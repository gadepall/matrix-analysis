
\numberwithin{equation}{subsection}
\subsection{Definitions}
\begin{definition}
	The {\em affine} transformation is given by 
    \begin{align}
	    \vec{x} &= \vec{P}\vec{y}+\vec{c} \quad \text{(Affine Transformation)}
\label{eq:conic_affine}
    \end{align}
	where $\vec{P}$ is invertible.
\end{definition}
\begin{definition}
	The eigenvalue decomposition of a symmetric matrix $\vec{V}$ is given by 
	%\cite{banchoff}
    \begin{align}
      \label{eq:conic_parmas_eig_def}
      \vec{P}^{\top}\vec{V}\vec{P} &= \vec{D}. \quad \text{(Eigenvalue Decomposition)}
      \\
      \vec{D} &= \myvec{\lambda_1 & 0\\ 0 & \lambda_2}, 
      \\
      \vec{P} &= \myvec{\vec{p}_1 & \vec{p}_2}, \quad \vec{P}^{\top}=\vec{P}^{-1},
      \label{eq:eigevecP}
    \end{align}
\end{definition}
\subsection{The Quadratic Form}

\begin{definition}
  Let $\vec{q}$ be a point such that the ratio of its distance from a fixed point $\vec{F}$ and the distance ($d$) from a fixed line 
	\begin{align}
L: \vec{n}^{\top}\vec{x}=c 
	\end{align}
		is constant, given by 
\label{conics/30/def}
\begin{align}
\frac{\norm{\vec{q}-\vec{F}}}{d} = e    
\end{align}
The locus of $\vec{q}$ is known as a conic section. The line $L$ is known as the directrix and the point $\vec{F}$ is the focus. $e$ is defined to be 
the eccentricity of the conic.  
\begin{enumerate}
    \item For $e = 1$, the conic is a parabola
    \item For $e < 1$, the conic is an ellipse
    \item For $e > 1$, the conic is a hyperbola
\end{enumerate}
\end{definition}
\begin{theorem}
The equation of  a conic with directrix $\vec{n}^{\top}\vec{x} = c$, eccentricity $e$ and focus $\vec{F}$ is given by 
\begin{align}
    \label{eq:conic_quad_form}
    \vec{x}^{\top}\vec{V}\vec{x}+2\vec{u}^{\top}\vec{x}+f=0
    \end{align}
where     
\begin{align}
  \label{eq:conic_quad_form_v}
\vec{V} &=\norm{\vec{n}}^2\vec{I}-e^2\vec{n}\vec{n}^{\top}, 
\\
\label{eq:conic_quad_form_u}
\vec{u} &= ce^2\vec{n}-\norm{\vec{n}}^2\vec{F}, 
\\
\label{eq:conic_quad_form_f}
f &= \norm{\vec{n}}^2\norm{\vec{F}}^2-c^2e^2
%\\
    \end{align}
    
% \begin{align}
% \vec{x}^{\top}(t\vec{I}-\vec{n}\vec{n}^{\top})\vec{x}+2(c\vec{n}-t\vec{F})^{\top}\vec{x}+t\norm{\vec{F}}^2-c^2&=0
% \end{align}
%
%and 
% where 
% \begin{align}
%     %t=\frac{\norm{\vec{n}}^2}{e^2}
%     \norm{\vec{n}} = 1
% \end{align}
%\end{theorem}
\end{theorem}
\begin{proof}
  Using Definition \ref{conics/30/def} and Lemma \ref{conics/30/lemma},  for any point $\vec{x}$ on the conic,
\begin{align}
\norm{\vec{x}-\vec{F}}^2&=e^2 \frac{\brak{{\vec{n}^{\top}\vec{x} - c}}^2}{\norm{\vec{n}}^2}\label{conics/30/eq:1} \\
\implies \norm{\vec{n}}^2\brak{\vec{x}-\vec{F}}^{\top}\brak{\vec{x}-\vec{F}}&=e^2\brak{\vec{n}^{\top}\vec{x} - c}^2
\\
\implies \norm{\vec{n}}^2\brak{\vec{x}^{\top}\vec{x}-2\vec{F}^{\top}\vec{x}+\norm{\vec{F}}^2}&=e^2\brak{c^2+\brak{\vec{n}^{\top}\vec{x} }^2-2c\vec{n}^{\top}\vec{x}} \\
&=e^2\brak{c^2+\brak{\vec{x}^{\top}\vec{n}\vec{n}^{\top}\vec{x} }-2c\vec{n}^{\top}\vec{x}}
% t\vec{x}^{\top}\vec{x}-(\vec{n}^{\top}\vec{x} )^2-2t\vec{F}^{\top}\vec{x}+2c\vec{n}^{\top}\vec{x}=c^2-t\norm{\vec{F}}^2\\
% t\vec{x}^{\top}\vec{I}\vec{x}-\vec{n}^{\top}\vec{x} \vec{n}^{\top}\vec{x}+2(c\vec{n}-t\vec{F})^{\top}\vec{x}=c^2-t\norm{\vec{F}}^2\\
% \vec{x}^{\top}(t\vec{I}-\vec{n}\vec{n}^{\top})\vec{x}+2(c\vec{n}-t\vec{F})^{\top}\vec{x}+t\norm{\vec{F}}^2-c^2=0
\end{align}
%
which can be expressed as \eqref{eq:conic_quad_form} after simplification.

% See Appendix \ref{app:conicdef}
\end{proof}
\begin{theorem}
  The eccentricity, directrices and foci of \eqref{eq:conic_quad_form} are given by 
%  \eqref{eq:conic_quad_form_e} -
%  \eqref{eq:conic_quad_form_F} 
%  \begin{figure*}[!hb]
%	  \centering
%	  \hrule
\begin{align}
  \label{eq:conic_quad_form_e} 
  e&= \sqrt{1-\frac{\lambda_1}{\lambda_2}}
\\
\label{eq:conic_quad_form_nc} 
	\begin{split}
  \vec{n}&= \sqrt{\lambda_2}\vec{p}_1,  
  \\
	c &= 
  \begin{cases}
    \frac{e\vec{u}^{\top}\vec{n} \pm \sqrt{e^2\brak{\vec{u}^{\top}\vec{n}}^2-\lambda_2\brak{e^2-1}\brak{\norm{\vec{u}}^2 - \lambda_2 f}}}{\lambda_2e\brak{e^2-1}} & e \ne 1
    \\
    \frac{\norm{\vec{u}}^2 - \lambda_2 f   }{2\vec{u}^{\top}\vec{n}} & e = 1
  \end{cases}
	\end{split}
  \\
  \label{eq:conic_quad_form_F} 
  \vec{F}  &= \frac{ce^2\vec{n}-\vec{u}}{\lambda_2}
\end{align}  
%  \end{figure*}
\end{theorem}
\begin{proof}
	See Appendix \ref{app:conic-parameters}
\end{proof}

\begin{theorem}
\eqref{eq:conic_quad_form} represents 
	\begin{enumerate}
		\item a parabola for $\mydet{\vec{V}} = 0 $,
		\item ellipse for $\mydet{\vec{V}} > 0 $ and 
		\item hyperbola for $\mydet{\vec{V}} < 0 $.
	\end{enumerate}
%		\item a pair of straight lines if
%\begin{align}
%\mydet{
%\vec{V}&\vec{u}
%\\
%\vec{u}^{\top}&f
%}
%=  0, \quad \mydet{\vec{V}} < 0
%\label{eq:quad_forms_pair_det}
%\end{align}
%			else, it represents
\end{theorem}
\begin{proof}
  From \eqref{eq:conic_quad_form_e},
\begin{align}
  \frac{\lambda_1}{\lambda_2} = 1 - e^2
\end{align}
Also, 
\begin{align}
	\mydet{\vec{V}} =   \lambda_1\lambda_2 
\end{align}
	yielding Table \ref{table:det}
\begin{table}[!h]
\centering
\input{tables/det.tex}
	\caption{}
\label{table:det}
\end{table}
\end{proof}
