%\subsection{The Quadratic Form}
%\numberwithin{equation}{subsection}
%\begin{enumerate}
\begin{enumerate}[label=\thesection.\arabic*.,ref=\thesection.\theenumi]

\item
  Let $\vec{q}$ be a point such that the ratio of its distance from a fixed point $\vec{F}$ and the distance ($d$) from a fixed line 
	\begin{align}
L: \vec{n}^{\top}\vec{x}=c 
	\end{align}
		is constant, given by 
\label{conics/30/def}
\begin{align}
\frac{\norm{\vec{q}-\vec{F}}}{d} = e    
\end{align}
The locus of $\vec{q}$ is known as a conic section. The line $L$ is known as the directrix and the point $\vec{F}$ is the focus. $e$ is defined to be 
the eccentricity of the conic.  
\begin{enumerate}
    \item For $e = 1$, the conic is a parabola
    \item For $e < 1$, the conic is an ellipse
    \item For $e > 1$, the conic is a hyperbola
\end{enumerate}

\item
The equation of  a conic with directrix $\vec{n}^{\top}\vec{x} = c$, eccentricity $e$ and focus $\vec{F}$ is given by 
\begin{align}
    \label{eq:conic_quad_form}
    \vec{x}^{\top}\vec{V}\vec{x}+2\vec{u}^{\top}\vec{x}+f=0
    \end{align}
where     
\begin{align}
  \label{eq:conic_quad_form_v}
\vec{V} &=\norm{\vec{n}}^2\vec{I}-e^2\vec{n}\vec{n}^{\top}, 
\\
\label{eq:conic_quad_form_u}
\vec{u} &= ce^2\vec{n}-\norm{\vec{n}}^2\vec{F}, 
\\
\label{eq:conic_quad_form_f}
f &= \norm{\vec{n}}^2\norm{\vec{F}}^2-c^2e^2
%\\
    \end{align}
    
% \begin{align}
% \vec{x}^{\top}(t\vec{I}-\vec{n}\vec{n}^{\top})\vec{x}+2(c\vec{n}-t\vec{F})^{\top}\vec{x}+t\norm{\vec{F}}^2-c^2&=0
% \end{align}
%
%and 
% where 
% \begin{align}
%     %t=\frac{\norm{\vec{n}}^2}{e^2}
%     \norm{\vec{n}} = 1
% \end{align}
%

\begin{proof}
  Using Definition \ref{conics/30/def} and Lemma \ref{conics/30/lemma},  for any point $\vec{x}$ on the conic,
\begin{align}
\norm{\vec{x}-\vec{F}}^2&=e^2 \frac{\brak{{\vec{n}^{\top}\vec{x} - c}}^2}{\norm{\vec{n}}^2}\label{conics/30/eq:1} \\
\implies \norm{\vec{n}}^2\brak{\vec{x}-\vec{F}}^{\top}\brak{\vec{x}-\vec{F}}&=e^2\brak{\vec{n}^{\top}\vec{x} - c}^2
\\
\implies \norm{\vec{n}}^2\brak{\vec{x}^{\top}\vec{x}-2\vec{F}^{\top}\vec{x}+\norm{\vec{F}}^2}&=e^2\brak{c^2+\brak{\vec{n}^{\top}\vec{x} }^2-2c\vec{n}^{\top}\vec{x}} \\
&=e^2\brak{c^2+\brak{\vec{x}^{\top}\vec{n}\vec{n}^{\top}\vec{x} }-2c\vec{n}^{\top}\vec{x}}
% t\vec{x}^{\top}\vec{x}-(\vec{n}^{\top}\vec{x} )^2-2t\vec{F}^{\top}\vec{x}+2c\vec{n}^{\top}\vec{x}=c^2-t\norm{\vec{F}}^2\\
% t\vec{x}^{\top}\vec{I}\vec{x}-\vec{n}^{\top}\vec{x} \vec{n}^{\top}\vec{x}+2(c\vec{n}-t\vec{F})^{\top}\vec{x}=c^2-t\norm{\vec{F}}^2\\
% \vec{x}^{\top}(t\vec{I}-\vec{n}\vec{n}^{\top})\vec{x}+2(c\vec{n}-t\vec{F})^{\top}\vec{x}+t\norm{\vec{F}}^2-c^2=0
\end{align}
%
which can be expressed as \eqref{eq:conic_quad_form} after simplification.

% See Appendix \ref{app:conicdef}
\end{proof}
\item
  The eccentricity, directrices and foci of \eqref{eq:conic_quad_form} are given by 
%  \eqref{eq:conic_quad_form_e} -
%  \eqref{eq:conic_quad_form_F} 
%  \begin{figure*}[!hb]
%	  \centering
%	  \hrule
\begin{align}
  \label{eq:conic_quad_form_e} 
  e&= \sqrt{1-\frac{\lambda_1}{\lambda_2}}
\\
\label{eq:conic_quad_form_nc} 
	\begin{split}
  \vec{n}&= \sqrt{\lambda_2}\vec{p}_1,  
  \\
	c &= 
  \begin{cases}
    \frac{e\vec{u}^{\top}\vec{n} \pm \sqrt{e^2\brak{\vec{u}^{\top}\vec{n}}^2-\lambda_2\brak{e^2-1}\brak{\norm{\vec{u}}^2 - \lambda_2 f}}}{\lambda_2e\brak{e^2-1}} & e \ne 1
    \\
    \frac{\norm{\vec{u}}^2 - \lambda_2 f   }{2\vec{u}^{\top}\vec{n}} & e = 1
  \end{cases}
	\end{split}
  \\
  \label{eq:conic_quad_form_F} 
  \vec{F}  &= \frac{ce^2\vec{n}-\vec{u}}{\lambda_2}
\end{align}  
%  \end{figure*}

\begin{proof}
	\label{app:conic-parameters}
	From \eqref{eq:conic_quad_form_v}, using the fact that $\vec{V}$ is symmetric with $\vec{V} = \vec{V}^{\top}$,
  \begin{align}
	  \vec{V}^{\top} \vec{V}&=\brak{\norm{\vec{n}}^2\vec{I}-e^2\vec{n}\vec{n}^{\top}}^{\top}
	  \brak{\norm{\vec{n}}^2\vec{I}-e^2\vec{n}\vec{n}^{\top}}
    \\
	  \implies \vec{V}^{2} &= \norm{\vec{n}}^4\vec{I}+e^4\vec{n}\vec{n}^{\top}\vec{n}\vec{n}^{\top}
	  -2e^2\norm{\vec{n}}^2\vec{n}\vec{n}^{\top}
    \\
	  &= \norm{\vec{n}}^4\vec{I} + e^4\norm{\vec{n}}^2\vec{n}\vec{n}^{\top}
	%  \\
	  - 2e^2\norm{\vec{n}}^2\vec{n}\vec{n}^{\top}
    \\
	  &= \norm{\vec{n}}^4\vec{I} + e^2\brak{e^2 - 2}\norm{\vec{n}}^2\vec{n}\vec{n}^{\top}
    \\
	  &= \norm{\vec{n}}^4\vec{I} + \brak{e^2 - 2}\norm{\vec{n}}^2\brak{\norm{\vec{n}}^2\vec{I}- \vec{V}}
    \end{align}
%    
which can be expressed as
\begin{align}
  \vec{V}^{2} + \brak{e^2 - 2}\norm{\vec{n}}^2\vec{V} - \brak{e^2 - 1}\norm{\vec{n}}^4\vec{I}=0
  \label{eq:conic_quad_form_e_cayley}
\end{align}
	Using the Cayley-Hamilton theorem,
%	\cite{banchoff}, 
	\eqref{eq:conic_quad_form_e_cayley} results in the characteristic equation, 
\begin{align}
  \lambda^{2} - \brak{2-e^2}\norm{\vec{n}}^2\lambda + \brak{1-e^2 }\norm{\vec{n}}^4=0
\end{align}
which can be expressed as
\begin{align}
\brak{\frac{\lambda}{\norm{\vec{n}}^2}}^2 - \brak{2-e^2 }\brak{\frac{\lambda}{\norm{\vec{n}}^2}} 
	+ \brak{1-e^2 } &= 0
	\\
	\implies \frac{\lambda}{\norm{\vec{n}}^2} &= 1-e^2, 1
  \\
	\text{or, }\lambda_2 = \norm{\vec{n}}^2, \lambda_1 &= \brak{1-e^2}\lambda_2 
  \label{eq:conic_quad_form_lam_cayley}
\end{align}
From   \eqref{eq:conic_quad_form_lam_cayley}, the eccentricity of \eqref{eq:conic_quad_form} is given by 
\eqref{eq:conic_quad_form_e}.   
%
% By inspection, we find that 
% \begin{align}
%   \frac{\lambda}}{\norm{\vec{n}}^2} = 1
%   \label{eq:conic_quad_form_lam2_cayley}
% \end{align}
%satisfies \eqref{eq:conic_quad_form_lam_cayley}.
Multiplying both sides of    \eqref{eq:conic_quad_form_v} by $\vec{n}$,
\begin{align}
\vec{V} \vec{n}&=\norm{\vec{n}}^2\vec{n}-e^2\vec{n}\vec{n}^{\top}\vec{n} 
\\
&=\norm{\vec{n}}^2\brak{1-e^2}\vec{n} 
 \\
% &=\frac{\lambda_1}{\lambda_2}\norm{\vec{n}}^2\vec{n} 
% \end{align}
% upon substituting from \eqref{eq:conic_quad_form_e}  and simplifying.  From the above, it is obvious that $\vec{n}$ is an eigenvector
% of $\vec{V}$.  Choosing 
% \begin{align}
%   \lambda_2 = \norm{\vec{n}}^2,
%   \label{eq:eigevecn_lam2}
% \end{align}  
% we obtain 
% \begin{align}
  &=\lambda_1 \vec{n} 
	\\
  \label{eq:eigevecn}
\end{align}  
from \eqref{eq:conic_quad_form_lam_cayley}.
Thus,  $\lambda_1$ is the corresponding eigenvalue for $\vec{n}$.  From       \eqref{eq:eigevecP} and \eqref{eq:eigevecn}, this implies that 
\begin{align}  
	\vec{p}_1 &= \frac{\vec{n}}{\norm{\vec{n}}} 
	\\
	\text{or, }
   \vec{n}&= \norm{\vec{n}}\vec{p}_1  = \sqrt{\lambda_2}\vec{p}_1 
\end{align}  
from   \eqref{eq:conic_quad_form_lam_cayley} .
From \eqref{eq:conic_quad_form_u} and \eqref{eq:conic_quad_form_lam_cayley},
\begin{align}
%   \label{eq:conic_quad_form_v}
% \vec{V} &=\norm{\vec{n}}^2\vec{I}-e^2\vec{n}\vec{n}^{\top}, 
% \\
%\label{eq:conic_quad_form_u}
\vec{F}  &= \frac{ce^2\vec{n}-\vec{u}}{\lambda_2}
 \\
 \implies \norm{\vec{F}}^2  &= \frac{\brak{ce^2\vec{n}-\vec{u}}^{\top}\brak{ce^2\vec{n}-\vec{u}}}{\lambda_2^2}
 \\
 \implies \lambda_2^2\norm{\vec{F}}^2  &= c^2e^4\lambda_2-2ce^2\vec{u}^{\top}\vec{n}+\norm{\vec{u}}^2
 \label{eq:conic_quad_form_u_temp}
% f &= \norm{\vec{n}}^2\norm{\vec{F}}^2-c^2e^2
% %\\
    \end{align}
    Also, \eqref{eq:conic_quad_form_f} can be expressed as
    \begin{align}
    \lambda_2\norm{\vec{F}}^2 &= f+c^2e^2
    \label{eq:conic_quad_form_f_temp}
\end{align}
From  \eqref{eq:conic_quad_form_u_temp} and     \eqref{eq:conic_quad_form_f_temp},
\begin{align}
c^2e^4\lambda_2-2ce^2\vec{u}^{\top}\vec{n}+\norm{\vec{u}}^2 = \lambda_2\brak{f+c^2e^2}
\end{align}
\begin{align}
\implies \lambda_2e^2\brak{e^2-1}c^2-2ce^2\vec{u}^{\top}\vec{n}
	+\norm{\vec{u}}^2 - \lambda_2 f = 0
\end{align}
yielding
  \eqref{eq:conic_quad_form_F}. 
%\begin{align}
%\text{or, } c = 
%\begin{cases}
%  \frac{e\vec{u}^{\top}\vec{n} \pm \sqrt{e^2\brak{\vec{u}^{\top}\vec{n}}^2-\lambda_2\brak{e^2-1}\brak{\norm{\vec{u}}^2 - \lambda_2 f}}}{\lambda_2e\brak{e^2-1}} & e \ne 1
%  \\
%  \frac{\norm{\vec{u}}^2 - \lambda_2 f   }{2e^2\vec{u}^{\top}\vec{n}} & e = 1
%\end{cases}
%\end{align}
	%See Appendix \ref{app:conic-parameters}
\end{proof}

\item
\eqref{eq:conic_quad_form} represents 
	\begin{enumerate}
		\item a parabola for $\mydet{\vec{V}} = 0 $,
		\item ellipse for $\mydet{\vec{V}} > 0 $ and 
		\item hyperbola for $\mydet{\vec{V}} < 0 $.
	\end{enumerate}
%		\item a pair of straight lines if
%\begin{align}
%\mydet{
%\vec{V}&\vec{u}
%\\
%\vec{u}^{\top}&f
%}
%=  0, \quad \mydet{\vec{V}} < 0
%\label{eq:quad_forms_pair_det}
%\end{align}
%			else, it represents

\begin{proof}
  From \eqref{eq:conic_quad_form_e},
\begin{align}
  \frac{\lambda_1}{\lambda_2} = 1 - e^2
\end{align}
Also, 
\begin{align}
	\mydet{\vec{V}} =   \lambda_1\lambda_2 
\end{align}
	yielding Table \ref{table:det}
\begin{table}[!h]
\centering
\input{tables/det.tex}
	\caption{}
\label{table:det}
\end{table}
\end{proof}
\end{enumerate}
