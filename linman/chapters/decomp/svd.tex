\section{Singular Value Decomposition}
\renewcommand{\theequation}{\theenumi}
\begin{enumerate}[label=\thesection.\arabic*.,ref=\thesection.\theenumi]
\numberwithin{equation}{enumi}

\item 
\iffalse
	We revisit \eqref{eq:pseudo_mat_eq}
%
\begin{align}
\myvec{
1 & 2
\\
-1 & 1
\\
1 & 2
}
\vec{x} =
\myvec{1 \\ -3 \\ -2}
\end{align}
\fi
\item Consider the rectangular equation
\begin{align}
	\vec{M}^{\top}\vec{x}=\vec{b}
\label{eq:pseudo_mat_eq}
\end{align}

\item Find $\vec{M}^T\vec{M}$ and $\vec{M}\vec{M}^T$.
\item Obtain the eigen decomposition 
\begin{align}
\vec{M}^T\vec{M} = \vec{P}_1\vec{D}_1\vec{P}_1^T
\end{align}
and 
\begin{align}
\vec{M}\vec{M}^T = \vec{P}_2\vec{D}_2\vec{P}_2^T
\end{align}
\item Perform the $QR$ decompositions
\begin{align}
\vec{P}_1 = \vec{U}\vec{R}_1,\,
\vec{P}_2 = \vec{V}\vec{R}_2
\end{align}
\item The singular value decomposition is the given by
\begin{align}
\vec{M} = \vec{U} \Sigma \vec{V}^T,
\end{align}
where $\Sigma$ has the same shape as $\vec{M}$ and
\begin{align}
\Sigma = \myvec{\vec{D}_1 & \vec{0} \\ \vec{0} & \vec{0}}
\end{align}
\item \eqref{eq:pseudo_mat_eq} can then be expressed as
\begin{align}
\vec{U} \Sigma \vec{V}^T \vec{x} &= \vec{b}
\\
\implies \vec{x} & = \vec{V}\Sigma^{-1} \vec{U}^T \vec{b}
\end{align}
%
where $\vec{\Sigma}^{-1}$ is obtained by inverting  only the non-zero elements of $\vec{\Sigma}$.
\iffalse
\item The relevant codes are available at
\begin{lstlisting}
codes/line/skew_builtin.py
codes/line/skew_svd.py
\end{lstlisting}
\fi
\end{enumerate}
