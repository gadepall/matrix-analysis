\iffalse
\documentclass{article}
\usepackage{setspace}
\usepackage{gensymb}
\usepackage{xcolor}
\usepackage{caption}
%\usepackage{subcaption}
%\doublespacing
\singlespacing
\usepackage{float}
\usepackage{graphicx}
\usepackage{amssymb}
\usepackage{relsize}
\usepackage[cmex10]{amsmath}
\usepackage{mathtools}
\usepackage{amsthm}
%\interdisplaylinepenalty=2500
%\savesymbol{iint}
%\usepackage{txfonts}
%\restoresymbol{TXF}{iint}
%\usepackage{wasysym}
\usepackage{amsthm}
%\usepackage{mathrsfs}
%\usepackage{txfonts}
%\usepackage{stfloats}
%\usepackage{cite}
%\usepackage{cases}
%\usepackage{subfig}
%\usepackage{xtab}
%\usepackage{longtable}
%\usepackage{multirow}
%\usepackage{algorithm}
%\usepackage{algpseudocode}
\usepackage{enumerate}
\usepackage{enumitem}
%\usepackage{mathtools}
%\usepackage{eenrc}
%\usepackage[framemethod=tikz]{mdframed}
%\usepackage{listings}
%\usepackage{listings}
    \usepackage[latin1]{inputenc}                                 %%
    \usepackage{color}                                            %%
    \usepackage{array}                                            %%
    \usepackage{longtable}                                        %%                                             %%
    \usepackage{multirow}                                         %%
    \usepackage{hhline}                                           %%
    \usepackage{ifthen}                                           %%
  %optionally (for landscape tables embedded in another document): %%
    \usepackage{lscape}     
\usepackage{wrapfig}
%\usepackage{ragged2e}
\usepackage{geometry}
\usepackage{multicol}
\usepackage{blindtext}
%\SetEnumitemKey{twocol}{
%	before=\raggedcolumns\begin{multicols}{2},
%	after=\end{multicols}}
%\SetEnumitemKey{fourcol}{
%	before=\raggedcolumns\begin{multicols}{4},
%	after=\end{multicols}}

\geometry{a4paper,total={200mm,225mm},left=25mm,right=30mm}
\begin{document}
\title{9$^{th}$ MATHS\\CHAPTER 9\\AREAS OF PARLLELOGRAMS AND TRIANGLES}
\date{}
\maketitle
\section*{EXERCISE 9.1}
\fi
Write the correct answer in each of the following:
\begin{enumerate}[label=\thesection.\arabic*,ref=\thesection.\theenumi]
\item The median of a triangle divides it into two 
\begin{enumerate}
\item triangles of equal area
\item congruent triangles
\item right triangles          
\item isosceles triangles
\end{enumerate}
\item In which of the following figures (Figure \ref{fig:exemplar/9.9.1Figure 1}), you find two polygons on the same base and between the same parallels?
\begin{figure}[!h]
\begin{center}
\includegraphics[width=\columnwidth]{exemplar/9.9.1/figs/three.jpg}
\end{center}
\caption{}
\label{fig:exemplar/9.9.1Figure 1}
\end{figure}
\item The figure obtained by joining the mid-points of the adjacent sides of a rectangle of sides 8cm and 6cm is:
	\begin{enumerate}
\item a rectangle of area $24 cm^2$ 
\item a square of area $25 cm^2$
\item a trapezium of area $24 cm^2$ 
\item a rhombus of area $24 cm^2$
\end{enumerate}

\item In Figure \ref{fig:exemplar/9.9.1Figure 2}, the area of parallelogram $ABCD$ is:
\begin{figure}[!h]
\begin{center}
\includegraphics[width=\columnwidth]{exemplar/9.9.1/figs/four.jpg}
\caption{}
\label{fig:exemplar/9.9.1Figure 2}
\end{center}
\end{figure}
\begin{enumerate}
\item AB x BM
\item BC x BN
\item DC x DL
\item AD x DL
\end{enumerate}
\item In Figure \ref{fig:exemplar/9.9.1Figure 3}, if parallelogram $ABCD$ and rectangle $ABEF$ of equal area, then:
\begin{enumerate}
\item Perimeter of $ABCD$ = Perimeter of $ABEM$
\item Perimeter of $ABCD$ $<$ Perimeter of $ABEM$
\item Perimeter of $ABCD$ $>$ Perimeter of $ABEM$
\item Perimeter of $ABCD$ = $\frac{1}{2}$ (Perimeter of $ABEM$)
\end{enumerate}
\begin{figure}[!h]
\begin{center}
\includegraphics[width=\columnwidth]{exemplar/9.9.1/figs/five.jpg}
\end{center}
\caption{}
\label{fig:exemplar/9.9.1Figure 3}
\end{figure}
\item The mid-point of the sides of a triangle along with any of the vertices as the fourth point make a parallelogram of area equal to
\begin{enumerate}
\item $\frac{1}{2} ar (ABC)$  
\item $\frac{1}{3} ar (ABC)$
\item $\frac{1}{4} ar (ABC)$
\item $ar (ABC)$
\end{enumerate}
\item Two parallelograms are on equal bases and between the same parallels. The ratio of their areas is
	\begin{enumerate}
\item $1:2$  \item $1:1$  \item $2:1$  \item $3:1$
\end{enumerate}
\item $ABCD$ is a quadrilateral whose diagonal $AC$ divides it into two parts, equal in area, then $ABCD$
\begin{enumerate}
\item is a rectangle       \item is always a rhombus
\item is a parallelogram   \item need not be any of (a), (b) or (c)
\end{enumerate}
\item If a triangle and a parallelogram are on the same base an between same parallels, then the ratio of the area of the triangle to the area of the parallelogram is
\begin{enumerate}
\item $1:3$  \item $1:2$  \item $3:1$  \item $1:4$
\end{enumerate}
\item $ABCD$ is a trapezium with parallel sides $AB = a cm$ and $DC = b cm$ (Figure \ref{fig:exemplar/9.9.1Figure 4}). E and F are the mid-points of the non-parallel sides. The ratio of $ar (ABFE)$ and $ar (EFCD)$ is
\begin{figure}[!h]
\begin{center}
\includegraphics[width=\columnwidth]{exemplar/9.9.1/figs/six.jpg}
\end{center}
\caption{}
\label{fig:exemplar/9.9.1Figure 4}
\end{figure}
\begin{enumerate}
\item $a:b$ 
\item $(3a+b):(a+3b)$  
\item $(a+3b):(3a+b)$
\item $(2a+b):(3a+b)$
\end{enumerate}
\end{enumerate}

