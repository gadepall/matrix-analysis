\begin{enumerate}[label=\thesection.\arabic*,ref=\thesection.\theenumi]
\numberwithin{equation}{enumi}
\numberwithin{figure}{enumi}
\numberwithin{table}{enumi}
\item In the Figure \ref{fig:9/9/2/1}, $ABCD$ is a parallelogram, $AE \perp DC$ and $CF \perp AD$. If $AB = 16 cm$, $AE = 8 cm$, and $CF = 10cm$, find $AD$.
	\begin{figure}[!h]
		\centering
 \includegraphics[width=\columnwidth]{chapters/9/9/2/1/figs/fig1.png}
		\caption{}
		\label{fig:9/9/2/1}
  	\end{figure}\\
\documentclass{article}
\usepackage{amsmath}
\usepackage{xcolor}
\usepackage{gensymb}
\usepackage{ragged2e}
\usepackage{graphicx}
\usepackage{gensymb}
\usepackage{mathtools}
\newcommand{\mydet}[1]{\ensuremath{\begin{vmatrix}#1\end{vmatrix}}}
\providecommand{\brak}[1]{\ensuremath{\left(#1\right)}}
\providecommand{\norm}[1]{\left\lVert#1\right\rVert}
\newcommand{\solution}{\noindent \textbf{Solution: }}
\newcommand{\myvec}[1]{\ensuremath{\begin{pmatrix}#1\end{pmatrix}}}
\let\vec\mathbf
\begin{document}
\begin{center}
        \textbf\large{CHAPTER-9 \\ AREAS OF PARALLELOGRAMS AND TRIANGLES}
\end{center}
\section{Exercise 9.2}
Q1. In the figure given below, $ABCD$ is a parallelogram, $AE \perp DC$ and $CF \perp AD$.If $AB = 16cm$, $AE = 8cm$ and $CF = 10cm$, find $AD$.\\
\textbf{Construction}
\begin{figure}[h]
 \begin{center}
  \includegraphics[width=\columnwidth]{fig1.png}
 \end{center}
 \caption{Parallelogram ABCD}
 \label{fig:Fig1}
\end{figure}\\
The following table displays the given input parameters :\\
\begin{table}[h]
	\centering
	\begin{tabular}{|c|c|c|}
  \hline
  \textbf{Symbol}&\textbf{Value}&\textbf{Description}\\
  \hline
  $a$ & 8 & $BC$\\
  \hline
	$\angle{B}$ & 45$\degree{}$ & $\angle{B}$ in $\triangle$$ABC$ \\
  \hline
	$k$ & 3.5 & $AB-AC$ i.e $c-b$ \\
  \hline 
	$\vec{e_2}$ & $\myvec{
			0\\
			1\\
			}$ & Basis vector\\
 \hline			
\end{tabular}

	\caption{Input Parameters}
	\label{tab:table1}
\end{table}\\
The lengths and angles which are to be found out are displayed in the table below along with their symbols :\\
\begin{table}[h]
	\centering
	\begin{tabular}{|c|c|}
\hline
Symbol & Description \\
\hline
$c$ & CD \\
\hline
$d$ & DE \\
\hline
$r$ & AD \\
\hline
$f$ & DF \\
\hline
$\theta$ & $\angle{D}$\\
\hline
\end{tabular}

	\caption{Unknown Parameters}
	\label{tab:table2}
\end{table}\\
The input co-ordinates of the above parallelogram is $\vec{D}$ which is at the origin.The rest of the point co-ordinates can be derived based on this assumption in the following way which is shown in the table below :\\
\begin{table}[h]
	\centering
	\begin{tabular}{|c|c|}
\hline
Point & Co-ordinates\\
\hline
$\vec{A}$ & $r\myvec{\cos{\theta}\\\sin{\theta}}$ \\
\hline
$\vec{B}$ & $\vec{A} + \vec{C}$ \\
\hline
$\vec{C}$ & $c\vec{e_1}$\\
\hline
$\vec{E}$ & $d\vec{e_1}$ \\
\hline
$\vec{F}$ & $f\myvec{\cos{\theta}\\\sin{\theta}}$ \\
\hline
\end{tabular}

	\caption{Unknown Co-ordinates}
	\label{tab:table3}
\end{table}\\
\textbf{Deriving the Unknown lengths and angles in terms of known and derived parameters :}\\
\begin{enumerate}
	\item \textbf{Deriving c:}
		From Figure\ref{fig:Fig1}, $c$ is  parallel to $l$(AB paralle to CD).So,
		\begin{align}
			c = l
			\label{eq:1}
		\end{align}
	\item \textbf{Deriving d:}
		From $\triangle{ADE}$,
		\begin{align}
			\cos{\theta} = \frac{DE}{AD} = \frac{d}{r}
			\implies d = r\cos{\theta}
			\label{eq:2}
		\end{align}
	\item \textbf{Deriving r:}
		From $\triangle{ADE}$,
		\begin{align}
			\sin{\theta} = \frac{AE}{AD} = \frac{b}{r}
			\implies r = \frac{b}{\sin{\theta}}
			\label{eq:3}
		\end{align}
	\item \textbf{Deriving f:}
		From $\triangle{DFC}$,
		\begin{align}
			\cos{\theta} = \frac{DF}{DC} = \frac{f}{c}
			\implies f = c\cos{\theta}
			\label{eq:4}
		\end{align}
	\item \textbf{Finding $\theta$:}
		From $\triangle{DFC}$,
		\begin{align}
			\sin{\theta} = \frac{CF}{CD} = \frac{a}{c}
			\implies \theta = \sin^{-1}\frac{a}{c}
			\label{eq:6}
		\end{align}
\end{enumerate}
From eq\ref{eq:1},eq\ref{eq:2},eq\ref{eq:3},eq\ref{eq:4} and eq\ref{eq:6}, table\ref{tab:table2} can be modified as :\\
\begin{table}[h]
	\centering
	\begin{tabular}{|c|c|c|}
\hline
Symbol & value & Description\\
\hline
$c$ & $l$ & DC\\
\hline
$r$ & $\frac{b}{\sin{\theta}}$ & AD \\
\hline
$d$ & $r\cos{\theta}$ & DE \\
\hline
$\theta$ & $\sin^{-1}\frac{a}{c}$ & $\angle{D}$ \\
\hline
$f$ & $c\cos{\theta}$ & DF\\
\hline
\end{tabular}


	\caption{Unknown parameters in terms of known and derived parameters}
	\label{tab:table4}
\end{table}\\
\textbf{Finding out unknown lengths and angles :}\\
\begin{enumerate}
	\item \textbf{Finding $\theta$:}
		From eq\ref{eq:6},
		\begin{align}
			\theta = \sin^{-1}\frac{10}{16} = 38.68\degree
			\label{eq:7}
		\end{align}
	\item \textbf{Finding c:}
		From eq\ref{eq:1},
		\begin{align}
			c = 16cm
			\label{eq:8}
		\end{align}
	\item \textbf{Finding r:}
		From \ref{eq:3}, the value of $r$ is :
		\begin{align}
			r = \frac{8}{\sin{38.68}} = 12.8cm
			\label{eq:9}
		\end{align}
	\item \textbf{Finding d:}
		From eq\ref{eq:2},
		\begin{align}
			d = (12.8)\cos{38.68} = 10cm
			\label{eq:10}
		\end{align}
	\item \textbf{Finding f:}
		From eq\ref{eq:4},
		\begin{align}
			f = (16)\cos{38.68} = 12.48cm
			\label{eq:11}
		\end{align}
\end{enumerate}
\textbf{Deriving co-ordinates in terms of known and derived parameters:}\\
Based on table\ref{tab:table4}, table\ref{tab:table3} can be modified as follows :\\
\begin{table}[h]
	\centering
	\begin{tabular}{|c|c|c|}
\hline
Point & Co-ordinates\\
\hline
$\vec{A}$ & $\frac{b}{\sin{\theta}}\myvec{\cos{\theta}\\\sin{\theta}}$\\
\hline
$\vec{B}$ & $\vec{A} + \vec{C}$\\
\hline
$\vec{C}$ & $c\vec{e_1}$\\
\hline
$\vec{E}$ & $r\cos{\theta}\vec{e_1}$\\
\hline
$\vec{F}$ & $c\cos{\theta}\myvec{\cos{\theta}\\\sin{\theta}}$\\
\hline
\end{tabular}

	\caption{Co-ordinates in terms of known and derived co-ordinates}
	\label{tab:table5}
\end{table}\\
Based on eq\ref{eq:7},eq\ref{eq:8},eq\ref{eq:9},eq\ref{eq:10} and eq\ref{eq:11} and table\ref{tab:table5}.The final co-ordinates of the  parallelogram are displayed in the table below:\\
\begin{table}[h]
	\centering
	\begin{tabular}{|c|c|}
\hline
Point & Co-ordinates\\
\hline
$\vec{A}$ & $\myvec{10\\8}$\\
\hline
$\vec{B}$ & $\myvec{26\\8}$\\
\hline
$\vec{C}$ & $\myvec{16\\0}$\\
\hline
$\vec{D}$ & $\myvec{0\\0}$\\
\hline
$\vec{E}$ & $\myvec{10\\0}$\\
\hline
$\vec{F}$ & $\myvec{9.75\\7.8}$\\
\hline
\end{tabular}

	\caption{Final Co-ordinates}
	\label{tab:table6}
\end{table}\\
From eq\ref{eq:9}, we got the length of AD = r = 12.8cm.
\end{document}

\item For a given Parallelogram $ABCD$, show that for any
point $\vec{P}$ inside the parallelogram,
\begin{enumerate}
	\item $Ar(APD)+Ar(PBC) = \frac{1}{2}Ar(ABCD)$
	\item $Ar(APD)+Ar(PBC) = Ar(APB)+Ar(PCD)$
\end{enumerate}
\documentclass[journal,10pt,twocolumn]{article}
\usepackage{graphicx}
\graphicspath{{./Figures/}}
\usepackage[margin=0.5in]{geometry}
\usepackage[cmex10]{amsmath}
\usepackage{amssymb}
\usepackage{array}
\usepackage{booktabs}

\title{\textbf{Line Assignment}}
\author{Bole Manideep}
\date{September 2022}

\providecommand{\norm}[1]{\left\lVert#1\right\rVert}
\providecommand{\abs}[1]{\left\vert#1\right\vert}
\let\vec\mathbf
\newcommand{\myvec}[1]{\ensuremath{\begin{pmatrix}#1\end{pmatrix}}}
\newcommand{\mydet}[1]{\ensuremath{\begin{vmatrix}#1\end{vmatrix}}}
\providecommand{\brak}[1]{\ensuremath{\left(#1\right)}}

\begin{document}
\maketitle
\paragraph{\textit{Problem Statement} - For a given Parallelogram ABCD, show that for any
point ’P’ inside the parallelogram,}
\begin{enumerate}
	\item $\boldsymbol{Ar(APD)+Ar(PBC) = \frac{1}{2}Ar(ABCD)}$
	\item $\boldsymbol{Ar(APD)+Ar(PBC) = Ar(APB)+Ar(PCD)}$
\end{enumerate}

\begin{figure}[h]
\centering
\includegraphics[width=1\columnwidth]{Question.png}
\caption{Parallelogram ABCD with interior point P}
\label{fig:Parallelogram}
\end{figure}

\section*{Solution}

\subsection*{Part 1}
WKT, area of a parallelogram with adjacent sides a \& b is,
\begin{equation}
\text{Area of parallelogram = } \norm{\vec{a} \times \vec{b}}
\label{eq-1-}
\end{equation}
And, area of a triangle with adjacent sides p \& q is,
\begin{equation}
\text{Area of triangle = } \frac{1}{2} \norm{\vec{p} \times \vec{q}}
\label{eq-2-}
\end{equation}
From Figure 1,\\ $\vec{(A-D)}\hspace{2mm}\&\hspace{2mm}\vec{(B-C)}$ are equal,\\
\begin{equation}
\norm{\vec{A-D}} = \norm{\vec{B-C}}
\label{eq-3-}
\end{equation}
Consider $\triangle$APD
\begin{equation}
\text{Area of } \triangle APD = \frac{1}{2}\norm{\vec{(A-D)} \times \vec{(A-P)}}
\label{eq-4-}
\end{equation}
Consider $\triangle$PBC
\begin{equation}
\text{Area of } \triangle BPC =  \frac{1}{2}\norm{\vec{(B-C)} \times \vec{(P-B)}}
\label{eq-5-}
\end{equation}
On adding \eqref{eq-4-} \& \eqref{eq-5-},
\begin{multline}
Ar(APD) + Ar(PBC) =\\
 \frac{1}{2}\norm{\vec{(A-D)} \times \vec{(A-P)}} + \frac{1}{2}\norm{\vec{(B-C)} \times \vec{(P-B)}}
\label{eq-6-}
\end{multline}
From equation \eqref{eq-3-},
\begin{multline}
Ar(APD) + Ar(PBC) =\\
 \frac{1}{2}\norm{\vec{(A-D)} \times \vec{(A-P)}} + \frac{1}{2}\norm{\vec{(A-D)} \times \vec{(P-B)}}
\label{eq-7-}
\end{multline}
\begin{multline}
\implies Ar(APD) + Ar(PBC) =\\ \frac{1}{2}\norm{\vec{(A-D)}\times[\vec{(A-P)} + \vec{(P-B)}]}
\label{eq-8-}
\end{multline}
Here, AP \& PB are adjacent sides of $\triangle$ APB
\\From Triangle law of vector addition, \\ $\vec{(A-P)} + \vec{(P-B)} = \vec{(A-B)}$
\begin{equation}
\implies Ar(APD) + Ar(PBC) = \frac{1}{2}\norm{\vec{(A-D)}\times\vec{(A-B)}}
\label{eq-9-}
\end{equation}
Since, $\vec{(A-D)} \hspace{2mm} \& \hspace{2mm} 1\vec{(A-B)}$ are adjacent sides of paralleogram ABCD
\\With reference to \eqref{eq-2-},
\begin{equation}
Ar(ABCD) = \norm{\vec{(A-D)}\times\vec{(A-B)}}
\label{eq-10-}
\end{equation}
From \eqref{eq-9-} \& \eqref{eq-10-}
\begin{equation}
\therefore \hspace{3mm} Ar(APD)+Ar(PBC) = \frac{1}{2}Ar(ABCD)
\label{eq-11-}
\end{equation}

\subsection*{Part 2}
Similarly, we can prove taht,
\begin{equation}
Ar(APB)+Ar(PBD) = \frac{1}{2}Ar(ABCD)
\label{eq-12-}
\end{equation}
On Comparing \eqref{eq-11-} and \eqref{eq-12-},
\begin{equation}
Ar(APD)+Ar(PBC) = Ar(APB)+Ar(PCD)
\label{eq-13-}
\end{equation}
\begin{center}
Hence Proved
\end{center}
\section*{Construction}
\raggedright A parallelogram ABCD is constructed unsing python,with the parameters that are mentioned in the table below.
\vspace{5mm}
\begin{center}
    \setlength{\arrayrulewidth}{0.1mm}
	\setlength{\tabcolsep}{12pt}
	\renewcommand{\arraystretch}{1.5}
\begin{tabular}{|c|c|c|}
	\hline 
    \textbf{Symbol} & \textbf{Value} & \textbf{Description}\\ 		\hline
    a & 4 & AB \\ \hline
    b & 2 & AD \\ \hline
    $\theta$ & 60$^{\circ}$ & $\angle$A \\ \hline
    $\vec{A}$ & $\myvec{0 \\ 0}$ & Vertex A \\ \hline
    $\vec{B}$ & $\myvec{a \\ 0}$ & Vertex B \\ \hline
    $\vec{D}$ & $b\myvec{\cos\theta \\ \sin\theta}$ & Vertex B \\ \hline
    $\vec{C}$ & $\vec{B+D}$ & Vertex C \\ \hline
    
\end{tabular}\\ \vspace{2mm}
Table 1: Parameter's Table
\end{center}

\section*{Proofs}
Triangle law of vector addition \\
Consider a triangle APB with vertices, \\ \vspace{2mm}
$\vec{A} = \myvec{0 \\ 0}$ \hspace{2mm}
$\vec{P} = \myvec{2 \\ 1}$ \hspace{2mm}
$\vec{B} = \myvec{0 \\ 4}$ \\ \vspace{2mm}
 
Vectors $\vec{(A-P)}, \vec{(P-B)} \hspace{1mm} \& \hspace{1mm} \vec{(A-B)} \hspace{1mm} are \hspace{1mm} sides \hspace{1mm} of \hspace{1mm} \triangle APB$\\
Let's consider,
\begin{equation}
\vec{(A-P)} + \vec{(P-B)} = \myvec{0 \\ 0} - \myvec{2 \\ 1} \hspace{2mm} + \hspace{2mm} \myvec{2 \\ 1} - \myvec{0 \\ 4}
\end{equation}
\begin{equation}
\implies \vec{(A-P)} + \vec{(P-B)} = \myvec{-2 \\ -1} \hspace{2mm} + \hspace{2mm} \myvec{2 \\ -3}
\end{equation}
\begin{equation}
\implies \vec{(A-P)} + \vec{(P-B)} = \myvec{0 \\ -4}
\end{equation}
\begin{equation}
\implies \vec{(A-P)} + \vec{(P-B)} = \myvec{0 \\ 0} - \myvec{0 \\ 4}
\end{equation} \vspace{1mm}
\begin{equation}
\therefore \vec{(A-P)} + \vec{(P-B)} = \vec{(A-B)}
\end{equation}  \\
\vspace{2mm}Thus, Triangle law of vector addition says taht sum of two adjacent side vectors of a triangle is equal to third side vector but in opposite direction.

\end{document}

\item In Fig.
		\ref{fig:9/9/2/5},
$PQRS$ and $ABRS$ are parallelograms
and $\vec{X}$ is any point on side $BR$. Show that  
\begin{enumerate}
    \item $ar (PQRS) = ar(ABRS)$
	    \label{prop:9/9/2/5}
    \item $ar(AXS) = \frac{1}{2} ar(PQRS)$
\end{enumerate}
	\begin{figure}[!h]
		\centering
 \includegraphics[width=\columnwidth]{chapters/9/9/2/5/figs/parallelogram1.pdf}
		\caption{}
		\label{fig:9/9/2/5}
  	\end{figure}
\iffalse
\documentclass[journal,10pt,twocolumn]{article}
\usepackage{graphicx}
\usepackage[margin=0.5in]{geometry}
\usepackage[cmex10]{amsmath}
\usepackage{array}
\usepackage{booktabs}
\usepackage{listings}
\title{\textbf{Line Assignment}}
\author{Bhavani Kanike}
\date{October 2022}

\providecommand{\norm}[1]{\left\lVert#1\right\rVert}
\providecommand{\abs}[1]{\left\vert#1\right\vert}
\let\vec\mathbf
\newcommand{\myvec}[1]{\ensuremath{\begin{pmatrix}#1\end{pmatrix}}}
\newcommand{\mydet}[1]{\ensuremath{\begin{vmatrix}#1\end{vmatrix}}}
\providecommand{\brak}[1]{\ensuremath{\left(#1\right)}}

\begin{document}

\maketitle
\paragraph{\textit{Problem Statement} 
\fi
ABCD is a quadrilateral in which $\vec{P}, \vec{Q}, \vec{R}$ and $\vec{S}$ are mid-points of the sides AB, BC, CD and DA (see Fig \ref{fig:9/8/2/1}). AC is a diagonal. 
		
Show that 
\begin{enumerate}
	\item $SR \parallel AC$ and $SR =\frac{1}{2} AC$
\item $PQ = SR$
\item $PQRS$ is a parallelogram.
\end{enumerate}
 	\begin{figure}
		\centering
 \includegraphics[width=\columnwidth]{chapters/9/8/2/1/figs/line1.pdf}
		\caption{}
		\label{fig:9/8/2/1}
  	\end{figure}
	\solution 
	Using 
	  \eqref{eq:section_formula},
	\begin{align}
		\label{eq:9/8/2/1}
		\begin{split}
		\vec{P} &= \frac{\vec{A}+\vec{B}}{2}\\
 \vec{Q} &= \frac{\vec{C}+\vec{B}}{2}\\
 \vec{R} &= \frac{\vec{C}+\vec{D}}{2}\\
 \vec{S} &= \frac{\vec{D}+\vec{A}}{2}
		\end{split}
	\end{align}
\begin{enumerate}
	\item
	Consequently, 
	\begin{align}
\vec{R}
		-\vec{S} &= \frac{\vec{C}-\vec{A}}{2}
		\\
		\implies SR &\parallel AC
	\end{align}
	Also, 
	\begin{align}
		\norm{\vec{R}
		-\vec{S}} &= \frac{\norm{\vec{C}-\vec{A}}}{2}
		\\
		\implies SR &= \frac{1}{2}AC
	\end{align}
\item 	From 
		\eqref{eq:9/8/2/1},
	\begin{align}
\vec{R}
		-\vec{S} = \vec{Q}-\vec{P}
	\end{align}
	which means that $PQRS$ is a parallelogram and $PQ = SR$.
\end{enumerate}
%
\iffalse
\begin{figure}[h]
\centering
\includegraphics[width=1\columnwidth]
\caption{Figure}
\label{fig:triangle}
\end{figure}

\section*{Solution}

$\boldsymbol Given :$  ABCD is a Quadrilateral P,Q,R and S are the midpoints of line AB,BC,CD,DA.We can obtain the points P,Q,R and S from A,B,C and D and are given by\\\\
\boldmath
\unboldmath
(3) To prove that PQRS is a parallelogram we need to prove  PQ // SR
To prove SR $\parallel$ PQ\\
Direction vector of line SR  $\boldsymbol {(R-S) =  \frac{(C-A)}{2}}$\\\\
Direction vector of line PQ  $\boldsymbol {(Q-P)= \frac{(C-A)}{2}}$\\\\
\begin{equation}
	\boldsymbol {(R-S) = (Q-P) = \frac{(C-A)}{2}}\\
\end{equation}
Since the direction vectors of line SR and PQ are in same direction\\\\
$SR \parallel PQ$\\
Therefore,
$\boldsymbol{ PQRS }$ is a parallelogram\\\\

	
(1)  Directional vector of line SR  = $\boldsymbol {(R-S)}$ = $\frac{\boldsymbol{(C-A)}}{2} $\\
Directional vector of line AC  = $\boldsymbol {(C-A)}$\\

It is observed that the constant k is $\frac{1}{2}$

Therefore
\begin{equation}
	SR \parallel AC
\end{equation} 

and from equation 1 
\begin{equation}
	\boldsymbol {SR = \frac{1}{2}AC}    
\end{equation}\\


(2)   To prove PQ = SR\\ 
		From euqation 1\\\\
\begin{equation}
		\boldsymbol{ (Q-P) = (R-S) = \frac{(C-A)}{2}}
\end{equation}
	 



\section{Execution}
The below python code realizes the construction:
\begin{lstlisting}
https://github.com/bhavani360/FWC_assignments
\end{lstlisting}
	
\section*{Construction}
The dimensions of the Quadrilateral ABCD are taken as below\\
{
\setlength\extrarowheight{2pt}
\centering
	\begin{tabular}{|c|c|}
	\hline
	\textbf{symbol}&\textbf{value}\\
	\hline
	r&8\\
	\hline
	$\theta$&pi/2.5\\
	\hline
	d&7\\
	\hline
	A&(0,0)\\
	\hline
	B&(d,0)\\
	\hline
	D&(rcos$\theta$,rsin$\theta$)\\
	\hline
	C&(D/1.5)+B\\
	\hline
\end{tabular}
}
\end{document}
\fi

%	
\item In quadrilateral $CBAD$,$CA = AD$ and $BA$ bisect $\angle{A}$ shown in figure \tabref{fig:chapters/9/7/1/1/Fig1}. Show that $\triangle{CAB} \cong \triangle{DAB}$. What can you say about $BC$ and $BD$? \\
	\solution
\iffalse
\documentclass{article}
\usepackage{amsmath}
\usepackage{xcolor}
\usepackage{gensymb}
\usepackage{ragged2e}
\usepackage{graphicx}
\usepackage{gensymb}
\usepackage{mathtools}
\newcommand{\mydet}[1]{\ensuremath{\begin{vmatrix}#1\end{vmatrix}}}
\providecommand{\brak}[1]{\ensuremath{\left(#1\right)}}
\providecommand{\norm}[1]{\left\lVert#1\right\rVert}
\newcommand{\solution}{\noindent \textbf{Solution: }}
\newcommand{\myvec}[1]{\ensuremath{\begin{pmatrix}#1\end{pmatrix}}}
\let\vec\mathbf
\begin{document}
\begin{center}
        \textbf\large{CHAPTER-7 \\ TRIANGLES}
\end{center}
\section{Exercise 7.1}
Q1. \textbf{Construction}\\
\fi
See 
	  \tabref{tab:9/7/1/1/Table1}.
\begin{figure}[h]
	\begin{center}
		\includegraphics[width=\columnwidth]{chapters/9/7/1/1/figs/fig.png}
	\end{center}
	\caption{Quadrilateral CBAD}
	\label{fig:chapters/9/7/1/1/Fig1}
\end{figure}
\begin{table}[h]
	  \centering
	  \begin{tabular}{|c|c|c|}
  \hline
  \textbf{Symbol}&\textbf{Value}&\textbf{Description}\\
  \hline
  $a$ & 8 & $BC$\\
  \hline
	$\angle{B}$ & 45$\degree{}$ & $\angle{B}$ in $\triangle$$ABC$ \\
  \hline
	$k$ & 3.5 & $AB-AC$ i.e $c-b$ \\
  \hline 
	$\vec{e_2}$ & $\myvec{
			0\\
			1\\
			}$ & Basis vector\\
 \hline			
\end{tabular}

	  \caption{Parameters}
	  \label{tab:9/7/1/1/Table1}
\end{table}
The vertices of the quadrilateral can be expressed as
\begin{align}
	\vec{A} = \myvec{0\\0},\vec{B} = a\vec{e_1},\vec{C} = \myvec{c\cos\theta\\c\sin\theta},\vec{D} = \myvec{c\cos\theta\\-c\sin\theta}
\end{align}
where
\begin{align}
	\vec{C}-\vec{A} &= \vec{A}-\vec{D}\\
	\angle{CAB} &= \angle{DAB}
\end{align}
\begin{align}
	AB:	\vec{n}^{\top}\vec{x} = 0,
\end{align}
where
\begin{align}
\vec{n} = \myvec{0\\1}
\end{align}
Letting
		\begin{align}
\theta_1=&\angle CBA 
		\end{align}
		and substituting numerical values,
		\begin{align}
\vec{m_1}=&\vec{B}-\vec{C}=\myvec{4.7\\-2.5}, \vec{m_2}=\vec{B}-\vec{A}=\myvec{9\\0}\\
\implies \theta_1=&\cos^{-1}\frac{\vec{m_1}^\top\vec{m_2}}{\norm{\vec{m_1}}\norm{\vec{m_2}}}\\
&=\cos^{-1}\frac{\myvec{4.7&-2.5}\myvec{9\\0}}{(9.2)(9)}=59.3\degree
\label{eq:chapters/9/7/1/1/1}
		\end{align}
		Similalry,  for
		\begin{align}
\theta_2=&\angle ABD, \\
\vec{n_1}=&\vec{D}-\vec{B}=\myvec{-4.7\\2.5}, \vec{n_2}=\vec{A}-\vec{B}=\myvec{-9\\0}\\
\implies \theta_2 =& \cos^{-1}\frac{\vec{n_1}^\top\vec{n_2}}{\norm{\vec{n_1}}\norm{\vec{n_2}}}\\
&=\cos^{-1}\frac{\myvec{-4.7&2.5}\myvec{-9\\0}}{(9.2)(9)}=59.3\degree
\label{eq:chapters/9/7/1/1/2}\\
\end{align}
From \eqref{eq:chapters/9/7/1/1/1} and \eqref{eq:chapters/9/7/1/1/2},
\begin{align}
\angle BAC = \angle BAD 
\end{align}
Similarly, equality can be shown for other sides and angles.

\item $ABCD$ is a quadrilateral in which $AD = BC$ and $\angle{DAB} = \angle{CBA}$ as shown in figure \ref{fig:chapters/9/7/1/2/Fig}. Prove that
\begin{enumerate}
\item $\triangle{ABD} \cong \triangle{BAC}$
  \item $BD = AC$
  \item $\angle{ABD} = \angle{BAC}$
\end{enumerate}
	\solution
\iffalse
\documentclass{article}
\usepackage{amsmath}
\usepackage{xcolor}
\usepackage{gensymb}
\usepackage{ragged2e}
\usepackage{graphicx}
\usepackage{gensymb}
\usepackage{mathtools}
\newcommand{\mydet}[1]{\ensuremath{\begin{vmatrix}#1\end{vmatrix}}}
\providecommand{\brak}[1]{\ensuremath{\left(#1\right)}}
\providecommand{\norm}[1]{\left\lVert#1\right\rVert}
\newcommand{\solution}{\noindent \textbf{Solution: }}
\newcommand{\myvec}[1]{\ensuremath{\begin{pmatrix}#1\end{pmatrix}}}
\let\vec\mathbf
\begin{document}
\begin{center}
        \textbf\large{CHAPTER-7 \\ TRIANGLES}
\end{center}
\section{Exercise 7.1}
Q2. \textbf{Construction}\\
\fi
\begin{figure}[h!]
	\begin{center}
		\includegraphics[width=\columnwidth]{chapters/9/7/1/2/figs/fig.png}
	\end{center}
	\caption{Quadrilateral ABCD}
	\label{fig:chapters/9/7/1/2/Fig}
\end{figure}
The input parameters for construction are shown in \tabref{tab:chapters/9/7/1/2/Table1}
\begin{table}[h!]
    \centering
    \begin{tabular}{|c|c|c|}
  \hline
  \textbf{Symbol}&\textbf{Value}&\textbf{Description}\\
  \hline
  $a$ & 8 & $BC$\\
  \hline
	$\angle{B}$ & 45$\degree{}$ & $\angle{B}$ in $\triangle$$ABC$ \\
  \hline
	$k$ & 3.5 & $AB-AC$ i.e $c-b$ \\
  \hline 
	$\vec{e_2}$ & $\myvec{
			0\\
			1\\
			}$ & Basis vector\\
 \hline			
\end{tabular}

    \caption{Parameters}
    \label{tab:chapters/9/7/1/2/Table1}
\end{table}
Let
\begin{align}
\vec{A} =& \myvec{0\\0},\vec{B} = \myvec{a\\0},\vec{C} = \myvec{c\cos\theta\\c\sin\theta},\vec{D} = \myvec{-c\cos\theta\\c\sin\theta}
\end{align}
\begin{align}
	AB: 	\vec{n}^{\top}\vec{x} = 0,\\
\end{align}
where
\begin{align}
\vec{n} = \myvec{1\\0}
\end{align}
  From the above assumptions, we get the coordinates of $C$ and $D$ as
  \begin{align}
\vec{C} =& \myvec{4.3\\-2.5},\vec{D} = \myvec{-4.3\\-2.5}
  \end{align}
Let 
    \begin{align}
\theta_1=&\angle ADB
    \end{align}
    Since
    \begin{align}
\vec{m_1}&=\vec{D}-\vec{A}=\myvec{-4.7\\-2.5}, \vec{m_2}=\vec{D}-\vec{B}=\myvec{-13.7\\-2.5}\\
\theta_1&=\cos^{-1}\frac{\vec{m_1}^\top\vec{m_2}}{\norm{\vec{m_1}}\norm{\vec{m_2}}}\\
&=\cos^{-1}\frac{\myvec{-4.7&-2.5}\myvec{-13.7\\-2.5}}{(9.2)(15.8)}=61\degree 
\label{eq:chapters/9/7/1/2/1}
    \end{align}
    Similalrly, letting
    \begin{align}
	    \theta_2&=\angle ACB,\\
\vec{n_1}&=\vec{C}-\vec{A}=\myvec{4.7\\-2.5}, \vec{n_2}=\vec{C}-\vec{B}=\myvec{13.7\\-2.5}\\
\theta_2 &=\cos^{-1}\frac{\vec{n_1}^\top\vec{n_2}}{\norm{\vec{n_1}}\norm{\vec{n_2}}}\\
&=\cos^{-1}\frac{\myvec{4.7&-2.5}\myvec{13.7\\-2.5}}{(9.2)(15.8)}=61\degree 
\label{eq:chapters/9/7/1/2/2}
\end{align}
From \eqref{eq:chapters/9/7/1/2/1} and \eqref{eq:chapters/9/7/1/2/2},
\begin{align}
\angle ABD = \angle CAB 
\end{align}
Since all the angles and sides of triangles $CAB$ and $CAD$ are equal,
\begin{align}
    \triangle{ACB} & \cong \triangle{ADB}
\end{align}

\item $AD$ and $BC$ are equal perpendiculars to a line segment $AB$. Show that $CD$ bisects $AB$.\\
	\solution
\iffalse
\documentclass{article}
\usepackage{amsmath}
\usepackage{xcolor}
\usepackage{gensymb}
\usepackage{ragged2e}
\usepackage{graphicx}
\usepackage{gensymb}
\usepackage{mathtools}
\newcommand{\mydet}[1]{\ensuremath{\begin{vmatrix}#1\end{vmatrix}}}
\providecommand{\brak}[1]{\ensuremath{\left(#1\right)}}
\providecommand{\norm}[1]{\left\lVert#1\right\rVert}
\newcommand{\solution}{\noindent \textbf{Solution: }}
\newcommand{\myvec}[1]{\ensuremath{\begin{pmatrix}#1\end{pmatrix}}}
\let\vec\mathbf 


\begin{document}
\begin{center}
        \textbf\large{CHAPTER-7 \\ TRIANGLES}
\end{center}
\section{Exercise 7.1}
Q3. 
\textbf{Construction}\\
\fi
See 
	\figref{fig:chapters/9/7/1/3/Fig1}
\begin{figure}[h]
	\begin{center}
		\includegraphics[width=\columnwidth]{chapters/9/7/1/3/figs/Figure1.png}
	\end{center}
	\caption{}
	\label{fig:chapters/9/7/1/3/Fig1}
\end{figure}
The input parameters for construction are shown in \tabref{tab:chapters/9/7/1/3/Table1}
\begin{table}[h]
	  \centering
	  \begin{tabular}{|c|c|c|}
  \hline
  \textbf{Symbol}&\textbf{Value}&\textbf{Description}\\
  \hline
  $a$ & 8 & $BC$\\
  \hline
	$\angle{B}$ & 45$\degree{}$ & $\angle{B}$ in $\triangle$$ABC$ \\
  \hline
	$k$ & 3.5 & $AB-AC$ i.e $c-b$ \\
  \hline 
	$\vec{e_2}$ & $\myvec{
			0\\
			1\\
			}$ & Basis vector\\
 \hline			
\end{tabular}

	  \caption{Parameters}
	  \label{tab:chapters/9/7/1/3/Table1}
\end{table}
Let
\begin{align}
	\vec{A} = a\vec{e_1},\vec{B} = \myvec{a\\b},\vec{C} = \myvec{2a\\b},\vec{D} = \myvec{0\\0}
\end{align}
\solution
Given
\begin{align}
	\norm{\vec{D}-\vec{A}}&=\norm{\vec{B}-\vec{C}}
	\\
	\angle DAB &= \angle CBA=90\degree
\end{align}
\begin{align}
	\frac{1}{2}(\vec{C}+\vec{D}) &= \frac{1}{2}\myvec{a \\ \frac{b}{2}}
	\\
	\frac{1}{2}(\vec{A}+\vec{B}) &= \frac{1}{2}\myvec{a \\ \frac{b}{2}}
		 \label{eq:chapters/9/7/1/3/1}\\
\end{align}
Thus, $CD$ bisects $AB$.


\end{enumerate}
