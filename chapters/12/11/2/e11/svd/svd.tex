\iffalse
\documentclass[journal,12pt,twocolumn]{IEEEtran}
\usepackage{romannum}
\usepackage{float}
\usepackage{setspace}
\usepackage{gensymb}
\singlespacing
\usepackage[cmex10]{amsmath}
\usepackage{amsthm}
\usepackage{mathrsfs}
\usepackage{txfonts}
\usepackage{stfloats}
\usepackage{bm}
\usepackage{cite}
\usepackage{cases}
\usepackage{subfig}
\usepackage{longtable}
\usepackage{multirow}
\usepackage{enumitem}
\usepackage{mathtools}
\usepackage{steinmetz}
\usepackage{tikz}
\usepackage{circuitikz}
\usepackage{verbatim}
\usepackage{tfrupee}
\usepackage[breaklinks=true]{hyperref}
\usepackage{tkz-euclide}
\usetikzlibrary{calc,math}
\usepackage{listings}
    \usepackage{color}                                            %%
    \usepackage{array}                                            %%
    \usepackage{longtable}                                        %%
    \usepackage{calc}                                             %%
    \usepackage{multirow}                                         %%
    \usepackage{hhline}                                           %%
    \usepackage{ifthen}                                           %%
  %optionally (for landscape tables embedded in another document): %%
    \usepackage{lscape}     
\usepackage{multicol}
\usepackage{chngcntr}
\DeclareMathOperator*{\Res}{Res}
\renewcommand\thesection{\arabic{section}}
\renewcommand\thesubsection{\thesection.\arabic{subsection}}
\renewcommand\thesubsubsection{\thesubsection.\arabic{subsubsection}}

\renewcommand\thesectiondis{\arabic{section}}
\renewcommand\thesubsectiondis{\thesectiondis.\arabic{subsection}}
\renewcommand\thesubsubsectiondis{\thesubsectiondis.\arabic{subsubsection}}

% correct bad hyphenation here
\hyphenation{op-tical net-works semi-conduc-tor}
\def\inputGnumericTable{}                                 %%

\lstset{
frame=single, 
breaklines=true,
columns=fullflexible
}

\begin{document}


\newtheorem{theorem}{Theorem}[section]
\newtheorem{problem}{Problem}
\newtheorem{proposition}{Proposition}[section]
\newtheorem{lemma}{Lemma}[section]
\newtheorem{corollary}[theorem]{Corollary}
\newtheorem{example}{Example}[section]
\newtheorem{definition}[problem]{Definition}
\newcommand{\BEQA}{\begin{eqnarray}}
\newcommand{\EEQA}{\end{eqnarray}}
\newcommand{\define}{\stackrel{\triangle}{=}}

\bibliographystyle{IEEEtran}
\providecommand{\mbf}{\mathbf}
\providecommand{\pr}[1]{\ensuremath{\Pr\left(#1\right)}}
\providecommand{\qfunc}[1]{\ensuremath{Q\left(#1\right)}}
\providecommand{\sbrak}[1]{\ensuremath{{}\left[#1\right]}}
\providecommand{\lsbrak}[1]{\ensuremath{{}\left[#1\right.}}
\providecommand{\rsbrak}[1]{\ensuremath{{}\left.#1\right]}}
\providecommand{\brak}[1]{\ensuremath{\left(#1\right)}}
\providecommand{\lbrak}[1]{\ensuremath{\left(#1\right.}}
\providecommand{\rbrak}[1]{\ensuremath{\left.#1\right)}}
\providecommand{\cbrak}[1]{\ensuremath{\left\{#1\right\}}}
\providecommand{\lcbrak}[1]{\ensuremath{\left\{#1\right.}}
\providecommand{\rcbrak}[1]{\ensuremath{\left.#1\right\}}}
\theoremstyle{remark}
\newtheorem{rem}{Remark}
\newcommand{\sgn}{\mathop{\mathrm{sgn}}}
\providecommand{\abs}[1]{\left\vert#1\right\vert}
\providecommand{\res}[1]{\Res\displaylimits_{#1}} 
\providecommand{\norm}[1]{\left\lVert#1\right\rVert}
\providecommand{\mtx}[1]{\mathbf{#1}}
\providecommand{\mean}[1]{E\left[ #1 \right]}
\providecommand{\fourier}{\overset{\mathcal{F}}{ \rightleftharpoons}}
\providecommand{\system}{\overset{\mathcal{H}}{ \longleftrightarrow}}
\newcommand{\solution}{\noindent \textbf{Solution: }}
\newcommand{\cosec}{\,\text{cosec}\,}
\providecommand{\dec}[2]{\ensuremath{\overset{#1}{\underset{#2}{\gtrless}}}}
\newcommand{\myvec}[1]{\ensuremath{\begin{pmatrix}#1\end{pmatrix}}}
\newcommand{\mydet}[1]{\ensuremath{\begin{vmatrix}#1\end{vmatrix}}}
\numberwithin{equation}{subsection}
\makeatletter
\@addtoreset{figure}{problem}
\makeatother

\let\StandardTheFigure\thefigure
\let\vec\mathbf
\renewcommand{\thefigure}{\theproblem}



\def\putbox#1#2#3{\makebox[0in][l]{\makebox[#1][l]{}\raisebox{\baselineskip}[0in][0in]{\raisebox{#2}[0in][0in]{#3}}}}
     \def\rightbox#1{\makebox[0in][r]{#1}}
     \def\centbox#1{\makebox[0in]{#1}}
     \def\topbox#1{\raisebox{-\baselineskip}[0in][0in]{#1}}
     \def\midbox#1{\raisebox{-0.5\baselineskip}[0in][0in]{#1}}

\vspace{3cm}


\title{Assignment 1}
\author{Jaswanth Chowdary Madala}





% make the title area
\maketitle

\newpage

%\tableofcontents

\bigskip

\renewcommand{\thefigure}{\theenumi}
\renewcommand{\thetable}{\theenumi}

\begin{enumerate}
\item Find the shortest distance between the lines $l_1$ and $l_2$ whose vector equations are ${\overrightarrow{r} = \hat{i}+\hat{j}+\lambda(2\hat{i}-\hat{j}+\hat{k})}$ and ${\overrightarrow{r} = 2\hat{i}+\hat{j}-\hat{k}+\mu(3\hat{i}-5\hat{j}+2\hat{k})}$

\textbf{Solution:} 
\fi
The shortest distance between the lines whose vector equations are
\begin{align}
L_1: \vec{x} = \vec{x_1} + \lambda_1\vec{m_1} \label{eq:chapters/12/11/2/311/svd/1} \\
L_2: \vec{x} = \vec{x_2} + \lambda_2\vec{m_2} \label{eq:chapters/12/11/2/311/svd/2}
\end{align}
is given by,
\begin{align}
d = \norm{\brak{\vec{U}\brak{\vec{\Sigma\Sigma}^{-1}}\vec{U}^\top-\vec{I}}\vec{x}}
\end{align}
with the parameter $\lambda$ given by
\begin{align}
\bm{\lambda} = \vec{V\Sigma}^{-1}\vec{U}^\top\vec{x} \label{eq:chapters/12/11/2/311/svd/lambda-sol}
\end{align}
where
\begin{align}
\vec{M} &\triangleq \myvec{\vec{m_1} & \vec{m_2}} \label{eq:chapters/12/11/2/311/svd/3}\\
\bm{\lambda} &\triangleq \myvec{\lambda_1\\-\lambda_2}\label{eq:chapters/12/11/2/311/svd/4} \\
\vec{x} &\triangleq \vec{x_2} - \vec{x_1} \label{eq:chapters/12/11/2/311/svd/5}
\end{align}

We use singular value decomposition of the matrix $\vec{M}$
\begin{align}
\vec{M} = \vec{U\Sigma V}^\top \label{eq:chapters/12/11/2/311/svd/6}
\end{align}
where $\vec{U}, \vec{V}$ are orthogonal and $\vec{\Sigma}$ is diagonal with nonnegative diagonal entries.

\begin{enumerate}
\item In this problem we have the lines $l_1$ and $l_2$ as
\begin{align}
\vec{x} &= \myvec{1\\1\\0} + \lambda_1\myvec{2\\-1\\1}\\
\vec{x} &= \myvec{2\\1\\-1} + \lambda_2\myvec{3\\-5\\2}
\end{align}

We first need to check whether the given lines are skew.
The lines \eqref{eq:chapters/12/11/2/311/svd/1}, \eqref{eq:chapters/12/11/2/311/svd/2} intersect if
\begin{align}
\vec{M}\bm{\lambda} &= \vec{x_2} - \vec{x_1}
\end{align}
Here we have,
\begin{align}
\vec{M} &= \myvec{2&3\\-1&-5\\1&2} \label{eq:chapters/12/11/2/311/svd/M}\\
\vec{x} = \vec{x_2} - \vec{x_1} &= \myvec{1\\0\\-1} \label{eq:chapters/12/11/2/311/svd/x}
\end{align}
We check whether the equation \eqref{eq:chapters/12/11/2/311/svd/7} has a solution
\begin{align}
\myvec{2&3\\-1&-5\\1&2}\bm{\lambda} = \myvec{1\\0\\-1}
\label{eq:chapters/12/11/2/311/svd/7}
\end{align}
the augmented matrix is given by,
\begin{align}
&\myvec{2&3&\vrule&1\\-1&-5&\vrule&0\\1&2&\vrule&-1}\\
\xleftrightarrow[R_3 \leftarrow R_3 - \frac{1}{2}R_1]{R_2 \leftarrow R_2 + \frac{1}{2}R_1}
&\myvec{2&3&\vrule&1\\&&\vrule\\0&-\frac{7}{2}&\vrule&\frac{1}{2}\\&&\vrule\\0&\frac{1}{2}&\vrule&-\frac{3}{2}}\\
\xleftrightarrow{R_3 \leftarrow R_3 + 7R_2}
&\myvec{2&3&\vrule&1\\&&\vrule\\0&-\frac{7}{2}&\vrule&\frac{1}{2}\\&&\vrule\\0&0&\vrule&-10}
\end{align}
The rank of the matrix is 3. So the given lines are skew.

\item From \eqref{eq:chapters/12/11/2/311/svd/M} we have
\begin{align}
\vec{M}^\top\vec{M} &= \myvec{2&-1&1\\3&-5&2}\myvec{2&3\\-1&-5\\1&2} \\ 
&= \myvec{6&13\\13&38} \label{eq:chapters/12/11/2/311/svd/MtM}
\end{align}
\begin{align}
\vec{MM}^\top &= \myvec{2&3\\-1&-5\\1&2}\myvec{2&-1&1\\3&-5&2}\\
&= \myvec{13&-17&8\\-17&26&-11\\8&-11&5} \label{eq:chapters/12/11/2/311/svd/MMt}
\end{align}
We perform the eigen decompositions for the matrics \eqref{eq:chapters/12/11/2/311/svd/MMt}, \eqref{eq:chapters/12/11/2/311/svd/MtM} and write them in the form
\begin{align}
    \vec{MM}^\top &= \vec{P_1D_1P_1}^\top \label{eq:chapters/12/11/2/311/svd/decomp-1} \\
    \vec{M}^\top\vec{M} &= \vec{P_2D_2P_2}^\top \label{eq:chapters/12/11/2/311/svd/decomp-2}
\end{align}
The characteristic polynomial of the matrix $\vec{MM}^\top$ is given by,
\begin{align}
\text{char}\brak{\vec{MM}^\top} &= \mydet{13-x&-17&8\\-17&26-x&-11\\8&-11&5-x} \\
&= -x^3 + 44x^2-59x
%\label{eq:chapters/12/11/2/311/svd/char-1}
\end{align}
Thus, the eigenvalues are given by
\begin{align}
\lambda_1 = 22+5\sqrt{17},\ \lambda_2 = 22-5\sqrt{17},\ \lambda_3 = 0
\end{align}
From the augmented matrix formed from the eigen value - eigen vector equation we get, the normalized eigen vectors as
\begin{align}
    \vec{p_1} &= \frac{\sqrt{5}}{\sqrt{68-6\sqrt{17}}} \myvec{\frac{12-\sqrt{17}}{5}\\\frac{1-3\sqrt{17}}{5}\\1}\\
    \vec{p_2} &= \frac{\sqrt{5}}{\sqrt{68+6\sqrt{17}}} \myvec{\frac{12+\sqrt{17}}{5}\\\frac{1+3\sqrt{17}}{5}\\1}\\
    \vec{p_3} &= \frac{1}{\sqrt{59}}\myvec{-3\\1\\7}
\end{align}
where $\vec{p_1},\vec{p_2},\vec{p_3}$ corresponds to the  eigen values $\lambda_1, \lambda_2, \lambda_3$ respectively. Using \eqref{eq:chapters/12/11/2/311/svd/decomp-1}, we get
\begin{align}
    \vec{P_1} &= \myvec{\frac{12-\sqrt{17}}{\sqrt{5}\sqrt{68-6\sqrt{17}}} & \frac{12+\sqrt{17}}{\sqrt{5}\sqrt{68+6\sqrt{17}}} & -\frac{3}{\sqrt{59}}\\
    \frac{1-3\sqrt{17}}{\sqrt{5}\sqrt{68-6\sqrt{17}}}&\frac{1+3\sqrt{17}}{\sqrt{5}\sqrt{68+6\sqrt{17}}} & \frac{1}{\sqrt{59}}\\
\frac{\sqrt{5}}{\sqrt{68-6\sqrt{17}}}&\frac{\sqrt{5}}{\sqrt{68+6\sqrt{17}}} & \frac{7}{\sqrt{59}} }
    \label{eq:chapters/12/11/2/311/svd/eig-params-1(a)}
\end{align}
\begin{align}
    \vec{D_1} &= \myvec{22+5\sqrt{17}&0&0\\0&22-5\sqrt{17}&0\\0&0&0}
    \label{eq:chapters/12/11/2/311/svd/eig-params-1(b)}
\end{align}

For $\vec{M}^\top\vec{M}$, the characteristic polynomial is
\begin{align}
    \text{char}\brak{\vec{M}^\top\vec{M}} &= \mydet{6-x&13\\13&38-x} \\&= x^2-44x+59
    \label{eq:chapters/12/11/2/311/svd/char-1}
\end{align}
Thus, the eigenvalues are given by
\begin{align}
    \lambda_1 = 22+5\sqrt{17},\ \lambda_2 = 22-5\sqrt{17}
\end{align}
From the augmented matrix formed from the eigen value - eigen vector equation we get, the normalized eigen vectors as
\begin{align}
\vec{p_1} &= \frac{13}{\sqrt{850-160\sqrt{17}}}\myvec{\frac{-16+5\sqrt{17}}{13}\\1}\\
\vec{p_2} &= \frac{13}{\sqrt{850+160\sqrt{17}}} \myvec{\frac{-16-5\sqrt{17}}{13}\\1}
\end{align}
where $\vec{p_1},\vec{p_2}$ corresponds to the  eigen values $\lambda_1, \lambda_2$ respectively. Using \eqref{eq:chapters/12/11/2/311/svd/decomp-2}, we get
\begin{align}
    \vec{P_2} &= \myvec{\frac{-16-5\sqrt{17}}{\sqrt{850+160\sqrt{17}}}&\frac{13}{\sqrt{850-160\sqrt{17}}}\\\frac{13}{\sqrt{850+160\sqrt{17}}}&\frac{-16+5\sqrt{17}}{\sqrt{850-160\sqrt{17}}}}
     \label{eq:chapters/12/11/2/311/svd/eig-params-2(a)}\\ 
    \vec{D_2} &= \myvec{22-5\sqrt{17}&0\\0&22+5\sqrt{17}}
    \label{eq:chapters/12/11/2/311/svd/eig-params-2(b)}
\end{align}
Therefore, from \eqref{eq:chapters/12/11/2/311/svd/6} we have
\begin{align}
    \vec{U} &= \vec{P_1} \\ 
    \vec{V} &= \vec{P_2} \\
    \vec{\Sigma} &= \myvec{\sqrt{22+5\sqrt{17}}&0\\0&\sqrt{22-5\sqrt{17}}\\0&0}
    \label{eq:chapters/12/11/2/311/svd/svd-params}
\end{align}
and substituting into \eqref{eq:chapters/12/11/2/311/svd/lambda-sol}, we get
\begin{align}
    \bm{\lambda} =  \myvec{\frac{25}{59}\\\\-\frac{7}{59}}
\end{align}
The minimum distance between the lines is given by,
\begin{align}
\norm{\vec{B}-\vec{A}} &= \norm{\frac{1}{59}\myvec{30\\-10\\-70}}\\
&= \frac{\sqrt{30^2+10^2+70^2}}{59}\\
&= \frac{10}{\sqrt{59}}
\end{align}
The shortest distance between the given lines is $\frac{10}{\sqrt{59}}$ units.
See Fig. 
	\ref{fig:chapters/12/11/2/e11/svd/}.
\begin{figure}[!ht]
\centering
\includegraphics[width=\columnwidth]{./chapters/12/11/2/e11/svd/figs/skew.png}
\caption{$AB$ is the required shortest distance.}
	\label{fig:chapters/12/11/2/e11/svd/}
\end{figure}
\end{enumerate} 
