\documentclass[12pt]{article}
\usepackage{graphicx}
\usepackage{amsmath}
\usepackage{mathtools}
\usepackage{gensymb}
\usepackage[utf8]{inputenc}
\usepackage{float}
\usepackage{setspace}
\newcommand{\mydet}[1]{\ensuremath{\begin{vmatrix}#1\end{vmatrix}}}
\providecommand{\brak}[1]{\ensuremath{\left(#1\right)}}
\providecommand{\norm}[1]{\left\lVert#1\right\rVert}
\newcommand{\solution}{\noindent \textbf{Solution: }}
\newcommand{\myvec}[1]{\ensuremath{\begin{pmatrix}#1\end{pmatrix}}}
\let\vec\mathbf

\begin{document}
\begin{center}
\textbf\large{CLASS-12 \\ CHAPTER-11 \\ THREE DIMENSIONAL GEOMETRY}
\end{center}
\section*{Excercise 11.2}

Q1. Show that the three lines with direction cosines $\frac{12}{13}, \frac{-3}{13}, \frac{-4}{13}; \frac{4}{13}, \frac{12}{13}, \frac{3}{13}; \frac{3}{13}, \frac{-4}{13}, \frac{12}{13}$ are mutually perpendicular.
\\
\solution
	\begin{align}
			\vec{A}=\myvec{\frac{12}{13}\\[2pt]\frac{-3}{13}\\[2pt]\frac{-4}{13}},\vec{B}=\myvec{\frac{4}{13}\\[2pt]\frac{12}{13}\\[2pt]\frac{3}{13}},\vec{C}=\myvec{\frac{3}{13}\\[2pt]\frac{-4}{13}\\[2pt]\frac{12}{13}}
		\end{align}
		Stack all three vectors into a single matrix $\vec{P}$:
			\begin{align}
		\vec{P}=\myvec{\frac{12}{13}&\frac{4}{13}    &\frac{3}{13}\\[2pt] \frac{-3}{13}&\frac{12}{13}&\frac{-4}{13}\\[2pt] \frac{-4}{13}&\frac{3}{13}&\frac{12}{13}}, \vec{P}^\top=\myvec{\frac{12}{13}&\frac{-3}{13}   &\frac{-4}{13}\\[2pt] \frac{4}{13}&\frac{12}{13}&\frac{3}{13}\\[2pt] \frac{3}{13}&\frac{-4}{13}&\frac{12}{13}}
			\end{align}
			\begin{enumerate}
				\item Check whether all three vectors are orthogonal to each other or not using: 
			\begin{align}
		\vec{P}\vec{P}^\top=\vec{I}\\
				\myvec{\frac{12}{13}&\frac{4}{13}    &\frac{3}{13}\\[2pt] \frac{-3}{13}&\frac{12}{13}&\frac{-4}{13}\\[2pt] \frac{-4}{13}&\frac{3}{13}&\frac{12}{13}}\myvec{\frac{12}{13}&\frac{-3}{13}   &\frac{-4}{13}\\[2pt] \frac{4}{13}&\frac{12}{13}&\frac{3}{13}\\[2pt] \frac{3}{13}&\frac{-4}{13}&\frac{12}{13}}&=\myvec{1&0&0\\[2pt]0&1&0\\[2pt]0&0&1}=\vec{I}
		\end{align}
	\item Check whether all three vectors are orthogonal to each other or not using:
\begin{align}
	\vec{P}^\top\vec{P}=\vec{I}\\
				\myvec{\frac{12}{13}&\frac{-3}{13}    &\frac{-4}{13}\\[2pt] \frac{4}{13}&\frac{12}{13}&\frac{3}{13}\\[2pt] \frac{3}{13}&\frac{-4}{13}&\frac{12}{13}}\myvec{\frac{12}{13}&\frac{4}{13}   &\frac{3}{13}\\[2pt] \frac{-3}{13}&\frac{12}{13}&\frac{-4}{13}\\[2pt] \frac{-4}{13}&\frac{3}{13}&\frac{12}{13}}&=\myvec{1&0&0\\[2pt]0&1&0\\[2pt]0&0&1}=\vec{I}
		\end{align}
			\end{enumerate}
					Hence, it satisfies both the above conditions, shows  that all three vectors are orthogonal(perpendicular) to each other.
			
\end{document}



