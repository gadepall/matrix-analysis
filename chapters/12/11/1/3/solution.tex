\documentclass[12pt]{article}
\usepackage{graphicx}
%\documentclass[journal,12pt,twocolumn]{IEEEtran}
\usepackage[none]{hyphenat}
\usepackage{graphicx}
\usepackage{listings}
\usepackage[english]{babel}
\usepackage{graphicx}
\usepackage{caption}
\usepackage[parfill]{parskip}
\usepackage{hyperref}
\usepackage{booktabs}
%\usepackage{setspace}\doublespacing\pagestyle{plain}
\def\inputGnumericTable{}
\usepackage{color}                                            %%
    \usepackage{array}                                            %%
    \usepackage{longtable}                                        %%
    \usepackage{calc}                                             %%
    \usepackage{multirow}                                         %%
    \usepackage{hhline}                                           %%
    \usepackage{ifthen}
\usepackage{array}
\usepackage{amsmath}   % for having text in math mode
\usepackage{parallel,enumitem}
\usepackage{listings}
\lstset{
language=tex,
frame=single,
breaklines=true
}
 
%Following 2 lines were added to remove the blank page at the beginning
\usepackage{atbegshi}% http://ctan.org/pkg/atbegshi
\AtBeginDocument{\AtBeginShipoutNext{\AtBeginShipoutDiscard}}
%
%New macro definitions
\newcommand{\mydet}[1]{\ensuremath{\begin{vmatrix}#1\end{vmatrix}}}
\providecommand{\brak}[1]{\ensuremath{\left(#1\right)}}
\providecommand{\norm}[1]{\left\lVert#1\right\rVert}
\newcommand{\solution}{\noindent \textbf{Solution: }}
\newcommand{\myvec}[1]{\ensuremath{\begin{pmatrix}#1\end{pmatrix}}}
\let\vec\mathbf
\begin{document}
\begin{center}
\enlargethispage{-4cm}
\title{\textbf{Three Dimensional Geometry}}
\date{\vspace{-5ex}} %Not to print date automatically
\maketitle
\end{center}
\setcounter{page}{1}
\section*{12$^{th}$ Maths - Chapter 11}
This is Problem-3 from Exercise 11.1
\begin{enumerate}
\item If a line has the direction ratios –18, 12, –4, then what are its direction cosines ?

\solution let $\vec{A}$ be the given vector
\begin{align}
	\vec{A} =\myvec{-18\\12\\-4}
\end{align}
Then $\vec{B}$ be the unit vector in the direction of $\vec{A}$ then direction cosine vector is given by  
\begin{align}
		\vec{B}=\frac{\vec{A}}{\norm{\vec{A}}}
\end{align}
		The magnitude for $\vec{A}$ is 
	\begin{align}
	\norm{\vec{A}}=22
	\end{align}
		Then direction cosine vector $\vec{B}$ is 
\begin{align}
	\vec{B}=\myvec{\frac{-9}{11}\\[2pt] \frac{6}{11}\\[2pt] \frac{-2}{11}}
\end{align}
\end{enumerate}
\end{document}
