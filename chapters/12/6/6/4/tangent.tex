\iffalse
\documentclass[journal,10pt,twocolumn]{article}
\usepackage{graphicx, float}
\usepackage[margin=0.5in]{geometry}
\usepackage{amsmath, bm}
\usepackage{array}
\usepackage{booktabs}
\usepackage[utf8]{inputenc}
\usepackage{amsfonts}
\usepackage{amssymb}
\usepackage{graphicx}
\usepackage{multicol}
\usepackage{tabularx}
\usepackage{hyperref}
\usepackage{mathtools}
\DeclareUnicodeCharacter{2212}{-}
\providecommand{\norm}[1]{\left\lVert#1\right\rVert}
\providecommand{\abs}[1]{\left\vert#1\right\vert}
\let\vec\mathbf
\newcommand{\myvec}[1]{\ensuremath{\myvec{#1}}}
\newcommand{\mydet}[1]{\ensuremath{\begin{vmatrix}#1\end{vmatrix}}}
\providecommand{\brak}[1]{\ensuremath{\left(#1\right)}}
\providecommand{\lbrak}[1]{\ensuremath{\left(#1\right.}}
\providecommand{\rbrak}[1]{\ensuremath{\left.#1\right)}}
\providecommand{\sbrak}[1]{\ensuremath{{}\left[#1\right]}}
%\providecommand{\norm}[1]{\left\lVert#1\right\rVert}
%\providecommand{\sbrak}[1]{\ensuremath{{}\left[#1\right]}}
%\providecommand{\lsbrak}[1]{\ensuremath{{}\left[#1\right.}}
%\providecommand{\rsbrak}[1]{\ensuremath{{}\left.#1\right]}}
%\providecommand{\brak}[1]{\ensuremath{\left(#1\right)}}
%\providecommand{\lbrak}[1]{\ensuremath{\left(#1\right.}}
%\providecommand{\rbrak}[1]{\ensuremath{\left.#1\right)}}
%\providecommand{\cbrak}[1]{\ensuremath{\left\{#1\right\}}}
%\providecommand{\lcbrak}[1]{\ensuremath{\left\{#1\right.}}
%\providecommand{\rcbrak}[1]{\ensuremath{\left.#1\right\}}}
%\newcommand{\myvec}[1]{\ensuremath{\myvec{#1}}}
%\let\vec\mathbf

\title{\textbf{Conic Assignment}}
\author{Navya Valmeekam \hspace{9cm} FWC22042}
\date{September 2022}

\begin{document}

\maketitle
\fi
Find the equation of the normal to curve $x^2 = 4y$ which passes through the point
(1, 2).
\\
\solution 
	\begin{figure}[!h]
		\centering
 \includegraphics[width=\columnwidth]{chapters/12/6/6/4/figs/conics.png}
		\caption{}
		\label{fig:12/6/6/4}
  	\end{figure}
\iffalse
\section*{\large Solution}

\begin{figure}[H]
\centering
\includegraphics[scale=0.45]{conics.png} 
\caption{Tangents from A to circle through B, C and D}
\label{fig:triangle}
\end{figure}

The given equation of parabola $x^2 = 4y$ can be written in the general quadratic form as
\begin{align}
    \label{eq:conic_quad_form}
    \vec{x}^{\top}\vec{V}\vec{x}+2\vec{u}^{\top}\vec{x}+f=0
    \end{align}
where
\fi
The conic parameters are
\begin{align}
	\vec{V} = \myvec{1 & 0\\0 & 0},
	\vec{u} = \myvec{0\\-2},
	f &= 0
	%\\
\end{align}
 Substituting these values in
	\eqref{eq:point_of_tangency-m},
% \eqref{eq:normal_solution}, 
 we obtain
\begin{align}
m = -1
\end{align}
as the only real solution.  Thus, 
\begin{align}
\vec{n} = \myvec{1 \\ 1}, \vec{m} = \myvec{-1 \\ 1}, 
    \mu = -1
\end{align}
upon substituting in 
	\eqref{eq:point_of_tangency-mu}
	and
	\eqref{eq:point_of_tangency-mn}.
	The equation of the normal is then obtained as
	\iffalse
	Substituting the above values in 
	\eqref{eq:point_of_tangency},
\begin{align}
    \vec{q} = \myvec{2 \\ 1}
\end{align}
and the 
equation of normal is given by 
\begin{align}
\vec{n^{\top}x=c}
\end{align}
\begin{align}
\myvec{
1 & 1
}
\myvec{
1 \\
2
}
	&=
c
\end{align}
We get c=3\\
Equation of normal is,
\fi
\begin{align}
	\vec{n}^{\top}\brak{\vec{x}-\vec{h}} &= 0
	\\
	\implies
\myvec{
1 & 1
}
		\vec{x}
	&=
\myvec{
1 & 1
}
	\myvec{1 \\2 }
	\\
	&= 3
\end{align}
