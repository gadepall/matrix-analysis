\documentclass[journal,10pt,twocolumn]{article}
\usepackage{graphicx, float}
\usepackage[margin=0.5in]{geometry}
\usepackage{amsmath, bm}
\usepackage{array}
\usepackage{booktabs}
\usepackage[utf8]{inputenc}
\usepackage{amsfonts}
\usepackage{amssymb}
\usepackage{graphicx}
\usepackage{multicol}
\usepackage{tabularx}
\usepackage{hyperref}
\usepackage{mathtools}
\DeclareUnicodeCharacter{2212}{-}
\providecommand{\norm}[1]{\left\lVert#1\right\rVert}
\providecommand{\abs}[1]{\left\vert#1\right\vert}
\let\vec\mathbf
\newcommand{\myvec}[1]{\ensuremath{\begin{pmatrix}#1\end{pmatrix}}}
\newcommand{\mydet}[1]{\ensuremath{\begin{vmatrix}#1\end{vmatrix}}}
\providecommand{\brak}[1]{\ensuremath{\left(#1\right)}}
\providecommand{\lbrak}[1]{\ensuremath{\left(#1\right.}}
\providecommand{\rbrak}[1]{\ensuremath{\left.#1\right)}}
\providecommand{\sbrak}[1]{\ensuremath{{}\left[#1\right]}}
%\providecommand{\norm}[1]{\left\lVert#1\right\rVert}
%\providecommand{\sbrak}[1]{\ensuremath{{}\left[#1\right]}}
%\providecommand{\lsbrak}[1]{\ensuremath{{}\left[#1\right.}}
%\providecommand{\rsbrak}[1]{\ensuremath{{}\left.#1\right]}}
%\providecommand{\brak}[1]{\ensuremath{\left(#1\right)}}
%\providecommand{\lbrak}[1]{\ensuremath{\left(#1\right.}}
%\providecommand{\rbrak}[1]{\ensuremath{\left.#1\right)}}
%\providecommand{\cbrak}[1]{\ensuremath{\left\{#1\right\}}}
%\providecommand{\lcbrak}[1]{\ensuremath{\left\{#1\right.}}
%\providecommand{\rcbrak}[1]{\ensuremath{\left.#1\right\}}}
%\newcommand{\myvec}[1]{\ensuremath{\begin{pmatrix}#1\end{pmatrix}}}
%\let\vec\mathbf

\title{\textbf{Conic Assignment}}
\author{Srinivas Dulla \hspace{9cm} FWC22041}
\date{September 2022}

\begin{document}

\maketitle
\paragraph{\textit{Problem Statement} - The normal at the point (1,1) on the curve \(2y+x^2=3\) :\\
(a)x+y=0  \hspace{2cm} (b)x-y=0\\ 
(c)x+y+1=0 \hspace{2cm}  (d)x-y=1\\}

\section*{\large Solution}

\begin{figure}[H]
\centering
\includegraphics[width=1\columnwidth]{cone.jpg}
\caption{Tangents from A to circle through B, C and D}
\label{fig:triangle}
\end{figure}

The given equation of parabola 2y$+$x^2 = 3$ can be written in the general quadratic form as
\begin{align}
    \label{eq:conic_quad_form}
    \vec{x}^{\top}\vec{V}\vec{x}+2\vec{u}^{\top}\vec{x}+f=0
    \end{align}
where
\begin{align}
	\label{eq:V_matrix}
	\vec{V} &= \myvec{1 & 0\\0 & 0},
	\\
	\label{eq:u_vector}
	\vec{u} &= \myvec{0\\-2},
	\\
	\label{eq:f_value}
	f &= 3
	%\\
\end{align}

The parabola in (\ref{eq:conic_quad_form}) can be expressed in standard form (center/vertex at origin, major-axis - $x$ axis) as
\begin{align}
	\label{eq:conic_simp_parab}
	\vec{y}^{\top}\vec{D}\vec{y} &=  -2\eta\vec{e}_1^{\top}\vec{y}   & \abs{V} &= 0
\end{align}
 where
\begin{align}
	\label{eq:conic_affine}
	\vec{x} = \vec{P}\vec{y}+\vec{c} \quad \text{(Affine Transformation)}
\end{align}
\begin{align}
	\label{eq:conic_parmas_eig_def}
	\vec{P}^{\top}\vec{V}\vec{P} &= \vec{D}. \quad \text{(Eigenvalue Decomposition)}
	\\
	\label{eq:eigevalV}
	\vec{D} &= \myvec{\lambda_1 & 0\\ 0 & \lambda_2}, 
	\\
	\vec{P} &= \myvec{\vec{p}_1 & \vec{p}_2}, \quad \vec{P}^{\top}=\vec{P}^{-1},
	\label{eq:eigevecP}
	\\
	\label{eq:eta}
	\eta &=\vec{u}^{\top}\vec{p}_1
	\\
	\vec{e}_1 &=\myvec{1 \\ 0}
	\end{align}

To find $\vec{c}$ which is the center of the parabola in (\ref{eq:conic_quad_form}), substitute (\ref{eq:conic_affine}) in (\ref{eq:conic_quad_form})
\begin{multline}
\brak{\vec{P}\vec{y}+\vec{c}}^T\vec{V}\brak{\vec{P}\vec{y}+\vec{c}}+2\vec{u}^T\brak{\vec{P}\vec{y}+\vec{c}} + f = 0, 
\end{multline}
yielding 
\begin{multline}
\vec{y}^T\vec{P}^T\vec{V}\vec{P}\vec{y}+2\brak{\vec{V}\vec{c}+\vec{u}}^T\vec{P}\vec{y} +  \vec{c}^T\vec{V}\vec{c} 
\\
+2\vec{u}^T\vec{c} + f= 0
\label{eq:conic_simp_one}
\end{multline}
%
From \eqref{eq:conic_simp_one} and \eqref{eq:conic_parmas_eig_def},
\begin{multline}
\vec{y}^T\vec{D}\vec{y}+2\brak{\vec{V}\vec{c}+\vec{u}}^T\vec{P}\vec{y} +  \vec{c}^T\brak{\vec{V}\vec{c} + \vec{u}}
\\
+ \vec{u}^T\vec{c} + f= 0
\label{eq:conic_simp}
\end{multline}
For a parabola $\abs{\vec{V}} = 0, \lambda_1 = 0$ and
\begin{align}
\vec{V}\vec{p}_1 = 0, 
\vec{V}\vec{p}_2 = \lambda_2\vec{p}_2.
\label{eq:conic_parab_eig_prop} 
\end{align}
where $\vec{p}_1,\vec{p}_2$ are the eigenvectors of $\vec{V}$ such that  \eqref{eq:conic_parmas_eig_def}
%
\begin{align}
\vec{P} = \myvec{\vec{p}_1 & \vec{p}_2},
\label{eq:eig_matrix}
\end{align}
Substituting \eqref{eq:eig_matrix}
in \eqref{eq:conic_simp},
\begin{multline}
	\vec{y}^T\vec{D}\vec{y}+2\brak{\vec{c}^T\vec{V}+\vec{u}^T}\myvec{\vec{p}_1 & \vec{p}_2}\vec{y}
\\
+  \vec{c}^T\brak{\vec{V}\vec{c} + \vec{u}}+ \vec{u}^T\vec{c} + f= 0
\\
\implies \vec{y}^T\vec{D}\vec{y}
\\
+2\myvec{\brak{\vec{c}^T\vec{V}+\vec{u}^T}\vec{p}_1  \brak{\vec{c}^T\vec{V}+\vec{u}^T}\vec{p}_2}\vec{y}
\\
+  \vec{c}^T\brak{\vec{V}\vec{c} + \vec{u}}+ \vec{u}^T\vec{c} + f= 0
\\
\implies \vec{y}^T\vec{D}\vec{y}
\\
+2\myvec{\vec{u}^T\vec{p}_1 & \brak{\lambda_2\vec{c}^T+\vec{u}^T}\vec{p}_2}\vec{y}
\\
+  \vec{c}^T\brak{\vec{V}\vec{c} + \vec{u}}+ \vec{u}^T\vec{c} + f= 0
\text{ from } \eqref{eq:conic_parab_eig_prop}     \nonumber \\
\\
\implies \lambda_2y_2^2+2\brak{\vec{u}^T\vec{p}_1}y_1+  2y_2\brak{\lambda_2\vec{c}+\vec{u}}^T\vec{p}_2
\\
+  \vec{c}^T\brak{\vec{V}\vec{c} + \vec{u}}+ \vec{u}^T\vec{c} + f= 0
\label{eq:conic_parab_foc_len_temp} 
\end{multline}
which is the equation of a parabola. 
Thus, \eqref{eq:conic_parab_foc_len_temp} 
can be expressed as \eqref{eq:conic_simp_parab} by choosing
\begin{align}
%\label{eq:eta}
\eta = \vec{u}^T\vec{p}_1
\end{align}
and $\vec{c}$ in \eqref{eq:conic_simp} such that
\begin{align}
\label{eq:conic_parab_one}
\vec{P}^{T}\brak{\vec{V}\vec{c}+\vec{u}} &= \eta\myvec{1\\0}
\\
\vec{c}^T\brak{\vec{V}\vec{c} + \vec{u}}+ \vec{u}^T\vec{c} + f&= 0
\label{eq:conic_parab_two}
\end{align}
Multiplying \eqref{eq:conic_parab_one} by $\vec{P}$ yields
\begin{align}
\label{eq:conic_parab_one_eig}
\brak{\vec{V}\vec{c}+\vec{u}} &= \eta\vec{p}_1,
\end{align}
which, upon substituting in \eqref{eq:conic_parab_two}
results in 
\begin{align}
\eta\vec{c}^T\vec{p}_1 + \vec{u}^T\vec{c} + f&= 0
\label{eq:conic_parab_two_eig}
\end{align}
\eqref{eq:conic_parab_one_eig} and \eqref{eq:conic_parab_two_eig} can be clubbed together to obtain \eqref{eq:conic_parab_c}.
\begin{align}
    \myvec{ \vec{u}^{\top}+\eta\vec{p}_1^{\top} \\ \vec{V}}\vec{c} &= \myvec{-f \\ \eta\vec{p}_1-\vec{u}}  &\abs{V} &= 0
    \label{eq:conic_parab_c}
    \end{align}
Substituting appropriate values from \eqref{eq:V_matrix}, \eqref{eq:u_vector}, \eqref{eq:f_value}, \eqref{eq:eigevecP}, and \eqref{eq:eta} into \eqref{eq:conic_parab_c}, the below matrix equation is obtained
\begin{align}
	\label{eq:vertex_system}
	\myvec{0&-4\\1& 0\\0& 0}\vec{c} = \myvec{0 \\0 \\0}\\
\end{align}
The augmented matrix for \eqref{eq:vertex_system} can be expressed as
\begin{align}
	\label{eq:vertex_solv1}
	\myvec{0&-4&\vrule&0\\1&0&\vrule&0\\0&0&\vrule&0}\\ 	
	\label{eq:vertex_solv2}
	\xleftrightarrow[]{R_1 \leftrightarrow R_2}\myvec{1&0&\vrule&0\\0&-4&\vrule&0\\0&0&\vrule&0}\\
	\label{eq:vertex_solv3}
	\xleftrightarrow[]{-\frac{R_2}{4} \leftarrow R_2}\myvec{1&0&\vrule&0\\0&1&\vrule&0\\0&0&\vrule&0}\\
	\label{eq:vertex_solv4}
	\implies\vec{c} = \myvec{0\\0}
\end{align}

Let the point from which normals are drawn be $\vec{h}$. Then, the equation of the normal can be written as
\begin{align}
	\vec{x} = \vec{h} + \lambda\vec{m}
	\label{eq:normal_chord}
\end{align}
Say the point of intersection of \eqref{eq:normal_chord} with the conic is $\vec{q}$. A tangent drawn at $\vec{q}$ satisfies the equation
\begin{align}
	\label{eq:tangency_condition}
	\vec{n}^\top(\vec{Vq}+\vec{u}) = 0
\end{align}
Where $\vec{n}$ is the direction vector of the tangent and is perpendicular to $\vec{m}$ in \eqref{eq:normal_chord}.\\\\
In general, the parameter values for points of intersection of a line given by \eqref{eq:normal_chord} with a conic is given by
{\tiny
\begin{multline}
\lambda_i = \frac{1}
{
\vec{m}^T\vec{V}\vec{m}
}
\lbrak{-\vec{m}^T\brak{\vec{V}\vec{h}+\vec{u}}}
\\
\pm
\rbrak{\sqrt{
\sbrak{
\vec{m}^T\brak{\vec{V}\vec{h}+\vec{u}}
}^2
-
\brak
{
\vec{h}^T\vec{V}\vec{h} + 2\vec{u}^T\vec{h} +f
}
\brak{\vec{m}^T\vec{V}\vec{m}}
}
}
\label{eq:tangent_roots}
\end{multline}
}
Using \eqref{eq:tangent_roots} and \eqref{eq:normal_chord}, the intersection point $\vec{q}$ can be written as
\begin{align}
	\label{eq:point_of_tangency}
	\vec{q} = \vec{h} + \lambda_i\vec{m}
\end{align}
Substituting \eqref{eq:point_of_tangency} in \eqref{eq:tangency_condition},
\begin{align}
	\label{eq:normal_simp_1}
	\vec{n}^\top(\vec{V}(\vec{h}+\lambda_i\vec{m})+\vec{u}) = 0\\
	\label{eq:normal_simp_2}
	\implies \lambda_i\vec{n}^\top\vec{V}\vec{m} = -\vec{n}^\top(\vec{Vh}+\vec{u})
\end{align}
Substituting value of $\lambda_i$ from \eqref{eq:tangent_roots} in \eqref{eq:normal_simp_2}
{\tiny
\begin{multline}
	\frac{1}{\vec{m}^\top\vec{V}\vec{m}}\lbrak{-\vec{m}^\top\brak{\vec{Vh}+\vec{u}}} \\ 
	\pm \rbrak{\sqrt{\sbrak{\vec{m}^T\brak{\vec{V}\vec{h}+\vec{u}}}^2-\brak{\vec{h}^T\vec{V}\vec{h} + 2\vec{u}^T\vec{h} +f}\brak{\vec{m}^T\vec{V}\vec{m}}}}\vec{n}^\top\vec{V}\vec{m} \\
	= -\vec{n}^\top\brak{\vec{Vh}+\vec{u}}
	\label{eq:normal_simp_3}
\end{multline}
}
Rearranging the terms,
{\tiny
\begin{multline}
	\pm \sqrt{\sbrak{\vec{m}^T\brak{\vec{V}\vec{h}+\vec{u}}}^2-\brak{\vec{h}^T\vec{V}\vec{h} + 2\vec{u}^T\vec{h} +f}\brak{\vec{m}^T\vec{V}\vec{m}}} \brak{\vec{n}^\top\vec{V}\vec{m}} \\ = \brak{\vec{Vh}+\vec{u}}^\top\brak{\brak{\vec{n}^\top\vec{V}\vec{m}}\vec{m}-\brak{\vec{m}^\top\vec{V}\vec{m}}\vec{n}}
\end{multline}
}
Squaring on both sides
{\tiny
\begin{multline}
	\sbrak{\sbrak{\vec{m}^T\brak{\vec{V}\vec{h}+\vec{u}}}^2-\brak{\vec{h}^T\vec{V}\vec{h} + 2\vec{u}^T\vec{h} +f}\brak{\vec{m}^T\vec{V}\vec{m}}}\brak{\vec{n}^\top\vec{V}\vec{m}}^2 \\ = \sbrak{\brak{\vec{Vh}+\vec{u}}^\top\brak{\brak{\vec{n}^\top\vec{V}\vec{m}}\vec{m}-\brak{\vec{m}^\top\vec{V}\vec{m}}\vec{n}}}^2
	\label{eq:normal_solution}
\end{multline}
}\\
If $\vec{n}$ is taken as $\myvec{-\mu \\ 1}$, then $\vec{m}$ is $\myvec{-1 \\ -\mu}$. Substituting these values in \eqref{eq:normal_solution} and solving for $\mu$, the different possible normals passing through $\vec{h}$ are obtained.\\\\
Thus after solving we get the following values for $\mu$ = {-1, 1/2 - sqrt(3)*I/2, 1/2 + sqrt(3)*I/2}\\\\
Taking $\mu$=1 we get,
\begin{center}
$\vec{n} = \myvec{1 \\ 1}$, $\vec{m} = \myvec{-1 \\ 1}$\\
\end{center}
By calculating $\lambda_i$ from \eqref{eq:normal_simp_2}, we get
\begin{center}
    $\lambda_i = -1$
\end{center}
We find out $\vec{q}$ from \eqref{eq:point_of_tangency},
\begin{center}
where $\vec{h} = \myvec{1 \\ 2}$, $\vec{m} = \myvec{-1 \\ 1}$, $\lambda_i = -1$
\end{center}
\begin{center}
    $\vec{q} = \myvec{1 \\ 2} + (-1)\myvec{-1 \\ 1} = $\myvec{2 \\ 1}
\end{center}
\begin{center}
    Thus $\vec{q}$ satisfies Option(a) i.e. $x+y=3$
\end{center} 

\section*{\large Construction}
{
\setlength\extrarowheight{5pt}
\begin{tabular}{|c|c|c|}
	\hline
	\textbf{Symbol}&\textbf{Value}&\textbf{Description}\\[5pt]
	\hline
	$\vec{h}$&$\myvec{1 \\ 2}$&Given point through which Normal is passing\\[5pt]
	\hline
	$\vec{q}$&$\myvec{2 \\ 1}$&Foot of Normal\\[5pt]
	\hline
	$\vec{m}$ & $\myvec{-1 \\ 1}$ & Direction Vector of Normal\\[5pt]
	\hline
	$\vec{n}$ & $\myvec{1 \\ 1}$ & Direction Vector of Tangent at $\myvec{q}$\\
	\hline
	$\vec{P}$&\myvec{0&1\\1&0}&eigenvectors of $\vec{V}$\\[5pt]
	\hline
\end{tabular}
}

\end{document}