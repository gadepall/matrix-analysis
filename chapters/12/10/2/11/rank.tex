\iffalse
\documentclass[12pt]{article}
\usepackage{graphicx}
\graphicspath{{./figs/}}{}
\usepackage{amsmath,amssymb,amsfonts,amsthm}
\newcommand{\myvec}[1]{\ensuremath{\begin{pmatrix}#1\end{pmatrix}}}
\providecommand{\norm}[1]{\lVert#1\rVert}
\usepackage{listings}
\usepackage{watermark}
\usepackage{titlesec}
\usepackage{caption}
\usepackage{extarrows}
\let\vec\mathbf
\lstset{
frame=single, 
breaklines=true,
columns=fullflexible
}
\thiswatermark{\centering \put(0,-105.0){\includegraphics[scale=0.15]{/sdcard/IITH/vectors/12.10.2.11/figs/logo.png}} }
\title{\mytitle}
\title{
Assignment - 12.10.2.11
}
\author{Surajit Sarkar}
\begin{document}
\maketitle
\tableofcontents
\bigskip
\section{\textbf{Problem}}
Show that the vectors $2\hat{i}+3\hat{j}+4\hat{k}$ and $-4\hat{i}+6\hat{j}-8\hat{k}$ are collinear.
\section{\textbf{Solution}}
\fi
Let
\begin{align}
\vec{A}=\myvec{2\\3\\4},\vec{B}=\myvec{-4\\6\\-8}\\
 \end{align}
 Forming the collinearity matrix
 \begin{align}        
\myvec{\vec{A}^{\top}\\ \vec{B}^{\top}}=\myvec{2&-3&4\\-4&6&-8}
 \xleftrightarrow{\frac{1}{2}R_1\to R_1}\myvec{1&-\frac{3}{2}&2\\-4&6&-8}\\
\xleftrightarrow{-\frac{1}{4}R_2\leftarrow R_2}\myvec{1&-\frac{3}{2}&2\\1&\frac{3}{2}&2}
\xleftrightarrow{R_2-1R_1\to R_2}\myvec{1&-\frac{3}{2}&2\\0&0&0}
\end{align}
Thus, the rank of the matrix is 1 and the vectors are collinear.
