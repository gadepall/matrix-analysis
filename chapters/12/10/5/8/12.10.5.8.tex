\documentclass[12pt]{article}
\usepackage[hidelinks]{hyperref}
%\documentclass[journal,12pt,twocolumn]{IEEEtran}
\usepackage[none]{hyphenat}
\usepackage{graphicx}
\usepackage{listings}
\usepackage[english]{babel}
\usepackage{graphicx}
\usepackage{caption} 
\usepackage{hyperref}
\usepackage{booktabs}
\def\inputGnumericTable{}
\usepackage{color}                                            %%
\usepackage{array}                                            %%
\usepackage{longtable}                                        %%
\usepackage{calc}                                             %%
\usepackage{multirow}                                         %%
\usepackage{hhline}                                           %%
\usepackage{ifthen}
\usepackage{array}
\usepackage{listings}
\usepackage{caption}
\usepackage{refstyle}
\graphicspath{{/sdcard/Download/vectors/tables}}
\lstset{
language=tex,
frame=single, 
breaklines=true
}
\usepackage{setspace}
\usepackage{gensymb}
\usepackage{xcolor}
\singlespacing
\usepackage{siunitx}
\usepackage[cmex10]{amsmath}
\usepackage{mathtools}
\usepackage{hyperref}
\usepackage{amsthm}
\usepackage{mathrsfs}
\usepackage{txfonts}
\usepackage{stfloats}
\usepackage{cite}
\usepackage{cases}
\usepackage{subfig}
\usepackage{longtable}
\usepackage{multirow}
\usepackage{enumitem}
\usepackage{mathtools}
\usepackage{listings}
\usepackage{tikz}
\usetikzlibrary{shapes,arrows,positioning}
\usepackage{circuitikz}
\let\vec\mathbf
\DeclareMathOperator*{\Res}{Res}
\graphicspath{{/sdcard/Download/vectors/figs}}
%Following 2 lines were added to remove the blank page at the beginning
\usepackage{atbegshi}% http://ctan.org/pkg/atbegshi
\AtBeginDocument{\AtBeginShipoutNext{\AtBeginShipoutDiscard}}
%


%New macro definitions
\newcommand{\mydet}[1]{\ensuremath{\begin{vmatrix}#1\end{vmatrix}}}
\providecommand{\brak}[1]{\ensuremath{\left(#1\right)}}
\providecommand{\norm}[1]{\left\lVert#1\right\rVert}
\newcommand{\solution}{\noindent \textbf{Solution: }}
\newcommand{\myvec}[1]{\ensuremath{\begin{pmatrix}#1\end{pmatrix}}}
\let\vec\mathbf


\begin{document}

\begin{center}
\title{\textbf{VECTOR ALGEBRA}}
\date{\vspace{-5ex}} %Not to print date automatically
\maketitle
\end{center}

\setcounter{page}{1}

\section*{12$^{th}$ Maths - Chapter 10}

This is Problem-8 from Exercise 5.8

 
\begin{enumerate}

	\item Show that the points A $\myvec{1\\-2\\-8}$, B $\myvec{5\\0\\-2}$ and C $\myvec{11\\3\\7}$ are collinear, and find the ratio in which B divides AC.\\


\solution \\The input parameters for this problem are available in Table \ref{Table-1}
\begin{table}[ht!]
\begin{tabular}{|c|c|p{5cm}|}
\hline
\textbf{Symbol} & \textbf{Value} & \textbf{Description} \\
\hline
$\theta$ & $30\degree$ & $\angle{BAP} = \angle{BAQ}$ \\
\hline
$a$ & $9$ & $AB$ \\
\hline
$c$ & $8$ & $AQ$ \\
\hline
$\vec{e}_1$ & $\myvec{1\\0}$ & Basis vector \\
\hline
\end{tabular}

\caption{}
\label{Table-1}	

\end{table}

		Points $\vec{A}$, $\vec{B}$ and $\vec{C}$ are on a line if
    \begin{align}
        \textrm{rank}\myvec{\vec{A} & \vec{B} & \vec{C}} < 3
        \label{eq:chapters/12/10/5/8rank-collinear}
    \end{align}
    Substituting, we must find the rank of
    \begin{align}
        \vec{M} = \myvec{1&5&11\\-2&0&3\\-8&-2&7}
    \end{align}
    Using row reduction methods to bring $\vec{M}$ into row-reduced echelon form,
    \begin{align}
        \myvec{1&5&11\\-2&0&3\\-8&-2&7}&\xleftrightarrow[]{R_2\rightarrow R_2+2R_1}
        \myvec{1&5&11\\0&10&25\\-8&-2&7} \\
                &\xleftrightarrow[]{R_3\rightarrow R_3+8R_1}\myvec{1&5&11\\0&10&25\\0&38&95} \\
                &\xleftrightarrow[]{R_3\rightarrow R_3-\frac{19}{5}R_2}\myvec{1&5&11\\0&10&25\\0&0&0}
                \label{eq:chapters/12/10/5/8row-red}
    \end{align}
    Clearly, the rank of $\vec{M}$ is 2, and hence the given points are 
    collinear. 
    Fig. \ref{fig:Fig1}  verifies that the three points are indeed 
    collinear as claimed.\\
	Let $\vec{B}$ divide $\vec{AC}$ in k:1 then,
	\begin{align}
		\frac{k\vec{C}+\vec{A}}{k+1} = \vec{B}
	\end{align}
		\begin{align}
			\implies k\vec{C}+\vec{A}=\vec{B}\brak{k+1}
			\implies k\brak{\vec{C}-\vec{B}}=\brak{\vec{B}-\vec{A}}
		\end{align}
			Multiplying with $\brak{\vec{C}-\vec{B}}^{\top}$ on both sides,\\
	
		\begin{align*}
			 k\brak{\vec{C}-\vec{B}}\brak{\vec{C}-\vec{B}}^{\top}=\brak{\vec{B}-\vec{A}}{\vec{C}-\vec{B}}^{\top}
		\end{align*}
			The value of k is as follows,
			\begin{align}
			k &=
			\frac{\brak{\vec{B}-\vec{A}}\brak{\vec{C}-\vec{B}}^{\top}}{\norm{\vec{C-B}}^2}
			\label{eq:7}
			\end{align}

			where,
			\begin{align}
				\brak{\vec{B-A}} &=
				\brak{\myvec{5\\0\\-2}-\myvec{1\\-2\\-8}} =
				\myvec{4\\2\\6}
			\end{align}
			
			\begin{align}
				\brak{\vec{C-B}} &=
				\brak{\myvec{11\\3\\7}-\myvec{5\\0\\-2}} =
				\myvec{6\\3\\9}
			\end{align}
			\begin{align*}
				\brak{\vec{C}-\vec{B}}^{\top} &=
				\myvec{6 & 3 & 9}
			\end{align*}
			Substituting the values in \eqref{7} the value of $k$ is $2/3$.
			\\    Hence, $\vec{B}$ divides $\vec{AC}$ in the ratio $2:3$.
	\begin{figure}[!h]
		\begin{center}
			\includegraphics[width=\columnwidth]{figs/line_3d.png}
		\end{center}
		\caption{}
		\label{fig:Fig1}
	\end{figure}


\end{enumerate}
\end{document}
