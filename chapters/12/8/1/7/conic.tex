\documentclass[10pt,a4paper]{report}
%\usepackage[latin1]{inputenc}
\usepackage[utf8]{inputenc}
\usepackage{amsmath}
\usepackage{amsfonts}
\usepackage{amssymb}
\usepackage{graphicx}
\usepackage{multicol}
\usepackage{tabularx}
\usepackage{tikz}
\usetikzlibrary{arrows,shapes,automata,petri,positioning,calc}
\usepackage{hyperref}
\usepackage{tikz}
\usetikzlibrary{matrix,calc}
\usepackage[margin=0.5in]{geometry}
% ---- power functions -----% 
\newcommand{\myvec}[1]{\ensuremath{\begin{pmatrix}#1\end{pmatrix}}}
\let\vec\mathbf

\providecommand{\norm}[1]{\left\lVert#1\right\rVert}
\providecommand{\abs}[1]{\left\vert#1\right\vert}
\let\vec\mathbf

\newcommand{\mydet}[1]{\ensuremath{\begin{vmatrix}#1\end{vmatrix}}}
\providecommand{\brak}[1]{\ensuremath{\left(#1\right)}}
\providecommand{\lbrak}[1]{\ensuremath{\left(#1\right.}}
\providecommand{\rbrak}[1]{\ensuremath{\left.#1\right)}}
\providecommand{\sbrak}[1]{\ensuremath{{}\left[#1\right]}}
%-------end power functions----%
\newenvironment{Figure}
  {\par\medskip\noindent\minipage{\linewidth}}
  {\endminipage\par\medskip}
\begin{document}
%--------------------logo figure-------------------------%
\begin{figure*}[!tbp]
  \centering
  \begin{minipage}[b]{0.4\textwidth}
    %\includegraphics[scale=0.05]{fig/iitlogo.jpg} 
  \end{minipage}
  \hfill
  \vspace{5mm}\begin{minipage}[b]{0.4\textwidth}
\raggedleft  \includegraphics[scale=0.05]{iitlogo.jpg}  \

  \end{minipage}\vspace{0.2cm}
\end{figure*}
%--------------------name & rollno-----------------------
\raggedright \textbf{Name}:\hspace{1mm} T.ManasaReddy\hspace{3cm} \Large \textbf{Conic Assignment}\hspace{2.5cm} % 
\normalsize \textbf{Roll No.} :\hspace{1mm} FWC22048\vspace{1cm}
\begin{multicols}{2}

%----------------problem statement--------------%
\raggedright \textbf{Problem Statement:}\vspace{2mm}
\raggedright \\Find the area of the smaller part of the circle $x^2+y^2=a^2 $ cut off by the line x=$\frac{a}{\sqrt{2}}$\\
\vspace{5mm}
%-----------------------------solution---------------------------
\raggedright \textbf{SOLUTION}:\vspace{2mm}\\

%---------given----------------%
\raggedright \textbf{Given}:\vspace{2mm}\\
Equation of Circle is \\\vspace{1mm}
\begin{align}
x^2+y^2=a^2
\end{align}
Equation of line is \\ \vspace{1mm}
\begin{align}
x=\frac{a}{\sqrt{2}}
\end{align}
%-------------To find ------------------%
\textbf{To Find }\vspace{2mm}\\
To find the intersection points and area of shaded region shown in figure\vspace{2mm}  \\ 
%--------------steps----------------------%
\textbf{STEP-1}\vspace{2mm}\\
The given circle can be expressed as conics with parameters,\\ \vspace{1mm}
\begin{align}
\myvec{
1 & 0\\
0 & 1
}
\end{align}

\begin{align}
\vec{u}=0
\end{align} 
\begin{align}
f=-a^2
\end{align} \vspace{2mm}


\textbf{STEP-2}\vspace{2mm}\\
the given line equation can be written as\\ 
\begin{align} 
	\vec{x}=\begin{pmatrix}\frac{a}{\sqrt{2}} \\ 0 \\ \end{pmatrix}+k\begin{pmatrix} 0 \\ -1 \\ \end{pmatrix}
\end{align}

\textbf{STEP-3}\vspace{2mm}\\
The points of intersection of the line, \\ 
\begin{align}
L: \quad \vec{x} = \vec{q} + \kappa \vec{m} \quad \kappa \in \mathbb{R}
\end{align}
with the conic section, \\ 
\begin{align}
	\vec{x}^{\top}\vec{V}\vec{x} + 2\vec{u}^{\top} \vec{x} + f = 0
\end{align}
are given by \\
\begin{align}
\vec{x}_i = \vec{q} + \kappa_i \vec{m}
\end{align}
where, \\
{\tiny
\begin{multline}
\kappa_i = \frac{1}
{
\vec{m}^T\vec{V}\vec{m}
}
\lbrak{-\vec{m}^T\brak{\vec{V}\vec{q}+\vec{u}}}
\\
\pm
\rbrak{\sqrt{
\sbrak{
\vec{m}^T\brak{\vec{V}\vec{q}+\vec{u}}
}^2
-
\brak
{
\vec{q}^T\vec{V}\vec{q} + 2\vec{u}^T\vec{q} +f
}
\brak{\vec{m}^T\vec{V}\vec{m}}
}
}
\end{multline}
}
On substituting\\
\begin{align}
\vec{q} &= \myvec{
\frac{a}{\sqrt{2}}\\
0
} 
\end{align}
\begin{align}
\vec{m} = \myvec{ 0 \\ -1 }
\end{align}
With the given circle  as in eq(3),(4),(5),\\ 

The value of $\kappa$ ,\\
\begin{align}
    \kappa =\frac{a}{\sqrt{2}},\frac{-a}{\sqrt{2}}
\end{align}
by substituting eq(13) in eq(6)we get the
points of intersection of line with Circle \\
\begin{align}
    \vec{A}=\myvec{
\frac{a}{\sqrt{2}}\\
\frac{-a}{\sqrt{2}}
    }
\end{align}
\begin{align}
    \vec{B}=\myvec{
\frac{a}{\sqrt{2}}\\
\frac{a}{\sqrt{2}}
    }
\end{align}
\textbf{Result}
\begin{center}
 \includegraphics[scale=0.5]{conic.png}    
 \end{center}\vspace{1mm}
 From the figure,\\ \vspace{1mm}
Total area of portion is given by, \\ \vspace{1mm}
area\textbf{ APQ}=2* area of \textbf{APR}

\subsection*{Area of APR}
Since \textbf{APR} is in first Quadrant \\
\begin{align}
y=\sqrt{a^2-x^2}
\end{align}
\subsection*{Area of Circle}

\begin{align} 
\implies APQ=2*\int_{0}^{\frac{a}{\sqrt{2}}}\sqrt{a^2-x^2}\,dx 
\end{align}
by solving the above equation we get area of  smaller part of the circle\\



\vspace{3mm}
$\implies APQ=\frac{a^2}{2}[1+\frac{\pi}{2}]$\vspace{3mm}
 \vspace{2mm} \textbf{Construction}
\begin{center}
\setlength{\arrayrulewidth}{0.5mm}
\setlength{\tabcolsep}{6pt}
\renewcommand{\arraystretch}{1.5}
    \begin{tabular}{|l|c|}
    \hline 
    \textbf{Points} & \textbf{coordinates} \\ \hline
   B & $\myvec{
\frac{a}{\sqrt{2}}\\
\frac{a}{\sqrt{2}}
   } $ \\\hline
   A & $
   \vec{A}=\myvec{
\frac{a}{\sqrt{2}}\\
\frac{-a}{\sqrt{2}}
   } $ 
   \\\hline
      \end{tabular}
  \end{center}
  \end{multicols}
 
Get the python code of the figures from

\begin{table}[h]
\large
\centering
\framebox{
\url{https://github.com/manasa/MANASA_FWC/blob/main/conics/code/conic.py}}
\bibliographystyle{ieeetr}
\end{table} 
\end{document}
