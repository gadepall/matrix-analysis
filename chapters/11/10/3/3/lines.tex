\iffalse
\documentclass[12pt]{article}
\usepackage{graphicx}
\usepackage{amsmath}
\usepackage{mathtools}
\usepackage{gensymb}
\usepackage[utf8]{inputenc}
\usepackage{float}
\newcommand{\mydet}[1]{\ensuremath{\begin{vmatrix}#1\end{vmatrix}}}
\providecommand{\brak}[1]{\ensuremath{\left(#1\right)}}
\providecommand{\norm}[1]{\left\lVert#1\right\rVert}
\newcommand{\solution}{\noindent \textbf{Solution: }}
\newcommand{\myvec}[1]{\ensuremath{\begin{pmatrix}#1\end{pmatrix}}}
\let\vec\mathbf

\begin{document}
\begin{center}
\textbf\large{CLASS-11 \\ CHAPTER-10 \\ STRAIGHT LINES}
\end{center}
\section*{Excercise 10.3}

Q3. Reduce the following equations into normal form. Find their perpendicular distances from the origin and angle between perpendicular and the positive $x$-axis.
\begin{enumerate}
	\item $x-\sqrt{3}y+8=0$ 
	\item $y-2=0$
	\item $x-y=4$
\end{enumerate}
\solution
\fi
\begin{enumerate}
\item The given equation can be expressed as
		\begin{align}
			\myvec{ 1 & -\sqrt{3}}\vec{x}= -8
			\end{align}
			yielding
\begin{align}
			\vec{n} = \myvec{ 1 & -\sqrt{3}}, c = -8
\end{align}
From the above, the	angle between perpendicular and the positive $x$-axis is given by
		\begin{align}
			\tan^{-1}\brak{-\sqrt{3}} = \frac{2\pi}{3}
		\end{align}
	The perpendicular distance from the origin to the line is given by
		\begin{align}
			d=\frac{\abs{c}}{\norm{\vec{n}}}=4
		\end{align}
\item In this case, the given equation becomes
          \begin{align}
		  \myvec{0 & 1}\vec{x} = 2
          \end{align}
	  yielding
                  \begin{align}
			  \vec{n}=\myvec{0\\1}, c = 2
                          \end{align}
          Angle between perpendicular and the positive $x$-axis is given by:
		\begin{align}  
\cos\theta&=\frac{\vec{e}_{1}^\top\vec{n}}{\norm{\vec{e}_{1}}\norm{\vec{n}}}\\
			&=\frac{\myvec{1&0}\myvec{0 \\ 1\\}}{1}\\
			&=0\\
			\implies	\theta&=90\degree       
                \end{align}      
 The perpendicular distance from the origin to the line is given by:    
                                      \begin{align}
					      d=\frac{|c|}{\norm{\vec{n}}}=\frac{2}{1}=2             
                  \end{align}
\begin{figure}[H]
\begin{center} 
	    \includegraphics[width=\columnwidth]{figs/line2.png}
	\end{center}
\caption{}
\label{fig:Fig2}
\end{figure}
\item From the given equation:
         \begin{align}                                    \vec{m}&=1\\                                  		c&=-4
         \end{align}                                                                                          The directional vector is given by:
          \begin{align}
                  \vec{m}&=\myvec{1\\1\\}
          \end{align}
          The normal vector is given by:
                  \begin{align}
         \vec{n}&=\myvec{-1\\1\\}\\
          \vec{n}^\top&=\myvec{-1 & 1}
                          \end{align}
          Angle between perpendicular and the positive $x$-axis is given by:
		\begin{align}   
              \cos\theta&=\frac{\vec{e}_{1}^\top\vec{n}}{\norm{\vec{e}_{1}}\norm{\vec{n}}}\\
			&=\frac{\myvec{1&0}\myvec{-1 \\ 1\\}}{\sqrt{2}}\\
			&=-\frac{1}{2}\\
			\implies	\theta&=315\degree
                \end{align}                                                                           The perpendicular distance from the origin to the line is given by:                                          \begin{align}
			d=\frac{|c|}{\norm{\vec{n}}}=\frac{4}{\sqrt{2}}=2\sqrt{2}                    
                  \end{align}
\begin{figure}[H]
	\begin{center} 
	    \includegraphics[width=\columnwidth]{figs/line3.png}
	\end{center}
\caption{}
\label{fig:Fig3}
\end{figure}
\end{enumerate}
\end{document}
 
