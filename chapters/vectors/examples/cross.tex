\begin{enumerate}[label=\thesection.\arabic*,ref=\thesection.\theenumi]
		\item Find $\abs{\overrightarrow{a}\times\overrightarrow{b}},\text{ if }\overrightarrow{a}=\hat{i}-7\hat{j}+7\hat{k}\text{ and } \overrightarrow{b}=3\hat{i}-2\hat{j}+2\hat{k}$.
	\\
		\solution
		\iffalse
\documentclass[12pt]{article}
\usepackage{graphicx}
%\documentclass[journal,12pt,twocolumn]{IEEEtran}
\usepackage[none]{hyphenat}
\usepackage{graphicx}
\usepackage{listings}
\usepackage[english]{babel}
\usepackage{graphicx}
\usepackage{caption} 
\usepackage{hyperref}
\usepackage{booktabs}
\def\inputGnumericTable{}
\usepackage{color}                                            %%
    \usepackage{array}                                            %%
    \usepackage{longtable}                                        %%
    \usepackage{calc}                                             %%
    \usepackage{multirow}                                         %%
    \usepackage{hhline}                                           %%
    \usepackage{ifthen}
\usepackage{array}
\usepackage{amsmath}   % for having text in math mode
\usepackage{listings}
\lstset{
language=tex,
frame=single, 
breaklines=true
}
  
%Following 2 lines were added to remove the blank page at the beginning
\usepackage{atbegshi}% http://ctan.org/pkg/atbegshi
\AtBeginDocument{\AtBeginShipoutNext{\AtBeginShipoutDiscard}}
%


%New macro definitions
\newcommand{\mydet}[1]{\ensuremath{\begin{vmatrix}#1\end{vmatrix}}}
\providecommand{\brak}[1]{\ensuremath{\left(#1\right)}}
\providecommand{\norm}[1]{\left\lVert#1\right\rVert}
\newcommand{\solution}{\noindent \textbf{Solution: }}
\newcommand{\myvec}[1]{\ensuremath{\begin{pmatrix}#1\end{pmatrix}}}
\let\vec\mathbf

\begin{document}

\begin{center}
\title{\textbf{Coordinate Geometry}}
\date{\vspace{-5ex}} %Not to print date automatically
\maketitle
\end{center}

\setcounter{page}{1}



\begin{enumerate}

\item\textbf{Problem statement :} Find the area of a rhombus of its vertices are $\myvec{3 ,0}$, $\myvec{4 ,5}$, $\myvec{-1 ,4}$ and $\myvec{-2 ,-1}$taken in order

\solution \\
\fi
The input vertices for this problem are given as
	\begin{align}
	\vec{A} = \myvec{
		3\\
		0
		},
	\vec{B} = \myvec{
		4\\
		5
		},
        \vec{C} = \myvec{
		-1\\
		4
		},
        \vec{D} = \myvec{
		-2\\
		-1
		}
	\end{align}
Since		
\begin{align}
 \vec{A-D}= \myvec{3 \\ 0} - \myvec{-2 \\-1}= \myvec{5\\1}
 \\
  \vec{B-A}= \myvec{4 \\ 5} - \myvec{3 \\0}= \myvec{1\\5}
\end{align}
the area of the rhombus is
\begin{align}
                \norm{\myvec{\vec{A-D}}\times \myvec{\vec{B-A}}}=\mydet{5 & 1\\1 & 5} = 24
\end{align}
See Fig. 
\ref{fig:chapters/10/7/2/10/gFig1}.
\begin{figure}[!h]
 \begin{center}
  \includegraphics[width=\columnwidth]{chapters/10/7/2/10/figs/fig.pdf}
 \end{center}
\caption{}
\label{fig:chapters/10/7/2/10/gFig1}
\end{figure}

\item Show that $$(\overrightarrow{a}-\overrightarrow{b})\times (\overrightarrow{a}+\overrightarrow{b})=2(\overrightarrow{a}\times \overrightarrow{b})$$
	\\
		\solution
		\iffalse
\documentclass[12pt]{article}
\usepackage{graphicx}
%\documentclass[journal,12pt,twocolumn]{IEEEtran}
\usepackage[none]{hyphenat}
\usepackage{graphicx}
\usepackage{listings}
\usepackage[english]{babel}
\usepackage{graphicx}
\usepackage{caption} 
\usepackage{hyperref}
\usepackage{booktabs}
\def\inputGnumericTable{}
\usepackage{color}                                            %%
    \usepackage{array}                                            %%
    \usepackage{longtable}                                        %%
    \usepackage{calc}                                             %%
    \usepackage{multirow}                                         %%
    \usepackage{hhline}                                           %%
    \usepackage{ifthen}
\usepackage{array}
\usepackage{amsmath}   % for having text in math mode
\usepackage{listings}
\lstset{
language=tex,
frame=single, 
breaklines=true
}
  
%Following 2 lines were added to remove the blank page at the beginning
\usepackage{atbegshi}% http://ctan.org/pkg/atbegshi
\AtBeginDocument{\AtBeginShipoutNext{\AtBeginShipoutDiscard}}
%


%New macro definitions
\newcommand{\mydet}[1]{\ensuremath{\begin{vmatrix}#1\end{vmatrix}}}
\providecommand{\brak}[1]{\ensuremath{\left(#1\right)}}
\providecommand{\norm}[1]{\left\lVert#1\right\rVert}
\newcommand{\solution}{\noindent \textbf{Solution: }}
\newcommand{\myvec}[1]{\ensuremath{\begin{pmatrix}#1\end{pmatrix}}}
\let\vec\mathbf

\begin{document}

\begin{center}
\title{\textbf{Coordinate Geometry}}
\date{\vspace{-5ex}} %Not to print date automatically
\maketitle
\end{center}

\setcounter{page}{1}



\begin{enumerate}

\item\textbf{Problem statement :} Find the area of a rhombus of its vertices are $\myvec{3 ,0}$, $\myvec{4 ,5}$, $\myvec{-1 ,4}$ and $\myvec{-2 ,-1}$taken in order

\solution \\
\fi
The input vertices for this problem are given as
	\begin{align}
	\vec{A} = \myvec{
		3\\
		0
		},
	\vec{B} = \myvec{
		4\\
		5
		},
        \vec{C} = \myvec{
		-1\\
		4
		},
        \vec{D} = \myvec{
		-2\\
		-1
		}
	\end{align}
Since		
\begin{align}
 \vec{A-D}= \myvec{3 \\ 0} - \myvec{-2 \\-1}= \myvec{5\\1}
 \\
  \vec{B-A}= \myvec{4 \\ 5} - \myvec{3 \\0}= \myvec{1\\5}
\end{align}
the area of the rhombus is
\begin{align}
                \norm{\myvec{\vec{A-D}}\times \myvec{\vec{B-A}}}=\mydet{5 & 1\\1 & 5} = 24
\end{align}
See Fig. 
\ref{fig:chapters/10/7/2/10/gFig1}.
\begin{figure}[!h]
 \begin{center}
  \includegraphics[width=\columnwidth]{chapters/10/7/2/10/figs/fig.pdf}
 \end{center}
\caption{}
\label{fig:chapters/10/7/2/10/gFig1}
\end{figure}

\item Find $\lambda$ and $\mu$ if $(2\hat{i}+6\hat{j}+27\hat{k})\times(\hat{i}+\lambda \hat{j} + \mu \hat{k})=\overrightarrow{0}$.
	\\
		\solution
		\iffalse
\documentclass[12pt]{article}
\usepackage{graphicx}
%\documentclass[journal,12pt,twocolumn]{IEEEtran}
\usepackage[none]{hyphenat}
\usepackage{graphicx}
\usepackage{listings}
\usepackage[english]{babel}
\usepackage{graphicx}
\usepackage{caption} 
\usepackage{hyperref}
\usepackage{booktabs}
\def\inputGnumericTable{}
\usepackage{color}                                            %%
    \usepackage{array}                                            %%
    \usepackage{longtable}                                        %%
    \usepackage{calc}                                             %%
    \usepackage{multirow}                                         %%
    \usepackage{hhline}                                           %%
    \usepackage{ifthen}
\usepackage{array}
\usepackage{amsmath}   % for having text in math mode
\usepackage{listings}
\lstset{
language=tex,
frame=single, 
breaklines=true
}
  
%Following 2 lines were added to remove the blank page at the beginning
\usepackage{atbegshi}% http://ctan.org/pkg/atbegshi
\AtBeginDocument{\AtBeginShipoutNext{\AtBeginShipoutDiscard}}
%


%New macro definitions
\newcommand{\mydet}[1]{\ensuremath{\begin{vmatrix}#1\end{vmatrix}}}
\providecommand{\brak}[1]{\ensuremath{\left(#1\right)}}
\providecommand{\norm}[1]{\left\lVert#1\right\rVert}
\newcommand{\solution}{\noindent \textbf{Solution: }}
\newcommand{\myvec}[1]{\ensuremath{\begin{pmatrix}#1\end{pmatrix}}}
\let\vec\mathbf

\begin{document}

\begin{center}
\title{\textbf{Coordinate Geometry}}
\date{\vspace{-5ex}} %Not to print date automatically
\maketitle
\end{center}

\setcounter{page}{1}



\begin{enumerate}

\item\textbf{Problem statement :} Find the area of a rhombus of its vertices are $\myvec{3 ,0}$, $\myvec{4 ,5}$, $\myvec{-1 ,4}$ and $\myvec{-2 ,-1}$taken in order

\solution \\
\fi
The input vertices for this problem are given as
	\begin{align}
	\vec{A} = \myvec{
		3\\
		0
		},
	\vec{B} = \myvec{
		4\\
		5
		},
        \vec{C} = \myvec{
		-1\\
		4
		},
        \vec{D} = \myvec{
		-2\\
		-1
		}
	\end{align}
Since		
\begin{align}
 \vec{A-D}= \myvec{3 \\ 0} - \myvec{-2 \\-1}= \myvec{5\\1}
 \\
  \vec{B-A}= \myvec{4 \\ 5} - \myvec{3 \\0}= \myvec{1\\5}
\end{align}
the area of the rhombus is
\begin{align}
                \norm{\myvec{\vec{A-D}}\times \myvec{\vec{B-A}}}=\mydet{5 & 1\\1 & 5} = 24
\end{align}
See Fig. 
\ref{fig:chapters/10/7/2/10/gFig1}.
\begin{figure}[!h]
 \begin{center}
  \includegraphics[width=\columnwidth]{chapters/10/7/2/10/figs/fig.pdf}
 \end{center}
\caption{}
\label{fig:chapters/10/7/2/10/gFig1}
\end{figure}

\item Given that $\overrightarrow{a} \cdot \overrightarrow{b} = 0$ and $\overrightarrow{a} \times \overrightarrow{b} = \overrightarrow{0}$. What can you conclude about the vectors $\overrightarrow{a} \text{ and }\overrightarrow{b}$?
\item Let the vectors be given as $\overrightarrow{a},\overrightarrow{b},\overrightarrow{c}\text{ be given as }\ a_1 \hat{i}+\ a_2 \hat{j}+\ a_3 \hat{k},\ b_1 \hat{i}+\ b_2 \hat{j}+\ b_3 \hat{k},\ c_1 \hat{i}+\ c_2 \hat{j}+\ c_3 \hat{k}$. Then show that $\overrightarrow{a} \times (\overrightarrow{b} + \overrightarrow{c}) = \overrightarrow{a} \times \overrightarrow{b}+\overrightarrow{a} \times \overrightarrow{c}$.
	\\
		\solution
		\iffalse
\documentclass[12pt]{article}
\usepackage{graphicx}
%\documentclass[journal,12pt,twocolumn]{IEEEtran}
\usepackage[none]{hyphenat}
\usepackage{graphicx}
\usepackage{listings}
\usepackage[english]{babel}
\usepackage{graphicx}
\usepackage{caption} 
\usepackage{hyperref}
\usepackage{booktabs}
\def\inputGnumericTable{}
\usepackage{color}                                            %%
    \usepackage{array}                                            %%
    \usepackage{longtable}                                        %%
    \usepackage{calc}                                             %%
    \usepackage{multirow}                                         %%
    \usepackage{hhline}                                           %%
    \usepackage{ifthen}
\usepackage{array}
\usepackage{amsmath}   % for having text in math mode
\usepackage{listings}
\lstset{
language=tex,
frame=single, 
breaklines=true
}
  
%Following 2 lines were added to remove the blank page at the beginning
\usepackage{atbegshi}% http://ctan.org/pkg/atbegshi
\AtBeginDocument{\AtBeginShipoutNext{\AtBeginShipoutDiscard}}
%


%New macro definitions
\newcommand{\mydet}[1]{\ensuremath{\begin{vmatrix}#1\end{vmatrix}}}
\providecommand{\brak}[1]{\ensuremath{\left(#1\right)}}
\providecommand{\norm}[1]{\left\lVert#1\right\rVert}
\newcommand{\solution}{\noindent \textbf{Solution: }}
\newcommand{\myvec}[1]{\ensuremath{\begin{pmatrix}#1\end{pmatrix}}}
\let\vec\mathbf

\begin{document}

\begin{center}
\title{\textbf{Coordinate Geometry}}
\date{\vspace{-5ex}} %Not to print date automatically
\maketitle
\end{center}

\setcounter{page}{1}



\begin{enumerate}

\item\textbf{Problem statement :} Find the area of a rhombus of its vertices are $\myvec{3 ,0}$, $\myvec{4 ,5}$, $\myvec{-1 ,4}$ and $\myvec{-2 ,-1}$taken in order

\solution \\
\fi
The input vertices for this problem are given as
	\begin{align}
	\vec{A} = \myvec{
		3\\
		0
		},
	\vec{B} = \myvec{
		4\\
		5
		},
        \vec{C} = \myvec{
		-1\\
		4
		},
        \vec{D} = \myvec{
		-2\\
		-1
		}
	\end{align}
Since		
\begin{align}
 \vec{A-D}= \myvec{3 \\ 0} - \myvec{-2 \\-1}= \myvec{5\\1}
 \\
  \vec{B-A}= \myvec{4 \\ 5} - \myvec{3 \\0}= \myvec{1\\5}
\end{align}
the area of the rhombus is
\begin{align}
                \norm{\myvec{\vec{A-D}}\times \myvec{\vec{B-A}}}=\mydet{5 & 1\\1 & 5} = 24
\end{align}
See Fig. 
\ref{fig:chapters/10/7/2/10/gFig1}.
\begin{figure}[!h]
 \begin{center}
  \includegraphics[width=\columnwidth]{chapters/10/7/2/10/figs/fig.pdf}
 \end{center}
\caption{}
\label{fig:chapters/10/7/2/10/gFig1}
\end{figure}

\item If either $\overrightarrow{a} = \overrightarrow{0}$ or $\overrightarrow{b} = \overrightarrow{0}$, then $\overrightarrow{a} \times \overrightarrow{b} = \overrightarrow{0}$. Is the converse true? Justify your answer with an example.
	\\
		\solution
		\iffalse
\documentclass[12pt]{article}
\usepackage{graphicx}
%\documentclass[journal,12pt,twocolumn]{IEEEtran}
\usepackage[none]{hyphenat}
\usepackage{graphicx}
\usepackage{listings}
\usepackage[english]{babel}
\usepackage{graphicx}
\usepackage{caption} 
\usepackage{hyperref}
\usepackage{booktabs}
\def\inputGnumericTable{}
\usepackage{color}                                            %%
    \usepackage{array}                                            %%
    \usepackage{longtable}                                        %%
    \usepackage{calc}                                             %%
    \usepackage{multirow}                                         %%
    \usepackage{hhline}                                           %%
    \usepackage{ifthen}
\usepackage{array}
\usepackage{amsmath}   % for having text in math mode
\usepackage{listings}
\lstset{
language=tex,
frame=single, 
breaklines=true
}
  
%Following 2 lines were added to remove the blank page at the beginning
\usepackage{atbegshi}% http://ctan.org/pkg/atbegshi
\AtBeginDocument{\AtBeginShipoutNext{\AtBeginShipoutDiscard}}
%


%New macro definitions
\newcommand{\mydet}[1]{\ensuremath{\begin{vmatrix}#1\end{vmatrix}}}
\providecommand{\brak}[1]{\ensuremath{\left(#1\right)}}
\providecommand{\norm}[1]{\left\lVert#1\right\rVert}
\newcommand{\solution}{\noindent \textbf{Solution: }}
\newcommand{\myvec}[1]{\ensuremath{\begin{pmatrix}#1\end{pmatrix}}}
\let\vec\mathbf

\begin{document}

\begin{center}
\title{\textbf{Coordinate Geometry}}
\date{\vspace{-5ex}} %Not to print date automatically
\maketitle
\end{center}

\setcounter{page}{1}



\begin{enumerate}

\item\textbf{Problem statement :} Find the area of a rhombus of its vertices are $\myvec{3 ,0}$, $\myvec{4 ,5}$, $\myvec{-1 ,4}$ and $\myvec{-2 ,-1}$taken in order

\solution \\
\fi
The input vertices for this problem are given as
	\begin{align}
	\vec{A} = \myvec{
		3\\
		0
		},
	\vec{B} = \myvec{
		4\\
		5
		},
        \vec{C} = \myvec{
		-1\\
		4
		},
        \vec{D} = \myvec{
		-2\\
		-1
		}
	\end{align}
Since		
\begin{align}
 \vec{A-D}= \myvec{3 \\ 0} - \myvec{-2 \\-1}= \myvec{5\\1}
 \\
  \vec{B-A}= \myvec{4 \\ 5} - \myvec{3 \\0}= \myvec{1\\5}
\end{align}
the area of the rhombus is
\begin{align}
                \norm{\myvec{\vec{A-D}}\times \myvec{\vec{B-A}}}=\mydet{5 & 1\\1 & 5} = 24
\end{align}
See Fig. 
\ref{fig:chapters/10/7/2/10/gFig1}.
\begin{figure}[!h]
 \begin{center}
  \includegraphics[width=\columnwidth]{chapters/10/7/2/10/figs/fig.pdf}
 \end{center}
\caption{}
\label{fig:chapters/10/7/2/10/gFig1}
\end{figure}

\item Find the area of the triangle with vertices $A(1, 1, 2)$, $B(2, 3, 5)$, and $C(1, 5, 5)$
	\\
		\solution
		\iffalse
\documentclass[12pt]{article}
\usepackage{graphicx}
%\documentclass[journal,12pt,twocolumn]{IEEEtran}
\usepackage[none]{hyphenat}
\usepackage{graphicx}
\usepackage{listings}
\usepackage[english]{babel}
\usepackage{graphicx}
\usepackage{caption} 
\usepackage{hyperref}
\usepackage{booktabs}
\def\inputGnumericTable{}
\usepackage{color}                                            %%
    \usepackage{array}                                            %%
    \usepackage{longtable}                                        %%
    \usepackage{calc}                                             %%
    \usepackage{multirow}                                         %%
    \usepackage{hhline}                                           %%
    \usepackage{ifthen}
\usepackage{array}
\usepackage{amsmath}   % for having text in math mode
\usepackage{listings}
\lstset{
language=tex,
frame=single, 
breaklines=true
}
  
%Following 2 lines were added to remove the blank page at the beginning
\usepackage{atbegshi}% http://ctan.org/pkg/atbegshi
\AtBeginDocument{\AtBeginShipoutNext{\AtBeginShipoutDiscard}}
%


%New macro definitions
\newcommand{\mydet}[1]{\ensuremath{\begin{vmatrix}#1\end{vmatrix}}}
\providecommand{\brak}[1]{\ensuremath{\left(#1\right)}}
\providecommand{\norm}[1]{\left\lVert#1\right\rVert}
\newcommand{\solution}{\noindent \textbf{Solution: }}
\newcommand{\myvec}[1]{\ensuremath{\begin{pmatrix}#1\end{pmatrix}}}
\let\vec\mathbf

\begin{document}

\begin{center}
\title{\textbf{Coordinate Geometry}}
\date{\vspace{-5ex}} %Not to print date automatically
\maketitle
\end{center}

\setcounter{page}{1}



\begin{enumerate}

\item\textbf{Problem statement :} Find the area of a rhombus of its vertices are $\myvec{3 ,0}$, $\myvec{4 ,5}$, $\myvec{-1 ,4}$ and $\myvec{-2 ,-1}$taken in order

\solution \\
\fi
The input vertices for this problem are given as
	\begin{align}
	\vec{A} = \myvec{
		3\\
		0
		},
	\vec{B} = \myvec{
		4\\
		5
		},
        \vec{C} = \myvec{
		-1\\
		4
		},
        \vec{D} = \myvec{
		-2\\
		-1
		}
	\end{align}
Since		
\begin{align}
 \vec{A-D}= \myvec{3 \\ 0} - \myvec{-2 \\-1}= \myvec{5\\1}
 \\
  \vec{B-A}= \myvec{4 \\ 5} - \myvec{3 \\0}= \myvec{1\\5}
\end{align}
the area of the rhombus is
\begin{align}
                \norm{\myvec{\vec{A-D}}\times \myvec{\vec{B-A}}}=\mydet{5 & 1\\1 & 5} = 24
\end{align}
See Fig. 
\ref{fig:chapters/10/7/2/10/gFig1}.
\begin{figure}[!h]
 \begin{center}
  \includegraphics[width=\columnwidth]{chapters/10/7/2/10/figs/fig.pdf}
 \end{center}
\caption{}
\label{fig:chapters/10/7/2/10/gFig1}
\end{figure}

\item Find the area of the parallelogram whose adjacent sides are determined by the vectors $\overrightarrow{a}=\hat{i}-\hat{j}+3\hat{k}$ and $\overrightarrow{b}=2\hat{i}-7\hat{j}+\hat{k}$.
	\\
		\solution
		\iffalse
\documentclass[12pt]{article}
\usepackage{graphicx}
%\documentclass[journal,12pt,twocolumn]{IEEEtran}
\usepackage[none]{hyphenat}
\usepackage{graphicx}
\usepackage{listings}
\usepackage[english]{babel}
\usepackage{graphicx}
\usepackage{caption} 
\usepackage{hyperref}
\usepackage{booktabs}
\def\inputGnumericTable{}
\usepackage{color}                                            %%
    \usepackage{array}                                            %%
    \usepackage{longtable}                                        %%
    \usepackage{calc}                                             %%
    \usepackage{multirow}                                         %%
    \usepackage{hhline}                                           %%
    \usepackage{ifthen}
\usepackage{array}
\usepackage{amsmath}   % for having text in math mode
\usepackage{listings}
\lstset{
language=tex,
frame=single, 
breaklines=true
}
  
%Following 2 lines were added to remove the blank page at the beginning
\usepackage{atbegshi}% http://ctan.org/pkg/atbegshi
\AtBeginDocument{\AtBeginShipoutNext{\AtBeginShipoutDiscard}}
%


%New macro definitions
\newcommand{\mydet}[1]{\ensuremath{\begin{vmatrix}#1\end{vmatrix}}}
\providecommand{\brak}[1]{\ensuremath{\left(#1\right)}}
\providecommand{\norm}[1]{\left\lVert#1\right\rVert}
\newcommand{\solution}{\noindent \textbf{Solution: }}
\newcommand{\myvec}[1]{\ensuremath{\begin{pmatrix}#1\end{pmatrix}}}
\let\vec\mathbf

\begin{document}

\begin{center}
\title{\textbf{Coordinate Geometry}}
\date{\vspace{-5ex}} %Not to print date automatically
\maketitle
\end{center}

\setcounter{page}{1}



\begin{enumerate}

\item\textbf{Problem statement :} Find the area of a rhombus of its vertices are $\myvec{3 ,0}$, $\myvec{4 ,5}$, $\myvec{-1 ,4}$ and $\myvec{-2 ,-1}$taken in order

\solution \\
\fi
The input vertices for this problem are given as
	\begin{align}
	\vec{A} = \myvec{
		3\\
		0
		},
	\vec{B} = \myvec{
		4\\
		5
		},
        \vec{C} = \myvec{
		-1\\
		4
		},
        \vec{D} = \myvec{
		-2\\
		-1
		}
	\end{align}
Since		
\begin{align}
 \vec{A-D}= \myvec{3 \\ 0} - \myvec{-2 \\-1}= \myvec{5\\1}
 \\
  \vec{B-A}= \myvec{4 \\ 5} - \myvec{3 \\0}= \myvec{1\\5}
\end{align}
the area of the rhombus is
\begin{align}
                \norm{\myvec{\vec{A-D}}\times \myvec{\vec{B-A}}}=\mydet{5 & 1\\1 & 5} = 24
\end{align}
See Fig. 
\ref{fig:chapters/10/7/2/10/gFig1}.
\begin{figure}[!h]
 \begin{center}
  \includegraphics[width=\columnwidth]{chapters/10/7/2/10/figs/fig.pdf}
 \end{center}
\caption{}
\label{fig:chapters/10/7/2/10/gFig1}
\end{figure}

\item Let the vectors $\overrightarrow{a}$ and $\overrightarrow{b}$ be such that $|\overrightarrow{a}| = 3$ and $|\overrightarrow{b}| = \dfrac{\sqrt{2}}{3}$, then $\overrightarrow{a} \times \overrightarrow{b}$ is a unit vector, if the angle between $\overrightarrow{a}$ and $\overrightarrow{b}$ is
\begin{enumerate}
\item $\dfrac{\pi}{6}$
\item $\dfrac{\pi}{4}$
\item $\dfrac{\pi}{3}$
\item $\dfrac{\pi}{2}$
\end{enumerate}
		\solution
		\iffalse
\documentclass[12pt]{article}
\usepackage{graphicx}
%\documentclass[journal,12pt,twocolumn]{IEEEtran}
\usepackage[none]{hyphenat}
\usepackage{graphicx}
\usepackage{listings}
\usepackage[english]{babel}
\usepackage{graphicx}
\usepackage{caption} 
\usepackage{hyperref}
\usepackage{booktabs}
\def\inputGnumericTable{}
\usepackage{color}                                            %%
    \usepackage{array}                                            %%
    \usepackage{longtable}                                        %%
    \usepackage{calc}                                             %%
    \usepackage{multirow}                                         %%
    \usepackage{hhline}                                           %%
    \usepackage{ifthen}
\usepackage{array}
\usepackage{amsmath}   % for having text in math mode
\usepackage{listings}
\lstset{
language=tex,
frame=single, 
breaklines=true
}
  
%Following 2 lines were added to remove the blank page at the beginning
\usepackage{atbegshi}% http://ctan.org/pkg/atbegshi
\AtBeginDocument{\AtBeginShipoutNext{\AtBeginShipoutDiscard}}
%


%New macro definitions
\newcommand{\mydet}[1]{\ensuremath{\begin{vmatrix}#1\end{vmatrix}}}
\providecommand{\brak}[1]{\ensuremath{\left(#1\right)}}
\providecommand{\norm}[1]{\left\lVert#1\right\rVert}
\newcommand{\solution}{\noindent \textbf{Solution: }}
\newcommand{\myvec}[1]{\ensuremath{\begin{pmatrix}#1\end{pmatrix}}}
\let\vec\mathbf

\begin{document}

\begin{center}
\title{\textbf{Coordinate Geometry}}
\date{\vspace{-5ex}} %Not to print date automatically
\maketitle
\end{center}

\setcounter{page}{1}



\begin{enumerate}

\item\textbf{Problem statement :} Find the area of a rhombus of its vertices are $\myvec{3 ,0}$, $\myvec{4 ,5}$, $\myvec{-1 ,4}$ and $\myvec{-2 ,-1}$taken in order

\solution \\
\fi
The input vertices for this problem are given as
	\begin{align}
	\vec{A} = \myvec{
		3\\
		0
		},
	\vec{B} = \myvec{
		4\\
		5
		},
        \vec{C} = \myvec{
		-1\\
		4
		},
        \vec{D} = \myvec{
		-2\\
		-1
		}
	\end{align}
Since		
\begin{align}
 \vec{A-D}= \myvec{3 \\ 0} - \myvec{-2 \\-1}= \myvec{5\\1}
 \\
  \vec{B-A}= \myvec{4 \\ 5} - \myvec{3 \\0}= \myvec{1\\5}
\end{align}
the area of the rhombus is
\begin{align}
                \norm{\myvec{\vec{A-D}}\times \myvec{\vec{B-A}}}=\mydet{5 & 1\\1 & 5} = 24
\end{align}
See Fig. 
\ref{fig:chapters/10/7/2/10/gFig1}.
\begin{figure}[!h]
 \begin{center}
  \includegraphics[width=\columnwidth]{chapters/10/7/2/10/figs/fig.pdf}
 \end{center}
\caption{}
\label{fig:chapters/10/7/2/10/gFig1}
\end{figure}

\item Area of a rectangle having vertices A, B, C and D with position vectors $ -\hat{i}+ \dfrac{1}{2} \hat{j}+4\hat{k},\hat{i}+ \dfrac{1}{2} \hat{j}+4\hat{k},\hat{i}-\dfrac{1}{2} \hat{j}+4\hat{k}\text{ and }-\hat{i}- \dfrac{1}{2} \hat{j}+4\hat{k}$, respectively is
\begin{enumerate}
\item $\dfrac{1}{2}$
\item 1
\item 2
\item 4
\end{enumerate}
		\solution
		\iffalse
\documentclass[12pt]{article}
\usepackage{graphicx}
%\documentclass[journal,12pt,twocolumn]{IEEEtran}
\usepackage[none]{hyphenat}
\usepackage{graphicx}
\usepackage{listings}
\usepackage[english]{babel}
\usepackage{graphicx}
\usepackage{caption} 
\usepackage{hyperref}
\usepackage{booktabs}
\def\inputGnumericTable{}
\usepackage{color}                                            %%
    \usepackage{array}                                            %%
    \usepackage{longtable}                                        %%
    \usepackage{calc}                                             %%
    \usepackage{multirow}                                         %%
    \usepackage{hhline}                                           %%
    \usepackage{ifthen}
\usepackage{array}
\usepackage{amsmath}   % for having text in math mode
\usepackage{listings}
\lstset{
language=tex,
frame=single, 
breaklines=true
}
  
%Following 2 lines were added to remove the blank page at the beginning
\usepackage{atbegshi}% http://ctan.org/pkg/atbegshi
\AtBeginDocument{\AtBeginShipoutNext{\AtBeginShipoutDiscard}}
%


%New macro definitions
\newcommand{\mydet}[1]{\ensuremath{\begin{vmatrix}#1\end{vmatrix}}}
\providecommand{\brak}[1]{\ensuremath{\left(#1\right)}}
\providecommand{\norm}[1]{\left\lVert#1\right\rVert}
\newcommand{\solution}{\noindent \textbf{Solution: }}
\newcommand{\myvec}[1]{\ensuremath{\begin{pmatrix}#1\end{pmatrix}}}
\let\vec\mathbf

\begin{document}

\begin{center}
\title{\textbf{Coordinate Geometry}}
\date{\vspace{-5ex}} %Not to print date automatically
\maketitle
\end{center}

\setcounter{page}{1}



\begin{enumerate}

\item\textbf{Problem statement :} Find the area of a rhombus of its vertices are $\myvec{3 ,0}$, $\myvec{4 ,5}$, $\myvec{-1 ,4}$ and $\myvec{-2 ,-1}$taken in order

\solution \\
\fi
The input vertices for this problem are given as
	\begin{align}
	\vec{A} = \myvec{
		3\\
		0
		},
	\vec{B} = \myvec{
		4\\
		5
		},
        \vec{C} = \myvec{
		-1\\
		4
		},
        \vec{D} = \myvec{
		-2\\
		-1
		}
	\end{align}
Since		
\begin{align}
 \vec{A-D}= \myvec{3 \\ 0} - \myvec{-2 \\-1}= \myvec{5\\1}
 \\
  \vec{B-A}= \myvec{4 \\ 5} - \myvec{3 \\0}= \myvec{1\\5}
\end{align}
the area of the rhombus is
\begin{align}
                \norm{\myvec{\vec{A-D}}\times \myvec{\vec{B-A}}}=\mydet{5 & 1\\1 & 5} = 24
\end{align}
See Fig. 
\ref{fig:chapters/10/7/2/10/gFig1}.
\begin{figure}[!h]
 \begin{center}
  \includegraphics[width=\columnwidth]{chapters/10/7/2/10/figs/fig.pdf}
 \end{center}
\caption{}
\label{fig:chapters/10/7/2/10/gFig1}
\end{figure}

\item Find the area of the triangle whose vertices are 
\begin{enumerate}
\item $(2, 3), (–1, 0), (2, – 4)$
\item $(–5, –1), (3, –5), (5, 2)$ 
\end{enumerate}
		\label{10/7/3/1}
\solution
		\iffalse
\documentclass[12pt]{article}
\usepackage{graphicx}
%\documentclass[journal,12pt,twocolumn]{IEEEtran}
\usepackage[none]{hyphenat}
\usepackage{graphicx}
\usepackage{listings}
\usepackage[english]{babel}
\usepackage{graphicx}
\usepackage{caption} 
\usepackage{hyperref}
\usepackage{booktabs}
\usepackage{array}
\usepackage{amsmath}   % for having text in math mode

%Following 2 lines were added to remove the blank page at the beginning
\usepackage{atbegshi}% http://ctan.org/pkg/atbegshi
\AtBeginDocument{\AtBeginShipoutNext{\AtBeginShipoutDiscard}}
%


%New macro definitions
\newcommand{\mydet}[1]{\ensuremath{\begin{vmatrix}#1\end{vmatrix}}}
\providecommand{\brak}[1]{\ensuremath{\left(#1\right)}}
\providecommand{\norm}[1]{\left\lVert#1\right\rVert}
\newcommand{\solution}{\noindent \textbf{Solution: }}
\newcommand{\myvec}[1]{\ensuremath{\begin{pmatrix}#1\end{pmatrix}}}
\let\vec\mathbf

\begin{document}

\begin{center}
\title{\textbf{Area of a Traingle}}
\date{\vspace{-5ex}} %Not to print date automatically
\maketitle
\end{center}

\setcounter{page}{1}



\section{10$^{th}$ Maths - Chapter 7}

This is Problem-1 from Exercise 7.3

\begin{enumerate}
\item Find the area of the triangle whose vertices are :
	\fi
\begin{enumerate}
\item 
In this case, the area  is given by  
  \label{prop:10/7/3/1area2d}
  \begin{align}
    \label{eq:10/7/3/1area2d}
	\frac{1}{2}\norm{\brak{\vec{A}-\vec{B}} \times \brak{\vec{A}-\vec{C}}} \\
  \end{align}
  Since
  \begin{align}
	 \vec{A}-\vec{B} =  \myvec{
  2 \\
  3 \\
 } - \myvec{
  -1 \\
  0 \\
 } = \myvec{
 3 \\
 3 \\
 }
 \\
 \vec{A}-\vec{C} =  \myvec{
  2 \\
  3 \\
 } - \myvec{
  2 \\
  -4 \\
 } = \myvec{
 0 \\
 7 \\
 }
 \end{align}
 the desired area is given by 
 \iffalse
The value of the cross product of two vectors is given by
\begin{align}
  \label{eq:10/7/3/1det2d}
  \mydet{\vec{M}} &= \mydet{\vec{A} & \vec{B}} 
  \\
  &= \mydet{a_1 & b_1\\a_2 & b_2} = a_1b_2 - a_2 b_1
\end{align}

		Therefore, \eqref{eq:10/7/3/1area2d} equals \\
		\fi
\begin{align}
	\frac{1}{2}\mydet{3 & 0\\3 & 7}  
	&=	\frac{21}{2}
\end{align}
\iffalse
\begin{figure}[!h]
	\begin{center}
		\includegraphics[width=\columnwidth]{./figs/problem1a.pdf}
	\end{center}
\caption{}
\label{fig:Fig1}
\end{figure}
\fi

\item In this case, 
	\iffalse
\solution The area of the triangle with vertices $\vec{A}, \vec{B}, \vec{C}$ is given by  
  \label{prop:10/7/3/1area2e}
  \begin{align}
    \label{eq:10/7/3/1area2e}
	\frac{1}{2}\norm{\brak{\vec{A}-\vec{B}} \times \brak{\vec{A}-\vec{C}}} \\
	\fi
  \begin{align}
	 \vec{A}-\vec{B} =  \myvec{
  -5 \\
  -1 \\
 } - \myvec{
  3 \\
  -5 \\
 } = \myvec{
 -8 \\
 4 \\
 }
 \\
 \vec{A}-\vec{C} =  \myvec{
  -5 \\
  -1 \\
 } - \myvec{
  5 \\
  2 \\
 } = \myvec{
 -10 \\
 -3 \\
 }
 \\
	  \implies
\text{Area} =	\frac{1}{2}\mydet{-8 & -10\\4 & -3}  
	=  32 
\end{align}
\iffalse
\begin{figure}[!h]
	\begin{center}
		\includegraphics[width=\columnwidth]{./figs/problem1b.pdf}
	\end{center}
\caption{}
\label{fig:Fig2}
\end{figure}
\fi
\end{enumerate}



\item Find the area of the triangle formed by joining the mid-points of the sides of the triangle whose vertices are $(0, –1), (2, 1) \text{ and } (0, 3)$. Find the ratio of this area to the area of the given triangle.
	\\
\solution
		\iffalse
\documentclass[12pt]{article}
\usepackage{graphicx}
\usepackage{amsmath}
\usepackage{mathtools}
\usepackage{gensymb}

\newcommand{\mydet}[1]{\ensuremath{\begin{vmatrix}#1\end{vmatrix}}}
\providecommand{\brak}[1]{\ensuremath{\left(#1\right)}}
\providecommand{\norm}[1]{\left\lVert#1\right\rVert}
\newcommand{\solution}{\noindent \textbf{Solution: }}
\newcommand{\myvec}[1]{\ensuremath{\begin{pmatrix}#1\end{pmatrix}}}
\let\vec\mathbf

\begin{document}
\begin{center}
\textbf\large{CHAPTER-7 \\ COORDINATE GEOMETRY}
\end{center}
\section*{Excercise 7.2}

Q3. Find the area of the triangle formed by joining the mid-points of the sides of the triangle
whose vertices are $\vec(0, –1), \vec(2, 1) \text{ and } \vec(0, 3)$. Find the ratio of this area to the area of the
given triangle
\\
\solution
\\
\fi
The coordinates are given as
	\begin{align}
	\vec{A} = \myvec{
		0\\
		-1\\
		},
	\vec{B} = \myvec{
		2\\
		1\\
		},
	\vec{C} = \myvec{
		0\\
		3\\
		}
	\end{align}
Calculating midpoints,
	\begin{align}
		\vec{P} = \frac{1}{2}\vec(\vec{A}+\vec{B}) = \frac{1}{2}\myvec{2\\0\\} = \myvec{1\\0\\}\\
		\vec{Q} = \frac{1}{2}\vec(\vec{B}+\vec{C}) = \frac{1}{2}\myvec{2\\4\\} = \myvec{1\\2\\}\\
		\vec{R} = \frac{1}{2}\vec(\vec{A}+\vec{C}) = \frac{1}{2}\myvec{0\\2\\} = \myvec{0\\1\\}
	\end{align}
	Since
	\begin{align}
		\vec{P}-\vec{Q} &=  \myvec{
  1 \\
  0 
 } - \myvec{
  1 \\
  2 
 } = \myvec{
 0 \\
 -2 
 }
		\\
		\vec{Q}-\vec{R} &=  \myvec{
  1 \\
  2 \\
 } - \myvec{
  0 \\
  1 \\
 } = \myvec{
 1 \\
 1 \\
 }
	\end{align}
	the area is obtained as
	\begin{align}
		ar(PQR)&=\frac{1}{2}{\norm{\vec(\vec{P}-\vec{Q})\times\vec(\vec{Q}-\vec{R})}}
		\\
		&=\frac{1}{2}\mydet{0 & 1\\-2 & 1}
		=1
	\end{align}
	Similarly, 
	\begin{align}
		\vec{A}-\vec{B} &=  \myvec{
  0 \\
  -1 \\
 } - \myvec{
  2 \\
  1 \\
 } = \myvec{
 -2 \\
 -2 \\
 }
 \\
		\vec{A}-\vec{C} &=  \myvec{
  0 \\
  -1 \\
 } - \myvec{
  0 \\
  3 \\
 } = \myvec{
 0 \\
 -4 \\
 }
	\end{align}
 the area is obained as
	\begin{align}
		ar(ABC)&=\frac{1}{2}{\norm{\vec(\vec{A}-\vec{B})\times\vec(\vec{A}-\vec{C})}}\\
		&=\frac{1}{2}\mydet{-2 & 0\\-2 & -4}
=4
	\end{align}
	Thus, the resultant ratio of two areas is 1:4.
	See Fig.
\ref{fig:10/7/3/3Fig}
\begin{figure}[!h]
	\begin{center} 
	    \includegraphics[width=\columnwidth]{chapters/10/7/3/3/figs/trigraph.png}
	\end{center}
\caption{}
\label{fig:10/7/3/3Fig}
\end{figure}


\item Find the area of the quadrilateral whose vertices, taken in order, are $(– 4, – 2), (– 3, – 5), (3, – 2)$  and $ (2, 3)$.
	\\
\solution
		\iffalse
\documentclass[12pt]{article}
\usepackage{graphicx}
%\documentclass[journal,12pt,twocolumn]{IEEEtran}
\usepackage[none]{hyphenat}
\usepackage{graphicx}
\usepackage{listings}
\usepackage[english]{babel}
\usepackage{graphicx}
\usepackage{caption} 
\usepackage{hyperref}
\usepackage{booktabs}
\def\inputGnumericTable{}
\usepackage{color}                                            %%
    \usepackage{array}                                            %%
    \usepackage{longtable}                                        %%
    \usepackage{calc}                                             %%
    \usepackage{multirow}                                         %%
    \usepackage{hhline}                                           %%
    \usepackage{ifthen}
\usepackage{array}
\usepackage{amsmath}   % for having text in math mode
\usepackage{listings}
\lstset{
language=tex,
frame=single, 
breaklines=true
}
  
%Following 2 lines were added to remove the blank page at the beginning
\usepackage{atbegshi}% http://ctan.org/pkg/atbegshi
\AtBeginDocument{\AtBeginShipoutNext{\AtBeginShipoutDiscard}}
%


%New macro definitions
\newcommand{\mydet}[1]{\ensuremath{\begin{vmatrix}#1\end{vmatrix}}}
\providecommand{\brak}[1]{\ensuremath{\left(#1\right)}}
\providecommand{\norm}[1]{\left\lVert#1\right\rVert}
\newcommand{\solution}{\noindent \textbf{Solution: }}
\newcommand{\myvec}[1]{\ensuremath{\begin{pmatrix}#1\end{pmatrix}}}
\let\vec\mathbf

\begin{document}

\begin{center}
\title{\textbf{Coordinate Geometry}}
\date{\vspace{-5ex}} %Not to print date automatically
\maketitle
\end{center}

\setcounter{page}{1}



\begin{enumerate}

\item\textbf{Problem statement :} Find the area of a rhombus of its vertices are $\myvec{3 ,0}$, $\myvec{4 ,5}$, $\myvec{-1 ,4}$ and $\myvec{-2 ,-1}$taken in order

\solution \\
\fi
The input vertices for this problem are given as
	\begin{align}
	\vec{A} = \myvec{
		3\\
		0
		},
	\vec{B} = \myvec{
		4\\
		5
		},
        \vec{C} = \myvec{
		-1\\
		4
		},
        \vec{D} = \myvec{
		-2\\
		-1
		}
	\end{align}
Since		
\begin{align}
 \vec{A-D}= \myvec{3 \\ 0} - \myvec{-2 \\-1}= \myvec{5\\1}
 \\
  \vec{B-A}= \myvec{4 \\ 5} - \myvec{3 \\0}= \myvec{1\\5}
\end{align}
the area of the rhombus is
\begin{align}
                \norm{\myvec{\vec{A-D}}\times \myvec{\vec{B-A}}}=\mydet{5 & 1\\1 & 5} = 24
\end{align}
See Fig. 
\ref{fig:chapters/10/7/2/10/gFig1}.
\begin{figure}[!h]
 \begin{center}
  \includegraphics[width=\columnwidth]{chapters/10/7/2/10/figs/fig.pdf}
 \end{center}
\caption{}
\label{fig:chapters/10/7/2/10/gFig1}
\end{figure}


\item Verify that a median of a triangle divides it into two triangles of equal areas for $\triangle ABC$ whose vertices are $\vec{A}(4, -6), \vec{B}(3, 2), \text{ and } \vec{C}(5, 2)$. 
		\label{10/7/3/5}
		\\
\solution
		\iffalse
\documentclass[12pt]{article}
\usepackage{graphicx}
\usepackage[none]{hyphenat}
\usepackage{graphicx}
\usepackage{listings}
\usepackage[english]{babel}
\usepackage{graphicx}
\usepackage{caption} 
\usepackage{booktabs}
\usepackage{array}
\usepackage{amssymb} % for \because
\usepackage{amsmath}   % for having text in math mode
\usepackage{extarrows} % for Row operations arrows
\usepackage{listings}
\usepackage[utf8]{inputenc}
\lstset{
  frame=single,
  breaklines=true
}
\usepackage{hyperref}
  
%Following 2 lines were added to remove the blank page at the beginning
\usepackage{atbegshi}% http://ctan.org/pkg/atbegshi
\AtBeginDocument{\AtBeginShipoutNext{\AtBeginShipoutDiscard}}


%New macro definitions
\newcommand{\mydet}[1]{\ensuremath{\begin{vmatrix}#1\end{vmatrix}}}
\providecommand{\brak}[1]{\ensuremath{\left(#1\right)}}
\newcommand{\solution}{\noindent \textbf{Solution: }}
\newcommand{\myvec}[1]{\ensuremath{\begin{pmatrix}#1\end{pmatrix}}}
\providecommand{\norm}[1]{\left\lVert#1\right\rVert}
\providecommand{\abs}[1]{\left\vert#1\right\vert}
\let\vec\mathbf

\begin{document}

\begin{center}
\title{\textbf{VECTORS}}
\date{\vspace{-5ex}} %Not to print date automatically
\maketitle
\end{center}

\section{10$^{th}$ Maths - EXERCISE-7.3}

\begin{enumerate}
\item That a median of a triangle divides it into two triangles  of equal areas. verify this result for $\triangle ABC$ whose vertices are $\vec{A}(4,-6),\vec{B}(3,-2)\text{ and }\vec{C}(5,2)$.
\end{enumerate}

\section{SOLUTION}
Given points are
\begin{align}
\vec{A}=\myvec{4\\ -6} ,
\vec{B}=\myvec{3\\ -2} ,
\vec{C}=\myvec{5\\ 2}
\end{align}
\fi
The median of the triangle 
\begin{align}
\vec{D}&=\frac{\vec{B}+\vec{C}}{2}\\
&=\myvec{4\\ 0}
\end{align}
Since 
\begin{align}
	\vec{A}- \vec{B} &= \myvec{4\\ -6}-\myvec{3\\ -2}=\myvec{1\\ -4}\label{eq:10/7/3/5/7}\\
	  \vec{A}- \vec{D} &= \myvec{4\\ -6}-\myvec{4\\ 0}=\myvec{0\\ -6}\label{eq:10/7/3/5/8}
  \end{align}
 \begin{align}
  ar(ABD)&=\frac{1}{2} \norm{\brak{\vec{A}-\vec{B}}  \times 
   \brak{\vec{A}- \vec{D}}} \label{eq:10/7/3/5/6} 
   \\
&=\frac{1}{2}\mydet{1 & 0\\-4 & -6}
	       =3	
\end{align}
upon
Substituting from \eqref{eq:10/7/3/5/7} and \eqref{eq:10/7/3/5/8} in \eqref{eq:10/7/3/5/6}.
		Similarly, 
\begin{align}
	\vec{A}- \vec{C} &= \myvec{4\\ -6}-\myvec{5\\ 2}=\myvec{-1\\ -8}\label{eq:10/7/3/5/13} \\
	  \vec{A}- \vec{D} &= \myvec{4\\ -6}-\myvec{4\\ 0}=\myvec{0\\ -6}\label{eq:10/7/3/5/14} 
  \end{align}
  yielding
  \begin{align}
  ar(ACD)&=\frac{1}{2} \norm{\brak{\vec{A}-\vec{C}}  \times 
   \brak{\vec{A}- \vec{D}}} \label{eq:10/7/3/5/12}
   \\
	&=\frac{1}{2}\mydet{-1 & 0\\-8 & -6}= 3
\end{align}
upon substituting from \eqref{eq:10/7/3/5/13} and \eqref{eq:10/7/3/5/14} in \eqref{eq:10/7/3/5/12}.
Thus,
\begin{align}
ar(ABD)=ar(ACD)
\end{align}
See Fig. 
\ref{fig:10/7/3/5/}.
\begin{figure}[h!]
\centering
\includegraphics[width=\columnwidth]{chapters/10/7/3/5/figs/fig.pdf}
\caption{}
\label{fig:10/7/3/5/}
\end{figure} 


\item The two adjacent sides of a parallelogram are 
$2\hat{i}-4\hat{j}+5\hat{k}$  and  $\hat{i}-2\hat{j}-3\hat{k}$.
Find the unit vector parallel to its diagonal. Also, find its area.\\
	\solution
		\iffalse
\documentclass[12pt]{article}
\usepackage{graphicx}
%\documentclass[journal,12pt,twocolumn]{IEEEtran}
\usepackage[none]{hyphenat}
\usepackage{graphicx}
\usepackage{listings}
\usepackage[english]{babel}
\usepackage{graphicx}
\usepackage{caption} 
\usepackage{hyperref}
\usepackage{booktabs}
\def\inputGnumericTable{}
\usepackage{color}                                            %%
    \usepackage{array}                                            %%
    \usepackage{longtable}                                        %%
    \usepackage{calc}                                             %%
    \usepackage{multirow}                                         %%
    \usepackage{hhline}                                           %%
    \usepackage{ifthen}
\usepackage{array}
\usepackage{amsmath}   % for having text in math mode
\usepackage{listings}
\lstset{
language=tex,
frame=single, 
breaklines=true
}
  
%Following 2 lines were added to remove the blank page at the beginning
\usepackage{atbegshi}% http://ctan.org/pkg/atbegshi
\AtBeginDocument{\AtBeginShipoutNext{\AtBeginShipoutDiscard}}
%


%New macro definitions
\newcommand{\mydet}[1]{\ensuremath{\begin{vmatrix}#1\end{vmatrix}}}
\providecommand{\brak}[1]{\ensuremath{\left(#1\right)}}
\providecommand{\norm}[1]{\left\lVert#1\right\rVert}
\newcommand{\solution}{\noindent \textbf{Solution: }}
\newcommand{\myvec}[1]{\ensuremath{\begin{pmatrix}#1\end{pmatrix}}}
\let\vec\mathbf

\begin{document}

\begin{center}
\title{\textbf{Coordinate Geometry}}
\date{\vspace{-5ex}} %Not to print date automatically
\maketitle
\end{center}

\setcounter{page}{1}



\begin{enumerate}

\item\textbf{Problem statement :} Find the area of a rhombus of its vertices are $\myvec{3 ,0}$, $\myvec{4 ,5}$, $\myvec{-1 ,4}$ and $\myvec{-2 ,-1}$taken in order

\solution \\
\fi
The input vertices for this problem are given as
	\begin{align}
	\vec{A} = \myvec{
		3\\
		0
		},
	\vec{B} = \myvec{
		4\\
		5
		},
        \vec{C} = \myvec{
		-1\\
		4
		},
        \vec{D} = \myvec{
		-2\\
		-1
		}
	\end{align}
Since		
\begin{align}
 \vec{A-D}= \myvec{3 \\ 0} - \myvec{-2 \\-1}= \myvec{5\\1}
 \\
  \vec{B-A}= \myvec{4 \\ 5} - \myvec{3 \\0}= \myvec{1\\5}
\end{align}
the area of the rhombus is
\begin{align}
                \norm{\myvec{\vec{A-D}}\times \myvec{\vec{B-A}}}=\mydet{5 & 1\\1 & 5} = 24
\end{align}
See Fig. 
\ref{fig:chapters/10/7/2/10/gFig1}.
\begin{figure}[!h]
 \begin{center}
  \includegraphics[width=\columnwidth]{chapters/10/7/2/10/figs/fig.pdf}
 \end{center}
\caption{}
\label{fig:chapters/10/7/2/10/gFig1}
\end{figure}

\item The vertices of a $\triangle ABC$ are $\vec{A}(4,6), \vec{B}(1,5)$ and  $\vec{C}(7,2)$. A line is drawn to intersect sides $AB$ and $AC$ at $\vec{D}$ and $\vec{E}$ respectively, such that $\frac{AD}{AB} = \frac{AE}{AC} = \frac{1}{4}$. Calculate the area of $\triangle ADE$ and compare it with the area of the $\triangle ABC$.
\\
\solution
	\begin{enumerate}[label=\thesection.\arabic*,ref=\thesection.\theenumi]
\numberwithin{equation}{enumi}
\numberwithin{figure}{enumi}
\numberwithin{table}{enumi}

\item Find the coordinates of the point which divides the join of $(-1,7) \text{ and } (4,-3)$ in the ratio 2:3.
	\\
		\solution
	\iffalse
\documentclass[12pt]{article}
\usepackage{graphicx}
\usepackage{amsmath}
\usepackage{mathtools}
\usepackage{gensymb}

\newcommand{\mydet}[1]{\ensuremath{\begin{vmatrix}#1\end{vmatrix}}}
\providecommand{\brak}[1]{\ensuremath{\left(#1\right)}}
\providecommand{\norm}[1]{\left\lVert#1\right\rVert}
\newcommand{\solution}{\noindent \textbf{Solution: }}
\newcommand{\myvec}[1]{\ensuremath{\begin{pmatrix}#1\end{pmatrix}}}
\let\vec\mathbf

\begin{document}
\begin{center}
\textbf\large{CHAPTER-7 \\ COORDINATE GEOMETRY}
\end{center}
\section*{Excercise 7.2}

1. Find the coordinates of the point which divides the join $\vec(-1,7) \text{ and } \vec(4,-3)$ in the ratio 2:3 :
\\
\\
\solution\\		
\fi
The coordinates and ratio are given as
\begin{align}
\vec{P}=\myvec{-1\\7\\},
\vec{Q}=\myvec{4\\-3\\},
n=\frac{3}{2}
\end{align}
Using section formula
\begin{align}
\vec{R}&=\frac{\vec{Q}+n\vec{P}}{1+n}\\
&=\frac{1}{1+\frac{3}{2}}  \myvec{\myvec{
4\\
-3\\
}
  +
   \frac{3}{2}\myvec{
-1\\
7\\
}}\\
&=\myvec{
1\\
3
}
\end{align}
See Fig. 
\ref{fig:chapters/10/7/2/1/Fig}
\begin{figure}[!h]
\begin{center}
   \includegraphics[width=\columnwidth]{chapters/10/7/2/1/figs/linefig.png}
\end{center}
\caption{}
\label{fig:chapters/10/7/2/1/Fig}
\end{figure}


\item Find the coordinates of the points of trisection of the line segment joining $(4,-1) \text{ and } (-2,3)$.
	\\
		\solution
	\begin{enumerate}[label=\thesection.\arabic*,ref=\thesection.\theenumi]
\numberwithin{equation}{enumi}
\numberwithin{figure}{enumi}
\numberwithin{table}{enumi}

\item Find the coordinates of the point which divides the join of $(-1,7) \text{ and } (4,-3)$ in the ratio 2:3.
	\\
		\solution
	\iffalse
\documentclass[12pt]{article}
\usepackage{graphicx}
\usepackage{amsmath}
\usepackage{mathtools}
\usepackage{gensymb}

\newcommand{\mydet}[1]{\ensuremath{\begin{vmatrix}#1\end{vmatrix}}}
\providecommand{\brak}[1]{\ensuremath{\left(#1\right)}}
\providecommand{\norm}[1]{\left\lVert#1\right\rVert}
\newcommand{\solution}{\noindent \textbf{Solution: }}
\newcommand{\myvec}[1]{\ensuremath{\begin{pmatrix}#1\end{pmatrix}}}
\let\vec\mathbf

\begin{document}
\begin{center}
\textbf\large{CHAPTER-7 \\ COORDINATE GEOMETRY}
\end{center}
\section*{Excercise 7.2}

1. Find the coordinates of the point which divides the join $\vec(-1,7) \text{ and } \vec(4,-3)$ in the ratio 2:3 :
\\
\\
\solution\\		
\fi
The coordinates and ratio are given as
\begin{align}
\vec{P}=\myvec{-1\\7\\},
\vec{Q}=\myvec{4\\-3\\},
n=\frac{3}{2}
\end{align}
Using section formula
\begin{align}
\vec{R}&=\frac{\vec{Q}+n\vec{P}}{1+n}\\
&=\frac{1}{1+\frac{3}{2}}  \myvec{\myvec{
4\\
-3\\
}
  +
   \frac{3}{2}\myvec{
-1\\
7\\
}}\\
&=\myvec{
1\\
3
}
\end{align}
See Fig. 
\ref{fig:chapters/10/7/2/1/Fig}
\begin{figure}[!h]
\begin{center}
   \includegraphics[width=\columnwidth]{chapters/10/7/2/1/figs/linefig.png}
\end{center}
\caption{}
\label{fig:chapters/10/7/2/1/Fig}
\end{figure}


\item Find the coordinates of the points of trisection of the line segment joining $(4,-1) \text{ and } (-2,3)$.
	\\
		\solution
	\begin{enumerate}[label=\thesection.\arabic*,ref=\thesection.\theenumi]
\numberwithin{equation}{enumi}
\numberwithin{figure}{enumi}
\numberwithin{table}{enumi}

\item Find the coordinates of the point which divides the join of $(-1,7) \text{ and } (4,-3)$ in the ratio 2:3.
	\\
		\solution
	\input{chapters/10/7/2/1/section.tex}
\item Find the coordinates of the points of trisection of the line segment joining $(4,-1) \text{ and } (-2,3)$.
	\\
		\solution
	\input{chapters/10/7/2/2/section.tex}
\item
	\iffalse
\item To conduct Sports Day activities, in your rectangular shaped school                   
ground ABCD, lines have 
drawn with chalk powder at a                 
distance of 1m each. 100 flower pots have been placed at a distance of 1m 
from each other along AD, as shown 
in Fig. 7.12. Niharika runs $ \frac {1}{4} $th the 
distance AD on the 2nd line and 
posts a green flag. Preet runs $ \frac {1}{5} $th 
the distance AD on the eighth line 
and posts a red flag. What is the 
distance between both the flags? If 
Rashmi has to post a blue flag exactly 
halfway between the line segment 
joining the two flags, where should 
she post her flag?
\begin{figure}[h!]
  \centering
  \includegraphics[width=\columnwidth]{sc.png}
  \caption{}
\label{fig:10/7/12Fig1}
\end{figure}               
\fi
      
\item Find the ratio in which the line segment joining the points $(-3,10) \text{ and } (6,-8)$ $\text{ is divided by } (-1,6)$.
	\\
		\solution
	\input{chapters/10/7/2/4/section.tex}
\item Find the ratio in which the line segment joining $A(1,-5) \text{ and } B(-4,5)$ $\text{is divided by the x-axis}$. Also find the coordinates of the point of division.
\item If $(1,2), (4,y), (x,6), (3,5)$ are the vertices of a parallelogram taken in order, find x and y.
	\\
		\solution
	\input{chapters/10/7/2/6/para1.tex}
\item Find the coordinates of a point A, where AB is the diameter of a circle whose centre is $(2,-3) \text{ and }$ B is $(1,4)$.
	\\
		\solution
	\input{chapters/10/7/2/7/section.tex}
\item If A \text{ and } B are $(-2,-2) \text{ and } (2,-4)$, respectively, find the coordinates of P such that AP= $\frac {3}{7}$AB $\text{ and }$ P lies on the line segment AB.
	\\
		\solution
	\input{chapters/10/7/2/8/section.tex}
\item Find the coordinates of the points which divide the line segment joining $A(-2,2) \text{ and } B(2,8)$ into four equal parts.
	\\
		\solution
	\input{chapters/10/7/2/9/section.tex}
\item Find the area of a rhombus if its vertices are $(3,0), (4,5), (-1,4) \text{ and } (-2,-1)$ taken in order. [$\vec{Hint}$ : Area of rhombus =$\frac {1}{2}$(product of its diagonals)]
	\\
		\solution
	\input{chapters/10/7/2/10/cross.tex}
\item Find the position vector of a point R which divides the line joining two points $\vec{P}$
and $\vec{Q}$ whose position vectors are $\hat{i}+2\hat{j}-\hat{k}$ and $-\hat{i}+\hat{j}+\hat{k}$ respectively, in the
ratio 2 : 1
\begin{enumerate}
    \item  internally
    \item  externally
\end{enumerate}
\solution
		\input{chapters/12/10/2/15/section.tex}
\item Find the position vector of the mid point of the vector joining the points $\vec{P}$(2, 3, 4)
and $\vec{Q}$(4, 1, –2).
\\
\solution
		\input{chapters/12/10/2/16/section.tex}
\item Determine the ratio in which the line $2x+y  - 4=0$ divides the line segment joining the points $\vec{A}(2, - 2)$  and  $\vec{B}(3, 7)$.
\\
\solution
	\input{chapters/10/7/4/1/section.tex}
\item Let $\vec{A}(4, 2), \vec{B}(6, 5)$  and $ \vec{C}(1, 4)$ be the vertices of $\triangle ABC$.
\begin{enumerate}
\item The median from $\vec{A}$ meets $BC$ at $\vec{D}$. Find the coordinates of the point $\vec{D}$.
\item Find the coordinates of the point $\vec{P}$ on $AD$ such that $AP : PD = 2 : 1$.
\item Find the coordinates of points $\vec{Q}$ and $\vec{R}$ on medians $BE$ and $CF$ respectively such that $BQ : QE = 2 : 1$  and  $CR : RF = 2 : 1$.
\item What do you observe?
\item If $\vec{A}, \vec{B}$ and $\vec{C}$  are the vertices of $\triangle ABC$, find the coordinates of the centroid of the triangle.
\end{enumerate}
\solution
	\input{chapters/10/7/4/7/section.tex}
\item Find the slope of a line, which passes through the origin and the mid point of the line segment joining the points $\vec{P}$(0,-4) and $\vec{B}$(8,0).
\label{chapters/11/10/1/5}
\input{chapters/11/10/1/5/matrix.tex}
\item Find the position vector of a point R which divides the line joining two points P and Q whose position vectors are $(2\vec{a}+\vec{b})$ and $(\vec{a}-3\vec{b})$
externally in the ratio 1 : 2. Also, show that P is the mid point of the line segment RQ.\\
	\solution
%		\input{chapters/12/10/5/9/section.tex}

\end{enumerate}


\item
	\iffalse
\item To conduct Sports Day activities, in your rectangular shaped school                   
ground ABCD, lines have 
drawn with chalk powder at a                 
distance of 1m each. 100 flower pots have been placed at a distance of 1m 
from each other along AD, as shown 
in Fig. 7.12. Niharika runs $ \frac {1}{4} $th the 
distance AD on the 2nd line and 
posts a green flag. Preet runs $ \frac {1}{5} $th 
the distance AD on the eighth line 
and posts a red flag. What is the 
distance between both the flags? If 
Rashmi has to post a blue flag exactly 
halfway between the line segment 
joining the two flags, where should 
she post her flag?
\begin{figure}[h!]
  \centering
  \includegraphics[width=\columnwidth]{sc.png}
  \caption{}
\label{fig:10/7/12Fig1}
\end{figure}               
\fi
      
\item Find the ratio in which the line segment joining the points $(-3,10) \text{ and } (6,-8)$ $\text{ is divided by } (-1,6)$.
	\\
		\solution
	\iffalse
\documentclass[12pt]{article}
\usepackage{graphicx}
%\documentclass[journal,12pt,twocolumn]{IEEEtran}
\usepackage[none]{hyphenat}
\usepackage{graphicx}
\usepackage{listings}
\usepackage[english]{babel}
\usepackage{graphicx}
\usepackage{caption} 
\usepackage{hyperref}
\usepackage{booktabs}
\def\inputGnumericTable{}
\usepackage{color}                                            %%
    \usepackage{array}                                            %%
    \usepackage{longtable}                                        %%
    \usepackage{calc}                                             %%
    \usepackage{multirow}                                         %%
    \usepackage{hhline}                                           %%
    \usepackage{ifthen}
\usepackage{array}
\usepackage{amsmath}   % for having text in math mode
\usepackage{listings}
\lstset{
language=tex,
frame=single, 
breaklines=true
}
  
%Following 2 lines were added to remove the blank page at the beginning
\usepackage{atbegshi}% http://ctan.org/pkg/atbegshi
\AtBeginDocument{\AtBeginShipoutNext{\AtBeginShipoutDiscard}}
%
%New macro definitions
\newcommand{\mydet}[1]{\ensuremath{\begin{vmatrix}#1\end{vmatrix}}}
\providecommand{\brak}[1]{\ensuremath{\left(#1\right)}}
\providecommand{\norm}[1]{\left\lVert#1\right\rVert}
\newcommand{\solution}{\noindent \textbf{Solution: }}
\newcommand{\myvec}[1]{\ensuremath{\begin{pmatrix}#1\end{pmatrix}}}
\let\vec\mathbf
\begin{document}
\begin{center}
\title{\textbf{Coordinate Geometry}}
\date{\vspace{-5ex}} %Not to print date automatically
\maketitle
\end{center}
\setcounter{page}{1}
\section*{10$^{th}$ Maths - Chapter 7}
This is Problem-4 from Exercise 7.2
\begin{enumerate}
\item Find the ratio in which the line segement joining the points $\myvec{-3 \\ 10}$ and $\myvec{6\\-8}$ is divided by $\myvec{-1\\6}$.\\
\solution \\
\fi
		The input parameters for this problem are available in Table \eqref{tab:10/7/2/4-1}.
\begin{table}[ht!]
\input{chapters/10/7/2/4/tables/table.tex}
\caption{}
\label{tab:10/7/2/4-1} 
\end{table}
Using section formula,
\begin{align}
         \vec{R} &=\frac{\vec{Q}+n\vec{P}}{1+n}\label{eq:chapters/10/7/2/4/1}
\end{align}
Substituting the values of $\vec{P},\vec{Q}$ and $\vec{R}$ in \eqref{eq:chapters/10/7/2/4/1}
\begin{align}
         \myvec{-1\\6} &=\frac{{\myvec{-3\\10}+n\myvec{6\\-8}}}{1+n}\\
 &=\frac{1}{1+n}\brak{{\myvec{-3\\10}+n\myvec{6\\-8}}} \\
 &=\frac{1}{1+n}\myvec{-3+6n\\10-8n} \label{eq:chapters/10/7/2/4/4}
\end{align}
Simplifying \eqref{eq:chapters/10/7/2/4/4} yeilds,
\begin{align}
          -1 &=\frac{-3+6n}{1+n}\\
\implies          n &=\frac{2}{7}
\end{align}
Also,
\begin{align}
          6 &=\frac{10-8n}{1+n}\\
    \implies      n &=\frac{2}{7}
\end{align}
Hence the desired ratio is $\dfrac{2}{7}$.  
\begin{figure}[!h]
 \begin{center}
  \includegraphics[width=\columnwidth]{chapters/10/7/2/4/figs/fig.png}
 \end{center}
\caption{}
\label{fig:10/7/2/4Fig1}
\end{figure}

\item Find the ratio in which the line segment joining $A(1,-5) \text{ and } B(-4,5)$ $\text{is divided by the x-axis}$. Also find the coordinates of the point of division.
\item If $(1,2), (4,y), (x,6), (3,5)$ are the vertices of a parallelogram taken in order, find x and y.
	\\
		\solution
	\iffalse
\documentclass[12pt]{article}
\usepackage{graphicx}
%\documentclass[journal,12pt,twocolumn]{IEEEtran}
\def\inputGnumericTable{}
\usepackage{color}                                            %%
    \usepackage{array}                                            %%
    \usepackage{longtable}                                        %%
    \usepackage{calc}                                             %%
    \usepackage{multirow}                                         %%
    \usepackage{hhline}                                           %%
    \usepackage{ifthen}
\usepackage[none]{hyphenat}
\usepackage{graphicx}
\usepackage{listings}
\usepackage[english]{babel}
\usepackage{graphicx}
\usepackage{caption} 
\usepackage{hyperref}
\usepackage{booktabs}
\usepackage{array}
\usepackage{amsmath}   % for having text in math mode
\usepackage{listings}
\lstset{
  frame=single,
  breaklines=true
}
  
%Following 2 lines were added to remove the blank page at the beginning
\usepackage{atbegshi}% http://ctan.org/pkg/atbegshi
\AtBeginDocument{\AtBeginShipoutNext{\AtBeginShipoutDiscard}}
%


%New macro definitions
\newcommand{\mydet}[1]{\ensuremath{\begin{vmatrix}#1\end{vmatrix}}}
\providecommand{\brak}[1]{\ensuremath{\left(#1\right)}}
\providecommand{\norm}[1]{\left\lVert#1\right\rVert}
\newcommand{\solution}{\noindent \textbf{Solution: }}
\newcommand{\myvec}[1]{\ensuremath{\begin{pmatrix}#1\end{pmatrix}}}
\let\vec\mathbf

\begin{document}

\begin{center}
\title{\textbf{Properties of Parallelegram}}
\date{\vspace{-5ex}} %Not to print date automatically
\maketitle
\end{center}

\setcounter{page}{1}

\section{10$^{th}$ Maths - Chapter 7}

This is Problem-6 from Exercise 7.2

\begin{enumerate}
\item If $\vec{A}(1, 2),\vec{B}(4, x),\vec{C}(y, 6) \text{and } \vec{D}(3, 5)$ are the vertices of a parallelogram taken in order,find x and y.
\end{enumerate}
\fi

The input parameters for this problem are available in
\ref{table:chapters/10/7/2/6/tables/}.	
\begin{table}[!ht]
	\centering
	\input{chapters/10/7/2/6/tables/table.tex}
\caption{}
\label{table:chapters/10/7/2/6/tables/}	
\end{table}
From the given information,
\begin{align}
  \label{eq:chapters/10/7/2/6/tables/det2f}
	\vec{B}-\vec{A} &= \myvec{4 \\y } - \myvec{1 \\2 }  = \myvec{3 \\y-2 }\\
	\vec{C}-\vec{D} &= \myvec{x \\6 } - \myvec{3 \\5 }  = \myvec{x-3 \\1}
\end{align}
Since $ABCD$ is a parallellogram,
\begin{align}
	\myvec{3\\y-2}&=\myvec{x-3\\1}\\
	\implies x&=6 ,y=3
\end{align}
Fig. \ref{fig:chapters/10/7/2/6/Fig3}
provides a verification.
\begin{figure}[h!]
	\begin{center}
  \includegraphics[width=\columnwidth]{chapters/10/7/2/6/figs/para.pdf}
	\end{center}
\caption{}
\label{fig:chapters/10/7/2/6/Fig3}
\end{figure}


\item Find the coordinates of a point A, where AB is the diameter of a circle whose centre is $(2,-3) \text{ and }$ B is $(1,4)$.
	\\
		\solution
	\iffalse
\documentclass[12pt]{article}
\usepackage{graphicx}
\usepackage{amsmath}
\usepackage{mathtools}
\usepackage{gensymb}

\newcommand{\mydet}[1]{\ensuremath{\begin{vmatrix}#1\end{vmatrix}}}
\providecommand{\brak}[1]{\ensuremath{\left(#1\right)}}
\providecommand{\norm}[1]{\left\lVert#1\right\rVert}
\newcommand{\solution}{\noindent \textbf{Solution: }}
\newcommand{\myvec}[1]{\ensuremath{\begin{pmatrix}#1\end{pmatrix}}}
\let\vec\mathbf

\begin{document}
\begin{center}
\section*{CHAPTER 7 - COORDINATE GEOMETRY}

\end{center}
\section*{Excercise 7.2}

Q7.Find the coordinates of point $\vec{A}$, where AB is the diameter of a circle where the center is (2,-3) and $\vec{B}$ is the point (1,4):

\solution
\begin{enumerate}
\item The coordinates $\vec{B}$ and center $\vec{C}$ are given, where:
	\fi
	Let
	\begin{align}
	\vec{B} = \myvec{
		1\\
	    4\\
		},
	\vec{C} = \myvec{
	    2\\
	   -3\\
		}
	\end{align}
	\iffalse
Let us assume the coordinates of $\vec{A}$. Now, $\vec{C}$ is the center which is midpoint of line AB and $\vec{B}$ is one of the coordinate of diameter AB of a circle.
	\fi	
Hence,	
	\begin{align}
	\vec{C} &= \frac{\vec{A+B}}{2} \\
\implies	2\vec{C} &= \vec{A}+\vec{B} \\
		\text{or, }	\vec{A} &= 2\vec{C}-\vec{B} \\
	 &= \myvec{3\\-10\\}	
	\end{align}       
	See Fig. 
\ref{fig:chapters/10/7/2/7Fig}.
\begin{figure}[!h]
\begin{center}	
	\includegraphics[width=\columnwidth]{chapters/10/7/2/7/figs/Vector1.png}
\end{center}
\caption{}
\label{fig:chapters/10/7/2/7Fig}
\end{figure}
	

\item If A \text{ and } B are $(-2,-2) \text{ and } (2,-4)$, respectively, find the coordinates of P such that AP= $\frac {3}{7}$AB $\text{ and }$ P lies on the line segment AB.
	\\
		\solution
	\iffalse
\documentclass[journal,10pt,twocolumn]{article}
\usepackage{graphicx}
\usepackage[none]{hyphenat}
\usepackage{graphicx}
\usepackage{listings}
\usepackage[english]{babel}
\usepackage{graphicx}
\usepackage{caption} 
\usepackage{booktabs}
\usepackage{array}
\usepackage{amssymb} % for \because
\usepackage{amsmath}   % for having text in math mode
\usepackage{extarrows} % for Row operations arrows
\usepackage{listings}
\usepackage[utf8]{inputenc}
\lstset{
  frame=single,
  breaklines=true
}
\usepackage{hyperref}
  
%Following 2 lines were added to remove the blank page at the beginning
\usepackage{atbegshi}% http://ctan.org/pkg/atbegshi
\AtBeginDocument{\AtBeginShipoutNext{\AtBeginShipoutDiscard}}


%New macro definitions
\newcommand{\mydet}[1]{\ensuremath{\begin{vmatrix}#1\end{vmatrix}}}
\providecommand{\brak}[1]{\ensuremath{\left(#1\right)}}
\newcommand{\solution}{\noindent \textbf{Solution: }}
\newcommand{\myvec}[1]{\ensuremath{\begin{pmatrix}#1\end{pmatrix}}}
\providecommand{\norm}[1]{\left\lVert#1\right\rVert}
\providecommand{\abs}[1]{\left\vert#1\right\vert}
\let\vec\mathbf

\begin{document}

\begin{center}
\title{\textbf{VECTORS}}
\date{\vspace{-5ex}} %Not to print date automatically
\maketitle
\end{center}

\section{10$^{th}$ Maths - EXERCISE-7.2}

\begin{enumerate}
\item If A and B are $(– 2, – 2)\text{ and }(2, – 4)$, respectively, find the coordinates of P such that $AP =\frac{3}{7}AB$ and P lies on the line segment AB. 

\section{SOLUTION}
Given points are
\begin{align}
\vec{A}=\myvec{-2\\ -2} ,
\vec{B}=\myvec{2\\ -4}
\end{align}
The equation of the formula is
\fi
Using section formula, 
\begin{align}
\vec{P}&=\frac{\vec{A}+n\vec{B}}{1+n}
\end{align}
where
\begin{align}
	n =\frac{3}{4}
\end{align}
Thus,
\begin{align}
\vec{P}&=\frac{1}{1+\frac{3}{4}}\brak{\myvec{-2\\-2}+\frac{3}{4}\myvec{2\\-4}}\\
&=\myvec{\frac{-2}{7}\\[1pt] \frac{-20}{7}}
\end{align}
See Fig. 
   \ref{fig:chapters/10/7/2/8/vec.png}
\begin{figure}
   \centering 
 \includegraphics[width=\columnwidth]{chapters/10/7/2/8/figs/vec.png}
   \caption{}
   \label{fig:chapters/10/7/2/8/vec.png}
   \end{figure}

\item Find the coordinates of the points which divide the line segment joining $A(-2,2) \text{ and } B(2,8)$ into four equal parts.
	\\
		\solution
	\begin{enumerate}[label=\thesection.\arabic*,ref=\thesection.\theenumi]
\numberwithin{equation}{enumi}
\numberwithin{figure}{enumi}
\numberwithin{table}{enumi}

\item Find the coordinates of the point which divides the join of $(-1,7) \text{ and } (4,-3)$ in the ratio 2:3.
	\\
		\solution
	\input{chapters/10/7/2/1/section.tex}
\item Find the coordinates of the points of trisection of the line segment joining $(4,-1) \text{ and } (-2,3)$.
	\\
		\solution
	\input{chapters/10/7/2/2/section.tex}
\item
	\iffalse
\item To conduct Sports Day activities, in your rectangular shaped school                   
ground ABCD, lines have 
drawn with chalk powder at a                 
distance of 1m each. 100 flower pots have been placed at a distance of 1m 
from each other along AD, as shown 
in Fig. 7.12. Niharika runs $ \frac {1}{4} $th the 
distance AD on the 2nd line and 
posts a green flag. Preet runs $ \frac {1}{5} $th 
the distance AD on the eighth line 
and posts a red flag. What is the 
distance between both the flags? If 
Rashmi has to post a blue flag exactly 
halfway between the line segment 
joining the two flags, where should 
she post her flag?
\begin{figure}[h!]
  \centering
  \includegraphics[width=\columnwidth]{sc.png}
  \caption{}
\label{fig:10/7/12Fig1}
\end{figure}               
\fi
      
\item Find the ratio in which the line segment joining the points $(-3,10) \text{ and } (6,-8)$ $\text{ is divided by } (-1,6)$.
	\\
		\solution
	\input{chapters/10/7/2/4/section.tex}
\item Find the ratio in which the line segment joining $A(1,-5) \text{ and } B(-4,5)$ $\text{is divided by the x-axis}$. Also find the coordinates of the point of division.
\item If $(1,2), (4,y), (x,6), (3,5)$ are the vertices of a parallelogram taken in order, find x and y.
	\\
		\solution
	\input{chapters/10/7/2/6/para1.tex}
\item Find the coordinates of a point A, where AB is the diameter of a circle whose centre is $(2,-3) \text{ and }$ B is $(1,4)$.
	\\
		\solution
	\input{chapters/10/7/2/7/section.tex}
\item If A \text{ and } B are $(-2,-2) \text{ and } (2,-4)$, respectively, find the coordinates of P such that AP= $\frac {3}{7}$AB $\text{ and }$ P lies on the line segment AB.
	\\
		\solution
	\input{chapters/10/7/2/8/section.tex}
\item Find the coordinates of the points which divide the line segment joining $A(-2,2) \text{ and } B(2,8)$ into four equal parts.
	\\
		\solution
	\input{chapters/10/7/2/9/section.tex}
\item Find the area of a rhombus if its vertices are $(3,0), (4,5), (-1,4) \text{ and } (-2,-1)$ taken in order. [$\vec{Hint}$ : Area of rhombus =$\frac {1}{2}$(product of its diagonals)]
	\\
		\solution
	\input{chapters/10/7/2/10/cross.tex}
\item Find the position vector of a point R which divides the line joining two points $\vec{P}$
and $\vec{Q}$ whose position vectors are $\hat{i}+2\hat{j}-\hat{k}$ and $-\hat{i}+\hat{j}+\hat{k}$ respectively, in the
ratio 2 : 1
\begin{enumerate}
    \item  internally
    \item  externally
\end{enumerate}
\solution
		\input{chapters/12/10/2/15/section.tex}
\item Find the position vector of the mid point of the vector joining the points $\vec{P}$(2, 3, 4)
and $\vec{Q}$(4, 1, –2).
\\
\solution
		\input{chapters/12/10/2/16/section.tex}
\item Determine the ratio in which the line $2x+y  - 4=0$ divides the line segment joining the points $\vec{A}(2, - 2)$  and  $\vec{B}(3, 7)$.
\\
\solution
	\input{chapters/10/7/4/1/section.tex}
\item Let $\vec{A}(4, 2), \vec{B}(6, 5)$  and $ \vec{C}(1, 4)$ be the vertices of $\triangle ABC$.
\begin{enumerate}
\item The median from $\vec{A}$ meets $BC$ at $\vec{D}$. Find the coordinates of the point $\vec{D}$.
\item Find the coordinates of the point $\vec{P}$ on $AD$ such that $AP : PD = 2 : 1$.
\item Find the coordinates of points $\vec{Q}$ and $\vec{R}$ on medians $BE$ and $CF$ respectively such that $BQ : QE = 2 : 1$  and  $CR : RF = 2 : 1$.
\item What do you observe?
\item If $\vec{A}, \vec{B}$ and $\vec{C}$  are the vertices of $\triangle ABC$, find the coordinates of the centroid of the triangle.
\end{enumerate}
\solution
	\input{chapters/10/7/4/7/section.tex}
\item Find the slope of a line, which passes through the origin and the mid point of the line segment joining the points $\vec{P}$(0,-4) and $\vec{B}$(8,0).
\label{chapters/11/10/1/5}
\input{chapters/11/10/1/5/matrix.tex}
\item Find the position vector of a point R which divides the line joining two points P and Q whose position vectors are $(2\vec{a}+\vec{b})$ and $(\vec{a}-3\vec{b})$
externally in the ratio 1 : 2. Also, show that P is the mid point of the line segment RQ.\\
	\solution
%		\input{chapters/12/10/5/9/section.tex}

\end{enumerate}


\item Find the area of a rhombus if its vertices are $(3,0), (4,5), (-1,4) \text{ and } (-2,-1)$ taken in order. [$\vec{Hint}$ : Area of rhombus =$\frac {1}{2}$(product of its diagonals)]
	\\
		\solution
	\iffalse
\documentclass[12pt]{article}
\usepackage{graphicx}
%\documentclass[journal,12pt,twocolumn]{IEEEtran}
\usepackage[none]{hyphenat}
\usepackage{graphicx}
\usepackage{listings}
\usepackage[english]{babel}
\usepackage{graphicx}
\usepackage{caption} 
\usepackage{hyperref}
\usepackage{booktabs}
\def\inputGnumericTable{}
\usepackage{color}                                            %%
    \usepackage{array}                                            %%
    \usepackage{longtable}                                        %%
    \usepackage{calc}                                             %%
    \usepackage{multirow}                                         %%
    \usepackage{hhline}                                           %%
    \usepackage{ifthen}
\usepackage{array}
\usepackage{amsmath}   % for having text in math mode
\usepackage{listings}
\lstset{
language=tex,
frame=single, 
breaklines=true
}
  
%Following 2 lines were added to remove the blank page at the beginning
\usepackage{atbegshi}% http://ctan.org/pkg/atbegshi
\AtBeginDocument{\AtBeginShipoutNext{\AtBeginShipoutDiscard}}
%


%New macro definitions
\newcommand{\mydet}[1]{\ensuremath{\begin{vmatrix}#1\end{vmatrix}}}
\providecommand{\brak}[1]{\ensuremath{\left(#1\right)}}
\providecommand{\norm}[1]{\left\lVert#1\right\rVert}
\newcommand{\solution}{\noindent \textbf{Solution: }}
\newcommand{\myvec}[1]{\ensuremath{\begin{pmatrix}#1\end{pmatrix}}}
\let\vec\mathbf

\begin{document}

\begin{center}
\title{\textbf{Coordinate Geometry}}
\date{\vspace{-5ex}} %Not to print date automatically
\maketitle
\end{center}

\setcounter{page}{1}



\begin{enumerate}

\item\textbf{Problem statement :} Find the area of a rhombus of its vertices are $\myvec{3 ,0}$, $\myvec{4 ,5}$, $\myvec{-1 ,4}$ and $\myvec{-2 ,-1}$taken in order

\solution \\
\fi
The input vertices for this problem are given as
	\begin{align}
	\vec{A} = \myvec{
		3\\
		0
		},
	\vec{B} = \myvec{
		4\\
		5
		},
        \vec{C} = \myvec{
		-1\\
		4
		},
        \vec{D} = \myvec{
		-2\\
		-1
		}
	\end{align}
Since		
\begin{align}
 \vec{A-D}= \myvec{3 \\ 0} - \myvec{-2 \\-1}= \myvec{5\\1}
 \\
  \vec{B-A}= \myvec{4 \\ 5} - \myvec{3 \\0}= \myvec{1\\5}
\end{align}
the area of the rhombus is
\begin{align}
                \norm{\myvec{\vec{A-D}}\times \myvec{\vec{B-A}}}=\mydet{5 & 1\\1 & 5} = 24
\end{align}
See Fig. 
\ref{fig:chapters/10/7/2/10/gFig1}.
\begin{figure}[!h]
 \begin{center}
  \includegraphics[width=\columnwidth]{chapters/10/7/2/10/figs/fig.pdf}
 \end{center}
\caption{}
\label{fig:chapters/10/7/2/10/gFig1}
\end{figure}

\item Find the position vector of a point R which divides the line joining two points $\vec{P}$
and $\vec{Q}$ whose position vectors are $\hat{i}+2\hat{j}-\hat{k}$ and $-\hat{i}+\hat{j}+\hat{k}$ respectively, in the
ratio 2 : 1
\begin{enumerate}
    \item  internally
    \item  externally
\end{enumerate}
\solution
		\begin{enumerate}[label=\thesection.\arabic*,ref=\thesection.\theenumi]
\numberwithin{equation}{enumi}
\numberwithin{figure}{enumi}
\numberwithin{table}{enumi}

\item Find the coordinates of the point which divides the join of $(-1,7) \text{ and } (4,-3)$ in the ratio 2:3.
	\\
		\solution
	\input{chapters/10/7/2/1/section.tex}
\item Find the coordinates of the points of trisection of the line segment joining $(4,-1) \text{ and } (-2,3)$.
	\\
		\solution
	\input{chapters/10/7/2/2/section.tex}
\item
	\iffalse
\item To conduct Sports Day activities, in your rectangular shaped school                   
ground ABCD, lines have 
drawn with chalk powder at a                 
distance of 1m each. 100 flower pots have been placed at a distance of 1m 
from each other along AD, as shown 
in Fig. 7.12. Niharika runs $ \frac {1}{4} $th the 
distance AD on the 2nd line and 
posts a green flag. Preet runs $ \frac {1}{5} $th 
the distance AD on the eighth line 
and posts a red flag. What is the 
distance between both the flags? If 
Rashmi has to post a blue flag exactly 
halfway between the line segment 
joining the two flags, where should 
she post her flag?
\begin{figure}[h!]
  \centering
  \includegraphics[width=\columnwidth]{sc.png}
  \caption{}
\label{fig:10/7/12Fig1}
\end{figure}               
\fi
      
\item Find the ratio in which the line segment joining the points $(-3,10) \text{ and } (6,-8)$ $\text{ is divided by } (-1,6)$.
	\\
		\solution
	\input{chapters/10/7/2/4/section.tex}
\item Find the ratio in which the line segment joining $A(1,-5) \text{ and } B(-4,5)$ $\text{is divided by the x-axis}$. Also find the coordinates of the point of division.
\item If $(1,2), (4,y), (x,6), (3,5)$ are the vertices of a parallelogram taken in order, find x and y.
	\\
		\solution
	\input{chapters/10/7/2/6/para1.tex}
\item Find the coordinates of a point A, where AB is the diameter of a circle whose centre is $(2,-3) \text{ and }$ B is $(1,4)$.
	\\
		\solution
	\input{chapters/10/7/2/7/section.tex}
\item If A \text{ and } B are $(-2,-2) \text{ and } (2,-4)$, respectively, find the coordinates of P such that AP= $\frac {3}{7}$AB $\text{ and }$ P lies on the line segment AB.
	\\
		\solution
	\input{chapters/10/7/2/8/section.tex}
\item Find the coordinates of the points which divide the line segment joining $A(-2,2) \text{ and } B(2,8)$ into four equal parts.
	\\
		\solution
	\input{chapters/10/7/2/9/section.tex}
\item Find the area of a rhombus if its vertices are $(3,0), (4,5), (-1,4) \text{ and } (-2,-1)$ taken in order. [$\vec{Hint}$ : Area of rhombus =$\frac {1}{2}$(product of its diagonals)]
	\\
		\solution
	\input{chapters/10/7/2/10/cross.tex}
\item Find the position vector of a point R which divides the line joining two points $\vec{P}$
and $\vec{Q}$ whose position vectors are $\hat{i}+2\hat{j}-\hat{k}$ and $-\hat{i}+\hat{j}+\hat{k}$ respectively, in the
ratio 2 : 1
\begin{enumerate}
    \item  internally
    \item  externally
\end{enumerate}
\solution
		\input{chapters/12/10/2/15/section.tex}
\item Find the position vector of the mid point of the vector joining the points $\vec{P}$(2, 3, 4)
and $\vec{Q}$(4, 1, –2).
\\
\solution
		\input{chapters/12/10/2/16/section.tex}
\item Determine the ratio in which the line $2x+y  - 4=0$ divides the line segment joining the points $\vec{A}(2, - 2)$  and  $\vec{B}(3, 7)$.
\\
\solution
	\input{chapters/10/7/4/1/section.tex}
\item Let $\vec{A}(4, 2), \vec{B}(6, 5)$  and $ \vec{C}(1, 4)$ be the vertices of $\triangle ABC$.
\begin{enumerate}
\item The median from $\vec{A}$ meets $BC$ at $\vec{D}$. Find the coordinates of the point $\vec{D}$.
\item Find the coordinates of the point $\vec{P}$ on $AD$ such that $AP : PD = 2 : 1$.
\item Find the coordinates of points $\vec{Q}$ and $\vec{R}$ on medians $BE$ and $CF$ respectively such that $BQ : QE = 2 : 1$  and  $CR : RF = 2 : 1$.
\item What do you observe?
\item If $\vec{A}, \vec{B}$ and $\vec{C}$  are the vertices of $\triangle ABC$, find the coordinates of the centroid of the triangle.
\end{enumerate}
\solution
	\input{chapters/10/7/4/7/section.tex}
\item Find the slope of a line, which passes through the origin and the mid point of the line segment joining the points $\vec{P}$(0,-4) and $\vec{B}$(8,0).
\label{chapters/11/10/1/5}
\input{chapters/11/10/1/5/matrix.tex}
\item Find the position vector of a point R which divides the line joining two points P and Q whose position vectors are $(2\vec{a}+\vec{b})$ and $(\vec{a}-3\vec{b})$
externally in the ratio 1 : 2. Also, show that P is the mid point of the line segment RQ.\\
	\solution
%		\input{chapters/12/10/5/9/section.tex}

\end{enumerate}


\item Find the position vector of the mid point of the vector joining the points $\vec{P}$(2, 3, 4)
and $\vec{Q}$(4, 1, –2).
\\
\solution
		\begin{enumerate}[label=\thesection.\arabic*,ref=\thesection.\theenumi]
\numberwithin{equation}{enumi}
\numberwithin{figure}{enumi}
\numberwithin{table}{enumi}

\item Find the coordinates of the point which divides the join of $(-1,7) \text{ and } (4,-3)$ in the ratio 2:3.
	\\
		\solution
	\input{chapters/10/7/2/1/section.tex}
\item Find the coordinates of the points of trisection of the line segment joining $(4,-1) \text{ and } (-2,3)$.
	\\
		\solution
	\input{chapters/10/7/2/2/section.tex}
\item
	\iffalse
\item To conduct Sports Day activities, in your rectangular shaped school                   
ground ABCD, lines have 
drawn with chalk powder at a                 
distance of 1m each. 100 flower pots have been placed at a distance of 1m 
from each other along AD, as shown 
in Fig. 7.12. Niharika runs $ \frac {1}{4} $th the 
distance AD on the 2nd line and 
posts a green flag. Preet runs $ \frac {1}{5} $th 
the distance AD on the eighth line 
and posts a red flag. What is the 
distance between both the flags? If 
Rashmi has to post a blue flag exactly 
halfway between the line segment 
joining the two flags, where should 
she post her flag?
\begin{figure}[h!]
  \centering
  \includegraphics[width=\columnwidth]{sc.png}
  \caption{}
\label{fig:10/7/12Fig1}
\end{figure}               
\fi
      
\item Find the ratio in which the line segment joining the points $(-3,10) \text{ and } (6,-8)$ $\text{ is divided by } (-1,6)$.
	\\
		\solution
	\input{chapters/10/7/2/4/section.tex}
\item Find the ratio in which the line segment joining $A(1,-5) \text{ and } B(-4,5)$ $\text{is divided by the x-axis}$. Also find the coordinates of the point of division.
\item If $(1,2), (4,y), (x,6), (3,5)$ are the vertices of a parallelogram taken in order, find x and y.
	\\
		\solution
	\input{chapters/10/7/2/6/para1.tex}
\item Find the coordinates of a point A, where AB is the diameter of a circle whose centre is $(2,-3) \text{ and }$ B is $(1,4)$.
	\\
		\solution
	\input{chapters/10/7/2/7/section.tex}
\item If A \text{ and } B are $(-2,-2) \text{ and } (2,-4)$, respectively, find the coordinates of P such that AP= $\frac {3}{7}$AB $\text{ and }$ P lies on the line segment AB.
	\\
		\solution
	\input{chapters/10/7/2/8/section.tex}
\item Find the coordinates of the points which divide the line segment joining $A(-2,2) \text{ and } B(2,8)$ into four equal parts.
	\\
		\solution
	\input{chapters/10/7/2/9/section.tex}
\item Find the area of a rhombus if its vertices are $(3,0), (4,5), (-1,4) \text{ and } (-2,-1)$ taken in order. [$\vec{Hint}$ : Area of rhombus =$\frac {1}{2}$(product of its diagonals)]
	\\
		\solution
	\input{chapters/10/7/2/10/cross.tex}
\item Find the position vector of a point R which divides the line joining two points $\vec{P}$
and $\vec{Q}$ whose position vectors are $\hat{i}+2\hat{j}-\hat{k}$ and $-\hat{i}+\hat{j}+\hat{k}$ respectively, in the
ratio 2 : 1
\begin{enumerate}
    \item  internally
    \item  externally
\end{enumerate}
\solution
		\input{chapters/12/10/2/15/section.tex}
\item Find the position vector of the mid point of the vector joining the points $\vec{P}$(2, 3, 4)
and $\vec{Q}$(4, 1, –2).
\\
\solution
		\input{chapters/12/10/2/16/section.tex}
\item Determine the ratio in which the line $2x+y  - 4=0$ divides the line segment joining the points $\vec{A}(2, - 2)$  and  $\vec{B}(3, 7)$.
\\
\solution
	\input{chapters/10/7/4/1/section.tex}
\item Let $\vec{A}(4, 2), \vec{B}(6, 5)$  and $ \vec{C}(1, 4)$ be the vertices of $\triangle ABC$.
\begin{enumerate}
\item The median from $\vec{A}$ meets $BC$ at $\vec{D}$. Find the coordinates of the point $\vec{D}$.
\item Find the coordinates of the point $\vec{P}$ on $AD$ such that $AP : PD = 2 : 1$.
\item Find the coordinates of points $\vec{Q}$ and $\vec{R}$ on medians $BE$ and $CF$ respectively such that $BQ : QE = 2 : 1$  and  $CR : RF = 2 : 1$.
\item What do you observe?
\item If $\vec{A}, \vec{B}$ and $\vec{C}$  are the vertices of $\triangle ABC$, find the coordinates of the centroid of the triangle.
\end{enumerate}
\solution
	\input{chapters/10/7/4/7/section.tex}
\item Find the slope of a line, which passes through the origin and the mid point of the line segment joining the points $\vec{P}$(0,-4) and $\vec{B}$(8,0).
\label{chapters/11/10/1/5}
\input{chapters/11/10/1/5/matrix.tex}
\item Find the position vector of a point R which divides the line joining two points P and Q whose position vectors are $(2\vec{a}+\vec{b})$ and $(\vec{a}-3\vec{b})$
externally in the ratio 1 : 2. Also, show that P is the mid point of the line segment RQ.\\
	\solution
%		\input{chapters/12/10/5/9/section.tex}

\end{enumerate}


\item Determine the ratio in which the line $2x+y  - 4=0$ divides the line segment joining the points $\vec{A}(2, - 2)$  and  $\vec{B}(3, 7)$.
\\
\solution
	\iffalse
\documentclass[journal,12pt,twocolumn]{IEEEtran}
\usepackage{graphicx}
\graphicspath{{./chapters/10/7/4/1/figs/}}{}
\usepackage{amsmath,amssymb,amsfonts,amsthm}
\newcommand{\myvec}[1]{\ensuremath{\begin{pmatrix}#1\end{pmatrix}}}
\providecommand{\norm}[1]{\lVert#1\rVert}
\usepackage{listings}
\usepackage{watermark}
\usepackage{titlesec}
\usepackage{caption}
\let\vec\mathbf
\lstset{
frame=single, 
breaklines=true,
columns=fullflexible
}
\thiswatermark{\centering \put(0,-105.0){\includegraphics[scale=0.15]{/sdcard/IITH/vector/vectpr-4/chapters/10/7/4/1/figs/logo.png}} }
\title{\mytitle}
\title{
Assignment - Vector-4
}
\author{Surajit Sarkar}
\begin{document}
\maketitle
%\tableofcontents
\bigskip
\section{\textbf{Problem}}
Determine the ratio in which the line 2x+y–4=0 divides the line segment joining the points A(2,–2) and B(3,7).
\section{\textbf{Solution}}
\begin{table}[h]
    \centering
    \begin{tabular}{|c|c|}
       \hline
       \textbf{Symbol}&\textbf{Value}  \\
       \hline
	    $\vec{A}$ & $\myvec{2\\-2}$\\
        \hline
	    $\vec{B}$ & $\myvec{3\\7}$\\
        \hline
	    c&$4$\\
        \hline
       $\vec{n}$ & $\myvec{2\\1}$\\
       \hline
    \end{tabular}
    \caption{Parameters}
    \label{tab:my_label}
\end{table}
Given equation
\fi
The given equation can be expressed as
\begin{align}
    \myvec{2&1}\vec{x}&=4\\
\end{align}
Using section formula, the point of division 
\begin{align}
    \vec{P} = \frac{k\vec{B+A}}{k+1}
\end{align}
which upon substitution in the equation of a line yields
\begin{align}
    \implies\vec{n}^{\top}\myvec{\frac{k\vec{B+A}}{k+1}}&=c\\
    \implies k&=\frac{c-\vec{n}^{\top}\vec{A}}{\vec{n}^{\top}\vec{B}-c}\\
\end{align}
upon simplification.  Substituting numerical values, 
\begin{align}
    k=\frac{2}{9}
\end{align}
See Fig. 
\ref{fig:chapters/10/7/4/1vec}.
\begin{figure}[!h]
\centering
\includegraphics[width=\columnwidth]{chapters/10/7/4/1/figs/vec.pdf}
\caption{}
\label{fig:chapters/10/7/4/1vec}
\end{figure}


\item Let $\vec{A}(4, 2), \vec{B}(6, 5)$  and $ \vec{C}(1, 4)$ be the vertices of $\triangle ABC$.
\begin{enumerate}
\item The median from $\vec{A}$ meets $BC$ at $\vec{D}$. Find the coordinates of the point $\vec{D}$.
\item Find the coordinates of the point $\vec{P}$ on $AD$ such that $AP : PD = 2 : 1$.
\item Find the coordinates of points $\vec{Q}$ and $\vec{R}$ on medians $BE$ and $CF$ respectively such that $BQ : QE = 2 : 1$  and  $CR : RF = 2 : 1$.
\item What do you observe?
\item If $\vec{A}, \vec{B}$ and $\vec{C}$  are the vertices of $\triangle ABC$, find the coordinates of the centroid of the triangle.
\end{enumerate}
\solution
	\iffalse
\documentclass[12pt]{article}
\usepackage{graphicx}
\usepackage[none]{hyphenat}
\usepackage{graphicx}
\usepackage{listings}
\usepackage[english]{babel}
\usepackage{graphicx}
\usepackage{caption} 
\usepackage{booktabs}
\usepackage{array}
\usepackage{amssymb} % for \because
\usepackage{amsmath}   % for having text in math mode
\usepackage{extarrows} % for Row operations arrows
\usepackage{listings}
\usepackage[utf8]{inputenc}
\lstset{
  frame=single,
  breaklines=true
}
\usepackage{hyperref}
  
%Following 2 lines were added to remove the blank page at the beginning
\usepackage{atbegshi}% http://ctan.org/pkg/atbegshi
\AtBeginDocument{\AtBeginShipoutNext{\AtBeginShipoutDiscard}}


%New macro definitions
\newcommand{\mydet}[1]{\ensuremath{\begin{vmatrix}#1\end{vmatrix}}}
\providecommand{\brak}[1]{\ensuremath{\left(#1\right)}}
\newcommand{\solution}{\noindent \textbf{Solution: }}
\newcommand{\myvec}[1]{\ensuremath{\begin{pmatrix}#1\end{pmatrix}}}
\providecommand{\norm}[1]{\left\lVert#1\right\rVert}
\providecommand{\abs}[1]{\left\vert#1\right\vert}
\let\vec\mathbf

\begin{document}

\begin{center}
\title{\textbf{VECTORS}}
\date{\vspace{-5ex}} %Not to print date automatically
\maketitle
\end{center}

\section{10$^{th}$ Maths - EXERCISE-7.4}

Let A(4, 2), B(6, 5) and C(1, 4) be the vertices of $\triangle ABC$
\begin{enumerate}
\item The median from A meets BC at D. Find the coordinates of the point D.
\item Find the coordinates of the point P on AD such that $AP : PD = 2 : 1$
\item Find the coordinates of points Q and R on medians BE and CF respectively such
that $BQ : QE = 2 : 1 \text{and} CR : RF = 2 : 1.$
\item What do yo observe?
\item If $A(x_1, y_1), B(x_2, y_2) \text{and} C(x_3, y_3)$ are the vertices of $\triangle ABC$, find the coordinates of the centroid of the triangle.
\end{enumerate}

Given points are
\begin{align}
\vec{A}=\myvec{4\\ 2} ,
\vec{B}=\myvec{6\\ 5} ,
\vec{C}=\myvec{1\\ 4}
\end{align}
\fi

\begin{enumerate}
\item 
\begin{align}
\vec{D}&=\frac{\vec{B}+\vec{C}}{2}\\
&=\myvec{\frac{7}{2}\\[2pt] \frac{9}{2}}\\
\vec{E}&=\frac{\vec{A}+\vec{C}}{2}\\
&=\myvec{\frac{5}{2}\\ 3}\\
\vec{F}&=\frac{\vec{A}+\vec{B}}{2}\\
&=\myvec{5\\ \frac{7}{2}}
\end{align}

\item 
	For
$n=2$,
\begin{align}
\vec{P}&=\frac{1}{1+n}\brak{\myvec{\vec{A}+n\vec{D}}}\\
&=\frac{1}{3}\myvec{11\\11}
\end{align}

\item 
\begin{align}
\vec{Q}&=\frac{1}{1+n}\brak{\myvec{\vec{B}+n\vec{E}}}\\
&=\frac{1}{3}\myvec{11\\11}\\
\vec{R}&=\frac{1}{1+n}\brak{\myvec{\vec{C}+n\vec{F}}}\\
&=\frac{1}{3}\myvec{11\\11}\\
\end{align}

\item 
 $\vec{P},\vec{Q},\vec{R}$ are the same point.
   
\item 
\begin{align}
\vec{G}&=\frac{\vec{D}+\vec{E}+\vec{F}}{3}\\
&=\frac{1}{3}\myvec{11\\11}\\
\end{align} 
\end{enumerate}
See Fig.  
  \ref{fig:chapters/10/7/4/7/Figure}.
\begin{figure}[h!]
\centering
\includegraphics[width=\columnwidth]{chapters/10/7/4/7/figs/dj.pdf}
\caption{}
  \label{fig:chapters/10/7/4/7/Figure}
\end{figure}

\item Find the slope of a line, which passes through the origin and the mid point of the line segment joining the points $\vec{P}$(0,-4) and $\vec{B}$(8,0).
\label{chapters/11/10/1/5}
\iffalse
\documentclass[journal,12pt,twocolumn]{IEEEtran}
\usepackage{graphicx}
\graphicspath{{./figs/}}{}
\usepackage{amsmath,amssymb,amsfonts,amsthm}
\newcommand{\myvec}[1]{\ensuremath{\begin{pmatrix}#1\end{pmatrix}}}

\let\vec\mathbf

\title{
Matrix-Lines
}
\author{Jyothsna Paluchuri-FWC22059\\}
\begin{document}
\maketitle
\tableofcontents
\bigskip
\section{Problem Statement}
\fi
	\begin{figure}[!ht]
		\centering
 \includegraphics[width=\columnwidth]{chapters/11/10/1/5/figs/line.png}
		\caption{}
		\label{fig:11/10/1/5}
  	\end{figure}
	\\
	\solution
\iffalse
\section{Construction}
\begin{figure}[h]
    \centering
\includegraphics[width=\columnwidth]{line.png}
    \caption{Equation of the slope}
    \label{fig:my_label}
\end{figure}
\vspace{2cm}
\begin{table}[h]
    \centering
    \begin{tabular}{|c|c|c|c|}
       \hline
       \textbf{Symbol}&\textbf{Value}&\textbf{Description}  \\
       \hline
	    $\vec{P}$ & $\myvec{
		    0\\
		    -4}$
	    & Point on Y-axis\\
        \hline
	    $\vec{B}$ & $\myvec{8\\0}$
 & Point on X-axis\\
        \hline
	    $\vec{0}$ & $\myvec{0\\0}$
 & Origin\\
        \hline
    \end{tabular}
    \caption{Parameters}
    \label{tab:my_label}
\end{table}


\section{Solution}
Given that resultant line passes through origin and mid point of the line segment joining point P(0,-4) and B(8,0) \\
\\
\\
given ${\vec{P}}$=$\myvec{
  0\\
  -4}$
 , ${\vec{B}}$=$\myvec{
  8\\
  0}$
  
 \fi 
The mid point of $PB$ is
\begin{align}
\vec{M} &=\frac{1}{2}(\vec{P}+\vec{B})
	= \myvec{4 \\ -2}  
\end{align}
The direction vector of line joining $\vec{O}, \vec{M}$ is 
\begin{align}
\vec{m}&=\vec{O}-\vec{M}
 = -\vec{M}
\end{align}
which can be expressed as
\begin{align}
	\myvec{1 \\ -\frac{1}{2}}
\end{align}
Thus the slope is
\begin{align}
	m = -\frac{1}{2}
\end{align}
\iffalse
\textbf{The direction vector of a line expressed as}
\begin{align}
\implies\vec{m} &= \begin{pmatrix}1 \\ m \\ \end{pmatrix}
\end{align}

\textbf{By solving equation (5) and (6),we get the slope of $\vec{O}$ $\vec{M}$ line}
\begin{align}
        \boxed{m=-0.5}
 \end{align}

\section{Software}
Download the following code using,
\begin{table}[h]
    \centering
    \begin{tabular}{|c|}
    \hline \\
   https://github.com/jyothsna777/jyothsna-fwc.git  \\
         \\
\hline
    \end{tabular}
\end{table}
\\
and execute the code by using command
\begin{center}
\textbf{Python3 lines.py}\\
\end{center}

\section{Conclusion}
Hence the slope of line $\vec{O}$ $\vec{M}$ lineis $\vec{m}$=-0.5

\end{document}
\fi

\item Find the position vector of a point R which divides the line joining two points P and Q whose position vectors are $(2\vec{a}+\vec{b})$ and $(\vec{a}-3\vec{b})$
externally in the ratio 1 : 2. Also, show that P is the mid point of the line segment RQ.\\
	\solution
%		\begin{enumerate}[label=\thesection.\arabic*,ref=\thesection.\theenumi]
\numberwithin{equation}{enumi}
\numberwithin{figure}{enumi}
\numberwithin{table}{enumi}

\item Find the coordinates of the point which divides the join of $(-1,7) \text{ and } (4,-3)$ in the ratio 2:3.
	\\
		\solution
	\input{chapters/10/7/2/1/section.tex}
\item Find the coordinates of the points of trisection of the line segment joining $(4,-1) \text{ and } (-2,3)$.
	\\
		\solution
	\input{chapters/10/7/2/2/section.tex}
\item
	\iffalse
\item To conduct Sports Day activities, in your rectangular shaped school                   
ground ABCD, lines have 
drawn with chalk powder at a                 
distance of 1m each. 100 flower pots have been placed at a distance of 1m 
from each other along AD, as shown 
in Fig. 7.12. Niharika runs $ \frac {1}{4} $th the 
distance AD on the 2nd line and 
posts a green flag. Preet runs $ \frac {1}{5} $th 
the distance AD on the eighth line 
and posts a red flag. What is the 
distance between both the flags? If 
Rashmi has to post a blue flag exactly 
halfway between the line segment 
joining the two flags, where should 
she post her flag?
\begin{figure}[h!]
  \centering
  \includegraphics[width=\columnwidth]{sc.png}
  \caption{}
\label{fig:10/7/12Fig1}
\end{figure}               
\fi
      
\item Find the ratio in which the line segment joining the points $(-3,10) \text{ and } (6,-8)$ $\text{ is divided by } (-1,6)$.
	\\
		\solution
	\input{chapters/10/7/2/4/section.tex}
\item Find the ratio in which the line segment joining $A(1,-5) \text{ and } B(-4,5)$ $\text{is divided by the x-axis}$. Also find the coordinates of the point of division.
\item If $(1,2), (4,y), (x,6), (3,5)$ are the vertices of a parallelogram taken in order, find x and y.
	\\
		\solution
	\input{chapters/10/7/2/6/para1.tex}
\item Find the coordinates of a point A, where AB is the diameter of a circle whose centre is $(2,-3) \text{ and }$ B is $(1,4)$.
	\\
		\solution
	\input{chapters/10/7/2/7/section.tex}
\item If A \text{ and } B are $(-2,-2) \text{ and } (2,-4)$, respectively, find the coordinates of P such that AP= $\frac {3}{7}$AB $\text{ and }$ P lies on the line segment AB.
	\\
		\solution
	\input{chapters/10/7/2/8/section.tex}
\item Find the coordinates of the points which divide the line segment joining $A(-2,2) \text{ and } B(2,8)$ into four equal parts.
	\\
		\solution
	\input{chapters/10/7/2/9/section.tex}
\item Find the area of a rhombus if its vertices are $(3,0), (4,5), (-1,4) \text{ and } (-2,-1)$ taken in order. [$\vec{Hint}$ : Area of rhombus =$\frac {1}{2}$(product of its diagonals)]
	\\
		\solution
	\input{chapters/10/7/2/10/cross.tex}
\item Find the position vector of a point R which divides the line joining two points $\vec{P}$
and $\vec{Q}$ whose position vectors are $\hat{i}+2\hat{j}-\hat{k}$ and $-\hat{i}+\hat{j}+\hat{k}$ respectively, in the
ratio 2 : 1
\begin{enumerate}
    \item  internally
    \item  externally
\end{enumerate}
\solution
		\input{chapters/12/10/2/15/section.tex}
\item Find the position vector of the mid point of the vector joining the points $\vec{P}$(2, 3, 4)
and $\vec{Q}$(4, 1, –2).
\\
\solution
		\input{chapters/12/10/2/16/section.tex}
\item Determine the ratio in which the line $2x+y  - 4=0$ divides the line segment joining the points $\vec{A}(2, - 2)$  and  $\vec{B}(3, 7)$.
\\
\solution
	\input{chapters/10/7/4/1/section.tex}
\item Let $\vec{A}(4, 2), \vec{B}(6, 5)$  and $ \vec{C}(1, 4)$ be the vertices of $\triangle ABC$.
\begin{enumerate}
\item The median from $\vec{A}$ meets $BC$ at $\vec{D}$. Find the coordinates of the point $\vec{D}$.
\item Find the coordinates of the point $\vec{P}$ on $AD$ such that $AP : PD = 2 : 1$.
\item Find the coordinates of points $\vec{Q}$ and $\vec{R}$ on medians $BE$ and $CF$ respectively such that $BQ : QE = 2 : 1$  and  $CR : RF = 2 : 1$.
\item What do you observe?
\item If $\vec{A}, \vec{B}$ and $\vec{C}$  are the vertices of $\triangle ABC$, find the coordinates of the centroid of the triangle.
\end{enumerate}
\solution
	\input{chapters/10/7/4/7/section.tex}
\item Find the slope of a line, which passes through the origin and the mid point of the line segment joining the points $\vec{P}$(0,-4) and $\vec{B}$(8,0).
\label{chapters/11/10/1/5}
\input{chapters/11/10/1/5/matrix.tex}
\item Find the position vector of a point R which divides the line joining two points P and Q whose position vectors are $(2\vec{a}+\vec{b})$ and $(\vec{a}-3\vec{b})$
externally in the ratio 1 : 2. Also, show that P is the mid point of the line segment RQ.\\
	\solution
%		\input{chapters/12/10/5/9/section.tex}

\end{enumerate}



\end{enumerate}


\item
	\iffalse
\item To conduct Sports Day activities, in your rectangular shaped school                   
ground ABCD, lines have 
drawn with chalk powder at a                 
distance of 1m each. 100 flower pots have been placed at a distance of 1m 
from each other along AD, as shown 
in Fig. 7.12. Niharika runs $ \frac {1}{4} $th the 
distance AD on the 2nd line and 
posts a green flag. Preet runs $ \frac {1}{5} $th 
the distance AD on the eighth line 
and posts a red flag. What is the 
distance between both the flags? If 
Rashmi has to post a blue flag exactly 
halfway between the line segment 
joining the two flags, where should 
she post her flag?
\begin{figure}[h!]
  \centering
  \includegraphics[width=\columnwidth]{sc.png}
  \caption{}
\label{fig:10/7/12Fig1}
\end{figure}               
\fi
      
\item Find the ratio in which the line segment joining the points $(-3,10) \text{ and } (6,-8)$ $\text{ is divided by } (-1,6)$.
	\\
		\solution
	\iffalse
\documentclass[12pt]{article}
\usepackage{graphicx}
%\documentclass[journal,12pt,twocolumn]{IEEEtran}
\usepackage[none]{hyphenat}
\usepackage{graphicx}
\usepackage{listings}
\usepackage[english]{babel}
\usepackage{graphicx}
\usepackage{caption} 
\usepackage{hyperref}
\usepackage{booktabs}
\def\inputGnumericTable{}
\usepackage{color}                                            %%
    \usepackage{array}                                            %%
    \usepackage{longtable}                                        %%
    \usepackage{calc}                                             %%
    \usepackage{multirow}                                         %%
    \usepackage{hhline}                                           %%
    \usepackage{ifthen}
\usepackage{array}
\usepackage{amsmath}   % for having text in math mode
\usepackage{listings}
\lstset{
language=tex,
frame=single, 
breaklines=true
}
  
%Following 2 lines were added to remove the blank page at the beginning
\usepackage{atbegshi}% http://ctan.org/pkg/atbegshi
\AtBeginDocument{\AtBeginShipoutNext{\AtBeginShipoutDiscard}}
%
%New macro definitions
\newcommand{\mydet}[1]{\ensuremath{\begin{vmatrix}#1\end{vmatrix}}}
\providecommand{\brak}[1]{\ensuremath{\left(#1\right)}}
\providecommand{\norm}[1]{\left\lVert#1\right\rVert}
\newcommand{\solution}{\noindent \textbf{Solution: }}
\newcommand{\myvec}[1]{\ensuremath{\begin{pmatrix}#1\end{pmatrix}}}
\let\vec\mathbf
\begin{document}
\begin{center}
\title{\textbf{Coordinate Geometry}}
\date{\vspace{-5ex}} %Not to print date automatically
\maketitle
\end{center}
\setcounter{page}{1}
\section*{10$^{th}$ Maths - Chapter 7}
This is Problem-4 from Exercise 7.2
\begin{enumerate}
\item Find the ratio in which the line segement joining the points $\myvec{-3 \\ 10}$ and $\myvec{6\\-8}$ is divided by $\myvec{-1\\6}$.\\
\solution \\
\fi
		The input parameters for this problem are available in Table \eqref{tab:10/7/2/4-1}.
\begin{table}[ht!]
\begin{tabular}{|c|c|p{5cm}|}
\hline
\textbf{Symbol} & \textbf{Value} & \textbf{Description} \\
\hline
$\theta$ & $30\degree$ & $\angle{BAP} = \angle{BAQ}$ \\
\hline
$a$ & $9$ & $AB$ \\
\hline
$c$ & $8$ & $AQ$ \\
\hline
$\vec{e}_1$ & $\myvec{1\\0}$ & Basis vector \\
\hline
\end{tabular}

\caption{}
\label{tab:10/7/2/4-1} 
\end{table}
Using section formula,
\begin{align}
         \vec{R} &=\frac{\vec{Q}+n\vec{P}}{1+n}\label{eq:chapters/10/7/2/4/1}
\end{align}
Substituting the values of $\vec{P},\vec{Q}$ and $\vec{R}$ in \eqref{eq:chapters/10/7/2/4/1}
\begin{align}
         \myvec{-1\\6} &=\frac{{\myvec{-3\\10}+n\myvec{6\\-8}}}{1+n}\\
 &=\frac{1}{1+n}\brak{{\myvec{-3\\10}+n\myvec{6\\-8}}} \\
 &=\frac{1}{1+n}\myvec{-3+6n\\10-8n} \label{eq:chapters/10/7/2/4/4}
\end{align}
Simplifying \eqref{eq:chapters/10/7/2/4/4} yeilds,
\begin{align}
          -1 &=\frac{-3+6n}{1+n}\\
\implies          n &=\frac{2}{7}
\end{align}
Also,
\begin{align}
          6 &=\frac{10-8n}{1+n}\\
    \implies      n &=\frac{2}{7}
\end{align}
Hence the desired ratio is $\dfrac{2}{7}$.  
\begin{figure}[!h]
 \begin{center}
  \includegraphics[width=\columnwidth]{chapters/10/7/2/4/figs/fig.png}
 \end{center}
\caption{}
\label{fig:10/7/2/4Fig1}
\end{figure}

\item Find the ratio in which the line segment joining $A(1,-5) \text{ and } B(-4,5)$ $\text{is divided by the x-axis}$. Also find the coordinates of the point of division.
\item If $(1,2), (4,y), (x,6), (3,5)$ are the vertices of a parallelogram taken in order, find x and y.
	\\
		\solution
	\iffalse
\documentclass[12pt]{article}
\usepackage{graphicx}
%\documentclass[journal,12pt,twocolumn]{IEEEtran}
\def\inputGnumericTable{}
\usepackage{color}                                            %%
    \usepackage{array}                                            %%
    \usepackage{longtable}                                        %%
    \usepackage{calc}                                             %%
    \usepackage{multirow}                                         %%
    \usepackage{hhline}                                           %%
    \usepackage{ifthen}
\usepackage[none]{hyphenat}
\usepackage{graphicx}
\usepackage{listings}
\usepackage[english]{babel}
\usepackage{graphicx}
\usepackage{caption} 
\usepackage{hyperref}
\usepackage{booktabs}
\usepackage{array}
\usepackage{amsmath}   % for having text in math mode
\usepackage{listings}
\lstset{
  frame=single,
  breaklines=true
}
  
%Following 2 lines were added to remove the blank page at the beginning
\usepackage{atbegshi}% http://ctan.org/pkg/atbegshi
\AtBeginDocument{\AtBeginShipoutNext{\AtBeginShipoutDiscard}}
%


%New macro definitions
\newcommand{\mydet}[1]{\ensuremath{\begin{vmatrix}#1\end{vmatrix}}}
\providecommand{\brak}[1]{\ensuremath{\left(#1\right)}}
\providecommand{\norm}[1]{\left\lVert#1\right\rVert}
\newcommand{\solution}{\noindent \textbf{Solution: }}
\newcommand{\myvec}[1]{\ensuremath{\begin{pmatrix}#1\end{pmatrix}}}
\let\vec\mathbf

\begin{document}

\begin{center}
\title{\textbf{Properties of Parallelegram}}
\date{\vspace{-5ex}} %Not to print date automatically
\maketitle
\end{center}

\setcounter{page}{1}

\section{10$^{th}$ Maths - Chapter 7}

This is Problem-6 from Exercise 7.2

\begin{enumerate}
\item If $\vec{A}(1, 2),\vec{B}(4, x),\vec{C}(y, 6) \text{and } \vec{D}(3, 5)$ are the vertices of a parallelogram taken in order,find x and y.
\end{enumerate}
\fi

The input parameters for this problem are available in
\ref{table:chapters/10/7/2/6/tables/}.	
\begin{table}[!ht]
	\centering
	\begin{tabular}{|c|c|p{5cm}|}
\hline
\textbf{Symbol} & \textbf{Value} & \textbf{Description} \\
\hline
$\theta$ & $30\degree$ & $\angle{BAP} = \angle{BAQ}$ \\
\hline
$a$ & $9$ & $AB$ \\
\hline
$c$ & $8$ & $AQ$ \\
\hline
$\vec{e}_1$ & $\myvec{1\\0}$ & Basis vector \\
\hline
\end{tabular}

\caption{}
\label{table:chapters/10/7/2/6/tables/}	
\end{table}
From the given information,
\begin{align}
  \label{eq:chapters/10/7/2/6/tables/det2f}
	\vec{B}-\vec{A} &= \myvec{4 \\y } - \myvec{1 \\2 }  = \myvec{3 \\y-2 }\\
	\vec{C}-\vec{D} &= \myvec{x \\6 } - \myvec{3 \\5 }  = \myvec{x-3 \\1}
\end{align}
Since $ABCD$ is a parallellogram,
\begin{align}
	\myvec{3\\y-2}&=\myvec{x-3\\1}\\
	\implies x&=6 ,y=3
\end{align}
Fig. \ref{fig:chapters/10/7/2/6/Fig3}
provides a verification.
\begin{figure}[h!]
	\begin{center}
  \includegraphics[width=\columnwidth]{chapters/10/7/2/6/figs/para.pdf}
	\end{center}
\caption{}
\label{fig:chapters/10/7/2/6/Fig3}
\end{figure}


\item Find the coordinates of a point A, where AB is the diameter of a circle whose centre is $(2,-3) \text{ and }$ B is $(1,4)$.
	\\
		\solution
	\iffalse
\documentclass[12pt]{article}
\usepackage{graphicx}
\usepackage{amsmath}
\usepackage{mathtools}
\usepackage{gensymb}

\newcommand{\mydet}[1]{\ensuremath{\begin{vmatrix}#1\end{vmatrix}}}
\providecommand{\brak}[1]{\ensuremath{\left(#1\right)}}
\providecommand{\norm}[1]{\left\lVert#1\right\rVert}
\newcommand{\solution}{\noindent \textbf{Solution: }}
\newcommand{\myvec}[1]{\ensuremath{\begin{pmatrix}#1\end{pmatrix}}}
\let\vec\mathbf

\begin{document}
\begin{center}
\section*{CHAPTER 7 - COORDINATE GEOMETRY}

\end{center}
\section*{Excercise 7.2}

Q7.Find the coordinates of point $\vec{A}$, where AB is the diameter of a circle where the center is (2,-3) and $\vec{B}$ is the point (1,4):

\solution
\begin{enumerate}
\item The coordinates $\vec{B}$ and center $\vec{C}$ are given, where:
	\fi
	Let
	\begin{align}
	\vec{B} = \myvec{
		1\\
	    4\\
		},
	\vec{C} = \myvec{
	    2\\
	   -3\\
		}
	\end{align}
	\iffalse
Let us assume the coordinates of $\vec{A}$. Now, $\vec{C}$ is the center which is midpoint of line AB and $\vec{B}$ is one of the coordinate of diameter AB of a circle.
	\fi	
Hence,	
	\begin{align}
	\vec{C} &= \frac{\vec{A+B}}{2} \\
\implies	2\vec{C} &= \vec{A}+\vec{B} \\
		\text{or, }	\vec{A} &= 2\vec{C}-\vec{B} \\
	 &= \myvec{3\\-10\\}	
	\end{align}       
	See Fig. 
\ref{fig:chapters/10/7/2/7Fig}.
\begin{figure}[!h]
\begin{center}	
	\includegraphics[width=\columnwidth]{chapters/10/7/2/7/figs/Vector1.png}
\end{center}
\caption{}
\label{fig:chapters/10/7/2/7Fig}
\end{figure}
	

\item If A \text{ and } B are $(-2,-2) \text{ and } (2,-4)$, respectively, find the coordinates of P such that AP= $\frac {3}{7}$AB $\text{ and }$ P lies on the line segment AB.
	\\
		\solution
	\iffalse
\documentclass[journal,10pt,twocolumn]{article}
\usepackage{graphicx}
\usepackage[none]{hyphenat}
\usepackage{graphicx}
\usepackage{listings}
\usepackage[english]{babel}
\usepackage{graphicx}
\usepackage{caption} 
\usepackage{booktabs}
\usepackage{array}
\usepackage{amssymb} % for \because
\usepackage{amsmath}   % for having text in math mode
\usepackage{extarrows} % for Row operations arrows
\usepackage{listings}
\usepackage[utf8]{inputenc}
\lstset{
  frame=single,
  breaklines=true
}
\usepackage{hyperref}
  
%Following 2 lines were added to remove the blank page at the beginning
\usepackage{atbegshi}% http://ctan.org/pkg/atbegshi
\AtBeginDocument{\AtBeginShipoutNext{\AtBeginShipoutDiscard}}


%New macro definitions
\newcommand{\mydet}[1]{\ensuremath{\begin{vmatrix}#1\end{vmatrix}}}
\providecommand{\brak}[1]{\ensuremath{\left(#1\right)}}
\newcommand{\solution}{\noindent \textbf{Solution: }}
\newcommand{\myvec}[1]{\ensuremath{\begin{pmatrix}#1\end{pmatrix}}}
\providecommand{\norm}[1]{\left\lVert#1\right\rVert}
\providecommand{\abs}[1]{\left\vert#1\right\vert}
\let\vec\mathbf

\begin{document}

\begin{center}
\title{\textbf{VECTORS}}
\date{\vspace{-5ex}} %Not to print date automatically
\maketitle
\end{center}

\section{10$^{th}$ Maths - EXERCISE-7.2}

\begin{enumerate}
\item If A and B are $(– 2, – 2)\text{ and }(2, – 4)$, respectively, find the coordinates of P such that $AP =\frac{3}{7}AB$ and P lies on the line segment AB. 

\section{SOLUTION}
Given points are
\begin{align}
\vec{A}=\myvec{-2\\ -2} ,
\vec{B}=\myvec{2\\ -4}
\end{align}
The equation of the formula is
\fi
Using section formula, 
\begin{align}
\vec{P}&=\frac{\vec{A}+n\vec{B}}{1+n}
\end{align}
where
\begin{align}
	n =\frac{3}{4}
\end{align}
Thus,
\begin{align}
\vec{P}&=\frac{1}{1+\frac{3}{4}}\brak{\myvec{-2\\-2}+\frac{3}{4}\myvec{2\\-4}}\\
&=\myvec{\frac{-2}{7}\\[1pt] \frac{-20}{7}}
\end{align}
See Fig. 
   \ref{fig:chapters/10/7/2/8/vec.png}
\begin{figure}
   \centering 
 \includegraphics[width=\columnwidth]{chapters/10/7/2/8/figs/vec.png}
   \caption{}
   \label{fig:chapters/10/7/2/8/vec.png}
   \end{figure}

\item Find the coordinates of the points which divide the line segment joining $A(-2,2) \text{ and } B(2,8)$ into four equal parts.
	\\
		\solution
	\begin{enumerate}[label=\thesection.\arabic*,ref=\thesection.\theenumi]
\numberwithin{equation}{enumi}
\numberwithin{figure}{enumi}
\numberwithin{table}{enumi}

\item Find the coordinates of the point which divides the join of $(-1,7) \text{ and } (4,-3)$ in the ratio 2:3.
	\\
		\solution
	\iffalse
\documentclass[12pt]{article}
\usepackage{graphicx}
\usepackage{amsmath}
\usepackage{mathtools}
\usepackage{gensymb}

\newcommand{\mydet}[1]{\ensuremath{\begin{vmatrix}#1\end{vmatrix}}}
\providecommand{\brak}[1]{\ensuremath{\left(#1\right)}}
\providecommand{\norm}[1]{\left\lVert#1\right\rVert}
\newcommand{\solution}{\noindent \textbf{Solution: }}
\newcommand{\myvec}[1]{\ensuremath{\begin{pmatrix}#1\end{pmatrix}}}
\let\vec\mathbf

\begin{document}
\begin{center}
\textbf\large{CHAPTER-7 \\ COORDINATE GEOMETRY}
\end{center}
\section*{Excercise 7.2}

1. Find the coordinates of the point which divides the join $\vec(-1,7) \text{ and } \vec(4,-3)$ in the ratio 2:3 :
\\
\\
\solution\\		
\fi
The coordinates and ratio are given as
\begin{align}
\vec{P}=\myvec{-1\\7\\},
\vec{Q}=\myvec{4\\-3\\},
n=\frac{3}{2}
\end{align}
Using section formula
\begin{align}
\vec{R}&=\frac{\vec{Q}+n\vec{P}}{1+n}\\
&=\frac{1}{1+\frac{3}{2}}  \myvec{\myvec{
4\\
-3\\
}
  +
   \frac{3}{2}\myvec{
-1\\
7\\
}}\\
&=\myvec{
1\\
3
}
\end{align}
See Fig. 
\ref{fig:chapters/10/7/2/1/Fig}
\begin{figure}[!h]
\begin{center}
   \includegraphics[width=\columnwidth]{chapters/10/7/2/1/figs/linefig.png}
\end{center}
\caption{}
\label{fig:chapters/10/7/2/1/Fig}
\end{figure}


\item Find the coordinates of the points of trisection of the line segment joining $(4,-1) \text{ and } (-2,3)$.
	\\
		\solution
	\begin{enumerate}[label=\thesection.\arabic*,ref=\thesection.\theenumi]
\numberwithin{equation}{enumi}
\numberwithin{figure}{enumi}
\numberwithin{table}{enumi}

\item Find the coordinates of the point which divides the join of $(-1,7) \text{ and } (4,-3)$ in the ratio 2:3.
	\\
		\solution
	\input{chapters/10/7/2/1/section.tex}
\item Find the coordinates of the points of trisection of the line segment joining $(4,-1) \text{ and } (-2,3)$.
	\\
		\solution
	\input{chapters/10/7/2/2/section.tex}
\item
	\iffalse
\item To conduct Sports Day activities, in your rectangular shaped school                   
ground ABCD, lines have 
drawn with chalk powder at a                 
distance of 1m each. 100 flower pots have been placed at a distance of 1m 
from each other along AD, as shown 
in Fig. 7.12. Niharika runs $ \frac {1}{4} $th the 
distance AD on the 2nd line and 
posts a green flag. Preet runs $ \frac {1}{5} $th 
the distance AD on the eighth line 
and posts a red flag. What is the 
distance between both the flags? If 
Rashmi has to post a blue flag exactly 
halfway between the line segment 
joining the two flags, where should 
she post her flag?
\begin{figure}[h!]
  \centering
  \includegraphics[width=\columnwidth]{sc.png}
  \caption{}
\label{fig:10/7/12Fig1}
\end{figure}               
\fi
      
\item Find the ratio in which the line segment joining the points $(-3,10) \text{ and } (6,-8)$ $\text{ is divided by } (-1,6)$.
	\\
		\solution
	\input{chapters/10/7/2/4/section.tex}
\item Find the ratio in which the line segment joining $A(1,-5) \text{ and } B(-4,5)$ $\text{is divided by the x-axis}$. Also find the coordinates of the point of division.
\item If $(1,2), (4,y), (x,6), (3,5)$ are the vertices of a parallelogram taken in order, find x and y.
	\\
		\solution
	\input{chapters/10/7/2/6/para1.tex}
\item Find the coordinates of a point A, where AB is the diameter of a circle whose centre is $(2,-3) \text{ and }$ B is $(1,4)$.
	\\
		\solution
	\input{chapters/10/7/2/7/section.tex}
\item If A \text{ and } B are $(-2,-2) \text{ and } (2,-4)$, respectively, find the coordinates of P such that AP= $\frac {3}{7}$AB $\text{ and }$ P lies on the line segment AB.
	\\
		\solution
	\input{chapters/10/7/2/8/section.tex}
\item Find the coordinates of the points which divide the line segment joining $A(-2,2) \text{ and } B(2,8)$ into four equal parts.
	\\
		\solution
	\input{chapters/10/7/2/9/section.tex}
\item Find the area of a rhombus if its vertices are $(3,0), (4,5), (-1,4) \text{ and } (-2,-1)$ taken in order. [$\vec{Hint}$ : Area of rhombus =$\frac {1}{2}$(product of its diagonals)]
	\\
		\solution
	\input{chapters/10/7/2/10/cross.tex}
\item Find the position vector of a point R which divides the line joining two points $\vec{P}$
and $\vec{Q}$ whose position vectors are $\hat{i}+2\hat{j}-\hat{k}$ and $-\hat{i}+\hat{j}+\hat{k}$ respectively, in the
ratio 2 : 1
\begin{enumerate}
    \item  internally
    \item  externally
\end{enumerate}
\solution
		\input{chapters/12/10/2/15/section.tex}
\item Find the position vector of the mid point of the vector joining the points $\vec{P}$(2, 3, 4)
and $\vec{Q}$(4, 1, –2).
\\
\solution
		\input{chapters/12/10/2/16/section.tex}
\item Determine the ratio in which the line $2x+y  - 4=0$ divides the line segment joining the points $\vec{A}(2, - 2)$  and  $\vec{B}(3, 7)$.
\\
\solution
	\input{chapters/10/7/4/1/section.tex}
\item Let $\vec{A}(4, 2), \vec{B}(6, 5)$  and $ \vec{C}(1, 4)$ be the vertices of $\triangle ABC$.
\begin{enumerate}
\item The median from $\vec{A}$ meets $BC$ at $\vec{D}$. Find the coordinates of the point $\vec{D}$.
\item Find the coordinates of the point $\vec{P}$ on $AD$ such that $AP : PD = 2 : 1$.
\item Find the coordinates of points $\vec{Q}$ and $\vec{R}$ on medians $BE$ and $CF$ respectively such that $BQ : QE = 2 : 1$  and  $CR : RF = 2 : 1$.
\item What do you observe?
\item If $\vec{A}, \vec{B}$ and $\vec{C}$  are the vertices of $\triangle ABC$, find the coordinates of the centroid of the triangle.
\end{enumerate}
\solution
	\input{chapters/10/7/4/7/section.tex}
\item Find the slope of a line, which passes through the origin and the mid point of the line segment joining the points $\vec{P}$(0,-4) and $\vec{B}$(8,0).
\label{chapters/11/10/1/5}
\input{chapters/11/10/1/5/matrix.tex}
\item Find the position vector of a point R which divides the line joining two points P and Q whose position vectors are $(2\vec{a}+\vec{b})$ and $(\vec{a}-3\vec{b})$
externally in the ratio 1 : 2. Also, show that P is the mid point of the line segment RQ.\\
	\solution
%		\input{chapters/12/10/5/9/section.tex}

\end{enumerate}


\item
	\iffalse
\item To conduct Sports Day activities, in your rectangular shaped school                   
ground ABCD, lines have 
drawn with chalk powder at a                 
distance of 1m each. 100 flower pots have been placed at a distance of 1m 
from each other along AD, as shown 
in Fig. 7.12. Niharika runs $ \frac {1}{4} $th the 
distance AD on the 2nd line and 
posts a green flag. Preet runs $ \frac {1}{5} $th 
the distance AD on the eighth line 
and posts a red flag. What is the 
distance between both the flags? If 
Rashmi has to post a blue flag exactly 
halfway between the line segment 
joining the two flags, where should 
she post her flag?
\begin{figure}[h!]
  \centering
  \includegraphics[width=\columnwidth]{sc.png}
  \caption{}
\label{fig:10/7/12Fig1}
\end{figure}               
\fi
      
\item Find the ratio in which the line segment joining the points $(-3,10) \text{ and } (6,-8)$ $\text{ is divided by } (-1,6)$.
	\\
		\solution
	\iffalse
\documentclass[12pt]{article}
\usepackage{graphicx}
%\documentclass[journal,12pt,twocolumn]{IEEEtran}
\usepackage[none]{hyphenat}
\usepackage{graphicx}
\usepackage{listings}
\usepackage[english]{babel}
\usepackage{graphicx}
\usepackage{caption} 
\usepackage{hyperref}
\usepackage{booktabs}
\def\inputGnumericTable{}
\usepackage{color}                                            %%
    \usepackage{array}                                            %%
    \usepackage{longtable}                                        %%
    \usepackage{calc}                                             %%
    \usepackage{multirow}                                         %%
    \usepackage{hhline}                                           %%
    \usepackage{ifthen}
\usepackage{array}
\usepackage{amsmath}   % for having text in math mode
\usepackage{listings}
\lstset{
language=tex,
frame=single, 
breaklines=true
}
  
%Following 2 lines were added to remove the blank page at the beginning
\usepackage{atbegshi}% http://ctan.org/pkg/atbegshi
\AtBeginDocument{\AtBeginShipoutNext{\AtBeginShipoutDiscard}}
%
%New macro definitions
\newcommand{\mydet}[1]{\ensuremath{\begin{vmatrix}#1\end{vmatrix}}}
\providecommand{\brak}[1]{\ensuremath{\left(#1\right)}}
\providecommand{\norm}[1]{\left\lVert#1\right\rVert}
\newcommand{\solution}{\noindent \textbf{Solution: }}
\newcommand{\myvec}[1]{\ensuremath{\begin{pmatrix}#1\end{pmatrix}}}
\let\vec\mathbf
\begin{document}
\begin{center}
\title{\textbf{Coordinate Geometry}}
\date{\vspace{-5ex}} %Not to print date automatically
\maketitle
\end{center}
\setcounter{page}{1}
\section*{10$^{th}$ Maths - Chapter 7}
This is Problem-4 from Exercise 7.2
\begin{enumerate}
\item Find the ratio in which the line segement joining the points $\myvec{-3 \\ 10}$ and $\myvec{6\\-8}$ is divided by $\myvec{-1\\6}$.\\
\solution \\
\fi
		The input parameters for this problem are available in Table \eqref{tab:10/7/2/4-1}.
\begin{table}[ht!]
\input{chapters/10/7/2/4/tables/table.tex}
\caption{}
\label{tab:10/7/2/4-1} 
\end{table}
Using section formula,
\begin{align}
         \vec{R} &=\frac{\vec{Q}+n\vec{P}}{1+n}\label{eq:chapters/10/7/2/4/1}
\end{align}
Substituting the values of $\vec{P},\vec{Q}$ and $\vec{R}$ in \eqref{eq:chapters/10/7/2/4/1}
\begin{align}
         \myvec{-1\\6} &=\frac{{\myvec{-3\\10}+n\myvec{6\\-8}}}{1+n}\\
 &=\frac{1}{1+n}\brak{{\myvec{-3\\10}+n\myvec{6\\-8}}} \\
 &=\frac{1}{1+n}\myvec{-3+6n\\10-8n} \label{eq:chapters/10/7/2/4/4}
\end{align}
Simplifying \eqref{eq:chapters/10/7/2/4/4} yeilds,
\begin{align}
          -1 &=\frac{-3+6n}{1+n}\\
\implies          n &=\frac{2}{7}
\end{align}
Also,
\begin{align}
          6 &=\frac{10-8n}{1+n}\\
    \implies      n &=\frac{2}{7}
\end{align}
Hence the desired ratio is $\dfrac{2}{7}$.  
\begin{figure}[!h]
 \begin{center}
  \includegraphics[width=\columnwidth]{chapters/10/7/2/4/figs/fig.png}
 \end{center}
\caption{}
\label{fig:10/7/2/4Fig1}
\end{figure}

\item Find the ratio in which the line segment joining $A(1,-5) \text{ and } B(-4,5)$ $\text{is divided by the x-axis}$. Also find the coordinates of the point of division.
\item If $(1,2), (4,y), (x,6), (3,5)$ are the vertices of a parallelogram taken in order, find x and y.
	\\
		\solution
	\iffalse
\documentclass[12pt]{article}
\usepackage{graphicx}
%\documentclass[journal,12pt,twocolumn]{IEEEtran}
\def\inputGnumericTable{}
\usepackage{color}                                            %%
    \usepackage{array}                                            %%
    \usepackage{longtable}                                        %%
    \usepackage{calc}                                             %%
    \usepackage{multirow}                                         %%
    \usepackage{hhline}                                           %%
    \usepackage{ifthen}
\usepackage[none]{hyphenat}
\usepackage{graphicx}
\usepackage{listings}
\usepackage[english]{babel}
\usepackage{graphicx}
\usepackage{caption} 
\usepackage{hyperref}
\usepackage{booktabs}
\usepackage{array}
\usepackage{amsmath}   % for having text in math mode
\usepackage{listings}
\lstset{
  frame=single,
  breaklines=true
}
  
%Following 2 lines were added to remove the blank page at the beginning
\usepackage{atbegshi}% http://ctan.org/pkg/atbegshi
\AtBeginDocument{\AtBeginShipoutNext{\AtBeginShipoutDiscard}}
%


%New macro definitions
\newcommand{\mydet}[1]{\ensuremath{\begin{vmatrix}#1\end{vmatrix}}}
\providecommand{\brak}[1]{\ensuremath{\left(#1\right)}}
\providecommand{\norm}[1]{\left\lVert#1\right\rVert}
\newcommand{\solution}{\noindent \textbf{Solution: }}
\newcommand{\myvec}[1]{\ensuremath{\begin{pmatrix}#1\end{pmatrix}}}
\let\vec\mathbf

\begin{document}

\begin{center}
\title{\textbf{Properties of Parallelegram}}
\date{\vspace{-5ex}} %Not to print date automatically
\maketitle
\end{center}

\setcounter{page}{1}

\section{10$^{th}$ Maths - Chapter 7}

This is Problem-6 from Exercise 7.2

\begin{enumerate}
\item If $\vec{A}(1, 2),\vec{B}(4, x),\vec{C}(y, 6) \text{and } \vec{D}(3, 5)$ are the vertices of a parallelogram taken in order,find x and y.
\end{enumerate}
\fi

The input parameters for this problem are available in
\ref{table:chapters/10/7/2/6/tables/}.	
\begin{table}[!ht]
	\centering
	\input{chapters/10/7/2/6/tables/table.tex}
\caption{}
\label{table:chapters/10/7/2/6/tables/}	
\end{table}
From the given information,
\begin{align}
  \label{eq:chapters/10/7/2/6/tables/det2f}
	\vec{B}-\vec{A} &= \myvec{4 \\y } - \myvec{1 \\2 }  = \myvec{3 \\y-2 }\\
	\vec{C}-\vec{D} &= \myvec{x \\6 } - \myvec{3 \\5 }  = \myvec{x-3 \\1}
\end{align}
Since $ABCD$ is a parallellogram,
\begin{align}
	\myvec{3\\y-2}&=\myvec{x-3\\1}\\
	\implies x&=6 ,y=3
\end{align}
Fig. \ref{fig:chapters/10/7/2/6/Fig3}
provides a verification.
\begin{figure}[h!]
	\begin{center}
  \includegraphics[width=\columnwidth]{chapters/10/7/2/6/figs/para.pdf}
	\end{center}
\caption{}
\label{fig:chapters/10/7/2/6/Fig3}
\end{figure}


\item Find the coordinates of a point A, where AB is the diameter of a circle whose centre is $(2,-3) \text{ and }$ B is $(1,4)$.
	\\
		\solution
	\iffalse
\documentclass[12pt]{article}
\usepackage{graphicx}
\usepackage{amsmath}
\usepackage{mathtools}
\usepackage{gensymb}

\newcommand{\mydet}[1]{\ensuremath{\begin{vmatrix}#1\end{vmatrix}}}
\providecommand{\brak}[1]{\ensuremath{\left(#1\right)}}
\providecommand{\norm}[1]{\left\lVert#1\right\rVert}
\newcommand{\solution}{\noindent \textbf{Solution: }}
\newcommand{\myvec}[1]{\ensuremath{\begin{pmatrix}#1\end{pmatrix}}}
\let\vec\mathbf

\begin{document}
\begin{center}
\section*{CHAPTER 7 - COORDINATE GEOMETRY}

\end{center}
\section*{Excercise 7.2}

Q7.Find the coordinates of point $\vec{A}$, where AB is the diameter of a circle where the center is (2,-3) and $\vec{B}$ is the point (1,4):

\solution
\begin{enumerate}
\item The coordinates $\vec{B}$ and center $\vec{C}$ are given, where:
	\fi
	Let
	\begin{align}
	\vec{B} = \myvec{
		1\\
	    4\\
		},
	\vec{C} = \myvec{
	    2\\
	   -3\\
		}
	\end{align}
	\iffalse
Let us assume the coordinates of $\vec{A}$. Now, $\vec{C}$ is the center which is midpoint of line AB and $\vec{B}$ is one of the coordinate of diameter AB of a circle.
	\fi	
Hence,	
	\begin{align}
	\vec{C} &= \frac{\vec{A+B}}{2} \\
\implies	2\vec{C} &= \vec{A}+\vec{B} \\
		\text{or, }	\vec{A} &= 2\vec{C}-\vec{B} \\
	 &= \myvec{3\\-10\\}	
	\end{align}       
	See Fig. 
\ref{fig:chapters/10/7/2/7Fig}.
\begin{figure}[!h]
\begin{center}	
	\includegraphics[width=\columnwidth]{chapters/10/7/2/7/figs/Vector1.png}
\end{center}
\caption{}
\label{fig:chapters/10/7/2/7Fig}
\end{figure}
	

\item If A \text{ and } B are $(-2,-2) \text{ and } (2,-4)$, respectively, find the coordinates of P such that AP= $\frac {3}{7}$AB $\text{ and }$ P lies on the line segment AB.
	\\
		\solution
	\iffalse
\documentclass[journal,10pt,twocolumn]{article}
\usepackage{graphicx}
\usepackage[none]{hyphenat}
\usepackage{graphicx}
\usepackage{listings}
\usepackage[english]{babel}
\usepackage{graphicx}
\usepackage{caption} 
\usepackage{booktabs}
\usepackage{array}
\usepackage{amssymb} % for \because
\usepackage{amsmath}   % for having text in math mode
\usepackage{extarrows} % for Row operations arrows
\usepackage{listings}
\usepackage[utf8]{inputenc}
\lstset{
  frame=single,
  breaklines=true
}
\usepackage{hyperref}
  
%Following 2 lines were added to remove the blank page at the beginning
\usepackage{atbegshi}% http://ctan.org/pkg/atbegshi
\AtBeginDocument{\AtBeginShipoutNext{\AtBeginShipoutDiscard}}


%New macro definitions
\newcommand{\mydet}[1]{\ensuremath{\begin{vmatrix}#1\end{vmatrix}}}
\providecommand{\brak}[1]{\ensuremath{\left(#1\right)}}
\newcommand{\solution}{\noindent \textbf{Solution: }}
\newcommand{\myvec}[1]{\ensuremath{\begin{pmatrix}#1\end{pmatrix}}}
\providecommand{\norm}[1]{\left\lVert#1\right\rVert}
\providecommand{\abs}[1]{\left\vert#1\right\vert}
\let\vec\mathbf

\begin{document}

\begin{center}
\title{\textbf{VECTORS}}
\date{\vspace{-5ex}} %Not to print date automatically
\maketitle
\end{center}

\section{10$^{th}$ Maths - EXERCISE-7.2}

\begin{enumerate}
\item If A and B are $(– 2, – 2)\text{ and }(2, – 4)$, respectively, find the coordinates of P such that $AP =\frac{3}{7}AB$ and P lies on the line segment AB. 

\section{SOLUTION}
Given points are
\begin{align}
\vec{A}=\myvec{-2\\ -2} ,
\vec{B}=\myvec{2\\ -4}
\end{align}
The equation of the formula is
\fi
Using section formula, 
\begin{align}
\vec{P}&=\frac{\vec{A}+n\vec{B}}{1+n}
\end{align}
where
\begin{align}
	n =\frac{3}{4}
\end{align}
Thus,
\begin{align}
\vec{P}&=\frac{1}{1+\frac{3}{4}}\brak{\myvec{-2\\-2}+\frac{3}{4}\myvec{2\\-4}}\\
&=\myvec{\frac{-2}{7}\\[1pt] \frac{-20}{7}}
\end{align}
See Fig. 
   \ref{fig:chapters/10/7/2/8/vec.png}
\begin{figure}
   \centering 
 \includegraphics[width=\columnwidth]{chapters/10/7/2/8/figs/vec.png}
   \caption{}
   \label{fig:chapters/10/7/2/8/vec.png}
   \end{figure}

\item Find the coordinates of the points which divide the line segment joining $A(-2,2) \text{ and } B(2,8)$ into four equal parts.
	\\
		\solution
	\begin{enumerate}[label=\thesection.\arabic*,ref=\thesection.\theenumi]
\numberwithin{equation}{enumi}
\numberwithin{figure}{enumi}
\numberwithin{table}{enumi}

\item Find the coordinates of the point which divides the join of $(-1,7) \text{ and } (4,-3)$ in the ratio 2:3.
	\\
		\solution
	\input{chapters/10/7/2/1/section.tex}
\item Find the coordinates of the points of trisection of the line segment joining $(4,-1) \text{ and } (-2,3)$.
	\\
		\solution
	\input{chapters/10/7/2/2/section.tex}
\item
	\iffalse
\item To conduct Sports Day activities, in your rectangular shaped school                   
ground ABCD, lines have 
drawn with chalk powder at a                 
distance of 1m each. 100 flower pots have been placed at a distance of 1m 
from each other along AD, as shown 
in Fig. 7.12. Niharika runs $ \frac {1}{4} $th the 
distance AD on the 2nd line and 
posts a green flag. Preet runs $ \frac {1}{5} $th 
the distance AD on the eighth line 
and posts a red flag. What is the 
distance between both the flags? If 
Rashmi has to post a blue flag exactly 
halfway between the line segment 
joining the two flags, where should 
she post her flag?
\begin{figure}[h!]
  \centering
  \includegraphics[width=\columnwidth]{sc.png}
  \caption{}
\label{fig:10/7/12Fig1}
\end{figure}               
\fi
      
\item Find the ratio in which the line segment joining the points $(-3,10) \text{ and } (6,-8)$ $\text{ is divided by } (-1,6)$.
	\\
		\solution
	\input{chapters/10/7/2/4/section.tex}
\item Find the ratio in which the line segment joining $A(1,-5) \text{ and } B(-4,5)$ $\text{is divided by the x-axis}$. Also find the coordinates of the point of division.
\item If $(1,2), (4,y), (x,6), (3,5)$ are the vertices of a parallelogram taken in order, find x and y.
	\\
		\solution
	\input{chapters/10/7/2/6/para1.tex}
\item Find the coordinates of a point A, where AB is the diameter of a circle whose centre is $(2,-3) \text{ and }$ B is $(1,4)$.
	\\
		\solution
	\input{chapters/10/7/2/7/section.tex}
\item If A \text{ and } B are $(-2,-2) \text{ and } (2,-4)$, respectively, find the coordinates of P such that AP= $\frac {3}{7}$AB $\text{ and }$ P lies on the line segment AB.
	\\
		\solution
	\input{chapters/10/7/2/8/section.tex}
\item Find the coordinates of the points which divide the line segment joining $A(-2,2) \text{ and } B(2,8)$ into four equal parts.
	\\
		\solution
	\input{chapters/10/7/2/9/section.tex}
\item Find the area of a rhombus if its vertices are $(3,0), (4,5), (-1,4) \text{ and } (-2,-1)$ taken in order. [$\vec{Hint}$ : Area of rhombus =$\frac {1}{2}$(product of its diagonals)]
	\\
		\solution
	\input{chapters/10/7/2/10/cross.tex}
\item Find the position vector of a point R which divides the line joining two points $\vec{P}$
and $\vec{Q}$ whose position vectors are $\hat{i}+2\hat{j}-\hat{k}$ and $-\hat{i}+\hat{j}+\hat{k}$ respectively, in the
ratio 2 : 1
\begin{enumerate}
    \item  internally
    \item  externally
\end{enumerate}
\solution
		\input{chapters/12/10/2/15/section.tex}
\item Find the position vector of the mid point of the vector joining the points $\vec{P}$(2, 3, 4)
and $\vec{Q}$(4, 1, –2).
\\
\solution
		\input{chapters/12/10/2/16/section.tex}
\item Determine the ratio in which the line $2x+y  - 4=0$ divides the line segment joining the points $\vec{A}(2, - 2)$  and  $\vec{B}(3, 7)$.
\\
\solution
	\input{chapters/10/7/4/1/section.tex}
\item Let $\vec{A}(4, 2), \vec{B}(6, 5)$  and $ \vec{C}(1, 4)$ be the vertices of $\triangle ABC$.
\begin{enumerate}
\item The median from $\vec{A}$ meets $BC$ at $\vec{D}$. Find the coordinates of the point $\vec{D}$.
\item Find the coordinates of the point $\vec{P}$ on $AD$ such that $AP : PD = 2 : 1$.
\item Find the coordinates of points $\vec{Q}$ and $\vec{R}$ on medians $BE$ and $CF$ respectively such that $BQ : QE = 2 : 1$  and  $CR : RF = 2 : 1$.
\item What do you observe?
\item If $\vec{A}, \vec{B}$ and $\vec{C}$  are the vertices of $\triangle ABC$, find the coordinates of the centroid of the triangle.
\end{enumerate}
\solution
	\input{chapters/10/7/4/7/section.tex}
\item Find the slope of a line, which passes through the origin and the mid point of the line segment joining the points $\vec{P}$(0,-4) and $\vec{B}$(8,0).
\label{chapters/11/10/1/5}
\input{chapters/11/10/1/5/matrix.tex}
\item Find the position vector of a point R which divides the line joining two points P and Q whose position vectors are $(2\vec{a}+\vec{b})$ and $(\vec{a}-3\vec{b})$
externally in the ratio 1 : 2. Also, show that P is the mid point of the line segment RQ.\\
	\solution
%		\input{chapters/12/10/5/9/section.tex}

\end{enumerate}


\item Find the area of a rhombus if its vertices are $(3,0), (4,5), (-1,4) \text{ and } (-2,-1)$ taken in order. [$\vec{Hint}$ : Area of rhombus =$\frac {1}{2}$(product of its diagonals)]
	\\
		\solution
	\iffalse
\documentclass[12pt]{article}
\usepackage{graphicx}
%\documentclass[journal,12pt,twocolumn]{IEEEtran}
\usepackage[none]{hyphenat}
\usepackage{graphicx}
\usepackage{listings}
\usepackage[english]{babel}
\usepackage{graphicx}
\usepackage{caption} 
\usepackage{hyperref}
\usepackage{booktabs}
\def\inputGnumericTable{}
\usepackage{color}                                            %%
    \usepackage{array}                                            %%
    \usepackage{longtable}                                        %%
    \usepackage{calc}                                             %%
    \usepackage{multirow}                                         %%
    \usepackage{hhline}                                           %%
    \usepackage{ifthen}
\usepackage{array}
\usepackage{amsmath}   % for having text in math mode
\usepackage{listings}
\lstset{
language=tex,
frame=single, 
breaklines=true
}
  
%Following 2 lines were added to remove the blank page at the beginning
\usepackage{atbegshi}% http://ctan.org/pkg/atbegshi
\AtBeginDocument{\AtBeginShipoutNext{\AtBeginShipoutDiscard}}
%


%New macro definitions
\newcommand{\mydet}[1]{\ensuremath{\begin{vmatrix}#1\end{vmatrix}}}
\providecommand{\brak}[1]{\ensuremath{\left(#1\right)}}
\providecommand{\norm}[1]{\left\lVert#1\right\rVert}
\newcommand{\solution}{\noindent \textbf{Solution: }}
\newcommand{\myvec}[1]{\ensuremath{\begin{pmatrix}#1\end{pmatrix}}}
\let\vec\mathbf

\begin{document}

\begin{center}
\title{\textbf{Coordinate Geometry}}
\date{\vspace{-5ex}} %Not to print date automatically
\maketitle
\end{center}

\setcounter{page}{1}



\begin{enumerate}

\item\textbf{Problem statement :} Find the area of a rhombus of its vertices are $\myvec{3 ,0}$, $\myvec{4 ,5}$, $\myvec{-1 ,4}$ and $\myvec{-2 ,-1}$taken in order

\solution \\
\fi
The input vertices for this problem are given as
	\begin{align}
	\vec{A} = \myvec{
		3\\
		0
		},
	\vec{B} = \myvec{
		4\\
		5
		},
        \vec{C} = \myvec{
		-1\\
		4
		},
        \vec{D} = \myvec{
		-2\\
		-1
		}
	\end{align}
Since		
\begin{align}
 \vec{A-D}= \myvec{3 \\ 0} - \myvec{-2 \\-1}= \myvec{5\\1}
 \\
  \vec{B-A}= \myvec{4 \\ 5} - \myvec{3 \\0}= \myvec{1\\5}
\end{align}
the area of the rhombus is
\begin{align}
                \norm{\myvec{\vec{A-D}}\times \myvec{\vec{B-A}}}=\mydet{5 & 1\\1 & 5} = 24
\end{align}
See Fig. 
\ref{fig:chapters/10/7/2/10/gFig1}.
\begin{figure}[!h]
 \begin{center}
  \includegraphics[width=\columnwidth]{chapters/10/7/2/10/figs/fig.pdf}
 \end{center}
\caption{}
\label{fig:chapters/10/7/2/10/gFig1}
\end{figure}

\item Find the position vector of a point R which divides the line joining two points $\vec{P}$
and $\vec{Q}$ whose position vectors are $\hat{i}+2\hat{j}-\hat{k}$ and $-\hat{i}+\hat{j}+\hat{k}$ respectively, in the
ratio 2 : 1
\begin{enumerate}
    \item  internally
    \item  externally
\end{enumerate}
\solution
		\begin{enumerate}[label=\thesection.\arabic*,ref=\thesection.\theenumi]
\numberwithin{equation}{enumi}
\numberwithin{figure}{enumi}
\numberwithin{table}{enumi}

\item Find the coordinates of the point which divides the join of $(-1,7) \text{ and } (4,-3)$ in the ratio 2:3.
	\\
		\solution
	\input{chapters/10/7/2/1/section.tex}
\item Find the coordinates of the points of trisection of the line segment joining $(4,-1) \text{ and } (-2,3)$.
	\\
		\solution
	\input{chapters/10/7/2/2/section.tex}
\item
	\iffalse
\item To conduct Sports Day activities, in your rectangular shaped school                   
ground ABCD, lines have 
drawn with chalk powder at a                 
distance of 1m each. 100 flower pots have been placed at a distance of 1m 
from each other along AD, as shown 
in Fig. 7.12. Niharika runs $ \frac {1}{4} $th the 
distance AD on the 2nd line and 
posts a green flag. Preet runs $ \frac {1}{5} $th 
the distance AD on the eighth line 
and posts a red flag. What is the 
distance between both the flags? If 
Rashmi has to post a blue flag exactly 
halfway between the line segment 
joining the two flags, where should 
she post her flag?
\begin{figure}[h!]
  \centering
  \includegraphics[width=\columnwidth]{sc.png}
  \caption{}
\label{fig:10/7/12Fig1}
\end{figure}               
\fi
      
\item Find the ratio in which the line segment joining the points $(-3,10) \text{ and } (6,-8)$ $\text{ is divided by } (-1,6)$.
	\\
		\solution
	\input{chapters/10/7/2/4/section.tex}
\item Find the ratio in which the line segment joining $A(1,-5) \text{ and } B(-4,5)$ $\text{is divided by the x-axis}$. Also find the coordinates of the point of division.
\item If $(1,2), (4,y), (x,6), (3,5)$ are the vertices of a parallelogram taken in order, find x and y.
	\\
		\solution
	\input{chapters/10/7/2/6/para1.tex}
\item Find the coordinates of a point A, where AB is the diameter of a circle whose centre is $(2,-3) \text{ and }$ B is $(1,4)$.
	\\
		\solution
	\input{chapters/10/7/2/7/section.tex}
\item If A \text{ and } B are $(-2,-2) \text{ and } (2,-4)$, respectively, find the coordinates of P such that AP= $\frac {3}{7}$AB $\text{ and }$ P lies on the line segment AB.
	\\
		\solution
	\input{chapters/10/7/2/8/section.tex}
\item Find the coordinates of the points which divide the line segment joining $A(-2,2) \text{ and } B(2,8)$ into four equal parts.
	\\
		\solution
	\input{chapters/10/7/2/9/section.tex}
\item Find the area of a rhombus if its vertices are $(3,0), (4,5), (-1,4) \text{ and } (-2,-1)$ taken in order. [$\vec{Hint}$ : Area of rhombus =$\frac {1}{2}$(product of its diagonals)]
	\\
		\solution
	\input{chapters/10/7/2/10/cross.tex}
\item Find the position vector of a point R which divides the line joining two points $\vec{P}$
and $\vec{Q}$ whose position vectors are $\hat{i}+2\hat{j}-\hat{k}$ and $-\hat{i}+\hat{j}+\hat{k}$ respectively, in the
ratio 2 : 1
\begin{enumerate}
    \item  internally
    \item  externally
\end{enumerate}
\solution
		\input{chapters/12/10/2/15/section.tex}
\item Find the position vector of the mid point of the vector joining the points $\vec{P}$(2, 3, 4)
and $\vec{Q}$(4, 1, –2).
\\
\solution
		\input{chapters/12/10/2/16/section.tex}
\item Determine the ratio in which the line $2x+y  - 4=0$ divides the line segment joining the points $\vec{A}(2, - 2)$  and  $\vec{B}(3, 7)$.
\\
\solution
	\input{chapters/10/7/4/1/section.tex}
\item Let $\vec{A}(4, 2), \vec{B}(6, 5)$  and $ \vec{C}(1, 4)$ be the vertices of $\triangle ABC$.
\begin{enumerate}
\item The median from $\vec{A}$ meets $BC$ at $\vec{D}$. Find the coordinates of the point $\vec{D}$.
\item Find the coordinates of the point $\vec{P}$ on $AD$ such that $AP : PD = 2 : 1$.
\item Find the coordinates of points $\vec{Q}$ and $\vec{R}$ on medians $BE$ and $CF$ respectively such that $BQ : QE = 2 : 1$  and  $CR : RF = 2 : 1$.
\item What do you observe?
\item If $\vec{A}, \vec{B}$ and $\vec{C}$  are the vertices of $\triangle ABC$, find the coordinates of the centroid of the triangle.
\end{enumerate}
\solution
	\input{chapters/10/7/4/7/section.tex}
\item Find the slope of a line, which passes through the origin and the mid point of the line segment joining the points $\vec{P}$(0,-4) and $\vec{B}$(8,0).
\label{chapters/11/10/1/5}
\input{chapters/11/10/1/5/matrix.tex}
\item Find the position vector of a point R which divides the line joining two points P and Q whose position vectors are $(2\vec{a}+\vec{b})$ and $(\vec{a}-3\vec{b})$
externally in the ratio 1 : 2. Also, show that P is the mid point of the line segment RQ.\\
	\solution
%		\input{chapters/12/10/5/9/section.tex}

\end{enumerate}


\item Find the position vector of the mid point of the vector joining the points $\vec{P}$(2, 3, 4)
and $\vec{Q}$(4, 1, –2).
\\
\solution
		\begin{enumerate}[label=\thesection.\arabic*,ref=\thesection.\theenumi]
\numberwithin{equation}{enumi}
\numberwithin{figure}{enumi}
\numberwithin{table}{enumi}

\item Find the coordinates of the point which divides the join of $(-1,7) \text{ and } (4,-3)$ in the ratio 2:3.
	\\
		\solution
	\input{chapters/10/7/2/1/section.tex}
\item Find the coordinates of the points of trisection of the line segment joining $(4,-1) \text{ and } (-2,3)$.
	\\
		\solution
	\input{chapters/10/7/2/2/section.tex}
\item
	\iffalse
\item To conduct Sports Day activities, in your rectangular shaped school                   
ground ABCD, lines have 
drawn with chalk powder at a                 
distance of 1m each. 100 flower pots have been placed at a distance of 1m 
from each other along AD, as shown 
in Fig. 7.12. Niharika runs $ \frac {1}{4} $th the 
distance AD on the 2nd line and 
posts a green flag. Preet runs $ \frac {1}{5} $th 
the distance AD on the eighth line 
and posts a red flag. What is the 
distance between both the flags? If 
Rashmi has to post a blue flag exactly 
halfway between the line segment 
joining the two flags, where should 
she post her flag?
\begin{figure}[h!]
  \centering
  \includegraphics[width=\columnwidth]{sc.png}
  \caption{}
\label{fig:10/7/12Fig1}
\end{figure}               
\fi
      
\item Find the ratio in which the line segment joining the points $(-3,10) \text{ and } (6,-8)$ $\text{ is divided by } (-1,6)$.
	\\
		\solution
	\input{chapters/10/7/2/4/section.tex}
\item Find the ratio in which the line segment joining $A(1,-5) \text{ and } B(-4,5)$ $\text{is divided by the x-axis}$. Also find the coordinates of the point of division.
\item If $(1,2), (4,y), (x,6), (3,5)$ are the vertices of a parallelogram taken in order, find x and y.
	\\
		\solution
	\input{chapters/10/7/2/6/para1.tex}
\item Find the coordinates of a point A, where AB is the diameter of a circle whose centre is $(2,-3) \text{ and }$ B is $(1,4)$.
	\\
		\solution
	\input{chapters/10/7/2/7/section.tex}
\item If A \text{ and } B are $(-2,-2) \text{ and } (2,-4)$, respectively, find the coordinates of P such that AP= $\frac {3}{7}$AB $\text{ and }$ P lies on the line segment AB.
	\\
		\solution
	\input{chapters/10/7/2/8/section.tex}
\item Find the coordinates of the points which divide the line segment joining $A(-2,2) \text{ and } B(2,8)$ into four equal parts.
	\\
		\solution
	\input{chapters/10/7/2/9/section.tex}
\item Find the area of a rhombus if its vertices are $(3,0), (4,5), (-1,4) \text{ and } (-2,-1)$ taken in order. [$\vec{Hint}$ : Area of rhombus =$\frac {1}{2}$(product of its diagonals)]
	\\
		\solution
	\input{chapters/10/7/2/10/cross.tex}
\item Find the position vector of a point R which divides the line joining two points $\vec{P}$
and $\vec{Q}$ whose position vectors are $\hat{i}+2\hat{j}-\hat{k}$ and $-\hat{i}+\hat{j}+\hat{k}$ respectively, in the
ratio 2 : 1
\begin{enumerate}
    \item  internally
    \item  externally
\end{enumerate}
\solution
		\input{chapters/12/10/2/15/section.tex}
\item Find the position vector of the mid point of the vector joining the points $\vec{P}$(2, 3, 4)
and $\vec{Q}$(4, 1, –2).
\\
\solution
		\input{chapters/12/10/2/16/section.tex}
\item Determine the ratio in which the line $2x+y  - 4=0$ divides the line segment joining the points $\vec{A}(2, - 2)$  and  $\vec{B}(3, 7)$.
\\
\solution
	\input{chapters/10/7/4/1/section.tex}
\item Let $\vec{A}(4, 2), \vec{B}(6, 5)$  and $ \vec{C}(1, 4)$ be the vertices of $\triangle ABC$.
\begin{enumerate}
\item The median from $\vec{A}$ meets $BC$ at $\vec{D}$. Find the coordinates of the point $\vec{D}$.
\item Find the coordinates of the point $\vec{P}$ on $AD$ such that $AP : PD = 2 : 1$.
\item Find the coordinates of points $\vec{Q}$ and $\vec{R}$ on medians $BE$ and $CF$ respectively such that $BQ : QE = 2 : 1$  and  $CR : RF = 2 : 1$.
\item What do you observe?
\item If $\vec{A}, \vec{B}$ and $\vec{C}$  are the vertices of $\triangle ABC$, find the coordinates of the centroid of the triangle.
\end{enumerate}
\solution
	\input{chapters/10/7/4/7/section.tex}
\item Find the slope of a line, which passes through the origin and the mid point of the line segment joining the points $\vec{P}$(0,-4) and $\vec{B}$(8,0).
\label{chapters/11/10/1/5}
\input{chapters/11/10/1/5/matrix.tex}
\item Find the position vector of a point R which divides the line joining two points P and Q whose position vectors are $(2\vec{a}+\vec{b})$ and $(\vec{a}-3\vec{b})$
externally in the ratio 1 : 2. Also, show that P is the mid point of the line segment RQ.\\
	\solution
%		\input{chapters/12/10/5/9/section.tex}

\end{enumerate}


\item Determine the ratio in which the line $2x+y  - 4=0$ divides the line segment joining the points $\vec{A}(2, - 2)$  and  $\vec{B}(3, 7)$.
\\
\solution
	\iffalse
\documentclass[journal,12pt,twocolumn]{IEEEtran}
\usepackage{graphicx}
\graphicspath{{./chapters/10/7/4/1/figs/}}{}
\usepackage{amsmath,amssymb,amsfonts,amsthm}
\newcommand{\myvec}[1]{\ensuremath{\begin{pmatrix}#1\end{pmatrix}}}
\providecommand{\norm}[1]{\lVert#1\rVert}
\usepackage{listings}
\usepackage{watermark}
\usepackage{titlesec}
\usepackage{caption}
\let\vec\mathbf
\lstset{
frame=single, 
breaklines=true,
columns=fullflexible
}
\thiswatermark{\centering \put(0,-105.0){\includegraphics[scale=0.15]{/sdcard/IITH/vector/vectpr-4/chapters/10/7/4/1/figs/logo.png}} }
\title{\mytitle}
\title{
Assignment - Vector-4
}
\author{Surajit Sarkar}
\begin{document}
\maketitle
%\tableofcontents
\bigskip
\section{\textbf{Problem}}
Determine the ratio in which the line 2x+y–4=0 divides the line segment joining the points A(2,–2) and B(3,7).
\section{\textbf{Solution}}
\begin{table}[h]
    \centering
    \begin{tabular}{|c|c|}
       \hline
       \textbf{Symbol}&\textbf{Value}  \\
       \hline
	    $\vec{A}$ & $\myvec{2\\-2}$\\
        \hline
	    $\vec{B}$ & $\myvec{3\\7}$\\
        \hline
	    c&$4$\\
        \hline
       $\vec{n}$ & $\myvec{2\\1}$\\
       \hline
    \end{tabular}
    \caption{Parameters}
    \label{tab:my_label}
\end{table}
Given equation
\fi
The given equation can be expressed as
\begin{align}
    \myvec{2&1}\vec{x}&=4\\
\end{align}
Using section formula, the point of division 
\begin{align}
    \vec{P} = \frac{k\vec{B+A}}{k+1}
\end{align}
which upon substitution in the equation of a line yields
\begin{align}
    \implies\vec{n}^{\top}\myvec{\frac{k\vec{B+A}}{k+1}}&=c\\
    \implies k&=\frac{c-\vec{n}^{\top}\vec{A}}{\vec{n}^{\top}\vec{B}-c}\\
\end{align}
upon simplification.  Substituting numerical values, 
\begin{align}
    k=\frac{2}{9}
\end{align}
See Fig. 
\ref{fig:chapters/10/7/4/1vec}.
\begin{figure}[!h]
\centering
\includegraphics[width=\columnwidth]{chapters/10/7/4/1/figs/vec.pdf}
\caption{}
\label{fig:chapters/10/7/4/1vec}
\end{figure}


\item Let $\vec{A}(4, 2), \vec{B}(6, 5)$  and $ \vec{C}(1, 4)$ be the vertices of $\triangle ABC$.
\begin{enumerate}
\item The median from $\vec{A}$ meets $BC$ at $\vec{D}$. Find the coordinates of the point $\vec{D}$.
\item Find the coordinates of the point $\vec{P}$ on $AD$ such that $AP : PD = 2 : 1$.
\item Find the coordinates of points $\vec{Q}$ and $\vec{R}$ on medians $BE$ and $CF$ respectively such that $BQ : QE = 2 : 1$  and  $CR : RF = 2 : 1$.
\item What do you observe?
\item If $\vec{A}, \vec{B}$ and $\vec{C}$  are the vertices of $\triangle ABC$, find the coordinates of the centroid of the triangle.
\end{enumerate}
\solution
	\iffalse
\documentclass[12pt]{article}
\usepackage{graphicx}
\usepackage[none]{hyphenat}
\usepackage{graphicx}
\usepackage{listings}
\usepackage[english]{babel}
\usepackage{graphicx}
\usepackage{caption} 
\usepackage{booktabs}
\usepackage{array}
\usepackage{amssymb} % for \because
\usepackage{amsmath}   % for having text in math mode
\usepackage{extarrows} % for Row operations arrows
\usepackage{listings}
\usepackage[utf8]{inputenc}
\lstset{
  frame=single,
  breaklines=true
}
\usepackage{hyperref}
  
%Following 2 lines were added to remove the blank page at the beginning
\usepackage{atbegshi}% http://ctan.org/pkg/atbegshi
\AtBeginDocument{\AtBeginShipoutNext{\AtBeginShipoutDiscard}}


%New macro definitions
\newcommand{\mydet}[1]{\ensuremath{\begin{vmatrix}#1\end{vmatrix}}}
\providecommand{\brak}[1]{\ensuremath{\left(#1\right)}}
\newcommand{\solution}{\noindent \textbf{Solution: }}
\newcommand{\myvec}[1]{\ensuremath{\begin{pmatrix}#1\end{pmatrix}}}
\providecommand{\norm}[1]{\left\lVert#1\right\rVert}
\providecommand{\abs}[1]{\left\vert#1\right\vert}
\let\vec\mathbf

\begin{document}

\begin{center}
\title{\textbf{VECTORS}}
\date{\vspace{-5ex}} %Not to print date automatically
\maketitle
\end{center}

\section{10$^{th}$ Maths - EXERCISE-7.4}

Let A(4, 2), B(6, 5) and C(1, 4) be the vertices of $\triangle ABC$
\begin{enumerate}
\item The median from A meets BC at D. Find the coordinates of the point D.
\item Find the coordinates of the point P on AD such that $AP : PD = 2 : 1$
\item Find the coordinates of points Q and R on medians BE and CF respectively such
that $BQ : QE = 2 : 1 \text{and} CR : RF = 2 : 1.$
\item What do yo observe?
\item If $A(x_1, y_1), B(x_2, y_2) \text{and} C(x_3, y_3)$ are the vertices of $\triangle ABC$, find the coordinates of the centroid of the triangle.
\end{enumerate}

Given points are
\begin{align}
\vec{A}=\myvec{4\\ 2} ,
\vec{B}=\myvec{6\\ 5} ,
\vec{C}=\myvec{1\\ 4}
\end{align}
\fi

\begin{enumerate}
\item 
\begin{align}
\vec{D}&=\frac{\vec{B}+\vec{C}}{2}\\
&=\myvec{\frac{7}{2}\\[2pt] \frac{9}{2}}\\
\vec{E}&=\frac{\vec{A}+\vec{C}}{2}\\
&=\myvec{\frac{5}{2}\\ 3}\\
\vec{F}&=\frac{\vec{A}+\vec{B}}{2}\\
&=\myvec{5\\ \frac{7}{2}}
\end{align}

\item 
	For
$n=2$,
\begin{align}
\vec{P}&=\frac{1}{1+n}\brak{\myvec{\vec{A}+n\vec{D}}}\\
&=\frac{1}{3}\myvec{11\\11}
\end{align}

\item 
\begin{align}
\vec{Q}&=\frac{1}{1+n}\brak{\myvec{\vec{B}+n\vec{E}}}\\
&=\frac{1}{3}\myvec{11\\11}\\
\vec{R}&=\frac{1}{1+n}\brak{\myvec{\vec{C}+n\vec{F}}}\\
&=\frac{1}{3}\myvec{11\\11}\\
\end{align}

\item 
 $\vec{P},\vec{Q},\vec{R}$ are the same point.
   
\item 
\begin{align}
\vec{G}&=\frac{\vec{D}+\vec{E}+\vec{F}}{3}\\
&=\frac{1}{3}\myvec{11\\11}\\
\end{align} 
\end{enumerate}
See Fig.  
  \ref{fig:chapters/10/7/4/7/Figure}.
\begin{figure}[h!]
\centering
\includegraphics[width=\columnwidth]{chapters/10/7/4/7/figs/dj.pdf}
\caption{}
  \label{fig:chapters/10/7/4/7/Figure}
\end{figure}

\item Find the slope of a line, which passes through the origin and the mid point of the line segment joining the points $\vec{P}$(0,-4) and $\vec{B}$(8,0).
\label{chapters/11/10/1/5}
\iffalse
\documentclass[journal,12pt,twocolumn]{IEEEtran}
\usepackage{graphicx}
\graphicspath{{./figs/}}{}
\usepackage{amsmath,amssymb,amsfonts,amsthm}
\newcommand{\myvec}[1]{\ensuremath{\begin{pmatrix}#1\end{pmatrix}}}

\let\vec\mathbf

\title{
Matrix-Lines
}
\author{Jyothsna Paluchuri-FWC22059\\}
\begin{document}
\maketitle
\tableofcontents
\bigskip
\section{Problem Statement}
\fi
	\begin{figure}[!ht]
		\centering
 \includegraphics[width=\columnwidth]{chapters/11/10/1/5/figs/line.png}
		\caption{}
		\label{fig:11/10/1/5}
  	\end{figure}
	\\
	\solution
\iffalse
\section{Construction}
\begin{figure}[h]
    \centering
\includegraphics[width=\columnwidth]{line.png}
    \caption{Equation of the slope}
    \label{fig:my_label}
\end{figure}
\vspace{2cm}
\begin{table}[h]
    \centering
    \begin{tabular}{|c|c|c|c|}
       \hline
       \textbf{Symbol}&\textbf{Value}&\textbf{Description}  \\
       \hline
	    $\vec{P}$ & $\myvec{
		    0\\
		    -4}$
	    & Point on Y-axis\\
        \hline
	    $\vec{B}$ & $\myvec{8\\0}$
 & Point on X-axis\\
        \hline
	    $\vec{0}$ & $\myvec{0\\0}$
 & Origin\\
        \hline
    \end{tabular}
    \caption{Parameters}
    \label{tab:my_label}
\end{table}


\section{Solution}
Given that resultant line passes through origin and mid point of the line segment joining point P(0,-4) and B(8,0) \\
\\
\\
given ${\vec{P}}$=$\myvec{
  0\\
  -4}$
 , ${\vec{B}}$=$\myvec{
  8\\
  0}$
  
 \fi 
The mid point of $PB$ is
\begin{align}
\vec{M} &=\frac{1}{2}(\vec{P}+\vec{B})
	= \myvec{4 \\ -2}  
\end{align}
The direction vector of line joining $\vec{O}, \vec{M}$ is 
\begin{align}
\vec{m}&=\vec{O}-\vec{M}
 = -\vec{M}
\end{align}
which can be expressed as
\begin{align}
	\myvec{1 \\ -\frac{1}{2}}
\end{align}
Thus the slope is
\begin{align}
	m = -\frac{1}{2}
\end{align}
\iffalse
\textbf{The direction vector of a line expressed as}
\begin{align}
\implies\vec{m} &= \begin{pmatrix}1 \\ m \\ \end{pmatrix}
\end{align}

\textbf{By solving equation (5) and (6),we get the slope of $\vec{O}$ $\vec{M}$ line}
\begin{align}
        \boxed{m=-0.5}
 \end{align}

\section{Software}
Download the following code using,
\begin{table}[h]
    \centering
    \begin{tabular}{|c|}
    \hline \\
   https://github.com/jyothsna777/jyothsna-fwc.git  \\
         \\
\hline
    \end{tabular}
\end{table}
\\
and execute the code by using command
\begin{center}
\textbf{Python3 lines.py}\\
\end{center}

\section{Conclusion}
Hence the slope of line $\vec{O}$ $\vec{M}$ lineis $\vec{m}$=-0.5

\end{document}
\fi

\item Find the position vector of a point R which divides the line joining two points P and Q whose position vectors are $(2\vec{a}+\vec{b})$ and $(\vec{a}-3\vec{b})$
externally in the ratio 1 : 2. Also, show that P is the mid point of the line segment RQ.\\
	\solution
%		\begin{enumerate}[label=\thesection.\arabic*,ref=\thesection.\theenumi]
\numberwithin{equation}{enumi}
\numberwithin{figure}{enumi}
\numberwithin{table}{enumi}

\item Find the coordinates of the point which divides the join of $(-1,7) \text{ and } (4,-3)$ in the ratio 2:3.
	\\
		\solution
	\input{chapters/10/7/2/1/section.tex}
\item Find the coordinates of the points of trisection of the line segment joining $(4,-1) \text{ and } (-2,3)$.
	\\
		\solution
	\input{chapters/10/7/2/2/section.tex}
\item
	\iffalse
\item To conduct Sports Day activities, in your rectangular shaped school                   
ground ABCD, lines have 
drawn with chalk powder at a                 
distance of 1m each. 100 flower pots have been placed at a distance of 1m 
from each other along AD, as shown 
in Fig. 7.12. Niharika runs $ \frac {1}{4} $th the 
distance AD on the 2nd line and 
posts a green flag. Preet runs $ \frac {1}{5} $th 
the distance AD on the eighth line 
and posts a red flag. What is the 
distance between both the flags? If 
Rashmi has to post a blue flag exactly 
halfway between the line segment 
joining the two flags, where should 
she post her flag?
\begin{figure}[h!]
  \centering
  \includegraphics[width=\columnwidth]{sc.png}
  \caption{}
\label{fig:10/7/12Fig1}
\end{figure}               
\fi
      
\item Find the ratio in which the line segment joining the points $(-3,10) \text{ and } (6,-8)$ $\text{ is divided by } (-1,6)$.
	\\
		\solution
	\input{chapters/10/7/2/4/section.tex}
\item Find the ratio in which the line segment joining $A(1,-5) \text{ and } B(-4,5)$ $\text{is divided by the x-axis}$. Also find the coordinates of the point of division.
\item If $(1,2), (4,y), (x,6), (3,5)$ are the vertices of a parallelogram taken in order, find x and y.
	\\
		\solution
	\input{chapters/10/7/2/6/para1.tex}
\item Find the coordinates of a point A, where AB is the diameter of a circle whose centre is $(2,-3) \text{ and }$ B is $(1,4)$.
	\\
		\solution
	\input{chapters/10/7/2/7/section.tex}
\item If A \text{ and } B are $(-2,-2) \text{ and } (2,-4)$, respectively, find the coordinates of P such that AP= $\frac {3}{7}$AB $\text{ and }$ P lies on the line segment AB.
	\\
		\solution
	\input{chapters/10/7/2/8/section.tex}
\item Find the coordinates of the points which divide the line segment joining $A(-2,2) \text{ and } B(2,8)$ into four equal parts.
	\\
		\solution
	\input{chapters/10/7/2/9/section.tex}
\item Find the area of a rhombus if its vertices are $(3,0), (4,5), (-1,4) \text{ and } (-2,-1)$ taken in order. [$\vec{Hint}$ : Area of rhombus =$\frac {1}{2}$(product of its diagonals)]
	\\
		\solution
	\input{chapters/10/7/2/10/cross.tex}
\item Find the position vector of a point R which divides the line joining two points $\vec{P}$
and $\vec{Q}$ whose position vectors are $\hat{i}+2\hat{j}-\hat{k}$ and $-\hat{i}+\hat{j}+\hat{k}$ respectively, in the
ratio 2 : 1
\begin{enumerate}
    \item  internally
    \item  externally
\end{enumerate}
\solution
		\input{chapters/12/10/2/15/section.tex}
\item Find the position vector of the mid point of the vector joining the points $\vec{P}$(2, 3, 4)
and $\vec{Q}$(4, 1, –2).
\\
\solution
		\input{chapters/12/10/2/16/section.tex}
\item Determine the ratio in which the line $2x+y  - 4=0$ divides the line segment joining the points $\vec{A}(2, - 2)$  and  $\vec{B}(3, 7)$.
\\
\solution
	\input{chapters/10/7/4/1/section.tex}
\item Let $\vec{A}(4, 2), \vec{B}(6, 5)$  and $ \vec{C}(1, 4)$ be the vertices of $\triangle ABC$.
\begin{enumerate}
\item The median from $\vec{A}$ meets $BC$ at $\vec{D}$. Find the coordinates of the point $\vec{D}$.
\item Find the coordinates of the point $\vec{P}$ on $AD$ such that $AP : PD = 2 : 1$.
\item Find the coordinates of points $\vec{Q}$ and $\vec{R}$ on medians $BE$ and $CF$ respectively such that $BQ : QE = 2 : 1$  and  $CR : RF = 2 : 1$.
\item What do you observe?
\item If $\vec{A}, \vec{B}$ and $\vec{C}$  are the vertices of $\triangle ABC$, find the coordinates of the centroid of the triangle.
\end{enumerate}
\solution
	\input{chapters/10/7/4/7/section.tex}
\item Find the slope of a line, which passes through the origin and the mid point of the line segment joining the points $\vec{P}$(0,-4) and $\vec{B}$(8,0).
\label{chapters/11/10/1/5}
\input{chapters/11/10/1/5/matrix.tex}
\item Find the position vector of a point R which divides the line joining two points P and Q whose position vectors are $(2\vec{a}+\vec{b})$ and $(\vec{a}-3\vec{b})$
externally in the ratio 1 : 2. Also, show that P is the mid point of the line segment RQ.\\
	\solution
%		\input{chapters/12/10/5/9/section.tex}

\end{enumerate}



\end{enumerate}


\item Find the area of a rhombus if its vertices are $(3,0), (4,5), (-1,4) \text{ and } (-2,-1)$ taken in order. [$\vec{Hint}$ : Area of rhombus =$\frac {1}{2}$(product of its diagonals)]
	\\
		\solution
	\iffalse
\documentclass[12pt]{article}
\usepackage{graphicx}
%\documentclass[journal,12pt,twocolumn]{IEEEtran}
\usepackage[none]{hyphenat}
\usepackage{graphicx}
\usepackage{listings}
\usepackage[english]{babel}
\usepackage{graphicx}
\usepackage{caption} 
\usepackage{hyperref}
\usepackage{booktabs}
\def\inputGnumericTable{}
\usepackage{color}                                            %%
    \usepackage{array}                                            %%
    \usepackage{longtable}                                        %%
    \usepackage{calc}                                             %%
    \usepackage{multirow}                                         %%
    \usepackage{hhline}                                           %%
    \usepackage{ifthen}
\usepackage{array}
\usepackage{amsmath}   % for having text in math mode
\usepackage{listings}
\lstset{
language=tex,
frame=single, 
breaklines=true
}
  
%Following 2 lines were added to remove the blank page at the beginning
\usepackage{atbegshi}% http://ctan.org/pkg/atbegshi
\AtBeginDocument{\AtBeginShipoutNext{\AtBeginShipoutDiscard}}
%


%New macro definitions
\newcommand{\mydet}[1]{\ensuremath{\begin{vmatrix}#1\end{vmatrix}}}
\providecommand{\brak}[1]{\ensuremath{\left(#1\right)}}
\providecommand{\norm}[1]{\left\lVert#1\right\rVert}
\newcommand{\solution}{\noindent \textbf{Solution: }}
\newcommand{\myvec}[1]{\ensuremath{\begin{pmatrix}#1\end{pmatrix}}}
\let\vec\mathbf

\begin{document}

\begin{center}
\title{\textbf{Coordinate Geometry}}
\date{\vspace{-5ex}} %Not to print date automatically
\maketitle
\end{center}

\setcounter{page}{1}



\begin{enumerate}

\item\textbf{Problem statement :} Find the area of a rhombus of its vertices are $\myvec{3 ,0}$, $\myvec{4 ,5}$, $\myvec{-1 ,4}$ and $\myvec{-2 ,-1}$taken in order

\solution \\
\fi
The input vertices for this problem are given as
	\begin{align}
	\vec{A} = \myvec{
		3\\
		0
		},
	\vec{B} = \myvec{
		4\\
		5
		},
        \vec{C} = \myvec{
		-1\\
		4
		},
        \vec{D} = \myvec{
		-2\\
		-1
		}
	\end{align}
Since		
\begin{align}
 \vec{A-D}= \myvec{3 \\ 0} - \myvec{-2 \\-1}= \myvec{5\\1}
 \\
  \vec{B-A}= \myvec{4 \\ 5} - \myvec{3 \\0}= \myvec{1\\5}
\end{align}
the area of the rhombus is
\begin{align}
                \norm{\myvec{\vec{A-D}}\times \myvec{\vec{B-A}}}=\mydet{5 & 1\\1 & 5} = 24
\end{align}
See Fig. 
\ref{fig:chapters/10/7/2/10/gFig1}.
\begin{figure}[!h]
 \begin{center}
  \includegraphics[width=\columnwidth]{chapters/10/7/2/10/figs/fig.pdf}
 \end{center}
\caption{}
\label{fig:chapters/10/7/2/10/gFig1}
\end{figure}

\item Find the position vector of a point R which divides the line joining two points $\vec{P}$
and $\vec{Q}$ whose position vectors are $\hat{i}+2\hat{j}-\hat{k}$ and $-\hat{i}+\hat{j}+\hat{k}$ respectively, in the
ratio 2 : 1
\begin{enumerate}
    \item  internally
    \item  externally
\end{enumerate}
\solution
		\begin{enumerate}[label=\thesection.\arabic*,ref=\thesection.\theenumi]
\numberwithin{equation}{enumi}
\numberwithin{figure}{enumi}
\numberwithin{table}{enumi}

\item Find the coordinates of the point which divides the join of $(-1,7) \text{ and } (4,-3)$ in the ratio 2:3.
	\\
		\solution
	\iffalse
\documentclass[12pt]{article}
\usepackage{graphicx}
\usepackage{amsmath}
\usepackage{mathtools}
\usepackage{gensymb}

\newcommand{\mydet}[1]{\ensuremath{\begin{vmatrix}#1\end{vmatrix}}}
\providecommand{\brak}[1]{\ensuremath{\left(#1\right)}}
\providecommand{\norm}[1]{\left\lVert#1\right\rVert}
\newcommand{\solution}{\noindent \textbf{Solution: }}
\newcommand{\myvec}[1]{\ensuremath{\begin{pmatrix}#1\end{pmatrix}}}
\let\vec\mathbf

\begin{document}
\begin{center}
\textbf\large{CHAPTER-7 \\ COORDINATE GEOMETRY}
\end{center}
\section*{Excercise 7.2}

1. Find the coordinates of the point which divides the join $\vec(-1,7) \text{ and } \vec(4,-3)$ in the ratio 2:3 :
\\
\\
\solution\\		
\fi
The coordinates and ratio are given as
\begin{align}
\vec{P}=\myvec{-1\\7\\},
\vec{Q}=\myvec{4\\-3\\},
n=\frac{3}{2}
\end{align}
Using section formula
\begin{align}
\vec{R}&=\frac{\vec{Q}+n\vec{P}}{1+n}\\
&=\frac{1}{1+\frac{3}{2}}  \myvec{\myvec{
4\\
-3\\
}
  +
   \frac{3}{2}\myvec{
-1\\
7\\
}}\\
&=\myvec{
1\\
3
}
\end{align}
See Fig. 
\ref{fig:chapters/10/7/2/1/Fig}
\begin{figure}[!h]
\begin{center}
   \includegraphics[width=\columnwidth]{chapters/10/7/2/1/figs/linefig.png}
\end{center}
\caption{}
\label{fig:chapters/10/7/2/1/Fig}
\end{figure}


\item Find the coordinates of the points of trisection of the line segment joining $(4,-1) \text{ and } (-2,3)$.
	\\
		\solution
	\begin{enumerate}[label=\thesection.\arabic*,ref=\thesection.\theenumi]
\numberwithin{equation}{enumi}
\numberwithin{figure}{enumi}
\numberwithin{table}{enumi}

\item Find the coordinates of the point which divides the join of $(-1,7) \text{ and } (4,-3)$ in the ratio 2:3.
	\\
		\solution
	\input{chapters/10/7/2/1/section.tex}
\item Find the coordinates of the points of trisection of the line segment joining $(4,-1) \text{ and } (-2,3)$.
	\\
		\solution
	\input{chapters/10/7/2/2/section.tex}
\item
	\iffalse
\item To conduct Sports Day activities, in your rectangular shaped school                   
ground ABCD, lines have 
drawn with chalk powder at a                 
distance of 1m each. 100 flower pots have been placed at a distance of 1m 
from each other along AD, as shown 
in Fig. 7.12. Niharika runs $ \frac {1}{4} $th the 
distance AD on the 2nd line and 
posts a green flag. Preet runs $ \frac {1}{5} $th 
the distance AD on the eighth line 
and posts a red flag. What is the 
distance between both the flags? If 
Rashmi has to post a blue flag exactly 
halfway between the line segment 
joining the two flags, where should 
she post her flag?
\begin{figure}[h!]
  \centering
  \includegraphics[width=\columnwidth]{sc.png}
  \caption{}
\label{fig:10/7/12Fig1}
\end{figure}               
\fi
      
\item Find the ratio in which the line segment joining the points $(-3,10) \text{ and } (6,-8)$ $\text{ is divided by } (-1,6)$.
	\\
		\solution
	\input{chapters/10/7/2/4/section.tex}
\item Find the ratio in which the line segment joining $A(1,-5) \text{ and } B(-4,5)$ $\text{is divided by the x-axis}$. Also find the coordinates of the point of division.
\item If $(1,2), (4,y), (x,6), (3,5)$ are the vertices of a parallelogram taken in order, find x and y.
	\\
		\solution
	\input{chapters/10/7/2/6/para1.tex}
\item Find the coordinates of a point A, where AB is the diameter of a circle whose centre is $(2,-3) \text{ and }$ B is $(1,4)$.
	\\
		\solution
	\input{chapters/10/7/2/7/section.tex}
\item If A \text{ and } B are $(-2,-2) \text{ and } (2,-4)$, respectively, find the coordinates of P such that AP= $\frac {3}{7}$AB $\text{ and }$ P lies on the line segment AB.
	\\
		\solution
	\input{chapters/10/7/2/8/section.tex}
\item Find the coordinates of the points which divide the line segment joining $A(-2,2) \text{ and } B(2,8)$ into four equal parts.
	\\
		\solution
	\input{chapters/10/7/2/9/section.tex}
\item Find the area of a rhombus if its vertices are $(3,0), (4,5), (-1,4) \text{ and } (-2,-1)$ taken in order. [$\vec{Hint}$ : Area of rhombus =$\frac {1}{2}$(product of its diagonals)]
	\\
		\solution
	\input{chapters/10/7/2/10/cross.tex}
\item Find the position vector of a point R which divides the line joining two points $\vec{P}$
and $\vec{Q}$ whose position vectors are $\hat{i}+2\hat{j}-\hat{k}$ and $-\hat{i}+\hat{j}+\hat{k}$ respectively, in the
ratio 2 : 1
\begin{enumerate}
    \item  internally
    \item  externally
\end{enumerate}
\solution
		\input{chapters/12/10/2/15/section.tex}
\item Find the position vector of the mid point of the vector joining the points $\vec{P}$(2, 3, 4)
and $\vec{Q}$(4, 1, –2).
\\
\solution
		\input{chapters/12/10/2/16/section.tex}
\item Determine the ratio in which the line $2x+y  - 4=0$ divides the line segment joining the points $\vec{A}(2, - 2)$  and  $\vec{B}(3, 7)$.
\\
\solution
	\input{chapters/10/7/4/1/section.tex}
\item Let $\vec{A}(4, 2), \vec{B}(6, 5)$  and $ \vec{C}(1, 4)$ be the vertices of $\triangle ABC$.
\begin{enumerate}
\item The median from $\vec{A}$ meets $BC$ at $\vec{D}$. Find the coordinates of the point $\vec{D}$.
\item Find the coordinates of the point $\vec{P}$ on $AD$ such that $AP : PD = 2 : 1$.
\item Find the coordinates of points $\vec{Q}$ and $\vec{R}$ on medians $BE$ and $CF$ respectively such that $BQ : QE = 2 : 1$  and  $CR : RF = 2 : 1$.
\item What do you observe?
\item If $\vec{A}, \vec{B}$ and $\vec{C}$  are the vertices of $\triangle ABC$, find the coordinates of the centroid of the triangle.
\end{enumerate}
\solution
	\input{chapters/10/7/4/7/section.tex}
\item Find the slope of a line, which passes through the origin and the mid point of the line segment joining the points $\vec{P}$(0,-4) and $\vec{B}$(8,0).
\label{chapters/11/10/1/5}
\input{chapters/11/10/1/5/matrix.tex}
\item Find the position vector of a point R which divides the line joining two points P and Q whose position vectors are $(2\vec{a}+\vec{b})$ and $(\vec{a}-3\vec{b})$
externally in the ratio 1 : 2. Also, show that P is the mid point of the line segment RQ.\\
	\solution
%		\input{chapters/12/10/5/9/section.tex}

\end{enumerate}


\item
	\iffalse
\item To conduct Sports Day activities, in your rectangular shaped school                   
ground ABCD, lines have 
drawn with chalk powder at a                 
distance of 1m each. 100 flower pots have been placed at a distance of 1m 
from each other along AD, as shown 
in Fig. 7.12. Niharika runs $ \frac {1}{4} $th the 
distance AD on the 2nd line and 
posts a green flag. Preet runs $ \frac {1}{5} $th 
the distance AD on the eighth line 
and posts a red flag. What is the 
distance between both the flags? If 
Rashmi has to post a blue flag exactly 
halfway between the line segment 
joining the two flags, where should 
she post her flag?
\begin{figure}[h!]
  \centering
  \includegraphics[width=\columnwidth]{sc.png}
  \caption{}
\label{fig:10/7/12Fig1}
\end{figure}               
\fi
      
\item Find the ratio in which the line segment joining the points $(-3,10) \text{ and } (6,-8)$ $\text{ is divided by } (-1,6)$.
	\\
		\solution
	\iffalse
\documentclass[12pt]{article}
\usepackage{graphicx}
%\documentclass[journal,12pt,twocolumn]{IEEEtran}
\usepackage[none]{hyphenat}
\usepackage{graphicx}
\usepackage{listings}
\usepackage[english]{babel}
\usepackage{graphicx}
\usepackage{caption} 
\usepackage{hyperref}
\usepackage{booktabs}
\def\inputGnumericTable{}
\usepackage{color}                                            %%
    \usepackage{array}                                            %%
    \usepackage{longtable}                                        %%
    \usepackage{calc}                                             %%
    \usepackage{multirow}                                         %%
    \usepackage{hhline}                                           %%
    \usepackage{ifthen}
\usepackage{array}
\usepackage{amsmath}   % for having text in math mode
\usepackage{listings}
\lstset{
language=tex,
frame=single, 
breaklines=true
}
  
%Following 2 lines were added to remove the blank page at the beginning
\usepackage{atbegshi}% http://ctan.org/pkg/atbegshi
\AtBeginDocument{\AtBeginShipoutNext{\AtBeginShipoutDiscard}}
%
%New macro definitions
\newcommand{\mydet}[1]{\ensuremath{\begin{vmatrix}#1\end{vmatrix}}}
\providecommand{\brak}[1]{\ensuremath{\left(#1\right)}}
\providecommand{\norm}[1]{\left\lVert#1\right\rVert}
\newcommand{\solution}{\noindent \textbf{Solution: }}
\newcommand{\myvec}[1]{\ensuremath{\begin{pmatrix}#1\end{pmatrix}}}
\let\vec\mathbf
\begin{document}
\begin{center}
\title{\textbf{Coordinate Geometry}}
\date{\vspace{-5ex}} %Not to print date automatically
\maketitle
\end{center}
\setcounter{page}{1}
\section*{10$^{th}$ Maths - Chapter 7}
This is Problem-4 from Exercise 7.2
\begin{enumerate}
\item Find the ratio in which the line segement joining the points $\myvec{-3 \\ 10}$ and $\myvec{6\\-8}$ is divided by $\myvec{-1\\6}$.\\
\solution \\
\fi
		The input parameters for this problem are available in Table \eqref{tab:10/7/2/4-1}.
\begin{table}[ht!]
\input{chapters/10/7/2/4/tables/table.tex}
\caption{}
\label{tab:10/7/2/4-1} 
\end{table}
Using section formula,
\begin{align}
         \vec{R} &=\frac{\vec{Q}+n\vec{P}}{1+n}\label{eq:chapters/10/7/2/4/1}
\end{align}
Substituting the values of $\vec{P},\vec{Q}$ and $\vec{R}$ in \eqref{eq:chapters/10/7/2/4/1}
\begin{align}
         \myvec{-1\\6} &=\frac{{\myvec{-3\\10}+n\myvec{6\\-8}}}{1+n}\\
 &=\frac{1}{1+n}\brak{{\myvec{-3\\10}+n\myvec{6\\-8}}} \\
 &=\frac{1}{1+n}\myvec{-3+6n\\10-8n} \label{eq:chapters/10/7/2/4/4}
\end{align}
Simplifying \eqref{eq:chapters/10/7/2/4/4} yeilds,
\begin{align}
          -1 &=\frac{-3+6n}{1+n}\\
\implies          n &=\frac{2}{7}
\end{align}
Also,
\begin{align}
          6 &=\frac{10-8n}{1+n}\\
    \implies      n &=\frac{2}{7}
\end{align}
Hence the desired ratio is $\dfrac{2}{7}$.  
\begin{figure}[!h]
 \begin{center}
  \includegraphics[width=\columnwidth]{chapters/10/7/2/4/figs/fig.png}
 \end{center}
\caption{}
\label{fig:10/7/2/4Fig1}
\end{figure}

\item Find the ratio in which the line segment joining $A(1,-5) \text{ and } B(-4,5)$ $\text{is divided by the x-axis}$. Also find the coordinates of the point of division.
\item If $(1,2), (4,y), (x,6), (3,5)$ are the vertices of a parallelogram taken in order, find x and y.
	\\
		\solution
	\iffalse
\documentclass[12pt]{article}
\usepackage{graphicx}
%\documentclass[journal,12pt,twocolumn]{IEEEtran}
\def\inputGnumericTable{}
\usepackage{color}                                            %%
    \usepackage{array}                                            %%
    \usepackage{longtable}                                        %%
    \usepackage{calc}                                             %%
    \usepackage{multirow}                                         %%
    \usepackage{hhline}                                           %%
    \usepackage{ifthen}
\usepackage[none]{hyphenat}
\usepackage{graphicx}
\usepackage{listings}
\usepackage[english]{babel}
\usepackage{graphicx}
\usepackage{caption} 
\usepackage{hyperref}
\usepackage{booktabs}
\usepackage{array}
\usepackage{amsmath}   % for having text in math mode
\usepackage{listings}
\lstset{
  frame=single,
  breaklines=true
}
  
%Following 2 lines were added to remove the blank page at the beginning
\usepackage{atbegshi}% http://ctan.org/pkg/atbegshi
\AtBeginDocument{\AtBeginShipoutNext{\AtBeginShipoutDiscard}}
%


%New macro definitions
\newcommand{\mydet}[1]{\ensuremath{\begin{vmatrix}#1\end{vmatrix}}}
\providecommand{\brak}[1]{\ensuremath{\left(#1\right)}}
\providecommand{\norm}[1]{\left\lVert#1\right\rVert}
\newcommand{\solution}{\noindent \textbf{Solution: }}
\newcommand{\myvec}[1]{\ensuremath{\begin{pmatrix}#1\end{pmatrix}}}
\let\vec\mathbf

\begin{document}

\begin{center}
\title{\textbf{Properties of Parallelegram}}
\date{\vspace{-5ex}} %Not to print date automatically
\maketitle
\end{center}

\setcounter{page}{1}

\section{10$^{th}$ Maths - Chapter 7}

This is Problem-6 from Exercise 7.2

\begin{enumerate}
\item If $\vec{A}(1, 2),\vec{B}(4, x),\vec{C}(y, 6) \text{and } \vec{D}(3, 5)$ are the vertices of a parallelogram taken in order,find x and y.
\end{enumerate}
\fi

The input parameters for this problem are available in
\ref{table:chapters/10/7/2/6/tables/}.	
\begin{table}[!ht]
	\centering
	\input{chapters/10/7/2/6/tables/table.tex}
\caption{}
\label{table:chapters/10/7/2/6/tables/}	
\end{table}
From the given information,
\begin{align}
  \label{eq:chapters/10/7/2/6/tables/det2f}
	\vec{B}-\vec{A} &= \myvec{4 \\y } - \myvec{1 \\2 }  = \myvec{3 \\y-2 }\\
	\vec{C}-\vec{D} &= \myvec{x \\6 } - \myvec{3 \\5 }  = \myvec{x-3 \\1}
\end{align}
Since $ABCD$ is a parallellogram,
\begin{align}
	\myvec{3\\y-2}&=\myvec{x-3\\1}\\
	\implies x&=6 ,y=3
\end{align}
Fig. \ref{fig:chapters/10/7/2/6/Fig3}
provides a verification.
\begin{figure}[h!]
	\begin{center}
  \includegraphics[width=\columnwidth]{chapters/10/7/2/6/figs/para.pdf}
	\end{center}
\caption{}
\label{fig:chapters/10/7/2/6/Fig3}
\end{figure}


\item Find the coordinates of a point A, where AB is the diameter of a circle whose centre is $(2,-3) \text{ and }$ B is $(1,4)$.
	\\
		\solution
	\iffalse
\documentclass[12pt]{article}
\usepackage{graphicx}
\usepackage{amsmath}
\usepackage{mathtools}
\usepackage{gensymb}

\newcommand{\mydet}[1]{\ensuremath{\begin{vmatrix}#1\end{vmatrix}}}
\providecommand{\brak}[1]{\ensuremath{\left(#1\right)}}
\providecommand{\norm}[1]{\left\lVert#1\right\rVert}
\newcommand{\solution}{\noindent \textbf{Solution: }}
\newcommand{\myvec}[1]{\ensuremath{\begin{pmatrix}#1\end{pmatrix}}}
\let\vec\mathbf

\begin{document}
\begin{center}
\section*{CHAPTER 7 - COORDINATE GEOMETRY}

\end{center}
\section*{Excercise 7.2}

Q7.Find the coordinates of point $\vec{A}$, where AB is the diameter of a circle where the center is (2,-3) and $\vec{B}$ is the point (1,4):

\solution
\begin{enumerate}
\item The coordinates $\vec{B}$ and center $\vec{C}$ are given, where:
	\fi
	Let
	\begin{align}
	\vec{B} = \myvec{
		1\\
	    4\\
		},
	\vec{C} = \myvec{
	    2\\
	   -3\\
		}
	\end{align}
	\iffalse
Let us assume the coordinates of $\vec{A}$. Now, $\vec{C}$ is the center which is midpoint of line AB and $\vec{B}$ is one of the coordinate of diameter AB of a circle.
	\fi	
Hence,	
	\begin{align}
	\vec{C} &= \frac{\vec{A+B}}{2} \\
\implies	2\vec{C} &= \vec{A}+\vec{B} \\
		\text{or, }	\vec{A} &= 2\vec{C}-\vec{B} \\
	 &= \myvec{3\\-10\\}	
	\end{align}       
	See Fig. 
\ref{fig:chapters/10/7/2/7Fig}.
\begin{figure}[!h]
\begin{center}	
	\includegraphics[width=\columnwidth]{chapters/10/7/2/7/figs/Vector1.png}
\end{center}
\caption{}
\label{fig:chapters/10/7/2/7Fig}
\end{figure}
	

\item If A \text{ and } B are $(-2,-2) \text{ and } (2,-4)$, respectively, find the coordinates of P such that AP= $\frac {3}{7}$AB $\text{ and }$ P lies on the line segment AB.
	\\
		\solution
	\iffalse
\documentclass[journal,10pt,twocolumn]{article}
\usepackage{graphicx}
\usepackage[none]{hyphenat}
\usepackage{graphicx}
\usepackage{listings}
\usepackage[english]{babel}
\usepackage{graphicx}
\usepackage{caption} 
\usepackage{booktabs}
\usepackage{array}
\usepackage{amssymb} % for \because
\usepackage{amsmath}   % for having text in math mode
\usepackage{extarrows} % for Row operations arrows
\usepackage{listings}
\usepackage[utf8]{inputenc}
\lstset{
  frame=single,
  breaklines=true
}
\usepackage{hyperref}
  
%Following 2 lines were added to remove the blank page at the beginning
\usepackage{atbegshi}% http://ctan.org/pkg/atbegshi
\AtBeginDocument{\AtBeginShipoutNext{\AtBeginShipoutDiscard}}


%New macro definitions
\newcommand{\mydet}[1]{\ensuremath{\begin{vmatrix}#1\end{vmatrix}}}
\providecommand{\brak}[1]{\ensuremath{\left(#1\right)}}
\newcommand{\solution}{\noindent \textbf{Solution: }}
\newcommand{\myvec}[1]{\ensuremath{\begin{pmatrix}#1\end{pmatrix}}}
\providecommand{\norm}[1]{\left\lVert#1\right\rVert}
\providecommand{\abs}[1]{\left\vert#1\right\vert}
\let\vec\mathbf

\begin{document}

\begin{center}
\title{\textbf{VECTORS}}
\date{\vspace{-5ex}} %Not to print date automatically
\maketitle
\end{center}

\section{10$^{th}$ Maths - EXERCISE-7.2}

\begin{enumerate}
\item If A and B are $(– 2, – 2)\text{ and }(2, – 4)$, respectively, find the coordinates of P such that $AP =\frac{3}{7}AB$ and P lies on the line segment AB. 

\section{SOLUTION}
Given points are
\begin{align}
\vec{A}=\myvec{-2\\ -2} ,
\vec{B}=\myvec{2\\ -4}
\end{align}
The equation of the formula is
\fi
Using section formula, 
\begin{align}
\vec{P}&=\frac{\vec{A}+n\vec{B}}{1+n}
\end{align}
where
\begin{align}
	n =\frac{3}{4}
\end{align}
Thus,
\begin{align}
\vec{P}&=\frac{1}{1+\frac{3}{4}}\brak{\myvec{-2\\-2}+\frac{3}{4}\myvec{2\\-4}}\\
&=\myvec{\frac{-2}{7}\\[1pt] \frac{-20}{7}}
\end{align}
See Fig. 
   \ref{fig:chapters/10/7/2/8/vec.png}
\begin{figure}
   \centering 
 \includegraphics[width=\columnwidth]{chapters/10/7/2/8/figs/vec.png}
   \caption{}
   \label{fig:chapters/10/7/2/8/vec.png}
   \end{figure}

\item Find the coordinates of the points which divide the line segment joining $A(-2,2) \text{ and } B(2,8)$ into four equal parts.
	\\
		\solution
	\begin{enumerate}[label=\thesection.\arabic*,ref=\thesection.\theenumi]
\numberwithin{equation}{enumi}
\numberwithin{figure}{enumi}
\numberwithin{table}{enumi}

\item Find the coordinates of the point which divides the join of $(-1,7) \text{ and } (4,-3)$ in the ratio 2:3.
	\\
		\solution
	\input{chapters/10/7/2/1/section.tex}
\item Find the coordinates of the points of trisection of the line segment joining $(4,-1) \text{ and } (-2,3)$.
	\\
		\solution
	\input{chapters/10/7/2/2/section.tex}
\item
	\iffalse
\item To conduct Sports Day activities, in your rectangular shaped school                   
ground ABCD, lines have 
drawn with chalk powder at a                 
distance of 1m each. 100 flower pots have been placed at a distance of 1m 
from each other along AD, as shown 
in Fig. 7.12. Niharika runs $ \frac {1}{4} $th the 
distance AD on the 2nd line and 
posts a green flag. Preet runs $ \frac {1}{5} $th 
the distance AD on the eighth line 
and posts a red flag. What is the 
distance between both the flags? If 
Rashmi has to post a blue flag exactly 
halfway between the line segment 
joining the two flags, where should 
she post her flag?
\begin{figure}[h!]
  \centering
  \includegraphics[width=\columnwidth]{sc.png}
  \caption{}
\label{fig:10/7/12Fig1}
\end{figure}               
\fi
      
\item Find the ratio in which the line segment joining the points $(-3,10) \text{ and } (6,-8)$ $\text{ is divided by } (-1,6)$.
	\\
		\solution
	\input{chapters/10/7/2/4/section.tex}
\item Find the ratio in which the line segment joining $A(1,-5) \text{ and } B(-4,5)$ $\text{is divided by the x-axis}$. Also find the coordinates of the point of division.
\item If $(1,2), (4,y), (x,6), (3,5)$ are the vertices of a parallelogram taken in order, find x and y.
	\\
		\solution
	\input{chapters/10/7/2/6/para1.tex}
\item Find the coordinates of a point A, where AB is the diameter of a circle whose centre is $(2,-3) \text{ and }$ B is $(1,4)$.
	\\
		\solution
	\input{chapters/10/7/2/7/section.tex}
\item If A \text{ and } B are $(-2,-2) \text{ and } (2,-4)$, respectively, find the coordinates of P such that AP= $\frac {3}{7}$AB $\text{ and }$ P lies on the line segment AB.
	\\
		\solution
	\input{chapters/10/7/2/8/section.tex}
\item Find the coordinates of the points which divide the line segment joining $A(-2,2) \text{ and } B(2,8)$ into four equal parts.
	\\
		\solution
	\input{chapters/10/7/2/9/section.tex}
\item Find the area of a rhombus if its vertices are $(3,0), (4,5), (-1,4) \text{ and } (-2,-1)$ taken in order. [$\vec{Hint}$ : Area of rhombus =$\frac {1}{2}$(product of its diagonals)]
	\\
		\solution
	\input{chapters/10/7/2/10/cross.tex}
\item Find the position vector of a point R which divides the line joining two points $\vec{P}$
and $\vec{Q}$ whose position vectors are $\hat{i}+2\hat{j}-\hat{k}$ and $-\hat{i}+\hat{j}+\hat{k}$ respectively, in the
ratio 2 : 1
\begin{enumerate}
    \item  internally
    \item  externally
\end{enumerate}
\solution
		\input{chapters/12/10/2/15/section.tex}
\item Find the position vector of the mid point of the vector joining the points $\vec{P}$(2, 3, 4)
and $\vec{Q}$(4, 1, –2).
\\
\solution
		\input{chapters/12/10/2/16/section.tex}
\item Determine the ratio in which the line $2x+y  - 4=0$ divides the line segment joining the points $\vec{A}(2, - 2)$  and  $\vec{B}(3, 7)$.
\\
\solution
	\input{chapters/10/7/4/1/section.tex}
\item Let $\vec{A}(4, 2), \vec{B}(6, 5)$  and $ \vec{C}(1, 4)$ be the vertices of $\triangle ABC$.
\begin{enumerate}
\item The median from $\vec{A}$ meets $BC$ at $\vec{D}$. Find the coordinates of the point $\vec{D}$.
\item Find the coordinates of the point $\vec{P}$ on $AD$ such that $AP : PD = 2 : 1$.
\item Find the coordinates of points $\vec{Q}$ and $\vec{R}$ on medians $BE$ and $CF$ respectively such that $BQ : QE = 2 : 1$  and  $CR : RF = 2 : 1$.
\item What do you observe?
\item If $\vec{A}, \vec{B}$ and $\vec{C}$  are the vertices of $\triangle ABC$, find the coordinates of the centroid of the triangle.
\end{enumerate}
\solution
	\input{chapters/10/7/4/7/section.tex}
\item Find the slope of a line, which passes through the origin and the mid point of the line segment joining the points $\vec{P}$(0,-4) and $\vec{B}$(8,0).
\label{chapters/11/10/1/5}
\input{chapters/11/10/1/5/matrix.tex}
\item Find the position vector of a point R which divides the line joining two points P and Q whose position vectors are $(2\vec{a}+\vec{b})$ and $(\vec{a}-3\vec{b})$
externally in the ratio 1 : 2. Also, show that P is the mid point of the line segment RQ.\\
	\solution
%		\input{chapters/12/10/5/9/section.tex}

\end{enumerate}


\item Find the area of a rhombus if its vertices are $(3,0), (4,5), (-1,4) \text{ and } (-2,-1)$ taken in order. [$\vec{Hint}$ : Area of rhombus =$\frac {1}{2}$(product of its diagonals)]
	\\
		\solution
	\iffalse
\documentclass[12pt]{article}
\usepackage{graphicx}
%\documentclass[journal,12pt,twocolumn]{IEEEtran}
\usepackage[none]{hyphenat}
\usepackage{graphicx}
\usepackage{listings}
\usepackage[english]{babel}
\usepackage{graphicx}
\usepackage{caption} 
\usepackage{hyperref}
\usepackage{booktabs}
\def\inputGnumericTable{}
\usepackage{color}                                            %%
    \usepackage{array}                                            %%
    \usepackage{longtable}                                        %%
    \usepackage{calc}                                             %%
    \usepackage{multirow}                                         %%
    \usepackage{hhline}                                           %%
    \usepackage{ifthen}
\usepackage{array}
\usepackage{amsmath}   % for having text in math mode
\usepackage{listings}
\lstset{
language=tex,
frame=single, 
breaklines=true
}
  
%Following 2 lines were added to remove the blank page at the beginning
\usepackage{atbegshi}% http://ctan.org/pkg/atbegshi
\AtBeginDocument{\AtBeginShipoutNext{\AtBeginShipoutDiscard}}
%


%New macro definitions
\newcommand{\mydet}[1]{\ensuremath{\begin{vmatrix}#1\end{vmatrix}}}
\providecommand{\brak}[1]{\ensuremath{\left(#1\right)}}
\providecommand{\norm}[1]{\left\lVert#1\right\rVert}
\newcommand{\solution}{\noindent \textbf{Solution: }}
\newcommand{\myvec}[1]{\ensuremath{\begin{pmatrix}#1\end{pmatrix}}}
\let\vec\mathbf

\begin{document}

\begin{center}
\title{\textbf{Coordinate Geometry}}
\date{\vspace{-5ex}} %Not to print date automatically
\maketitle
\end{center}

\setcounter{page}{1}



\begin{enumerate}

\item\textbf{Problem statement :} Find the area of a rhombus of its vertices are $\myvec{3 ,0}$, $\myvec{4 ,5}$, $\myvec{-1 ,4}$ and $\myvec{-2 ,-1}$taken in order

\solution \\
\fi
The input vertices for this problem are given as
	\begin{align}
	\vec{A} = \myvec{
		3\\
		0
		},
	\vec{B} = \myvec{
		4\\
		5
		},
        \vec{C} = \myvec{
		-1\\
		4
		},
        \vec{D} = \myvec{
		-2\\
		-1
		}
	\end{align}
Since		
\begin{align}
 \vec{A-D}= \myvec{3 \\ 0} - \myvec{-2 \\-1}= \myvec{5\\1}
 \\
  \vec{B-A}= \myvec{4 \\ 5} - \myvec{3 \\0}= \myvec{1\\5}
\end{align}
the area of the rhombus is
\begin{align}
                \norm{\myvec{\vec{A-D}}\times \myvec{\vec{B-A}}}=\mydet{5 & 1\\1 & 5} = 24
\end{align}
See Fig. 
\ref{fig:chapters/10/7/2/10/gFig1}.
\begin{figure}[!h]
 \begin{center}
  \includegraphics[width=\columnwidth]{chapters/10/7/2/10/figs/fig.pdf}
 \end{center}
\caption{}
\label{fig:chapters/10/7/2/10/gFig1}
\end{figure}

\item Find the position vector of a point R which divides the line joining two points $\vec{P}$
and $\vec{Q}$ whose position vectors are $\hat{i}+2\hat{j}-\hat{k}$ and $-\hat{i}+\hat{j}+\hat{k}$ respectively, in the
ratio 2 : 1
\begin{enumerate}
    \item  internally
    \item  externally
\end{enumerate}
\solution
		\begin{enumerate}[label=\thesection.\arabic*,ref=\thesection.\theenumi]
\numberwithin{equation}{enumi}
\numberwithin{figure}{enumi}
\numberwithin{table}{enumi}

\item Find the coordinates of the point which divides the join of $(-1,7) \text{ and } (4,-3)$ in the ratio 2:3.
	\\
		\solution
	\input{chapters/10/7/2/1/section.tex}
\item Find the coordinates of the points of trisection of the line segment joining $(4,-1) \text{ and } (-2,3)$.
	\\
		\solution
	\input{chapters/10/7/2/2/section.tex}
\item
	\iffalse
\item To conduct Sports Day activities, in your rectangular shaped school                   
ground ABCD, lines have 
drawn with chalk powder at a                 
distance of 1m each. 100 flower pots have been placed at a distance of 1m 
from each other along AD, as shown 
in Fig. 7.12. Niharika runs $ \frac {1}{4} $th the 
distance AD on the 2nd line and 
posts a green flag. Preet runs $ \frac {1}{5} $th 
the distance AD on the eighth line 
and posts a red flag. What is the 
distance between both the flags? If 
Rashmi has to post a blue flag exactly 
halfway between the line segment 
joining the two flags, where should 
she post her flag?
\begin{figure}[h!]
  \centering
  \includegraphics[width=\columnwidth]{sc.png}
  \caption{}
\label{fig:10/7/12Fig1}
\end{figure}               
\fi
      
\item Find the ratio in which the line segment joining the points $(-3,10) \text{ and } (6,-8)$ $\text{ is divided by } (-1,6)$.
	\\
		\solution
	\input{chapters/10/7/2/4/section.tex}
\item Find the ratio in which the line segment joining $A(1,-5) \text{ and } B(-4,5)$ $\text{is divided by the x-axis}$. Also find the coordinates of the point of division.
\item If $(1,2), (4,y), (x,6), (3,5)$ are the vertices of a parallelogram taken in order, find x and y.
	\\
		\solution
	\input{chapters/10/7/2/6/para1.tex}
\item Find the coordinates of a point A, where AB is the diameter of a circle whose centre is $(2,-3) \text{ and }$ B is $(1,4)$.
	\\
		\solution
	\input{chapters/10/7/2/7/section.tex}
\item If A \text{ and } B are $(-2,-2) \text{ and } (2,-4)$, respectively, find the coordinates of P such that AP= $\frac {3}{7}$AB $\text{ and }$ P lies on the line segment AB.
	\\
		\solution
	\input{chapters/10/7/2/8/section.tex}
\item Find the coordinates of the points which divide the line segment joining $A(-2,2) \text{ and } B(2,8)$ into four equal parts.
	\\
		\solution
	\input{chapters/10/7/2/9/section.tex}
\item Find the area of a rhombus if its vertices are $(3,0), (4,5), (-1,4) \text{ and } (-2,-1)$ taken in order. [$\vec{Hint}$ : Area of rhombus =$\frac {1}{2}$(product of its diagonals)]
	\\
		\solution
	\input{chapters/10/7/2/10/cross.tex}
\item Find the position vector of a point R which divides the line joining two points $\vec{P}$
and $\vec{Q}$ whose position vectors are $\hat{i}+2\hat{j}-\hat{k}$ and $-\hat{i}+\hat{j}+\hat{k}$ respectively, in the
ratio 2 : 1
\begin{enumerate}
    \item  internally
    \item  externally
\end{enumerate}
\solution
		\input{chapters/12/10/2/15/section.tex}
\item Find the position vector of the mid point of the vector joining the points $\vec{P}$(2, 3, 4)
and $\vec{Q}$(4, 1, –2).
\\
\solution
		\input{chapters/12/10/2/16/section.tex}
\item Determine the ratio in which the line $2x+y  - 4=0$ divides the line segment joining the points $\vec{A}(2, - 2)$  and  $\vec{B}(3, 7)$.
\\
\solution
	\input{chapters/10/7/4/1/section.tex}
\item Let $\vec{A}(4, 2), \vec{B}(6, 5)$  and $ \vec{C}(1, 4)$ be the vertices of $\triangle ABC$.
\begin{enumerate}
\item The median from $\vec{A}$ meets $BC$ at $\vec{D}$. Find the coordinates of the point $\vec{D}$.
\item Find the coordinates of the point $\vec{P}$ on $AD$ such that $AP : PD = 2 : 1$.
\item Find the coordinates of points $\vec{Q}$ and $\vec{R}$ on medians $BE$ and $CF$ respectively such that $BQ : QE = 2 : 1$  and  $CR : RF = 2 : 1$.
\item What do you observe?
\item If $\vec{A}, \vec{B}$ and $\vec{C}$  are the vertices of $\triangle ABC$, find the coordinates of the centroid of the triangle.
\end{enumerate}
\solution
	\input{chapters/10/7/4/7/section.tex}
\item Find the slope of a line, which passes through the origin and the mid point of the line segment joining the points $\vec{P}$(0,-4) and $\vec{B}$(8,0).
\label{chapters/11/10/1/5}
\input{chapters/11/10/1/5/matrix.tex}
\item Find the position vector of a point R which divides the line joining two points P and Q whose position vectors are $(2\vec{a}+\vec{b})$ and $(\vec{a}-3\vec{b})$
externally in the ratio 1 : 2. Also, show that P is the mid point of the line segment RQ.\\
	\solution
%		\input{chapters/12/10/5/9/section.tex}

\end{enumerate}


\item Find the position vector of the mid point of the vector joining the points $\vec{P}$(2, 3, 4)
and $\vec{Q}$(4, 1, –2).
\\
\solution
		\begin{enumerate}[label=\thesection.\arabic*,ref=\thesection.\theenumi]
\numberwithin{equation}{enumi}
\numberwithin{figure}{enumi}
\numberwithin{table}{enumi}

\item Find the coordinates of the point which divides the join of $(-1,7) \text{ and } (4,-3)$ in the ratio 2:3.
	\\
		\solution
	\input{chapters/10/7/2/1/section.tex}
\item Find the coordinates of the points of trisection of the line segment joining $(4,-1) \text{ and } (-2,3)$.
	\\
		\solution
	\input{chapters/10/7/2/2/section.tex}
\item
	\iffalse
\item To conduct Sports Day activities, in your rectangular shaped school                   
ground ABCD, lines have 
drawn with chalk powder at a                 
distance of 1m each. 100 flower pots have been placed at a distance of 1m 
from each other along AD, as shown 
in Fig. 7.12. Niharika runs $ \frac {1}{4} $th the 
distance AD on the 2nd line and 
posts a green flag. Preet runs $ \frac {1}{5} $th 
the distance AD on the eighth line 
and posts a red flag. What is the 
distance between both the flags? If 
Rashmi has to post a blue flag exactly 
halfway between the line segment 
joining the two flags, where should 
she post her flag?
\begin{figure}[h!]
  \centering
  \includegraphics[width=\columnwidth]{sc.png}
  \caption{}
\label{fig:10/7/12Fig1}
\end{figure}               
\fi
      
\item Find the ratio in which the line segment joining the points $(-3,10) \text{ and } (6,-8)$ $\text{ is divided by } (-1,6)$.
	\\
		\solution
	\input{chapters/10/7/2/4/section.tex}
\item Find the ratio in which the line segment joining $A(1,-5) \text{ and } B(-4,5)$ $\text{is divided by the x-axis}$. Also find the coordinates of the point of division.
\item If $(1,2), (4,y), (x,6), (3,5)$ are the vertices of a parallelogram taken in order, find x and y.
	\\
		\solution
	\input{chapters/10/7/2/6/para1.tex}
\item Find the coordinates of a point A, where AB is the diameter of a circle whose centre is $(2,-3) \text{ and }$ B is $(1,4)$.
	\\
		\solution
	\input{chapters/10/7/2/7/section.tex}
\item If A \text{ and } B are $(-2,-2) \text{ and } (2,-4)$, respectively, find the coordinates of P such that AP= $\frac {3}{7}$AB $\text{ and }$ P lies on the line segment AB.
	\\
		\solution
	\input{chapters/10/7/2/8/section.tex}
\item Find the coordinates of the points which divide the line segment joining $A(-2,2) \text{ and } B(2,8)$ into four equal parts.
	\\
		\solution
	\input{chapters/10/7/2/9/section.tex}
\item Find the area of a rhombus if its vertices are $(3,0), (4,5), (-1,4) \text{ and } (-2,-1)$ taken in order. [$\vec{Hint}$ : Area of rhombus =$\frac {1}{2}$(product of its diagonals)]
	\\
		\solution
	\input{chapters/10/7/2/10/cross.tex}
\item Find the position vector of a point R which divides the line joining two points $\vec{P}$
and $\vec{Q}$ whose position vectors are $\hat{i}+2\hat{j}-\hat{k}$ and $-\hat{i}+\hat{j}+\hat{k}$ respectively, in the
ratio 2 : 1
\begin{enumerate}
    \item  internally
    \item  externally
\end{enumerate}
\solution
		\input{chapters/12/10/2/15/section.tex}
\item Find the position vector of the mid point of the vector joining the points $\vec{P}$(2, 3, 4)
and $\vec{Q}$(4, 1, –2).
\\
\solution
		\input{chapters/12/10/2/16/section.tex}
\item Determine the ratio in which the line $2x+y  - 4=0$ divides the line segment joining the points $\vec{A}(2, - 2)$  and  $\vec{B}(3, 7)$.
\\
\solution
	\input{chapters/10/7/4/1/section.tex}
\item Let $\vec{A}(4, 2), \vec{B}(6, 5)$  and $ \vec{C}(1, 4)$ be the vertices of $\triangle ABC$.
\begin{enumerate}
\item The median from $\vec{A}$ meets $BC$ at $\vec{D}$. Find the coordinates of the point $\vec{D}$.
\item Find the coordinates of the point $\vec{P}$ on $AD$ such that $AP : PD = 2 : 1$.
\item Find the coordinates of points $\vec{Q}$ and $\vec{R}$ on medians $BE$ and $CF$ respectively such that $BQ : QE = 2 : 1$  and  $CR : RF = 2 : 1$.
\item What do you observe?
\item If $\vec{A}, \vec{B}$ and $\vec{C}$  are the vertices of $\triangle ABC$, find the coordinates of the centroid of the triangle.
\end{enumerate}
\solution
	\input{chapters/10/7/4/7/section.tex}
\item Find the slope of a line, which passes through the origin and the mid point of the line segment joining the points $\vec{P}$(0,-4) and $\vec{B}$(8,0).
\label{chapters/11/10/1/5}
\input{chapters/11/10/1/5/matrix.tex}
\item Find the position vector of a point R which divides the line joining two points P and Q whose position vectors are $(2\vec{a}+\vec{b})$ and $(\vec{a}-3\vec{b})$
externally in the ratio 1 : 2. Also, show that P is the mid point of the line segment RQ.\\
	\solution
%		\input{chapters/12/10/5/9/section.tex}

\end{enumerate}


\item Determine the ratio in which the line $2x+y  - 4=0$ divides the line segment joining the points $\vec{A}(2, - 2)$  and  $\vec{B}(3, 7)$.
\\
\solution
	\iffalse
\documentclass[journal,12pt,twocolumn]{IEEEtran}
\usepackage{graphicx}
\graphicspath{{./chapters/10/7/4/1/figs/}}{}
\usepackage{amsmath,amssymb,amsfonts,amsthm}
\newcommand{\myvec}[1]{\ensuremath{\begin{pmatrix}#1\end{pmatrix}}}
\providecommand{\norm}[1]{\lVert#1\rVert}
\usepackage{listings}
\usepackage{watermark}
\usepackage{titlesec}
\usepackage{caption}
\let\vec\mathbf
\lstset{
frame=single, 
breaklines=true,
columns=fullflexible
}
\thiswatermark{\centering \put(0,-105.0){\includegraphics[scale=0.15]{/sdcard/IITH/vector/vectpr-4/chapters/10/7/4/1/figs/logo.png}} }
\title{\mytitle}
\title{
Assignment - Vector-4
}
\author{Surajit Sarkar}
\begin{document}
\maketitle
%\tableofcontents
\bigskip
\section{\textbf{Problem}}
Determine the ratio in which the line 2x+y–4=0 divides the line segment joining the points A(2,–2) and B(3,7).
\section{\textbf{Solution}}
\begin{table}[h]
    \centering
    \begin{tabular}{|c|c|}
       \hline
       \textbf{Symbol}&\textbf{Value}  \\
       \hline
	    $\vec{A}$ & $\myvec{2\\-2}$\\
        \hline
	    $\vec{B}$ & $\myvec{3\\7}$\\
        \hline
	    c&$4$\\
        \hline
       $\vec{n}$ & $\myvec{2\\1}$\\
       \hline
    \end{tabular}
    \caption{Parameters}
    \label{tab:my_label}
\end{table}
Given equation
\fi
The given equation can be expressed as
\begin{align}
    \myvec{2&1}\vec{x}&=4\\
\end{align}
Using section formula, the point of division 
\begin{align}
    \vec{P} = \frac{k\vec{B+A}}{k+1}
\end{align}
which upon substitution in the equation of a line yields
\begin{align}
    \implies\vec{n}^{\top}\myvec{\frac{k\vec{B+A}}{k+1}}&=c\\
    \implies k&=\frac{c-\vec{n}^{\top}\vec{A}}{\vec{n}^{\top}\vec{B}-c}\\
\end{align}
upon simplification.  Substituting numerical values, 
\begin{align}
    k=\frac{2}{9}
\end{align}
See Fig. 
\ref{fig:chapters/10/7/4/1vec}.
\begin{figure}[!h]
\centering
\includegraphics[width=\columnwidth]{chapters/10/7/4/1/figs/vec.pdf}
\caption{}
\label{fig:chapters/10/7/4/1vec}
\end{figure}


\item Let $\vec{A}(4, 2), \vec{B}(6, 5)$  and $ \vec{C}(1, 4)$ be the vertices of $\triangle ABC$.
\begin{enumerate}
\item The median from $\vec{A}$ meets $BC$ at $\vec{D}$. Find the coordinates of the point $\vec{D}$.
\item Find the coordinates of the point $\vec{P}$ on $AD$ such that $AP : PD = 2 : 1$.
\item Find the coordinates of points $\vec{Q}$ and $\vec{R}$ on medians $BE$ and $CF$ respectively such that $BQ : QE = 2 : 1$  and  $CR : RF = 2 : 1$.
\item What do you observe?
\item If $\vec{A}, \vec{B}$ and $\vec{C}$  are the vertices of $\triangle ABC$, find the coordinates of the centroid of the triangle.
\end{enumerate}
\solution
	\iffalse
\documentclass[12pt]{article}
\usepackage{graphicx}
\usepackage[none]{hyphenat}
\usepackage{graphicx}
\usepackage{listings}
\usepackage[english]{babel}
\usepackage{graphicx}
\usepackage{caption} 
\usepackage{booktabs}
\usepackage{array}
\usepackage{amssymb} % for \because
\usepackage{amsmath}   % for having text in math mode
\usepackage{extarrows} % for Row operations arrows
\usepackage{listings}
\usepackage[utf8]{inputenc}
\lstset{
  frame=single,
  breaklines=true
}
\usepackage{hyperref}
  
%Following 2 lines were added to remove the blank page at the beginning
\usepackage{atbegshi}% http://ctan.org/pkg/atbegshi
\AtBeginDocument{\AtBeginShipoutNext{\AtBeginShipoutDiscard}}


%New macro definitions
\newcommand{\mydet}[1]{\ensuremath{\begin{vmatrix}#1\end{vmatrix}}}
\providecommand{\brak}[1]{\ensuremath{\left(#1\right)}}
\newcommand{\solution}{\noindent \textbf{Solution: }}
\newcommand{\myvec}[1]{\ensuremath{\begin{pmatrix}#1\end{pmatrix}}}
\providecommand{\norm}[1]{\left\lVert#1\right\rVert}
\providecommand{\abs}[1]{\left\vert#1\right\vert}
\let\vec\mathbf

\begin{document}

\begin{center}
\title{\textbf{VECTORS}}
\date{\vspace{-5ex}} %Not to print date automatically
\maketitle
\end{center}

\section{10$^{th}$ Maths - EXERCISE-7.4}

Let A(4, 2), B(6, 5) and C(1, 4) be the vertices of $\triangle ABC$
\begin{enumerate}
\item The median from A meets BC at D. Find the coordinates of the point D.
\item Find the coordinates of the point P on AD such that $AP : PD = 2 : 1$
\item Find the coordinates of points Q and R on medians BE and CF respectively such
that $BQ : QE = 2 : 1 \text{and} CR : RF = 2 : 1.$
\item What do yo observe?
\item If $A(x_1, y_1), B(x_2, y_2) \text{and} C(x_3, y_3)$ are the vertices of $\triangle ABC$, find the coordinates of the centroid of the triangle.
\end{enumerate}

Given points are
\begin{align}
\vec{A}=\myvec{4\\ 2} ,
\vec{B}=\myvec{6\\ 5} ,
\vec{C}=\myvec{1\\ 4}
\end{align}
\fi

\begin{enumerate}
\item 
\begin{align}
\vec{D}&=\frac{\vec{B}+\vec{C}}{2}\\
&=\myvec{\frac{7}{2}\\[2pt] \frac{9}{2}}\\
\vec{E}&=\frac{\vec{A}+\vec{C}}{2}\\
&=\myvec{\frac{5}{2}\\ 3}\\
\vec{F}&=\frac{\vec{A}+\vec{B}}{2}\\
&=\myvec{5\\ \frac{7}{2}}
\end{align}

\item 
	For
$n=2$,
\begin{align}
\vec{P}&=\frac{1}{1+n}\brak{\myvec{\vec{A}+n\vec{D}}}\\
&=\frac{1}{3}\myvec{11\\11}
\end{align}

\item 
\begin{align}
\vec{Q}&=\frac{1}{1+n}\brak{\myvec{\vec{B}+n\vec{E}}}\\
&=\frac{1}{3}\myvec{11\\11}\\
\vec{R}&=\frac{1}{1+n}\brak{\myvec{\vec{C}+n\vec{F}}}\\
&=\frac{1}{3}\myvec{11\\11}\\
\end{align}

\item 
 $\vec{P},\vec{Q},\vec{R}$ are the same point.
   
\item 
\begin{align}
\vec{G}&=\frac{\vec{D}+\vec{E}+\vec{F}}{3}\\
&=\frac{1}{3}\myvec{11\\11}\\
\end{align} 
\end{enumerate}
See Fig.  
  \ref{fig:chapters/10/7/4/7/Figure}.
\begin{figure}[h!]
\centering
\includegraphics[width=\columnwidth]{chapters/10/7/4/7/figs/dj.pdf}
\caption{}
  \label{fig:chapters/10/7/4/7/Figure}
\end{figure}

\item Find the slope of a line, which passes through the origin and the mid point of the line segment joining the points $\vec{P}$(0,-4) and $\vec{B}$(8,0).
\label{chapters/11/10/1/5}
\iffalse
\documentclass[journal,12pt,twocolumn]{IEEEtran}
\usepackage{graphicx}
\graphicspath{{./figs/}}{}
\usepackage{amsmath,amssymb,amsfonts,amsthm}
\newcommand{\myvec}[1]{\ensuremath{\begin{pmatrix}#1\end{pmatrix}}}

\let\vec\mathbf

\title{
Matrix-Lines
}
\author{Jyothsna Paluchuri-FWC22059\\}
\begin{document}
\maketitle
\tableofcontents
\bigskip
\section{Problem Statement}
\fi
	\begin{figure}[!ht]
		\centering
 \includegraphics[width=\columnwidth]{chapters/11/10/1/5/figs/line.png}
		\caption{}
		\label{fig:11/10/1/5}
  	\end{figure}
	\\
	\solution
\iffalse
\section{Construction}
\begin{figure}[h]
    \centering
\includegraphics[width=\columnwidth]{line.png}
    \caption{Equation of the slope}
    \label{fig:my_label}
\end{figure}
\vspace{2cm}
\begin{table}[h]
    \centering
    \begin{tabular}{|c|c|c|c|}
       \hline
       \textbf{Symbol}&\textbf{Value}&\textbf{Description}  \\
       \hline
	    $\vec{P}$ & $\myvec{
		    0\\
		    -4}$
	    & Point on Y-axis\\
        \hline
	    $\vec{B}$ & $\myvec{8\\0}$
 & Point on X-axis\\
        \hline
	    $\vec{0}$ & $\myvec{0\\0}$
 & Origin\\
        \hline
    \end{tabular}
    \caption{Parameters}
    \label{tab:my_label}
\end{table}


\section{Solution}
Given that resultant line passes through origin and mid point of the line segment joining point P(0,-4) and B(8,0) \\
\\
\\
given ${\vec{P}}$=$\myvec{
  0\\
  -4}$
 , ${\vec{B}}$=$\myvec{
  8\\
  0}$
  
 \fi 
The mid point of $PB$ is
\begin{align}
\vec{M} &=\frac{1}{2}(\vec{P}+\vec{B})
	= \myvec{4 \\ -2}  
\end{align}
The direction vector of line joining $\vec{O}, \vec{M}$ is 
\begin{align}
\vec{m}&=\vec{O}-\vec{M}
 = -\vec{M}
\end{align}
which can be expressed as
\begin{align}
	\myvec{1 \\ -\frac{1}{2}}
\end{align}
Thus the slope is
\begin{align}
	m = -\frac{1}{2}
\end{align}
\iffalse
\textbf{The direction vector of a line expressed as}
\begin{align}
\implies\vec{m} &= \begin{pmatrix}1 \\ m \\ \end{pmatrix}
\end{align}

\textbf{By solving equation (5) and (6),we get the slope of $\vec{O}$ $\vec{M}$ line}
\begin{align}
        \boxed{m=-0.5}
 \end{align}

\section{Software}
Download the following code using,
\begin{table}[h]
    \centering
    \begin{tabular}{|c|}
    \hline \\
   https://github.com/jyothsna777/jyothsna-fwc.git  \\
         \\
\hline
    \end{tabular}
\end{table}
\\
and execute the code by using command
\begin{center}
\textbf{Python3 lines.py}\\
\end{center}

\section{Conclusion}
Hence the slope of line $\vec{O}$ $\vec{M}$ lineis $\vec{m}$=-0.5

\end{document}
\fi

\item Find the position vector of a point R which divides the line joining two points P and Q whose position vectors are $(2\vec{a}+\vec{b})$ and $(\vec{a}-3\vec{b})$
externally in the ratio 1 : 2. Also, show that P is the mid point of the line segment RQ.\\
	\solution
%		\begin{enumerate}[label=\thesection.\arabic*,ref=\thesection.\theenumi]
\numberwithin{equation}{enumi}
\numberwithin{figure}{enumi}
\numberwithin{table}{enumi}

\item Find the coordinates of the point which divides the join of $(-1,7) \text{ and } (4,-3)$ in the ratio 2:3.
	\\
		\solution
	\input{chapters/10/7/2/1/section.tex}
\item Find the coordinates of the points of trisection of the line segment joining $(4,-1) \text{ and } (-2,3)$.
	\\
		\solution
	\input{chapters/10/7/2/2/section.tex}
\item
	\iffalse
\item To conduct Sports Day activities, in your rectangular shaped school                   
ground ABCD, lines have 
drawn with chalk powder at a                 
distance of 1m each. 100 flower pots have been placed at a distance of 1m 
from each other along AD, as shown 
in Fig. 7.12. Niharika runs $ \frac {1}{4} $th the 
distance AD on the 2nd line and 
posts a green flag. Preet runs $ \frac {1}{5} $th 
the distance AD on the eighth line 
and posts a red flag. What is the 
distance between both the flags? If 
Rashmi has to post a blue flag exactly 
halfway between the line segment 
joining the two flags, where should 
she post her flag?
\begin{figure}[h!]
  \centering
  \includegraphics[width=\columnwidth]{sc.png}
  \caption{}
\label{fig:10/7/12Fig1}
\end{figure}               
\fi
      
\item Find the ratio in which the line segment joining the points $(-3,10) \text{ and } (6,-8)$ $\text{ is divided by } (-1,6)$.
	\\
		\solution
	\input{chapters/10/7/2/4/section.tex}
\item Find the ratio in which the line segment joining $A(1,-5) \text{ and } B(-4,5)$ $\text{is divided by the x-axis}$. Also find the coordinates of the point of division.
\item If $(1,2), (4,y), (x,6), (3,5)$ are the vertices of a parallelogram taken in order, find x and y.
	\\
		\solution
	\input{chapters/10/7/2/6/para1.tex}
\item Find the coordinates of a point A, where AB is the diameter of a circle whose centre is $(2,-3) \text{ and }$ B is $(1,4)$.
	\\
		\solution
	\input{chapters/10/7/2/7/section.tex}
\item If A \text{ and } B are $(-2,-2) \text{ and } (2,-4)$, respectively, find the coordinates of P such that AP= $\frac {3}{7}$AB $\text{ and }$ P lies on the line segment AB.
	\\
		\solution
	\input{chapters/10/7/2/8/section.tex}
\item Find the coordinates of the points which divide the line segment joining $A(-2,2) \text{ and } B(2,8)$ into four equal parts.
	\\
		\solution
	\input{chapters/10/7/2/9/section.tex}
\item Find the area of a rhombus if its vertices are $(3,0), (4,5), (-1,4) \text{ and } (-2,-1)$ taken in order. [$\vec{Hint}$ : Area of rhombus =$\frac {1}{2}$(product of its diagonals)]
	\\
		\solution
	\input{chapters/10/7/2/10/cross.tex}
\item Find the position vector of a point R which divides the line joining two points $\vec{P}$
and $\vec{Q}$ whose position vectors are $\hat{i}+2\hat{j}-\hat{k}$ and $-\hat{i}+\hat{j}+\hat{k}$ respectively, in the
ratio 2 : 1
\begin{enumerate}
    \item  internally
    \item  externally
\end{enumerate}
\solution
		\input{chapters/12/10/2/15/section.tex}
\item Find the position vector of the mid point of the vector joining the points $\vec{P}$(2, 3, 4)
and $\vec{Q}$(4, 1, –2).
\\
\solution
		\input{chapters/12/10/2/16/section.tex}
\item Determine the ratio in which the line $2x+y  - 4=0$ divides the line segment joining the points $\vec{A}(2, - 2)$  and  $\vec{B}(3, 7)$.
\\
\solution
	\input{chapters/10/7/4/1/section.tex}
\item Let $\vec{A}(4, 2), \vec{B}(6, 5)$  and $ \vec{C}(1, 4)$ be the vertices of $\triangle ABC$.
\begin{enumerate}
\item The median from $\vec{A}$ meets $BC$ at $\vec{D}$. Find the coordinates of the point $\vec{D}$.
\item Find the coordinates of the point $\vec{P}$ on $AD$ such that $AP : PD = 2 : 1$.
\item Find the coordinates of points $\vec{Q}$ and $\vec{R}$ on medians $BE$ and $CF$ respectively such that $BQ : QE = 2 : 1$  and  $CR : RF = 2 : 1$.
\item What do you observe?
\item If $\vec{A}, \vec{B}$ and $\vec{C}$  are the vertices of $\triangle ABC$, find the coordinates of the centroid of the triangle.
\end{enumerate}
\solution
	\input{chapters/10/7/4/7/section.tex}
\item Find the slope of a line, which passes through the origin and the mid point of the line segment joining the points $\vec{P}$(0,-4) and $\vec{B}$(8,0).
\label{chapters/11/10/1/5}
\input{chapters/11/10/1/5/matrix.tex}
\item Find the position vector of a point R which divides the line joining two points P and Q whose position vectors are $(2\vec{a}+\vec{b})$ and $(\vec{a}-3\vec{b})$
externally in the ratio 1 : 2. Also, show that P is the mid point of the line segment RQ.\\
	\solution
%		\input{chapters/12/10/5/9/section.tex}

\end{enumerate}



\end{enumerate}


\item Find the position vector of the mid point of the vector joining the points $\vec{P}$(2, 3, 4)
and $\vec{Q}$(4, 1, –2).
\\
\solution
		\begin{enumerate}[label=\thesection.\arabic*,ref=\thesection.\theenumi]
\numberwithin{equation}{enumi}
\numberwithin{figure}{enumi}
\numberwithin{table}{enumi}

\item Find the coordinates of the point which divides the join of $(-1,7) \text{ and } (4,-3)$ in the ratio 2:3.
	\\
		\solution
	\iffalse
\documentclass[12pt]{article}
\usepackage{graphicx}
\usepackage{amsmath}
\usepackage{mathtools}
\usepackage{gensymb}

\newcommand{\mydet}[1]{\ensuremath{\begin{vmatrix}#1\end{vmatrix}}}
\providecommand{\brak}[1]{\ensuremath{\left(#1\right)}}
\providecommand{\norm}[1]{\left\lVert#1\right\rVert}
\newcommand{\solution}{\noindent \textbf{Solution: }}
\newcommand{\myvec}[1]{\ensuremath{\begin{pmatrix}#1\end{pmatrix}}}
\let\vec\mathbf

\begin{document}
\begin{center}
\textbf\large{CHAPTER-7 \\ COORDINATE GEOMETRY}
\end{center}
\section*{Excercise 7.2}

1. Find the coordinates of the point which divides the join $\vec(-1,7) \text{ and } \vec(4,-3)$ in the ratio 2:3 :
\\
\\
\solution\\		
\fi
The coordinates and ratio are given as
\begin{align}
\vec{P}=\myvec{-1\\7\\},
\vec{Q}=\myvec{4\\-3\\},
n=\frac{3}{2}
\end{align}
Using section formula
\begin{align}
\vec{R}&=\frac{\vec{Q}+n\vec{P}}{1+n}\\
&=\frac{1}{1+\frac{3}{2}}  \myvec{\myvec{
4\\
-3\\
}
  +
   \frac{3}{2}\myvec{
-1\\
7\\
}}\\
&=\myvec{
1\\
3
}
\end{align}
See Fig. 
\ref{fig:chapters/10/7/2/1/Fig}
\begin{figure}[!h]
\begin{center}
   \includegraphics[width=\columnwidth]{chapters/10/7/2/1/figs/linefig.png}
\end{center}
\caption{}
\label{fig:chapters/10/7/2/1/Fig}
\end{figure}


\item Find the coordinates of the points of trisection of the line segment joining $(4,-1) \text{ and } (-2,3)$.
	\\
		\solution
	\begin{enumerate}[label=\thesection.\arabic*,ref=\thesection.\theenumi]
\numberwithin{equation}{enumi}
\numberwithin{figure}{enumi}
\numberwithin{table}{enumi}

\item Find the coordinates of the point which divides the join of $(-1,7) \text{ and } (4,-3)$ in the ratio 2:3.
	\\
		\solution
	\input{chapters/10/7/2/1/section.tex}
\item Find the coordinates of the points of trisection of the line segment joining $(4,-1) \text{ and } (-2,3)$.
	\\
		\solution
	\input{chapters/10/7/2/2/section.tex}
\item
	\iffalse
\item To conduct Sports Day activities, in your rectangular shaped school                   
ground ABCD, lines have 
drawn with chalk powder at a                 
distance of 1m each. 100 flower pots have been placed at a distance of 1m 
from each other along AD, as shown 
in Fig. 7.12. Niharika runs $ \frac {1}{4} $th the 
distance AD on the 2nd line and 
posts a green flag. Preet runs $ \frac {1}{5} $th 
the distance AD on the eighth line 
and posts a red flag. What is the 
distance between both the flags? If 
Rashmi has to post a blue flag exactly 
halfway between the line segment 
joining the two flags, where should 
she post her flag?
\begin{figure}[h!]
  \centering
  \includegraphics[width=\columnwidth]{sc.png}
  \caption{}
\label{fig:10/7/12Fig1}
\end{figure}               
\fi
      
\item Find the ratio in which the line segment joining the points $(-3,10) \text{ and } (6,-8)$ $\text{ is divided by } (-1,6)$.
	\\
		\solution
	\input{chapters/10/7/2/4/section.tex}
\item Find the ratio in which the line segment joining $A(1,-5) \text{ and } B(-4,5)$ $\text{is divided by the x-axis}$. Also find the coordinates of the point of division.
\item If $(1,2), (4,y), (x,6), (3,5)$ are the vertices of a parallelogram taken in order, find x and y.
	\\
		\solution
	\input{chapters/10/7/2/6/para1.tex}
\item Find the coordinates of a point A, where AB is the diameter of a circle whose centre is $(2,-3) \text{ and }$ B is $(1,4)$.
	\\
		\solution
	\input{chapters/10/7/2/7/section.tex}
\item If A \text{ and } B are $(-2,-2) \text{ and } (2,-4)$, respectively, find the coordinates of P such that AP= $\frac {3}{7}$AB $\text{ and }$ P lies on the line segment AB.
	\\
		\solution
	\input{chapters/10/7/2/8/section.tex}
\item Find the coordinates of the points which divide the line segment joining $A(-2,2) \text{ and } B(2,8)$ into four equal parts.
	\\
		\solution
	\input{chapters/10/7/2/9/section.tex}
\item Find the area of a rhombus if its vertices are $(3,0), (4,5), (-1,4) \text{ and } (-2,-1)$ taken in order. [$\vec{Hint}$ : Area of rhombus =$\frac {1}{2}$(product of its diagonals)]
	\\
		\solution
	\input{chapters/10/7/2/10/cross.tex}
\item Find the position vector of a point R which divides the line joining two points $\vec{P}$
and $\vec{Q}$ whose position vectors are $\hat{i}+2\hat{j}-\hat{k}$ and $-\hat{i}+\hat{j}+\hat{k}$ respectively, in the
ratio 2 : 1
\begin{enumerate}
    \item  internally
    \item  externally
\end{enumerate}
\solution
		\input{chapters/12/10/2/15/section.tex}
\item Find the position vector of the mid point of the vector joining the points $\vec{P}$(2, 3, 4)
and $\vec{Q}$(4, 1, –2).
\\
\solution
		\input{chapters/12/10/2/16/section.tex}
\item Determine the ratio in which the line $2x+y  - 4=0$ divides the line segment joining the points $\vec{A}(2, - 2)$  and  $\vec{B}(3, 7)$.
\\
\solution
	\input{chapters/10/7/4/1/section.tex}
\item Let $\vec{A}(4, 2), \vec{B}(6, 5)$  and $ \vec{C}(1, 4)$ be the vertices of $\triangle ABC$.
\begin{enumerate}
\item The median from $\vec{A}$ meets $BC$ at $\vec{D}$. Find the coordinates of the point $\vec{D}$.
\item Find the coordinates of the point $\vec{P}$ on $AD$ such that $AP : PD = 2 : 1$.
\item Find the coordinates of points $\vec{Q}$ and $\vec{R}$ on medians $BE$ and $CF$ respectively such that $BQ : QE = 2 : 1$  and  $CR : RF = 2 : 1$.
\item What do you observe?
\item If $\vec{A}, \vec{B}$ and $\vec{C}$  are the vertices of $\triangle ABC$, find the coordinates of the centroid of the triangle.
\end{enumerate}
\solution
	\input{chapters/10/7/4/7/section.tex}
\item Find the slope of a line, which passes through the origin and the mid point of the line segment joining the points $\vec{P}$(0,-4) and $\vec{B}$(8,0).
\label{chapters/11/10/1/5}
\input{chapters/11/10/1/5/matrix.tex}
\item Find the position vector of a point R which divides the line joining two points P and Q whose position vectors are $(2\vec{a}+\vec{b})$ and $(\vec{a}-3\vec{b})$
externally in the ratio 1 : 2. Also, show that P is the mid point of the line segment RQ.\\
	\solution
%		\input{chapters/12/10/5/9/section.tex}

\end{enumerate}


\item
	\iffalse
\item To conduct Sports Day activities, in your rectangular shaped school                   
ground ABCD, lines have 
drawn with chalk powder at a                 
distance of 1m each. 100 flower pots have been placed at a distance of 1m 
from each other along AD, as shown 
in Fig. 7.12. Niharika runs $ \frac {1}{4} $th the 
distance AD on the 2nd line and 
posts a green flag. Preet runs $ \frac {1}{5} $th 
the distance AD on the eighth line 
and posts a red flag. What is the 
distance between both the flags? If 
Rashmi has to post a blue flag exactly 
halfway between the line segment 
joining the two flags, where should 
she post her flag?
\begin{figure}[h!]
  \centering
  \includegraphics[width=\columnwidth]{sc.png}
  \caption{}
\label{fig:10/7/12Fig1}
\end{figure}               
\fi
      
\item Find the ratio in which the line segment joining the points $(-3,10) \text{ and } (6,-8)$ $\text{ is divided by } (-1,6)$.
	\\
		\solution
	\iffalse
\documentclass[12pt]{article}
\usepackage{graphicx}
%\documentclass[journal,12pt,twocolumn]{IEEEtran}
\usepackage[none]{hyphenat}
\usepackage{graphicx}
\usepackage{listings}
\usepackage[english]{babel}
\usepackage{graphicx}
\usepackage{caption} 
\usepackage{hyperref}
\usepackage{booktabs}
\def\inputGnumericTable{}
\usepackage{color}                                            %%
    \usepackage{array}                                            %%
    \usepackage{longtable}                                        %%
    \usepackage{calc}                                             %%
    \usepackage{multirow}                                         %%
    \usepackage{hhline}                                           %%
    \usepackage{ifthen}
\usepackage{array}
\usepackage{amsmath}   % for having text in math mode
\usepackage{listings}
\lstset{
language=tex,
frame=single, 
breaklines=true
}
  
%Following 2 lines were added to remove the blank page at the beginning
\usepackage{atbegshi}% http://ctan.org/pkg/atbegshi
\AtBeginDocument{\AtBeginShipoutNext{\AtBeginShipoutDiscard}}
%
%New macro definitions
\newcommand{\mydet}[1]{\ensuremath{\begin{vmatrix}#1\end{vmatrix}}}
\providecommand{\brak}[1]{\ensuremath{\left(#1\right)}}
\providecommand{\norm}[1]{\left\lVert#1\right\rVert}
\newcommand{\solution}{\noindent \textbf{Solution: }}
\newcommand{\myvec}[1]{\ensuremath{\begin{pmatrix}#1\end{pmatrix}}}
\let\vec\mathbf
\begin{document}
\begin{center}
\title{\textbf{Coordinate Geometry}}
\date{\vspace{-5ex}} %Not to print date automatically
\maketitle
\end{center}
\setcounter{page}{1}
\section*{10$^{th}$ Maths - Chapter 7}
This is Problem-4 from Exercise 7.2
\begin{enumerate}
\item Find the ratio in which the line segement joining the points $\myvec{-3 \\ 10}$ and $\myvec{6\\-8}$ is divided by $\myvec{-1\\6}$.\\
\solution \\
\fi
		The input parameters for this problem are available in Table \eqref{tab:10/7/2/4-1}.
\begin{table}[ht!]
\input{chapters/10/7/2/4/tables/table.tex}
\caption{}
\label{tab:10/7/2/4-1} 
\end{table}
Using section formula,
\begin{align}
         \vec{R} &=\frac{\vec{Q}+n\vec{P}}{1+n}\label{eq:chapters/10/7/2/4/1}
\end{align}
Substituting the values of $\vec{P},\vec{Q}$ and $\vec{R}$ in \eqref{eq:chapters/10/7/2/4/1}
\begin{align}
         \myvec{-1\\6} &=\frac{{\myvec{-3\\10}+n\myvec{6\\-8}}}{1+n}\\
 &=\frac{1}{1+n}\brak{{\myvec{-3\\10}+n\myvec{6\\-8}}} \\
 &=\frac{1}{1+n}\myvec{-3+6n\\10-8n} \label{eq:chapters/10/7/2/4/4}
\end{align}
Simplifying \eqref{eq:chapters/10/7/2/4/4} yeilds,
\begin{align}
          -1 &=\frac{-3+6n}{1+n}\\
\implies          n &=\frac{2}{7}
\end{align}
Also,
\begin{align}
          6 &=\frac{10-8n}{1+n}\\
    \implies      n &=\frac{2}{7}
\end{align}
Hence the desired ratio is $\dfrac{2}{7}$.  
\begin{figure}[!h]
 \begin{center}
  \includegraphics[width=\columnwidth]{chapters/10/7/2/4/figs/fig.png}
 \end{center}
\caption{}
\label{fig:10/7/2/4Fig1}
\end{figure}

\item Find the ratio in which the line segment joining $A(1,-5) \text{ and } B(-4,5)$ $\text{is divided by the x-axis}$. Also find the coordinates of the point of division.
\item If $(1,2), (4,y), (x,6), (3,5)$ are the vertices of a parallelogram taken in order, find x and y.
	\\
		\solution
	\iffalse
\documentclass[12pt]{article}
\usepackage{graphicx}
%\documentclass[journal,12pt,twocolumn]{IEEEtran}
\def\inputGnumericTable{}
\usepackage{color}                                            %%
    \usepackage{array}                                            %%
    \usepackage{longtable}                                        %%
    \usepackage{calc}                                             %%
    \usepackage{multirow}                                         %%
    \usepackage{hhline}                                           %%
    \usepackage{ifthen}
\usepackage[none]{hyphenat}
\usepackage{graphicx}
\usepackage{listings}
\usepackage[english]{babel}
\usepackage{graphicx}
\usepackage{caption} 
\usepackage{hyperref}
\usepackage{booktabs}
\usepackage{array}
\usepackage{amsmath}   % for having text in math mode
\usepackage{listings}
\lstset{
  frame=single,
  breaklines=true
}
  
%Following 2 lines were added to remove the blank page at the beginning
\usepackage{atbegshi}% http://ctan.org/pkg/atbegshi
\AtBeginDocument{\AtBeginShipoutNext{\AtBeginShipoutDiscard}}
%


%New macro definitions
\newcommand{\mydet}[1]{\ensuremath{\begin{vmatrix}#1\end{vmatrix}}}
\providecommand{\brak}[1]{\ensuremath{\left(#1\right)}}
\providecommand{\norm}[1]{\left\lVert#1\right\rVert}
\newcommand{\solution}{\noindent \textbf{Solution: }}
\newcommand{\myvec}[1]{\ensuremath{\begin{pmatrix}#1\end{pmatrix}}}
\let\vec\mathbf

\begin{document}

\begin{center}
\title{\textbf{Properties of Parallelegram}}
\date{\vspace{-5ex}} %Not to print date automatically
\maketitle
\end{center}

\setcounter{page}{1}

\section{10$^{th}$ Maths - Chapter 7}

This is Problem-6 from Exercise 7.2

\begin{enumerate}
\item If $\vec{A}(1, 2),\vec{B}(4, x),\vec{C}(y, 6) \text{and } \vec{D}(3, 5)$ are the vertices of a parallelogram taken in order,find x and y.
\end{enumerate}
\fi

The input parameters for this problem are available in
\ref{table:chapters/10/7/2/6/tables/}.	
\begin{table}[!ht]
	\centering
	\input{chapters/10/7/2/6/tables/table.tex}
\caption{}
\label{table:chapters/10/7/2/6/tables/}	
\end{table}
From the given information,
\begin{align}
  \label{eq:chapters/10/7/2/6/tables/det2f}
	\vec{B}-\vec{A} &= \myvec{4 \\y } - \myvec{1 \\2 }  = \myvec{3 \\y-2 }\\
	\vec{C}-\vec{D} &= \myvec{x \\6 } - \myvec{3 \\5 }  = \myvec{x-3 \\1}
\end{align}
Since $ABCD$ is a parallellogram,
\begin{align}
	\myvec{3\\y-2}&=\myvec{x-3\\1}\\
	\implies x&=6 ,y=3
\end{align}
Fig. \ref{fig:chapters/10/7/2/6/Fig3}
provides a verification.
\begin{figure}[h!]
	\begin{center}
  \includegraphics[width=\columnwidth]{chapters/10/7/2/6/figs/para.pdf}
	\end{center}
\caption{}
\label{fig:chapters/10/7/2/6/Fig3}
\end{figure}


\item Find the coordinates of a point A, where AB is the diameter of a circle whose centre is $(2,-3) \text{ and }$ B is $(1,4)$.
	\\
		\solution
	\iffalse
\documentclass[12pt]{article}
\usepackage{graphicx}
\usepackage{amsmath}
\usepackage{mathtools}
\usepackage{gensymb}

\newcommand{\mydet}[1]{\ensuremath{\begin{vmatrix}#1\end{vmatrix}}}
\providecommand{\brak}[1]{\ensuremath{\left(#1\right)}}
\providecommand{\norm}[1]{\left\lVert#1\right\rVert}
\newcommand{\solution}{\noindent \textbf{Solution: }}
\newcommand{\myvec}[1]{\ensuremath{\begin{pmatrix}#1\end{pmatrix}}}
\let\vec\mathbf

\begin{document}
\begin{center}
\section*{CHAPTER 7 - COORDINATE GEOMETRY}

\end{center}
\section*{Excercise 7.2}

Q7.Find the coordinates of point $\vec{A}$, where AB is the diameter of a circle where the center is (2,-3) and $\vec{B}$ is the point (1,4):

\solution
\begin{enumerate}
\item The coordinates $\vec{B}$ and center $\vec{C}$ are given, where:
	\fi
	Let
	\begin{align}
	\vec{B} = \myvec{
		1\\
	    4\\
		},
	\vec{C} = \myvec{
	    2\\
	   -3\\
		}
	\end{align}
	\iffalse
Let us assume the coordinates of $\vec{A}$. Now, $\vec{C}$ is the center which is midpoint of line AB and $\vec{B}$ is one of the coordinate of diameter AB of a circle.
	\fi	
Hence,	
	\begin{align}
	\vec{C} &= \frac{\vec{A+B}}{2} \\
\implies	2\vec{C} &= \vec{A}+\vec{B} \\
		\text{or, }	\vec{A} &= 2\vec{C}-\vec{B} \\
	 &= \myvec{3\\-10\\}	
	\end{align}       
	See Fig. 
\ref{fig:chapters/10/7/2/7Fig}.
\begin{figure}[!h]
\begin{center}	
	\includegraphics[width=\columnwidth]{chapters/10/7/2/7/figs/Vector1.png}
\end{center}
\caption{}
\label{fig:chapters/10/7/2/7Fig}
\end{figure}
	

\item If A \text{ and } B are $(-2,-2) \text{ and } (2,-4)$, respectively, find the coordinates of P such that AP= $\frac {3}{7}$AB $\text{ and }$ P lies on the line segment AB.
	\\
		\solution
	\iffalse
\documentclass[journal,10pt,twocolumn]{article}
\usepackage{graphicx}
\usepackage[none]{hyphenat}
\usepackage{graphicx}
\usepackage{listings}
\usepackage[english]{babel}
\usepackage{graphicx}
\usepackage{caption} 
\usepackage{booktabs}
\usepackage{array}
\usepackage{amssymb} % for \because
\usepackage{amsmath}   % for having text in math mode
\usepackage{extarrows} % for Row operations arrows
\usepackage{listings}
\usepackage[utf8]{inputenc}
\lstset{
  frame=single,
  breaklines=true
}
\usepackage{hyperref}
  
%Following 2 lines were added to remove the blank page at the beginning
\usepackage{atbegshi}% http://ctan.org/pkg/atbegshi
\AtBeginDocument{\AtBeginShipoutNext{\AtBeginShipoutDiscard}}


%New macro definitions
\newcommand{\mydet}[1]{\ensuremath{\begin{vmatrix}#1\end{vmatrix}}}
\providecommand{\brak}[1]{\ensuremath{\left(#1\right)}}
\newcommand{\solution}{\noindent \textbf{Solution: }}
\newcommand{\myvec}[1]{\ensuremath{\begin{pmatrix}#1\end{pmatrix}}}
\providecommand{\norm}[1]{\left\lVert#1\right\rVert}
\providecommand{\abs}[1]{\left\vert#1\right\vert}
\let\vec\mathbf

\begin{document}

\begin{center}
\title{\textbf{VECTORS}}
\date{\vspace{-5ex}} %Not to print date automatically
\maketitle
\end{center}

\section{10$^{th}$ Maths - EXERCISE-7.2}

\begin{enumerate}
\item If A and B are $(– 2, – 2)\text{ and }(2, – 4)$, respectively, find the coordinates of P such that $AP =\frac{3}{7}AB$ and P lies on the line segment AB. 

\section{SOLUTION}
Given points are
\begin{align}
\vec{A}=\myvec{-2\\ -2} ,
\vec{B}=\myvec{2\\ -4}
\end{align}
The equation of the formula is
\fi
Using section formula, 
\begin{align}
\vec{P}&=\frac{\vec{A}+n\vec{B}}{1+n}
\end{align}
where
\begin{align}
	n =\frac{3}{4}
\end{align}
Thus,
\begin{align}
\vec{P}&=\frac{1}{1+\frac{3}{4}}\brak{\myvec{-2\\-2}+\frac{3}{4}\myvec{2\\-4}}\\
&=\myvec{\frac{-2}{7}\\[1pt] \frac{-20}{7}}
\end{align}
See Fig. 
   \ref{fig:chapters/10/7/2/8/vec.png}
\begin{figure}
   \centering 
 \includegraphics[width=\columnwidth]{chapters/10/7/2/8/figs/vec.png}
   \caption{}
   \label{fig:chapters/10/7/2/8/vec.png}
   \end{figure}

\item Find the coordinates of the points which divide the line segment joining $A(-2,2) \text{ and } B(2,8)$ into four equal parts.
	\\
		\solution
	\begin{enumerate}[label=\thesection.\arabic*,ref=\thesection.\theenumi]
\numberwithin{equation}{enumi}
\numberwithin{figure}{enumi}
\numberwithin{table}{enumi}

\item Find the coordinates of the point which divides the join of $(-1,7) \text{ and } (4,-3)$ in the ratio 2:3.
	\\
		\solution
	\input{chapters/10/7/2/1/section.tex}
\item Find the coordinates of the points of trisection of the line segment joining $(4,-1) \text{ and } (-2,3)$.
	\\
		\solution
	\input{chapters/10/7/2/2/section.tex}
\item
	\iffalse
\item To conduct Sports Day activities, in your rectangular shaped school                   
ground ABCD, lines have 
drawn with chalk powder at a                 
distance of 1m each. 100 flower pots have been placed at a distance of 1m 
from each other along AD, as shown 
in Fig. 7.12. Niharika runs $ \frac {1}{4} $th the 
distance AD on the 2nd line and 
posts a green flag. Preet runs $ \frac {1}{5} $th 
the distance AD on the eighth line 
and posts a red flag. What is the 
distance between both the flags? If 
Rashmi has to post a blue flag exactly 
halfway between the line segment 
joining the two flags, where should 
she post her flag?
\begin{figure}[h!]
  \centering
  \includegraphics[width=\columnwidth]{sc.png}
  \caption{}
\label{fig:10/7/12Fig1}
\end{figure}               
\fi
      
\item Find the ratio in which the line segment joining the points $(-3,10) \text{ and } (6,-8)$ $\text{ is divided by } (-1,6)$.
	\\
		\solution
	\input{chapters/10/7/2/4/section.tex}
\item Find the ratio in which the line segment joining $A(1,-5) \text{ and } B(-4,5)$ $\text{is divided by the x-axis}$. Also find the coordinates of the point of division.
\item If $(1,2), (4,y), (x,6), (3,5)$ are the vertices of a parallelogram taken in order, find x and y.
	\\
		\solution
	\input{chapters/10/7/2/6/para1.tex}
\item Find the coordinates of a point A, where AB is the diameter of a circle whose centre is $(2,-3) \text{ and }$ B is $(1,4)$.
	\\
		\solution
	\input{chapters/10/7/2/7/section.tex}
\item If A \text{ and } B are $(-2,-2) \text{ and } (2,-4)$, respectively, find the coordinates of P such that AP= $\frac {3}{7}$AB $\text{ and }$ P lies on the line segment AB.
	\\
		\solution
	\input{chapters/10/7/2/8/section.tex}
\item Find the coordinates of the points which divide the line segment joining $A(-2,2) \text{ and } B(2,8)$ into four equal parts.
	\\
		\solution
	\input{chapters/10/7/2/9/section.tex}
\item Find the area of a rhombus if its vertices are $(3,0), (4,5), (-1,4) \text{ and } (-2,-1)$ taken in order. [$\vec{Hint}$ : Area of rhombus =$\frac {1}{2}$(product of its diagonals)]
	\\
		\solution
	\input{chapters/10/7/2/10/cross.tex}
\item Find the position vector of a point R which divides the line joining two points $\vec{P}$
and $\vec{Q}$ whose position vectors are $\hat{i}+2\hat{j}-\hat{k}$ and $-\hat{i}+\hat{j}+\hat{k}$ respectively, in the
ratio 2 : 1
\begin{enumerate}
    \item  internally
    \item  externally
\end{enumerate}
\solution
		\input{chapters/12/10/2/15/section.tex}
\item Find the position vector of the mid point of the vector joining the points $\vec{P}$(2, 3, 4)
and $\vec{Q}$(4, 1, –2).
\\
\solution
		\input{chapters/12/10/2/16/section.tex}
\item Determine the ratio in which the line $2x+y  - 4=0$ divides the line segment joining the points $\vec{A}(2, - 2)$  and  $\vec{B}(3, 7)$.
\\
\solution
	\input{chapters/10/7/4/1/section.tex}
\item Let $\vec{A}(4, 2), \vec{B}(6, 5)$  and $ \vec{C}(1, 4)$ be the vertices of $\triangle ABC$.
\begin{enumerate}
\item The median from $\vec{A}$ meets $BC$ at $\vec{D}$. Find the coordinates of the point $\vec{D}$.
\item Find the coordinates of the point $\vec{P}$ on $AD$ such that $AP : PD = 2 : 1$.
\item Find the coordinates of points $\vec{Q}$ and $\vec{R}$ on medians $BE$ and $CF$ respectively such that $BQ : QE = 2 : 1$  and  $CR : RF = 2 : 1$.
\item What do you observe?
\item If $\vec{A}, \vec{B}$ and $\vec{C}$  are the vertices of $\triangle ABC$, find the coordinates of the centroid of the triangle.
\end{enumerate}
\solution
	\input{chapters/10/7/4/7/section.tex}
\item Find the slope of a line, which passes through the origin and the mid point of the line segment joining the points $\vec{P}$(0,-4) and $\vec{B}$(8,0).
\label{chapters/11/10/1/5}
\input{chapters/11/10/1/5/matrix.tex}
\item Find the position vector of a point R which divides the line joining two points P and Q whose position vectors are $(2\vec{a}+\vec{b})$ and $(\vec{a}-3\vec{b})$
externally in the ratio 1 : 2. Also, show that P is the mid point of the line segment RQ.\\
	\solution
%		\input{chapters/12/10/5/9/section.tex}

\end{enumerate}


\item Find the area of a rhombus if its vertices are $(3,0), (4,5), (-1,4) \text{ and } (-2,-1)$ taken in order. [$\vec{Hint}$ : Area of rhombus =$\frac {1}{2}$(product of its diagonals)]
	\\
		\solution
	\iffalse
\documentclass[12pt]{article}
\usepackage{graphicx}
%\documentclass[journal,12pt,twocolumn]{IEEEtran}
\usepackage[none]{hyphenat}
\usepackage{graphicx}
\usepackage{listings}
\usepackage[english]{babel}
\usepackage{graphicx}
\usepackage{caption} 
\usepackage{hyperref}
\usepackage{booktabs}
\def\inputGnumericTable{}
\usepackage{color}                                            %%
    \usepackage{array}                                            %%
    \usepackage{longtable}                                        %%
    \usepackage{calc}                                             %%
    \usepackage{multirow}                                         %%
    \usepackage{hhline}                                           %%
    \usepackage{ifthen}
\usepackage{array}
\usepackage{amsmath}   % for having text in math mode
\usepackage{listings}
\lstset{
language=tex,
frame=single, 
breaklines=true
}
  
%Following 2 lines were added to remove the blank page at the beginning
\usepackage{atbegshi}% http://ctan.org/pkg/atbegshi
\AtBeginDocument{\AtBeginShipoutNext{\AtBeginShipoutDiscard}}
%


%New macro definitions
\newcommand{\mydet}[1]{\ensuremath{\begin{vmatrix}#1\end{vmatrix}}}
\providecommand{\brak}[1]{\ensuremath{\left(#1\right)}}
\providecommand{\norm}[1]{\left\lVert#1\right\rVert}
\newcommand{\solution}{\noindent \textbf{Solution: }}
\newcommand{\myvec}[1]{\ensuremath{\begin{pmatrix}#1\end{pmatrix}}}
\let\vec\mathbf

\begin{document}

\begin{center}
\title{\textbf{Coordinate Geometry}}
\date{\vspace{-5ex}} %Not to print date automatically
\maketitle
\end{center}

\setcounter{page}{1}



\begin{enumerate}

\item\textbf{Problem statement :} Find the area of a rhombus of its vertices are $\myvec{3 ,0}$, $\myvec{4 ,5}$, $\myvec{-1 ,4}$ and $\myvec{-2 ,-1}$taken in order

\solution \\
\fi
The input vertices for this problem are given as
	\begin{align}
	\vec{A} = \myvec{
		3\\
		0
		},
	\vec{B} = \myvec{
		4\\
		5
		},
        \vec{C} = \myvec{
		-1\\
		4
		},
        \vec{D} = \myvec{
		-2\\
		-1
		}
	\end{align}
Since		
\begin{align}
 \vec{A-D}= \myvec{3 \\ 0} - \myvec{-2 \\-1}= \myvec{5\\1}
 \\
  \vec{B-A}= \myvec{4 \\ 5} - \myvec{3 \\0}= \myvec{1\\5}
\end{align}
the area of the rhombus is
\begin{align}
                \norm{\myvec{\vec{A-D}}\times \myvec{\vec{B-A}}}=\mydet{5 & 1\\1 & 5} = 24
\end{align}
See Fig. 
\ref{fig:chapters/10/7/2/10/gFig1}.
\begin{figure}[!h]
 \begin{center}
  \includegraphics[width=\columnwidth]{chapters/10/7/2/10/figs/fig.pdf}
 \end{center}
\caption{}
\label{fig:chapters/10/7/2/10/gFig1}
\end{figure}

\item Find the position vector of a point R which divides the line joining two points $\vec{P}$
and $\vec{Q}$ whose position vectors are $\hat{i}+2\hat{j}-\hat{k}$ and $-\hat{i}+\hat{j}+\hat{k}$ respectively, in the
ratio 2 : 1
\begin{enumerate}
    \item  internally
    \item  externally
\end{enumerate}
\solution
		\begin{enumerate}[label=\thesection.\arabic*,ref=\thesection.\theenumi]
\numberwithin{equation}{enumi}
\numberwithin{figure}{enumi}
\numberwithin{table}{enumi}

\item Find the coordinates of the point which divides the join of $(-1,7) \text{ and } (4,-3)$ in the ratio 2:3.
	\\
		\solution
	\input{chapters/10/7/2/1/section.tex}
\item Find the coordinates of the points of trisection of the line segment joining $(4,-1) \text{ and } (-2,3)$.
	\\
		\solution
	\input{chapters/10/7/2/2/section.tex}
\item
	\iffalse
\item To conduct Sports Day activities, in your rectangular shaped school                   
ground ABCD, lines have 
drawn with chalk powder at a                 
distance of 1m each. 100 flower pots have been placed at a distance of 1m 
from each other along AD, as shown 
in Fig. 7.12. Niharika runs $ \frac {1}{4} $th the 
distance AD on the 2nd line and 
posts a green flag. Preet runs $ \frac {1}{5} $th 
the distance AD on the eighth line 
and posts a red flag. What is the 
distance between both the flags? If 
Rashmi has to post a blue flag exactly 
halfway between the line segment 
joining the two flags, where should 
she post her flag?
\begin{figure}[h!]
  \centering
  \includegraphics[width=\columnwidth]{sc.png}
  \caption{}
\label{fig:10/7/12Fig1}
\end{figure}               
\fi
      
\item Find the ratio in which the line segment joining the points $(-3,10) \text{ and } (6,-8)$ $\text{ is divided by } (-1,6)$.
	\\
		\solution
	\input{chapters/10/7/2/4/section.tex}
\item Find the ratio in which the line segment joining $A(1,-5) \text{ and } B(-4,5)$ $\text{is divided by the x-axis}$. Also find the coordinates of the point of division.
\item If $(1,2), (4,y), (x,6), (3,5)$ are the vertices of a parallelogram taken in order, find x and y.
	\\
		\solution
	\input{chapters/10/7/2/6/para1.tex}
\item Find the coordinates of a point A, where AB is the diameter of a circle whose centre is $(2,-3) \text{ and }$ B is $(1,4)$.
	\\
		\solution
	\input{chapters/10/7/2/7/section.tex}
\item If A \text{ and } B are $(-2,-2) \text{ and } (2,-4)$, respectively, find the coordinates of P such that AP= $\frac {3}{7}$AB $\text{ and }$ P lies on the line segment AB.
	\\
		\solution
	\input{chapters/10/7/2/8/section.tex}
\item Find the coordinates of the points which divide the line segment joining $A(-2,2) \text{ and } B(2,8)$ into four equal parts.
	\\
		\solution
	\input{chapters/10/7/2/9/section.tex}
\item Find the area of a rhombus if its vertices are $(3,0), (4,5), (-1,4) \text{ and } (-2,-1)$ taken in order. [$\vec{Hint}$ : Area of rhombus =$\frac {1}{2}$(product of its diagonals)]
	\\
		\solution
	\input{chapters/10/7/2/10/cross.tex}
\item Find the position vector of a point R which divides the line joining two points $\vec{P}$
and $\vec{Q}$ whose position vectors are $\hat{i}+2\hat{j}-\hat{k}$ and $-\hat{i}+\hat{j}+\hat{k}$ respectively, in the
ratio 2 : 1
\begin{enumerate}
    \item  internally
    \item  externally
\end{enumerate}
\solution
		\input{chapters/12/10/2/15/section.tex}
\item Find the position vector of the mid point of the vector joining the points $\vec{P}$(2, 3, 4)
and $\vec{Q}$(4, 1, –2).
\\
\solution
		\input{chapters/12/10/2/16/section.tex}
\item Determine the ratio in which the line $2x+y  - 4=0$ divides the line segment joining the points $\vec{A}(2, - 2)$  and  $\vec{B}(3, 7)$.
\\
\solution
	\input{chapters/10/7/4/1/section.tex}
\item Let $\vec{A}(4, 2), \vec{B}(6, 5)$  and $ \vec{C}(1, 4)$ be the vertices of $\triangle ABC$.
\begin{enumerate}
\item The median from $\vec{A}$ meets $BC$ at $\vec{D}$. Find the coordinates of the point $\vec{D}$.
\item Find the coordinates of the point $\vec{P}$ on $AD$ such that $AP : PD = 2 : 1$.
\item Find the coordinates of points $\vec{Q}$ and $\vec{R}$ on medians $BE$ and $CF$ respectively such that $BQ : QE = 2 : 1$  and  $CR : RF = 2 : 1$.
\item What do you observe?
\item If $\vec{A}, \vec{B}$ and $\vec{C}$  are the vertices of $\triangle ABC$, find the coordinates of the centroid of the triangle.
\end{enumerate}
\solution
	\input{chapters/10/7/4/7/section.tex}
\item Find the slope of a line, which passes through the origin and the mid point of the line segment joining the points $\vec{P}$(0,-4) and $\vec{B}$(8,0).
\label{chapters/11/10/1/5}
\input{chapters/11/10/1/5/matrix.tex}
\item Find the position vector of a point R which divides the line joining two points P and Q whose position vectors are $(2\vec{a}+\vec{b})$ and $(\vec{a}-3\vec{b})$
externally in the ratio 1 : 2. Also, show that P is the mid point of the line segment RQ.\\
	\solution
%		\input{chapters/12/10/5/9/section.tex}

\end{enumerate}


\item Find the position vector of the mid point of the vector joining the points $\vec{P}$(2, 3, 4)
and $\vec{Q}$(4, 1, –2).
\\
\solution
		\begin{enumerate}[label=\thesection.\arabic*,ref=\thesection.\theenumi]
\numberwithin{equation}{enumi}
\numberwithin{figure}{enumi}
\numberwithin{table}{enumi}

\item Find the coordinates of the point which divides the join of $(-1,7) \text{ and } (4,-3)$ in the ratio 2:3.
	\\
		\solution
	\input{chapters/10/7/2/1/section.tex}
\item Find the coordinates of the points of trisection of the line segment joining $(4,-1) \text{ and } (-2,3)$.
	\\
		\solution
	\input{chapters/10/7/2/2/section.tex}
\item
	\iffalse
\item To conduct Sports Day activities, in your rectangular shaped school                   
ground ABCD, lines have 
drawn with chalk powder at a                 
distance of 1m each. 100 flower pots have been placed at a distance of 1m 
from each other along AD, as shown 
in Fig. 7.12. Niharika runs $ \frac {1}{4} $th the 
distance AD on the 2nd line and 
posts a green flag. Preet runs $ \frac {1}{5} $th 
the distance AD on the eighth line 
and posts a red flag. What is the 
distance between both the flags? If 
Rashmi has to post a blue flag exactly 
halfway between the line segment 
joining the two flags, where should 
she post her flag?
\begin{figure}[h!]
  \centering
  \includegraphics[width=\columnwidth]{sc.png}
  \caption{}
\label{fig:10/7/12Fig1}
\end{figure}               
\fi
      
\item Find the ratio in which the line segment joining the points $(-3,10) \text{ and } (6,-8)$ $\text{ is divided by } (-1,6)$.
	\\
		\solution
	\input{chapters/10/7/2/4/section.tex}
\item Find the ratio in which the line segment joining $A(1,-5) \text{ and } B(-4,5)$ $\text{is divided by the x-axis}$. Also find the coordinates of the point of division.
\item If $(1,2), (4,y), (x,6), (3,5)$ are the vertices of a parallelogram taken in order, find x and y.
	\\
		\solution
	\input{chapters/10/7/2/6/para1.tex}
\item Find the coordinates of a point A, where AB is the diameter of a circle whose centre is $(2,-3) \text{ and }$ B is $(1,4)$.
	\\
		\solution
	\input{chapters/10/7/2/7/section.tex}
\item If A \text{ and } B are $(-2,-2) \text{ and } (2,-4)$, respectively, find the coordinates of P such that AP= $\frac {3}{7}$AB $\text{ and }$ P lies on the line segment AB.
	\\
		\solution
	\input{chapters/10/7/2/8/section.tex}
\item Find the coordinates of the points which divide the line segment joining $A(-2,2) \text{ and } B(2,8)$ into four equal parts.
	\\
		\solution
	\input{chapters/10/7/2/9/section.tex}
\item Find the area of a rhombus if its vertices are $(3,0), (4,5), (-1,4) \text{ and } (-2,-1)$ taken in order. [$\vec{Hint}$ : Area of rhombus =$\frac {1}{2}$(product of its diagonals)]
	\\
		\solution
	\input{chapters/10/7/2/10/cross.tex}
\item Find the position vector of a point R which divides the line joining two points $\vec{P}$
and $\vec{Q}$ whose position vectors are $\hat{i}+2\hat{j}-\hat{k}$ and $-\hat{i}+\hat{j}+\hat{k}$ respectively, in the
ratio 2 : 1
\begin{enumerate}
    \item  internally
    \item  externally
\end{enumerate}
\solution
		\input{chapters/12/10/2/15/section.tex}
\item Find the position vector of the mid point of the vector joining the points $\vec{P}$(2, 3, 4)
and $\vec{Q}$(4, 1, –2).
\\
\solution
		\input{chapters/12/10/2/16/section.tex}
\item Determine the ratio in which the line $2x+y  - 4=0$ divides the line segment joining the points $\vec{A}(2, - 2)$  and  $\vec{B}(3, 7)$.
\\
\solution
	\input{chapters/10/7/4/1/section.tex}
\item Let $\vec{A}(4, 2), \vec{B}(6, 5)$  and $ \vec{C}(1, 4)$ be the vertices of $\triangle ABC$.
\begin{enumerate}
\item The median from $\vec{A}$ meets $BC$ at $\vec{D}$. Find the coordinates of the point $\vec{D}$.
\item Find the coordinates of the point $\vec{P}$ on $AD$ such that $AP : PD = 2 : 1$.
\item Find the coordinates of points $\vec{Q}$ and $\vec{R}$ on medians $BE$ and $CF$ respectively such that $BQ : QE = 2 : 1$  and  $CR : RF = 2 : 1$.
\item What do you observe?
\item If $\vec{A}, \vec{B}$ and $\vec{C}$  are the vertices of $\triangle ABC$, find the coordinates of the centroid of the triangle.
\end{enumerate}
\solution
	\input{chapters/10/7/4/7/section.tex}
\item Find the slope of a line, which passes through the origin and the mid point of the line segment joining the points $\vec{P}$(0,-4) and $\vec{B}$(8,0).
\label{chapters/11/10/1/5}
\input{chapters/11/10/1/5/matrix.tex}
\item Find the position vector of a point R which divides the line joining two points P and Q whose position vectors are $(2\vec{a}+\vec{b})$ and $(\vec{a}-3\vec{b})$
externally in the ratio 1 : 2. Also, show that P is the mid point of the line segment RQ.\\
	\solution
%		\input{chapters/12/10/5/9/section.tex}

\end{enumerate}


\item Determine the ratio in which the line $2x+y  - 4=0$ divides the line segment joining the points $\vec{A}(2, - 2)$  and  $\vec{B}(3, 7)$.
\\
\solution
	\iffalse
\documentclass[journal,12pt,twocolumn]{IEEEtran}
\usepackage{graphicx}
\graphicspath{{./chapters/10/7/4/1/figs/}}{}
\usepackage{amsmath,amssymb,amsfonts,amsthm}
\newcommand{\myvec}[1]{\ensuremath{\begin{pmatrix}#1\end{pmatrix}}}
\providecommand{\norm}[1]{\lVert#1\rVert}
\usepackage{listings}
\usepackage{watermark}
\usepackage{titlesec}
\usepackage{caption}
\let\vec\mathbf
\lstset{
frame=single, 
breaklines=true,
columns=fullflexible
}
\thiswatermark{\centering \put(0,-105.0){\includegraphics[scale=0.15]{/sdcard/IITH/vector/vectpr-4/chapters/10/7/4/1/figs/logo.png}} }
\title{\mytitle}
\title{
Assignment - Vector-4
}
\author{Surajit Sarkar}
\begin{document}
\maketitle
%\tableofcontents
\bigskip
\section{\textbf{Problem}}
Determine the ratio in which the line 2x+y–4=0 divides the line segment joining the points A(2,–2) and B(3,7).
\section{\textbf{Solution}}
\begin{table}[h]
    \centering
    \begin{tabular}{|c|c|}
       \hline
       \textbf{Symbol}&\textbf{Value}  \\
       \hline
	    $\vec{A}$ & $\myvec{2\\-2}$\\
        \hline
	    $\vec{B}$ & $\myvec{3\\7}$\\
        \hline
	    c&$4$\\
        \hline
       $\vec{n}$ & $\myvec{2\\1}$\\
       \hline
    \end{tabular}
    \caption{Parameters}
    \label{tab:my_label}
\end{table}
Given equation
\fi
The given equation can be expressed as
\begin{align}
    \myvec{2&1}\vec{x}&=4\\
\end{align}
Using section formula, the point of division 
\begin{align}
    \vec{P} = \frac{k\vec{B+A}}{k+1}
\end{align}
which upon substitution in the equation of a line yields
\begin{align}
    \implies\vec{n}^{\top}\myvec{\frac{k\vec{B+A}}{k+1}}&=c\\
    \implies k&=\frac{c-\vec{n}^{\top}\vec{A}}{\vec{n}^{\top}\vec{B}-c}\\
\end{align}
upon simplification.  Substituting numerical values, 
\begin{align}
    k=\frac{2}{9}
\end{align}
See Fig. 
\ref{fig:chapters/10/7/4/1vec}.
\begin{figure}[!h]
\centering
\includegraphics[width=\columnwidth]{chapters/10/7/4/1/figs/vec.pdf}
\caption{}
\label{fig:chapters/10/7/4/1vec}
\end{figure}


\item Let $\vec{A}(4, 2), \vec{B}(6, 5)$  and $ \vec{C}(1, 4)$ be the vertices of $\triangle ABC$.
\begin{enumerate}
\item The median from $\vec{A}$ meets $BC$ at $\vec{D}$. Find the coordinates of the point $\vec{D}$.
\item Find the coordinates of the point $\vec{P}$ on $AD$ such that $AP : PD = 2 : 1$.
\item Find the coordinates of points $\vec{Q}$ and $\vec{R}$ on medians $BE$ and $CF$ respectively such that $BQ : QE = 2 : 1$  and  $CR : RF = 2 : 1$.
\item What do you observe?
\item If $\vec{A}, \vec{B}$ and $\vec{C}$  are the vertices of $\triangle ABC$, find the coordinates of the centroid of the triangle.
\end{enumerate}
\solution
	\iffalse
\documentclass[12pt]{article}
\usepackage{graphicx}
\usepackage[none]{hyphenat}
\usepackage{graphicx}
\usepackage{listings}
\usepackage[english]{babel}
\usepackage{graphicx}
\usepackage{caption} 
\usepackage{booktabs}
\usepackage{array}
\usepackage{amssymb} % for \because
\usepackage{amsmath}   % for having text in math mode
\usepackage{extarrows} % for Row operations arrows
\usepackage{listings}
\usepackage[utf8]{inputenc}
\lstset{
  frame=single,
  breaklines=true
}
\usepackage{hyperref}
  
%Following 2 lines were added to remove the blank page at the beginning
\usepackage{atbegshi}% http://ctan.org/pkg/atbegshi
\AtBeginDocument{\AtBeginShipoutNext{\AtBeginShipoutDiscard}}


%New macro definitions
\newcommand{\mydet}[1]{\ensuremath{\begin{vmatrix}#1\end{vmatrix}}}
\providecommand{\brak}[1]{\ensuremath{\left(#1\right)}}
\newcommand{\solution}{\noindent \textbf{Solution: }}
\newcommand{\myvec}[1]{\ensuremath{\begin{pmatrix}#1\end{pmatrix}}}
\providecommand{\norm}[1]{\left\lVert#1\right\rVert}
\providecommand{\abs}[1]{\left\vert#1\right\vert}
\let\vec\mathbf

\begin{document}

\begin{center}
\title{\textbf{VECTORS}}
\date{\vspace{-5ex}} %Not to print date automatically
\maketitle
\end{center}

\section{10$^{th}$ Maths - EXERCISE-7.4}

Let A(4, 2), B(6, 5) and C(1, 4) be the vertices of $\triangle ABC$
\begin{enumerate}
\item The median from A meets BC at D. Find the coordinates of the point D.
\item Find the coordinates of the point P on AD such that $AP : PD = 2 : 1$
\item Find the coordinates of points Q and R on medians BE and CF respectively such
that $BQ : QE = 2 : 1 \text{and} CR : RF = 2 : 1.$
\item What do yo observe?
\item If $A(x_1, y_1), B(x_2, y_2) \text{and} C(x_3, y_3)$ are the vertices of $\triangle ABC$, find the coordinates of the centroid of the triangle.
\end{enumerate}

Given points are
\begin{align}
\vec{A}=\myvec{4\\ 2} ,
\vec{B}=\myvec{6\\ 5} ,
\vec{C}=\myvec{1\\ 4}
\end{align}
\fi

\begin{enumerate}
\item 
\begin{align}
\vec{D}&=\frac{\vec{B}+\vec{C}}{2}\\
&=\myvec{\frac{7}{2}\\[2pt] \frac{9}{2}}\\
\vec{E}&=\frac{\vec{A}+\vec{C}}{2}\\
&=\myvec{\frac{5}{2}\\ 3}\\
\vec{F}&=\frac{\vec{A}+\vec{B}}{2}\\
&=\myvec{5\\ \frac{7}{2}}
\end{align}

\item 
	For
$n=2$,
\begin{align}
\vec{P}&=\frac{1}{1+n}\brak{\myvec{\vec{A}+n\vec{D}}}\\
&=\frac{1}{3}\myvec{11\\11}
\end{align}

\item 
\begin{align}
\vec{Q}&=\frac{1}{1+n}\brak{\myvec{\vec{B}+n\vec{E}}}\\
&=\frac{1}{3}\myvec{11\\11}\\
\vec{R}&=\frac{1}{1+n}\brak{\myvec{\vec{C}+n\vec{F}}}\\
&=\frac{1}{3}\myvec{11\\11}\\
\end{align}

\item 
 $\vec{P},\vec{Q},\vec{R}$ are the same point.
   
\item 
\begin{align}
\vec{G}&=\frac{\vec{D}+\vec{E}+\vec{F}}{3}\\
&=\frac{1}{3}\myvec{11\\11}\\
\end{align} 
\end{enumerate}
See Fig.  
  \ref{fig:chapters/10/7/4/7/Figure}.
\begin{figure}[h!]
\centering
\includegraphics[width=\columnwidth]{chapters/10/7/4/7/figs/dj.pdf}
\caption{}
  \label{fig:chapters/10/7/4/7/Figure}
\end{figure}

\item Find the slope of a line, which passes through the origin and the mid point of the line segment joining the points $\vec{P}$(0,-4) and $\vec{B}$(8,0).
\label{chapters/11/10/1/5}
\iffalse
\documentclass[journal,12pt,twocolumn]{IEEEtran}
\usepackage{graphicx}
\graphicspath{{./figs/}}{}
\usepackage{amsmath,amssymb,amsfonts,amsthm}
\newcommand{\myvec}[1]{\ensuremath{\begin{pmatrix}#1\end{pmatrix}}}

\let\vec\mathbf

\title{
Matrix-Lines
}
\author{Jyothsna Paluchuri-FWC22059\\}
\begin{document}
\maketitle
\tableofcontents
\bigskip
\section{Problem Statement}
\fi
	\begin{figure}[!ht]
		\centering
 \includegraphics[width=\columnwidth]{chapters/11/10/1/5/figs/line.png}
		\caption{}
		\label{fig:11/10/1/5}
  	\end{figure}
	\\
	\solution
\iffalse
\section{Construction}
\begin{figure}[h]
    \centering
\includegraphics[width=\columnwidth]{line.png}
    \caption{Equation of the slope}
    \label{fig:my_label}
\end{figure}
\vspace{2cm}
\begin{table}[h]
    \centering
    \begin{tabular}{|c|c|c|c|}
       \hline
       \textbf{Symbol}&\textbf{Value}&\textbf{Description}  \\
       \hline
	    $\vec{P}$ & $\myvec{
		    0\\
		    -4}$
	    & Point on Y-axis\\
        \hline
	    $\vec{B}$ & $\myvec{8\\0}$
 & Point on X-axis\\
        \hline
	    $\vec{0}$ & $\myvec{0\\0}$
 & Origin\\
        \hline
    \end{tabular}
    \caption{Parameters}
    \label{tab:my_label}
\end{table}


\section{Solution}
Given that resultant line passes through origin and mid point of the line segment joining point P(0,-4) and B(8,0) \\
\\
\\
given ${\vec{P}}$=$\myvec{
  0\\
  -4}$
 , ${\vec{B}}$=$\myvec{
  8\\
  0}$
  
 \fi 
The mid point of $PB$ is
\begin{align}
\vec{M} &=\frac{1}{2}(\vec{P}+\vec{B})
	= \myvec{4 \\ -2}  
\end{align}
The direction vector of line joining $\vec{O}, \vec{M}$ is 
\begin{align}
\vec{m}&=\vec{O}-\vec{M}
 = -\vec{M}
\end{align}
which can be expressed as
\begin{align}
	\myvec{1 \\ -\frac{1}{2}}
\end{align}
Thus the slope is
\begin{align}
	m = -\frac{1}{2}
\end{align}
\iffalse
\textbf{The direction vector of a line expressed as}
\begin{align}
\implies\vec{m} &= \begin{pmatrix}1 \\ m \\ \end{pmatrix}
\end{align}

\textbf{By solving equation (5) and (6),we get the slope of $\vec{O}$ $\vec{M}$ line}
\begin{align}
        \boxed{m=-0.5}
 \end{align}

\section{Software}
Download the following code using,
\begin{table}[h]
    \centering
    \begin{tabular}{|c|}
    \hline \\
   https://github.com/jyothsna777/jyothsna-fwc.git  \\
         \\
\hline
    \end{tabular}
\end{table}
\\
and execute the code by using command
\begin{center}
\textbf{Python3 lines.py}\\
\end{center}

\section{Conclusion}
Hence the slope of line $\vec{O}$ $\vec{M}$ lineis $\vec{m}$=-0.5

\end{document}
\fi

\item Find the position vector of a point R which divides the line joining two points P and Q whose position vectors are $(2\vec{a}+\vec{b})$ and $(\vec{a}-3\vec{b})$
externally in the ratio 1 : 2. Also, show that P is the mid point of the line segment RQ.\\
	\solution
%		\begin{enumerate}[label=\thesection.\arabic*,ref=\thesection.\theenumi]
\numberwithin{equation}{enumi}
\numberwithin{figure}{enumi}
\numberwithin{table}{enumi}

\item Find the coordinates of the point which divides the join of $(-1,7) \text{ and } (4,-3)$ in the ratio 2:3.
	\\
		\solution
	\input{chapters/10/7/2/1/section.tex}
\item Find the coordinates of the points of trisection of the line segment joining $(4,-1) \text{ and } (-2,3)$.
	\\
		\solution
	\input{chapters/10/7/2/2/section.tex}
\item
	\iffalse
\item To conduct Sports Day activities, in your rectangular shaped school                   
ground ABCD, lines have 
drawn with chalk powder at a                 
distance of 1m each. 100 flower pots have been placed at a distance of 1m 
from each other along AD, as shown 
in Fig. 7.12. Niharika runs $ \frac {1}{4} $th the 
distance AD on the 2nd line and 
posts a green flag. Preet runs $ \frac {1}{5} $th 
the distance AD on the eighth line 
and posts a red flag. What is the 
distance between both the flags? If 
Rashmi has to post a blue flag exactly 
halfway between the line segment 
joining the two flags, where should 
she post her flag?
\begin{figure}[h!]
  \centering
  \includegraphics[width=\columnwidth]{sc.png}
  \caption{}
\label{fig:10/7/12Fig1}
\end{figure}               
\fi
      
\item Find the ratio in which the line segment joining the points $(-3,10) \text{ and } (6,-8)$ $\text{ is divided by } (-1,6)$.
	\\
		\solution
	\input{chapters/10/7/2/4/section.tex}
\item Find the ratio in which the line segment joining $A(1,-5) \text{ and } B(-4,5)$ $\text{is divided by the x-axis}$. Also find the coordinates of the point of division.
\item If $(1,2), (4,y), (x,6), (3,5)$ are the vertices of a parallelogram taken in order, find x and y.
	\\
		\solution
	\input{chapters/10/7/2/6/para1.tex}
\item Find the coordinates of a point A, where AB is the diameter of a circle whose centre is $(2,-3) \text{ and }$ B is $(1,4)$.
	\\
		\solution
	\input{chapters/10/7/2/7/section.tex}
\item If A \text{ and } B are $(-2,-2) \text{ and } (2,-4)$, respectively, find the coordinates of P such that AP= $\frac {3}{7}$AB $\text{ and }$ P lies on the line segment AB.
	\\
		\solution
	\input{chapters/10/7/2/8/section.tex}
\item Find the coordinates of the points which divide the line segment joining $A(-2,2) \text{ and } B(2,8)$ into four equal parts.
	\\
		\solution
	\input{chapters/10/7/2/9/section.tex}
\item Find the area of a rhombus if its vertices are $(3,0), (4,5), (-1,4) \text{ and } (-2,-1)$ taken in order. [$\vec{Hint}$ : Area of rhombus =$\frac {1}{2}$(product of its diagonals)]
	\\
		\solution
	\input{chapters/10/7/2/10/cross.tex}
\item Find the position vector of a point R which divides the line joining two points $\vec{P}$
and $\vec{Q}$ whose position vectors are $\hat{i}+2\hat{j}-\hat{k}$ and $-\hat{i}+\hat{j}+\hat{k}$ respectively, in the
ratio 2 : 1
\begin{enumerate}
    \item  internally
    \item  externally
\end{enumerate}
\solution
		\input{chapters/12/10/2/15/section.tex}
\item Find the position vector of the mid point of the vector joining the points $\vec{P}$(2, 3, 4)
and $\vec{Q}$(4, 1, –2).
\\
\solution
		\input{chapters/12/10/2/16/section.tex}
\item Determine the ratio in which the line $2x+y  - 4=0$ divides the line segment joining the points $\vec{A}(2, - 2)$  and  $\vec{B}(3, 7)$.
\\
\solution
	\input{chapters/10/7/4/1/section.tex}
\item Let $\vec{A}(4, 2), \vec{B}(6, 5)$  and $ \vec{C}(1, 4)$ be the vertices of $\triangle ABC$.
\begin{enumerate}
\item The median from $\vec{A}$ meets $BC$ at $\vec{D}$. Find the coordinates of the point $\vec{D}$.
\item Find the coordinates of the point $\vec{P}$ on $AD$ such that $AP : PD = 2 : 1$.
\item Find the coordinates of points $\vec{Q}$ and $\vec{R}$ on medians $BE$ and $CF$ respectively such that $BQ : QE = 2 : 1$  and  $CR : RF = 2 : 1$.
\item What do you observe?
\item If $\vec{A}, \vec{B}$ and $\vec{C}$  are the vertices of $\triangle ABC$, find the coordinates of the centroid of the triangle.
\end{enumerate}
\solution
	\input{chapters/10/7/4/7/section.tex}
\item Find the slope of a line, which passes through the origin and the mid point of the line segment joining the points $\vec{P}$(0,-4) and $\vec{B}$(8,0).
\label{chapters/11/10/1/5}
\input{chapters/11/10/1/5/matrix.tex}
\item Find the position vector of a point R which divides the line joining two points P and Q whose position vectors are $(2\vec{a}+\vec{b})$ and $(\vec{a}-3\vec{b})$
externally in the ratio 1 : 2. Also, show that P is the mid point of the line segment RQ.\\
	\solution
%		\input{chapters/12/10/5/9/section.tex}

\end{enumerate}



\end{enumerate}


\item Determine the ratio in which the line $2x+y  - 4=0$ divides the line segment joining the points $\vec{A}(2, - 2)$  and  $\vec{B}(3, 7)$.
\\
\solution
	\iffalse
\documentclass[journal,12pt,twocolumn]{IEEEtran}
\usepackage{graphicx}
\graphicspath{{./chapters/10/7/4/1/figs/}}{}
\usepackage{amsmath,amssymb,amsfonts,amsthm}
\newcommand{\myvec}[1]{\ensuremath{\begin{pmatrix}#1\end{pmatrix}}}
\providecommand{\norm}[1]{\lVert#1\rVert}
\usepackage{listings}
\usepackage{watermark}
\usepackage{titlesec}
\usepackage{caption}
\let\vec\mathbf
\lstset{
frame=single, 
breaklines=true,
columns=fullflexible
}
\thiswatermark{\centering \put(0,-105.0){\includegraphics[scale=0.15]{/sdcard/IITH/vector/vectpr-4/chapters/10/7/4/1/figs/logo.png}} }
\title{\mytitle}
\title{
Assignment - Vector-4
}
\author{Surajit Sarkar}
\begin{document}
\maketitle
%\tableofcontents
\bigskip
\section{\textbf{Problem}}
Determine the ratio in which the line 2x+y–4=0 divides the line segment joining the points A(2,–2) and B(3,7).
\section{\textbf{Solution}}
\begin{table}[h]
    \centering
    \begin{tabular}{|c|c|}
       \hline
       \textbf{Symbol}&\textbf{Value}  \\
       \hline
	    $\vec{A}$ & $\myvec{2\\-2}$\\
        \hline
	    $\vec{B}$ & $\myvec{3\\7}$\\
        \hline
	    c&$4$\\
        \hline
       $\vec{n}$ & $\myvec{2\\1}$\\
       \hline
    \end{tabular}
    \caption{Parameters}
    \label{tab:my_label}
\end{table}
Given equation
\fi
The given equation can be expressed as
\begin{align}
    \myvec{2&1}\vec{x}&=4\\
\end{align}
Using section formula, the point of division 
\begin{align}
    \vec{P} = \frac{k\vec{B+A}}{k+1}
\end{align}
which upon substitution in the equation of a line yields
\begin{align}
    \implies\vec{n}^{\top}\myvec{\frac{k\vec{B+A}}{k+1}}&=c\\
    \implies k&=\frac{c-\vec{n}^{\top}\vec{A}}{\vec{n}^{\top}\vec{B}-c}\\
\end{align}
upon simplification.  Substituting numerical values, 
\begin{align}
    k=\frac{2}{9}
\end{align}
See Fig. 
\ref{fig:chapters/10/7/4/1vec}.
\begin{figure}[!h]
\centering
\includegraphics[width=\columnwidth]{chapters/10/7/4/1/figs/vec.pdf}
\caption{}
\label{fig:chapters/10/7/4/1vec}
\end{figure}


\item Let $\vec{A}(4, 2), \vec{B}(6, 5)$  and $ \vec{C}(1, 4)$ be the vertices of $\triangle ABC$.
\begin{enumerate}
\item The median from $\vec{A}$ meets $BC$ at $\vec{D}$. Find the coordinates of the point $\vec{D}$.
\item Find the coordinates of the point $\vec{P}$ on $AD$ such that $AP : PD = 2 : 1$.
\item Find the coordinates of points $\vec{Q}$ and $\vec{R}$ on medians $BE$ and $CF$ respectively such that $BQ : QE = 2 : 1$  and  $CR : RF = 2 : 1$.
\item What do you observe?
\item If $\vec{A}, \vec{B}$ and $\vec{C}$  are the vertices of $\triangle ABC$, find the coordinates of the centroid of the triangle.
\end{enumerate}
\solution
	\iffalse
\documentclass[12pt]{article}
\usepackage{graphicx}
\usepackage[none]{hyphenat}
\usepackage{graphicx}
\usepackage{listings}
\usepackage[english]{babel}
\usepackage{graphicx}
\usepackage{caption} 
\usepackage{booktabs}
\usepackage{array}
\usepackage{amssymb} % for \because
\usepackage{amsmath}   % for having text in math mode
\usepackage{extarrows} % for Row operations arrows
\usepackage{listings}
\usepackage[utf8]{inputenc}
\lstset{
  frame=single,
  breaklines=true
}
\usepackage{hyperref}
  
%Following 2 lines were added to remove the blank page at the beginning
\usepackage{atbegshi}% http://ctan.org/pkg/atbegshi
\AtBeginDocument{\AtBeginShipoutNext{\AtBeginShipoutDiscard}}


%New macro definitions
\newcommand{\mydet}[1]{\ensuremath{\begin{vmatrix}#1\end{vmatrix}}}
\providecommand{\brak}[1]{\ensuremath{\left(#1\right)}}
\newcommand{\solution}{\noindent \textbf{Solution: }}
\newcommand{\myvec}[1]{\ensuremath{\begin{pmatrix}#1\end{pmatrix}}}
\providecommand{\norm}[1]{\left\lVert#1\right\rVert}
\providecommand{\abs}[1]{\left\vert#1\right\vert}
\let\vec\mathbf

\begin{document}

\begin{center}
\title{\textbf{VECTORS}}
\date{\vspace{-5ex}} %Not to print date automatically
\maketitle
\end{center}

\section{10$^{th}$ Maths - EXERCISE-7.4}

Let A(4, 2), B(6, 5) and C(1, 4) be the vertices of $\triangle ABC$
\begin{enumerate}
\item The median from A meets BC at D. Find the coordinates of the point D.
\item Find the coordinates of the point P on AD such that $AP : PD = 2 : 1$
\item Find the coordinates of points Q and R on medians BE and CF respectively such
that $BQ : QE = 2 : 1 \text{and} CR : RF = 2 : 1.$
\item What do yo observe?
\item If $A(x_1, y_1), B(x_2, y_2) \text{and} C(x_3, y_3)$ are the vertices of $\triangle ABC$, find the coordinates of the centroid of the triangle.
\end{enumerate}

Given points are
\begin{align}
\vec{A}=\myvec{4\\ 2} ,
\vec{B}=\myvec{6\\ 5} ,
\vec{C}=\myvec{1\\ 4}
\end{align}
\fi

\begin{enumerate}
\item 
\begin{align}
\vec{D}&=\frac{\vec{B}+\vec{C}}{2}\\
&=\myvec{\frac{7}{2}\\[2pt] \frac{9}{2}}\\
\vec{E}&=\frac{\vec{A}+\vec{C}}{2}\\
&=\myvec{\frac{5}{2}\\ 3}\\
\vec{F}&=\frac{\vec{A}+\vec{B}}{2}\\
&=\myvec{5\\ \frac{7}{2}}
\end{align}

\item 
	For
$n=2$,
\begin{align}
\vec{P}&=\frac{1}{1+n}\brak{\myvec{\vec{A}+n\vec{D}}}\\
&=\frac{1}{3}\myvec{11\\11}
\end{align}

\item 
\begin{align}
\vec{Q}&=\frac{1}{1+n}\brak{\myvec{\vec{B}+n\vec{E}}}\\
&=\frac{1}{3}\myvec{11\\11}\\
\vec{R}&=\frac{1}{1+n}\brak{\myvec{\vec{C}+n\vec{F}}}\\
&=\frac{1}{3}\myvec{11\\11}\\
\end{align}

\item 
 $\vec{P},\vec{Q},\vec{R}$ are the same point.
   
\item 
\begin{align}
\vec{G}&=\frac{\vec{D}+\vec{E}+\vec{F}}{3}\\
&=\frac{1}{3}\myvec{11\\11}\\
\end{align} 
\end{enumerate}
See Fig.  
  \ref{fig:chapters/10/7/4/7/Figure}.
\begin{figure}[h!]
\centering
\includegraphics[width=\columnwidth]{chapters/10/7/4/7/figs/dj.pdf}
\caption{}
  \label{fig:chapters/10/7/4/7/Figure}
\end{figure}

\item Find the slope of a line, which passes through the origin and the mid point of the line segment joining the points $\vec{P}$(0,-4) and $\vec{B}$(8,0).
\label{chapters/11/10/1/5}
\iffalse
\documentclass[journal,12pt,twocolumn]{IEEEtran}
\usepackage{graphicx}
\graphicspath{{./figs/}}{}
\usepackage{amsmath,amssymb,amsfonts,amsthm}
\newcommand{\myvec}[1]{\ensuremath{\begin{pmatrix}#1\end{pmatrix}}}

\let\vec\mathbf

\title{
Matrix-Lines
}
\author{Jyothsna Paluchuri-FWC22059\\}
\begin{document}
\maketitle
\tableofcontents
\bigskip
\section{Problem Statement}
\fi
	\begin{figure}[!ht]
		\centering
 \includegraphics[width=\columnwidth]{chapters/11/10/1/5/figs/line.png}
		\caption{}
		\label{fig:11/10/1/5}
  	\end{figure}
	\\
	\solution
\iffalse
\section{Construction}
\begin{figure}[h]
    \centering
\includegraphics[width=\columnwidth]{line.png}
    \caption{Equation of the slope}
    \label{fig:my_label}
\end{figure}
\vspace{2cm}
\begin{table}[h]
    \centering
    \begin{tabular}{|c|c|c|c|}
       \hline
       \textbf{Symbol}&\textbf{Value}&\textbf{Description}  \\
       \hline
	    $\vec{P}$ & $\myvec{
		    0\\
		    -4}$
	    & Point on Y-axis\\
        \hline
	    $\vec{B}$ & $\myvec{8\\0}$
 & Point on X-axis\\
        \hline
	    $\vec{0}$ & $\myvec{0\\0}$
 & Origin\\
        \hline
    \end{tabular}
    \caption{Parameters}
    \label{tab:my_label}
\end{table}


\section{Solution}
Given that resultant line passes through origin and mid point of the line segment joining point P(0,-4) and B(8,0) \\
\\
\\
given ${\vec{P}}$=$\myvec{
  0\\
  -4}$
 , ${\vec{B}}$=$\myvec{
  8\\
  0}$
  
 \fi 
The mid point of $PB$ is
\begin{align}
\vec{M} &=\frac{1}{2}(\vec{P}+\vec{B})
	= \myvec{4 \\ -2}  
\end{align}
The direction vector of line joining $\vec{O}, \vec{M}$ is 
\begin{align}
\vec{m}&=\vec{O}-\vec{M}
 = -\vec{M}
\end{align}
which can be expressed as
\begin{align}
	\myvec{1 \\ -\frac{1}{2}}
\end{align}
Thus the slope is
\begin{align}
	m = -\frac{1}{2}
\end{align}
\iffalse
\textbf{The direction vector of a line expressed as}
\begin{align}
\implies\vec{m} &= \begin{pmatrix}1 \\ m \\ \end{pmatrix}
\end{align}

\textbf{By solving equation (5) and (6),we get the slope of $\vec{O}$ $\vec{M}$ line}
\begin{align}
        \boxed{m=-0.5}
 \end{align}

\section{Software}
Download the following code using,
\begin{table}[h]
    \centering
    \begin{tabular}{|c|}
    \hline \\
   https://github.com/jyothsna777/jyothsna-fwc.git  \\
         \\
\hline
    \end{tabular}
\end{table}
\\
and execute the code by using command
\begin{center}
\textbf{Python3 lines.py}\\
\end{center}

\section{Conclusion}
Hence the slope of line $\vec{O}$ $\vec{M}$ lineis $\vec{m}$=-0.5

\end{document}
\fi

\item Find the position vector of a point R which divides the line joining two points P and Q whose position vectors are $(2\vec{a}+\vec{b})$ and $(\vec{a}-3\vec{b})$
externally in the ratio 1 : 2. Also, show that P is the mid point of the line segment RQ.\\
	\solution
%		\begin{enumerate}[label=\thesection.\arabic*,ref=\thesection.\theenumi]
\numberwithin{equation}{enumi}
\numberwithin{figure}{enumi}
\numberwithin{table}{enumi}

\item Find the coordinates of the point which divides the join of $(-1,7) \text{ and } (4,-3)$ in the ratio 2:3.
	\\
		\solution
	\iffalse
\documentclass[12pt]{article}
\usepackage{graphicx}
\usepackage{amsmath}
\usepackage{mathtools}
\usepackage{gensymb}

\newcommand{\mydet}[1]{\ensuremath{\begin{vmatrix}#1\end{vmatrix}}}
\providecommand{\brak}[1]{\ensuremath{\left(#1\right)}}
\providecommand{\norm}[1]{\left\lVert#1\right\rVert}
\newcommand{\solution}{\noindent \textbf{Solution: }}
\newcommand{\myvec}[1]{\ensuremath{\begin{pmatrix}#1\end{pmatrix}}}
\let\vec\mathbf

\begin{document}
\begin{center}
\textbf\large{CHAPTER-7 \\ COORDINATE GEOMETRY}
\end{center}
\section*{Excercise 7.2}

1. Find the coordinates of the point which divides the join $\vec(-1,7) \text{ and } \vec(4,-3)$ in the ratio 2:3 :
\\
\\
\solution\\		
\fi
The coordinates and ratio are given as
\begin{align}
\vec{P}=\myvec{-1\\7\\},
\vec{Q}=\myvec{4\\-3\\},
n=\frac{3}{2}
\end{align}
Using section formula
\begin{align}
\vec{R}&=\frac{\vec{Q}+n\vec{P}}{1+n}\\
&=\frac{1}{1+\frac{3}{2}}  \myvec{\myvec{
4\\
-3\\
}
  +
   \frac{3}{2}\myvec{
-1\\
7\\
}}\\
&=\myvec{
1\\
3
}
\end{align}
See Fig. 
\ref{fig:chapters/10/7/2/1/Fig}
\begin{figure}[!h]
\begin{center}
   \includegraphics[width=\columnwidth]{chapters/10/7/2/1/figs/linefig.png}
\end{center}
\caption{}
\label{fig:chapters/10/7/2/1/Fig}
\end{figure}


\item Find the coordinates of the points of trisection of the line segment joining $(4,-1) \text{ and } (-2,3)$.
	\\
		\solution
	\begin{enumerate}[label=\thesection.\arabic*,ref=\thesection.\theenumi]
\numberwithin{equation}{enumi}
\numberwithin{figure}{enumi}
\numberwithin{table}{enumi}

\item Find the coordinates of the point which divides the join of $(-1,7) \text{ and } (4,-3)$ in the ratio 2:3.
	\\
		\solution
	\input{chapters/10/7/2/1/section.tex}
\item Find the coordinates of the points of trisection of the line segment joining $(4,-1) \text{ and } (-2,3)$.
	\\
		\solution
	\input{chapters/10/7/2/2/section.tex}
\item
	\iffalse
\item To conduct Sports Day activities, in your rectangular shaped school                   
ground ABCD, lines have 
drawn with chalk powder at a                 
distance of 1m each. 100 flower pots have been placed at a distance of 1m 
from each other along AD, as shown 
in Fig. 7.12. Niharika runs $ \frac {1}{4} $th the 
distance AD on the 2nd line and 
posts a green flag. Preet runs $ \frac {1}{5} $th 
the distance AD on the eighth line 
and posts a red flag. What is the 
distance between both the flags? If 
Rashmi has to post a blue flag exactly 
halfway between the line segment 
joining the two flags, where should 
she post her flag?
\begin{figure}[h!]
  \centering
  \includegraphics[width=\columnwidth]{sc.png}
  \caption{}
\label{fig:10/7/12Fig1}
\end{figure}               
\fi
      
\item Find the ratio in which the line segment joining the points $(-3,10) \text{ and } (6,-8)$ $\text{ is divided by } (-1,6)$.
	\\
		\solution
	\input{chapters/10/7/2/4/section.tex}
\item Find the ratio in which the line segment joining $A(1,-5) \text{ and } B(-4,5)$ $\text{is divided by the x-axis}$. Also find the coordinates of the point of division.
\item If $(1,2), (4,y), (x,6), (3,5)$ are the vertices of a parallelogram taken in order, find x and y.
	\\
		\solution
	\input{chapters/10/7/2/6/para1.tex}
\item Find the coordinates of a point A, where AB is the diameter of a circle whose centre is $(2,-3) \text{ and }$ B is $(1,4)$.
	\\
		\solution
	\input{chapters/10/7/2/7/section.tex}
\item If A \text{ and } B are $(-2,-2) \text{ and } (2,-4)$, respectively, find the coordinates of P such that AP= $\frac {3}{7}$AB $\text{ and }$ P lies on the line segment AB.
	\\
		\solution
	\input{chapters/10/7/2/8/section.tex}
\item Find the coordinates of the points which divide the line segment joining $A(-2,2) \text{ and } B(2,8)$ into four equal parts.
	\\
		\solution
	\input{chapters/10/7/2/9/section.tex}
\item Find the area of a rhombus if its vertices are $(3,0), (4,5), (-1,4) \text{ and } (-2,-1)$ taken in order. [$\vec{Hint}$ : Area of rhombus =$\frac {1}{2}$(product of its diagonals)]
	\\
		\solution
	\input{chapters/10/7/2/10/cross.tex}
\item Find the position vector of a point R which divides the line joining two points $\vec{P}$
and $\vec{Q}$ whose position vectors are $\hat{i}+2\hat{j}-\hat{k}$ and $-\hat{i}+\hat{j}+\hat{k}$ respectively, in the
ratio 2 : 1
\begin{enumerate}
    \item  internally
    \item  externally
\end{enumerate}
\solution
		\input{chapters/12/10/2/15/section.tex}
\item Find the position vector of the mid point of the vector joining the points $\vec{P}$(2, 3, 4)
and $\vec{Q}$(4, 1, –2).
\\
\solution
		\input{chapters/12/10/2/16/section.tex}
\item Determine the ratio in which the line $2x+y  - 4=0$ divides the line segment joining the points $\vec{A}(2, - 2)$  and  $\vec{B}(3, 7)$.
\\
\solution
	\input{chapters/10/7/4/1/section.tex}
\item Let $\vec{A}(4, 2), \vec{B}(6, 5)$  and $ \vec{C}(1, 4)$ be the vertices of $\triangle ABC$.
\begin{enumerate}
\item The median from $\vec{A}$ meets $BC$ at $\vec{D}$. Find the coordinates of the point $\vec{D}$.
\item Find the coordinates of the point $\vec{P}$ on $AD$ such that $AP : PD = 2 : 1$.
\item Find the coordinates of points $\vec{Q}$ and $\vec{R}$ on medians $BE$ and $CF$ respectively such that $BQ : QE = 2 : 1$  and  $CR : RF = 2 : 1$.
\item What do you observe?
\item If $\vec{A}, \vec{B}$ and $\vec{C}$  are the vertices of $\triangle ABC$, find the coordinates of the centroid of the triangle.
\end{enumerate}
\solution
	\input{chapters/10/7/4/7/section.tex}
\item Find the slope of a line, which passes through the origin and the mid point of the line segment joining the points $\vec{P}$(0,-4) and $\vec{B}$(8,0).
\label{chapters/11/10/1/5}
\input{chapters/11/10/1/5/matrix.tex}
\item Find the position vector of a point R which divides the line joining two points P and Q whose position vectors are $(2\vec{a}+\vec{b})$ and $(\vec{a}-3\vec{b})$
externally in the ratio 1 : 2. Also, show that P is the mid point of the line segment RQ.\\
	\solution
%		\input{chapters/12/10/5/9/section.tex}

\end{enumerate}


\item
	\iffalse
\item To conduct Sports Day activities, in your rectangular shaped school                   
ground ABCD, lines have 
drawn with chalk powder at a                 
distance of 1m each. 100 flower pots have been placed at a distance of 1m 
from each other along AD, as shown 
in Fig. 7.12. Niharika runs $ \frac {1}{4} $th the 
distance AD on the 2nd line and 
posts a green flag. Preet runs $ \frac {1}{5} $th 
the distance AD on the eighth line 
and posts a red flag. What is the 
distance between both the flags? If 
Rashmi has to post a blue flag exactly 
halfway between the line segment 
joining the two flags, where should 
she post her flag?
\begin{figure}[h!]
  \centering
  \includegraphics[width=\columnwidth]{sc.png}
  \caption{}
\label{fig:10/7/12Fig1}
\end{figure}               
\fi
      
\item Find the ratio in which the line segment joining the points $(-3,10) \text{ and } (6,-8)$ $\text{ is divided by } (-1,6)$.
	\\
		\solution
	\iffalse
\documentclass[12pt]{article}
\usepackage{graphicx}
%\documentclass[journal,12pt,twocolumn]{IEEEtran}
\usepackage[none]{hyphenat}
\usepackage{graphicx}
\usepackage{listings}
\usepackage[english]{babel}
\usepackage{graphicx}
\usepackage{caption} 
\usepackage{hyperref}
\usepackage{booktabs}
\def\inputGnumericTable{}
\usepackage{color}                                            %%
    \usepackage{array}                                            %%
    \usepackage{longtable}                                        %%
    \usepackage{calc}                                             %%
    \usepackage{multirow}                                         %%
    \usepackage{hhline}                                           %%
    \usepackage{ifthen}
\usepackage{array}
\usepackage{amsmath}   % for having text in math mode
\usepackage{listings}
\lstset{
language=tex,
frame=single, 
breaklines=true
}
  
%Following 2 lines were added to remove the blank page at the beginning
\usepackage{atbegshi}% http://ctan.org/pkg/atbegshi
\AtBeginDocument{\AtBeginShipoutNext{\AtBeginShipoutDiscard}}
%
%New macro definitions
\newcommand{\mydet}[1]{\ensuremath{\begin{vmatrix}#1\end{vmatrix}}}
\providecommand{\brak}[1]{\ensuremath{\left(#1\right)}}
\providecommand{\norm}[1]{\left\lVert#1\right\rVert}
\newcommand{\solution}{\noindent \textbf{Solution: }}
\newcommand{\myvec}[1]{\ensuremath{\begin{pmatrix}#1\end{pmatrix}}}
\let\vec\mathbf
\begin{document}
\begin{center}
\title{\textbf{Coordinate Geometry}}
\date{\vspace{-5ex}} %Not to print date automatically
\maketitle
\end{center}
\setcounter{page}{1}
\section*{10$^{th}$ Maths - Chapter 7}
This is Problem-4 from Exercise 7.2
\begin{enumerate}
\item Find the ratio in which the line segement joining the points $\myvec{-3 \\ 10}$ and $\myvec{6\\-8}$ is divided by $\myvec{-1\\6}$.\\
\solution \\
\fi
		The input parameters for this problem are available in Table \eqref{tab:10/7/2/4-1}.
\begin{table}[ht!]
\input{chapters/10/7/2/4/tables/table.tex}
\caption{}
\label{tab:10/7/2/4-1} 
\end{table}
Using section formula,
\begin{align}
         \vec{R} &=\frac{\vec{Q}+n\vec{P}}{1+n}\label{eq:chapters/10/7/2/4/1}
\end{align}
Substituting the values of $\vec{P},\vec{Q}$ and $\vec{R}$ in \eqref{eq:chapters/10/7/2/4/1}
\begin{align}
         \myvec{-1\\6} &=\frac{{\myvec{-3\\10}+n\myvec{6\\-8}}}{1+n}\\
 &=\frac{1}{1+n}\brak{{\myvec{-3\\10}+n\myvec{6\\-8}}} \\
 &=\frac{1}{1+n}\myvec{-3+6n\\10-8n} \label{eq:chapters/10/7/2/4/4}
\end{align}
Simplifying \eqref{eq:chapters/10/7/2/4/4} yeilds,
\begin{align}
          -1 &=\frac{-3+6n}{1+n}\\
\implies          n &=\frac{2}{7}
\end{align}
Also,
\begin{align}
          6 &=\frac{10-8n}{1+n}\\
    \implies      n &=\frac{2}{7}
\end{align}
Hence the desired ratio is $\dfrac{2}{7}$.  
\begin{figure}[!h]
 \begin{center}
  \includegraphics[width=\columnwidth]{chapters/10/7/2/4/figs/fig.png}
 \end{center}
\caption{}
\label{fig:10/7/2/4Fig1}
\end{figure}

\item Find the ratio in which the line segment joining $A(1,-5) \text{ and } B(-4,5)$ $\text{is divided by the x-axis}$. Also find the coordinates of the point of division.
\item If $(1,2), (4,y), (x,6), (3,5)$ are the vertices of a parallelogram taken in order, find x and y.
	\\
		\solution
	\iffalse
\documentclass[12pt]{article}
\usepackage{graphicx}
%\documentclass[journal,12pt,twocolumn]{IEEEtran}
\def\inputGnumericTable{}
\usepackage{color}                                            %%
    \usepackage{array}                                            %%
    \usepackage{longtable}                                        %%
    \usepackage{calc}                                             %%
    \usepackage{multirow}                                         %%
    \usepackage{hhline}                                           %%
    \usepackage{ifthen}
\usepackage[none]{hyphenat}
\usepackage{graphicx}
\usepackage{listings}
\usepackage[english]{babel}
\usepackage{graphicx}
\usepackage{caption} 
\usepackage{hyperref}
\usepackage{booktabs}
\usepackage{array}
\usepackage{amsmath}   % for having text in math mode
\usepackage{listings}
\lstset{
  frame=single,
  breaklines=true
}
  
%Following 2 lines were added to remove the blank page at the beginning
\usepackage{atbegshi}% http://ctan.org/pkg/atbegshi
\AtBeginDocument{\AtBeginShipoutNext{\AtBeginShipoutDiscard}}
%


%New macro definitions
\newcommand{\mydet}[1]{\ensuremath{\begin{vmatrix}#1\end{vmatrix}}}
\providecommand{\brak}[1]{\ensuremath{\left(#1\right)}}
\providecommand{\norm}[1]{\left\lVert#1\right\rVert}
\newcommand{\solution}{\noindent \textbf{Solution: }}
\newcommand{\myvec}[1]{\ensuremath{\begin{pmatrix}#1\end{pmatrix}}}
\let\vec\mathbf

\begin{document}

\begin{center}
\title{\textbf{Properties of Parallelegram}}
\date{\vspace{-5ex}} %Not to print date automatically
\maketitle
\end{center}

\setcounter{page}{1}

\section{10$^{th}$ Maths - Chapter 7}

This is Problem-6 from Exercise 7.2

\begin{enumerate}
\item If $\vec{A}(1, 2),\vec{B}(4, x),\vec{C}(y, 6) \text{and } \vec{D}(3, 5)$ are the vertices of a parallelogram taken in order,find x and y.
\end{enumerate}
\fi

The input parameters for this problem are available in
\ref{table:chapters/10/7/2/6/tables/}.	
\begin{table}[!ht]
	\centering
	\input{chapters/10/7/2/6/tables/table.tex}
\caption{}
\label{table:chapters/10/7/2/6/tables/}	
\end{table}
From the given information,
\begin{align}
  \label{eq:chapters/10/7/2/6/tables/det2f}
	\vec{B}-\vec{A} &= \myvec{4 \\y } - \myvec{1 \\2 }  = \myvec{3 \\y-2 }\\
	\vec{C}-\vec{D} &= \myvec{x \\6 } - \myvec{3 \\5 }  = \myvec{x-3 \\1}
\end{align}
Since $ABCD$ is a parallellogram,
\begin{align}
	\myvec{3\\y-2}&=\myvec{x-3\\1}\\
	\implies x&=6 ,y=3
\end{align}
Fig. \ref{fig:chapters/10/7/2/6/Fig3}
provides a verification.
\begin{figure}[h!]
	\begin{center}
  \includegraphics[width=\columnwidth]{chapters/10/7/2/6/figs/para.pdf}
	\end{center}
\caption{}
\label{fig:chapters/10/7/2/6/Fig3}
\end{figure}


\item Find the coordinates of a point A, where AB is the diameter of a circle whose centre is $(2,-3) \text{ and }$ B is $(1,4)$.
	\\
		\solution
	\iffalse
\documentclass[12pt]{article}
\usepackage{graphicx}
\usepackage{amsmath}
\usepackage{mathtools}
\usepackage{gensymb}

\newcommand{\mydet}[1]{\ensuremath{\begin{vmatrix}#1\end{vmatrix}}}
\providecommand{\brak}[1]{\ensuremath{\left(#1\right)}}
\providecommand{\norm}[1]{\left\lVert#1\right\rVert}
\newcommand{\solution}{\noindent \textbf{Solution: }}
\newcommand{\myvec}[1]{\ensuremath{\begin{pmatrix}#1\end{pmatrix}}}
\let\vec\mathbf

\begin{document}
\begin{center}
\section*{CHAPTER 7 - COORDINATE GEOMETRY}

\end{center}
\section*{Excercise 7.2}

Q7.Find the coordinates of point $\vec{A}$, where AB is the diameter of a circle where the center is (2,-3) and $\vec{B}$ is the point (1,4):

\solution
\begin{enumerate}
\item The coordinates $\vec{B}$ and center $\vec{C}$ are given, where:
	\fi
	Let
	\begin{align}
	\vec{B} = \myvec{
		1\\
	    4\\
		},
	\vec{C} = \myvec{
	    2\\
	   -3\\
		}
	\end{align}
	\iffalse
Let us assume the coordinates of $\vec{A}$. Now, $\vec{C}$ is the center which is midpoint of line AB and $\vec{B}$ is one of the coordinate of diameter AB of a circle.
	\fi	
Hence,	
	\begin{align}
	\vec{C} &= \frac{\vec{A+B}}{2} \\
\implies	2\vec{C} &= \vec{A}+\vec{B} \\
		\text{or, }	\vec{A} &= 2\vec{C}-\vec{B} \\
	 &= \myvec{3\\-10\\}	
	\end{align}       
	See Fig. 
\ref{fig:chapters/10/7/2/7Fig}.
\begin{figure}[!h]
\begin{center}	
	\includegraphics[width=\columnwidth]{chapters/10/7/2/7/figs/Vector1.png}
\end{center}
\caption{}
\label{fig:chapters/10/7/2/7Fig}
\end{figure}
	

\item If A \text{ and } B are $(-2,-2) \text{ and } (2,-4)$, respectively, find the coordinates of P such that AP= $\frac {3}{7}$AB $\text{ and }$ P lies on the line segment AB.
	\\
		\solution
	\iffalse
\documentclass[journal,10pt,twocolumn]{article}
\usepackage{graphicx}
\usepackage[none]{hyphenat}
\usepackage{graphicx}
\usepackage{listings}
\usepackage[english]{babel}
\usepackage{graphicx}
\usepackage{caption} 
\usepackage{booktabs}
\usepackage{array}
\usepackage{amssymb} % for \because
\usepackage{amsmath}   % for having text in math mode
\usepackage{extarrows} % for Row operations arrows
\usepackage{listings}
\usepackage[utf8]{inputenc}
\lstset{
  frame=single,
  breaklines=true
}
\usepackage{hyperref}
  
%Following 2 lines were added to remove the blank page at the beginning
\usepackage{atbegshi}% http://ctan.org/pkg/atbegshi
\AtBeginDocument{\AtBeginShipoutNext{\AtBeginShipoutDiscard}}


%New macro definitions
\newcommand{\mydet}[1]{\ensuremath{\begin{vmatrix}#1\end{vmatrix}}}
\providecommand{\brak}[1]{\ensuremath{\left(#1\right)}}
\newcommand{\solution}{\noindent \textbf{Solution: }}
\newcommand{\myvec}[1]{\ensuremath{\begin{pmatrix}#1\end{pmatrix}}}
\providecommand{\norm}[1]{\left\lVert#1\right\rVert}
\providecommand{\abs}[1]{\left\vert#1\right\vert}
\let\vec\mathbf

\begin{document}

\begin{center}
\title{\textbf{VECTORS}}
\date{\vspace{-5ex}} %Not to print date automatically
\maketitle
\end{center}

\section{10$^{th}$ Maths - EXERCISE-7.2}

\begin{enumerate}
\item If A and B are $(– 2, – 2)\text{ and }(2, – 4)$, respectively, find the coordinates of P such that $AP =\frac{3}{7}AB$ and P lies on the line segment AB. 

\section{SOLUTION}
Given points are
\begin{align}
\vec{A}=\myvec{-2\\ -2} ,
\vec{B}=\myvec{2\\ -4}
\end{align}
The equation of the formula is
\fi
Using section formula, 
\begin{align}
\vec{P}&=\frac{\vec{A}+n\vec{B}}{1+n}
\end{align}
where
\begin{align}
	n =\frac{3}{4}
\end{align}
Thus,
\begin{align}
\vec{P}&=\frac{1}{1+\frac{3}{4}}\brak{\myvec{-2\\-2}+\frac{3}{4}\myvec{2\\-4}}\\
&=\myvec{\frac{-2}{7}\\[1pt] \frac{-20}{7}}
\end{align}
See Fig. 
   \ref{fig:chapters/10/7/2/8/vec.png}
\begin{figure}
   \centering 
 \includegraphics[width=\columnwidth]{chapters/10/7/2/8/figs/vec.png}
   \caption{}
   \label{fig:chapters/10/7/2/8/vec.png}
   \end{figure}

\item Find the coordinates of the points which divide the line segment joining $A(-2,2) \text{ and } B(2,8)$ into four equal parts.
	\\
		\solution
	\begin{enumerate}[label=\thesection.\arabic*,ref=\thesection.\theenumi]
\numberwithin{equation}{enumi}
\numberwithin{figure}{enumi}
\numberwithin{table}{enumi}

\item Find the coordinates of the point which divides the join of $(-1,7) \text{ and } (4,-3)$ in the ratio 2:3.
	\\
		\solution
	\input{chapters/10/7/2/1/section.tex}
\item Find the coordinates of the points of trisection of the line segment joining $(4,-1) \text{ and } (-2,3)$.
	\\
		\solution
	\input{chapters/10/7/2/2/section.tex}
\item
	\iffalse
\item To conduct Sports Day activities, in your rectangular shaped school                   
ground ABCD, lines have 
drawn with chalk powder at a                 
distance of 1m each. 100 flower pots have been placed at a distance of 1m 
from each other along AD, as shown 
in Fig. 7.12. Niharika runs $ \frac {1}{4} $th the 
distance AD on the 2nd line and 
posts a green flag. Preet runs $ \frac {1}{5} $th 
the distance AD on the eighth line 
and posts a red flag. What is the 
distance between both the flags? If 
Rashmi has to post a blue flag exactly 
halfway between the line segment 
joining the two flags, where should 
she post her flag?
\begin{figure}[h!]
  \centering
  \includegraphics[width=\columnwidth]{sc.png}
  \caption{}
\label{fig:10/7/12Fig1}
\end{figure}               
\fi
      
\item Find the ratio in which the line segment joining the points $(-3,10) \text{ and } (6,-8)$ $\text{ is divided by } (-1,6)$.
	\\
		\solution
	\input{chapters/10/7/2/4/section.tex}
\item Find the ratio in which the line segment joining $A(1,-5) \text{ and } B(-4,5)$ $\text{is divided by the x-axis}$. Also find the coordinates of the point of division.
\item If $(1,2), (4,y), (x,6), (3,5)$ are the vertices of a parallelogram taken in order, find x and y.
	\\
		\solution
	\input{chapters/10/7/2/6/para1.tex}
\item Find the coordinates of a point A, where AB is the diameter of a circle whose centre is $(2,-3) \text{ and }$ B is $(1,4)$.
	\\
		\solution
	\input{chapters/10/7/2/7/section.tex}
\item If A \text{ and } B are $(-2,-2) \text{ and } (2,-4)$, respectively, find the coordinates of P such that AP= $\frac {3}{7}$AB $\text{ and }$ P lies on the line segment AB.
	\\
		\solution
	\input{chapters/10/7/2/8/section.tex}
\item Find the coordinates of the points which divide the line segment joining $A(-2,2) \text{ and } B(2,8)$ into four equal parts.
	\\
		\solution
	\input{chapters/10/7/2/9/section.tex}
\item Find the area of a rhombus if its vertices are $(3,0), (4,5), (-1,4) \text{ and } (-2,-1)$ taken in order. [$\vec{Hint}$ : Area of rhombus =$\frac {1}{2}$(product of its diagonals)]
	\\
		\solution
	\input{chapters/10/7/2/10/cross.tex}
\item Find the position vector of a point R which divides the line joining two points $\vec{P}$
and $\vec{Q}$ whose position vectors are $\hat{i}+2\hat{j}-\hat{k}$ and $-\hat{i}+\hat{j}+\hat{k}$ respectively, in the
ratio 2 : 1
\begin{enumerate}
    \item  internally
    \item  externally
\end{enumerate}
\solution
		\input{chapters/12/10/2/15/section.tex}
\item Find the position vector of the mid point of the vector joining the points $\vec{P}$(2, 3, 4)
and $\vec{Q}$(4, 1, –2).
\\
\solution
		\input{chapters/12/10/2/16/section.tex}
\item Determine the ratio in which the line $2x+y  - 4=0$ divides the line segment joining the points $\vec{A}(2, - 2)$  and  $\vec{B}(3, 7)$.
\\
\solution
	\input{chapters/10/7/4/1/section.tex}
\item Let $\vec{A}(4, 2), \vec{B}(6, 5)$  and $ \vec{C}(1, 4)$ be the vertices of $\triangle ABC$.
\begin{enumerate}
\item The median from $\vec{A}$ meets $BC$ at $\vec{D}$. Find the coordinates of the point $\vec{D}$.
\item Find the coordinates of the point $\vec{P}$ on $AD$ such that $AP : PD = 2 : 1$.
\item Find the coordinates of points $\vec{Q}$ and $\vec{R}$ on medians $BE$ and $CF$ respectively such that $BQ : QE = 2 : 1$  and  $CR : RF = 2 : 1$.
\item What do you observe?
\item If $\vec{A}, \vec{B}$ and $\vec{C}$  are the vertices of $\triangle ABC$, find the coordinates of the centroid of the triangle.
\end{enumerate}
\solution
	\input{chapters/10/7/4/7/section.tex}
\item Find the slope of a line, which passes through the origin and the mid point of the line segment joining the points $\vec{P}$(0,-4) and $\vec{B}$(8,0).
\label{chapters/11/10/1/5}
\input{chapters/11/10/1/5/matrix.tex}
\item Find the position vector of a point R which divides the line joining two points P and Q whose position vectors are $(2\vec{a}+\vec{b})$ and $(\vec{a}-3\vec{b})$
externally in the ratio 1 : 2. Also, show that P is the mid point of the line segment RQ.\\
	\solution
%		\input{chapters/12/10/5/9/section.tex}

\end{enumerate}


\item Find the area of a rhombus if its vertices are $(3,0), (4,5), (-1,4) \text{ and } (-2,-1)$ taken in order. [$\vec{Hint}$ : Area of rhombus =$\frac {1}{2}$(product of its diagonals)]
	\\
		\solution
	\iffalse
\documentclass[12pt]{article}
\usepackage{graphicx}
%\documentclass[journal,12pt,twocolumn]{IEEEtran}
\usepackage[none]{hyphenat}
\usepackage{graphicx}
\usepackage{listings}
\usepackage[english]{babel}
\usepackage{graphicx}
\usepackage{caption} 
\usepackage{hyperref}
\usepackage{booktabs}
\def\inputGnumericTable{}
\usepackage{color}                                            %%
    \usepackage{array}                                            %%
    \usepackage{longtable}                                        %%
    \usepackage{calc}                                             %%
    \usepackage{multirow}                                         %%
    \usepackage{hhline}                                           %%
    \usepackage{ifthen}
\usepackage{array}
\usepackage{amsmath}   % for having text in math mode
\usepackage{listings}
\lstset{
language=tex,
frame=single, 
breaklines=true
}
  
%Following 2 lines were added to remove the blank page at the beginning
\usepackage{atbegshi}% http://ctan.org/pkg/atbegshi
\AtBeginDocument{\AtBeginShipoutNext{\AtBeginShipoutDiscard}}
%


%New macro definitions
\newcommand{\mydet}[1]{\ensuremath{\begin{vmatrix}#1\end{vmatrix}}}
\providecommand{\brak}[1]{\ensuremath{\left(#1\right)}}
\providecommand{\norm}[1]{\left\lVert#1\right\rVert}
\newcommand{\solution}{\noindent \textbf{Solution: }}
\newcommand{\myvec}[1]{\ensuremath{\begin{pmatrix}#1\end{pmatrix}}}
\let\vec\mathbf

\begin{document}

\begin{center}
\title{\textbf{Coordinate Geometry}}
\date{\vspace{-5ex}} %Not to print date automatically
\maketitle
\end{center}

\setcounter{page}{1}



\begin{enumerate}

\item\textbf{Problem statement :} Find the area of a rhombus of its vertices are $\myvec{3 ,0}$, $\myvec{4 ,5}$, $\myvec{-1 ,4}$ and $\myvec{-2 ,-1}$taken in order

\solution \\
\fi
The input vertices for this problem are given as
	\begin{align}
	\vec{A} = \myvec{
		3\\
		0
		},
	\vec{B} = \myvec{
		4\\
		5
		},
        \vec{C} = \myvec{
		-1\\
		4
		},
        \vec{D} = \myvec{
		-2\\
		-1
		}
	\end{align}
Since		
\begin{align}
 \vec{A-D}= \myvec{3 \\ 0} - \myvec{-2 \\-1}= \myvec{5\\1}
 \\
  \vec{B-A}= \myvec{4 \\ 5} - \myvec{3 \\0}= \myvec{1\\5}
\end{align}
the area of the rhombus is
\begin{align}
                \norm{\myvec{\vec{A-D}}\times \myvec{\vec{B-A}}}=\mydet{5 & 1\\1 & 5} = 24
\end{align}
See Fig. 
\ref{fig:chapters/10/7/2/10/gFig1}.
\begin{figure}[!h]
 \begin{center}
  \includegraphics[width=\columnwidth]{chapters/10/7/2/10/figs/fig.pdf}
 \end{center}
\caption{}
\label{fig:chapters/10/7/2/10/gFig1}
\end{figure}

\item Find the position vector of a point R which divides the line joining two points $\vec{P}$
and $\vec{Q}$ whose position vectors are $\hat{i}+2\hat{j}-\hat{k}$ and $-\hat{i}+\hat{j}+\hat{k}$ respectively, in the
ratio 2 : 1
\begin{enumerate}
    \item  internally
    \item  externally
\end{enumerate}
\solution
		\begin{enumerate}[label=\thesection.\arabic*,ref=\thesection.\theenumi]
\numberwithin{equation}{enumi}
\numberwithin{figure}{enumi}
\numberwithin{table}{enumi}

\item Find the coordinates of the point which divides the join of $(-1,7) \text{ and } (4,-3)$ in the ratio 2:3.
	\\
		\solution
	\input{chapters/10/7/2/1/section.tex}
\item Find the coordinates of the points of trisection of the line segment joining $(4,-1) \text{ and } (-2,3)$.
	\\
		\solution
	\input{chapters/10/7/2/2/section.tex}
\item
	\iffalse
\item To conduct Sports Day activities, in your rectangular shaped school                   
ground ABCD, lines have 
drawn with chalk powder at a                 
distance of 1m each. 100 flower pots have been placed at a distance of 1m 
from each other along AD, as shown 
in Fig. 7.12. Niharika runs $ \frac {1}{4} $th the 
distance AD on the 2nd line and 
posts a green flag. Preet runs $ \frac {1}{5} $th 
the distance AD on the eighth line 
and posts a red flag. What is the 
distance between both the flags? If 
Rashmi has to post a blue flag exactly 
halfway between the line segment 
joining the two flags, where should 
she post her flag?
\begin{figure}[h!]
  \centering
  \includegraphics[width=\columnwidth]{sc.png}
  \caption{}
\label{fig:10/7/12Fig1}
\end{figure}               
\fi
      
\item Find the ratio in which the line segment joining the points $(-3,10) \text{ and } (6,-8)$ $\text{ is divided by } (-1,6)$.
	\\
		\solution
	\input{chapters/10/7/2/4/section.tex}
\item Find the ratio in which the line segment joining $A(1,-5) \text{ and } B(-4,5)$ $\text{is divided by the x-axis}$. Also find the coordinates of the point of division.
\item If $(1,2), (4,y), (x,6), (3,5)$ are the vertices of a parallelogram taken in order, find x and y.
	\\
		\solution
	\input{chapters/10/7/2/6/para1.tex}
\item Find the coordinates of a point A, where AB is the diameter of a circle whose centre is $(2,-3) \text{ and }$ B is $(1,4)$.
	\\
		\solution
	\input{chapters/10/7/2/7/section.tex}
\item If A \text{ and } B are $(-2,-2) \text{ and } (2,-4)$, respectively, find the coordinates of P such that AP= $\frac {3}{7}$AB $\text{ and }$ P lies on the line segment AB.
	\\
		\solution
	\input{chapters/10/7/2/8/section.tex}
\item Find the coordinates of the points which divide the line segment joining $A(-2,2) \text{ and } B(2,8)$ into four equal parts.
	\\
		\solution
	\input{chapters/10/7/2/9/section.tex}
\item Find the area of a rhombus if its vertices are $(3,0), (4,5), (-1,4) \text{ and } (-2,-1)$ taken in order. [$\vec{Hint}$ : Area of rhombus =$\frac {1}{2}$(product of its diagonals)]
	\\
		\solution
	\input{chapters/10/7/2/10/cross.tex}
\item Find the position vector of a point R which divides the line joining two points $\vec{P}$
and $\vec{Q}$ whose position vectors are $\hat{i}+2\hat{j}-\hat{k}$ and $-\hat{i}+\hat{j}+\hat{k}$ respectively, in the
ratio 2 : 1
\begin{enumerate}
    \item  internally
    \item  externally
\end{enumerate}
\solution
		\input{chapters/12/10/2/15/section.tex}
\item Find the position vector of the mid point of the vector joining the points $\vec{P}$(2, 3, 4)
and $\vec{Q}$(4, 1, –2).
\\
\solution
		\input{chapters/12/10/2/16/section.tex}
\item Determine the ratio in which the line $2x+y  - 4=0$ divides the line segment joining the points $\vec{A}(2, - 2)$  and  $\vec{B}(3, 7)$.
\\
\solution
	\input{chapters/10/7/4/1/section.tex}
\item Let $\vec{A}(4, 2), \vec{B}(6, 5)$  and $ \vec{C}(1, 4)$ be the vertices of $\triangle ABC$.
\begin{enumerate}
\item The median from $\vec{A}$ meets $BC$ at $\vec{D}$. Find the coordinates of the point $\vec{D}$.
\item Find the coordinates of the point $\vec{P}$ on $AD$ such that $AP : PD = 2 : 1$.
\item Find the coordinates of points $\vec{Q}$ and $\vec{R}$ on medians $BE$ and $CF$ respectively such that $BQ : QE = 2 : 1$  and  $CR : RF = 2 : 1$.
\item What do you observe?
\item If $\vec{A}, \vec{B}$ and $\vec{C}$  are the vertices of $\triangle ABC$, find the coordinates of the centroid of the triangle.
\end{enumerate}
\solution
	\input{chapters/10/7/4/7/section.tex}
\item Find the slope of a line, which passes through the origin and the mid point of the line segment joining the points $\vec{P}$(0,-4) and $\vec{B}$(8,0).
\label{chapters/11/10/1/5}
\input{chapters/11/10/1/5/matrix.tex}
\item Find the position vector of a point R which divides the line joining two points P and Q whose position vectors are $(2\vec{a}+\vec{b})$ and $(\vec{a}-3\vec{b})$
externally in the ratio 1 : 2. Also, show that P is the mid point of the line segment RQ.\\
	\solution
%		\input{chapters/12/10/5/9/section.tex}

\end{enumerate}


\item Find the position vector of the mid point of the vector joining the points $\vec{P}$(2, 3, 4)
and $\vec{Q}$(4, 1, –2).
\\
\solution
		\begin{enumerate}[label=\thesection.\arabic*,ref=\thesection.\theenumi]
\numberwithin{equation}{enumi}
\numberwithin{figure}{enumi}
\numberwithin{table}{enumi}

\item Find the coordinates of the point which divides the join of $(-1,7) \text{ and } (4,-3)$ in the ratio 2:3.
	\\
		\solution
	\input{chapters/10/7/2/1/section.tex}
\item Find the coordinates of the points of trisection of the line segment joining $(4,-1) \text{ and } (-2,3)$.
	\\
		\solution
	\input{chapters/10/7/2/2/section.tex}
\item
	\iffalse
\item To conduct Sports Day activities, in your rectangular shaped school                   
ground ABCD, lines have 
drawn with chalk powder at a                 
distance of 1m each. 100 flower pots have been placed at a distance of 1m 
from each other along AD, as shown 
in Fig. 7.12. Niharika runs $ \frac {1}{4} $th the 
distance AD on the 2nd line and 
posts a green flag. Preet runs $ \frac {1}{5} $th 
the distance AD on the eighth line 
and posts a red flag. What is the 
distance between both the flags? If 
Rashmi has to post a blue flag exactly 
halfway between the line segment 
joining the two flags, where should 
she post her flag?
\begin{figure}[h!]
  \centering
  \includegraphics[width=\columnwidth]{sc.png}
  \caption{}
\label{fig:10/7/12Fig1}
\end{figure}               
\fi
      
\item Find the ratio in which the line segment joining the points $(-3,10) \text{ and } (6,-8)$ $\text{ is divided by } (-1,6)$.
	\\
		\solution
	\input{chapters/10/7/2/4/section.tex}
\item Find the ratio in which the line segment joining $A(1,-5) \text{ and } B(-4,5)$ $\text{is divided by the x-axis}$. Also find the coordinates of the point of division.
\item If $(1,2), (4,y), (x,6), (3,5)$ are the vertices of a parallelogram taken in order, find x and y.
	\\
		\solution
	\input{chapters/10/7/2/6/para1.tex}
\item Find the coordinates of a point A, where AB is the diameter of a circle whose centre is $(2,-3) \text{ and }$ B is $(1,4)$.
	\\
		\solution
	\input{chapters/10/7/2/7/section.tex}
\item If A \text{ and } B are $(-2,-2) \text{ and } (2,-4)$, respectively, find the coordinates of P such that AP= $\frac {3}{7}$AB $\text{ and }$ P lies on the line segment AB.
	\\
		\solution
	\input{chapters/10/7/2/8/section.tex}
\item Find the coordinates of the points which divide the line segment joining $A(-2,2) \text{ and } B(2,8)$ into four equal parts.
	\\
		\solution
	\input{chapters/10/7/2/9/section.tex}
\item Find the area of a rhombus if its vertices are $(3,0), (4,5), (-1,4) \text{ and } (-2,-1)$ taken in order. [$\vec{Hint}$ : Area of rhombus =$\frac {1}{2}$(product of its diagonals)]
	\\
		\solution
	\input{chapters/10/7/2/10/cross.tex}
\item Find the position vector of a point R which divides the line joining two points $\vec{P}$
and $\vec{Q}$ whose position vectors are $\hat{i}+2\hat{j}-\hat{k}$ and $-\hat{i}+\hat{j}+\hat{k}$ respectively, in the
ratio 2 : 1
\begin{enumerate}
    \item  internally
    \item  externally
\end{enumerate}
\solution
		\input{chapters/12/10/2/15/section.tex}
\item Find the position vector of the mid point of the vector joining the points $\vec{P}$(2, 3, 4)
and $\vec{Q}$(4, 1, –2).
\\
\solution
		\input{chapters/12/10/2/16/section.tex}
\item Determine the ratio in which the line $2x+y  - 4=0$ divides the line segment joining the points $\vec{A}(2, - 2)$  and  $\vec{B}(3, 7)$.
\\
\solution
	\input{chapters/10/7/4/1/section.tex}
\item Let $\vec{A}(4, 2), \vec{B}(6, 5)$  and $ \vec{C}(1, 4)$ be the vertices of $\triangle ABC$.
\begin{enumerate}
\item The median from $\vec{A}$ meets $BC$ at $\vec{D}$. Find the coordinates of the point $\vec{D}$.
\item Find the coordinates of the point $\vec{P}$ on $AD$ such that $AP : PD = 2 : 1$.
\item Find the coordinates of points $\vec{Q}$ and $\vec{R}$ on medians $BE$ and $CF$ respectively such that $BQ : QE = 2 : 1$  and  $CR : RF = 2 : 1$.
\item What do you observe?
\item If $\vec{A}, \vec{B}$ and $\vec{C}$  are the vertices of $\triangle ABC$, find the coordinates of the centroid of the triangle.
\end{enumerate}
\solution
	\input{chapters/10/7/4/7/section.tex}
\item Find the slope of a line, which passes through the origin and the mid point of the line segment joining the points $\vec{P}$(0,-4) and $\vec{B}$(8,0).
\label{chapters/11/10/1/5}
\input{chapters/11/10/1/5/matrix.tex}
\item Find the position vector of a point R which divides the line joining two points P and Q whose position vectors are $(2\vec{a}+\vec{b})$ and $(\vec{a}-3\vec{b})$
externally in the ratio 1 : 2. Also, show that P is the mid point of the line segment RQ.\\
	\solution
%		\input{chapters/12/10/5/9/section.tex}

\end{enumerate}


\item Determine the ratio in which the line $2x+y  - 4=0$ divides the line segment joining the points $\vec{A}(2, - 2)$  and  $\vec{B}(3, 7)$.
\\
\solution
	\iffalse
\documentclass[journal,12pt,twocolumn]{IEEEtran}
\usepackage{graphicx}
\graphicspath{{./chapters/10/7/4/1/figs/}}{}
\usepackage{amsmath,amssymb,amsfonts,amsthm}
\newcommand{\myvec}[1]{\ensuremath{\begin{pmatrix}#1\end{pmatrix}}}
\providecommand{\norm}[1]{\lVert#1\rVert}
\usepackage{listings}
\usepackage{watermark}
\usepackage{titlesec}
\usepackage{caption}
\let\vec\mathbf
\lstset{
frame=single, 
breaklines=true,
columns=fullflexible
}
\thiswatermark{\centering \put(0,-105.0){\includegraphics[scale=0.15]{/sdcard/IITH/vector/vectpr-4/chapters/10/7/4/1/figs/logo.png}} }
\title{\mytitle}
\title{
Assignment - Vector-4
}
\author{Surajit Sarkar}
\begin{document}
\maketitle
%\tableofcontents
\bigskip
\section{\textbf{Problem}}
Determine the ratio in which the line 2x+y–4=0 divides the line segment joining the points A(2,–2) and B(3,7).
\section{\textbf{Solution}}
\begin{table}[h]
    \centering
    \begin{tabular}{|c|c|}
       \hline
       \textbf{Symbol}&\textbf{Value}  \\
       \hline
	    $\vec{A}$ & $\myvec{2\\-2}$\\
        \hline
	    $\vec{B}$ & $\myvec{3\\7}$\\
        \hline
	    c&$4$\\
        \hline
       $\vec{n}$ & $\myvec{2\\1}$\\
       \hline
    \end{tabular}
    \caption{Parameters}
    \label{tab:my_label}
\end{table}
Given equation
\fi
The given equation can be expressed as
\begin{align}
    \myvec{2&1}\vec{x}&=4\\
\end{align}
Using section formula, the point of division 
\begin{align}
    \vec{P} = \frac{k\vec{B+A}}{k+1}
\end{align}
which upon substitution in the equation of a line yields
\begin{align}
    \implies\vec{n}^{\top}\myvec{\frac{k\vec{B+A}}{k+1}}&=c\\
    \implies k&=\frac{c-\vec{n}^{\top}\vec{A}}{\vec{n}^{\top}\vec{B}-c}\\
\end{align}
upon simplification.  Substituting numerical values, 
\begin{align}
    k=\frac{2}{9}
\end{align}
See Fig. 
\ref{fig:chapters/10/7/4/1vec}.
\begin{figure}[!h]
\centering
\includegraphics[width=\columnwidth]{chapters/10/7/4/1/figs/vec.pdf}
\caption{}
\label{fig:chapters/10/7/4/1vec}
\end{figure}


\item Let $\vec{A}(4, 2), \vec{B}(6, 5)$  and $ \vec{C}(1, 4)$ be the vertices of $\triangle ABC$.
\begin{enumerate}
\item The median from $\vec{A}$ meets $BC$ at $\vec{D}$. Find the coordinates of the point $\vec{D}$.
\item Find the coordinates of the point $\vec{P}$ on $AD$ such that $AP : PD = 2 : 1$.
\item Find the coordinates of points $\vec{Q}$ and $\vec{R}$ on medians $BE$ and $CF$ respectively such that $BQ : QE = 2 : 1$  and  $CR : RF = 2 : 1$.
\item What do you observe?
\item If $\vec{A}, \vec{B}$ and $\vec{C}$  are the vertices of $\triangle ABC$, find the coordinates of the centroid of the triangle.
\end{enumerate}
\solution
	\iffalse
\documentclass[12pt]{article}
\usepackage{graphicx}
\usepackage[none]{hyphenat}
\usepackage{graphicx}
\usepackage{listings}
\usepackage[english]{babel}
\usepackage{graphicx}
\usepackage{caption} 
\usepackage{booktabs}
\usepackage{array}
\usepackage{amssymb} % for \because
\usepackage{amsmath}   % for having text in math mode
\usepackage{extarrows} % for Row operations arrows
\usepackage{listings}
\usepackage[utf8]{inputenc}
\lstset{
  frame=single,
  breaklines=true
}
\usepackage{hyperref}
  
%Following 2 lines were added to remove the blank page at the beginning
\usepackage{atbegshi}% http://ctan.org/pkg/atbegshi
\AtBeginDocument{\AtBeginShipoutNext{\AtBeginShipoutDiscard}}


%New macro definitions
\newcommand{\mydet}[1]{\ensuremath{\begin{vmatrix}#1\end{vmatrix}}}
\providecommand{\brak}[1]{\ensuremath{\left(#1\right)}}
\newcommand{\solution}{\noindent \textbf{Solution: }}
\newcommand{\myvec}[1]{\ensuremath{\begin{pmatrix}#1\end{pmatrix}}}
\providecommand{\norm}[1]{\left\lVert#1\right\rVert}
\providecommand{\abs}[1]{\left\vert#1\right\vert}
\let\vec\mathbf

\begin{document}

\begin{center}
\title{\textbf{VECTORS}}
\date{\vspace{-5ex}} %Not to print date automatically
\maketitle
\end{center}

\section{10$^{th}$ Maths - EXERCISE-7.4}

Let A(4, 2), B(6, 5) and C(1, 4) be the vertices of $\triangle ABC$
\begin{enumerate}
\item The median from A meets BC at D. Find the coordinates of the point D.
\item Find the coordinates of the point P on AD such that $AP : PD = 2 : 1$
\item Find the coordinates of points Q and R on medians BE and CF respectively such
that $BQ : QE = 2 : 1 \text{and} CR : RF = 2 : 1.$
\item What do yo observe?
\item If $A(x_1, y_1), B(x_2, y_2) \text{and} C(x_3, y_3)$ are the vertices of $\triangle ABC$, find the coordinates of the centroid of the triangle.
\end{enumerate}

Given points are
\begin{align}
\vec{A}=\myvec{4\\ 2} ,
\vec{B}=\myvec{6\\ 5} ,
\vec{C}=\myvec{1\\ 4}
\end{align}
\fi

\begin{enumerate}
\item 
\begin{align}
\vec{D}&=\frac{\vec{B}+\vec{C}}{2}\\
&=\myvec{\frac{7}{2}\\[2pt] \frac{9}{2}}\\
\vec{E}&=\frac{\vec{A}+\vec{C}}{2}\\
&=\myvec{\frac{5}{2}\\ 3}\\
\vec{F}&=\frac{\vec{A}+\vec{B}}{2}\\
&=\myvec{5\\ \frac{7}{2}}
\end{align}

\item 
	For
$n=2$,
\begin{align}
\vec{P}&=\frac{1}{1+n}\brak{\myvec{\vec{A}+n\vec{D}}}\\
&=\frac{1}{3}\myvec{11\\11}
\end{align}

\item 
\begin{align}
\vec{Q}&=\frac{1}{1+n}\brak{\myvec{\vec{B}+n\vec{E}}}\\
&=\frac{1}{3}\myvec{11\\11}\\
\vec{R}&=\frac{1}{1+n}\brak{\myvec{\vec{C}+n\vec{F}}}\\
&=\frac{1}{3}\myvec{11\\11}\\
\end{align}

\item 
 $\vec{P},\vec{Q},\vec{R}$ are the same point.
   
\item 
\begin{align}
\vec{G}&=\frac{\vec{D}+\vec{E}+\vec{F}}{3}\\
&=\frac{1}{3}\myvec{11\\11}\\
\end{align} 
\end{enumerate}
See Fig.  
  \ref{fig:chapters/10/7/4/7/Figure}.
\begin{figure}[h!]
\centering
\includegraphics[width=\columnwidth]{chapters/10/7/4/7/figs/dj.pdf}
\caption{}
  \label{fig:chapters/10/7/4/7/Figure}
\end{figure}

\item Find the slope of a line, which passes through the origin and the mid point of the line segment joining the points $\vec{P}$(0,-4) and $\vec{B}$(8,0).
\label{chapters/11/10/1/5}
\iffalse
\documentclass[journal,12pt,twocolumn]{IEEEtran}
\usepackage{graphicx}
\graphicspath{{./figs/}}{}
\usepackage{amsmath,amssymb,amsfonts,amsthm}
\newcommand{\myvec}[1]{\ensuremath{\begin{pmatrix}#1\end{pmatrix}}}

\let\vec\mathbf

\title{
Matrix-Lines
}
\author{Jyothsna Paluchuri-FWC22059\\}
\begin{document}
\maketitle
\tableofcontents
\bigskip
\section{Problem Statement}
\fi
	\begin{figure}[!ht]
		\centering
 \includegraphics[width=\columnwidth]{chapters/11/10/1/5/figs/line.png}
		\caption{}
		\label{fig:11/10/1/5}
  	\end{figure}
	\\
	\solution
\iffalse
\section{Construction}
\begin{figure}[h]
    \centering
\includegraphics[width=\columnwidth]{line.png}
    \caption{Equation of the slope}
    \label{fig:my_label}
\end{figure}
\vspace{2cm}
\begin{table}[h]
    \centering
    \begin{tabular}{|c|c|c|c|}
       \hline
       \textbf{Symbol}&\textbf{Value}&\textbf{Description}  \\
       \hline
	    $\vec{P}$ & $\myvec{
		    0\\
		    -4}$
	    & Point on Y-axis\\
        \hline
	    $\vec{B}$ & $\myvec{8\\0}$
 & Point on X-axis\\
        \hline
	    $\vec{0}$ & $\myvec{0\\0}$
 & Origin\\
        \hline
    \end{tabular}
    \caption{Parameters}
    \label{tab:my_label}
\end{table}


\section{Solution}
Given that resultant line passes through origin and mid point of the line segment joining point P(0,-4) and B(8,0) \\
\\
\\
given ${\vec{P}}$=$\myvec{
  0\\
  -4}$
 , ${\vec{B}}$=$\myvec{
  8\\
  0}$
  
 \fi 
The mid point of $PB$ is
\begin{align}
\vec{M} &=\frac{1}{2}(\vec{P}+\vec{B})
	= \myvec{4 \\ -2}  
\end{align}
The direction vector of line joining $\vec{O}, \vec{M}$ is 
\begin{align}
\vec{m}&=\vec{O}-\vec{M}
 = -\vec{M}
\end{align}
which can be expressed as
\begin{align}
	\myvec{1 \\ -\frac{1}{2}}
\end{align}
Thus the slope is
\begin{align}
	m = -\frac{1}{2}
\end{align}
\iffalse
\textbf{The direction vector of a line expressed as}
\begin{align}
\implies\vec{m} &= \begin{pmatrix}1 \\ m \\ \end{pmatrix}
\end{align}

\textbf{By solving equation (5) and (6),we get the slope of $\vec{O}$ $\vec{M}$ line}
\begin{align}
        \boxed{m=-0.5}
 \end{align}

\section{Software}
Download the following code using,
\begin{table}[h]
    \centering
    \begin{tabular}{|c|}
    \hline \\
   https://github.com/jyothsna777/jyothsna-fwc.git  \\
         \\
\hline
    \end{tabular}
\end{table}
\\
and execute the code by using command
\begin{center}
\textbf{Python3 lines.py}\\
\end{center}

\section{Conclusion}
Hence the slope of line $\vec{O}$ $\vec{M}$ lineis $\vec{m}$=-0.5

\end{document}
\fi

\item Find the position vector of a point R which divides the line joining two points P and Q whose position vectors are $(2\vec{a}+\vec{b})$ and $(\vec{a}-3\vec{b})$
externally in the ratio 1 : 2. Also, show that P is the mid point of the line segment RQ.\\
	\solution
%		\begin{enumerate}[label=\thesection.\arabic*,ref=\thesection.\theenumi]
\numberwithin{equation}{enumi}
\numberwithin{figure}{enumi}
\numberwithin{table}{enumi}

\item Find the coordinates of the point which divides the join of $(-1,7) \text{ and } (4,-3)$ in the ratio 2:3.
	\\
		\solution
	\input{chapters/10/7/2/1/section.tex}
\item Find the coordinates of the points of trisection of the line segment joining $(4,-1) \text{ and } (-2,3)$.
	\\
		\solution
	\input{chapters/10/7/2/2/section.tex}
\item
	\iffalse
\item To conduct Sports Day activities, in your rectangular shaped school                   
ground ABCD, lines have 
drawn with chalk powder at a                 
distance of 1m each. 100 flower pots have been placed at a distance of 1m 
from each other along AD, as shown 
in Fig. 7.12. Niharika runs $ \frac {1}{4} $th the 
distance AD on the 2nd line and 
posts a green flag. Preet runs $ \frac {1}{5} $th 
the distance AD on the eighth line 
and posts a red flag. What is the 
distance between both the flags? If 
Rashmi has to post a blue flag exactly 
halfway between the line segment 
joining the two flags, where should 
she post her flag?
\begin{figure}[h!]
  \centering
  \includegraphics[width=\columnwidth]{sc.png}
  \caption{}
\label{fig:10/7/12Fig1}
\end{figure}               
\fi
      
\item Find the ratio in which the line segment joining the points $(-3,10) \text{ and } (6,-8)$ $\text{ is divided by } (-1,6)$.
	\\
		\solution
	\input{chapters/10/7/2/4/section.tex}
\item Find the ratio in which the line segment joining $A(1,-5) \text{ and } B(-4,5)$ $\text{is divided by the x-axis}$. Also find the coordinates of the point of division.
\item If $(1,2), (4,y), (x,6), (3,5)$ are the vertices of a parallelogram taken in order, find x and y.
	\\
		\solution
	\input{chapters/10/7/2/6/para1.tex}
\item Find the coordinates of a point A, where AB is the diameter of a circle whose centre is $(2,-3) \text{ and }$ B is $(1,4)$.
	\\
		\solution
	\input{chapters/10/7/2/7/section.tex}
\item If A \text{ and } B are $(-2,-2) \text{ and } (2,-4)$, respectively, find the coordinates of P such that AP= $\frac {3}{7}$AB $\text{ and }$ P lies on the line segment AB.
	\\
		\solution
	\input{chapters/10/7/2/8/section.tex}
\item Find the coordinates of the points which divide the line segment joining $A(-2,2) \text{ and } B(2,8)$ into four equal parts.
	\\
		\solution
	\input{chapters/10/7/2/9/section.tex}
\item Find the area of a rhombus if its vertices are $(3,0), (4,5), (-1,4) \text{ and } (-2,-1)$ taken in order. [$\vec{Hint}$ : Area of rhombus =$\frac {1}{2}$(product of its diagonals)]
	\\
		\solution
	\input{chapters/10/7/2/10/cross.tex}
\item Find the position vector of a point R which divides the line joining two points $\vec{P}$
and $\vec{Q}$ whose position vectors are $\hat{i}+2\hat{j}-\hat{k}$ and $-\hat{i}+\hat{j}+\hat{k}$ respectively, in the
ratio 2 : 1
\begin{enumerate}
    \item  internally
    \item  externally
\end{enumerate}
\solution
		\input{chapters/12/10/2/15/section.tex}
\item Find the position vector of the mid point of the vector joining the points $\vec{P}$(2, 3, 4)
and $\vec{Q}$(4, 1, –2).
\\
\solution
		\input{chapters/12/10/2/16/section.tex}
\item Determine the ratio in which the line $2x+y  - 4=0$ divides the line segment joining the points $\vec{A}(2, - 2)$  and  $\vec{B}(3, 7)$.
\\
\solution
	\input{chapters/10/7/4/1/section.tex}
\item Let $\vec{A}(4, 2), \vec{B}(6, 5)$  and $ \vec{C}(1, 4)$ be the vertices of $\triangle ABC$.
\begin{enumerate}
\item The median from $\vec{A}$ meets $BC$ at $\vec{D}$. Find the coordinates of the point $\vec{D}$.
\item Find the coordinates of the point $\vec{P}$ on $AD$ such that $AP : PD = 2 : 1$.
\item Find the coordinates of points $\vec{Q}$ and $\vec{R}$ on medians $BE$ and $CF$ respectively such that $BQ : QE = 2 : 1$  and  $CR : RF = 2 : 1$.
\item What do you observe?
\item If $\vec{A}, \vec{B}$ and $\vec{C}$  are the vertices of $\triangle ABC$, find the coordinates of the centroid of the triangle.
\end{enumerate}
\solution
	\input{chapters/10/7/4/7/section.tex}
\item Find the slope of a line, which passes through the origin and the mid point of the line segment joining the points $\vec{P}$(0,-4) and $\vec{B}$(8,0).
\label{chapters/11/10/1/5}
\input{chapters/11/10/1/5/matrix.tex}
\item Find the position vector of a point R which divides the line joining two points P and Q whose position vectors are $(2\vec{a}+\vec{b})$ and $(\vec{a}-3\vec{b})$
externally in the ratio 1 : 2. Also, show that P is the mid point of the line segment RQ.\\
	\solution
%		\input{chapters/12/10/5/9/section.tex}

\end{enumerate}



\end{enumerate}



\end{enumerate}


    \item Draw a quadrilateral in the Cartesian plane, whose vertices are 
    \begin{align}
        \vec{A} = \myvec{-4\\5} \quad \vec{B} = \myvec{0\\7} \\
        \vec{C} = \myvec{5\\-5} \quad \vec{D} = \myvec{-4\\-2}
    \end{align}
    Also, find its area.
\label{chapters/11/10/1/1}
   \\ 
    \solution 
\iffalse
\documentclass[journal,12pt,twocolumn]{IEEEtran}
\usepackage{setspace}
\usepackage{gensymb}
\usepackage{xcolor}
\usepackage{caption}
\singlespacing
\usepackage{siunitx}
\usepackage[cmex10]{amsmath}
\usepackage{mathtools}
\usepackage{hyperref}
\usepackage{amsthm}
\usepackage{mathrsfs}
\usepackage{txfonts}
\usepackage{stfloats}
\usepackage{cite}
\usepackage{cases}
\usepackage{subfig}
\usepackage{longtable}
\usepackage{multirow}
\usepackage{enumitem}
\usepackage{mathtools}
\usepackage{listings}
\usepackage{tikz}
\usetikzlibrary{shapes,arrows,positioning}
\usepackage{circuitikz}
\let\vec\mathbf
\DeclareMathOperator*{\Res}{Res}
\renewcommand\thesection{\arabic{section}}
\renewcommand\thesubsection{\thesection.\arabic{subsection}}
\renewcommand\thesubsubsection{\thesubsection.\arabic{subsubsection}}

\renewcommand\thesectiondis{\arabic{section}}
\renewcommand\thesubsectiondis{\thesectiondis.\arabic{subsection}}
\renewcommand\thesubsubsectiondis{\thesubsectiondis.\arabic{subsubsection}}
\hyphenation{op-tical net-works semi-conduc-tor}

\lstset{
language=Python,
frame=single, 
breaklines=true,
columns=fullflexible
}
\begin{document}
\theoremstyle{definition}
\newtheorem{theorem}{Theorem}[section]
\newtheorem{problem}{Problem}
\newtheorem{proposition}{Proposition}[section]
\newtheorem{lemma}{Lemma}[section]
\newtheorem{corollary}[theorem]{Corollary}
\newtheorem{example}{Example}[section]
\newtheorem{definition}{Definition}[section]
\newcommand{\BEQA}{\begin{eqnarray}}
\newcommand{\EEQA}{\end{eqnarray}}
\newcommand{\define}{\stackrel{\triangle}{=}}
\newcommand{\myvec}[1]{\ensuremath{\begin{pmatrix}#1\end{pmatrix}}}
\newcommand{\mydet}[1]{\ensuremath{\begin{vmatrix}#1\end{vmatrix}}}

\bibliographystyle{IEEEtran}
\providecommand{\nCr}[2]{\,^{#1}C_{#2}} % nCr
\providecommand{\nPr}[2]{\,^{#1}P_{#2}} % nPr
\providecommand{\mbf}{\mathbf}
\providecommand{\pr}[1]{\ensuremath{\Pr\left(#1\right)}}
\providecommand{\qfunc}[1]{\ensuremath{Q\left(#1\right)}}
\providecommand{\sbrak}[1]{\ensuremath{{}\left[#1\right]}}
\providecommand{\lsbrak}[1]{\ensuremath{{}\left[#1\right.}}
\providecommand{\rsbrak}[1]{\ensuremath{{}\left.#1\right]}}
\providecommand{\brak}[1]{\ensuremath{\left(#1\right)}}
\providecommand{\lbrak}[1]{\ensuremath{\left(#1\right.}}
\providecommand{\rbrak}[1]{\ensuremath{\left.#1\right)}}
\providecommand{\cbrak}[1]{\ensuremath{\left\{#1\right\}}}
\providecommand{\lcbrak}[1]{\ensuremath{\left\{#1\right.}}
\providecommand{\rcbrak}[1]{\ensuremath{\left.#1\right\}}}
\theoremstyle{remark}
\newtheorem{rem}{Remark}
\newcommand{\sgn}{\mathop{\mathrm{sgn}}}
\newcommand{\rect}{\mathop{\mathrm{rect}}}
\newcommand{\sinc}{\mathop{\mathrm{sinc}}}
\providecommand{\abs}[1]{\left\vert#1\right\vert}
\providecommand{\res}[1]{\Res\displaylimits_{#1}} 
\providecommand{\norm}[1]{\left\Vert#1\right\Vert}
\providecommand{\mtx}[1]{\mathbf{#1}}
\providecommand{\mean}[1]{E\left[ #1 \right]}
\providecommand{\fourier}{\overset{\mathcal{F}}{ \rightleftharpoons}}
\providecommand{\ztrans}{\overset{\mathcal{Z}}{ \rightleftharpoons}}
\providecommand{\system}[1]{\overset{\mathcal{#1}}{ \longleftrightarrow}}
\newcommand{\solution}{\noindent \textbf{Solution: }}
\providecommand{\dec}[2]{\ensuremath{\overset{#1}{\underset{#2}{\gtrless}}}}
\let\StandardTheFigure\thefigure
\def\putbox#1#2#3{\makebox[0in][l]{\makebox[#1][l]{}\raisebox{\baselineskip}[0in][0in]{\raisebox{#2}[0in][0in]{#3}}}}
     \def\rightbox#1{\makebox[0in][r]{#1}}
     \def\centbox#1{\makebox[0in]{#1}}
     \def\topbox#1{\raisebox{-\baselineskip}[0in][0in]{#1}}
     \def\midbox#1{\raisebox{-0.5\baselineskip}[0in][0in]{#1}}

\vspace{3cm}
\title{Straight Lines Assignment}
\author{Gautam Singh}
\maketitle
\bigskip

\begin{abstract}
    This document contains the solution to Question 1 of Exercise 1 in Chapter
    10 of the class 11 NCERT textbook.
\end{abstract}

\begin{enumerate}
\fi
		The points are plotted in Fig. \ref{fig:11/10/1/1quad}. The plot is 
    generated using the Python code \texttt{codes/quad.py}.

    The area vector (denoted by $\vec{R_X}$ for region $X$) of the quadrilateral 
    is perpendicular to the plane of the quadrilateral and its orientation is 
    assumed to be in the positive $z$-direction here.
    \begin{align}
        &\vec{R_{ABCD}} = \vec{R_{ABC}} + \vec{R_{ACD}} \\
        &= \frac{1}{2}\brak{\brak{\vec{B}-\vec{A}}\times\brak{\vec{C}-\vec{A}} + 
        \brak{\vec{C}-\vec{A}}\times\brak{\vec{D}-\vec{A}}} \\
        &= \frac{1}{2}\brak{\brak{\vec{C}-\vec{A}}\times
        \brak{\vec{D}-\vec{A}+\vec{A}-\vec{B}}} \\
        &= \frac{1}{2}\brak{\brak{\vec{C}-\vec{A}}\times\brak{\vec{D}-\vec{B}}} \\
        \label{eq:11/10/1/1area-diag} 
    \end{align}
    Thus the area of quadrilateral ABCD is
    \begin{align}
        \textrm{ar}\brak{ABCD} &= \norm{\vec{R_{ABCD}}} \\
                               &= \frac{1}{2}\norm{\brak{\vec{C}-\vec{A}}\times\brak{\vec{D}-\vec{B}}} \\ 
                               &= \frac{1}{2}\mydet{9&-4\\-10&-9} \\
                               &= 60.5\ \textrm{sq. units.}
        \label{eq:11/10/1/1ans}
    \end{align}
    \begin{figure}[!htb]
        \centering
        \includegraphics[width=\columnwidth]{chapters/11/10/1/1/figs/quad.png}
        \caption{Plot of quadrilateral $ABCD$}
        \label{fig:11/10/1/1quad}
    \end{figure}

\item Find the area of region bounded by the triangle whose
	vertices are $(1, 0), (2, 2) \text{ and } (3, 1)$. 
\item Find the area of region bounded by the triangle whose vertices
	are $(– 1, 0), (1, 3) \text{ and } (3, 2)$. 
\item Find the area of the $\triangle ABC$, coordinates of whose vertices are $\vec{A}(2, 0), \vec{B}(4, 5), \text{ and } \vec{C}(6, 3)$.


\item 
\documentclass[journal,12pt,twocolumn]{IEEEtran}
\usepackage{setspace}
\usepackage{gensymb}
\usepackage{xcolor}
\usepackage{caption}
\singlespacing
\usepackage{siunitx}
\usepackage[cmex10]{amsmath}
\usepackage{mathtools}
\usepackage{hyperref}
\usepackage{amsthm}
\usepackage{mathrsfs}
\usepackage{txfonts}
\usepackage{stfloats}
\usepackage{cite}
\usepackage{cases}
\usepackage{subfig}
\usepackage{longtable}
\usepackage{multirow}
\usepackage{enumitem}
\usepackage{bm}
\usepackage{mathtools}
\usepackage{listings}
\usepackage{tikz}
\usetikzlibrary{shapes,arrows,positioning}
\usepackage{circuitikz}
\renewcommand{\vec}[1]{\boldsymbol{\mathbf{#1}}}
\DeclareMathOperator*{\Res}{Res}
\renewcommand\thesection{\arabic{section}}
\renewcommand\thesubsection{\thesection.\arabic{subsection}}
\renewcommand\thesubsubsection{\thesubsection.\arabic{subsubsection}}

\renewcommand\thesectiondis{\arabic{section}}
\renewcommand\thesubsectiondis{\thesectiondis.\arabic{subsection}}
\renewcommand\thesubsubsectiondis{\thesubsectiondis.\arabic{subsubsection}}
\hyphenation{op-tical net-works semi-conduc-tor}

\lstset{
language=Python,
frame=single, 
breaklines=true,
columns=fullflexible
}
\begin{document}
\theoremstyle{definition}
\newtheorem{theorem}{Theorem}[section]
\newtheorem{problem}{Problem}
\newtheorem{proposition}{Proposition}[section]
\newtheorem{lemma}{Lemma}[section]
\newtheorem{corollary}[theorem]{Corollary}
\newtheorem{example}{Example}[section]
\newtheorem{definition}{Definition}[section]
\newcommand{\BEQA}{\begin{eqnarray}}
        \newcommand{\EEQA}{\end{eqnarray}}
\newcommand{\define}{\stackrel{\triangle}{=}}
\newcommand{\myvec}[1]{\ensuremath{\begin{pmatrix}#1\end{pmatrix}}}
\newcommand{\mydet}[1]{\ensuremath{\begin{vmatrix}#1\end{vmatrix}}}
\bibliographystyle{IEEEtran}
\providecommand{\nCr}[2]{\,^{#1}C_{#2}} % nCr
\providecommand{\nPr}[2]{\,^{#1}P_{#2}} % nPr
\providecommand{\mbf}{\mathbf}
\providecommand{\pr}[1]{\ensuremath{\Pr\left(#1\right)}}
\providecommand{\qfunc}[1]{\ensuremath{Q\left(#1\right)}}
\providecommand{\sbrak}[1]{\ensuremath{{}\left[#1\right]}}
\providecommand{\lsbrak}[1]{\ensuremath{{}\left[#1\right.}}
\providecommand{\rsbrak}[1]{\ensuremath{{}\left.#1\right]}}
\providecommand{\brak}[1]{\ensuremath{\left(#1\right)}}
\providecommand{\lbrak}[1]{\ensuremath{\left(#1\right.}}
\providecommand{\rbrak}[1]{\ensuremath{\left.#1\right)}}
\providecommand{\cbrak}[1]{\ensuremath{\left\{#1\right\}}}
\providecommand{\lcbrak}[1]{\ensuremath{\left\{#1\right.}}
\providecommand{\rcbrak}[1]{\ensuremath{\left.#1\right\}}}
\theoremstyle{remark}
\newtheorem{rem}{Remark}
\newcommand{\sgn}{\mathop{\mathrm{sgn}}}
\newcommand{\rect}{\mathop{\mathrm{rect}}}
\newcommand{\sinc}{\mathop{\mathrm{sinc}}}
\providecommand{\abs}[1]{\left\vert#1\right\vert}
\providecommand{\res}[1]{\Res\displaylimits_{#1}}
\providecommand{\norm}[1]{\lVert#1\rVert}
\providecommand{\mtx}[1]{\mathbf{#1}}
\providecommand{\mean}[1]{E\left[ #1 \right]}
\providecommand{\fourier}{\overset{\mathcal{F}}{ \rightleftharpoons}}
\providecommand{\ztrans}{\overset{\mathcal{Z}}{ \rightleftharpoons}}
\providecommand{\system}[1]{\overset{\mathcal{#1}}{ \longleftrightarrow}}
\newcommand{\solution}{\noindent \textbf{Solution: }}
\providecommand{\dec}[2]{\ensuremath{\overset{#1}{\underset{#2}{\gtrless}}}}
\let\StandardTheFigure\thefigure
\def\putbox#1#2#3{\makebox[0in][l]{\makebox[#1][l]{}\raisebox{\baselineskip}[0in][0in]{\raisebox{#2}[0in][0in]{#3}}}}
\def\rightbox#1{\makebox[0in][r]{#1}}
\def\centbox#1{\makebox[0in]{#1}}
\def\topbox#1{\raisebox{-\baselineskip}[0in][0in]{#1}}
\def\midbox#1{\raisebox{-0.5\baselineskip}[0in][0in]{#1}}

\vspace{3cm}
\title{12.11.3.9}
\author{Lokesh Surana}
\maketitle
\section*{Class 12, Chapter 11, Exercise 4.19}

Q. Find the vector equation of the line passing through $\myvec{1\\2\\3}$ and parallel to the planes $\myvec{1\\-1\\2}^{\top}\vec{r} = 5$ and $\myvec{3\\1\\1}^{\top}\vec{r} = 6$.  

\solution
The line equations are given as
\begin{align}
    \label{eq:1} \vec{r} = \vec{A} + \lambda\vec{m}
\end{align}
where $\vec{m}$ is the direction vector of the line and $\vec{A}$ is any point on the line. 

The planes are given as
\begin{align}
    \label{eq:2} {P}_1: \myvec{1&-1&2}\vec{r} = 5 \\
    \label{eq:3} \implies \vec{n}_1 = \myvec{1\\-1\\2}\\
    \label{eq:4} {P}_2: \myvec{3&1&1}\vec{r} = 6 \\
    \label{eq:5} \implies \vec{n}_2 = \myvec{3\\1\\1}
\end{align}
The expected line is parallel to both the planes, then the direction vector of the line must be perpendicular to both the normal vectors. This means that

\begin{align}
    \label{eq:6} \vec{n}_1^{\top}\vec{m} = 0 \\
    \label{eq:7} \vec{n}_2^{\top}\vec{m} = 0 \\
    \label{eq:8} \implies \myvec{1&-1&2 \\ 3&1&1}\vec{m} = 0
\end{align}

Let's reduce the matrix from equation \eqref{eq:8} to row-echelon form:
\begin{align}
    \label{eq:9} \myvec{1&-1&2 \\ 3&1&1} &\xleftrightarrow[]{R_2\rightarrow -\frac{3}{4}{R_1} + \frac{1}{4}{R_2}} \myvec{1&-1&2 \\ 0&1&-\frac{5}{4}}\\
    \label{eq:10} \myvec{1&-1&2 \\ 0&1&-\frac{5}{4}} &\xleftrightarrow[]{R_1\rightarrow {R_1} + {R_2}} \myvec{1&0&\frac{3}{4} \\ 0&1&-\frac{5}{4}}
\end{align}

Using \eqref{eq:8}, \eqref{eq:9} and \eqref{eq:10}, we get:
\begin{align}
    \implies \myvec{1&0&\frac{3}{4} \\ 0&1&-\frac{5}{4}}\vec{m} &= 0 \\
    \implies \myvec{{m}_1\\{m}_2\\{m}_3} &= \myvec{-\frac{3}{4}{m}_3\\\frac{5}{4}{m}_3\\{m}_3} \\
    \implies \myvec{{m}_1\\{m}_2\\{m}_3} &= {m}_3\myvec{-\frac{3}{4}\\\frac{5}{4}\\1} \\
    \implies \vec{m} = \myvec{-3\\5\\4}
\end{align}

It is given that line passes through point $\myvec{1\\2\\3}$, so the final equation of line implies
\begin{align}
    \vec{r} = \myvec{1\\2\\3} + \lambda\myvec{-3\\5\\4} \\
\end{align}

\end{document}
\end{enumerate}

