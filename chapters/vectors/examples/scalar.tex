\begin{enumerate}[label=\thesection.\arabic*,ref=\thesection.\theenumi]
\numberwithin{equation}{enumi}
\numberwithin{figure}{enumi}
\numberwithin{table}{enumi}
\item Find the angle between two vectors $\overrightarrow{a}$ and $\overrightarrow {b} $ with magnitudes $\sqrt{3}$ and 2 respectively having $\overrightarrow {a}.\overrightarrow {b}=\sqrt{6}$.
	\\
	\solution
		\iffalse
\documentclass[10pt]{article}
\usepackage{graphicx}
\usepackage[none]{hyphenat}
\usepackage{graphicx}
\usepackage{listings}
\usepackage[english]{babel}
\usepackage{siunitx}
\usepackage{graphicx}
\usepackage{caption} 
\usepackage{booktabs}
\usepackage{array}
\usepackage{amssymb} % for \because
\usepackage{amsmath}   % for having text in math mode
\usepackage{extarrows} % for Row operations arrows
\usepackage{listings}
\usepackage[utf8]{inputenc}
\lstset{
  frame=single,
  breaklines=true
}
\usepackage{hyperref}
  
%Following 2 lines were added to remove the blank page at the beginning
\usepackage{atbegshi}% http://ctan.org/pkg/atbegshi
\AtBeginDocument{\AtBeginShipoutNext{\AtBeginShipoutDiscard}}


%New macro definitions
\newcommand{\mydet}[1]{\ensuremath{\begin{vmatrix}#1\end{vmatrix}}}
\providecommand{\brak}[1]{\ensuremath{\left(#1\right)}}
\newcommand{\solution}{\noindent \textbf{Solution: }}
\newcommand{\myvec}[1]{\ensuremath{\begin{pmatrix}#1\end{pmatrix}}}
\providecommand{\norm}[1]{\left\lVert#1\right\rVert}
\providecommand{\abs}[1]{\left\vert#1\right\vert}
\let\vec\mathbf{}
\begin{document}

\begin{center}
\title{\textbf{VECTORS}}
\date{\vspace{-5ex}} %Not to print date automatically
\maketitle
\end{center}

\section{12$^{th}$ Maths - EXERCISE-10.3}

\begin{enumerate}
\item Find the angle between two vectors $\overrightarrow{a}$ and $\overrightarrow {b} $ with magnitudes $\sqrt{3}$ and 2 respectively having $\overrightarrow {a}.\overrightarrow {b}=\sqrt{6}$.\\  

\solution
\fi
From the given information,
\begin{align}
\norm{\vec{a}}&=\sqrt{3}\\
\norm{\vec{b}}&= 2\\
{\vec{a}^{\top}}{\vec{b}}&=\sqrt{6}  
\end{align}
Thus,
\begin{align}
\cos\theta&=\frac{{\vec{a}^{\top}}{\vec{b}}}{\norm{\vec{a}}\norm{\vec{b}}}\\
&=\frac{1}{\sqrt{2}}\\
\implies\theta&={45}\degree
\end{align}

\item Find the angle between the the vectors $\hat{i}-2\hat{j}+3\hat{k}$ and $3\hat{i}-2\hat{j}+\hat{k}$.
	\\
	\solution
		\iffalse
\documentclass[journal,12pt]{article}
\usepackage{graphicx}
\usepackage[none]{hyphenat}
\usepackage{graphicx}
\usepackage{listings}
\usepackage[english]{babel}
\usepackage{siunitx}
\usepackage{graphicx}
\usepackage{caption} 
\usepackage{booktabs}
\usepackage{array}
\usepackage{amssymb} % for \because
\usepackage{amsmath}   % for having text in math mode
\usepackage{extarrows} % for Row operations arrows
\usepackage{listings}
\usepackage[utf8]{inputenc}
\lstset{
  frame=single,
  breaklines=true
}
\usepackage{hyperref}
  
%Following 2 lines were added to remove the blank page at the beginning
\usepackage{atbegshi}% http://ctan.org/pkg/atbegshi
\AtBeginDocument{\AtBeginShipoutNext{\AtBeginShipoutDiscard}}


%New macro definitions
\newcommand{\mydet}[1]{\ensuremath{\begin{vmatrix}#1\end{vmatrix}}}
\providecommand{\brak}[1]{\ensuremath{\left(#1\right)}}
\newcommand{\solution}{\noindent \textbf{Solution: }}
\newcommand{\myvec}[1]{\ensuremath{\begin{pmatrix}#1\end{pmatrix}}}
\providecommand{\norm}[1]{\left\lVert#1\right\rVert}
\providecommand{\abs}[1]{\left\vert#1\right\vert}
\let\vec\mathbf

\begin{document}

\begin{center}
\title{\textbf{VECTORS}}
\date{\vspace{-5ex}} %Not to print date automatically
\maketitle
\end{center}

\section*{$12^{th}$ Maths - EXERCISE-10.3}

\begin{enumerate}
\item Find the angle between the vectors $\\ \overrightarrow{a}=\hat{i}-2\hat{j}+3\hat{k}$ and $\overrightarrow{b}=3\hat{i}-2\hat{j}+\hat{k}$  
\section*{solution}
\fi
Let the given points be
\begin{align}
	\vec{a} &= \myvec{1\\-2\\3} , \vec{b} = \myvec{3\\ -2 \\ 1},
\end{align}
Since 
\begin{align}
\vec{a}^{\top}\vec{b}=10, \,
\norm{\vec{a}}=\sqrt{14}, \, 
	\norm{\vec{b}}&=\sqrt{14}, 
\label{eq:chapters/12/10/3/2/5}
\\
\cos\theta=\frac{\vec{a}^{\top}\vec{b}}{\norm{\vec{a}}\norm{\vec{b}}}
	&= \frac{5}{7}
	\\
	\implies \theta&= \cos^{-1}\frac{5}{7}
\end{align}

\item Find $\abs{\overrightarrow {a}}$ and $\abs{\overrightarrow {b}}$,if ($\overrightarrow {a}+\overrightarrow {b}).(\overrightarrow {a}-\overrightarrow {b})=8$ and $\abs{\overrightarrow {a}}=8\abs{\overrightarrow {b}}$.
	\\
	\solution
		\iffalse
\documentclass[12pt]{article}
\usepackage{graphicx}
%\documentclass[journal,12pt,twocolumn]{IEEEtran}
\usepackage[none]{hyphenat}
\usepackage{graphicx}
\usepackage{listings}
\usepackage[english]{babel}
\usepackage{graphicx}
\usepackage{caption} 
\usepackage{hyperref}
\usepackage{booktabs}
\usepackage{array}
\usepackage{amsmath}   % for having text in math mode
\usepackage{listings}
\lstset{
  frame=single,
  breaklines=true
}
  
%Following 2 lines were added to remove the blank page at the beginning
\usepackage{atbegshi}% http://ctan.org/pkg/atbegshi
\AtBeginDocument{\AtBeginShipoutNext{\AtBeginShipoutDiscard}}
%


%New macro definitions
\newcommand{\mydet}[1]{\ensuremath{\begin{vmatrix}#1\end{vmatrix}}}
\providecommand{\brak}[1]{\ensuremath{\left(#1\right)}}
\providecommand{\norm}[1]{\left\lVert#1\right\rVert}
\newcommand{\solution}{\noindent \textbf{Solution: }}
\newcommand{\myvec}[1]{\ensuremath{\begin{pmatrix}#1\end{pmatrix}}}
\let\vec\mathbf

\begin{document}

\begin{center}
\title{\textbf{Vector Dot Product}}
\date{\vspace{-5ex}} %Not to print date automatically
\maketitle
\end{center}
\setcounter{page}{1}

\section{12$^{th}$ Maths - Chapter 10}
This is Problem-9 from Exercise 10.3
\begin{enumerate}
\item Find $\norm{\vec{x}}$, if for a unit vector $\vec{a}$, $\brak{\vec{x}-\vec{a}}.\brak{\vec{x}+\vec{a}} = 12$.\\
	\fi
\solution 
From the given information,
\begin{align}
  \label{eq:12/10/3/9det2f}
  \brak{\vec{x}-\vec{a}}^\top\brak{\vec{x}+\vec{a}} &= 12 \\
  \implies \vec{x}^\top\vec{x} - \vec{a}^\top\vec{x} + \vec{x}^\top\vec{a} - \vec{a}^\top\vec{a} &= 12 \\
  \implies \norm{\vec{x}}^{2} - \norm{\vec{a}}^{2} &= 12 \\
\implies   \norm{\vec{x}}^{2} - 1 &= 12  \\
	\text{or, }  
	\norm{\vec{x}} &= \sqrt{13}
\end{align}

\item Evaluate the product(3$\overrightarrow {a}-5\overrightarrow {b}).(2\overrightarrow {a}+7\overrightarrow {b}$).
	\\
	\solution
		\iffalse
\documentclass[12pt]{article}\usepackage{graphicx}
\graphicspath{{./figs/}}{}

\usepackage{amsmath,amssymb,amsfonts,amsthm}
\newcommand{\myvec}[1]{\ensuremath{\begin{pmatrix}#1\end{pmatrix}}}
%\providecommand{\norm}[1]{\lVert#1\rVert}3
\usepackage{listings}
\usepackage{watermark}
\usepackage{titlesec}
\usepackage{caption}
\let\vec\mathbf
\lstset{
frame=single, 
breaklines=true,
columns=fullflexible
}
\usepackage{atbegshi}% http://ctan.org/pkg/atbegshi
\AtBeginDocument{\AtBeginShipoutNext{\AtBeginShipoutDiscard}}
\let\vec\mathbf
\providecommand{\norm}[1]{\left\lVert#1\right\rVert}
\providecommand{\qfunc}[1]{\ensuremath{Q\left(#1\right)}}
\providecommand{\sbrak}[1]{\ensuremath{{}\left[#1\right]}}
\providecommand{\lsbrak}[1]{\ensuremath{{}\left[#1\right.}}
\providecommand{\rsbrak}[1]{\ensuremath{{}\left.#1\right]}}
\providecommand{\brak}[1]{\ensuremath{\left(#1\right)}}
\providecommand{\lbrak}[1]{\ensuremath{\left(#1\right.}}
\providecommand{\rbrak}[1]{\ensuremath{\left.#1\right)}}
\providecommand{\cbrak}[1]{\ensuremath{\left\{#1\right\}}}
\providecommand{\lcbrak}[1]{\ensuremath{\left\{#1\right.}}
\providecommand{\rcbrak}[1]{\ensuremath{\left.#1\right\}}}
\newcommand{\solution}{\noindent \textbf{Solution: }}
\newcommand{\mydet}[1]{\ensuremath{\begin{vmatrix}#1\end{vmatrix}}}
\title{\mytitle}
\begin{document}
\begin{center}
\title{\textbf{VECTOR ALGEBRA}}
\maketitle
\end{center}
\begin{enumerate}
\item\textbf{Problem statement :} Evaluate the product $\brak{3\overrightarrow{a}-5\overrightarrow{b}}\cdot\brak{2\overrightarrow{a}+7\overrightarrow{b}}$
\solution
\fi
\begin{multline}
    \brak{3\vec{a}-5\vec{b}}^{\top}\brak{2\vec{a}+7\vec{b}}= \brak{3\vec{a}^{\top}}\brak{2\vec{a}}+\brak{3\vec{a}^{\top}}\brak{7\vec{b}}-\brak{5\vec{b}^{\top}}\brak{2\vec{a}}-\brak{5\vec{b}^{\top}}\brak{7\vec{b}}
    \\
     = 6\norm{\vec{a}}^2 +21\vec{a}^{\top}\vec{b}-10\vec{b}^{\top}\vec{a}-35\norm{\vec{b}}^2 
     =6\norm{\vec{a}}^2-35\norm{\vec{b}}^2+11\vec{a}^{\top}\vec{b}
\end{multline}

\item Find the magnitude of two vectors $\overrightarrow {a}$ and $\overrightarrow {b}$, having the same magnitude and such that the angle between them is $60\degree$ and their scalar product is $\frac{1}{2}$
	\\
	\solution
		\iffalse
\documentclass[10pt]{article}
\usepackage{graphicx}
\usepackage[none]{hyphenat}
\usepackage{listings}
\usepackage[english]{babel}
\usepackage{siunitx}
\usepackage{caption}
\usepackage{booktabs}
\usepackage{array}
\usepackage{extarrows}
\usepackage{enumerate}
\usepackage{enumitem}
\usepackage{amsmath}
\usepackage{commath}
\usepackage{gensymb}
\usepackage{amssymb}
\usepackage{multicol}
\usepackage[utf8]{inputenc}
\lstset{
	frame=single,
	breaklines=true
}
\usepackage{hyperref}
%\usepackage[margin=0.8in]{geometry}
%\usepackage{exsheets}% also loads the `tasks' package
\usepackage{atbegshi}
\AtBeginDocument{\AtBeginShipoutNext{\AtBeginShipoutDiscard}}

%new macro definitions
\newcommand{\mydet}[1]{\ensuremath{\begin{vmatrix}#1\end{vmatrix}}}
\providecommand{\brak}[1]{\ensuremath{\left(#1\right)}}
\newcommand{\solution}{\noindent \textbf{Solution: }}
\newcommand{\myvet}[1]{\ensuremath{\begin{pmatrix}#1\end{pmatrix}}}
\providecommand{\norm}[1]{\left\1Vert#1\right\rVert}
\let\vec\mathbf{}


%\SetEnumitemKey{twocol}{
%	before=\raggedcolumns\begin{multicols}{2},
%	after=\end{multicols}}
%\SetEnumitemKey{fourcol}{
%	before=\raggedcolumns\begin{multicols}{4},
%	after=\end{multicols}}	


\begin{document}
\begin{center}
\title{\textbf{VECTORS}}
\date{\vspace{-5ex}}
\maketitle
\end{center}
\section*{12$^{th}$Math - Chapter 10}
This is Problem-8 from Exercise 10.3\\\\
Find the magnitude of two vectors $\overrightarrow{a}$ and $\overrightarrow{b}$, having the same magnitude and such that the angle between them is 60 $\degree$ and their scalar product is $\dfrac{1}{2}$.

\solution
\fi
Given 
\begin{align}
\norm{\vec{a}}\norm{\vec{b}}&=\frac{{{\vec{a}}^{\top}}{\vec{b}}}{\cos\theta}\\
\implies 
\norm{\vec{a}}&=\sqrt{\frac{{\vec{a}}^{\top}{\vec{b}}}{\cos\theta}}\\
\end{align}
since 
\begin{align}
\norm{\vec{a}}=\norm{\vec{b}}
\end{align}
Substituting numerical values,
\begin{align}
\norm{\vec{a}}
= \norm{\vec{b}}=1
\end{align}

\item Find $\abs{\overrightarrow {x}}$,if for a unit vector $\overrightarrow {a},(\overrightarrow {x}-\overrightarrow {a}).(\overrightarrow {x}+\overrightarrow {a}$)=12.
	\\
		\iffalse
\documentclass[12pt]{article}
\usepackage{graphicx}
%\documentclass[journal,12pt,twocolumn]{IEEEtran}
\usepackage[none]{hyphenat}
\usepackage{graphicx}
\usepackage{listings}
\usepackage[english]{babel}
\usepackage{graphicx}
\usepackage{caption} 
\usepackage{hyperref}
\usepackage{booktabs}
\usepackage{array}
\usepackage{amsmath}   % for having text in math mode
\usepackage{listings}
\lstset{
  frame=single,
  breaklines=true
}
  
%Following 2 lines were added to remove the blank page at the beginning
\usepackage{atbegshi}% http://ctan.org/pkg/atbegshi
\AtBeginDocument{\AtBeginShipoutNext{\AtBeginShipoutDiscard}}
%


%New macro definitions
\newcommand{\mydet}[1]{\ensuremath{\begin{vmatrix}#1\end{vmatrix}}}
\providecommand{\brak}[1]{\ensuremath{\left(#1\right)}}
\providecommand{\norm}[1]{\left\lVert#1\right\rVert}
\newcommand{\solution}{\noindent \textbf{Solution: }}
\newcommand{\myvec}[1]{\ensuremath{\begin{pmatrix}#1\end{pmatrix}}}
\let\vec\mathbf

\begin{document}

\begin{center}
\title{\textbf{Vector Dot Product}}
\date{\vspace{-5ex}} %Not to print date automatically
\maketitle
\end{center}
\setcounter{page}{1}

\section{12$^{th}$ Maths - Chapter 10}
This is Problem-9 from Exercise 10.3
\begin{enumerate}
\item Find $\norm{\vec{x}}$, if for a unit vector $\vec{a}$, $\brak{\vec{x}-\vec{a}}.\brak{\vec{x}+\vec{a}} = 12$.\\
	\fi
\solution 
From the given information,
\begin{align}
  \label{eq:12/10/3/9det2f}
  \brak{\vec{x}-\vec{a}}^\top\brak{\vec{x}+\vec{a}} &= 12 \\
  \implies \vec{x}^\top\vec{x} - \vec{a}^\top\vec{x} + \vec{x}^\top\vec{a} - \vec{a}^\top\vec{a} &= 12 \\
  \implies \norm{\vec{x}}^{2} - \norm{\vec{a}}^{2} &= 12 \\
\implies   \norm{\vec{x}}^{2} - 1 &= 12  \\
	\text{or, }  
	\norm{\vec{x}} &= \sqrt{13}
\end{align}

\item If the vertices A,B,C of a triangle ABC are (1,2,3),(-1,0,0)(0,1,2), respectively , then find  $\angle{ABC}. [\angle{ABC}$ is the angle between the vectors $\overrightarrow{BA}$ and $\overrightarrow{BC}$].
	\\
	\solution
		\iffalse
\documentclass[12pt]{article}
\usepackage{graphicx}
\usepackage{amsmath}
\usepackage{mathtools}
\usepackage{gensymb}

\newcommand{\mydet}[1]{\ensuremath{\begin{vmatrix}#1\end{vmatrix}}}
\providecommand{\brak}[1]{\ensuremath{\left(#1\right)}}
\providecommand{\norm}[1]{\left\lVert#1\right\rVert}
\newcommand{\solution}{\noindent \textbf{Solution: }}
\newcommand{\myvec}[1]{\ensuremath{\begin{pmatrix}#1\end{pmatrix}}}
\let\vec\mathbf

\begin{document}
\begin{center}
\textbf\large{CHAPTER-7 \\ COORDINATE GEOMETRY}

\end{center}
\section*{Excercise 7.1}

Q6.Name the type of quadilateral formed,if any, by the following points, and give reasons for your answer:
\begin{enumerate}
	\item $\brak{-1,-2}, \brak{1,0}, \brak{-1,2}, \brak{-3,0}$ 
	\item $\brak{-3,5}, \brak{3,1}, \brak{0,3}, \brak{-1,-4}$
	\item $\brak{4,5}, \brak{7,6}, \brak{4,3}, \brak{1,2}$
\end{enumerate}
\solution
\fi
\begin{enumerate}
\item The coordinates are given as
	\begin{align}
	\vec{A} = \myvec{
		-1\\
		-2\\
		},
	\vec{B} = \myvec{
		1\\
		0\\
		},
	\vec{C} = \myvec{
		-1\\
		2\\
		} \text{ and }
	\vec{D} = \myvec{
		-3\\
		0\\
		}
	\end{align}
	\begin{align}
		\vec{B} - \vec{A} &= \myvec{1\\0} - \myvec{-1\\-2} = \myvec{2\\2}\\
		\vec{C} - \vec{B} &= \myvec{-1\\2} - \myvec{1\\0} = \myvec{-2\\2}\\
		\vec{C} - \vec{D} &= \myvec{-1\\2} - \myvec{-3\\0} = \myvec{2\\2}\\
		\vec{D} - \vec{A} &= \myvec{-3\\0} - \myvec{-1\\-2} = \myvec{-2\\2}
	\end{align}
	\begin{align}	
		\vec{C} - \vec{A} &= \myvec{-1\\2} - \myvec{-1\\-2} = \myvec{0\\4}\\
		\vec{D} - \vec{B} &= \myvec{-3\\0} - \myvec{1\\0} = \myvec{-4\\0}
	\end{align}
	\begin{align}	
		\vec{B}-\vec{A} = \vec{C}-\vec{D} \text{ and } \vec{C}-\vec{B} = \vec{D}-\vec{A}.
	\end{align}
	Hence, $ABCD$ is a parallelogram.
	\begin{enumerate}
		\item Now checking if the adjacent sides are orthogonal to each other
	\begin{align}
		(\vec{B}-\vec{A})^\top (\vec{C}-\vec{B}) = \myvec{2&2} \myvec{-2\\2} = -4+4 = 0
	\end{align}
		\item Now checking if the diagonals are also orthogonal then it is a square else a rectangle.
	\end{enumerate}	
	\begin{align}
		(\vec{C}-\vec{A})^\top (\vec{D}-\vec{B}) = \myvec{0&4} \myvec{-4\\0} = 0
	\end{align}
	Hence the diagonals are orthogonal to each other.

	So, we can conclude that $ABCD$ is a square.

	As shown in Figure \ref{fig:10/7/1/6/Fig1} we can see that $ABCD$ is a square hence we can conclude that our theoritical result is verified.
 
\begin{figure}[!h]
	\begin{center} 
	    \includegraphics[width=\columnwidth]{chapters/10/7/1/6/figs/quad1}
	\end{center}
\caption{}
\label{fig:10/7/1/6/Fig1}
\end{figure}

\item The coordinates are given as
	\begin{align}
	\vec{A} = \myvec{
		-3\\
		5\\
		},
	\vec{B} = \myvec{
		3\\
		1\\
		},
	\vec{C} = \myvec{
		0\\
		3\\
		} \text{ and }
	\vec{D} = \myvec{
		-1\\
		-4\\
		}
	\end{align}
	\begin{align}
		\vec{B} - \vec{A} &= \myvec{3\\1} - \myvec{-3\\5} = \myvec{6\\-4}\\
		\vec{C} - \vec{B} &= \myvec{0\\3} - \myvec{3\\1} = \myvec{-3\\2}\\
		\vec{C} - \vec{D} &= \myvec{0\\3} - \myvec{-1\\-4} = \myvec{1\\7}\\
		\vec{D} - \vec{A} &= \myvec{-1\\-4} - \myvec{-3\\5} = \myvec{2\\-9}
	\end{align}
	\begin{align}
		\vec{C} - \vec{A} &= \myvec{0\\3} - \myvec{-3\\5} = \myvec{3\\-2}\\
		\vec{D} - \vec{B} &= \myvec{-1\\-4} - \myvec{3\\1} = \myvec{-4\\-5}
	\end{align}
	\begin{align}
	\vec{B}-\vec{A} \neq \vec{C}-\vec{D} \text{ and } \vec{C}-\vec{B} \neq \vec{D}-\vec{A},
	\end{align}
	Hence, $ABCD$ is not a parallelogram, it can be a irregular quadilateral.
	\begin{enumerate}
		\item Now to check if any three points are collinear,

	if rank of $\myvec{\vec{B}-\vec{A} & \vec{C}-\vec{B}} = 1$ then points are collinear

	Forming the collinearity matrix
	\begin{align}
		\myvec{6&-3\\-4&2} \xleftrightarrow{R_{2}\rightarrow R_{2}+\frac{2}{3}R_{1}}= \myvec{6&-3\\0&0}
	\end{align}
	\end{enumerate}
	Hence, rank = 1

	Since none of the opposite sides are parallel to each other and three points are collinear so these does not form a quadilateral.

	As shown in Figure \ref{fig:10/7/1/6/Fig2} we can see that $ABCD$ does not form a quadilateral and three points are collinear hence, our theoritical result is verified.
	
\begin{figure}[!h]
	\begin{center} 
	    \includegraphics[width=\columnwidth]{chapters/10/7/1/6/figs/quad2}
	\end{center}
\caption{}
\label{fig:10/7/1/6/Fig2}
\end{figure}
	
\item The coordinates are given as
	\begin{align}
	\vec{A} = \myvec{
		4\\
		5\\
		},
	\vec{B} = \myvec{
		7\\
		6\\
		},
	\vec{C} = \myvec{
		4\\
		3\\
		} \text{ and }
	\vec{D} = \myvec{
		1\\
		2\\
		}
	\end{align}
	\begin{align}
		\vec{B} - \vec{A} &= \myvec{7\\6} - \myvec{4\\5} = \myvec{3\\1}\\
		\vec{C} - \vec{B} &= \myvec{4\\3} - \myvec{7\\6} = \myvec{-3\\-3}\\
		\vec{C} - \vec{D} &= \myvec{4\\3} - \myvec{1\\2} = \myvec{3\\1}\\
		\vec{D} - \vec{A} &= \myvec{1\\2} - \myvec{4\\5} = \myvec{-3\\-3}
	\end{align}
	\begin{align}
		\vec{C} - \vec{A} &= \myvec{4\\3} - \myvec{4\\5} = \myvec{0\\-2}\\
		\vec{D} - \vec{B} &= \myvec{1\\2} - \myvec{7\\6} = \myvec{-6\\-4}
	\end{align}
	\begin{align}
		\vec{B}-\vec{A} = \vec{C}-\vec{D} \text{ and } \vec{C}-\vec{B} = \vec{D}-\vec{A},
	\end{align}
	Hence, $ABCD$ is a parallelogram.
	\begin{enumerate}
		\item Now checking if the adjacent sides are orthogonal to each other
	\begin{align}
		(\vec{B}-\vec{A})^\top (\vec{C}-\vec{B}) = \myvec{3&1} \myvec{-3\\-3} = -9-3 = -12
	\end{align}
	Since inner product is not zero so adjacent sides are not orthogonal.

	Hence, we can say that $ABCD$ is neither a rectangle nor a square.

		\item Now checking if the diagonals are orthogonal then it is a Rhombus.
	\begin{align}
		(\vec{C}- \vec{A})^\top (\vec{D}-\vec{B}) = \myvec{0&-2} \myvec{-6\\-4} = 0+8 = 8
	\end{align}
	\end{enumerate}		
	Hence the diagonals are also not orthogonal so we conclude that $ABCD$ is a parallelogram.

	As shown in Figure \ref{fig:10/7/1/6/Fig3} we can see that $ABCD$ forms a parallelogram hence, our theoritical result is verified.

\begin{figure}[!h]
	\begin{center} 
	    \includegraphics[width=\columnwidth]{chapters/10/7/1/6/figs/quad3}
	\end{center}
\caption{}
\label{fig:10/7/1/6/Fig3}
\end{figure}
\end{enumerate}



    \item Find the direction cosines of a line which makes equal angles with the coordinate
    axes.
		\\
		\solution
		\iffalse
\documentclass[12pt]{article}
\usepackage{graphicx}
%\documentclass[journal,12pt,twocolumn]{IEEEtran}
\usepackage[none]{hyphenat}
\usepackage{graphicx}
\usepackage{listings}
\usepackage[english]{babel}
\usepackage{graphicx}
\usepackage{caption} 
\usepackage{hyperref}
\usepackage{booktabs}
\usepackage{array}
\usepackage{amsmath}   % for having text in math mode
\usepackage{listings}
\lstset{
  frame=single,
  breaklines=true
}
  
%Following 2 lines were added to remove the blank page at the beginning
\usepackage{atbegshi}% http://ctan.org/pkg/atbegshi
\AtBeginDocument{\AtBeginShipoutNext{\AtBeginShipoutDiscard}}
%


%New macro definitions
\newcommand{\mydet}[1]{\ensuremath{\begin{vmatrix}#1\end{vmatrix}}}
\providecommand{\brak}[1]{\ensuremath{\left(#1\right)}}
\providecommand{\norm}[1]{\left\lVert#1\right\rVert}
\newcommand{\solution}{\noindent \textbf{Solution: }}
\newcommand{\myvec}[1]{\ensuremath{\begin{pmatrix}#1\end{pmatrix}}}
\let\vec\mathbf

\begin{document}

\begin{center}
\title{\textbf{Vector Dot Product}}
\date{\vspace{-5ex}} %Not to print date automatically
\maketitle
\end{center}
\setcounter{page}{1}

\section{12$^{th}$ Maths - Chapter 10}
This is Problem-9 from Exercise 10.3
\begin{enumerate}
\item Find $\norm{\vec{x}}$, if for a unit vector $\vec{a}$, $\brak{\vec{x}-\vec{a}}.\brak{\vec{x}+\vec{a}} = 12$.\\
	\fi
\solution 
From the given information,
\begin{align}
  \label{eq:12/10/3/9det2f}
  \brak{\vec{x}-\vec{a}}^\top\brak{\vec{x}+\vec{a}} &= 12 \\
  \implies \vec{x}^\top\vec{x} - \vec{a}^\top\vec{x} + \vec{x}^\top\vec{a} - \vec{a}^\top\vec{a} &= 12 \\
  \implies \norm{\vec{x}}^{2} - \norm{\vec{a}}^{2} &= 12 \\
\implies   \norm{\vec{x}}^{2} - 1 &= 12  \\
	\text{or, }  
	\norm{\vec{x}} &= \sqrt{13}
\end{align}

\item Find a unit vector perpendicular to each of the vector $\overrightarrow{a}+\overrightarrow{b}\text{ and }\overrightarrow{a}-\overrightarrow{b},\text{ where } \overrightarrow{a}=3\hat{i}+2\hat{j}+2\hat{k}\text{ and } \overrightarrow{b}=\hat{i}+2\hat{j}-2\hat{k}$. 
	\\
		\solution
		\iffalse
\documentclass[12pt]{article}
\usepackage{graphicx}
\usepackage[none]{hyphenat}
\usepackage{graphicx}
\usepackage{listings}
\usepackage[english]{babel}
\usepackage{graphicx}
\usepackage{caption} 
\usepackage{booktabs}
\usepackage{array}
\usepackage{amssymb} % for \because
\usepackage{amsmath}   % for having text in math mode
\usepackage{extarrows} % for Row operations arrows
\usepackage{listings}
\usepackage[utf8]{inputenc}
\lstset{
  frame=single,
  breaklines=true
}
\usepackage{hyperref}
  
%Following 2 lines were added to remove the blank page at the beginning
\usepackage{atbegshi}% http://ctan.org/pkg/atbegshi
\AtBeginDocument{\AtBeginShipoutNext{\AtBeginShipoutDiscard}}


%New macro definitions
\newcommand{\mydet}[1]{\ensuremath{\begin{vmatrix}#1\end{vmatrix}}}
\providecommand{\brak}[1]{\ensuremath{\left(#1\right)}}
\newcommand{\solution}{\noindent \textbf{Solution: }}
\newcommand{\myvec}[1]{\ensuremath{\begin{pmatrix}#1\end{pmatrix}}}
\providecommand{\norm}[1]{\left\lVert#1\right\rVert}
\providecommand{\abs}[1]{\left\vert#1\right\vert}
\let\vec\mathbf
\begin{document}
\begin{center}
\title{\textbf{  Unit Vector Perpendicular}}
\date{\vspace{-5ex}} %Not to print date automatically
\maketitle
\end{center}
\setcounter{page}{1}
\section{12$^{th}$ Maths - Chapter 10}
\textbf{This is Problem-2 from Exercise 10.4}
\begin{enumerate}

\item Find a unit vector  perpendicular to each of a vector $\bar{a}+\bar{b} \text{ and }\bar{a}-\bar{b}$ where  $\overrightarrow{a}=3\hat{i}+2\hat{j}+2\hat{k}\text{ and }\overrightarrow{b}=\hat{i}+2\hat{j}-2\hat{k}$
\section{Solution}
\fi
Since
\begin{align}
	\vec{a}+\vec{b}=\myvec{4\\4\\0},\,
	\vec{a}-\vec{b}=\myvec{2\\0\\4}
\end{align}
the desired vector is obtained as
\begin{align} 
\myvec{\vec{a}+\vec{b}& \vec{a}-\vec{b}}^\top\vec{x}=0\\
\implies
\myvec{
4&4&0\\
2&0&4
}
\xleftrightarrow[]{R_1=\frac{R_1}{4}}
\myvec{
1&1&0\\
2&0&4
}\xleftrightarrow[]{R_2=\frac{R_2}{2}}
\myvec{
1&1&0\\
1&0&2
}\\
\xleftrightarrow[]{R_2={R_1}-{R_2}}
\myvec{
1&1&0\\
0&-1&2
}
\xleftrightarrow[]{R_2=\frac{R_2}{-1}}
\myvec{
1&1&0\\
0&1&-2
}
\xleftrightarrow[]{R_1={R_1}-{R_2}}
\myvec{
1&0&2\\
0&1&-2
}
\end{align}
yielding
\begin{align}
\begin{split}
x_1+2x_3=0\\
x_2-2x_3=0
\end{split}
\implies 
\vec{x}
=x_3\myvec{-2\\2\\1}
\end{align}

\item If a unit vector $\overrightarrow{a}$ makes angles $\dfrac{\pi}{3}\text{ with }\hat{i}, \dfrac{\pi}{4}\text{ with }\hat{j}$ and an acute angle $\theta \text{ with }\hat{k},\text{ then find } \theta$ and hence, the components of $\overrightarrow{a}$.
	\\
		\solution
		\iffalse
\documentclass[12pt]{article}
\usepackage{graphicx}
%\documentclass[journal,12pt,twocolumn]{IEEEtran}
\usepackage[none]{hyphenat}
\usepackage{graphicx}
\usepackage{listings}
\usepackage[english]{babel}
\usepackage{graphicx}
\usepackage{caption}
\usepackage[parfill]{parskip}
\usepackage{hyperref}
\usepackage{booktabs}
\usepackage{gensymb}
%\usepackage{setspace}\doublespacing\pagestyle{plain}
\def\inputGnumericTable{}
\usepackage{color}                                            %%
    \usepackage{array}                                            %%
    \usepackage{longtable}                                        %%
    \usepackage{calc}                                             %%
    \usepackage{multirow}                                         %%
    \usepackage{hhline}                                           %%
    \usepackage{ifthen}
\usepackage{array}
\usepackage{amsmath}   % for having text in math mode
\usepackage{parallel,enumitem}
\usepackage{listings}
\lstset{
language=tex,
frame=single,
breaklines=true
}
 
%Following 2 lines were added to remove the blank page at the beginning
\usepackage{atbegshi}% http://ctan.org/pkg/atbegshi
\AtBeginDocument{\AtBeginShipoutNext{\AtBeginShipoutDiscard}}
%
%New macro definitions
\newcommand{\mydet}[1]{\ensuremath{\begin{vmatrix}#1\end{vmatrix}}}
\providecommand{\brak}[1]{\ensuremath{\left(#1\right)}}
\providecommand{\norm}[1]{\left\lVert#1\right\rVert}
\newcommand{\solution}{\noindent \textbf{Solution: }}
\newcommand{\myvec}[1]{\ensuremath{\begin{pmatrix}#1\end{pmatrix}}}
\let\vec\mathbf
\begin{document}
\begin{center}
\enlargethispage{-4cm}
\title{\textbf{Vector Algebra}}
\date{\vspace{-5ex}} %Not to print date automatically
\maketitle
\end{center}
\setcounter{page}{1}
\section*{12$^{th}$ Maths - Chapter 10}
This is Problem-3 from Exercise 10.4
\begin{enumerate}
\item If unit vector $\overrightarrow{a}$ makes angles $\frac{\pi}{3}$ with $\hat{i}$, $\frac{\pi}{4}$ with $\hat{j}$ and an acute angle $\theta$ with $\hat{k}$, then find $\theta$ and hence, the components of $\overrightarrow{a}$.

\solution
\fi
		Let 
		\begin{align}
			\vec{A}=\myvec{\cos\theta_1\\\cos\theta_2\\\cos\theta_3}
		\end{align}
		where
		\begin{align}
		\cos\theta_1 &=\cos\frac{\pi}{3}
			=\frac{1}{2}\\
			\cos\theta_2 &=\cos\frac{\pi}{4}\\
			=\frac{1}{\sqrt{2}}
		\end{align}
		Since
\begin{align}
    \norm{\vec{A}}&=1,
\sqrt{\cos^2\theta_1+\cos^2\theta_2+\cos^2\theta_3}&=1
    \implies\sqrt{\frac{1}{2}^2+\frac{1}{\sqrt{2}}^2+\cos^2\theta_3 }&=1\\
    \implies\cos\theta_3 &=\pm\frac{1}{2}
\end{align}
Since $\theta_3$ is an acute angle
\begin{align}
 \cos\theta_3=\frac{1}{2}
\end{align}
    Hence 
\begin{align}
		\vec{A}=\myvec{\frac{1}{2}\\[2pt] \frac{1}{\sqrt{2}}\\[2pt] \frac{1}{2}}
\end{align}

\item Show that the direction cosines of a vector equally inclined to the axes OX, OY and OZ are \textpm $\sbrak{\frac{1}{\sqrt{3}},\frac{1}{\sqrt{3}},\frac{1}{\sqrt{3}}}$.\\
	\solution
		\iffalse
\documentclass[12pt]{article}
\usepackage{graphicx}
%\documentclass[journal,12pt,twocolumn]{IEEEtran}
\usepackage[none]{hyphenat}
\usepackage{graphicx}
\usepackage{listings}
\usepackage[english]{babel}
\usepackage{graphicx}
\usepackage{caption} 
\usepackage{hyperref}
\usepackage{booktabs}
\usepackage{array}
\usepackage{amsmath}   % for having text in math mode
\usepackage{listings}
\lstset{
  frame=single,
  breaklines=true
}
  
%Following 2 lines were added to remove the blank page at the beginning
\usepackage{atbegshi}% http://ctan.org/pkg/atbegshi
\AtBeginDocument{\AtBeginShipoutNext{\AtBeginShipoutDiscard}}
%


%New macro definitions
\newcommand{\mydet}[1]{\ensuremath{\begin{vmatrix}#1\end{vmatrix}}}
\providecommand{\brak}[1]{\ensuremath{\left(#1\right)}}
\providecommand{\norm}[1]{\left\lVert#1\right\rVert}
\newcommand{\solution}{\noindent \textbf{Solution: }}
\newcommand{\myvec}[1]{\ensuremath{\begin{pmatrix}#1\end{pmatrix}}}
\let\vec\mathbf

\begin{document}

\begin{center}
\title{\textbf{Vector Dot Product}}
\date{\vspace{-5ex}} %Not to print date automatically
\maketitle
\end{center}
\setcounter{page}{1}

\section{12$^{th}$ Maths - Chapter 10}
This is Problem-9 from Exercise 10.3
\begin{enumerate}
\item Find $\norm{\vec{x}}$, if for a unit vector $\vec{a}$, $\brak{\vec{x}-\vec{a}}.\brak{\vec{x}+\vec{a}} = 12$.\\
	\fi
\solution 
From the given information,
\begin{align}
  \label{eq:12/10/3/9det2f}
  \brak{\vec{x}-\vec{a}}^\top\brak{\vec{x}+\vec{a}} &= 12 \\
  \implies \vec{x}^\top\vec{x} - \vec{a}^\top\vec{x} + \vec{x}^\top\vec{a} - \vec{a}^\top\vec{a} &= 12 \\
  \implies \norm{\vec{x}}^{2} - \norm{\vec{a}}^{2} &= 12 \\
\implies   \norm{\vec{x}}^{2} - 1 &= 12  \\
	\text{or, }  
	\norm{\vec{x}} &= \sqrt{13}
\end{align}

\item Write down a unit vector in XY-plane, making an angle of 30$^{\circ}$ with the positive direction of x-axis.\\
\item The scalar product of the vector $\hat{i}+\hat{j}+\hat{k}$ with a unit vector along the sum of vectors $2\hat{i}+4\hat{j}-5\hat{k}$ and $\lambda\hat{i}+2\hat{j}+3\hat{k}$ is equal to one. Find the value of $\lambda$.
\item If $\theta$ is the angle between two vectors $\vec{a}$ and $\vec{b}$,then $\vec{a}.\vec{b}\geq0$ only when 
\begin{enumerate}
\item \label{itm:chapters/12/10/5/161} $0<\theta<\frac{\pi}{2}$
\item \label{itm:chapters/12/10/5/162} $0\le\theta\le\frac{\pi}{2}$
\item \label{itm:chapters/12/10/5/163} $0<\theta<\pi$
\item \label{itm:chapters/12/10/5/164} $0\le\theta\le\pi$
\end{enumerate}
	\solution
		\iffalse
\documentclass[12pt]{article}
\usepackage{graphicx}
\usepackage{amsmath}
\usepackage{mathtools}
\usepackage{gensymb}

\newcommand{\mydet}[1]{\ensuremath{\begin{vmatrix}#1\end{vmatrix}}}
\providecommand{\brak}[1]{\ensuremath{\left(#1\right)}}
\providecommand{\norm}[1]{\left\lVert#1\right\rVert}
\newcommand{\solution}{\noindent \textbf{Solution: }}
\newcommand{\myvec}[1]{\ensuremath{\begin{pmatrix}#1\end{pmatrix}}}
\let\vec\mathbf

\begin{document}
\begin{center}
\textbf\large{CHAPTER-7 \\ COORDINATE GEOMETRY}

\end{center}
\section*{Excercise 7.1}

Q6.Name the type of quadilateral formed,if any, by the following points, and give reasons for your answer:
\begin{enumerate}
	\item $\brak{-1,-2}, \brak{1,0}, \brak{-1,2}, \brak{-3,0}$ 
	\item $\brak{-3,5}, \brak{3,1}, \brak{0,3}, \brak{-1,-4}$
	\item $\brak{4,5}, \brak{7,6}, \brak{4,3}, \brak{1,2}$
\end{enumerate}
\solution
\fi
\begin{enumerate}
\item The coordinates are given as
	\begin{align}
	\vec{A} = \myvec{
		-1\\
		-2\\
		},
	\vec{B} = \myvec{
		1\\
		0\\
		},
	\vec{C} = \myvec{
		-1\\
		2\\
		} \text{ and }
	\vec{D} = \myvec{
		-3\\
		0\\
		}
	\end{align}
	\begin{align}
		\vec{B} - \vec{A} &= \myvec{1\\0} - \myvec{-1\\-2} = \myvec{2\\2}\\
		\vec{C} - \vec{B} &= \myvec{-1\\2} - \myvec{1\\0} = \myvec{-2\\2}\\
		\vec{C} - \vec{D} &= \myvec{-1\\2} - \myvec{-3\\0} = \myvec{2\\2}\\
		\vec{D} - \vec{A} &= \myvec{-3\\0} - \myvec{-1\\-2} = \myvec{-2\\2}
	\end{align}
	\begin{align}	
		\vec{C} - \vec{A} &= \myvec{-1\\2} - \myvec{-1\\-2} = \myvec{0\\4}\\
		\vec{D} - \vec{B} &= \myvec{-3\\0} - \myvec{1\\0} = \myvec{-4\\0}
	\end{align}
	\begin{align}	
		\vec{B}-\vec{A} = \vec{C}-\vec{D} \text{ and } \vec{C}-\vec{B} = \vec{D}-\vec{A}.
	\end{align}
	Hence, $ABCD$ is a parallelogram.
	\begin{enumerate}
		\item Now checking if the adjacent sides are orthogonal to each other
	\begin{align}
		(\vec{B}-\vec{A})^\top (\vec{C}-\vec{B}) = \myvec{2&2} \myvec{-2\\2} = -4+4 = 0
	\end{align}
		\item Now checking if the diagonals are also orthogonal then it is a square else a rectangle.
	\end{enumerate}	
	\begin{align}
		(\vec{C}-\vec{A})^\top (\vec{D}-\vec{B}) = \myvec{0&4} \myvec{-4\\0} = 0
	\end{align}
	Hence the diagonals are orthogonal to each other.

	So, we can conclude that $ABCD$ is a square.

	As shown in Figure \ref{fig:10/7/1/6/Fig1} we can see that $ABCD$ is a square hence we can conclude that our theoritical result is verified.
 
\begin{figure}[!h]
	\begin{center} 
	    \includegraphics[width=\columnwidth]{chapters/10/7/1/6/figs/quad1}
	\end{center}
\caption{}
\label{fig:10/7/1/6/Fig1}
\end{figure}

\item The coordinates are given as
	\begin{align}
	\vec{A} = \myvec{
		-3\\
		5\\
		},
	\vec{B} = \myvec{
		3\\
		1\\
		},
	\vec{C} = \myvec{
		0\\
		3\\
		} \text{ and }
	\vec{D} = \myvec{
		-1\\
		-4\\
		}
	\end{align}
	\begin{align}
		\vec{B} - \vec{A} &= \myvec{3\\1} - \myvec{-3\\5} = \myvec{6\\-4}\\
		\vec{C} - \vec{B} &= \myvec{0\\3} - \myvec{3\\1} = \myvec{-3\\2}\\
		\vec{C} - \vec{D} &= \myvec{0\\3} - \myvec{-1\\-4} = \myvec{1\\7}\\
		\vec{D} - \vec{A} &= \myvec{-1\\-4} - \myvec{-3\\5} = \myvec{2\\-9}
	\end{align}
	\begin{align}
		\vec{C} - \vec{A} &= \myvec{0\\3} - \myvec{-3\\5} = \myvec{3\\-2}\\
		\vec{D} - \vec{B} &= \myvec{-1\\-4} - \myvec{3\\1} = \myvec{-4\\-5}
	\end{align}
	\begin{align}
	\vec{B}-\vec{A} \neq \vec{C}-\vec{D} \text{ and } \vec{C}-\vec{B} \neq \vec{D}-\vec{A},
	\end{align}
	Hence, $ABCD$ is not a parallelogram, it can be a irregular quadilateral.
	\begin{enumerate}
		\item Now to check if any three points are collinear,

	if rank of $\myvec{\vec{B}-\vec{A} & \vec{C}-\vec{B}} = 1$ then points are collinear

	Forming the collinearity matrix
	\begin{align}
		\myvec{6&-3\\-4&2} \xleftrightarrow{R_{2}\rightarrow R_{2}+\frac{2}{3}R_{1}}= \myvec{6&-3\\0&0}
	\end{align}
	\end{enumerate}
	Hence, rank = 1

	Since none of the opposite sides are parallel to each other and three points are collinear so these does not form a quadilateral.

	As shown in Figure \ref{fig:10/7/1/6/Fig2} we can see that $ABCD$ does not form a quadilateral and three points are collinear hence, our theoritical result is verified.
	
\begin{figure}[!h]
	\begin{center} 
	    \includegraphics[width=\columnwidth]{chapters/10/7/1/6/figs/quad2}
	\end{center}
\caption{}
\label{fig:10/7/1/6/Fig2}
\end{figure}
	
\item The coordinates are given as
	\begin{align}
	\vec{A} = \myvec{
		4\\
		5\\
		},
	\vec{B} = \myvec{
		7\\
		6\\
		},
	\vec{C} = \myvec{
		4\\
		3\\
		} \text{ and }
	\vec{D} = \myvec{
		1\\
		2\\
		}
	\end{align}
	\begin{align}
		\vec{B} - \vec{A} &= \myvec{7\\6} - \myvec{4\\5} = \myvec{3\\1}\\
		\vec{C} - \vec{B} &= \myvec{4\\3} - \myvec{7\\6} = \myvec{-3\\-3}\\
		\vec{C} - \vec{D} &= \myvec{4\\3} - \myvec{1\\2} = \myvec{3\\1}\\
		\vec{D} - \vec{A} &= \myvec{1\\2} - \myvec{4\\5} = \myvec{-3\\-3}
	\end{align}
	\begin{align}
		\vec{C} - \vec{A} &= \myvec{4\\3} - \myvec{4\\5} = \myvec{0\\-2}\\
		\vec{D} - \vec{B} &= \myvec{1\\2} - \myvec{7\\6} = \myvec{-6\\-4}
	\end{align}
	\begin{align}
		\vec{B}-\vec{A} = \vec{C}-\vec{D} \text{ and } \vec{C}-\vec{B} = \vec{D}-\vec{A},
	\end{align}
	Hence, $ABCD$ is a parallelogram.
	\begin{enumerate}
		\item Now checking if the adjacent sides are orthogonal to each other
	\begin{align}
		(\vec{B}-\vec{A})^\top (\vec{C}-\vec{B}) = \myvec{3&1} \myvec{-3\\-3} = -9-3 = -12
	\end{align}
	Since inner product is not zero so adjacent sides are not orthogonal.

	Hence, we can say that $ABCD$ is neither a rectangle nor a square.

		\item Now checking if the diagonals are orthogonal then it is a Rhombus.
	\begin{align}
		(\vec{C}- \vec{A})^\top (\vec{D}-\vec{B}) = \myvec{0&-2} \myvec{-6\\-4} = 0+8 = 8
	\end{align}
	\end{enumerate}		
	Hence the diagonals are also not orthogonal so we conclude that $ABCD$ is a parallelogram.

	As shown in Figure \ref{fig:10/7/1/6/Fig3} we can see that $ABCD$ forms a parallelogram hence, our theoritical result is verified.

\begin{figure}[!h]
	\begin{center} 
	    \includegraphics[width=\columnwidth]{chapters/10/7/1/6/figs/quad3}
	\end{center}
\caption{}
\label{fig:10/7/1/6/Fig3}
\end{figure}
\end{enumerate}



\item Find the slope of the line, which makes an angle of 30 degrees with the positive direction of y-axis measures anticlockwise.
\label{chapters/11/10/1/7}\\
\solution
\iffalse
\documentclass[12pt]{article}
\usepackage{graphicx}
\usepackage{amsmath}
\usepackage{mathtools}
\usepackage{gensymb}

\newcommand{\mydet}[1]{\ensuremath{\begin{vmatrix}#1\end{vmatrix}}}
\providecommand{\brak}[1]{\ensuremath{\left(#1\right)}}
\providecommand{\norm}[1]{\left\lVert#1\right\rVert}
\newcommand{\solution}{\noindent \textbf{Solution: }}
\newcommand{\myvec}[1]{\ensuremath{\begin{pmatrix}#1\end{pmatrix}}}
\let\vec\mathbf

\begin{document}
\begin{center}
\textbf\large{CHAPTER-7 \\ COORDINATE GEOMETRY}

\end{center}
\section*{Excercise 7.1}

Q6.Name the type of quadilateral formed,if any, by the following points, and give reasons for your answer:
\begin{enumerate}
	\item $\brak{-1,-2}, \brak{1,0}, \brak{-1,2}, \brak{-3,0}$ 
	\item $\brak{-3,5}, \brak{3,1}, \brak{0,3}, \brak{-1,-4}$
	\item $\brak{4,5}, \brak{7,6}, \brak{4,3}, \brak{1,2}$
\end{enumerate}
\solution
\fi
\begin{enumerate}
\item The coordinates are given as
	\begin{align}
	\vec{A} = \myvec{
		-1\\
		-2\\
		},
	\vec{B} = \myvec{
		1\\
		0\\
		},
	\vec{C} = \myvec{
		-1\\
		2\\
		} \text{ and }
	\vec{D} = \myvec{
		-3\\
		0\\
		}
	\end{align}
	\begin{align}
		\vec{B} - \vec{A} &= \myvec{1\\0} - \myvec{-1\\-2} = \myvec{2\\2}\\
		\vec{C} - \vec{B} &= \myvec{-1\\2} - \myvec{1\\0} = \myvec{-2\\2}\\
		\vec{C} - \vec{D} &= \myvec{-1\\2} - \myvec{-3\\0} = \myvec{2\\2}\\
		\vec{D} - \vec{A} &= \myvec{-3\\0} - \myvec{-1\\-2} = \myvec{-2\\2}
	\end{align}
	\begin{align}	
		\vec{C} - \vec{A} &= \myvec{-1\\2} - \myvec{-1\\-2} = \myvec{0\\4}\\
		\vec{D} - \vec{B} &= \myvec{-3\\0} - \myvec{1\\0} = \myvec{-4\\0}
	\end{align}
	\begin{align}	
		\vec{B}-\vec{A} = \vec{C}-\vec{D} \text{ and } \vec{C}-\vec{B} = \vec{D}-\vec{A}.
	\end{align}
	Hence, $ABCD$ is a parallelogram.
	\begin{enumerate}
		\item Now checking if the adjacent sides are orthogonal to each other
	\begin{align}
		(\vec{B}-\vec{A})^\top (\vec{C}-\vec{B}) = \myvec{2&2} \myvec{-2\\2} = -4+4 = 0
	\end{align}
		\item Now checking if the diagonals are also orthogonal then it is a square else a rectangle.
	\end{enumerate}	
	\begin{align}
		(\vec{C}-\vec{A})^\top (\vec{D}-\vec{B}) = \myvec{0&4} \myvec{-4\\0} = 0
	\end{align}
	Hence the diagonals are orthogonal to each other.

	So, we can conclude that $ABCD$ is a square.

	As shown in Figure \ref{fig:10/7/1/6/Fig1} we can see that $ABCD$ is a square hence we can conclude that our theoritical result is verified.
 
\begin{figure}[!h]
	\begin{center} 
	    \includegraphics[width=\columnwidth]{chapters/10/7/1/6/figs/quad1}
	\end{center}
\caption{}
\label{fig:10/7/1/6/Fig1}
\end{figure}

\item The coordinates are given as
	\begin{align}
	\vec{A} = \myvec{
		-3\\
		5\\
		},
	\vec{B} = \myvec{
		3\\
		1\\
		},
	\vec{C} = \myvec{
		0\\
		3\\
		} \text{ and }
	\vec{D} = \myvec{
		-1\\
		-4\\
		}
	\end{align}
	\begin{align}
		\vec{B} - \vec{A} &= \myvec{3\\1} - \myvec{-3\\5} = \myvec{6\\-4}\\
		\vec{C} - \vec{B} &= \myvec{0\\3} - \myvec{3\\1} = \myvec{-3\\2}\\
		\vec{C} - \vec{D} &= \myvec{0\\3} - \myvec{-1\\-4} = \myvec{1\\7}\\
		\vec{D} - \vec{A} &= \myvec{-1\\-4} - \myvec{-3\\5} = \myvec{2\\-9}
	\end{align}
	\begin{align}
		\vec{C} - \vec{A} &= \myvec{0\\3} - \myvec{-3\\5} = \myvec{3\\-2}\\
		\vec{D} - \vec{B} &= \myvec{-1\\-4} - \myvec{3\\1} = \myvec{-4\\-5}
	\end{align}
	\begin{align}
	\vec{B}-\vec{A} \neq \vec{C}-\vec{D} \text{ and } \vec{C}-\vec{B} \neq \vec{D}-\vec{A},
	\end{align}
	Hence, $ABCD$ is not a parallelogram, it can be a irregular quadilateral.
	\begin{enumerate}
		\item Now to check if any three points are collinear,

	if rank of $\myvec{\vec{B}-\vec{A} & \vec{C}-\vec{B}} = 1$ then points are collinear

	Forming the collinearity matrix
	\begin{align}
		\myvec{6&-3\\-4&2} \xleftrightarrow{R_{2}\rightarrow R_{2}+\frac{2}{3}R_{1}}= \myvec{6&-3\\0&0}
	\end{align}
	\end{enumerate}
	Hence, rank = 1

	Since none of the opposite sides are parallel to each other and three points are collinear so these does not form a quadilateral.

	As shown in Figure \ref{fig:10/7/1/6/Fig2} we can see that $ABCD$ does not form a quadilateral and three points are collinear hence, our theoritical result is verified.
	
\begin{figure}[!h]
	\begin{center} 
	    \includegraphics[width=\columnwidth]{chapters/10/7/1/6/figs/quad2}
	\end{center}
\caption{}
\label{fig:10/7/1/6/Fig2}
\end{figure}
	
\item The coordinates are given as
	\begin{align}
	\vec{A} = \myvec{
		4\\
		5\\
		},
	\vec{B} = \myvec{
		7\\
		6\\
		},
	\vec{C} = \myvec{
		4\\
		3\\
		} \text{ and }
	\vec{D} = \myvec{
		1\\
		2\\
		}
	\end{align}
	\begin{align}
		\vec{B} - \vec{A} &= \myvec{7\\6} - \myvec{4\\5} = \myvec{3\\1}\\
		\vec{C} - \vec{B} &= \myvec{4\\3} - \myvec{7\\6} = \myvec{-3\\-3}\\
		\vec{C} - \vec{D} &= \myvec{4\\3} - \myvec{1\\2} = \myvec{3\\1}\\
		\vec{D} - \vec{A} &= \myvec{1\\2} - \myvec{4\\5} = \myvec{-3\\-3}
	\end{align}
	\begin{align}
		\vec{C} - \vec{A} &= \myvec{4\\3} - \myvec{4\\5} = \myvec{0\\-2}\\
		\vec{D} - \vec{B} &= \myvec{1\\2} - \myvec{7\\6} = \myvec{-6\\-4}
	\end{align}
	\begin{align}
		\vec{B}-\vec{A} = \vec{C}-\vec{D} \text{ and } \vec{C}-\vec{B} = \vec{D}-\vec{A},
	\end{align}
	Hence, $ABCD$ is a parallelogram.
	\begin{enumerate}
		\item Now checking if the adjacent sides are orthogonal to each other
	\begin{align}
		(\vec{B}-\vec{A})^\top (\vec{C}-\vec{B}) = \myvec{3&1} \myvec{-3\\-3} = -9-3 = -12
	\end{align}
	Since inner product is not zero so adjacent sides are not orthogonal.

	Hence, we can say that $ABCD$ is neither a rectangle nor a square.

		\item Now checking if the diagonals are orthogonal then it is a Rhombus.
	\begin{align}
		(\vec{C}- \vec{A})^\top (\vec{D}-\vec{B}) = \myvec{0&-2} \myvec{-6\\-4} = 0+8 = 8
	\end{align}
	\end{enumerate}		
	Hence the diagonals are also not orthogonal so we conclude that $ABCD$ is a parallelogram.

	As shown in Figure \ref{fig:10/7/1/6/Fig3} we can see that $ABCD$ forms a parallelogram hence, our theoritical result is verified.

\begin{figure}[!h]
	\begin{center} 
	    \includegraphics[width=\columnwidth]{chapters/10/7/1/6/figs/quad3}
	\end{center}
\caption{}
\label{fig:10/7/1/6/Fig3}
\end{figure}
\end{enumerate}



\item Find the angle between x-axis and the line joining points (3,-1) and (4,-2).
\label{chapters/11/10/1/10}
\iffalse
\documentclass[journal,12pt,twocolumn]{IEEEtran}
\usepackage{graphicx}
\graphicspath{{./figs/}}{}
\usepackage{amsmath,amssymb,amsfonts,amsthm}
\newcommand{\myvec}[1]{\ensuremath{\begin{pmatrix}#1\end{pmatrix}}}

\let\vec\mathbf

\title{
Matrix-Lines
}
\author{Jyothsna Paluchuri-FWC22059\\}
\begin{document}
\maketitle
\tableofcontents
\bigskip
\section{Problem Statement}
\fi
	\begin{figure}[!ht]
		\centering
 \includegraphics[width=\columnwidth]{chapters/11/10/1/5/figs/line.png}
		\caption{}
		\label{fig:11/10/1/5}
  	\end{figure}
	\\
	\solution
\iffalse
\section{Construction}
\begin{figure}[h]
    \centering
\includegraphics[width=\columnwidth]{line.png}
    \caption{Equation of the slope}
    \label{fig:my_label}
\end{figure}
\vspace{2cm}
\begin{table}[h]
    \centering
    \begin{tabular}{|c|c|c|c|}
       \hline
       \textbf{Symbol}&\textbf{Value}&\textbf{Description}  \\
       \hline
	    $\vec{P}$ & $\myvec{
		    0\\
		    -4}$
	    & Point on Y-axis\\
        \hline
	    $\vec{B}$ & $\myvec{8\\0}$
 & Point on X-axis\\
        \hline
	    $\vec{0}$ & $\myvec{0\\0}$
 & Origin\\
        \hline
    \end{tabular}
    \caption{Parameters}
    \label{tab:my_label}
\end{table}


\section{Solution}
Given that resultant line passes through origin and mid point of the line segment joining point P(0,-4) and B(8,0) \\
\\
\\
given ${\vec{P}}$=$\myvec{
  0\\
  -4}$
 , ${\vec{B}}$=$\myvec{
  8\\
  0}$
  
 \fi 
The mid point of $PB$ is
\begin{align}
\vec{M} &=\frac{1}{2}(\vec{P}+\vec{B})
	= \myvec{4 \\ -2}  
\end{align}
The direction vector of line joining $\vec{O}, \vec{M}$ is 
\begin{align}
\vec{m}&=\vec{O}-\vec{M}
 = -\vec{M}
\end{align}
which can be expressed as
\begin{align}
	\myvec{1 \\ -\frac{1}{2}}
\end{align}
Thus the slope is
\begin{align}
	m = -\frac{1}{2}
\end{align}
\iffalse
\textbf{The direction vector of a line expressed as}
\begin{align}
\implies\vec{m} &= \begin{pmatrix}1 \\ m \\ \end{pmatrix}
\end{align}

\textbf{By solving equation (5) and (6),we get the slope of $\vec{O}$ $\vec{M}$ line}
\begin{align}
        \boxed{m=-0.5}
 \end{align}

\section{Software}
Download the following code using,
\begin{table}[h]
    \centering
    \begin{tabular}{|c|}
    \hline \\
   https://github.com/jyothsna777/jyothsna-fwc.git  \\
         \\
\hline
    \end{tabular}
\end{table}
\\
and execute the code by using command
\begin{center}
\textbf{Python3 lines.py}\\
\end{center}

\section{Conclusion}
Hence the slope of line $\vec{O}$ $\vec{M}$ lineis $\vec{m}$=-0.5

\end{document}
\fi

	\item The slope of a line is double of the slope of another line. If tangent of the angle between them is 1/3, find the slopes of the lines.
\label{chapters/11/10/1/11}
\iffalse
\documentclass[journal,12pt,twocolumn]{IEEEtran}
\usepackage{graphicx}
\graphicspath{{./figs/}}{}
\usepackage{amsmath,amssymb,amsfonts,amsthm}
\newcommand{\myvec}[1]{\ensuremath{\begin{pmatrix}#1\end{pmatrix}}}

\let\vec\mathbf

\title{
Matrix-Lines
}
\author{Jyothsna Paluchuri-FWC22059\\}
\begin{document}
\maketitle
\tableofcontents
\bigskip
\section{Problem Statement}
\fi
	\begin{figure}[!ht]
		\centering
 \includegraphics[width=\columnwidth]{chapters/11/10/1/5/figs/line.png}
		\caption{}
		\label{fig:11/10/1/5}
  	\end{figure}
	\\
	\solution
\iffalse
\section{Construction}
\begin{figure}[h]
    \centering
\includegraphics[width=\columnwidth]{line.png}
    \caption{Equation of the slope}
    \label{fig:my_label}
\end{figure}
\vspace{2cm}
\begin{table}[h]
    \centering
    \begin{tabular}{|c|c|c|c|}
       \hline
       \textbf{Symbol}&\textbf{Value}&\textbf{Description}  \\
       \hline
	    $\vec{P}$ & $\myvec{
		    0\\
		    -4}$
	    & Point on Y-axis\\
        \hline
	    $\vec{B}$ & $\myvec{8\\0}$
 & Point on X-axis\\
        \hline
	    $\vec{0}$ & $\myvec{0\\0}$
 & Origin\\
        \hline
    \end{tabular}
    \caption{Parameters}
    \label{tab:my_label}
\end{table}


\section{Solution}
Given that resultant line passes through origin and mid point of the line segment joining point P(0,-4) and B(8,0) \\
\\
\\
given ${\vec{P}}$=$\myvec{
  0\\
  -4}$
 , ${\vec{B}}$=$\myvec{
  8\\
  0}$
  
 \fi 
The mid point of $PB$ is
\begin{align}
\vec{M} &=\frac{1}{2}(\vec{P}+\vec{B})
	= \myvec{4 \\ -2}  
\end{align}
The direction vector of line joining $\vec{O}, \vec{M}$ is 
\begin{align}
\vec{m}&=\vec{O}-\vec{M}
 = -\vec{M}
\end{align}
which can be expressed as
\begin{align}
	\myvec{1 \\ -\frac{1}{2}}
\end{align}
Thus the slope is
\begin{align}
	m = -\frac{1}{2}
\end{align}
\iffalse
\textbf{The direction vector of a line expressed as}
\begin{align}
\implies\vec{m} &= \begin{pmatrix}1 \\ m \\ \end{pmatrix}
\end{align}

\textbf{By solving equation (5) and (6),we get the slope of $\vec{O}$ $\vec{M}$ line}
\begin{align}
        \boxed{m=-0.5}
 \end{align}

\section{Software}
Download the following code using,
\begin{table}[h]
    \centering
    \begin{tabular}{|c|}
    \hline \\
   https://github.com/jyothsna777/jyothsna-fwc.git  \\
         \\
\hline
    \end{tabular}
\end{table}
\\
and execute the code by using command
\begin{center}
\textbf{Python3 lines.py}\\
\end{center}

\section{Conclusion}
Hence the slope of line $\vec{O}$ $\vec{M}$ lineis $\vec{m}$=-0.5

\end{document}
\fi

\item    Find angle between the lines,$\sqrt{3}x+y=1$ and $x+\sqrt{3}y$=1.
\label{chapters/11/10/3/9}
\def\mytitle{LINE ASSIGNMENT}
\def\myauthor{G.Kumar}
\def\contact{kumargandhamaneni20016@gmail.com}
\def\mymodule{Future Wireless Communication (FWC)}
\documentclass[10pt, a4paper]{article}
\usepackage[a4paper,outer=1.5cm,inner=1.5cm,top=1.75cm,bottom=1.5cm]{geometry}
\twocolumn
\usepackage{graphicx}
\graphicspath{{./images/}}
\usepackage[colorlinks,linkcolor={black},citecolor={blue!80!black},urlcolor={blue!80!black}]{hyperref}
\usepackage[parfill]{parskip}
\usepackage{lmodern}
\usepackage{tikz}
\usepackage{physics}
\usepackage{tabularx}
\usetikzlibrary{calc}
\usepackage{amsmath}
\usepackage{amssymb}
\renewcommand*\familydefault{\sfdefault}
\usepackage{watermark}
\usepackage{lipsum}
\usepackage{xcolor}
\usepackage{listings}
\usepackage{float}
\usepackage{titlesec}
\providecommand{\mtx}[1]{\mathbf{#1}}
\titlespacing{\subsection}{1pt}{\parskip}{3pt}
\titlespacing{\subsubsection}{0pt}{\parskip}{-\parskip}
\titlespacing{\paragraph}{0pt}{\parskip}{\parskip}
\newcommand{\figuremacro}[5]{
    \begin{figure}[#1]
        \centering
        \includegraphics[width=#5\columnwidth]{#2}
        \caption[#3]{\textbf{#3}#4}
        \label{fig:#2}
    \end{figure}
}
\newcommand{\myvec}[1]{\ensuremath{\begin{pmatrix}#1\end{pmatrix}}}
\let\vec\mathbf
\lstset{
frame=single, 
breaklines=true,
columns=fullflexible
}
\thiswatermark{\centering \put(0,-105.0){\includegraphics[scale=0.35]{iith}} }
\title{\mytitle}
\author{\myauthor\hspace{1em}\\\contact\\IITH\hspace{0.5em}-\hspace{0.5em}\mymodule}
\date{}
\begin{document}
	\maketitle
\section*{Problem}
   Find angle between the lines,$\sqrt{3}$x+y=1 and x+$\sqrt{3}$y=1.
   \section*{Solution}
   \includegraphics[scale=0.55]{line.png}
   The input parameters for this construction are :
   \begin{center}
\begin{tabular}{|c|c|c|}
	\hline
	\textbf{Symbol}&\textbf{Value}&\textbf{Description}\\
	\hline
	P&$\
	\begin{pmatrix}
		0.57736 \\
		0 \\
	\end{pmatrix}$%
	&Point P\\ 
	\hline
	X&$\
	\begin{pmatrix}
		0.36603 \\
		0.36603 \\
	\end{pmatrix}$%
	&Point X\\
	\hline
	Q&$\
	\begin{pmatrix}
		1 \\
		0 \\
	\end{pmatrix}$%
	&Point Q\\
	
	\hline
\end{tabular}
\end{center}
   \subsection*{Step 1}
   Given two equations are, \\
   \begin{equation}
   \sqrt{3}x+y=1 
   \end{equation}
   \begin{equation}
   x+\sqrt{3}y=1 
   \end{equation}
   Equation(1) in vector form is given as,
   \begin{align}
   \myvec{\sqrt{3}&1}\vec{x}=1
   \end{align}
   From this, Normal vector to the line is given as,
   \begin{align*}
   \vec{n_1}=\myvec{\sqrt{3}\\1}
   \end{align*}
   So, the direction vector of the line is given as,
\begin{eqnarray*}
   \vec{m1}=\myvec{-1\\\sqrt{3}}
\end{eqnarray*} 
Similarly, Normal vector to the line(2) is given as,
   \begin{align*}
   \vec{n_2}=\myvec{1\\\sqrt{3}}
   \end{align*}
   So, the direction vector of the line is given as,
\begin{eqnarray*}
   \vec{m2}=\myvec{-\sqrt{3}\\1}
\end{eqnarray*}     

\subsection*{Step 2}
Now, Angle between any two lines,using their direction vectors, is given by, \\
\begin{eqnarray*}
 cos\theta=\frac{(\vec{m1})^T(\vec{m2})}{\norm{\vec{m1}}\norm{\vec{m2}}}
\end{eqnarray*}
So, Angle between the two lines is given by,
\begin{eqnarray}
 cos\angle{x}=\frac{(\vec{m1})^T(\vec{m2})}{\norm{\vec{m1}}\norm{\vec{m2}}}\
\end{eqnarray}
\begin{eqnarray}
 cos\angle{x}=\frac{\myvec{-1\\\sqrt{3}}^T\myvec{-\sqrt{3}\\1}}{\norm{\myvec{-1\\\sqrt{3}}}\norm{\myvec{-\sqrt{3}\\1}}}
\end{eqnarray}
By solving the above equation, we get, \\
\begin{equation}
cos\angle{x}=\frac{\sqrt{3}}{2} \\
\end{equation}
This Implies,
\begin{equation*}
\angle{x}=30^\circ
\end{equation*}
Therefore, the angle between given two lines is $30^\circ$. \\
\bibliographystyle{ieeetr}
\end{document}
\item Find the equation of the lines through the point (3, 2) which make an angle of 45\degree  with the line $x – 2y$ = 3.
\label{chapters/11/10/4/11}\\
\solution
%\documentclass{article}
\documentclass[10pt,a4paper]{report}
\usepackage{amsmath}
\usepackage{amssymb}
\usepackage{gensymb}
\usepackage{amsfonts}
\usepackage{setspace}
\usepackage{tasks}
\usepackage{graphicx}
\usepackage{float}
\usepackage{listings}
\newcommand{\myvec}[1]{\ensuremath{\begin{pmatrix}#1\end{pmatrix}}}
\let\vec\mathbf
\providecommand{\sbrak}[1]{\ensuremath{{}\left[#1\right]}}
\providecommand{\lsbrak}[1]{\ensuremath{{}\left[#1\right.}}
\providecommand{\rsbrak}[1]{\ensuremath{{}\left.#1\right]}}
\providecommand{\brak}[1]{\ensuremath{\left(#1\right)}}
\providecommand{\lbrak}[1]{\ensuremath{\left(#1\right.}}
\providecommand{\rbrak}[1]{\ensuremath{\left.#1\right)}}
\providecommand{\cbrak}[1]{\ensuremath{\left\{#1\right\}}}
\providecommand{\lcbrak}[1]{\ensuremath{\left\{#1\right.}}
\providecommand{\rcbrak}[1]{\ensuremath{\left.#1\right\}}}
\providecommand{\norm}[1]{\left\lVert#1\right\rVert}
\providecommand{\abs}[1]{\left\vert#1\right\vert}
\let\vec\mathbf
%\newcommand{\norm}[1]{\lVert#1\rVert}
\renewcommand{\vec}[1]{\textbf{#1}}
\begin{document}
\onehalfspacing
\begin{center}
	\section*{\textbf{Class 11}}
	\subsection*{Chapter 10 - STRAIGHT LINES}
\end{center}
The following problem is question 11 from exercise 10.4
\begin{enumerate}
    \item Find the equation of the lines through the point (3, 2) which make an angle of 45\degree  with the line x – 2y = 3.
\end{enumerate}
\textbf{Solution:}\\
The given line parameters are
\begin{align}
   \vec{n}=\myvec{1\\-2},c=-5\\
	\vec{P}=\myvec{3\\2}
\end{align}
yielding
\begin{align}
\vec{m}_1=\myvec{2\\1}\\
\vec{m}_2=\myvec{1\\m}
\end{align}
where  $m$ is defined to be the slope of the line. If the angle between the lines be $\theta$,

\begin{align}
\cos \theta = \frac{\vec{m}_1^\top \vec{m}_2}{\norm{\vec{m}_1}\norm{\vec{m}_2}}\\
	\text{given, } \theta = 45\degree\\
\implies \cos45\degree =  \frac{\vec{m}_1^\top \vec{m}_2}{\norm{\vec{m}_1}\norm{\vec{m}_2}}\\
\implies \frac{1}{\sqrt{2}} = \frac{\myvec{2 & 1} \myvec{1\\m}}{\norm{\myvec{2\\1}}\norm{\myvec{1\\m}}}
\end{align}
\begin{align}
\implies \frac{1}{\sqrt{2}}=\frac{2+m}{\sqrt{2^2 + 1}\sqrt{m^2 + 1}}\\
\implies \frac{1}{2}=\frac{m^2 + 4m +4}{5m^2 +5}\\
\text{or, } 3m^2 - 8m -3 = 0
\end{align}
yielding
\begin{align}
m= - \frac{1}{3}, 3
\end{align} 
when m=3,the equation of line passing through $\vec{P}$  is then obtained as
\begin{align}
\vec{n}^{\top} ({\vec{x}-\vec{P}}) = 0\\
\text{where,}{\vec{n}}=\myvec{m\\-1} \\
{\vec{n}}=\myvec{3\\-1} \\
\implies 
	\myvec{3&-1}\cbrak{\vec{x}-\myvec{3\\2}}&=0\\
	&=7 \\
 \implies 	\myvec{3 & -1}\vec{x} &= 7
\end{align}
And, when $m=-\frac{1}{3}$,the equation of the line passing through $\vec{P}$  and having a slope of $-\frac{1}{3}$is
\begin{align}
\vec{n}^{\top} ({\vec{x}-\vec{P}}) = 0\\
{\vec{n}}=\myvec{-\frac{1}{3}\\-1} \\
\implies {\vec{n}}=\myvec{1\\3} \\
\implies 
	\myvec{1&3}\cbrak{\vec{x}-\myvec{3\\2}}&=0\\
	&=9 \\
		\implies 	\myvec{1 & 3}\vec{x} &= 9
\end{align}
Therefore,the equations of the lines are 
\begin{align}
	\myvec{3 & -1}\vec{x} = 7  \text{ and }   \myvec{1 & 3}\vec{x} = 9 .
\end{align}
\begin{figure}[H]
\centering
\includegraphics[width=\columnwidth]{figs/stline.jpg}
\caption{STRAIGHT LINES}
\label{fig:strline.jpg}
\end{figure}




\end{document}

\begin{figure}[H]
\centering
\includegraphics[width=\columnwidth]{chapters/11/10/4/11/figs/strline.jpg}
\caption{STRAIGHT LINES}
\label{fig:chapters/11/10/4/11/figs/strline.jpg}
\end{figure}
\item\textbf{}The scalar product of the vector $\hat{i}+\hat{j}+\hat{k}$ with a unit vector along the sum of vectors $2\hat{i}+4\hat{j}-5\hat{k}$ and $\lambda\hat{i}+2\hat{j}+3\hat{k}$ is equal to one, Find the value of $\lambda$.
\\\\
\textbf{Generalized Construction:}\\
We now that \\
\begin{align}
   &\implies \vec{A}^\top = \frac{\brak{\vec{B}+\vec{C}}}{\norm{\vec{B}+\vec{C}}}\\
       &\implies \vec{A}^\top \brak{\vec{B}+\vec{C}}=\norm{\vec{B}+\vec{C}} \label{eq:Eqat2}\\
       &\implies \vec{C}=\lambda\vec{e}_1+\vec{D}\label{eq:EQT-C}
    \end{align}
    were,
    \begin{align}
       &\implies \norm{\vec{B}+\vec{C}}= \sqrt{\brak{\vec{B}+\vec{C}}^\top\brak{\vec{B}+\vec{C}}}
    \end{align}
From the Equation\eqref{eq:Eqat2},We can do
\begin{align}
   &\implies \vec{A}^\top \brak{\vec{B}+\vec{C}}=\sqrt{\brak{\vec{B}+\vec{C}}^\top\brak{\vec{B}+\vec{C}}}\\
&\implies \vec{A}^\top \brak{\vec{B}+\vec{C}}=\sqrt{\norm{\vec{B}}^2+2\sbrak{\vec{B}^{\top}\vec{C}}+\norm{\vec{C}}^2}\\
&\implies \vec{A}^\top \brak{\vec{B}+\vec{C}}=\sqrt{{\vec{B}^{\top}\vec{B}}+2\sbrak{\vec{B}^{\top}\vec{C}}+{\vec{C}^\top\vec{C}}}\label{eq:Eqt6}
\end{align}
Substitute the $\vec{C}$ Value in the Equation\eqref{eq:Eqt6},We get
\begin{align}
&\implies\vec{A}^{\top}\brak{\vec{B}+\lambda\vec{e}_1+\vec{D}}=\sqrt{\vec{B}^{\top}\vec{B}+2\vec{B}^{\top}\brak{\lambda\vec{e}_1+\vec{D}}+\brak{\lambda\vec{e}_1+\vec{D}}^{\top}\brak{\lambda\vec{e}_1+\vec{D}}}
\end{align}
S.O.B.S,we get
\begin{align}
&\implies\brak{\vec{A}^{\top}\brak{\vec{B}+\lambda\vec{e}_1+\vec{D}}}^{2}=\vec{B}^{\top}\vec{B}+2\vec{B}^{\top}\brak{\lambda\vec{e}_1+\vec{D}}+\brak{\brak{\lambda\vec{e}_1+\vec{D}}^{\top}\brak{\lambda\vec{e}_1+\vec{D}}} \\
&\implies\brak{\vec{A}^{\top}\lambda\vec{e}_1}^{2}+\brak{\vec{A}^{\top}\vec{B}+\vec{D}}^{2}+2\brak{\vec{A}^{\top}\lambda\vec{e}_1}\brak{\vec{A}^{\top}\brak{\vec{B}+\vec{D}}}=\vec{B}^{\top}\vec{B}+2\vec{B}^{\top}\brak{\lambda\vec{e}_1+\vec{D}}+\lambda^{2}+2\lambda\vec{e}_1^{\top}\vec{D}+\vec{D}^{\top}\vec{D}\\
&\implies\brak{\lambda^{2}}+\brak{\vec{A}^{\top}\brak{\vec{B}+\vec{D}}}^{2}+2\brak{\vec{A}^{\top}\lambda\vec{e}_1}\brak{\vec{A}^{\top}\brak{\vec{B}+\vec{D}}}=\vec{B}^{\top}\vec{B}+2\lambda\brak{\vec{B}^{\top}\vec{e}_1+\vec{e}_1^{\top}\vec{D}}+\vec{D}^{\top}\vec{D}+\lambda^{2}\\
&\implies2\lambda\sbrak{\vec{A}^\top\vec{e}_1\vec{A}^\top\brak{\vec{B}+\vec{D}}-\brak{\vec{B}^{\top}\vec{e}_1+\vec{e}_1^{\top}\vec{D}}}=\vec{B}^{\top}\vec{B}+2\lambda\brak{\vec{B}^{\top}\vec{e}_1+\vec{e}_1^{\top}\vec{D}}+\vec{D}^{\top}\vec{D}-\brak{\vec{A}^{\top}\brak{\vec{B}+\vec{D}}}^{2}\\
&\implies2\lambda=\frac{\vec{B}^{\top}\vec{B}+2\vec{B}^{\top}\vec{D}+\vec{D}^{\top}\vec{D}-\brak{\vec{A}^{\top}\brak{\vec{B}+\vec{D}}}^{2}}{\sbrak{\vec{A}^\top\vec{e}_1\vec{A}^\top\brak{\vec{B}+\vec{D}}-\brak{\vec{B}^{\top}\vec{e}_1+\vec{e}_1^{\top}\vec{D}}}}\\
&\implies\lambda=\frac{\vec{B}^{\top}\vec{B}+2\vec{B}^{\top}\vec{D}+\vec{D}^{\top}\vec{D}-\brak{\vec{A}^{\top}\brak{\vec{B}+\vec{D}}}^{2}}{2\sbrak{\vec{A}^\top\vec{e}_1\vec{A}^\top\brak{\vec{B}+\vec{D}}-\brak{\vec{B}^{\top}\vec{e}_1+\vec{e}_1^{\top}\vec{D}}}} \label{eq:EWQ77}
\end{align}
Substitute the Given Data in Equation\eqref{eq:EWQ77},
\begin{align*}
\vec{A}=\myvec{1\\1\\1};\vec{B}=\myvec{2\\4\\-5};\vec{C}=\myvec{\lambda\\2\\3}
\end{align*}
we get,
\begin{align}   
&\implies\lambda=\frac{45-14+13-36}{2\brak{1\brak{6}-2}}\\
&\implies\lambda=\frac{44-36}{8}\\
&\impliedby\lambda=\frac{8}{8}\\
 &\implies \lambda = 1
\end{align}
\end{enumerate}
