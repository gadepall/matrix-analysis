\begin{enumerate}[label=\thesection.\arabic*,ref=\thesection.\theenumi]
\numberwithin{equation}{enumi}
\numberwithin{figure}{enumi}
\numberwithin{table}{enumi}

\item 
\item 
\item  Reduce the following equations into normal form. Find their perpendicular distances from the origin and angle between perpendicular and the positive $x$-axis.
\label{chapters/11/10/3/3}
\begin{enumerate}
	\item $x-\sqrt{3}y+8=0$ 
	\item $y-2=0$
	\item $x-y=4$
\end{enumerate}
\solution
\begin{enumerate}[label=\thesection.\arabic*,ref=\thesection.\theenumi]
\numberwithin{equation}{enumi}
\numberwithin{figure}{enumi}
\numberwithin{table}{enumi}

	\item The distance of the point $\vec{P}(2, 3)$ from the x-axis is

\begin{enumerate}
\item 2
\item 3
\item 1
\item 5 
\end{enumerate}

\item Find the foot of perpendicular from the point $(2,3,-8)$ to the line  $\dfrac{4-x}{2}=\dfrac{y}{6}=\dfrac{1-z}{3}$.Also, find the perpendicular distance from the given point to the line.
\item Find the distance of a point $(2,4,-1)$ from the line $$\frac{x+5}{1}=\frac{y+3}{4}=\frac{z-6}{-9}$$
\item Find the length and the foot of perpendicular from the point $ \brak{1,\dfrac{3}{2} ,2 }$ to the plane $2x-2y+4z+5=0.$
\item Show that the points $(\hat{i}-\hat{j}+3\hat{k})$ and $3(\hat{i}+\hat{j}+\hat{k})$ are equidistant from the plane $\overrightarrow{r} \cdot (5\hat{i}+2\hat{j}-7\hat{k})+9=0$ and lies on opposite side of it.
\item The distance of the plane $\overrightarrow{r} \cdot \brak{ \dfrac{2}{7}\hat{i}+\dfrac{3}{7}\hat{j}-\dfrac{6}{7}\hat{k}}=1$ from the origin is 
\begin{enumerate}
	\item 1
	\item 7
	\item $\dfrac{1}{7}$
	\item None of these	
\end{enumerate}
\item If the foot of perpendicular drawn from the origin to a plane is $(5,-3,-2)$, then the equation of plane is $\overrightarrow{r} \cdot (5\hat{i}-3\hat{j}-2\hat{k})=38.$
\end{enumerate}

\item Find the distance of the point $(-1,1)$ from the line $12\brak{x+6} = 5\brak{y-2}$. 
\label{chapters/11/10/3/4}
\iffalse
\documentclass[12pt]{article}
\usepackage{graphicx}
\usepackage[none]{hyphenat}
\usepackage{graphicx}
\usepackage{listings}
\usepackage[english]{babel}
\usepackage{graphicx}
\usepackage{caption} 
\usepackage{booktabs}
\usepackage{array}
\usepackage{amssymb} % for \because
\usepackage{amsmath}   % for having text in math mode
\usepackage{extarrows} % for Row operations arrows
\usepackage{listings}
\lstset{
  frame=single,
  breaklines=true
}
\usepackage{hyperref}
  
%Following 2 lines were added to remove the blank page at the beginning
\usepackage{atbegshi}% http://ctan.org/pkg/atbegshi
\AtBeginDocument{\AtBeginShipoutNext{\AtBeginShipoutDiscard}}


%New macro definitions
\newcommand{\mydet}[1]{\ensuremath{\begin{vmatrix}#1\end{vmatrix}}}
\providecommand{\brak}[1]{\ensuremath{\left(#1\right)}}
\providecommand{\norm}[1]{\left\lVert#1\right\rVert}
\newcommand{\solution}{\noindent \textbf{Solution: }}
\newcommand{\myvec}[1]{\ensuremath{\begin{pmatrix}#1\end{pmatrix}}}
\providecommand{\abs}[1]{\left\vert#1\right\vert}
\let\vec\mathbf

\begin{document}

\begin{center}
\title{\textbf{Equation  of Line}}
\date{\vspace{-5ex}} %Not to print date automatically
\maketitle
\end{center}
\setcounter{page}{1}

\section{11$^{th}$ Maths - Chapter 10}
This is Problem-4 from Exercise 10.3
\begin{enumerate}
		\fi
\item Find the distance of the point $(-1,1)$ from the line $12\brak{x+6} = 5\brak{y-2}$. 
	\\
\solution 
\begin{enumerate}
\item The equation of the line is $12\brak{x+6} = 5\brak{y-2}$. Rearranging the equation, 
\begin{align}
12x-5y = -10-72 \\
12x-5y = -82
\end{align}

This can be equated to

\begin{align}
	\label{eq:11/10/3/4/2Dline}
	\vec{n}^\top\vec{x} = c 
\end{align}
\begin{align}
	\text{ where }
		\vec{n} = \myvec{
	  12 \\
	  -5 
	  } ,   c = -82 
\end{align}
		We need to compute the distance from a point $\vec{P}\myvec{-1 \\ 1}$ to the line. 
Without loss of generality, let $\vec{A}$ be the foot of the perpendicular from $\vec{P}$ to the line in Equation \eqref{eq:11/10/3/4/2Dline}. 
The equation of the normal to Equation \eqref{eq:11/10/3/4/2Dline} can then be expressed as 

\begin{align}
	\label{eq:11/10/3/4/dir_line_normal_dist}
	\vec{x} &= \vec{A} + \lambda \vec{n}
	\\
	\implies 
	\label{eq:11/10/3/4/dir_line_normal_dist_pa}
	\vec{P}- \vec{A} &=  \lambda \vec{n}
\end{align}

$\because \vec{P}$ lies on 
		\eqref{eq:11/10/3/4/dir_line_normal_dist}.
From the above, the desired distance can be expressed as 

\begin{align}
	\label{eq:11/10/3/4/dir_line_normal_dist_pa_d}
d = 	\norm{\vec{P}- \vec{A}}= \abs{\lambda} \norm{\vec{n}}
\end{align}

From 
	\eqref{eq:11/10/3/4/dir_line_normal_dist_pa},

\begin{align}
	\vec{n}^{\top}
	\brak{\vec{P}- \vec{A}} &=  \lambda \vec{n}^{\top}\vec{n} = \lambda\norm{\vec{n}}^2
	\\
	\implies \abs{\lambda}&= \frac{\abs{\vec{n}^{\top}
	\brak{\vec{P}- \vec{A}}}}{\norm{\vec{n}}^2} 
\end{align}

Substituting the above in \eqref{eq:11/10/3/4/dir_line_normal_dist_pa_d} and using the fact that

\begin{align}
   \vec{n}^{\top}\vec{A} = c
\end{align}

from 	\eqref{eq:11/10/3/4/2Dline}, yields 

\begin{align}
	\label{eq:11/10/3/4/line_dist_2d}
	d = \frac{\abs{   \vec{n}^{\top}\vec{P}-c }}{\norm{\vec{n}}}	
\end{align}

\begin{align}
	= \frac{\abs{  \myvec{12 & -5 }\myvec{-1 \\ 1}-\brak{-82} }}{\sqrt{12^2+\brak{-5}^2}} \\	
	= \frac{\abs{  -17 + 82 }}{\sqrt{169}}	
	= \frac{\abs{65 }}{13}
	= 5 \text{ units }
\end{align}
\item The foot of the perpendicular from $\vec{P}\myvec{-1 \\ 1}$ to line in \eqref{eq:11/10/3/4/2Dline} is expressed as
\begin{align}
	\label{eq:11/10/3/4/foot_of_perpendicular}
	\myvec{\vec{m} & \vec{n}}^\top\vec{A} &= 
	   \myvec{
              \vec{m}^\top\vec{P}\\
	      c
	      }
\end{align}
where $\vec{m}$ is the direction vector of the given line
\begin{align}
    \because \vec{n} = \myvec{ 12 \\ -5},   
    \vec{m} = \myvec{ 5 \\ 12} \\ 
	\eqref{eq:11/10/3/4/foot_of_perpendicular} \implies \myvec{5&12 \\ 12 & -5}\vec{A} &= \myvec{\myvec{5 & 12}\myvec{-1 \\ 1}\\ -82} \\
	\label{eq:11/10/3/4/sysEq1}
	\myvec{5&12 \\ 12 & -5}\vec{A} &= \myvec{7 \\ -82} 
\end{align}	
The augmented matrix for the system equations in \eqref{eq:11/10/3/4/sysEq1} is expressed as
\begin{align}
	\myvec{5&12 & \vrule & 7 \\ 12 & -5 & \vrule & -82} 
\end{align}
Performing sequence of row operations to transform into RREF form
\begin{align}
        \xleftrightarrow[]{{R_2\rightarrow R_2-\frac{12}{5}R_1}}  
	\myvec{5&12 & \vrule & 7 \\ 0 & -\frac{169}{5} & \vrule & -\frac{494}{5}} \\
	\xleftrightarrow[{R_1\rightarrow \frac{1}{5}}R_1]{{R_2\rightarrow \frac{-5}{169}R_2}}  
	\myvec{1 & \frac{12}{5} & \vrule & \frac{7}{5} \\ 0 & 1 & \vrule & \frac{38}{13}} \\
	\xleftrightarrow[]{{R_1\rightarrow R_1-\frac{12}{5}R_2}}  
	\myvec{1 & 0 & \vrule & -\frac{73}{13} \\ 0 & 1 & \vrule & \frac{38}{13}} \\
	\vec{A} = \myvec{ -\frac{73}{13} \\ \frac{38}{13} }
\end{align}
\end{enumerate}
The desired line and the perpendicular line from $\vec{P}$ is shown as in Fig. \ref{fig:11/10/3/4/Fig1}
\begin{figure}[!h]
	\begin{center}
		\includegraphics[width=\columnwidth]{chapters/11/10/3/4/figs/problem4.pdf}
	\end{center}
\caption{}
\label{fig:11/10/3/4/Fig1}
\end{figure}

\item Find the points on the x-axis, whose distances from the line $\frac{x}{3}+\frac{y}{4}=1$ are 4 units.
\label{chapters/11/10/3/5}
	\\
	\solution
\iffalse
\documentclass[12pt]{article}
\usepackage{graphicx}
\usepackage{amsmath}
\usepackage{mathtools}
\usepackage{gensymb}
\usepackage{amssymb}

\newcommand{\mydet}[1]{\ensuremath{\begin{vmatrix}#1\end{vmatrix}}}
\providecommand{\brak}[1]{\ensuremath{\left(#1\right)}}
\providecommand{\norm}[1]{\left\lVert#1\right\rVert}
\newcommand{\solution}{\noindent \textbf{Solution: }}
\newcommand{\myvec}[1]{\ensuremath{\begin{pmatrix}#1\end{pmatrix}}}
\providecommand{\abs}[1]{\left\vert#1\right\vert}	
\let\vec\mathbf

\begin{document}
\begin{center}
\textbf\large{CLASS 11 CHAPTER-11 \\ LINES}

\end{center}
\section*{Exercise 10.3}


\solution
\fi
The given line can be expressed as 
\begin{align}
	\vec{n}^{\top}\vec{x}&=c,
	\text{ where }
		\vec{n} &= \myvec{4\\3} , c = 12
\end{align}
The distance formula is given by
\begin{align}
	d = \frac{\abs{\vec{n}^\top\vec{P}-c}}{\norm{\vec{n}}}
\end{align}
Let the desired point be
\begin{align}
	\vec{P} = x\vec{e}_{1} = \myvec{x\\0}
\end{align}
Substituting the values in the distance formula, 
\begin{align}
	d &= \frac{\abs{\vec{n}^\top\vec{P}-c}}{\norm{\vec{n}}}\\
	  &= \frac{\abs{x\vec{n}^\top\vec{e}_{1}-c}}{\norm{\vec{n}}}
	  \\
	  \implies 
	\abs{x\vec{n}^\top\vec{e}_{1}-c} &= d\norm{\vec{n}}
	\\
	\text{or, }	x = \frac{\pm d\norm{\vec{n}}+c}{\vec{n}^\top\vec{e}_{1}}
\end{align}
Since 
\begin{align}
	d &= 4,
\end{align}
substituting numerical values, 
\begin{align}
	x = 8,
	 -2
\end{align}
This is verified in Fig. 
\ref{fig:11/10/3/5/Fig1}.	
\begin{figure}[!h]
	\begin{center} 
	    \includegraphics[width=\columnwidth]{chapters/11/10/3/5/figs/line2}
	\end{center}
\caption{}
\label{fig:11/10/3/5/Fig1}
\end{figure}



\item Find the distance between parallel lines
\label{chapters/11/10/3/6}
\begin{enumerate}
	\item $15x+8y-34=0$ and  $15x+8y+31=0$ \\
	\item  $l(x+y)+p=0$ and  $l(x+y)-r=0$
\end{enumerate}
	\solution
\iffalse
\documentclass[10pt]{article}
       \usepackage[latin1]{inputenc}
       \usepackage{fullpage}
       \usepackage{color}
       \usepackage{array}
       \usepackage{longtable}
       \usepackage{calc}
       \usepackage{multirow}
       \usepackage{hhline}
       \usepackage{ifthen}
\usepackage{graphicx}
\def\inputGnumericTable{}
\usepackage[none]{hyphenat}
\usepackage{graphicx}
\usepackage{listings}
\usepackage[english]{babel}
\usepackage{graphicx}
\usepackage{caption} 
\usepackage{booktabs}
\usepackage{gensymb}
\usepackage{array}
\usepackage{amssymb} % for \because
\usepackage{amsmath}   % for having text in math mode
\usepackage{extarrows} % for Row operations arrows
\usepackage{listings}
\lstset{
  frame=single,
  breaklines=true
}
\usepackage{hyperref}
%Following 2 lines were added to remove the blank page at the beginning
\usepackage{atbegshi}% http://ctan.org/pkg/atbegshi
\AtBeginDocument{\AtBeginShipoutNext{\AtBeginShipoutDiscard}}
%New macro definitions
\newcommand{\mydet}[1]{\ensuremath{\begin{vmatrix}#1\end{vmatrix}}}
\providecommand{\brak}[1]{\ensuremath{\left(#1\right)}}
\providecommand{\norm}[1]{\left\lVert#1\right\rVert}
\newcommand{\solution}{\noindent \textbf{Solution: }}
\newcommand{\myvec}[1]{\ensuremath{\begin{pmatrix}#1\end{pmatrix}}}
\providecommand{\abs}[1]{\left\vert#1\right\vert}
\let\vec\mathbf
\begin{document}

\begin{center}
\title{\textbf{STRAIGHT LINES}}
\date{\vspace{-5ex}} %Not to print date automatically
\maketitle
\end{center}

\section{11$^{th}$ Maths - Chapter 10}
This is Problem 5 from Exercise-10.4
\begin{enumerate}

\solution
\fi
Let
\begin{align}
	\vec{A}=\myvec{\cos\theta\\\sin\theta},\vec{B}&=\myvec{\cos\phi\\\sin\phi}\\
\implies	\vec{m}=\vec{B}-\vec{A}&=\myvec{\cos\phi-\cos\theta\\\sin\phi-\sin\theta}
\end{align}
The normal vector is then given by,
\begin{align}
\vec{n}=\myvec{\sin\phi-\sin\theta\\\cos\theta-\cos\phi} \implies
\norm{\vec{n}}=2\sin\brak{\frac{\phi-\theta}{2}}
\end{align}
The equation of the line is
\begin{align}
\vec{n}^\top\brak{\vec{x}-\vec{A}}&=0\\
\implies\myvec{\sin\phi-\sin\theta&\cos\theta-\cos\phi}\vec{x}&=\sin\brak{\phi-\theta}
\label{eq:chapters/11/10/4/5/1}
\end{align}
Thus, 
\begin{align}
c=\sin\brak{\phi-\theta}
\end{align}
The perpendicular distance from the origin to the line is
\begin{align}
d&=\frac{\abs{c}}{\norm{\vec{n}}}\\
\implies d&=\frac{\sin\brak{\phi-\theta}}{2\sin\brak{\frac{\phi-\theta}{2}}} = \cos\brak{\frac{\phi-\theta}{2}}
\label{eq:chapters/11/10/4/5/2}
\end{align}

\item Find the coordinates of the foot of the perpendicular from $(-1, 3)$ to the line $3x-4y-16=0$.  
\label{chapters/11/10/3/14}
\\
\solution
\iffalse
\documentclass[12pt]{article}
\usepackage{graphicx}
%\documentclass[journal,12pt,twocolumn]{IEEEtran}
\usepackage[none]{hyphenat}
\usepackage{graphicx}
\usepackage{listings}
\usepackage[english]{babel}
\usepackage{graphicx}
\usepackage{caption} 
\usepackage{hyperref}
\usepackage{booktabs}
\usepackage{commath}
\usepackage{gensymb}
\usepackage{array}
\usepackage{amsmath}   % for having text in math mode
\usepackage{mathtools}
\usepackage{listings}
\let\vec\mathbf
\lstset{
  frame=single,
  breaklines=true
}
  
%Following 2 lines were added to remove the blank page at the beginning
\usepackage{atbegshi}% http://ctan.org/pkg/atbegshi
\AtBeginDocument{\AtBeginShipoutNext{\AtBeginShipoutDiscard}}
%
%New macro definitions
\newcommand{\mydet}[1]{\ensuremath{\begin{vmatrix}#1\end{vmatrix}}}
\providecommand{\brak}[1]{\ensuremath{\left(#1\right)}}
\providecommand{\norm}[1]{\left\lVert#1\right\rVert}
\newcommand{\solution}{\noindent \textbf{Solution: }}
\newcommand{\myvec}[1]{\ensuremath{\begin{pmatrix}#1\end{pmatrix}}}
\let\vec\mathbf
\begin{document}
\begin{center}
\title{\textbf{LINES}}
\date{\vspace{-5ex}} %Not to print date automatically
\maketitle
\end{center}
\setcounter{page}{1}
\section*{CHAPTER 11 - STRAIGHT LINES}
\section*{Excercise 10.3}
\solution 
\begin{enumerate}
\section{Solution}
		\fi
		Let
\begin{align}
 \vec{P}=\myvec{
-1\\
3
}
\end{align}
The line parameters are
\begin{align}
\vec{n}=\myvec{
3\\
-4
}, c=16
\end{align}
The desired foot of the perpendicular is then given by 
\begin{align}
\myvec{4&3\\3&-4}\vec{A}&=\myvec{\myvec{4&3}\myvec{-1\\3}\\16}\\
&=\myvec{5\\16}  
\end{align}
The augmented matrix for the above system is
\begin{align}
  \myvec{
   4 &  3  & 5\\
   3 & -4  & 16} 
  \xleftrightarrow[]{R_2=R_2-\frac{3}{4}R_1}
  \myvec{
  4 & 3 & 5\\
  0 & \frac{-25}{4} & \frac{49}{4}} 
\\
  \xleftrightarrow{R_2=\frac{-4}{25}}
  \myvec{
  4 & 3 & 5\\
  0 & 1 & \frac{-49}{25}}
  \xleftrightarrow{R_1=\frac{1}{4}R_1}
  \myvec{
  1 & \frac{3}{4} & \frac{5}{4}\\
  0 & 1 & \frac{-49}{25}}
\\
  \xleftrightarrow{R_1=R_1-\frac{3}{4}R_2}
  \myvec{
  1 & 0 & \frac{68}{25}\\
  0 & 1 & \frac{-49}{25}}          
\end{align}
yielding
\begin{align}
\vec{A}=\myvec{
\frac{68}{25}\\[1pt]
\frac{-49}{25}
}
\end{align}
See Fig.
\ref{fig:chapters/11/10/3/14/Fig}.
\begin{figure}[!h]
	\begin{center} 
	    \includegraphics[width=\columnwidth]{chapters/11/10/3/14/figs/lines.png}
	\end{center}
\caption{}
\label{fig:chapters/11/10/3/14/Fig}
\end{figure}

\item  If ${p}$ and ${q}$ are the lengths of perpendiculars from the origin to the lines ${x}\cos\theta - {y}\sin\theta =  {k}\cos2\theta$ and ${x}\sec\theta + {y}\cosec\theta = {k}$, respectively, prove that ${p}^2 + 4{q}^2 = {k}^2$
\label{chapters/11/10/3/16}
\\
\solution
\iffalse
\documentclass[journal,10pt,twocolumn]{article}
\usepackage{graphicx}
\usepackage[margin=0.5in]{geometry}
\usepackage[cmex10]{amsmath}
\usepackage{array}
\usepackage{booktabs}
\usepackage{listings}
\title{\textbf{Line Assignment}}
\author{Bhavani Kanike}
\date{October 2022}

\providecommand{\norm}[1]{\left\lVert#1\right\rVert}
\providecommand{\abs}[1]{\left\vert#1\right\vert}
\let\vec\mathbf
\newcommand{\myvec}[1]{\ensuremath{\begin{pmatrix}#1\end{pmatrix}}}
\newcommand{\mydet}[1]{\ensuremath{\begin{vmatrix}#1\end{vmatrix}}}
\providecommand{\brak}[1]{\ensuremath{\left(#1\right)}}

\begin{document}

\maketitle
\paragraph{\textit{Problem Statement} 
\fi
ABCD is a quadrilateral in which $\vec{P}, \vec{Q}, \vec{R}$ and $\vec{S}$ are mid-points of the sides AB, BC, CD and DA (see Fig \ref{fig:9/8/2/1}). AC is a diagonal. 
		
Show that 
\begin{enumerate}
	\item $SR \parallel AC$ and $SR =\frac{1}{2} AC$
\item $PQ = SR$
\item $PQRS$ is a parallelogram.
\end{enumerate}
 	\begin{figure}
		\centering
 \includegraphics[width=\columnwidth]{chapters/9/8/2/1/figs/line1.pdf}
		\caption{}
		\label{fig:9/8/2/1}
  	\end{figure}
	\solution 
	Using 
	  \eqref{eq:section_formula},
	\begin{align}
		\label{eq:9/8/2/1}
		\begin{split}
		\vec{P} &= \frac{\vec{A}+\vec{B}}{2}\\
 \vec{Q} &= \frac{\vec{C}+\vec{B}}{2}\\
 \vec{R} &= \frac{\vec{C}+\vec{D}}{2}\\
 \vec{S} &= \frac{\vec{D}+\vec{A}}{2}
		\end{split}
	\end{align}
\begin{enumerate}
	\item
	Consequently, 
	\begin{align}
\vec{R}
		-\vec{S} &= \frac{\vec{C}-\vec{A}}{2}
		\\
		\implies SR &\parallel AC
	\end{align}
	Also, 
	\begin{align}
		\norm{\vec{R}
		-\vec{S}} &= \frac{\norm{\vec{C}-\vec{A}}}{2}
		\\
		\implies SR &= \frac{1}{2}AC
	\end{align}
\item 	From 
		\eqref{eq:9/8/2/1},
	\begin{align}
\vec{R}
		-\vec{S} = \vec{Q}-\vec{P}
	\end{align}
	which means that $PQRS$ is a parallelogram and $PQ = SR$.
\end{enumerate}
%
\iffalse
\begin{figure}[h]
\centering
\includegraphics[width=1\columnwidth]
\caption{Figure}
\label{fig:triangle}
\end{figure}

\section*{Solution}

$\boldsymbol Given :$  ABCD is a Quadrilateral P,Q,R and S are the midpoints of line AB,BC,CD,DA.We can obtain the points P,Q,R and S from A,B,C and D and are given by\\\\
\boldmath
\unboldmath
(3) To prove that PQRS is a parallelogram we need to prove  PQ // SR
To prove SR $\parallel$ PQ\\
Direction vector of line SR  $\boldsymbol {(R-S) =  \frac{(C-A)}{2}}$\\\\
Direction vector of line PQ  $\boldsymbol {(Q-P)= \frac{(C-A)}{2}}$\\\\
\begin{equation}
	\boldsymbol {(R-S) = (Q-P) = \frac{(C-A)}{2}}\\
\end{equation}
Since the direction vectors of line SR and PQ are in same direction\\\\
$SR \parallel PQ$\\
Therefore,
$\boldsymbol{ PQRS }$ is a parallelogram\\\\

	
(1)  Directional vector of line SR  = $\boldsymbol {(R-S)}$ = $\frac{\boldsymbol{(C-A)}}{2} $\\
Directional vector of line AC  = $\boldsymbol {(C-A)}$\\

It is observed that the constant k is $\frac{1}{2}$

Therefore
\begin{equation}
	SR \parallel AC
\end{equation} 

and from equation 1 
\begin{equation}
	\boldsymbol {SR = \frac{1}{2}AC}    
\end{equation}\\


(2)   To prove PQ = SR\\ 
		From euqation 1\\\\
\begin{equation}
		\boldsymbol{ (Q-P) = (R-S) = \frac{(C-A)}{2}}
\end{equation}
	 



\section{Execution}
The below python code realizes the construction:
\begin{lstlisting}
https://github.com/bhavani360/FWC_assignments
\end{lstlisting}
	
\section*{Construction}
The dimensions of the Quadrilateral ABCD are taken as below\\
{
\setlength\extrarowheight{2pt}
\centering
	\begin{tabular}{|c|c|}
	\hline
	\textbf{symbol}&\textbf{value}\\
	\hline
	r&8\\
	\hline
	$\theta$&pi/2.5\\
	\hline
	d&7\\
	\hline
	A&(0,0)\\
	\hline
	B&(d,0)\\
	\hline
	D&(rcos$\theta$,rsin$\theta$)\\
	\hline
	C&(D/1.5)+B\\
	\hline
\end{tabular}
}
\end{document}
\fi

\item In the triangle $ABC$ with vertices $\vec{A} \brak{2, 3}$, $\vec{B} \brak{4, –1}$ and $\vec{C} \brak{1, 2}$, find the equation and length of altitude from the vertex $\vec{A}$.
\label{chapters/11/10/3/17}
\\
\solution
\iffalse
\documentclass[journal,10pt,twocolumn]{article}
\usepackage{graphicx}
\usepackage[margin=0.5in]{geometry}
\usepackage[cmex10]{amsmath}
\usepackage{array}
\usepackage{booktabs}
\usepackage{listings}
\title{\textbf{Line Assignment}}
\author{Bhavani Kanike}
\date{October 2022}

\providecommand{\norm}[1]{\left\lVert#1\right\rVert}
\providecommand{\abs}[1]{\left\vert#1\right\vert}
\let\vec\mathbf
\newcommand{\myvec}[1]{\ensuremath{\begin{pmatrix}#1\end{pmatrix}}}
\newcommand{\mydet}[1]{\ensuremath{\begin{vmatrix}#1\end{vmatrix}}}
\providecommand{\brak}[1]{\ensuremath{\left(#1\right)}}

\begin{document}

\maketitle
\paragraph{\textit{Problem Statement} 
\fi
ABCD is a quadrilateral in which $\vec{P}, \vec{Q}, \vec{R}$ and $\vec{S}$ are mid-points of the sides AB, BC, CD and DA (see Fig \ref{fig:9/8/2/1}). AC is a diagonal. 
		
Show that 
\begin{enumerate}
	\item $SR \parallel AC$ and $SR =\frac{1}{2} AC$
\item $PQ = SR$
\item $PQRS$ is a parallelogram.
\end{enumerate}
 	\begin{figure}
		\centering
 \includegraphics[width=\columnwidth]{chapters/9/8/2/1/figs/line1.pdf}
		\caption{}
		\label{fig:9/8/2/1}
  	\end{figure}
	\solution 
	Using 
	  \eqref{eq:section_formula},
	\begin{align}
		\label{eq:9/8/2/1}
		\begin{split}
		\vec{P} &= \frac{\vec{A}+\vec{B}}{2}\\
 \vec{Q} &= \frac{\vec{C}+\vec{B}}{2}\\
 \vec{R} &= \frac{\vec{C}+\vec{D}}{2}\\
 \vec{S} &= \frac{\vec{D}+\vec{A}}{2}
		\end{split}
	\end{align}
\begin{enumerate}
	\item
	Consequently, 
	\begin{align}
\vec{R}
		-\vec{S} &= \frac{\vec{C}-\vec{A}}{2}
		\\
		\implies SR &\parallel AC
	\end{align}
	Also, 
	\begin{align}
		\norm{\vec{R}
		-\vec{S}} &= \frac{\norm{\vec{C}-\vec{A}}}{2}
		\\
		\implies SR &= \frac{1}{2}AC
	\end{align}
\item 	From 
		\eqref{eq:9/8/2/1},
	\begin{align}
\vec{R}
		-\vec{S} = \vec{Q}-\vec{P}
	\end{align}
	which means that $PQRS$ is a parallelogram and $PQ = SR$.
\end{enumerate}
%
\iffalse
\begin{figure}[h]
\centering
\includegraphics[width=1\columnwidth]
\caption{Figure}
\label{fig:triangle}
\end{figure}

\section*{Solution}

$\boldsymbol Given :$  ABCD is a Quadrilateral P,Q,R and S are the midpoints of line AB,BC,CD,DA.We can obtain the points P,Q,R and S from A,B,C and D and are given by\\\\
\boldmath
\unboldmath
(3) To prove that PQRS is a parallelogram we need to prove  PQ // SR
To prove SR $\parallel$ PQ\\
Direction vector of line SR  $\boldsymbol {(R-S) =  \frac{(C-A)}{2}}$\\\\
Direction vector of line PQ  $\boldsymbol {(Q-P)= \frac{(C-A)}{2}}$\\\\
\begin{equation}
	\boldsymbol {(R-S) = (Q-P) = \frac{(C-A)}{2}}\\
\end{equation}
Since the direction vectors of line SR and PQ are in same direction\\\\
$SR \parallel PQ$\\
Therefore,
$\boldsymbol{ PQRS }$ is a parallelogram\\\\

	
(1)  Directional vector of line SR  = $\boldsymbol {(R-S)}$ = $\frac{\boldsymbol{(C-A)}}{2} $\\
Directional vector of line AC  = $\boldsymbol {(C-A)}$\\

It is observed that the constant k is $\frac{1}{2}$

Therefore
\begin{equation}
	SR \parallel AC
\end{equation} 

and from equation 1 
\begin{equation}
	\boldsymbol {SR = \frac{1}{2}AC}    
\end{equation}\\


(2)   To prove PQ = SR\\ 
		From euqation 1\\\\
\begin{equation}
		\boldsymbol{ (Q-P) = (R-S) = \frac{(C-A)}{2}}
\end{equation}
	 



\section{Execution}
The below python code realizes the construction:
\begin{lstlisting}
https://github.com/bhavani360/FWC_assignments
\end{lstlisting}
	
\section*{Construction}
The dimensions of the Quadrilateral ABCD are taken as below\\
{
\setlength\extrarowheight{2pt}
\centering
	\begin{tabular}{|c|c|}
	\hline
	\textbf{symbol}&\textbf{value}\\
	\hline
	r&8\\
	\hline
	$\theta$&pi/2.5\\
	\hline
	d&7\\
	\hline
	A&(0,0)\\
	\hline
	B&(d,0)\\
	\hline
	D&(rcos$\theta$,rsin$\theta$)\\
	\hline
	C&(D/1.5)+B\\
	\hline
\end{tabular}
}
\end{document}
\fi

\item If $p$ is the length of perpendicular from origin to the line whose intercepts on the axes are $a$ and $b$, then show that 
\begin{align}
	\frac{1}{p^2} = \frac{1}{a^2}+ \frac{1}{b^2}
\end{align}
\label{chapters/11/10/3/18}
\iffalse
\documentclass[10pt]{article}
       \usepackage[latin1]{inputenc}
       \usepackage{fullpage}
       \usepackage{color}
       \usepackage{array}
       \usepackage{longtable}
       \usepackage{calc}
       \usepackage{multirow}
       \usepackage{hhline}
       \usepackage{ifthen}
\usepackage{graphicx}
\def\inputGnumericTable{}
\usepackage[none]{hyphenat}
\usepackage{graphicx}
\usepackage{listings}
\usepackage[english]{babel}
\usepackage{graphicx}
\usepackage{caption} 
\usepackage{booktabs}
\usepackage{gensymb}
\usepackage{array}
\usepackage{amssymb} % for \because
\usepackage{amsmath}   % for having text in math mode
\usepackage{extarrows} % for Row operations arrows
\usepackage{listings}
\lstset{
  frame=single,
  breaklines=true
}
\usepackage{hyperref}
%Following 2 lines were added to remove the blank page at the beginning
\usepackage{atbegshi}% http://ctan.org/pkg/atbegshi
\AtBeginDocument{\AtBeginShipoutNext{\AtBeginShipoutDiscard}}
%New macro definitions
\newcommand{\mydet}[1]{\ensuremath{\begin{vmatrix}#1\end{vmatrix}}}
\providecommand{\brak}[1]{\ensuremath{\left(#1\right)}}
\providecommand{\norm}[1]{\left\lVert#1\right\rVert}
\newcommand{\solution}{\noindent \textbf{Solution: }}
\newcommand{\myvec}[1]{\ensuremath{\begin{pmatrix}#1\end{pmatrix}}}
\providecommand{\abs}[1]{\left\vert#1\right\vert}
\let\vec\mathbf
\begin{document}

\begin{center}
\title{\textbf{STRAIGHT LINES}}
\date{\vspace{-5ex}} %Not to print date automatically
\maketitle
\end{center}

\section{11$^{th}$ Maths - Chapter 10}
This is Problem 5 from Exercise-10.4
\begin{enumerate}

\solution
\fi
Let
\begin{align}
	\vec{A}=\myvec{\cos\theta\\\sin\theta},\vec{B}&=\myvec{\cos\phi\\\sin\phi}\\
\implies	\vec{m}=\vec{B}-\vec{A}&=\myvec{\cos\phi-\cos\theta\\\sin\phi-\sin\theta}
\end{align}
The normal vector is then given by,
\begin{align}
\vec{n}=\myvec{\sin\phi-\sin\theta\\\cos\theta-\cos\phi} \implies
\norm{\vec{n}}=2\sin\brak{\frac{\phi-\theta}{2}}
\end{align}
The equation of the line is
\begin{align}
\vec{n}^\top\brak{\vec{x}-\vec{A}}&=0\\
\implies\myvec{\sin\phi-\sin\theta&\cos\theta-\cos\phi}\vec{x}&=\sin\brak{\phi-\theta}
\label{eq:chapters/11/10/4/5/1}
\end{align}
Thus, 
\begin{align}
c=\sin\brak{\phi-\theta}
\end{align}
The perpendicular distance from the origin to the line is
\begin{align}
d&=\frac{\abs{c}}{\norm{\vec{n}}}\\
\implies d&=\frac{\sin\brak{\phi-\theta}}{2\sin\brak{\frac{\phi-\theta}{2}}} = \cos\brak{\frac{\phi-\theta}{2}}
\label{eq:chapters/11/10/4/5/2}
\end{align}

\item What are the points on the y-axis whose distance from the line $\frac{x}{3}+\frac{y}{4}=1$ is 4 units.
\\
\solution
		\iffalse
\documentclass[12pt]{article}
\usepackage{graphicx}
\usepackage[none]{hyphenat}
\usepackage{graphicx}
\usepackage{listings}
\usepackage[english]{babel}
\usepackage{graphicx}
\usepackage{caption} 
\usepackage{booktabs}
\usepackage{array}
\usepackage{amssymb} % for \because
\usepackage{amsmath}   % for having text in math mode
\usepackage{extarrows} % for Row operations arrows
\usepackage{listings}
\usepackage[utf8]{inputenc}
\lstset{
  frame=single,
  breaklines=true
}
\usepackage{hyperref}
  
%Following 2 lines were added to remove the blank page at the beginning
\usepackage{atbegshi}% http://ctan.org/pkg/atbegshi
\AtBeginDocument{\AtBeginShipoutNext{\AtBeginShipoutDiscard}}


%New macro definitions
\newcommand{\mydet}[1]{\ensuremath{\begin{vmatrix}#1\end{vmatrix}}}
\providecommand{\brak}[1]{\ensuremath{\left(#1\right)}}
\newcommand{\solution}{\noindent \textbf{Solution: }}
\newcommand{\myvec}[1]{\ensuremath{\begin{pmatrix}#1\end{pmatrix}}}
\providecommand{\norm}[1]{\left\lVert#1\right\rVert}
\providecommand{\abs}[1]{\left\vert#1\right\vert}
\let\vec\mathbf

\begin{document}

\begin{center}
\title{\textbf{LINE}}
\date{\vspace{-5ex}} %Not to print date automatically
\maketitle
\end{center}

\section{11$^{th}$ Maths - EXERCISE-10.4}
\begin{enumerate}
\end{enumerate}
\section{SOLUTION}
\fi
Given line parameters are
\begin{align}
\vec{n}=\myvec{4\\3},\,
c=12.
\end{align}
The distance of the line from y-axis
\begin{align}
d&=\frac{\vec{n}^\top\vec{P}-c}{\abs{n}}\\
\implies\pm4&=\frac{\myvec{0\\ 3y}-12}{5}\\
	\implies y&= \frac{32}{3}\text{ or }y=\frac{-8}{3}
\end{align}
See Fig. 
		\ref{fig:chapters/11/10/4/4/Figure}.
\begin{figure}[h]
\centering
\includegraphics[width=\columnwidth]{chapters/11/10/4/4/figs/fig.png}
\caption{}
		\label{fig:chapters/11/10/4/4/Figure}
\end{figure}

\item Find perpendicular distance from the origin to the line joining the points$(\cos\theta,\sin\theta)$ and $(\cos\phi,\sin\phi)$.
\\
\solution
		\iffalse
\documentclass[10pt]{article}
       \usepackage[latin1]{inputenc}
       \usepackage{fullpage}
       \usepackage{color}
       \usepackage{array}
       \usepackage{longtable}
       \usepackage{calc}
       \usepackage{multirow}
       \usepackage{hhline}
       \usepackage{ifthen}
\usepackage{graphicx}
\def\inputGnumericTable{}
\usepackage[none]{hyphenat}
\usepackage{graphicx}
\usepackage{listings}
\usepackage[english]{babel}
\usepackage{graphicx}
\usepackage{caption} 
\usepackage{booktabs}
\usepackage{gensymb}
\usepackage{array}
\usepackage{amssymb} % for \because
\usepackage{amsmath}   % for having text in math mode
\usepackage{extarrows} % for Row operations arrows
\usepackage{listings}
\lstset{
  frame=single,
  breaklines=true
}
\usepackage{hyperref}
%Following 2 lines were added to remove the blank page at the beginning
\usepackage{atbegshi}% http://ctan.org/pkg/atbegshi
\AtBeginDocument{\AtBeginShipoutNext{\AtBeginShipoutDiscard}}
%New macro definitions
\newcommand{\mydet}[1]{\ensuremath{\begin{vmatrix}#1\end{vmatrix}}}
\providecommand{\brak}[1]{\ensuremath{\left(#1\right)}}
\providecommand{\norm}[1]{\left\lVert#1\right\rVert}
\newcommand{\solution}{\noindent \textbf{Solution: }}
\newcommand{\myvec}[1]{\ensuremath{\begin{pmatrix}#1\end{pmatrix}}}
\providecommand{\abs}[1]{\left\vert#1\right\vert}
\let\vec\mathbf
\begin{document}

\begin{center}
\title{\textbf{STRAIGHT LINES}}
\date{\vspace{-5ex}} %Not to print date automatically
\maketitle
\end{center}

\section{11$^{th}$ Maths - Chapter 10}
This is Problem 5 from Exercise-10.4
\begin{enumerate}

\solution
\fi
Let
\begin{align}
	\vec{A}=\myvec{\cos\theta\\\sin\theta},\vec{B}&=\myvec{\cos\phi\\\sin\phi}\\
\implies	\vec{m}=\vec{B}-\vec{A}&=\myvec{\cos\phi-\cos\theta\\\sin\phi-\sin\theta}
\end{align}
The normal vector is then given by,
\begin{align}
\vec{n}=\myvec{\sin\phi-\sin\theta\\\cos\theta-\cos\phi} \implies
\norm{\vec{n}}=2\sin\brak{\frac{\phi-\theta}{2}}
\end{align}
The equation of the line is
\begin{align}
\vec{n}^\top\brak{\vec{x}-\vec{A}}&=0\\
\implies\myvec{\sin\phi-\sin\theta&\cos\theta-\cos\phi}\vec{x}&=\sin\brak{\phi-\theta}
\label{eq:chapters/11/10/4/5/1}
\end{align}
Thus, 
\begin{align}
c=\sin\brak{\phi-\theta}
\end{align}
The perpendicular distance from the origin to the line is
\begin{align}
d&=\frac{\abs{c}}{\norm{\vec{n}}}\\
\implies d&=\frac{\sin\brak{\phi-\theta}}{2\sin\brak{\frac{\phi-\theta}{2}}} = \cos\brak{\frac{\phi-\theta}{2}}
\label{eq:chapters/11/10/4/5/2}
\end{align}

\item Find the equation of line which is equidistant from parallel lines $9x+6y-7=0$ and $3x+2y+6=0$.
\\
\solution
		\iffalse
\documentclass[journal,10pt,twocolumn]{article}
\usepackage{graphicx}
\usepackage[margin=0.5in]{geometry}
\usepackage[cmex10]{amsmath}
\usepackage{array}
\usepackage{booktabs}
\usepackage{listings}
\title{\textbf{Line Assignment}}
\author{Bhavani Kanike}
\date{October 2022}

\providecommand{\norm}[1]{\left\lVert#1\right\rVert}
\providecommand{\abs}[1]{\left\vert#1\right\vert}
\let\vec\mathbf
\newcommand{\myvec}[1]{\ensuremath{\begin{pmatrix}#1\end{pmatrix}}}
\newcommand{\mydet}[1]{\ensuremath{\begin{vmatrix}#1\end{vmatrix}}}
\providecommand{\brak}[1]{\ensuremath{\left(#1\right)}}

\begin{document}

\maketitle
\paragraph{\textit{Problem Statement} 
\fi
ABCD is a quadrilateral in which $\vec{P}, \vec{Q}, \vec{R}$ and $\vec{S}$ are mid-points of the sides AB, BC, CD and DA (see Fig \ref{fig:9/8/2/1}). AC is a diagonal. 
		
Show that 
\begin{enumerate}
	\item $SR \parallel AC$ and $SR =\frac{1}{2} AC$
\item $PQ = SR$
\item $PQRS$ is a parallelogram.
\end{enumerate}
 	\begin{figure}
		\centering
 \includegraphics[width=\columnwidth]{chapters/9/8/2/1/figs/line1.pdf}
		\caption{}
		\label{fig:9/8/2/1}
  	\end{figure}
	\solution 
	Using 
	  \eqref{eq:section_formula},
	\begin{align}
		\label{eq:9/8/2/1}
		\begin{split}
		\vec{P} &= \frac{\vec{A}+\vec{B}}{2}\\
 \vec{Q} &= \frac{\vec{C}+\vec{B}}{2}\\
 \vec{R} &= \frac{\vec{C}+\vec{D}}{2}\\
 \vec{S} &= \frac{\vec{D}+\vec{A}}{2}
		\end{split}
	\end{align}
\begin{enumerate}
	\item
	Consequently, 
	\begin{align}
\vec{R}
		-\vec{S} &= \frac{\vec{C}-\vec{A}}{2}
		\\
		\implies SR &\parallel AC
	\end{align}
	Also, 
	\begin{align}
		\norm{\vec{R}
		-\vec{S}} &= \frac{\norm{\vec{C}-\vec{A}}}{2}
		\\
		\implies SR &= \frac{1}{2}AC
	\end{align}
\item 	From 
		\eqref{eq:9/8/2/1},
	\begin{align}
\vec{R}
		-\vec{S} = \vec{Q}-\vec{P}
	\end{align}
	which means that $PQRS$ is a parallelogram and $PQ = SR$.
\end{enumerate}
%
\iffalse
\begin{figure}[h]
\centering
\includegraphics[width=1\columnwidth]
\caption{Figure}
\label{fig:triangle}
\end{figure}

\section*{Solution}

$\boldsymbol Given :$  ABCD is a Quadrilateral P,Q,R and S are the midpoints of line AB,BC,CD,DA.We can obtain the points P,Q,R and S from A,B,C and D and are given by\\\\
\boldmath
\unboldmath
(3) To prove that PQRS is a parallelogram we need to prove  PQ // SR
To prove SR $\parallel$ PQ\\
Direction vector of line SR  $\boldsymbol {(R-S) =  \frac{(C-A)}{2}}$\\\\
Direction vector of line PQ  $\boldsymbol {(Q-P)= \frac{(C-A)}{2}}$\\\\
\begin{equation}
	\boldsymbol {(R-S) = (Q-P) = \frac{(C-A)}{2}}\\
\end{equation}
Since the direction vectors of line SR and PQ are in same direction\\\\
$SR \parallel PQ$\\
Therefore,
$\boldsymbol{ PQRS }$ is a parallelogram\\\\

	
(1)  Directional vector of line SR  = $\boldsymbol {(R-S)}$ = $\frac{\boldsymbol{(C-A)}}{2} $\\
Directional vector of line AC  = $\boldsymbol {(C-A)}$\\

It is observed that the constant k is $\frac{1}{2}$

Therefore
\begin{equation}
	SR \parallel AC
\end{equation} 

and from equation 1 
\begin{equation}
	\boldsymbol {SR = \frac{1}{2}AC}    
\end{equation}\\


(2)   To prove PQ = SR\\ 
		From euqation 1\\\\
\begin{equation}
		\boldsymbol{ (Q-P) = (R-S) = \frac{(C-A)}{2}}
\end{equation}
	 



\section{Execution}
The below python code realizes the construction:
\begin{lstlisting}
https://github.com/bhavani360/FWC_assignments
\end{lstlisting}
	
\section*{Construction}
The dimensions of the Quadrilateral ABCD are taken as below\\
{
\setlength\extrarowheight{2pt}
\centering
	\begin{tabular}{|c|c|}
	\hline
	\textbf{symbol}&\textbf{value}\\
	\hline
	r&8\\
	\hline
	$\theta$&pi/2.5\\
	\hline
	d&7\\
	\hline
	A&(0,0)\\
	\hline
	B&(d,0)\\
	\hline
	D&(rcos$\theta$,rsin$\theta$)\\
	\hline
	C&(D/1.5)+B\\
	\hline
\end{tabular}
}
\end{document}
\fi

	\item Prove that the products of the lengths of the perpendiculars drawn from the points $\myvec{\sqrt{a^2-b^2}\\0}$ and $\myvec{-\sqrt{a^2-b^2} \\0} $ to the line $\frac{x}{a} \cos{\theta} + \frac{y}{b}\sin{\theta} =1 $ is $ b^2 $.
\\
    \solution 
		\iffalse
\documentclass[10pt]{article}
       \usepackage[latin1]{inputenc}
       \usepackage{fullpage}
       \usepackage{color}
       \usepackage{array}
       \usepackage{longtable}
       \usepackage{calc}
       \usepackage{multirow}
       \usepackage{hhline}
       \usepackage{ifthen}
\usepackage{graphicx}
\def\inputGnumericTable{}
\usepackage[none]{hyphenat}
\usepackage{graphicx}
\usepackage{listings}
\usepackage[english]{babel}
\usepackage{graphicx}
\usepackage{caption} 
\usepackage{booktabs}
\usepackage{gensymb}
\usepackage{array}
\usepackage{amssymb} % for \because
\usepackage{amsmath}   % for having text in math mode
\usepackage{extarrows} % for Row operations arrows
\usepackage{listings}
\lstset{
  frame=single,
  breaklines=true
}
\usepackage{hyperref}
%Following 2 lines were added to remove the blank page at the beginning
\usepackage{atbegshi}% http://ctan.org/pkg/atbegshi
\AtBeginDocument{\AtBeginShipoutNext{\AtBeginShipoutDiscard}}
%New macro definitions
\newcommand{\mydet}[1]{\ensuremath{\begin{vmatrix}#1\end{vmatrix}}}
\providecommand{\brak}[1]{\ensuremath{\left(#1\right)}}
\providecommand{\norm}[1]{\left\lVert#1\right\rVert}
\newcommand{\solution}{\noindent \textbf{Solution: }}
\newcommand{\myvec}[1]{\ensuremath{\begin{pmatrix}#1\end{pmatrix}}}
\providecommand{\abs}[1]{\left\vert#1\right\vert}
\let\vec\mathbf
\begin{document}

\begin{center}
\title{\textbf{STRAIGHT LINES}}
\date{\vspace{-5ex}} %Not to print date automatically
\maketitle
\end{center}

\section{11$^{th}$ Maths - Chapter 10}
This is Problem 5 from Exercise-10.4
\begin{enumerate}

\solution
\fi
Let
\begin{align}
	\vec{A}=\myvec{\cos\theta\\\sin\theta},\vec{B}&=\myvec{\cos\phi\\\sin\phi}\\
\implies	\vec{m}=\vec{B}-\vec{A}&=\myvec{\cos\phi-\cos\theta\\\sin\phi-\sin\theta}
\end{align}
The normal vector is then given by,
\begin{align}
\vec{n}=\myvec{\sin\phi-\sin\theta\\\cos\theta-\cos\phi} \implies
\norm{\vec{n}}=2\sin\brak{\frac{\phi-\theta}{2}}
\end{align}
The equation of the line is
\begin{align}
\vec{n}^\top\brak{\vec{x}-\vec{A}}&=0\\
\implies\myvec{\sin\phi-\sin\theta&\cos\theta-\cos\phi}\vec{x}&=\sin\brak{\phi-\theta}
\label{eq:chapters/11/10/4/5/1}
\end{align}
Thus, 
\begin{align}
c=\sin\brak{\phi-\theta}
\end{align}
The perpendicular distance from the origin to the line is
\begin{align}
d&=\frac{\abs{c}}{\norm{\vec{n}}}\\
\implies d&=\frac{\sin\brak{\phi-\theta}}{2\sin\brak{\frac{\phi-\theta}{2}}} = \cos\brak{\frac{\phi-\theta}{2}}
\label{eq:chapters/11/10/4/5/2}
\end{align}

\item Find the equation of line  drawn perpendicular to the line $\frac{x}{4}+\frac{y}{6}=1$ through the point where it meets the y-axis \\
\solution
		\iffalse
\documentclass[journal,10pt,twocolumn]{article}
\usepackage{graphicx}
\usepackage[margin=0.5in]{geometry}
\usepackage[cmex10]{amsmath}
\usepackage{array}
\usepackage{booktabs}
\usepackage{listings}
\title{\textbf{Line Assignment}}
\author{Bhavani Kanike}
\date{October 2022}

\providecommand{\norm}[1]{\left\lVert#1\right\rVert}
\providecommand{\abs}[1]{\left\vert#1\right\vert}
\let\vec\mathbf
\newcommand{\myvec}[1]{\ensuremath{\begin{pmatrix}#1\end{pmatrix}}}
\newcommand{\mydet}[1]{\ensuremath{\begin{vmatrix}#1\end{vmatrix}}}
\providecommand{\brak}[1]{\ensuremath{\left(#1\right)}}

\begin{document}

\maketitle
\paragraph{\textit{Problem Statement} 
\fi
ABCD is a quadrilateral in which $\vec{P}, \vec{Q}, \vec{R}$ and $\vec{S}$ are mid-points of the sides AB, BC, CD and DA (see Fig \ref{fig:9/8/2/1}). AC is a diagonal. 
		
Show that 
\begin{enumerate}
	\item $SR \parallel AC$ and $SR =\frac{1}{2} AC$
\item $PQ = SR$
\item $PQRS$ is a parallelogram.
\end{enumerate}
 	\begin{figure}
		\centering
 \includegraphics[width=\columnwidth]{chapters/9/8/2/1/figs/line1.pdf}
		\caption{}
		\label{fig:9/8/2/1}
  	\end{figure}
	\solution 
	Using 
	  \eqref{eq:section_formula},
	\begin{align}
		\label{eq:9/8/2/1}
		\begin{split}
		\vec{P} &= \frac{\vec{A}+\vec{B}}{2}\\
 \vec{Q} &= \frac{\vec{C}+\vec{B}}{2}\\
 \vec{R} &= \frac{\vec{C}+\vec{D}}{2}\\
 \vec{S} &= \frac{\vec{D}+\vec{A}}{2}
		\end{split}
	\end{align}
\begin{enumerate}
	\item
	Consequently, 
	\begin{align}
\vec{R}
		-\vec{S} &= \frac{\vec{C}-\vec{A}}{2}
		\\
		\implies SR &\parallel AC
	\end{align}
	Also, 
	\begin{align}
		\norm{\vec{R}
		-\vec{S}} &= \frac{\norm{\vec{C}-\vec{A}}}{2}
		\\
		\implies SR &= \frac{1}{2}AC
	\end{align}
\item 	From 
		\eqref{eq:9/8/2/1},
	\begin{align}
\vec{R}
		-\vec{S} = \vec{Q}-\vec{P}
	\end{align}
	which means that $PQRS$ is a parallelogram and $PQ = SR$.
\end{enumerate}
%
\iffalse
\begin{figure}[h]
\centering
\includegraphics[width=1\columnwidth]
\caption{Figure}
\label{fig:triangle}
\end{figure}

\section*{Solution}

$\boldsymbol Given :$  ABCD is a Quadrilateral P,Q,R and S are the midpoints of line AB,BC,CD,DA.We can obtain the points P,Q,R and S from A,B,C and D and are given by\\\\
\boldmath
\unboldmath
(3) To prove that PQRS is a parallelogram we need to prove  PQ // SR
To prove SR $\parallel$ PQ\\
Direction vector of line SR  $\boldsymbol {(R-S) =  \frac{(C-A)}{2}}$\\\\
Direction vector of line PQ  $\boldsymbol {(Q-P)= \frac{(C-A)}{2}}$\\\\
\begin{equation}
	\boldsymbol {(R-S) = (Q-P) = \frac{(C-A)}{2}}\\
\end{equation}
Since the direction vectors of line SR and PQ are in same direction\\\\
$SR \parallel PQ$\\
Therefore,
$\boldsymbol{ PQRS }$ is a parallelogram\\\\

	
(1)  Directional vector of line SR  = $\boldsymbol {(R-S)}$ = $\frac{\boldsymbol{(C-A)}}{2} $\\
Directional vector of line AC  = $\boldsymbol {(C-A)}$\\

It is observed that the constant k is $\frac{1}{2}$

Therefore
\begin{equation}
	SR \parallel AC
\end{equation} 

and from equation 1 
\begin{equation}
	\boldsymbol {SR = \frac{1}{2}AC}    
\end{equation}\\


(2)   To prove PQ = SR\\ 
		From euqation 1\\\\
\begin{equation}
		\boldsymbol{ (Q-P) = (R-S) = \frac{(C-A)}{2}}
\end{equation}
	 



\section{Execution}
The below python code realizes the construction:
\begin{lstlisting}
https://github.com/bhavani360/FWC_assignments
\end{lstlisting}
	
\section*{Construction}
The dimensions of the Quadrilateral ABCD are taken as below\\
{
\setlength\extrarowheight{2pt}
\centering
	\begin{tabular}{|c|c|}
	\hline
	\textbf{symbol}&\textbf{value}\\
	\hline
	r&8\\
	\hline
	$\theta$&pi/2.5\\
	\hline
	d&7\\
	\hline
	A&(0,0)\\
	\hline
	B&(d,0)\\
	\hline
	D&(rcos$\theta$,rsin$\theta$)\\
	\hline
	C&(D/1.5)+B\\
	\hline
\end{tabular}
}
\end{document}
\fi

 \item  In each of the following cases, determine the direction cosines of the normal to
the plane and the distance from the origin.
\begin{enumerate}
	\item $z=2$ 
	\item $x + y + z = 1$
	\item $2x + 3y – z = 5$
	\item $5y + 8 = 0$
\end{enumerate}
    \solution
		\iffalse
\documentclass[12pt]{article}
\usepackage{graphicx}
\usepackage{amsmath}
\usepackage{mathtools}
\usepackage{gensymb}
\usepackage[utf8]{inputenc}
\usepackage{float}
\newcommand{\mydet}[1]{\ensuremath{\begin{vmatrix}#1\end{vmatrix}}}
\providecommand{\brak}[1]{\ensuremath{\left(#1\right)}}
\providecommand{\norm}[1]{\left\lVert#1\right\rVert}
\newcommand{\solution}{\noindent \textbf{Solution: }}
\newcommand{\myvec}[1]{\ensuremath{\begin{pmatrix}#1\end{pmatrix}}}
\let\vec\mathbf

\begin{document}
\begin{center}
\textbf\large{CLASS-12 \\ CHAPTER-11 \\ THREE DIMENSIONAL GEOMETRY}
\end{center}
\section*{Excercise 11.3}

\solution
\fi
\begin{enumerate}
\item From the given equation
	\begin{align}
		\vec{n}=\myvec{0\\0\\1},c=2
	\end{align}
	The distance from the origin is given by:
		\begin{align}
			d=\frac{|c|}{\norm{\vec{n}}}=\frac{2}{1}=2
		\end{align}

\item From the given equation
         \begin{align}
		\vec{n}=\myvec{1\\1\\1},c=1
			\end{align}
	 
	The distance from the origin is given by
		\begin{align}
			d=\frac{|c|}{\norm{\vec{n}}}=\frac{1}{\sqrt{3}}
		\end{align}

\item From the given equation
         \begin{align}
		\vec{n}=\myvec{2\\3\\-1},c=5
			\end{align}
	
	The distance from the origin is given by
		\begin{align}
			d=\frac{|c|}{\norm{\vec{n}}}=\frac{5}{\sqrt{14}}
		\end{align}
		
\item From the given equation
         \begin{align}
		\vec{n}=\myvec{0\\-5\\0},c=8
			\end{align}
	
	The distance from the origin is given by
		\begin{align}
			d=\frac{|c|}{\norm{\vec{n}}}=\frac{8}{5}
		\end{align}

\end{enumerate}
 


\item
Find the angle between the lines whose direction ratios are $a,b,c$ and $b-c,c-a,a-b$.

\textbf{Solution :}
    \begin{align}
    \vec{m _1} &= \myvec{a\\b\\c}\\
    \vec{m_2} &= \myvec{b-c\\c-a\\a-b}\\
    \cos{\theta}&= \frac{\vec{m_1}^{\top}\vec{m_2}}{\vec{\norm{m_1}\norm{m_2}}
   } \\
   &=\frac{\myvec{a&b&c}\myvec{b-c\\c-a\\a-b}}{\sqrt{a^2+b^2+c^2}\sqrt{\brak{b-c}^2+\brak{c-a}^2+\brak{a-b}^2}}\\
   &=0\\
   or,\theta&=\frac{\pi}{2}
    \end{align}

\end{enumerate}
