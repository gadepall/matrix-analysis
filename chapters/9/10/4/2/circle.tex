\iffalse
\documentclass[10pt]{article}
       \usepackage[latin1]{inputenc}
       \usepackage{fullpage}
       \usepackage{color}
       \usepackage{array}
       \usepackage{longtable}
       \usepackage{calc}
       \usepackage{multirow}
       \usepackage{hhline}
       \usepackage{ifthen}
\usepackage{graphicx}
\def\inputGnumericTable{}
\usepackage[none]{hyphenat}
\usepackage{graphicx}
\usepackage{listings}
\usepackage[english]{babel}
\usepackage{graphicx}
\usepackage{caption} 
\usepackage{booktabs}
\usepackage{gensymb}
\usepackage{array}
\usepackage{amssymb} % for \because
\usepackage{amsmath}   % for having text in math mode
\usepackage{extarrows} % for Row operations arrows
\usepackage{listings}
\lstset{
  frame=single,
  breaklines=true
}
\usepackage{hyperref}
%Following 2 lines were added to remove the blank page at the beginning
\usepackage{atbegshi}% http://ctan.org/pkg/atbegshi
\AtBeginDocument{\AtBeginShipoutNext{\AtBeginShipoutDiscard}}
%New macro definitions
\newcommand{\mydet}[1]{\ensuremath{\begin{vmatrix}#1\end{vmatrix}}}
\providecommand{\brak}[1]{\ensuremath{\left(#1\right)}}
\providecommand{\norm}[1]{\left\lVert#1\right\rVert}
\newcommand{\solution}{\noindent \textbf{Solution: }}
\newcommand{\myvec}[1]{\ensuremath{\begin{pmatrix}#1\end{pmatrix}}}
\providecommand{\abs}[1]{\left\vert#1\right\vert}
\let\vec\mathbf
\begin{document}
\begin{center}
\title{\textbf{Properties of Circle}}
\date{\vspace{-5ex}} %Not to print date automatically
\maketitle
\end{center}
\setcounter{page}{1}
\section{9$^{th}$ Maths - Chapter 10}
       This is Problem-2 from Exercise 10.4
\begin{enumerate}
\item If two equal chords of a circle intersect within the circle, prove that the segments of one chord are equal to corresponding segments of other chord.

\solution:
\fi
\begin{figure}[h!]
	\begin{center} 
	  \includegraphics[width=\columnwidth]{chapters/9/10/4/2/figs/c.png}
	\end{center}
\caption{Two equal chords intersecting in a circle}
\label{fig:chapters/9/10/4/2/Fig1}
\end{figure} 
See Table 
\ref{tab:chapters/9/10/4/2/}
for the input  parameters.
\begin{table}[h!]
	\begin{tabular}{|c|c|p{5cm}|}
\hline
\textbf{Symbol} & \textbf{Value} & \textbf{Description} \\
\hline
$\theta$ & $30\degree$ & $\angle{BAP} = \angle{BAQ}$ \\
\hline
$a$ & $9$ & $AB$ \\
\hline
$c$ & $8$ & $AQ$ \\
\hline
$\vec{e}_1$ & $\myvec{1\\0}$ & Basis vector \\
\hline
\end{tabular}

\caption{}
\label{tab:chapters/9/10/4/2/}
\end{table}
Consider
\begin{align}
\vec{P}=\myvec{\cos \theta_1\\\sin \theta_1},\,
\vec{Q}=\myvec{\cos \theta_2\\\sin \theta_2},\,
\vec{R}=\myvec{\cos \theta_3\\\sin \theta_3},\,
\vec{S}=\myvec{\cos \theta_4\\\sin \theta_4}
\label{eq:chapters/9/10/4/2/table1}
\end{align}
such that 
\begin{align}
	\vec{P}-\vec{Q}&=\myvec{\cos\theta_1-\cos\theta_2 \\ \sin \theta_1-\sin \theta_2}
	\\
	\implies \norm{\vec{P}-\vec{Q}}^2 &= 
	\brak{\cos \theta_1-\cos \theta_2}^2+\brak{\sin \theta_1-\sin \theta_2}^2=d^2\\
\end{align}
yielding
\begin{align}
	\brak{\frac{\theta_1-\theta_2}{2}}=\sin^{-1}\brak{\frac{d}{2}}.
	\end{align}
	Similarly, 
\begin{align}
\brak{\frac{\theta_3-\theta_4}{2}}=\sin^{-1}\brak{\frac{d}{2}}.
\end{align}
The equations of $PQ$ and $RS$ are obtained using 
\begin{align}
\vec{{n}_1^{\top}}\brak{\vec{x}-\vec{P}}&=0\\
\vec{{n}_2^{\top}}\brak{\vec{x}-\vec{R}}&=0
\end{align}
where 
\begin{align}
\vec{n}_1
&=\myvec{\sin \theta_1-\sin \theta_2\\\cos \theta_2-\cos \theta_1}\\
\vec{n}_2
&=\myvec{\sin \theta_3-\sin \theta_4\\\cos \theta_4-\cos \theta_3}
\end{align}
Substiuting numerical values, the 
point of intersection of lines $PQ,RS$ is 
\begin{align}
\vec{T}=\myvec{0.68341409\\-0.04288508}
\end{align}
Thus, 
\begin{align}
\norm{\vec{P}-\vec{T}}=
\norm{\vec{S}-\vec{T}}&=0.5727
\end{align}
