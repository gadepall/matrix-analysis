\iffalse
\documentclass[12pt]{article}
\usepackage{graphicx}
\usepackage{amsmath}
\usepackage{mathtools}
\usepackage{gensymb}

\newcommand{\mydet}[1]{\ensuremath{\begin{vmatrix}#1\end{vmatrix}}}
\providecommand{\brak}[1]{\ensuremath{\left(#1\right)}}
\providecommand{\norm}[1]{\left\lVert#1\right\rVert}
\newcommand{\solution}{\noindent \textbf{Solution: }}
\newcommand{\myvec}[1]{\ensuremath{\begin{pmatrix}#1\end{pmatrix}}}
\let\vec\mathbf
\def\inputGnumericTable{}
\usepackage{color}                                            %%
    \usepackage{array}                                            %%
    \usepackage{longtable}                                        %%
    \usepackage{calc}                                             %%
    \usepackage{multirow}                                         %%
    \usepackage{hhline}                                           %%
    \usepackage{ifthen}
\usepackage{array}
\usepackage{amsmath}   % for having text in math mode
\usepackage{listings}
\lstset{
language=tex,
frame=single, 
breaklines=true
}
\begin{document}
\begin{center}
\textbf\large{CLASS-9\\CHAPTER-10 \\ CIRCLES}

\end{center}
\section*{Excercise 10.5}

Q1. section*{\large Solution}
\fi
\begin{figure}[h!]
\centering
\includegraphics[width=\columnwidth]{chapters/9/10/5/1/figs/circle1.pdf}
\caption{}
\label{fig:chapters/9/10/5/1/Fig1}
\end{figure}
%
\begin{table}[h!]
	\centering
	%\subimport{../chapters/9/10/5/1/tables/}{table1.tex}
     \begin{tabular}{|c|c|c|}
  \hline
  \textbf{Symbol}&\textbf{Value}&\textbf{Description}\\
  \hline
  $a$ & 8 & $BC$\\
  \hline
	$\angle{B}$ & 45$\degree{}$ & $\angle{B}$ in $\triangle$$ABC$ \\
  \hline
	$k$ & 3.5 & $AB-AC$ i.e $c-b$ \\
  \hline 
	$\vec{e_2}$ & $\myvec{
			0\\
			1\\
			}$ & Basis vector\\
 \hline			
\end{tabular}

%	\caption{}
	\label{table:chapters/9/10/5/1/table1}
	\end{table}
The input parameters are available in Table
	\ref{table:chapters/9/10/5/1/table1} yielding
\begin{align}
	\vec{C} =\vec{e}_1= \myvec{1\\0},\,
	\vec{A} = \myvec{\cos(\alpha+\beta)\\\sin(\alpha+\beta)},\,
	\vec{D} = \myvec{\cos\gamma\\\sin\gamma}.
\end{align}
Since
\begin{align}
	 \vec{A-D}& = \myvec{\cos(\alpha+\beta) - \cos\gamma\\\sin(\alpha+\beta) - \sin\gamma},
	 \vec{C-D} &= \myvec{1 - \cos\gamma\\-\sin\gamma},
	 \norm{\vec{A-D}}\norm{\vec{C-D}}& = 4 \sin\frac{\alpha+\gamma}2\sin\frac{\beta+\gamma}2,
	 \\
	\cos(\angle ADC) &= \frac{\vec{(A-D)^\top(C-D)}}{\norm{\vec{A-D}}\norm{\vec{C-D}}},
	\label{eq:2}
	\\
	&= 4\sin\frac{\alpha+\gamma}2\sin\frac{\beta+\gamma}2\cos\frac{\alpha+\beta}2
 = \cos\frac{\alpha+\beta}{2}
	\label{eq:7}
\end{align}
Substituting $\alpha$ and $\beta$ in \eqref{eq:7}
\begin{align}
\angle ADC = \frac{\alpha+\beta}{2}=\frac{(30\degree + 60\degree )}{2}=45\degree
\end{align}
See Fig. 
\ref{fig:chapters/9/10/5/1/Fig1}.


