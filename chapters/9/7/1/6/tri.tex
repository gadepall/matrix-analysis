\iffalse
\documentclass[10pt]{article}
\usepackage{graphicx}
\def\inputGnumericTable{}
\usepackage[latin1]{inputenc}
\usepackage{fullpage}
\usepackage{color}
\usepackage{array}
\usepackage{longtable}
\usepackage{calc}
\usepackage{multirow}
\usepackage{hhline}
\usepackage{ifthen}
\usepackage{amsmath}
\usepackage[none]{hyphenat}
\usepackage{listings}
\usepackage[english]{babel}
\usepackage{siunitx}
\usepackage{caption}
\usepackage{booktabs}
\usepackage{array}
\usepackage{extarrows}
\usepackage{enumerate}
\usepackage{enumitem}
\usepackage{amsmath}
\usepackage{commath}
\usepackage{gensymb}
\usepackage{amssymb}
\usepackage{multicol}
%\usepackage[utf8]{inputenc}
\lstset{
 frame=single,
 breaklines=true
}
\usepackage{hyperref}
\usepackage[margin=0.5in]{geometry}	 
%\usepackage{exsheets}% also loads the `tasks' package
\usepackage{atbegshi}
\AtBeginDocument{\AtBeginShipoutNext{\AtBeginShipoutDiscard}}

%new macro definitions
\renewcommand{\labelenumi}{(\roman{enumi})}
\newcommand{\mydet}[1]{\ensuremath{\begin{vmatrix}#1\end{vmatrix}}}
\providecommand{\brak}[1]{\ensuremath{\left(#1\right)}}
\newcommand{\solution}{\noindent \textbf{Solution: }}
\newcommand{\myvec}[1]{\ensuremath{\begin{pmatrix}#1\end{pmatrix}}}
\newenvironment{amatrix}[1]{%
	\left(\begin{array}{@{}*{#1}{c}|c@{}}
}{%
	\end{array}\right)
}

\newcommand{\myaugvec}[2]{\ensuremath{\begin{amatrix}{#1}#2\end{amatrix}}}
\providecommand{\norm}[1]{\left\1Vert#1\right\rVert}
\let\vec\mathbf{}


%\SetEnumitemKey{twocol}{
% before=\raggedcolumns\begin{multicols}{2},
% after=\end{multicols}}
%\SetEnumitemKey{fourcol}{
% before=\raggedcolumns\begin{multicols}{4},
% after=\end{multicols}} 


\begin{document}
\begin{center}
\title{\textbf{TRIANGLES}}
\date{\vspace{-5ex}}
\maketitle
\end{center}
\section*{9$^{th}$Math - Chapter 7}
\section*{Problem}
\section*{Construction}
\solution
\textbf{Given:}
\fi
The input parameters are available in Table 
\ref{tab:chapters/9/7/1/6/1}.
\begin{table}[!h]
\centering
\begin{tabular}{|c|c|p{5cm}|}
\hline
\textbf{Symbol} & \textbf{Value} & \textbf{Description} \\
\hline
$\theta$ & $30\degree$ & $\angle{BAP} = \angle{BAQ}$ \\
\hline
$a$ & $9$ & $AB$ \\
\hline
$c$ & $8$ & $AQ$ \\
\hline
$\vec{e}_1$ & $\myvec{1\\0}$ & Basis vector \\
\hline
\end{tabular}

\label{tab:chapters/9/7/1/6/1}
\end{table}
The vertices of $\triangle ABC$ are given by 
\begin{align}
	\vec{A}=c\myvec{\cos\frac{\theta}{2}\\\sin\frac{\theta}{2}},\,
\vec{B}=\myvec{0\\0},\,
\vec{C}=a\vec{e}_1,
\end{align}
and 
\begin{align}
\vec{D}=\brak{2c\sin\frac{\theta}{2}}\vec{e_1}
\end{align}
Then, using the cosine formula and  the fact that $\triangle AEC$ is iscosceles,
\begin{align}
AC = b=\sqrt{a^2+c^2-2ac\cos\theta},\, EC = 2b\sin\frac{\theta}{2}
\end{align}
Also, 
\begin{align}
\angle BCA=\cos^{-1}\brak{\frac{a^2+b^2-c^2}{2ab}},\,
\angle ACE=90\degree-\frac{\theta}{2}
\end{align}
Let $\phi$ be the angle made by the vector 	$EC$ with  the $x$-axis.  Then,  
\begin{align}
\phi=180\degree-\brak{\angle BCA+\angle ACE} = =90\degree-\frac{\theta}{2}\cos^{-1}\brak{\frac{a^2+b^2-c^2}{2ab}}
\end{align}
Consequently, 
Using vector addition,
\begin{align}
	\vec{E}-\vec{C}&=\brak{2b\sin\frac{\theta}{2}}\myvec{\cos\phi\\\sin\phi}\\
	\implies \vec{E}&=\vec{C}+2b\sin\frac{\theta}{2}\myvec{\cos\phi\\\sin\phi}\\
\end{align}
Substituting numerical values, 
\begin{align}
	\norm{\vec{B}-\vec{C}}
	=\norm{\vec{D}-\vec{E}}
\label{eq:chapters/9/7/1/6/5}
\end{align}
