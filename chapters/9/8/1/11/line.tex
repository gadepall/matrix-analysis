
\documentclass[10pt, a4paper]{article}
\usepackage[a4paper,outer=1.5cm,inner=1.5cm,top=1.75cm,bottom=1.5cm]{geometry}

\twocolumn
\usepackage{graphicx}
\usepackage{karnaugh-map}
\usepackage{tabularx}
\usepackage{hyperref}
\usepackage[utf8]{inputenc}
\usepackage{amsmath}
\usepackage{physics}
\usepackage{amssymb}

\begin{document}
\title{Assignment-4}
\author{Name:A.Gowri Priya\and Email :  \url{gowripriyaappayyagari@gmail.com}}
%\{ Wireless Communication (FWC)}
\date{}
\maketitle


  \section{Problem}
In  $\Delta$  ABC and  $\Delta$ DEF, AB = DE, AB $\parallel$ DE, BC = EF
and BC $\parallel$ EF. Vertices A, B and C are joined to
vertices D, E and F respectively (see Figure).\\
Show that\\
(i) quadrilateral ABED is a parallelogram\\
(ii) quadrilateral BEFC is a parallelogram\\
(iii) AD $\parallel$ CF and AD = CF\\
(iv) quadrilateral ACFD is a parallelogram\\
(v) AC = DF\\
(vi)$\Delta ABC \cong \Delta$  DEF.\\
\begin{figure}[h]
\centering
\includegraphics[scale=0.5]{fig.png} 
\caption{Given Figure}
\end{figure}

\section{Solution}
\begin{center}
The input parameters for this construction are
\begin{tabular}{|c|c|}
	\hline
	\textbf{Symbol}&\textbf{Value}\\
	\hline
	r1&2\\
	\hline
	r2&3\\
	\hline
	$\theta$&$\frac{{3}\pi}{10}$\\
	\hline
\end{tabular}
\boldmath
$$\vec{A}=\begin{pmatrix} r1\cos\theta\\ r2\sin\theta\ \end{pmatrix}$$
$$\vec{B}=\begin{pmatrix} 0\\ 0\ \end{pmatrix}$$
$$\vec{D}=\begin{pmatrix} 4\\ 0\ \end{pmatrix}$$
$$\vec{C}={\vec{B}+\vec{D}}/2$$
$$\vec{E}={\vec{B}+\vec{D}-\vec{A}}$$
$$\vec{F}={\vec{E}+\vec{C}-\vec{B}}$$
\unboldmath
\end{center}
\textbf{Direction vectors}

The Direction vectors are
\boldmath
$$\vec{m_1}={\vec{A}-\vec{B}} $$
$$\vec{m_2}={\vec{B}-\vec{C}} $$
$$\vec{m_3}={\vec{C}-\vec{A}} $$
$$\vec{n_1}={\vec{D}-\vec{E}} $$
$$\vec{n_2}={\vec{E}-\vec{F}} $$
$$\vec{n_3}={\vec{F}-\vec{D}} $$
$$\vec{o_1}={\vec{A}-\vec{D}} $$
$$\vec{o_2}={\vec{C}-\vec{F}} $$
\unboldmath
\textbf{To proove\\ i.Quadrilateral ABED is a parallelogram}\\
    Distance between A and B is $\norm{\vec{A-B}}$\\
	Distance between D and E is $\norm{\vec{D-E}}$\\
	if $\norm{\vec{A-B}}$ =  $\norm{\vec{D-E}}$\\
	then AB = DE..........(1)\\
	if $\vec{m_1} \times \vec{n_1}=0$\\
	then AB $\parallel$ DE...........(2)\\
	Because,Two vectors ara parallel when cross product of that two vectors is zero.\\
	From (1) and (2) we can say that ABED is a parallelogram.Because,If one pair of opposite sides of a quadrilateral are equal and parallel to each other,then it is a parallelogram.\\
	$\therefore$ Quadrilateral ABED is a parallelogram.\\ 
\textbf{ii.Quadrilateral BEFC is a parallelogram}\\
    Distance between B and C is $\norm{\vec{B-C}}$\\
	Distance between E and F is $\norm{\vec{E-F}}$\\
	if $\norm{\vec{B-C}}$ =  $\norm{\vec{E-F}}$\\
	then BC = EF..........(3)\\
	if $\vec{m_2} \times \vec{n_2}=0$\\
	then BC $\parallel$ EF...........(4)\\
	Because,Two vectors ara parallel when cross product of that two vectors is zero.\\
	From (3) and (4) we can say that BEFC is a parallelogram.Because,If one pair of opposite sides of a quadrilateral are equal and parallel to each other,then it is a parallelogram.\\
	$\therefore$ Quadrilateral BEFC is a parallelogram.\\ 	
\textbf{iii.AD$\parallel$CF and AD=CF}\\
    Distance between A and D is $\norm{\vec{A-D}}$\\
	Distance between C and F is $\norm{\vec{C-F}}$\\
	if $\norm{\vec{A-D}}$ =  $\norm{\vec{C-F}}$\\
	then AD = CF..........(5)\\
	if $\vec{O_1} \times \vec{O_2}=0$\\
	then AD $\parallel$ CF...........(6)\\
	Because,Two vectors ara parallel when cross product of that two vectors is zero.\\
	From (5) and (6) AD$\parallel$CF and AD=CF \\
\textbf{iv.Quadrilateral ACFD is a parallelogram}\\
   From (iii) we can say that AD$\parallel$CF and AD=CF 
	SO, we can say that ACFD is a parallelogram.Because,If one pair of opposite sides of a quadrilateral are equal and parallel to each other,then it is a parallelogram.\\
	$\therefore$ Quadrilateral ACFD is a parallelogram.\\
\textbf{v.AC=DF}\\
    Distance between A and C is $\norm{\vec{A-C}}$\\
	Distance between D and F is $\norm{\vec{D-F}}$\\
	if $\norm{\vec{A-C}}$ =  $\norm{\vec{D-F}}$\\
	then AC = DF\\
	$\therefore$ AC=DF\\
\textbf{vi.$\Delta ABC \cong \Delta DEF$}  \\
If $\norm{\vec{A-B}}$ =  $\norm{\vec{D-E}}$ and $\norm{\vec{B-C}}$ =  $\norm{\vec{E-F}}$ and $\norm{\vec{A-C}}$ =  $\norm{\vec{D-F}}$\\
Then,$\Delta ABC \cong \Delta DEF$ .Because, If three sides of one triangle are equal to three sides of another triangle, the triangles are congruent.(By SSS Rule)\\
$\therefore$ $\Delta ABC \cong \Delta DEF$

\section{Execution}
*Verify the above proofs in the following code.\\
\framebox{
\url{https://github.com/gowripriya-2002/FWC/blob/main/line_assignment/line.py}}	
\bibliographystyle{ieeetr}
\end{document}
