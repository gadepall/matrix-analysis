\iffalse
\documentclass[journal,12pt,twocolumn]{IEEEtran}
\usepackage{graphicx}
\graphicspath{{./figs/}}{}
\usepackage{amsmath,amssymb,amsfonts,amsthm}
\newcommand{\myvec}[1]{\ensuremath{\begin{pmatrix}#1\end{pmatrix}}}
\providecommand{\norm}[1]{\lVert#1\rVert}
\usepackage{listings}
\usepackage{watermark}
\usepackage{titlesec}
\usepackage{caption}
\usepackage{enumitem}
\usepackage{extarrows}
\let\vec\mathbf
\lstset{
frame=single, 
breaklines=true,
columns=fullflexible
}
\thiswatermark{\centering \put(0,-105.0){\includegraphics[scale=0.15]{/sdcard/IITH/vector/vector-3/figs/logo.png}} }
\title{\mytitle}
\title{
Assignment - Vector-3
}
\author{Surajit Sarkar}
\begin{document}
\maketitle
\tableofcontents
\bigskip
\section{\textbf{Problem}}
In each of the following, find the value of ’k’,\\ for which the points are collinear.
\begin{enumerate}[label=(\roman*)]
\item (7, –2), (5, 1), (3, k)
\item(8, 1), (k, –4), (2, –5)
\end{enumerate}
\section{\textbf{Solution}}
\fi
\begin{enumerate}
    \item Given
    \begin{align}
      \vec{A}=\myvec{7\\-2},\Vec{B}=\myvec{5\\1},\vec{C}=\myvec{3\\k}  
    \end{align}
    Then
    \begin{align}
        \myvec{\vec{A}-\vec{B}}&=\myvec{2\\-3}\\
        \myvec{\vec{A}-\vec{C}}&=\myvec{4\\2k}\
    \end{align}
    Forming the collinearity matrix
    \begin{align}
        \myvec{2&-3\\4&2k} \xleftrightarrow{R_1\rightarrow{R_1-1}}&\myvec{1&-2\\4&2k}\\
         \xleftrightarrow{R_1\rightarrow{R_1-1}}&\myvec{1&-2\\0&2k+8}\\
        k&=4
        \end{align}
    
    \item Given
     \begin{align}
      \vec{A}=\myvec{8\\1},\vec{B}=\myvec{k\\-4},\vec{C}=\myvec{2\\-5}  
    \end{align}
    Then
    \begin{align}
        \myvec{\vec{A}-\vec{B}}&=\myvec{-8k\\-5}\\
        \myvec{\vec{A}-\vec{C}}&=\myvec{6\\6}\
    \end{align}
    Forming the collinearity matrix
    \begin{align}
        \myvec{-8-k&-5\\6&6} \xleftrightarrow{R_1\rightarrow{R_1+8}}&\myvec{-k&3\\6&6}\\
        k&=3
        \end{align}
\end{enumerate}

