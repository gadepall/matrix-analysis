\iffalse
\documentclass[12pt]{article}
\usepackage{graphicx}
\usepackage{amsmath}

\begin{document}
\begin{center}
\textbf\large{CHAPTER-7 \\ COORDINATE GEOMETRY}

\end{center}
\section*{Excercise 7.4}
\fi
%\begin{enumerate}[label=\thechapter.\arabic*,ref=\thechapter.\theenumi]
\begin{enumerate}[label=\thesection.\arabic*,ref=\thesection.\theenumi]
\numberwithin{equation}{enumi}
\numberwithin{figure}{enumi}
\numberwithin{table}{enumi}
\item Determine the ratio in which the line $2x+y  - 4=0$ divides the line segment joining the points $\vec{A}(2, - 2)$  and  $\vec{B}(3, 7)$.
\item Find a relation between $x$ and $y$ if the points $(x, y), (1, 2)$  and  $(7, 0)$ are collinear.

\item Find the centre of a circle passing through the points $(6, – 6), (3, – 7)$ and $ (3, 3)$.

\item The two opposite vertices of a square are $(–1, 2)$  and $ (3, 2)$. Find the coordinates of the other two vertices.
\\
	\iffalse
\documentclass[12pt]{article}
\usepackage{graphicx}
\usepackage{amsmath}
\usepackage{mathtools}
\usepackage{gensymb}

\newcommand{\mydet}[1]{\ensuremath{\begin{vmatrix}#1\end{vmatrix}}}
\providecommand{\brak}[1]{\ensuremath{\left(#1\right)}}
\providecommand{\norm}[1]{\left\lVert#1\right\rVert}
\newcommand{\solution}{\noindent \textbf{Solution: }}
\newcommand{\myvec}[1]{\ensuremath{\begin{pmatrix}#1\end{pmatrix}}}
\let\vec\mathbf

\begin{document}
\begin{center}
\textbf\large{CHAPTER-7 \\ COORDINATE GEOMETRY}

\end{center}
\section*{Excercise 7.4}

Q4.The two opposite vertices of a square are $(–1, 2) \text{ and } (3, 2)$. Find the coordinates of the other two vertices.\\
\fi
\solution
Let
\begin{align}
\vec{A} = \myvec
{
-1 \\
 2\\
},
\vec{C} = 
\myvec
{
3\\
2\\
}
\end{align}

\begin{figure}[!h]
	\begin{center} 
	    \includegraphics[width=\columnwidth]{chapters/10/7/4/4/figs/square}
	\end{center}
\caption{}
\label{fig:7/4/4/4Fig1}
\end{figure}

Shifting $\vec{A}$ to origin with reference to Fig. \ref{fig:7/4/4/4Fig2},
\begin{align}
\vec{A^{\prime}} =
\myvec{
0 \\
0\\
},
\vec{C^{\prime}} = \vec{C}-\vec{A} = 
\myvec{
4 \\
0\\
}
\end{align}

\begin{figure}[!h]
	\begin{center} 
	    \includegraphics[width=\columnwidth]{chapters/10/7/4/4/figs/square1}
	\end{center}
\caption{}
\label{fig:7/4/4/4Fig2}
\end{figure}
\iffalse
In general,
the angle made by $AC$ with the x-axis is 
		\begin{align}
\beta = \theta + 45\degree
		\end{align}
\fi
Since
\begin{align}
\vec{C} - \vec{A} = \myvec{
4\\
0
} \equiv 
\myvec{
1\\
0
},
	\tan\theta&= \frac{0}{4} \implies 
\theta= 0\degree
\end{align}
		where
$\theta$ is the angle made by $AC$ with the x-axis.
Considering the rotation matrix 
\begin{align}
\vec{P} =
\myvec{
\cos\brak{\frac{\pi}{4}-\theta} & -\sin\brak{\frac{\pi}{4}-\theta} \\
\sin\brak{\frac{\pi}{4}-\theta} & \cos\brak{\frac{\pi}{4}-\theta} 
}
\end{align}
\iffalse
from Fig. \ref{fig:7/4/4/4Fig3},
\begin{align}
\vec{C^{\prime \prime}} = \vec{P}^\top (\vec{C}-\vec{A}) =
\myvec{
\frac{1}{\sqrt{2}} & -\frac{1}{\sqrt{2}} \\
\frac{1}{\sqrt{2}} & \frac{1}{\sqrt{2}}\\
}
\myvec{
4 \\
0\\
} = 
\myvec{
\frac{4}{\sqrt{2}} \\
\frac{4}{\sqrt{2}}\\
}
\end{align}
\begin{align}
\vec{B^{\prime \prime}} = \myvec{
 1&0\\
 0&0\\
}\vec{C^{\prime \prime}}=
\myvec{
 \frac{4}{\sqrt{2}}\\
 0\\
},
\vec{D^{\prime \prime}} = \myvec{
 0&0\\
 0&1\\
}\vec{C^{\prime \prime}}=
\myvec{
 0\\
 \frac{4}{\sqrt{2}}\\
} \text{ and }
\vec{A^{\prime \prime}} =
\myvec{
0 \\
0\\
}
\end{align}
\fi
\begin{figure}[!h]
	\begin{center} 
	    \includegraphics[width=\columnwidth]{chapters/10/7/4/4/figs/square2}
	\end{center}
\caption{}
\label{fig:7/4/4/4Fig3}
\end{figure}

\newpage
\iffalse
Again tranforming(rotating) the coordinates back to the original axis.

We know for anti-clockwise direction the rotation matrix is given as
\begin{align}
\vec{P} =
\myvec{
\cos\theta & -\sin\theta \\
\sin\theta & \cos\theta \\
}
\end{align}

Again we know that the angle is negative so the rotation will be in clockwise direction. So now the transformed(rotated) coordinates $\vec{B} \text{ and } \vec{D}$ are with refrence to 
\fi
from Figure 
%\ref{fig:7/4/4/4Fig4},
\ref{fig:7/4/4/4Fig3},
\begin{align}
	\vec{C^{\prime \prime}} &= \vec{P} (\vec{C}-\vec{A}) 
	\\
\label{eq:7/4/4/4bp}
	\vec{B^{\prime \prime}} &= \myvec{\vec{e}_1 & \vec{0}}\vec{C^{\prime \prime}}
	\\
\label{eq:7/4/4/4dp}
	\vec{D^{\prime \prime}} &= \myvec{ \vec{0} & \vec{e}_2}\vec{C^{\prime \prime}}
\end{align}
Now, 
\begin{align}
\label{eq:7/4/4/4b}
	\vec{B} = \vec{P}^{\top}\vec{B}^{\prime \prime}+\vec{A}
	\\
\label{eq:7/4/4/4d}
	\vec{D} = \vec{P}^{\top}\vec{D}^{\prime \prime}+\vec{A}
\end{align}
by reversing the process of translation and rotation.  Thus, 
from
\eqref{eq:7/4/4/4b}
\eqref{eq:7/4/4/4bp},
\eqref{eq:7/4/4/4d}
and
\eqref{eq:7/4/4/4dp}
\begin{align}
	\vec{B} = \vec{P}^{\top}\myvec{\vec{e}_1 & \vec{0}}\vec{P} (\vec{C}-\vec{A}) +\vec{A}
	\\
	\vec{D} = \vec{P}^{\top}\myvec{\vec{0} & \vec{e}_2  }\vec{P} (\vec{C}-\vec{A}) +\vec{A}
%	\vec{B} &= \brak{(\vec{C}-\vec{A})^{\top}\vec{P}^{\top} \vec{e}_1}\vec{P}^{\top}\vec{e}_1+\vec{A}
%	\\
%	\vec{D} &= \brak{(\vec{C}-\vec{A})^{\top}\vec{P}^{\top} \vec{e}_2}\vec{P}^{\top}\vec{e}_2+\vec{A}
\end{align}
yielding
		\begin{align}
\vec{B}=
\myvec{
1\\
0
},
\vec{D}
\myvec{
1\\
4
}.
		\end{align}
\iffalse
\begin{align}
\vec{B^{\prime}} &= \vec{P}\vec{B^{\prime \prime}} = \myvec{
\frac{1}{\sqrt{2}} & \frac{1}{\sqrt{2}} \\
-\frac{1}{\sqrt{2}} & \frac{1}{\sqrt{2}}\\
}
\myvec{
 \frac{4}{\sqrt{2}}\\
 0\\
} = 
\myvec{
2 \\
-2\\
}\\
\vec{D^{\prime}} &= \vec{P}\vec{D^{\prime \prime}} = \myvec{
\frac{1}{\sqrt{2}} & \frac{1}{\sqrt{2}} \\
-\frac{1}{\sqrt{2}} & \frac{1}{\sqrt{2}}\\
}
\myvec{
 0\\
 \frac{4}{\sqrt{2}}\\
} = 
\myvec{
2 \\
2 \\
}
\end{align}

\begin{figure}[!h]
	\begin{center} 
	    \includegraphics[width=\columnwidth]{chapters/10/7/4/4/figs/square3}
	\end{center}
\caption{}
\label{fig:7/4/4/4Fig4}
\end{figure}

Again transforming(shifting) the axis back to the original with refrence to Figure \ref{fig:7/4/4/4Fig5}
\begin{align}
\vec{B} &= \vec{B^{\prime}}+\vec{A} = \myvec{
2 \\
-2\\
}+\myvec{
-1 \\
2\\
} = 
\myvec{
1 \\
0\\
}\\
\vec{D} &= \vec{D^{\prime}}+\vec{A} = \myvec{
2 \\
2\\
}+\myvec{
-1 \\
2\\
} = 
\myvec{
1 \\
4 \\
}
\end{align}

Hence, the other two vertices are $\vec{B}(1,0) \text{ and } \vec{D}(1,4)$   

\begin{figure}[!h]
	\begin{center} 
	    \includegraphics[width=\columnwidth]{chapters/10/7/4/4/figs/square4}
	\end{center}
\caption{}
\label{fig:7/4/4/4Fig5}
\end{figure}
which can also be expressed as
\begin{align}
\vec{B} &= \vec{A} + \vec{P}\myvec{
\vec{e_{1}}&\vec{0}\\
}
\vec{P}^\top \brak{\vec{C}-\vec{A}}\\
\vec{D} &= \vec{A} + \vec{P}\myvec{
\vec{0}&\vec{e_{2}}\\
}
\vec{P}^\top \brak{\vec{C}-\vec{A}}\\
\end{align}
where $\vec{P}$ is the rotation matrix and $\vec{A} \text{ and } \vec{C}$ are the position vectors of opposite vertices.

Derivation of the above formulas:

We know that after shifting the axis and rotating by the required angle any arbitrary square will be aligned with the x and y axis so that we can directly get the vectors $\vec{B} \text{ and } \vec{D}$ as follows
\begin{align}
\vec{C^{\prime\prime}} &= \vec{P}^\top \brak{\vec{C} - \vec{A}}\\
\vec{B^{\prime\prime}} &= \myvec{
\vec{e_{1}} & \vec{0}
}\vec{C^{\prime\prime}} = \myvec{
\vec{e_{1}} & \vec{0}
}\vec{P}^\top \brak{\vec{C} - \vec{A}}\\
\vec{B^{\prime}} &= \vec{P} \vec{B^{\prime\prime}}  = \vec{P}\myvec{
\vec{e_{1}} & \vec{0}
}\vec{P}^\top \brak{\vec{C} - \vec{A}}\\
\vec{B} &= \vec{A}+\vec{B^{\prime}}\\
 &= \vec{A} + \vec{P}\myvec{
\vec{e_{1}}&\vec{0}\\
}
\vec{P}^\top\brak{\vec{C}-\vec{A}}
\end{align}

Similarly for D it can be derived as
\begin{align}
\vec{C^{\prime\prime}} &= \vec{P}^\top \brak{\vec{C} - \vec{A}}\\
\vec{D^{\prime\prime}} &= \myvec{
\vec{0} & \vec{e_{2}}
}\vec{C^{\prime\prime}} = \myvec{
\vec{0} & \vec{e_{2}}
}\vec{P}^\top \brak{\vec{C} - \vec{A}}\\
\vec{D^{\prime}} &= \vec{P} \vec{D^{\prime\prime}} = \vec{P} \myvec{
\vec{0} & \vec{e_{2}}
}\vec{P}^\top \brak{\vec{C} - \vec{A}}\\
\vec{D} &= \vec{A}+\vec{D^{\prime}}\\
 &= \vec{A} + \vec{P}\myvec{
\vec{0}&\vec{e_{2}}\\
}
\vec{P}^\top\brak{\vec{C}-\vec{A}}
\end{align}


Verification of the above formula for the given question

\begin{align}
\vec{B} &= \myvec{
-1\\
2\\
}+\myvec{
\frac{1}{\sqrt{2}} & \frac{1}{\sqrt{2}} \\
-\frac{1}{\sqrt{2}} & \frac{1}{\sqrt{2}}\\
}\myvec{
 1&0\\
 0&0\\
}\myvec{
\frac{1}{\sqrt{2}} & -\frac{1}{\sqrt{2}} \\
\frac{1}{\sqrt{2}} & \frac{1}{\sqrt{2}}\\
}\myvec{
4\\
0\\
}\\
 &= \myvec{
-1\\
2\\
}+\myvec{
\frac{1}{\sqrt{2}} & \frac{1}{\sqrt{2}} \\
-\frac{1}{\sqrt{2}} & \frac{1}{\sqrt{2}}\\
}\myvec{
 1&0\\
 0&0\\
}\myvec{
\frac{4}{\sqrt{2}}\\
\frac{4}{\sqrt{2}}\\
}\\
 &= \myvec{
-1\\
2\\
}+\myvec{
\frac{1}{\sqrt{2}} & \frac{1}{\sqrt{2}} \\
-\frac{1}{\sqrt{2}} & \frac{1}{\sqrt{2}}\\
}\myvec{
\frac{4}{\sqrt{2}}\\
0\\
}\\
 &= \myvec{
-1\\
2\\
}+\myvec{
2\\
-2\\
}\\
 &= \myvec{
1\\
0\\
}\\
\vec{D} &= \myvec{
-1\\
2\\
}+\myvec{
\frac{1}{\sqrt{2}} & \frac{1}{\sqrt{2}} \\
-\frac{1}{\sqrt{2}} & \frac{1}{\sqrt{2}}\\
}\myvec{
 0&0\\
 0&1\\
}\myvec{
\frac{1}{\sqrt{2}} & -\frac{1}{\sqrt{2}} \\
\frac{1}{\sqrt{2}} & \frac{1}{\sqrt{2}}\\
}\myvec{
4\\
0\\
}\\
 &= \myvec{
-1\\
2\\
}+\myvec{
\frac{1}{\sqrt{2}} & \frac{1}{\sqrt{2}} \\
-\frac{1}{\sqrt{2}} & \frac{1}{\sqrt{2}}\\
}\myvec{
 0&0\\
 0&1\\
}\myvec{
\frac{4}{\sqrt{2}}\\
\frac{4}{\sqrt{2}}\\
}\\
 &= \myvec{
-1\\
2\\
}+\myvec{
\frac{1}{\sqrt{2}} & \frac{1}{\sqrt{2}} \\
-\frac{1}{\sqrt{2}} & \frac{1}{\sqrt{2}}\\
}\myvec{
0\\
\frac{4}{\sqrt{2}}\\
}\\
 &= \myvec{
-1\\
2\\
}+\myvec{
2\\
2\\
}\\
 &= \myvec{
1\\
4\\
}
\end{align}
\fi








\iffalse
\item The Class X students of a secondary school in Krishinagar have been allotted a rectangular plot of land for their gardening activity. Sapling of Gulmohar are planted on the boundary at a distance of 1m from each other. There is a triangular grassy lawn in the plot as shown in fig.\ref{fig:Fig1}. The students are to sow seeds of flowering plants on the remaining area of the plot.\\

\begin{figure}[!h]
	\begin{center} 
	    \includegraphics[width=\columnwidth]{./ss}
	\end{center}
\caption{}
\label{fig:Fig1}
\end{figure}
\item Taking $\vec{A}$ as origin, find the coordinates of the vertices of the triangle.
\item What will be the coordinates of the vertices of $\triangle PQR$ if $\vec{C}$ is the origin?
Also calculate the areas of the triangles in these cases. What do you observe?
\end{enumerate}

\fi

\item The vertices of a $\triangle ABC$ are $\vec{A}(4,6), \vec{B}(1,5)$ and  $\vec{C}(7,2)$. A line is drawn to intersect sides $AB$ and $AC$ at $\vec{D}$ and $\vec{E}$ respectively, such that $\frac{AD}{AB} = \frac{AE}{AC} = \frac{1}{4}$. Calculate the area of $\triangle ADE$ and compare it with the area of the $\triangle ABC$.
\\
\solution
	\begin{enumerate}[label=\thesection.\arabic*,ref=\thesection.\theenumi]
\numberwithin{equation}{enumi}
\numberwithin{figure}{enumi}
\numberwithin{table}{enumi}

\item Find the coordinates of the point which divides the join of $(-1,7) \text{ and } (4,-3)$ in the ratio 2:3.
	\\
		\solution
	\iffalse
\documentclass[12pt]{article}
\usepackage{graphicx}
\usepackage{amsmath}
\usepackage{mathtools}
\usepackage{gensymb}

\newcommand{\mydet}[1]{\ensuremath{\begin{vmatrix}#1\end{vmatrix}}}
\providecommand{\brak}[1]{\ensuremath{\left(#1\right)}}
\providecommand{\norm}[1]{\left\lVert#1\right\rVert}
\newcommand{\solution}{\noindent \textbf{Solution: }}
\newcommand{\myvec}[1]{\ensuremath{\begin{pmatrix}#1\end{pmatrix}}}
\let\vec\mathbf

\begin{document}
\begin{center}
\textbf\large{CHAPTER-7 \\ COORDINATE GEOMETRY}
\end{center}
\section*{Excercise 7.2}

1. Find the coordinates of the point which divides the join $\vec(-1,7) \text{ and } \vec(4,-3)$ in the ratio 2:3 :
\\
\\
\solution\\		
\fi
The coordinates and ratio are given as
\begin{align}
\vec{P}=\myvec{-1\\7\\},
\vec{Q}=\myvec{4\\-3\\},
n=\frac{3}{2}
\end{align}
Using section formula
\begin{align}
\vec{R}&=\frac{\vec{Q}+n\vec{P}}{1+n}\\
&=\frac{1}{1+\frac{3}{2}}  \myvec{\myvec{
4\\
-3\\
}
  +
   \frac{3}{2}\myvec{
-1\\
7\\
}}\\
&=\myvec{
1\\
3
}
\end{align}
See Fig. 
\ref{fig:chapters/10/7/2/1/Fig}
\begin{figure}[!h]
\begin{center}
   \includegraphics[width=\columnwidth]{chapters/10/7/2/1/figs/linefig.png}
\end{center}
\caption{}
\label{fig:chapters/10/7/2/1/Fig}
\end{figure}


\item Find the coordinates of the points of trisection of the line segment joining $(4,-1) \text{ and } (-2,3)$.
	\\
		\solution
	\begin{enumerate}[label=\thesection.\arabic*,ref=\thesection.\theenumi]
\numberwithin{equation}{enumi}
\numberwithin{figure}{enumi}
\numberwithin{table}{enumi}

\item Find the coordinates of the point which divides the join of $(-1,7) \text{ and } (4,-3)$ in the ratio 2:3.
	\\
		\solution
	\iffalse
\documentclass[12pt]{article}
\usepackage{graphicx}
\usepackage{amsmath}
\usepackage{mathtools}
\usepackage{gensymb}

\newcommand{\mydet}[1]{\ensuremath{\begin{vmatrix}#1\end{vmatrix}}}
\providecommand{\brak}[1]{\ensuremath{\left(#1\right)}}
\providecommand{\norm}[1]{\left\lVert#1\right\rVert}
\newcommand{\solution}{\noindent \textbf{Solution: }}
\newcommand{\myvec}[1]{\ensuremath{\begin{pmatrix}#1\end{pmatrix}}}
\let\vec\mathbf

\begin{document}
\begin{center}
\textbf\large{CHAPTER-7 \\ COORDINATE GEOMETRY}
\end{center}
\section*{Excercise 7.2}

1. Find the coordinates of the point which divides the join $\vec(-1,7) \text{ and } \vec(4,-3)$ in the ratio 2:3 :
\\
\\
\solution\\		
\fi
The coordinates and ratio are given as
\begin{align}
\vec{P}=\myvec{-1\\7\\},
\vec{Q}=\myvec{4\\-3\\},
n=\frac{3}{2}
\end{align}
Using section formula
\begin{align}
\vec{R}&=\frac{\vec{Q}+n\vec{P}}{1+n}\\
&=\frac{1}{1+\frac{3}{2}}  \myvec{\myvec{
4\\
-3\\
}
  +
   \frac{3}{2}\myvec{
-1\\
7\\
}}\\
&=\myvec{
1\\
3
}
\end{align}
See Fig. 
\ref{fig:chapters/10/7/2/1/Fig}
\begin{figure}[!h]
\begin{center}
   \includegraphics[width=\columnwidth]{chapters/10/7/2/1/figs/linefig.png}
\end{center}
\caption{}
\label{fig:chapters/10/7/2/1/Fig}
\end{figure}


\item Find the coordinates of the points of trisection of the line segment joining $(4,-1) \text{ and } (-2,3)$.
	\\
		\solution
	\begin{enumerate}[label=\thesection.\arabic*,ref=\thesection.\theenumi]
\numberwithin{equation}{enumi}
\numberwithin{figure}{enumi}
\numberwithin{table}{enumi}

\item Find the coordinates of the point which divides the join of $(-1,7) \text{ and } (4,-3)$ in the ratio 2:3.
	\\
		\solution
	\input{chapters/10/7/2/1/section.tex}
\item Find the coordinates of the points of trisection of the line segment joining $(4,-1) \text{ and } (-2,3)$.
	\\
		\solution
	\input{chapters/10/7/2/2/section.tex}
\item
	\iffalse
\item To conduct Sports Day activities, in your rectangular shaped school                   
ground ABCD, lines have 
drawn with chalk powder at a                 
distance of 1m each. 100 flower pots have been placed at a distance of 1m 
from each other along AD, as shown 
in Fig. 7.12. Niharika runs $ \frac {1}{4} $th the 
distance AD on the 2nd line and 
posts a green flag. Preet runs $ \frac {1}{5} $th 
the distance AD on the eighth line 
and posts a red flag. What is the 
distance between both the flags? If 
Rashmi has to post a blue flag exactly 
halfway between the line segment 
joining the two flags, where should 
she post her flag?
\begin{figure}[h!]
  \centering
  \includegraphics[width=\columnwidth]{sc.png}
  \caption{}
\label{fig:10/7/12Fig1}
\end{figure}               
\fi
      
\item Find the ratio in which the line segment joining the points $(-3,10) \text{ and } (6,-8)$ $\text{ is divided by } (-1,6)$.
	\\
		\solution
	\input{chapters/10/7/2/4/section.tex}
\item Find the ratio in which the line segment joining $A(1,-5) \text{ and } B(-4,5)$ $\text{is divided by the x-axis}$. Also find the coordinates of the point of division.
\item If $(1,2), (4,y), (x,6), (3,5)$ are the vertices of a parallelogram taken in order, find x and y.
	\\
		\solution
	\input{chapters/10/7/2/6/para1.tex}
\item Find the coordinates of a point A, where AB is the diameter of a circle whose centre is $(2,-3) \text{ and }$ B is $(1,4)$.
	\\
		\solution
	\input{chapters/10/7/2/7/section.tex}
\item If A \text{ and } B are $(-2,-2) \text{ and } (2,-4)$, respectively, find the coordinates of P such that AP= $\frac {3}{7}$AB $\text{ and }$ P lies on the line segment AB.
	\\
		\solution
	\input{chapters/10/7/2/8/section.tex}
\item Find the coordinates of the points which divide the line segment joining $A(-2,2) \text{ and } B(2,8)$ into four equal parts.
	\\
		\solution
	\input{chapters/10/7/2/9/section.tex}
\item Find the area of a rhombus if its vertices are $(3,0), (4,5), (-1,4) \text{ and } (-2,-1)$ taken in order. [$\vec{Hint}$ : Area of rhombus =$\frac {1}{2}$(product of its diagonals)]
	\\
		\solution
	\input{chapters/10/7/2/10/cross.tex}
\item Find the position vector of a point R which divides the line joining two points $\vec{P}$
and $\vec{Q}$ whose position vectors are $\hat{i}+2\hat{j}-\hat{k}$ and $-\hat{i}+\hat{j}+\hat{k}$ respectively, in the
ratio 2 : 1
\begin{enumerate}
    \item  internally
    \item  externally
\end{enumerate}
\solution
		\input{chapters/12/10/2/15/section.tex}
\item Find the position vector of the mid point of the vector joining the points $\vec{P}$(2, 3, 4)
and $\vec{Q}$(4, 1, –2).
\\
\solution
		\input{chapters/12/10/2/16/section.tex}
\item Determine the ratio in which the line $2x+y  - 4=0$ divides the line segment joining the points $\vec{A}(2, - 2)$  and  $\vec{B}(3, 7)$.
\\
\solution
	\input{chapters/10/7/4/1/section.tex}
\item Let $\vec{A}(4, 2), \vec{B}(6, 5)$  and $ \vec{C}(1, 4)$ be the vertices of $\triangle ABC$.
\begin{enumerate}
\item The median from $\vec{A}$ meets $BC$ at $\vec{D}$. Find the coordinates of the point $\vec{D}$.
\item Find the coordinates of the point $\vec{P}$ on $AD$ such that $AP : PD = 2 : 1$.
\item Find the coordinates of points $\vec{Q}$ and $\vec{R}$ on medians $BE$ and $CF$ respectively such that $BQ : QE = 2 : 1$  and  $CR : RF = 2 : 1$.
\item What do you observe?
\item If $\vec{A}, \vec{B}$ and $\vec{C}$  are the vertices of $\triangle ABC$, find the coordinates of the centroid of the triangle.
\end{enumerate}
\solution
	\input{chapters/10/7/4/7/section.tex}
\item Find the slope of a line, which passes through the origin and the mid point of the line segment joining the points $\vec{P}$(0,-4) and $\vec{B}$(8,0).
\label{chapters/11/10/1/5}
\input{chapters/11/10/1/5/matrix.tex}
\item Find the position vector of a point R which divides the line joining two points P and Q whose position vectors are $(2\vec{a}+\vec{b})$ and $(\vec{a}-3\vec{b})$
externally in the ratio 1 : 2. Also, show that P is the mid point of the line segment RQ.\\
	\solution
%		\input{chapters/12/10/5/9/section.tex}

\end{enumerate}


\item
	\iffalse
\item To conduct Sports Day activities, in your rectangular shaped school                   
ground ABCD, lines have 
drawn with chalk powder at a                 
distance of 1m each. 100 flower pots have been placed at a distance of 1m 
from each other along AD, as shown 
in Fig. 7.12. Niharika runs $ \frac {1}{4} $th the 
distance AD on the 2nd line and 
posts a green flag. Preet runs $ \frac {1}{5} $th 
the distance AD on the eighth line 
and posts a red flag. What is the 
distance between both the flags? If 
Rashmi has to post a blue flag exactly 
halfway between the line segment 
joining the two flags, where should 
she post her flag?
\begin{figure}[h!]
  \centering
  \includegraphics[width=\columnwidth]{sc.png}
  \caption{}
\label{fig:10/7/12Fig1}
\end{figure}               
\fi
      
\item Find the ratio in which the line segment joining the points $(-3,10) \text{ and } (6,-8)$ $\text{ is divided by } (-1,6)$.
	\\
		\solution
	\iffalse
\documentclass[12pt]{article}
\usepackage{graphicx}
%\documentclass[journal,12pt,twocolumn]{IEEEtran}
\usepackage[none]{hyphenat}
\usepackage{graphicx}
\usepackage{listings}
\usepackage[english]{babel}
\usepackage{graphicx}
\usepackage{caption} 
\usepackage{hyperref}
\usepackage{booktabs}
\def\inputGnumericTable{}
\usepackage{color}                                            %%
    \usepackage{array}                                            %%
    \usepackage{longtable}                                        %%
    \usepackage{calc}                                             %%
    \usepackage{multirow}                                         %%
    \usepackage{hhline}                                           %%
    \usepackage{ifthen}
\usepackage{array}
\usepackage{amsmath}   % for having text in math mode
\usepackage{listings}
\lstset{
language=tex,
frame=single, 
breaklines=true
}
  
%Following 2 lines were added to remove the blank page at the beginning
\usepackage{atbegshi}% http://ctan.org/pkg/atbegshi
\AtBeginDocument{\AtBeginShipoutNext{\AtBeginShipoutDiscard}}
%
%New macro definitions
\newcommand{\mydet}[1]{\ensuremath{\begin{vmatrix}#1\end{vmatrix}}}
\providecommand{\brak}[1]{\ensuremath{\left(#1\right)}}
\providecommand{\norm}[1]{\left\lVert#1\right\rVert}
\newcommand{\solution}{\noindent \textbf{Solution: }}
\newcommand{\myvec}[1]{\ensuremath{\begin{pmatrix}#1\end{pmatrix}}}
\let\vec\mathbf
\begin{document}
\begin{center}
\title{\textbf{Coordinate Geometry}}
\date{\vspace{-5ex}} %Not to print date automatically
\maketitle
\end{center}
\setcounter{page}{1}
\section*{10$^{th}$ Maths - Chapter 7}
This is Problem-4 from Exercise 7.2
\begin{enumerate}
\item Find the ratio in which the line segement joining the points $\myvec{-3 \\ 10}$ and $\myvec{6\\-8}$ is divided by $\myvec{-1\\6}$.\\
\solution \\
\fi
		The input parameters for this problem are available in Table \eqref{tab:10/7/2/4-1}.
\begin{table}[ht!]
\input{chapters/10/7/2/4/tables/table.tex}
\caption{}
\label{tab:10/7/2/4-1} 
\end{table}
Using section formula,
\begin{align}
         \vec{R} &=\frac{\vec{Q}+n\vec{P}}{1+n}\label{eq:chapters/10/7/2/4/1}
\end{align}
Substituting the values of $\vec{P},\vec{Q}$ and $\vec{R}$ in \eqref{eq:chapters/10/7/2/4/1}
\begin{align}
         \myvec{-1\\6} &=\frac{{\myvec{-3\\10}+n\myvec{6\\-8}}}{1+n}\\
 &=\frac{1}{1+n}\brak{{\myvec{-3\\10}+n\myvec{6\\-8}}} \\
 &=\frac{1}{1+n}\myvec{-3+6n\\10-8n} \label{eq:chapters/10/7/2/4/4}
\end{align}
Simplifying \eqref{eq:chapters/10/7/2/4/4} yeilds,
\begin{align}
          -1 &=\frac{-3+6n}{1+n}\\
\implies          n &=\frac{2}{7}
\end{align}
Also,
\begin{align}
          6 &=\frac{10-8n}{1+n}\\
    \implies      n &=\frac{2}{7}
\end{align}
Hence the desired ratio is $\dfrac{2}{7}$.  
\begin{figure}[!h]
 \begin{center}
  \includegraphics[width=\columnwidth]{chapters/10/7/2/4/figs/fig.png}
 \end{center}
\caption{}
\label{fig:10/7/2/4Fig1}
\end{figure}

\item Find the ratio in which the line segment joining $A(1,-5) \text{ and } B(-4,5)$ $\text{is divided by the x-axis}$. Also find the coordinates of the point of division.
\item If $(1,2), (4,y), (x,6), (3,5)$ are the vertices of a parallelogram taken in order, find x and y.
	\\
		\solution
	\iffalse
\documentclass[12pt]{article}
\usepackage{graphicx}
%\documentclass[journal,12pt,twocolumn]{IEEEtran}
\def\inputGnumericTable{}
\usepackage{color}                                            %%
    \usepackage{array}                                            %%
    \usepackage{longtable}                                        %%
    \usepackage{calc}                                             %%
    \usepackage{multirow}                                         %%
    \usepackage{hhline}                                           %%
    \usepackage{ifthen}
\usepackage[none]{hyphenat}
\usepackage{graphicx}
\usepackage{listings}
\usepackage[english]{babel}
\usepackage{graphicx}
\usepackage{caption} 
\usepackage{hyperref}
\usepackage{booktabs}
\usepackage{array}
\usepackage{amsmath}   % for having text in math mode
\usepackage{listings}
\lstset{
  frame=single,
  breaklines=true
}
  
%Following 2 lines were added to remove the blank page at the beginning
\usepackage{atbegshi}% http://ctan.org/pkg/atbegshi
\AtBeginDocument{\AtBeginShipoutNext{\AtBeginShipoutDiscard}}
%


%New macro definitions
\newcommand{\mydet}[1]{\ensuremath{\begin{vmatrix}#1\end{vmatrix}}}
\providecommand{\brak}[1]{\ensuremath{\left(#1\right)}}
\providecommand{\norm}[1]{\left\lVert#1\right\rVert}
\newcommand{\solution}{\noindent \textbf{Solution: }}
\newcommand{\myvec}[1]{\ensuremath{\begin{pmatrix}#1\end{pmatrix}}}
\let\vec\mathbf

\begin{document}

\begin{center}
\title{\textbf{Properties of Parallelegram}}
\date{\vspace{-5ex}} %Not to print date automatically
\maketitle
\end{center}

\setcounter{page}{1}

\section{10$^{th}$ Maths - Chapter 7}

This is Problem-6 from Exercise 7.2

\begin{enumerate}
\item If $\vec{A}(1, 2),\vec{B}(4, x),\vec{C}(y, 6) \text{and } \vec{D}(3, 5)$ are the vertices of a parallelogram taken in order,find x and y.
\end{enumerate}
\fi

The input parameters for this problem are available in
\ref{table:chapters/10/7/2/6/tables/}.	
\begin{table}[!ht]
	\centering
	\input{chapters/10/7/2/6/tables/table.tex}
\caption{}
\label{table:chapters/10/7/2/6/tables/}	
\end{table}
From the given information,
\begin{align}
  \label{eq:chapters/10/7/2/6/tables/det2f}
	\vec{B}-\vec{A} &= \myvec{4 \\y } - \myvec{1 \\2 }  = \myvec{3 \\y-2 }\\
	\vec{C}-\vec{D} &= \myvec{x \\6 } - \myvec{3 \\5 }  = \myvec{x-3 \\1}
\end{align}
Since $ABCD$ is a parallellogram,
\begin{align}
	\myvec{3\\y-2}&=\myvec{x-3\\1}\\
	\implies x&=6 ,y=3
\end{align}
Fig. \ref{fig:chapters/10/7/2/6/Fig3}
provides a verification.
\begin{figure}[h!]
	\begin{center}
  \includegraphics[width=\columnwidth]{chapters/10/7/2/6/figs/para.pdf}
	\end{center}
\caption{}
\label{fig:chapters/10/7/2/6/Fig3}
\end{figure}


\item Find the coordinates of a point A, where AB is the diameter of a circle whose centre is $(2,-3) \text{ and }$ B is $(1,4)$.
	\\
		\solution
	\iffalse
\documentclass[12pt]{article}
\usepackage{graphicx}
\usepackage{amsmath}
\usepackage{mathtools}
\usepackage{gensymb}

\newcommand{\mydet}[1]{\ensuremath{\begin{vmatrix}#1\end{vmatrix}}}
\providecommand{\brak}[1]{\ensuremath{\left(#1\right)}}
\providecommand{\norm}[1]{\left\lVert#1\right\rVert}
\newcommand{\solution}{\noindent \textbf{Solution: }}
\newcommand{\myvec}[1]{\ensuremath{\begin{pmatrix}#1\end{pmatrix}}}
\let\vec\mathbf

\begin{document}
\begin{center}
\section*{CHAPTER 7 - COORDINATE GEOMETRY}

\end{center}
\section*{Excercise 7.2}

Q7.Find the coordinates of point $\vec{A}$, where AB is the diameter of a circle where the center is (2,-3) and $\vec{B}$ is the point (1,4):

\solution
\begin{enumerate}
\item The coordinates $\vec{B}$ and center $\vec{C}$ are given, where:
	\fi
	Let
	\begin{align}
	\vec{B} = \myvec{
		1\\
	    4\\
		},
	\vec{C} = \myvec{
	    2\\
	   -3\\
		}
	\end{align}
	\iffalse
Let us assume the coordinates of $\vec{A}$. Now, $\vec{C}$ is the center which is midpoint of line AB and $\vec{B}$ is one of the coordinate of diameter AB of a circle.
	\fi	
Hence,	
	\begin{align}
	\vec{C} &= \frac{\vec{A+B}}{2} \\
\implies	2\vec{C} &= \vec{A}+\vec{B} \\
		\text{or, }	\vec{A} &= 2\vec{C}-\vec{B} \\
	 &= \myvec{3\\-10\\}	
	\end{align}       
	See Fig. 
\ref{fig:chapters/10/7/2/7Fig}.
\begin{figure}[!h]
\begin{center}	
	\includegraphics[width=\columnwidth]{chapters/10/7/2/7/figs/Vector1.png}
\end{center}
\caption{}
\label{fig:chapters/10/7/2/7Fig}
\end{figure}
	

\item If A \text{ and } B are $(-2,-2) \text{ and } (2,-4)$, respectively, find the coordinates of P such that AP= $\frac {3}{7}$AB $\text{ and }$ P lies on the line segment AB.
	\\
		\solution
	\iffalse
\documentclass[journal,10pt,twocolumn]{article}
\usepackage{graphicx}
\usepackage[none]{hyphenat}
\usepackage{graphicx}
\usepackage{listings}
\usepackage[english]{babel}
\usepackage{graphicx}
\usepackage{caption} 
\usepackage{booktabs}
\usepackage{array}
\usepackage{amssymb} % for \because
\usepackage{amsmath}   % for having text in math mode
\usepackage{extarrows} % for Row operations arrows
\usepackage{listings}
\usepackage[utf8]{inputenc}
\lstset{
  frame=single,
  breaklines=true
}
\usepackage{hyperref}
  
%Following 2 lines were added to remove the blank page at the beginning
\usepackage{atbegshi}% http://ctan.org/pkg/atbegshi
\AtBeginDocument{\AtBeginShipoutNext{\AtBeginShipoutDiscard}}


%New macro definitions
\newcommand{\mydet}[1]{\ensuremath{\begin{vmatrix}#1\end{vmatrix}}}
\providecommand{\brak}[1]{\ensuremath{\left(#1\right)}}
\newcommand{\solution}{\noindent \textbf{Solution: }}
\newcommand{\myvec}[1]{\ensuremath{\begin{pmatrix}#1\end{pmatrix}}}
\providecommand{\norm}[1]{\left\lVert#1\right\rVert}
\providecommand{\abs}[1]{\left\vert#1\right\vert}
\let\vec\mathbf

\begin{document}

\begin{center}
\title{\textbf{VECTORS}}
\date{\vspace{-5ex}} %Not to print date automatically
\maketitle
\end{center}

\section{10$^{th}$ Maths - EXERCISE-7.2}

\begin{enumerate}
\item If A and B are $(– 2, – 2)\text{ and }(2, – 4)$, respectively, find the coordinates of P such that $AP =\frac{3}{7}AB$ and P lies on the line segment AB. 

\section{SOLUTION}
Given points are
\begin{align}
\vec{A}=\myvec{-2\\ -2} ,
\vec{B}=\myvec{2\\ -4}
\end{align}
The equation of the formula is
\fi
Using section formula, 
\begin{align}
\vec{P}&=\frac{\vec{A}+n\vec{B}}{1+n}
\end{align}
where
\begin{align}
	n =\frac{3}{4}
\end{align}
Thus,
\begin{align}
\vec{P}&=\frac{1}{1+\frac{3}{4}}\brak{\myvec{-2\\-2}+\frac{3}{4}\myvec{2\\-4}}\\
&=\myvec{\frac{-2}{7}\\[1pt] \frac{-20}{7}}
\end{align}
See Fig. 
   \ref{fig:chapters/10/7/2/8/vec.png}
\begin{figure}
   \centering 
 \includegraphics[width=\columnwidth]{chapters/10/7/2/8/figs/vec.png}
   \caption{}
   \label{fig:chapters/10/7/2/8/vec.png}
   \end{figure}

\item Find the coordinates of the points which divide the line segment joining $A(-2,2) \text{ and } B(2,8)$ into four equal parts.
	\\
		\solution
	\begin{enumerate}[label=\thesection.\arabic*,ref=\thesection.\theenumi]
\numberwithin{equation}{enumi}
\numberwithin{figure}{enumi}
\numberwithin{table}{enumi}

\item Find the coordinates of the point which divides the join of $(-1,7) \text{ and } (4,-3)$ in the ratio 2:3.
	\\
		\solution
	\input{chapters/10/7/2/1/section.tex}
\item Find the coordinates of the points of trisection of the line segment joining $(4,-1) \text{ and } (-2,3)$.
	\\
		\solution
	\input{chapters/10/7/2/2/section.tex}
\item
	\iffalse
\item To conduct Sports Day activities, in your rectangular shaped school                   
ground ABCD, lines have 
drawn with chalk powder at a                 
distance of 1m each. 100 flower pots have been placed at a distance of 1m 
from each other along AD, as shown 
in Fig. 7.12. Niharika runs $ \frac {1}{4} $th the 
distance AD on the 2nd line and 
posts a green flag. Preet runs $ \frac {1}{5} $th 
the distance AD on the eighth line 
and posts a red flag. What is the 
distance between both the flags? If 
Rashmi has to post a blue flag exactly 
halfway between the line segment 
joining the two flags, where should 
she post her flag?
\begin{figure}[h!]
  \centering
  \includegraphics[width=\columnwidth]{sc.png}
  \caption{}
\label{fig:10/7/12Fig1}
\end{figure}               
\fi
      
\item Find the ratio in which the line segment joining the points $(-3,10) \text{ and } (6,-8)$ $\text{ is divided by } (-1,6)$.
	\\
		\solution
	\input{chapters/10/7/2/4/section.tex}
\item Find the ratio in which the line segment joining $A(1,-5) \text{ and } B(-4,5)$ $\text{is divided by the x-axis}$. Also find the coordinates of the point of division.
\item If $(1,2), (4,y), (x,6), (3,5)$ are the vertices of a parallelogram taken in order, find x and y.
	\\
		\solution
	\input{chapters/10/7/2/6/para1.tex}
\item Find the coordinates of a point A, where AB is the diameter of a circle whose centre is $(2,-3) \text{ and }$ B is $(1,4)$.
	\\
		\solution
	\input{chapters/10/7/2/7/section.tex}
\item If A \text{ and } B are $(-2,-2) \text{ and } (2,-4)$, respectively, find the coordinates of P such that AP= $\frac {3}{7}$AB $\text{ and }$ P lies on the line segment AB.
	\\
		\solution
	\input{chapters/10/7/2/8/section.tex}
\item Find the coordinates of the points which divide the line segment joining $A(-2,2) \text{ and } B(2,8)$ into four equal parts.
	\\
		\solution
	\input{chapters/10/7/2/9/section.tex}
\item Find the area of a rhombus if its vertices are $(3,0), (4,5), (-1,4) \text{ and } (-2,-1)$ taken in order. [$\vec{Hint}$ : Area of rhombus =$\frac {1}{2}$(product of its diagonals)]
	\\
		\solution
	\input{chapters/10/7/2/10/cross.tex}
\item Find the position vector of a point R which divides the line joining two points $\vec{P}$
and $\vec{Q}$ whose position vectors are $\hat{i}+2\hat{j}-\hat{k}$ and $-\hat{i}+\hat{j}+\hat{k}$ respectively, in the
ratio 2 : 1
\begin{enumerate}
    \item  internally
    \item  externally
\end{enumerate}
\solution
		\input{chapters/12/10/2/15/section.tex}
\item Find the position vector of the mid point of the vector joining the points $\vec{P}$(2, 3, 4)
and $\vec{Q}$(4, 1, –2).
\\
\solution
		\input{chapters/12/10/2/16/section.tex}
\item Determine the ratio in which the line $2x+y  - 4=0$ divides the line segment joining the points $\vec{A}(2, - 2)$  and  $\vec{B}(3, 7)$.
\\
\solution
	\input{chapters/10/7/4/1/section.tex}
\item Let $\vec{A}(4, 2), \vec{B}(6, 5)$  and $ \vec{C}(1, 4)$ be the vertices of $\triangle ABC$.
\begin{enumerate}
\item The median from $\vec{A}$ meets $BC$ at $\vec{D}$. Find the coordinates of the point $\vec{D}$.
\item Find the coordinates of the point $\vec{P}$ on $AD$ such that $AP : PD = 2 : 1$.
\item Find the coordinates of points $\vec{Q}$ and $\vec{R}$ on medians $BE$ and $CF$ respectively such that $BQ : QE = 2 : 1$  and  $CR : RF = 2 : 1$.
\item What do you observe?
\item If $\vec{A}, \vec{B}$ and $\vec{C}$  are the vertices of $\triangle ABC$, find the coordinates of the centroid of the triangle.
\end{enumerate}
\solution
	\input{chapters/10/7/4/7/section.tex}
\item Find the slope of a line, which passes through the origin and the mid point of the line segment joining the points $\vec{P}$(0,-4) and $\vec{B}$(8,0).
\label{chapters/11/10/1/5}
\input{chapters/11/10/1/5/matrix.tex}
\item Find the position vector of a point R which divides the line joining two points P and Q whose position vectors are $(2\vec{a}+\vec{b})$ and $(\vec{a}-3\vec{b})$
externally in the ratio 1 : 2. Also, show that P is the mid point of the line segment RQ.\\
	\solution
%		\input{chapters/12/10/5/9/section.tex}

\end{enumerate}


\item Find the area of a rhombus if its vertices are $(3,0), (4,5), (-1,4) \text{ and } (-2,-1)$ taken in order. [$\vec{Hint}$ : Area of rhombus =$\frac {1}{2}$(product of its diagonals)]
	\\
		\solution
	\iffalse
\documentclass[12pt]{article}
\usepackage{graphicx}
%\documentclass[journal,12pt,twocolumn]{IEEEtran}
\usepackage[none]{hyphenat}
\usepackage{graphicx}
\usepackage{listings}
\usepackage[english]{babel}
\usepackage{graphicx}
\usepackage{caption} 
\usepackage{hyperref}
\usepackage{booktabs}
\def\inputGnumericTable{}
\usepackage{color}                                            %%
    \usepackage{array}                                            %%
    \usepackage{longtable}                                        %%
    \usepackage{calc}                                             %%
    \usepackage{multirow}                                         %%
    \usepackage{hhline}                                           %%
    \usepackage{ifthen}
\usepackage{array}
\usepackage{amsmath}   % for having text in math mode
\usepackage{listings}
\lstset{
language=tex,
frame=single, 
breaklines=true
}
  
%Following 2 lines were added to remove the blank page at the beginning
\usepackage{atbegshi}% http://ctan.org/pkg/atbegshi
\AtBeginDocument{\AtBeginShipoutNext{\AtBeginShipoutDiscard}}
%


%New macro definitions
\newcommand{\mydet}[1]{\ensuremath{\begin{vmatrix}#1\end{vmatrix}}}
\providecommand{\brak}[1]{\ensuremath{\left(#1\right)}}
\providecommand{\norm}[1]{\left\lVert#1\right\rVert}
\newcommand{\solution}{\noindent \textbf{Solution: }}
\newcommand{\myvec}[1]{\ensuremath{\begin{pmatrix}#1\end{pmatrix}}}
\let\vec\mathbf

\begin{document}

\begin{center}
\title{\textbf{Coordinate Geometry}}
\date{\vspace{-5ex}} %Not to print date automatically
\maketitle
\end{center}

\setcounter{page}{1}



\begin{enumerate}

\item\textbf{Problem statement :} Find the area of a rhombus of its vertices are $\myvec{3 ,0}$, $\myvec{4 ,5}$, $\myvec{-1 ,4}$ and $\myvec{-2 ,-1}$taken in order

\solution \\
\fi
The input vertices for this problem are given as
	\begin{align}
	\vec{A} = \myvec{
		3\\
		0
		},
	\vec{B} = \myvec{
		4\\
		5
		},
        \vec{C} = \myvec{
		-1\\
		4
		},
        \vec{D} = \myvec{
		-2\\
		-1
		}
	\end{align}
Since		
\begin{align}
 \vec{A-D}= \myvec{3 \\ 0} - \myvec{-2 \\-1}= \myvec{5\\1}
 \\
  \vec{B-A}= \myvec{4 \\ 5} - \myvec{3 \\0}= \myvec{1\\5}
\end{align}
the area of the rhombus is
\begin{align}
                \norm{\myvec{\vec{A-D}}\times \myvec{\vec{B-A}}}=\mydet{5 & 1\\1 & 5} = 24
\end{align}
See Fig. 
\ref{fig:chapters/10/7/2/10/gFig1}.
\begin{figure}[!h]
 \begin{center}
  \includegraphics[width=\columnwidth]{chapters/10/7/2/10/figs/fig.pdf}
 \end{center}
\caption{}
\label{fig:chapters/10/7/2/10/gFig1}
\end{figure}

\item Find the position vector of a point R which divides the line joining two points $\vec{P}$
and $\vec{Q}$ whose position vectors are $\hat{i}+2\hat{j}-\hat{k}$ and $-\hat{i}+\hat{j}+\hat{k}$ respectively, in the
ratio 2 : 1
\begin{enumerate}
    \item  internally
    \item  externally
\end{enumerate}
\solution
		\begin{enumerate}[label=\thesection.\arabic*,ref=\thesection.\theenumi]
\numberwithin{equation}{enumi}
\numberwithin{figure}{enumi}
\numberwithin{table}{enumi}

\item Find the coordinates of the point which divides the join of $(-1,7) \text{ and } (4,-3)$ in the ratio 2:3.
	\\
		\solution
	\input{chapters/10/7/2/1/section.tex}
\item Find the coordinates of the points of trisection of the line segment joining $(4,-1) \text{ and } (-2,3)$.
	\\
		\solution
	\input{chapters/10/7/2/2/section.tex}
\item
	\iffalse
\item To conduct Sports Day activities, in your rectangular shaped school                   
ground ABCD, lines have 
drawn with chalk powder at a                 
distance of 1m each. 100 flower pots have been placed at a distance of 1m 
from each other along AD, as shown 
in Fig. 7.12. Niharika runs $ \frac {1}{4} $th the 
distance AD on the 2nd line and 
posts a green flag. Preet runs $ \frac {1}{5} $th 
the distance AD on the eighth line 
and posts a red flag. What is the 
distance between both the flags? If 
Rashmi has to post a blue flag exactly 
halfway between the line segment 
joining the two flags, where should 
she post her flag?
\begin{figure}[h!]
  \centering
  \includegraphics[width=\columnwidth]{sc.png}
  \caption{}
\label{fig:10/7/12Fig1}
\end{figure}               
\fi
      
\item Find the ratio in which the line segment joining the points $(-3,10) \text{ and } (6,-8)$ $\text{ is divided by } (-1,6)$.
	\\
		\solution
	\input{chapters/10/7/2/4/section.tex}
\item Find the ratio in which the line segment joining $A(1,-5) \text{ and } B(-4,5)$ $\text{is divided by the x-axis}$. Also find the coordinates of the point of division.
\item If $(1,2), (4,y), (x,6), (3,5)$ are the vertices of a parallelogram taken in order, find x and y.
	\\
		\solution
	\input{chapters/10/7/2/6/para1.tex}
\item Find the coordinates of a point A, where AB is the diameter of a circle whose centre is $(2,-3) \text{ and }$ B is $(1,4)$.
	\\
		\solution
	\input{chapters/10/7/2/7/section.tex}
\item If A \text{ and } B are $(-2,-2) \text{ and } (2,-4)$, respectively, find the coordinates of P such that AP= $\frac {3}{7}$AB $\text{ and }$ P lies on the line segment AB.
	\\
		\solution
	\input{chapters/10/7/2/8/section.tex}
\item Find the coordinates of the points which divide the line segment joining $A(-2,2) \text{ and } B(2,8)$ into four equal parts.
	\\
		\solution
	\input{chapters/10/7/2/9/section.tex}
\item Find the area of a rhombus if its vertices are $(3,0), (4,5), (-1,4) \text{ and } (-2,-1)$ taken in order. [$\vec{Hint}$ : Area of rhombus =$\frac {1}{2}$(product of its diagonals)]
	\\
		\solution
	\input{chapters/10/7/2/10/cross.tex}
\item Find the position vector of a point R which divides the line joining two points $\vec{P}$
and $\vec{Q}$ whose position vectors are $\hat{i}+2\hat{j}-\hat{k}$ and $-\hat{i}+\hat{j}+\hat{k}$ respectively, in the
ratio 2 : 1
\begin{enumerate}
    \item  internally
    \item  externally
\end{enumerate}
\solution
		\input{chapters/12/10/2/15/section.tex}
\item Find the position vector of the mid point of the vector joining the points $\vec{P}$(2, 3, 4)
and $\vec{Q}$(4, 1, –2).
\\
\solution
		\input{chapters/12/10/2/16/section.tex}
\item Determine the ratio in which the line $2x+y  - 4=0$ divides the line segment joining the points $\vec{A}(2, - 2)$  and  $\vec{B}(3, 7)$.
\\
\solution
	\input{chapters/10/7/4/1/section.tex}
\item Let $\vec{A}(4, 2), \vec{B}(6, 5)$  and $ \vec{C}(1, 4)$ be the vertices of $\triangle ABC$.
\begin{enumerate}
\item The median from $\vec{A}$ meets $BC$ at $\vec{D}$. Find the coordinates of the point $\vec{D}$.
\item Find the coordinates of the point $\vec{P}$ on $AD$ such that $AP : PD = 2 : 1$.
\item Find the coordinates of points $\vec{Q}$ and $\vec{R}$ on medians $BE$ and $CF$ respectively such that $BQ : QE = 2 : 1$  and  $CR : RF = 2 : 1$.
\item What do you observe?
\item If $\vec{A}, \vec{B}$ and $\vec{C}$  are the vertices of $\triangle ABC$, find the coordinates of the centroid of the triangle.
\end{enumerate}
\solution
	\input{chapters/10/7/4/7/section.tex}
\item Find the slope of a line, which passes through the origin and the mid point of the line segment joining the points $\vec{P}$(0,-4) and $\vec{B}$(8,0).
\label{chapters/11/10/1/5}
\input{chapters/11/10/1/5/matrix.tex}
\item Find the position vector of a point R which divides the line joining two points P and Q whose position vectors are $(2\vec{a}+\vec{b})$ and $(\vec{a}-3\vec{b})$
externally in the ratio 1 : 2. Also, show that P is the mid point of the line segment RQ.\\
	\solution
%		\input{chapters/12/10/5/9/section.tex}

\end{enumerate}


\item Find the position vector of the mid point of the vector joining the points $\vec{P}$(2, 3, 4)
and $\vec{Q}$(4, 1, –2).
\\
\solution
		\begin{enumerate}[label=\thesection.\arabic*,ref=\thesection.\theenumi]
\numberwithin{equation}{enumi}
\numberwithin{figure}{enumi}
\numberwithin{table}{enumi}

\item Find the coordinates of the point which divides the join of $(-1,7) \text{ and } (4,-3)$ in the ratio 2:3.
	\\
		\solution
	\input{chapters/10/7/2/1/section.tex}
\item Find the coordinates of the points of trisection of the line segment joining $(4,-1) \text{ and } (-2,3)$.
	\\
		\solution
	\input{chapters/10/7/2/2/section.tex}
\item
	\iffalse
\item To conduct Sports Day activities, in your rectangular shaped school                   
ground ABCD, lines have 
drawn with chalk powder at a                 
distance of 1m each. 100 flower pots have been placed at a distance of 1m 
from each other along AD, as shown 
in Fig. 7.12. Niharika runs $ \frac {1}{4} $th the 
distance AD on the 2nd line and 
posts a green flag. Preet runs $ \frac {1}{5} $th 
the distance AD on the eighth line 
and posts a red flag. What is the 
distance between both the flags? If 
Rashmi has to post a blue flag exactly 
halfway between the line segment 
joining the two flags, where should 
she post her flag?
\begin{figure}[h!]
  \centering
  \includegraphics[width=\columnwidth]{sc.png}
  \caption{}
\label{fig:10/7/12Fig1}
\end{figure}               
\fi
      
\item Find the ratio in which the line segment joining the points $(-3,10) \text{ and } (6,-8)$ $\text{ is divided by } (-1,6)$.
	\\
		\solution
	\input{chapters/10/7/2/4/section.tex}
\item Find the ratio in which the line segment joining $A(1,-5) \text{ and } B(-4,5)$ $\text{is divided by the x-axis}$. Also find the coordinates of the point of division.
\item If $(1,2), (4,y), (x,6), (3,5)$ are the vertices of a parallelogram taken in order, find x and y.
	\\
		\solution
	\input{chapters/10/7/2/6/para1.tex}
\item Find the coordinates of a point A, where AB is the diameter of a circle whose centre is $(2,-3) \text{ and }$ B is $(1,4)$.
	\\
		\solution
	\input{chapters/10/7/2/7/section.tex}
\item If A \text{ and } B are $(-2,-2) \text{ and } (2,-4)$, respectively, find the coordinates of P such that AP= $\frac {3}{7}$AB $\text{ and }$ P lies on the line segment AB.
	\\
		\solution
	\input{chapters/10/7/2/8/section.tex}
\item Find the coordinates of the points which divide the line segment joining $A(-2,2) \text{ and } B(2,8)$ into four equal parts.
	\\
		\solution
	\input{chapters/10/7/2/9/section.tex}
\item Find the area of a rhombus if its vertices are $(3,0), (4,5), (-1,4) \text{ and } (-2,-1)$ taken in order. [$\vec{Hint}$ : Area of rhombus =$\frac {1}{2}$(product of its diagonals)]
	\\
		\solution
	\input{chapters/10/7/2/10/cross.tex}
\item Find the position vector of a point R which divides the line joining two points $\vec{P}$
and $\vec{Q}$ whose position vectors are $\hat{i}+2\hat{j}-\hat{k}$ and $-\hat{i}+\hat{j}+\hat{k}$ respectively, in the
ratio 2 : 1
\begin{enumerate}
    \item  internally
    \item  externally
\end{enumerate}
\solution
		\input{chapters/12/10/2/15/section.tex}
\item Find the position vector of the mid point of the vector joining the points $\vec{P}$(2, 3, 4)
and $\vec{Q}$(4, 1, –2).
\\
\solution
		\input{chapters/12/10/2/16/section.tex}
\item Determine the ratio in which the line $2x+y  - 4=0$ divides the line segment joining the points $\vec{A}(2, - 2)$  and  $\vec{B}(3, 7)$.
\\
\solution
	\input{chapters/10/7/4/1/section.tex}
\item Let $\vec{A}(4, 2), \vec{B}(6, 5)$  and $ \vec{C}(1, 4)$ be the vertices of $\triangle ABC$.
\begin{enumerate}
\item The median from $\vec{A}$ meets $BC$ at $\vec{D}$. Find the coordinates of the point $\vec{D}$.
\item Find the coordinates of the point $\vec{P}$ on $AD$ such that $AP : PD = 2 : 1$.
\item Find the coordinates of points $\vec{Q}$ and $\vec{R}$ on medians $BE$ and $CF$ respectively such that $BQ : QE = 2 : 1$  and  $CR : RF = 2 : 1$.
\item What do you observe?
\item If $\vec{A}, \vec{B}$ and $\vec{C}$  are the vertices of $\triangle ABC$, find the coordinates of the centroid of the triangle.
\end{enumerate}
\solution
	\input{chapters/10/7/4/7/section.tex}
\item Find the slope of a line, which passes through the origin and the mid point of the line segment joining the points $\vec{P}$(0,-4) and $\vec{B}$(8,0).
\label{chapters/11/10/1/5}
\input{chapters/11/10/1/5/matrix.tex}
\item Find the position vector of a point R which divides the line joining two points P and Q whose position vectors are $(2\vec{a}+\vec{b})$ and $(\vec{a}-3\vec{b})$
externally in the ratio 1 : 2. Also, show that P is the mid point of the line segment RQ.\\
	\solution
%		\input{chapters/12/10/5/9/section.tex}

\end{enumerate}


\item Determine the ratio in which the line $2x+y  - 4=0$ divides the line segment joining the points $\vec{A}(2, - 2)$  and  $\vec{B}(3, 7)$.
\\
\solution
	\iffalse
\documentclass[journal,12pt,twocolumn]{IEEEtran}
\usepackage{graphicx}
\graphicspath{{./chapters/10/7/4/1/figs/}}{}
\usepackage{amsmath,amssymb,amsfonts,amsthm}
\newcommand{\myvec}[1]{\ensuremath{\begin{pmatrix}#1\end{pmatrix}}}
\providecommand{\norm}[1]{\lVert#1\rVert}
\usepackage{listings}
\usepackage{watermark}
\usepackage{titlesec}
\usepackage{caption}
\let\vec\mathbf
\lstset{
frame=single, 
breaklines=true,
columns=fullflexible
}
\thiswatermark{\centering \put(0,-105.0){\includegraphics[scale=0.15]{/sdcard/IITH/vector/vectpr-4/chapters/10/7/4/1/figs/logo.png}} }
\title{\mytitle}
\title{
Assignment - Vector-4
}
\author{Surajit Sarkar}
\begin{document}
\maketitle
%\tableofcontents
\bigskip
\section{\textbf{Problem}}
Determine the ratio in which the line 2x+y–4=0 divides the line segment joining the points A(2,–2) and B(3,7).
\section{\textbf{Solution}}
\begin{table}[h]
    \centering
    \begin{tabular}{|c|c|}
       \hline
       \textbf{Symbol}&\textbf{Value}  \\
       \hline
	    $\vec{A}$ & $\myvec{2\\-2}$\\
        \hline
	    $\vec{B}$ & $\myvec{3\\7}$\\
        \hline
	    c&$4$\\
        \hline
       $\vec{n}$ & $\myvec{2\\1}$\\
       \hline
    \end{tabular}
    \caption{Parameters}
    \label{tab:my_label}
\end{table}
Given equation
\fi
The given equation can be expressed as
\begin{align}
    \myvec{2&1}\vec{x}&=4\\
\end{align}
Using section formula, the point of division 
\begin{align}
    \vec{P} = \frac{k\vec{B+A}}{k+1}
\end{align}
which upon substitution in the equation of a line yields
\begin{align}
    \implies\vec{n}^{\top}\myvec{\frac{k\vec{B+A}}{k+1}}&=c\\
    \implies k&=\frac{c-\vec{n}^{\top}\vec{A}}{\vec{n}^{\top}\vec{B}-c}\\
\end{align}
upon simplification.  Substituting numerical values, 
\begin{align}
    k=\frac{2}{9}
\end{align}
See Fig. 
\ref{fig:chapters/10/7/4/1vec}.
\begin{figure}[!h]
\centering
\includegraphics[width=\columnwidth]{chapters/10/7/4/1/figs/vec.pdf}
\caption{}
\label{fig:chapters/10/7/4/1vec}
\end{figure}


\item Let $\vec{A}(4, 2), \vec{B}(6, 5)$  and $ \vec{C}(1, 4)$ be the vertices of $\triangle ABC$.
\begin{enumerate}
\item The median from $\vec{A}$ meets $BC$ at $\vec{D}$. Find the coordinates of the point $\vec{D}$.
\item Find the coordinates of the point $\vec{P}$ on $AD$ such that $AP : PD = 2 : 1$.
\item Find the coordinates of points $\vec{Q}$ and $\vec{R}$ on medians $BE$ and $CF$ respectively such that $BQ : QE = 2 : 1$  and  $CR : RF = 2 : 1$.
\item What do you observe?
\item If $\vec{A}, \vec{B}$ and $\vec{C}$  are the vertices of $\triangle ABC$, find the coordinates of the centroid of the triangle.
\end{enumerate}
\solution
	\iffalse
\documentclass[12pt]{article}
\usepackage{graphicx}
\usepackage[none]{hyphenat}
\usepackage{graphicx}
\usepackage{listings}
\usepackage[english]{babel}
\usepackage{graphicx}
\usepackage{caption} 
\usepackage{booktabs}
\usepackage{array}
\usepackage{amssymb} % for \because
\usepackage{amsmath}   % for having text in math mode
\usepackage{extarrows} % for Row operations arrows
\usepackage{listings}
\usepackage[utf8]{inputenc}
\lstset{
  frame=single,
  breaklines=true
}
\usepackage{hyperref}
  
%Following 2 lines were added to remove the blank page at the beginning
\usepackage{atbegshi}% http://ctan.org/pkg/atbegshi
\AtBeginDocument{\AtBeginShipoutNext{\AtBeginShipoutDiscard}}


%New macro definitions
\newcommand{\mydet}[1]{\ensuremath{\begin{vmatrix}#1\end{vmatrix}}}
\providecommand{\brak}[1]{\ensuremath{\left(#1\right)}}
\newcommand{\solution}{\noindent \textbf{Solution: }}
\newcommand{\myvec}[1]{\ensuremath{\begin{pmatrix}#1\end{pmatrix}}}
\providecommand{\norm}[1]{\left\lVert#1\right\rVert}
\providecommand{\abs}[1]{\left\vert#1\right\vert}
\let\vec\mathbf

\begin{document}

\begin{center}
\title{\textbf{VECTORS}}
\date{\vspace{-5ex}} %Not to print date automatically
\maketitle
\end{center}

\section{10$^{th}$ Maths - EXERCISE-7.4}

Let A(4, 2), B(6, 5) and C(1, 4) be the vertices of $\triangle ABC$
\begin{enumerate}
\item The median from A meets BC at D. Find the coordinates of the point D.
\item Find the coordinates of the point P on AD such that $AP : PD = 2 : 1$
\item Find the coordinates of points Q and R on medians BE and CF respectively such
that $BQ : QE = 2 : 1 \text{and} CR : RF = 2 : 1.$
\item What do yo observe?
\item If $A(x_1, y_1), B(x_2, y_2) \text{and} C(x_3, y_3)$ are the vertices of $\triangle ABC$, find the coordinates of the centroid of the triangle.
\end{enumerate}

Given points are
\begin{align}
\vec{A}=\myvec{4\\ 2} ,
\vec{B}=\myvec{6\\ 5} ,
\vec{C}=\myvec{1\\ 4}
\end{align}
\fi

\begin{enumerate}
\item 
\begin{align}
\vec{D}&=\frac{\vec{B}+\vec{C}}{2}\\
&=\myvec{\frac{7}{2}\\[2pt] \frac{9}{2}}\\
\vec{E}&=\frac{\vec{A}+\vec{C}}{2}\\
&=\myvec{\frac{5}{2}\\ 3}\\
\vec{F}&=\frac{\vec{A}+\vec{B}}{2}\\
&=\myvec{5\\ \frac{7}{2}}
\end{align}

\item 
	For
$n=2$,
\begin{align}
\vec{P}&=\frac{1}{1+n}\brak{\myvec{\vec{A}+n\vec{D}}}\\
&=\frac{1}{3}\myvec{11\\11}
\end{align}

\item 
\begin{align}
\vec{Q}&=\frac{1}{1+n}\brak{\myvec{\vec{B}+n\vec{E}}}\\
&=\frac{1}{3}\myvec{11\\11}\\
\vec{R}&=\frac{1}{1+n}\brak{\myvec{\vec{C}+n\vec{F}}}\\
&=\frac{1}{3}\myvec{11\\11}\\
\end{align}

\item 
 $\vec{P},\vec{Q},\vec{R}$ are the same point.
   
\item 
\begin{align}
\vec{G}&=\frac{\vec{D}+\vec{E}+\vec{F}}{3}\\
&=\frac{1}{3}\myvec{11\\11}\\
\end{align} 
\end{enumerate}
See Fig.  
  \ref{fig:chapters/10/7/4/7/Figure}.
\begin{figure}[h!]
\centering
\includegraphics[width=\columnwidth]{chapters/10/7/4/7/figs/dj.pdf}
\caption{}
  \label{fig:chapters/10/7/4/7/Figure}
\end{figure}

\item Find the slope of a line, which passes through the origin and the mid point of the line segment joining the points $\vec{P}$(0,-4) and $\vec{B}$(8,0).
\label{chapters/11/10/1/5}
\iffalse
\documentclass[journal,12pt,twocolumn]{IEEEtran}
\usepackage{graphicx}
\graphicspath{{./figs/}}{}
\usepackage{amsmath,amssymb,amsfonts,amsthm}
\newcommand{\myvec}[1]{\ensuremath{\begin{pmatrix}#1\end{pmatrix}}}

\let\vec\mathbf

\title{
Matrix-Lines
}
\author{Jyothsna Paluchuri-FWC22059\\}
\begin{document}
\maketitle
\tableofcontents
\bigskip
\section{Problem Statement}
\fi
	\begin{figure}[!ht]
		\centering
 \includegraphics[width=\columnwidth]{chapters/11/10/1/5/figs/line.png}
		\caption{}
		\label{fig:11/10/1/5}
  	\end{figure}
	\\
	\solution
\iffalse
\section{Construction}
\begin{figure}[h]
    \centering
\includegraphics[width=\columnwidth]{line.png}
    \caption{Equation of the slope}
    \label{fig:my_label}
\end{figure}
\vspace{2cm}
\begin{table}[h]
    \centering
    \begin{tabular}{|c|c|c|c|}
       \hline
       \textbf{Symbol}&\textbf{Value}&\textbf{Description}  \\
       \hline
	    $\vec{P}$ & $\myvec{
		    0\\
		    -4}$
	    & Point on Y-axis\\
        \hline
	    $\vec{B}$ & $\myvec{8\\0}$
 & Point on X-axis\\
        \hline
	    $\vec{0}$ & $\myvec{0\\0}$
 & Origin\\
        \hline
    \end{tabular}
    \caption{Parameters}
    \label{tab:my_label}
\end{table}


\section{Solution}
Given that resultant line passes through origin and mid point of the line segment joining point P(0,-4) and B(8,0) \\
\\
\\
given ${\vec{P}}$=$\myvec{
  0\\
  -4}$
 , ${\vec{B}}$=$\myvec{
  8\\
  0}$
  
 \fi 
The mid point of $PB$ is
\begin{align}
\vec{M} &=\frac{1}{2}(\vec{P}+\vec{B})
	= \myvec{4 \\ -2}  
\end{align}
The direction vector of line joining $\vec{O}, \vec{M}$ is 
\begin{align}
\vec{m}&=\vec{O}-\vec{M}
 = -\vec{M}
\end{align}
which can be expressed as
\begin{align}
	\myvec{1 \\ -\frac{1}{2}}
\end{align}
Thus the slope is
\begin{align}
	m = -\frac{1}{2}
\end{align}
\iffalse
\textbf{The direction vector of a line expressed as}
\begin{align}
\implies\vec{m} &= \begin{pmatrix}1 \\ m \\ \end{pmatrix}
\end{align}

\textbf{By solving equation (5) and (6),we get the slope of $\vec{O}$ $\vec{M}$ line}
\begin{align}
        \boxed{m=-0.5}
 \end{align}

\section{Software}
Download the following code using,
\begin{table}[h]
    \centering
    \begin{tabular}{|c|}
    \hline \\
   https://github.com/jyothsna777/jyothsna-fwc.git  \\
         \\
\hline
    \end{tabular}
\end{table}
\\
and execute the code by using command
\begin{center}
\textbf{Python3 lines.py}\\
\end{center}

\section{Conclusion}
Hence the slope of line $\vec{O}$ $\vec{M}$ lineis $\vec{m}$=-0.5

\end{document}
\fi

\item Find the position vector of a point R which divides the line joining two points P and Q whose position vectors are $(2\vec{a}+\vec{b})$ and $(\vec{a}-3\vec{b})$
externally in the ratio 1 : 2. Also, show that P is the mid point of the line segment RQ.\\
	\solution
%		\begin{enumerate}[label=\thesection.\arabic*,ref=\thesection.\theenumi]
\numberwithin{equation}{enumi}
\numberwithin{figure}{enumi}
\numberwithin{table}{enumi}

\item Find the coordinates of the point which divides the join of $(-1,7) \text{ and } (4,-3)$ in the ratio 2:3.
	\\
		\solution
	\input{chapters/10/7/2/1/section.tex}
\item Find the coordinates of the points of trisection of the line segment joining $(4,-1) \text{ and } (-2,3)$.
	\\
		\solution
	\input{chapters/10/7/2/2/section.tex}
\item
	\iffalse
\item To conduct Sports Day activities, in your rectangular shaped school                   
ground ABCD, lines have 
drawn with chalk powder at a                 
distance of 1m each. 100 flower pots have been placed at a distance of 1m 
from each other along AD, as shown 
in Fig. 7.12. Niharika runs $ \frac {1}{4} $th the 
distance AD on the 2nd line and 
posts a green flag. Preet runs $ \frac {1}{5} $th 
the distance AD on the eighth line 
and posts a red flag. What is the 
distance between both the flags? If 
Rashmi has to post a blue flag exactly 
halfway between the line segment 
joining the two flags, where should 
she post her flag?
\begin{figure}[h!]
  \centering
  \includegraphics[width=\columnwidth]{sc.png}
  \caption{}
\label{fig:10/7/12Fig1}
\end{figure}               
\fi
      
\item Find the ratio in which the line segment joining the points $(-3,10) \text{ and } (6,-8)$ $\text{ is divided by } (-1,6)$.
	\\
		\solution
	\input{chapters/10/7/2/4/section.tex}
\item Find the ratio in which the line segment joining $A(1,-5) \text{ and } B(-4,5)$ $\text{is divided by the x-axis}$. Also find the coordinates of the point of division.
\item If $(1,2), (4,y), (x,6), (3,5)$ are the vertices of a parallelogram taken in order, find x and y.
	\\
		\solution
	\input{chapters/10/7/2/6/para1.tex}
\item Find the coordinates of a point A, where AB is the diameter of a circle whose centre is $(2,-3) \text{ and }$ B is $(1,4)$.
	\\
		\solution
	\input{chapters/10/7/2/7/section.tex}
\item If A \text{ and } B are $(-2,-2) \text{ and } (2,-4)$, respectively, find the coordinates of P such that AP= $\frac {3}{7}$AB $\text{ and }$ P lies on the line segment AB.
	\\
		\solution
	\input{chapters/10/7/2/8/section.tex}
\item Find the coordinates of the points which divide the line segment joining $A(-2,2) \text{ and } B(2,8)$ into four equal parts.
	\\
		\solution
	\input{chapters/10/7/2/9/section.tex}
\item Find the area of a rhombus if its vertices are $(3,0), (4,5), (-1,4) \text{ and } (-2,-1)$ taken in order. [$\vec{Hint}$ : Area of rhombus =$\frac {1}{2}$(product of its diagonals)]
	\\
		\solution
	\input{chapters/10/7/2/10/cross.tex}
\item Find the position vector of a point R which divides the line joining two points $\vec{P}$
and $\vec{Q}$ whose position vectors are $\hat{i}+2\hat{j}-\hat{k}$ and $-\hat{i}+\hat{j}+\hat{k}$ respectively, in the
ratio 2 : 1
\begin{enumerate}
    \item  internally
    \item  externally
\end{enumerate}
\solution
		\input{chapters/12/10/2/15/section.tex}
\item Find the position vector of the mid point of the vector joining the points $\vec{P}$(2, 3, 4)
and $\vec{Q}$(4, 1, –2).
\\
\solution
		\input{chapters/12/10/2/16/section.tex}
\item Determine the ratio in which the line $2x+y  - 4=0$ divides the line segment joining the points $\vec{A}(2, - 2)$  and  $\vec{B}(3, 7)$.
\\
\solution
	\input{chapters/10/7/4/1/section.tex}
\item Let $\vec{A}(4, 2), \vec{B}(6, 5)$  and $ \vec{C}(1, 4)$ be the vertices of $\triangle ABC$.
\begin{enumerate}
\item The median from $\vec{A}$ meets $BC$ at $\vec{D}$. Find the coordinates of the point $\vec{D}$.
\item Find the coordinates of the point $\vec{P}$ on $AD$ such that $AP : PD = 2 : 1$.
\item Find the coordinates of points $\vec{Q}$ and $\vec{R}$ on medians $BE$ and $CF$ respectively such that $BQ : QE = 2 : 1$  and  $CR : RF = 2 : 1$.
\item What do you observe?
\item If $\vec{A}, \vec{B}$ and $\vec{C}$  are the vertices of $\triangle ABC$, find the coordinates of the centroid of the triangle.
\end{enumerate}
\solution
	\input{chapters/10/7/4/7/section.tex}
\item Find the slope of a line, which passes through the origin and the mid point of the line segment joining the points $\vec{P}$(0,-4) and $\vec{B}$(8,0).
\label{chapters/11/10/1/5}
\input{chapters/11/10/1/5/matrix.tex}
\item Find the position vector of a point R which divides the line joining two points P and Q whose position vectors are $(2\vec{a}+\vec{b})$ and $(\vec{a}-3\vec{b})$
externally in the ratio 1 : 2. Also, show that P is the mid point of the line segment RQ.\\
	\solution
%		\input{chapters/12/10/5/9/section.tex}

\end{enumerate}



\end{enumerate}


\item
	\iffalse
\item To conduct Sports Day activities, in your rectangular shaped school                   
ground ABCD, lines have 
drawn with chalk powder at a                 
distance of 1m each. 100 flower pots have been placed at a distance of 1m 
from each other along AD, as shown 
in Fig. 7.12. Niharika runs $ \frac {1}{4} $th the 
distance AD on the 2nd line and 
posts a green flag. Preet runs $ \frac {1}{5} $th 
the distance AD on the eighth line 
and posts a red flag. What is the 
distance between both the flags? If 
Rashmi has to post a blue flag exactly 
halfway between the line segment 
joining the two flags, where should 
she post her flag?
\begin{figure}[h!]
  \centering
  \includegraphics[width=\columnwidth]{sc.png}
  \caption{}
\label{fig:10/7/12Fig1}
\end{figure}               
\fi
      
\item Find the ratio in which the line segment joining the points $(-3,10) \text{ and } (6,-8)$ $\text{ is divided by } (-1,6)$.
	\\
		\solution
	\iffalse
\documentclass[12pt]{article}
\usepackage{graphicx}
%\documentclass[journal,12pt,twocolumn]{IEEEtran}
\usepackage[none]{hyphenat}
\usepackage{graphicx}
\usepackage{listings}
\usepackage[english]{babel}
\usepackage{graphicx}
\usepackage{caption} 
\usepackage{hyperref}
\usepackage{booktabs}
\def\inputGnumericTable{}
\usepackage{color}                                            %%
    \usepackage{array}                                            %%
    \usepackage{longtable}                                        %%
    \usepackage{calc}                                             %%
    \usepackage{multirow}                                         %%
    \usepackage{hhline}                                           %%
    \usepackage{ifthen}
\usepackage{array}
\usepackage{amsmath}   % for having text in math mode
\usepackage{listings}
\lstset{
language=tex,
frame=single, 
breaklines=true
}
  
%Following 2 lines were added to remove the blank page at the beginning
\usepackage{atbegshi}% http://ctan.org/pkg/atbegshi
\AtBeginDocument{\AtBeginShipoutNext{\AtBeginShipoutDiscard}}
%
%New macro definitions
\newcommand{\mydet}[1]{\ensuremath{\begin{vmatrix}#1\end{vmatrix}}}
\providecommand{\brak}[1]{\ensuremath{\left(#1\right)}}
\providecommand{\norm}[1]{\left\lVert#1\right\rVert}
\newcommand{\solution}{\noindent \textbf{Solution: }}
\newcommand{\myvec}[1]{\ensuremath{\begin{pmatrix}#1\end{pmatrix}}}
\let\vec\mathbf
\begin{document}
\begin{center}
\title{\textbf{Coordinate Geometry}}
\date{\vspace{-5ex}} %Not to print date automatically
\maketitle
\end{center}
\setcounter{page}{1}
\section*{10$^{th}$ Maths - Chapter 7}
This is Problem-4 from Exercise 7.2
\begin{enumerate}
\item Find the ratio in which the line segement joining the points $\myvec{-3 \\ 10}$ and $\myvec{6\\-8}$ is divided by $\myvec{-1\\6}$.\\
\solution \\
\fi
		The input parameters for this problem are available in Table \eqref{tab:10/7/2/4-1}.
\begin{table}[ht!]
\begin{tabular}{|c|c|p{5cm}|}
\hline
\textbf{Symbol} & \textbf{Value} & \textbf{Description} \\
\hline
$\theta$ & $30\degree$ & $\angle{BAP} = \angle{BAQ}$ \\
\hline
$a$ & $9$ & $AB$ \\
\hline
$c$ & $8$ & $AQ$ \\
\hline
$\vec{e}_1$ & $\myvec{1\\0}$ & Basis vector \\
\hline
\end{tabular}

\caption{}
\label{tab:10/7/2/4-1} 
\end{table}
Using section formula,
\begin{align}
         \vec{R} &=\frac{\vec{Q}+n\vec{P}}{1+n}\label{eq:chapters/10/7/2/4/1}
\end{align}
Substituting the values of $\vec{P},\vec{Q}$ and $\vec{R}$ in \eqref{eq:chapters/10/7/2/4/1}
\begin{align}
         \myvec{-1\\6} &=\frac{{\myvec{-3\\10}+n\myvec{6\\-8}}}{1+n}\\
 &=\frac{1}{1+n}\brak{{\myvec{-3\\10}+n\myvec{6\\-8}}} \\
 &=\frac{1}{1+n}\myvec{-3+6n\\10-8n} \label{eq:chapters/10/7/2/4/4}
\end{align}
Simplifying \eqref{eq:chapters/10/7/2/4/4} yeilds,
\begin{align}
          -1 &=\frac{-3+6n}{1+n}\\
\implies          n &=\frac{2}{7}
\end{align}
Also,
\begin{align}
          6 &=\frac{10-8n}{1+n}\\
    \implies      n &=\frac{2}{7}
\end{align}
Hence the desired ratio is $\dfrac{2}{7}$.  
\begin{figure}[!h]
 \begin{center}
  \includegraphics[width=\columnwidth]{chapters/10/7/2/4/figs/fig.png}
 \end{center}
\caption{}
\label{fig:10/7/2/4Fig1}
\end{figure}

\item Find the ratio in which the line segment joining $A(1,-5) \text{ and } B(-4,5)$ $\text{is divided by the x-axis}$. Also find the coordinates of the point of division.
\item If $(1,2), (4,y), (x,6), (3,5)$ are the vertices of a parallelogram taken in order, find x and y.
	\\
		\solution
	\iffalse
\documentclass[12pt]{article}
\usepackage{graphicx}
%\documentclass[journal,12pt,twocolumn]{IEEEtran}
\def\inputGnumericTable{}
\usepackage{color}                                            %%
    \usepackage{array}                                            %%
    \usepackage{longtable}                                        %%
    \usepackage{calc}                                             %%
    \usepackage{multirow}                                         %%
    \usepackage{hhline}                                           %%
    \usepackage{ifthen}
\usepackage[none]{hyphenat}
\usepackage{graphicx}
\usepackage{listings}
\usepackage[english]{babel}
\usepackage{graphicx}
\usepackage{caption} 
\usepackage{hyperref}
\usepackage{booktabs}
\usepackage{array}
\usepackage{amsmath}   % for having text in math mode
\usepackage{listings}
\lstset{
  frame=single,
  breaklines=true
}
  
%Following 2 lines were added to remove the blank page at the beginning
\usepackage{atbegshi}% http://ctan.org/pkg/atbegshi
\AtBeginDocument{\AtBeginShipoutNext{\AtBeginShipoutDiscard}}
%


%New macro definitions
\newcommand{\mydet}[1]{\ensuremath{\begin{vmatrix}#1\end{vmatrix}}}
\providecommand{\brak}[1]{\ensuremath{\left(#1\right)}}
\providecommand{\norm}[1]{\left\lVert#1\right\rVert}
\newcommand{\solution}{\noindent \textbf{Solution: }}
\newcommand{\myvec}[1]{\ensuremath{\begin{pmatrix}#1\end{pmatrix}}}
\let\vec\mathbf

\begin{document}

\begin{center}
\title{\textbf{Properties of Parallelegram}}
\date{\vspace{-5ex}} %Not to print date automatically
\maketitle
\end{center}

\setcounter{page}{1}

\section{10$^{th}$ Maths - Chapter 7}

This is Problem-6 from Exercise 7.2

\begin{enumerate}
\item If $\vec{A}(1, 2),\vec{B}(4, x),\vec{C}(y, 6) \text{and } \vec{D}(3, 5)$ are the vertices of a parallelogram taken in order,find x and y.
\end{enumerate}
\fi

The input parameters for this problem are available in
\ref{table:chapters/10/7/2/6/tables/}.	
\begin{table}[!ht]
	\centering
	\begin{tabular}{|c|c|p{5cm}|}
\hline
\textbf{Symbol} & \textbf{Value} & \textbf{Description} \\
\hline
$\theta$ & $30\degree$ & $\angle{BAP} = \angle{BAQ}$ \\
\hline
$a$ & $9$ & $AB$ \\
\hline
$c$ & $8$ & $AQ$ \\
\hline
$\vec{e}_1$ & $\myvec{1\\0}$ & Basis vector \\
\hline
\end{tabular}

\caption{}
\label{table:chapters/10/7/2/6/tables/}	
\end{table}
From the given information,
\begin{align}
  \label{eq:chapters/10/7/2/6/tables/det2f}
	\vec{B}-\vec{A} &= \myvec{4 \\y } - \myvec{1 \\2 }  = \myvec{3 \\y-2 }\\
	\vec{C}-\vec{D} &= \myvec{x \\6 } - \myvec{3 \\5 }  = \myvec{x-3 \\1}
\end{align}
Since $ABCD$ is a parallellogram,
\begin{align}
	\myvec{3\\y-2}&=\myvec{x-3\\1}\\
	\implies x&=6 ,y=3
\end{align}
Fig. \ref{fig:chapters/10/7/2/6/Fig3}
provides a verification.
\begin{figure}[h!]
	\begin{center}
  \includegraphics[width=\columnwidth]{chapters/10/7/2/6/figs/para.pdf}
	\end{center}
\caption{}
\label{fig:chapters/10/7/2/6/Fig3}
\end{figure}


\item Find the coordinates of a point A, where AB is the diameter of a circle whose centre is $(2,-3) \text{ and }$ B is $(1,4)$.
	\\
		\solution
	\iffalse
\documentclass[12pt]{article}
\usepackage{graphicx}
\usepackage{amsmath}
\usepackage{mathtools}
\usepackage{gensymb}

\newcommand{\mydet}[1]{\ensuremath{\begin{vmatrix}#1\end{vmatrix}}}
\providecommand{\brak}[1]{\ensuremath{\left(#1\right)}}
\providecommand{\norm}[1]{\left\lVert#1\right\rVert}
\newcommand{\solution}{\noindent \textbf{Solution: }}
\newcommand{\myvec}[1]{\ensuremath{\begin{pmatrix}#1\end{pmatrix}}}
\let\vec\mathbf

\begin{document}
\begin{center}
\section*{CHAPTER 7 - COORDINATE GEOMETRY}

\end{center}
\section*{Excercise 7.2}

Q7.Find the coordinates of point $\vec{A}$, where AB is the diameter of a circle where the center is (2,-3) and $\vec{B}$ is the point (1,4):

\solution
\begin{enumerate}
\item The coordinates $\vec{B}$ and center $\vec{C}$ are given, where:
	\fi
	Let
	\begin{align}
	\vec{B} = \myvec{
		1\\
	    4\\
		},
	\vec{C} = \myvec{
	    2\\
	   -3\\
		}
	\end{align}
	\iffalse
Let us assume the coordinates of $\vec{A}$. Now, $\vec{C}$ is the center which is midpoint of line AB and $\vec{B}$ is one of the coordinate of diameter AB of a circle.
	\fi	
Hence,	
	\begin{align}
	\vec{C} &= \frac{\vec{A+B}}{2} \\
\implies	2\vec{C} &= \vec{A}+\vec{B} \\
		\text{or, }	\vec{A} &= 2\vec{C}-\vec{B} \\
	 &= \myvec{3\\-10\\}	
	\end{align}       
	See Fig. 
\ref{fig:chapters/10/7/2/7Fig}.
\begin{figure}[!h]
\begin{center}	
	\includegraphics[width=\columnwidth]{chapters/10/7/2/7/figs/Vector1.png}
\end{center}
\caption{}
\label{fig:chapters/10/7/2/7Fig}
\end{figure}
	

\item If A \text{ and } B are $(-2,-2) \text{ and } (2,-4)$, respectively, find the coordinates of P such that AP= $\frac {3}{7}$AB $\text{ and }$ P lies on the line segment AB.
	\\
		\solution
	\iffalse
\documentclass[journal,10pt,twocolumn]{article}
\usepackage{graphicx}
\usepackage[none]{hyphenat}
\usepackage{graphicx}
\usepackage{listings}
\usepackage[english]{babel}
\usepackage{graphicx}
\usepackage{caption} 
\usepackage{booktabs}
\usepackage{array}
\usepackage{amssymb} % for \because
\usepackage{amsmath}   % for having text in math mode
\usepackage{extarrows} % for Row operations arrows
\usepackage{listings}
\usepackage[utf8]{inputenc}
\lstset{
  frame=single,
  breaklines=true
}
\usepackage{hyperref}
  
%Following 2 lines were added to remove the blank page at the beginning
\usepackage{atbegshi}% http://ctan.org/pkg/atbegshi
\AtBeginDocument{\AtBeginShipoutNext{\AtBeginShipoutDiscard}}


%New macro definitions
\newcommand{\mydet}[1]{\ensuremath{\begin{vmatrix}#1\end{vmatrix}}}
\providecommand{\brak}[1]{\ensuremath{\left(#1\right)}}
\newcommand{\solution}{\noindent \textbf{Solution: }}
\newcommand{\myvec}[1]{\ensuremath{\begin{pmatrix}#1\end{pmatrix}}}
\providecommand{\norm}[1]{\left\lVert#1\right\rVert}
\providecommand{\abs}[1]{\left\vert#1\right\vert}
\let\vec\mathbf

\begin{document}

\begin{center}
\title{\textbf{VECTORS}}
\date{\vspace{-5ex}} %Not to print date automatically
\maketitle
\end{center}

\section{10$^{th}$ Maths - EXERCISE-7.2}

\begin{enumerate}
\item If A and B are $(– 2, – 2)\text{ and }(2, – 4)$, respectively, find the coordinates of P such that $AP =\frac{3}{7}AB$ and P lies on the line segment AB. 

\section{SOLUTION}
Given points are
\begin{align}
\vec{A}=\myvec{-2\\ -2} ,
\vec{B}=\myvec{2\\ -4}
\end{align}
The equation of the formula is
\fi
Using section formula, 
\begin{align}
\vec{P}&=\frac{\vec{A}+n\vec{B}}{1+n}
\end{align}
where
\begin{align}
	n =\frac{3}{4}
\end{align}
Thus,
\begin{align}
\vec{P}&=\frac{1}{1+\frac{3}{4}}\brak{\myvec{-2\\-2}+\frac{3}{4}\myvec{2\\-4}}\\
&=\myvec{\frac{-2}{7}\\[1pt] \frac{-20}{7}}
\end{align}
See Fig. 
   \ref{fig:chapters/10/7/2/8/vec.png}
\begin{figure}
   \centering 
 \includegraphics[width=\columnwidth]{chapters/10/7/2/8/figs/vec.png}
   \caption{}
   \label{fig:chapters/10/7/2/8/vec.png}
   \end{figure}

\item Find the coordinates of the points which divide the line segment joining $A(-2,2) \text{ and } B(2,8)$ into four equal parts.
	\\
		\solution
	\begin{enumerate}[label=\thesection.\arabic*,ref=\thesection.\theenumi]
\numberwithin{equation}{enumi}
\numberwithin{figure}{enumi}
\numberwithin{table}{enumi}

\item Find the coordinates of the point which divides the join of $(-1,7) \text{ and } (4,-3)$ in the ratio 2:3.
	\\
		\solution
	\iffalse
\documentclass[12pt]{article}
\usepackage{graphicx}
\usepackage{amsmath}
\usepackage{mathtools}
\usepackage{gensymb}

\newcommand{\mydet}[1]{\ensuremath{\begin{vmatrix}#1\end{vmatrix}}}
\providecommand{\brak}[1]{\ensuremath{\left(#1\right)}}
\providecommand{\norm}[1]{\left\lVert#1\right\rVert}
\newcommand{\solution}{\noindent \textbf{Solution: }}
\newcommand{\myvec}[1]{\ensuremath{\begin{pmatrix}#1\end{pmatrix}}}
\let\vec\mathbf

\begin{document}
\begin{center}
\textbf\large{CHAPTER-7 \\ COORDINATE GEOMETRY}
\end{center}
\section*{Excercise 7.2}

1. Find the coordinates of the point which divides the join $\vec(-1,7) \text{ and } \vec(4,-3)$ in the ratio 2:3 :
\\
\\
\solution\\		
\fi
The coordinates and ratio are given as
\begin{align}
\vec{P}=\myvec{-1\\7\\},
\vec{Q}=\myvec{4\\-3\\},
n=\frac{3}{2}
\end{align}
Using section formula
\begin{align}
\vec{R}&=\frac{\vec{Q}+n\vec{P}}{1+n}\\
&=\frac{1}{1+\frac{3}{2}}  \myvec{\myvec{
4\\
-3\\
}
  +
   \frac{3}{2}\myvec{
-1\\
7\\
}}\\
&=\myvec{
1\\
3
}
\end{align}
See Fig. 
\ref{fig:chapters/10/7/2/1/Fig}
\begin{figure}[!h]
\begin{center}
   \includegraphics[width=\columnwidth]{chapters/10/7/2/1/figs/linefig.png}
\end{center}
\caption{}
\label{fig:chapters/10/7/2/1/Fig}
\end{figure}


\item Find the coordinates of the points of trisection of the line segment joining $(4,-1) \text{ and } (-2,3)$.
	\\
		\solution
	\begin{enumerate}[label=\thesection.\arabic*,ref=\thesection.\theenumi]
\numberwithin{equation}{enumi}
\numberwithin{figure}{enumi}
\numberwithin{table}{enumi}

\item Find the coordinates of the point which divides the join of $(-1,7) \text{ and } (4,-3)$ in the ratio 2:3.
	\\
		\solution
	\input{chapters/10/7/2/1/section.tex}
\item Find the coordinates of the points of trisection of the line segment joining $(4,-1) \text{ and } (-2,3)$.
	\\
		\solution
	\input{chapters/10/7/2/2/section.tex}
\item
	\iffalse
\item To conduct Sports Day activities, in your rectangular shaped school                   
ground ABCD, lines have 
drawn with chalk powder at a                 
distance of 1m each. 100 flower pots have been placed at a distance of 1m 
from each other along AD, as shown 
in Fig. 7.12. Niharika runs $ \frac {1}{4} $th the 
distance AD on the 2nd line and 
posts a green flag. Preet runs $ \frac {1}{5} $th 
the distance AD on the eighth line 
and posts a red flag. What is the 
distance between both the flags? If 
Rashmi has to post a blue flag exactly 
halfway between the line segment 
joining the two flags, where should 
she post her flag?
\begin{figure}[h!]
  \centering
  \includegraphics[width=\columnwidth]{sc.png}
  \caption{}
\label{fig:10/7/12Fig1}
\end{figure}               
\fi
      
\item Find the ratio in which the line segment joining the points $(-3,10) \text{ and } (6,-8)$ $\text{ is divided by } (-1,6)$.
	\\
		\solution
	\input{chapters/10/7/2/4/section.tex}
\item Find the ratio in which the line segment joining $A(1,-5) \text{ and } B(-4,5)$ $\text{is divided by the x-axis}$. Also find the coordinates of the point of division.
\item If $(1,2), (4,y), (x,6), (3,5)$ are the vertices of a parallelogram taken in order, find x and y.
	\\
		\solution
	\input{chapters/10/7/2/6/para1.tex}
\item Find the coordinates of a point A, where AB is the diameter of a circle whose centre is $(2,-3) \text{ and }$ B is $(1,4)$.
	\\
		\solution
	\input{chapters/10/7/2/7/section.tex}
\item If A \text{ and } B are $(-2,-2) \text{ and } (2,-4)$, respectively, find the coordinates of P such that AP= $\frac {3}{7}$AB $\text{ and }$ P lies on the line segment AB.
	\\
		\solution
	\input{chapters/10/7/2/8/section.tex}
\item Find the coordinates of the points which divide the line segment joining $A(-2,2) \text{ and } B(2,8)$ into four equal parts.
	\\
		\solution
	\input{chapters/10/7/2/9/section.tex}
\item Find the area of a rhombus if its vertices are $(3,0), (4,5), (-1,4) \text{ and } (-2,-1)$ taken in order. [$\vec{Hint}$ : Area of rhombus =$\frac {1}{2}$(product of its diagonals)]
	\\
		\solution
	\input{chapters/10/7/2/10/cross.tex}
\item Find the position vector of a point R which divides the line joining two points $\vec{P}$
and $\vec{Q}$ whose position vectors are $\hat{i}+2\hat{j}-\hat{k}$ and $-\hat{i}+\hat{j}+\hat{k}$ respectively, in the
ratio 2 : 1
\begin{enumerate}
    \item  internally
    \item  externally
\end{enumerate}
\solution
		\input{chapters/12/10/2/15/section.tex}
\item Find the position vector of the mid point of the vector joining the points $\vec{P}$(2, 3, 4)
and $\vec{Q}$(4, 1, –2).
\\
\solution
		\input{chapters/12/10/2/16/section.tex}
\item Determine the ratio in which the line $2x+y  - 4=0$ divides the line segment joining the points $\vec{A}(2, - 2)$  and  $\vec{B}(3, 7)$.
\\
\solution
	\input{chapters/10/7/4/1/section.tex}
\item Let $\vec{A}(4, 2), \vec{B}(6, 5)$  and $ \vec{C}(1, 4)$ be the vertices of $\triangle ABC$.
\begin{enumerate}
\item The median from $\vec{A}$ meets $BC$ at $\vec{D}$. Find the coordinates of the point $\vec{D}$.
\item Find the coordinates of the point $\vec{P}$ on $AD$ such that $AP : PD = 2 : 1$.
\item Find the coordinates of points $\vec{Q}$ and $\vec{R}$ on medians $BE$ and $CF$ respectively such that $BQ : QE = 2 : 1$  and  $CR : RF = 2 : 1$.
\item What do you observe?
\item If $\vec{A}, \vec{B}$ and $\vec{C}$  are the vertices of $\triangle ABC$, find the coordinates of the centroid of the triangle.
\end{enumerate}
\solution
	\input{chapters/10/7/4/7/section.tex}
\item Find the slope of a line, which passes through the origin and the mid point of the line segment joining the points $\vec{P}$(0,-4) and $\vec{B}$(8,0).
\label{chapters/11/10/1/5}
\input{chapters/11/10/1/5/matrix.tex}
\item Find the position vector of a point R which divides the line joining two points P and Q whose position vectors are $(2\vec{a}+\vec{b})$ and $(\vec{a}-3\vec{b})$
externally in the ratio 1 : 2. Also, show that P is the mid point of the line segment RQ.\\
	\solution
%		\input{chapters/12/10/5/9/section.tex}

\end{enumerate}


\item
	\iffalse
\item To conduct Sports Day activities, in your rectangular shaped school                   
ground ABCD, lines have 
drawn with chalk powder at a                 
distance of 1m each. 100 flower pots have been placed at a distance of 1m 
from each other along AD, as shown 
in Fig. 7.12. Niharika runs $ \frac {1}{4} $th the 
distance AD on the 2nd line and 
posts a green flag. Preet runs $ \frac {1}{5} $th 
the distance AD on the eighth line 
and posts a red flag. What is the 
distance between both the flags? If 
Rashmi has to post a blue flag exactly 
halfway between the line segment 
joining the two flags, where should 
she post her flag?
\begin{figure}[h!]
  \centering
  \includegraphics[width=\columnwidth]{sc.png}
  \caption{}
\label{fig:10/7/12Fig1}
\end{figure}               
\fi
      
\item Find the ratio in which the line segment joining the points $(-3,10) \text{ and } (6,-8)$ $\text{ is divided by } (-1,6)$.
	\\
		\solution
	\iffalse
\documentclass[12pt]{article}
\usepackage{graphicx}
%\documentclass[journal,12pt,twocolumn]{IEEEtran}
\usepackage[none]{hyphenat}
\usepackage{graphicx}
\usepackage{listings}
\usepackage[english]{babel}
\usepackage{graphicx}
\usepackage{caption} 
\usepackage{hyperref}
\usepackage{booktabs}
\def\inputGnumericTable{}
\usepackage{color}                                            %%
    \usepackage{array}                                            %%
    \usepackage{longtable}                                        %%
    \usepackage{calc}                                             %%
    \usepackage{multirow}                                         %%
    \usepackage{hhline}                                           %%
    \usepackage{ifthen}
\usepackage{array}
\usepackage{amsmath}   % for having text in math mode
\usepackage{listings}
\lstset{
language=tex,
frame=single, 
breaklines=true
}
  
%Following 2 lines were added to remove the blank page at the beginning
\usepackage{atbegshi}% http://ctan.org/pkg/atbegshi
\AtBeginDocument{\AtBeginShipoutNext{\AtBeginShipoutDiscard}}
%
%New macro definitions
\newcommand{\mydet}[1]{\ensuremath{\begin{vmatrix}#1\end{vmatrix}}}
\providecommand{\brak}[1]{\ensuremath{\left(#1\right)}}
\providecommand{\norm}[1]{\left\lVert#1\right\rVert}
\newcommand{\solution}{\noindent \textbf{Solution: }}
\newcommand{\myvec}[1]{\ensuremath{\begin{pmatrix}#1\end{pmatrix}}}
\let\vec\mathbf
\begin{document}
\begin{center}
\title{\textbf{Coordinate Geometry}}
\date{\vspace{-5ex}} %Not to print date automatically
\maketitle
\end{center}
\setcounter{page}{1}
\section*{10$^{th}$ Maths - Chapter 7}
This is Problem-4 from Exercise 7.2
\begin{enumerate}
\item Find the ratio in which the line segement joining the points $\myvec{-3 \\ 10}$ and $\myvec{6\\-8}$ is divided by $\myvec{-1\\6}$.\\
\solution \\
\fi
		The input parameters for this problem are available in Table \eqref{tab:10/7/2/4-1}.
\begin{table}[ht!]
\input{chapters/10/7/2/4/tables/table.tex}
\caption{}
\label{tab:10/7/2/4-1} 
\end{table}
Using section formula,
\begin{align}
         \vec{R} &=\frac{\vec{Q}+n\vec{P}}{1+n}\label{eq:chapters/10/7/2/4/1}
\end{align}
Substituting the values of $\vec{P},\vec{Q}$ and $\vec{R}$ in \eqref{eq:chapters/10/7/2/4/1}
\begin{align}
         \myvec{-1\\6} &=\frac{{\myvec{-3\\10}+n\myvec{6\\-8}}}{1+n}\\
 &=\frac{1}{1+n}\brak{{\myvec{-3\\10}+n\myvec{6\\-8}}} \\
 &=\frac{1}{1+n}\myvec{-3+6n\\10-8n} \label{eq:chapters/10/7/2/4/4}
\end{align}
Simplifying \eqref{eq:chapters/10/7/2/4/4} yeilds,
\begin{align}
          -1 &=\frac{-3+6n}{1+n}\\
\implies          n &=\frac{2}{7}
\end{align}
Also,
\begin{align}
          6 &=\frac{10-8n}{1+n}\\
    \implies      n &=\frac{2}{7}
\end{align}
Hence the desired ratio is $\dfrac{2}{7}$.  
\begin{figure}[!h]
 \begin{center}
  \includegraphics[width=\columnwidth]{chapters/10/7/2/4/figs/fig.png}
 \end{center}
\caption{}
\label{fig:10/7/2/4Fig1}
\end{figure}

\item Find the ratio in which the line segment joining $A(1,-5) \text{ and } B(-4,5)$ $\text{is divided by the x-axis}$. Also find the coordinates of the point of division.
\item If $(1,2), (4,y), (x,6), (3,5)$ are the vertices of a parallelogram taken in order, find x and y.
	\\
		\solution
	\iffalse
\documentclass[12pt]{article}
\usepackage{graphicx}
%\documentclass[journal,12pt,twocolumn]{IEEEtran}
\def\inputGnumericTable{}
\usepackage{color}                                            %%
    \usepackage{array}                                            %%
    \usepackage{longtable}                                        %%
    \usepackage{calc}                                             %%
    \usepackage{multirow}                                         %%
    \usepackage{hhline}                                           %%
    \usepackage{ifthen}
\usepackage[none]{hyphenat}
\usepackage{graphicx}
\usepackage{listings}
\usepackage[english]{babel}
\usepackage{graphicx}
\usepackage{caption} 
\usepackage{hyperref}
\usepackage{booktabs}
\usepackage{array}
\usepackage{amsmath}   % for having text in math mode
\usepackage{listings}
\lstset{
  frame=single,
  breaklines=true
}
  
%Following 2 lines were added to remove the blank page at the beginning
\usepackage{atbegshi}% http://ctan.org/pkg/atbegshi
\AtBeginDocument{\AtBeginShipoutNext{\AtBeginShipoutDiscard}}
%


%New macro definitions
\newcommand{\mydet}[1]{\ensuremath{\begin{vmatrix}#1\end{vmatrix}}}
\providecommand{\brak}[1]{\ensuremath{\left(#1\right)}}
\providecommand{\norm}[1]{\left\lVert#1\right\rVert}
\newcommand{\solution}{\noindent \textbf{Solution: }}
\newcommand{\myvec}[1]{\ensuremath{\begin{pmatrix}#1\end{pmatrix}}}
\let\vec\mathbf

\begin{document}

\begin{center}
\title{\textbf{Properties of Parallelegram}}
\date{\vspace{-5ex}} %Not to print date automatically
\maketitle
\end{center}

\setcounter{page}{1}

\section{10$^{th}$ Maths - Chapter 7}

This is Problem-6 from Exercise 7.2

\begin{enumerate}
\item If $\vec{A}(1, 2),\vec{B}(4, x),\vec{C}(y, 6) \text{and } \vec{D}(3, 5)$ are the vertices of a parallelogram taken in order,find x and y.
\end{enumerate}
\fi

The input parameters for this problem are available in
\ref{table:chapters/10/7/2/6/tables/}.	
\begin{table}[!ht]
	\centering
	\input{chapters/10/7/2/6/tables/table.tex}
\caption{}
\label{table:chapters/10/7/2/6/tables/}	
\end{table}
From the given information,
\begin{align}
  \label{eq:chapters/10/7/2/6/tables/det2f}
	\vec{B}-\vec{A} &= \myvec{4 \\y } - \myvec{1 \\2 }  = \myvec{3 \\y-2 }\\
	\vec{C}-\vec{D} &= \myvec{x \\6 } - \myvec{3 \\5 }  = \myvec{x-3 \\1}
\end{align}
Since $ABCD$ is a parallellogram,
\begin{align}
	\myvec{3\\y-2}&=\myvec{x-3\\1}\\
	\implies x&=6 ,y=3
\end{align}
Fig. \ref{fig:chapters/10/7/2/6/Fig3}
provides a verification.
\begin{figure}[h!]
	\begin{center}
  \includegraphics[width=\columnwidth]{chapters/10/7/2/6/figs/para.pdf}
	\end{center}
\caption{}
\label{fig:chapters/10/7/2/6/Fig3}
\end{figure}


\item Find the coordinates of a point A, where AB is the diameter of a circle whose centre is $(2,-3) \text{ and }$ B is $(1,4)$.
	\\
		\solution
	\iffalse
\documentclass[12pt]{article}
\usepackage{graphicx}
\usepackage{amsmath}
\usepackage{mathtools}
\usepackage{gensymb}

\newcommand{\mydet}[1]{\ensuremath{\begin{vmatrix}#1\end{vmatrix}}}
\providecommand{\brak}[1]{\ensuremath{\left(#1\right)}}
\providecommand{\norm}[1]{\left\lVert#1\right\rVert}
\newcommand{\solution}{\noindent \textbf{Solution: }}
\newcommand{\myvec}[1]{\ensuremath{\begin{pmatrix}#1\end{pmatrix}}}
\let\vec\mathbf

\begin{document}
\begin{center}
\section*{CHAPTER 7 - COORDINATE GEOMETRY}

\end{center}
\section*{Excercise 7.2}

Q7.Find the coordinates of point $\vec{A}$, where AB is the diameter of a circle where the center is (2,-3) and $\vec{B}$ is the point (1,4):

\solution
\begin{enumerate}
\item The coordinates $\vec{B}$ and center $\vec{C}$ are given, where:
	\fi
	Let
	\begin{align}
	\vec{B} = \myvec{
		1\\
	    4\\
		},
	\vec{C} = \myvec{
	    2\\
	   -3\\
		}
	\end{align}
	\iffalse
Let us assume the coordinates of $\vec{A}$. Now, $\vec{C}$ is the center which is midpoint of line AB and $\vec{B}$ is one of the coordinate of diameter AB of a circle.
	\fi	
Hence,	
	\begin{align}
	\vec{C} &= \frac{\vec{A+B}}{2} \\
\implies	2\vec{C} &= \vec{A}+\vec{B} \\
		\text{or, }	\vec{A} &= 2\vec{C}-\vec{B} \\
	 &= \myvec{3\\-10\\}	
	\end{align}       
	See Fig. 
\ref{fig:chapters/10/7/2/7Fig}.
\begin{figure}[!h]
\begin{center}	
	\includegraphics[width=\columnwidth]{chapters/10/7/2/7/figs/Vector1.png}
\end{center}
\caption{}
\label{fig:chapters/10/7/2/7Fig}
\end{figure}
	

\item If A \text{ and } B are $(-2,-2) \text{ and } (2,-4)$, respectively, find the coordinates of P such that AP= $\frac {3}{7}$AB $\text{ and }$ P lies on the line segment AB.
	\\
		\solution
	\iffalse
\documentclass[journal,10pt,twocolumn]{article}
\usepackage{graphicx}
\usepackage[none]{hyphenat}
\usepackage{graphicx}
\usepackage{listings}
\usepackage[english]{babel}
\usepackage{graphicx}
\usepackage{caption} 
\usepackage{booktabs}
\usepackage{array}
\usepackage{amssymb} % for \because
\usepackage{amsmath}   % for having text in math mode
\usepackage{extarrows} % for Row operations arrows
\usepackage{listings}
\usepackage[utf8]{inputenc}
\lstset{
  frame=single,
  breaklines=true
}
\usepackage{hyperref}
  
%Following 2 lines were added to remove the blank page at the beginning
\usepackage{atbegshi}% http://ctan.org/pkg/atbegshi
\AtBeginDocument{\AtBeginShipoutNext{\AtBeginShipoutDiscard}}


%New macro definitions
\newcommand{\mydet}[1]{\ensuremath{\begin{vmatrix}#1\end{vmatrix}}}
\providecommand{\brak}[1]{\ensuremath{\left(#1\right)}}
\newcommand{\solution}{\noindent \textbf{Solution: }}
\newcommand{\myvec}[1]{\ensuremath{\begin{pmatrix}#1\end{pmatrix}}}
\providecommand{\norm}[1]{\left\lVert#1\right\rVert}
\providecommand{\abs}[1]{\left\vert#1\right\vert}
\let\vec\mathbf

\begin{document}

\begin{center}
\title{\textbf{VECTORS}}
\date{\vspace{-5ex}} %Not to print date automatically
\maketitle
\end{center}

\section{10$^{th}$ Maths - EXERCISE-7.2}

\begin{enumerate}
\item If A and B are $(– 2, – 2)\text{ and }(2, – 4)$, respectively, find the coordinates of P such that $AP =\frac{3}{7}AB$ and P lies on the line segment AB. 

\section{SOLUTION}
Given points are
\begin{align}
\vec{A}=\myvec{-2\\ -2} ,
\vec{B}=\myvec{2\\ -4}
\end{align}
The equation of the formula is
\fi
Using section formula, 
\begin{align}
\vec{P}&=\frac{\vec{A}+n\vec{B}}{1+n}
\end{align}
where
\begin{align}
	n =\frac{3}{4}
\end{align}
Thus,
\begin{align}
\vec{P}&=\frac{1}{1+\frac{3}{4}}\brak{\myvec{-2\\-2}+\frac{3}{4}\myvec{2\\-4}}\\
&=\myvec{\frac{-2}{7}\\[1pt] \frac{-20}{7}}
\end{align}
See Fig. 
   \ref{fig:chapters/10/7/2/8/vec.png}
\begin{figure}
   \centering 
 \includegraphics[width=\columnwidth]{chapters/10/7/2/8/figs/vec.png}
   \caption{}
   \label{fig:chapters/10/7/2/8/vec.png}
   \end{figure}

\item Find the coordinates of the points which divide the line segment joining $A(-2,2) \text{ and } B(2,8)$ into four equal parts.
	\\
		\solution
	\begin{enumerate}[label=\thesection.\arabic*,ref=\thesection.\theenumi]
\numberwithin{equation}{enumi}
\numberwithin{figure}{enumi}
\numberwithin{table}{enumi}

\item Find the coordinates of the point which divides the join of $(-1,7) \text{ and } (4,-3)$ in the ratio 2:3.
	\\
		\solution
	\input{chapters/10/7/2/1/section.tex}
\item Find the coordinates of the points of trisection of the line segment joining $(4,-1) \text{ and } (-2,3)$.
	\\
		\solution
	\input{chapters/10/7/2/2/section.tex}
\item
	\iffalse
\item To conduct Sports Day activities, in your rectangular shaped school                   
ground ABCD, lines have 
drawn with chalk powder at a                 
distance of 1m each. 100 flower pots have been placed at a distance of 1m 
from each other along AD, as shown 
in Fig. 7.12. Niharika runs $ \frac {1}{4} $th the 
distance AD on the 2nd line and 
posts a green flag. Preet runs $ \frac {1}{5} $th 
the distance AD on the eighth line 
and posts a red flag. What is the 
distance between both the flags? If 
Rashmi has to post a blue flag exactly 
halfway between the line segment 
joining the two flags, where should 
she post her flag?
\begin{figure}[h!]
  \centering
  \includegraphics[width=\columnwidth]{sc.png}
  \caption{}
\label{fig:10/7/12Fig1}
\end{figure}               
\fi
      
\item Find the ratio in which the line segment joining the points $(-3,10) \text{ and } (6,-8)$ $\text{ is divided by } (-1,6)$.
	\\
		\solution
	\input{chapters/10/7/2/4/section.tex}
\item Find the ratio in which the line segment joining $A(1,-5) \text{ and } B(-4,5)$ $\text{is divided by the x-axis}$. Also find the coordinates of the point of division.
\item If $(1,2), (4,y), (x,6), (3,5)$ are the vertices of a parallelogram taken in order, find x and y.
	\\
		\solution
	\input{chapters/10/7/2/6/para1.tex}
\item Find the coordinates of a point A, where AB is the diameter of a circle whose centre is $(2,-3) \text{ and }$ B is $(1,4)$.
	\\
		\solution
	\input{chapters/10/7/2/7/section.tex}
\item If A \text{ and } B are $(-2,-2) \text{ and } (2,-4)$, respectively, find the coordinates of P such that AP= $\frac {3}{7}$AB $\text{ and }$ P lies on the line segment AB.
	\\
		\solution
	\input{chapters/10/7/2/8/section.tex}
\item Find the coordinates of the points which divide the line segment joining $A(-2,2) \text{ and } B(2,8)$ into four equal parts.
	\\
		\solution
	\input{chapters/10/7/2/9/section.tex}
\item Find the area of a rhombus if its vertices are $(3,0), (4,5), (-1,4) \text{ and } (-2,-1)$ taken in order. [$\vec{Hint}$ : Area of rhombus =$\frac {1}{2}$(product of its diagonals)]
	\\
		\solution
	\input{chapters/10/7/2/10/cross.tex}
\item Find the position vector of a point R which divides the line joining two points $\vec{P}$
and $\vec{Q}$ whose position vectors are $\hat{i}+2\hat{j}-\hat{k}$ and $-\hat{i}+\hat{j}+\hat{k}$ respectively, in the
ratio 2 : 1
\begin{enumerate}
    \item  internally
    \item  externally
\end{enumerate}
\solution
		\input{chapters/12/10/2/15/section.tex}
\item Find the position vector of the mid point of the vector joining the points $\vec{P}$(2, 3, 4)
and $\vec{Q}$(4, 1, –2).
\\
\solution
		\input{chapters/12/10/2/16/section.tex}
\item Determine the ratio in which the line $2x+y  - 4=0$ divides the line segment joining the points $\vec{A}(2, - 2)$  and  $\vec{B}(3, 7)$.
\\
\solution
	\input{chapters/10/7/4/1/section.tex}
\item Let $\vec{A}(4, 2), \vec{B}(6, 5)$  and $ \vec{C}(1, 4)$ be the vertices of $\triangle ABC$.
\begin{enumerate}
\item The median from $\vec{A}$ meets $BC$ at $\vec{D}$. Find the coordinates of the point $\vec{D}$.
\item Find the coordinates of the point $\vec{P}$ on $AD$ such that $AP : PD = 2 : 1$.
\item Find the coordinates of points $\vec{Q}$ and $\vec{R}$ on medians $BE$ and $CF$ respectively such that $BQ : QE = 2 : 1$  and  $CR : RF = 2 : 1$.
\item What do you observe?
\item If $\vec{A}, \vec{B}$ and $\vec{C}$  are the vertices of $\triangle ABC$, find the coordinates of the centroid of the triangle.
\end{enumerate}
\solution
	\input{chapters/10/7/4/7/section.tex}
\item Find the slope of a line, which passes through the origin and the mid point of the line segment joining the points $\vec{P}$(0,-4) and $\vec{B}$(8,0).
\label{chapters/11/10/1/5}
\input{chapters/11/10/1/5/matrix.tex}
\item Find the position vector of a point R which divides the line joining two points P and Q whose position vectors are $(2\vec{a}+\vec{b})$ and $(\vec{a}-3\vec{b})$
externally in the ratio 1 : 2. Also, show that P is the mid point of the line segment RQ.\\
	\solution
%		\input{chapters/12/10/5/9/section.tex}

\end{enumerate}


\item Find the area of a rhombus if its vertices are $(3,0), (4,5), (-1,4) \text{ and } (-2,-1)$ taken in order. [$\vec{Hint}$ : Area of rhombus =$\frac {1}{2}$(product of its diagonals)]
	\\
		\solution
	\iffalse
\documentclass[12pt]{article}
\usepackage{graphicx}
%\documentclass[journal,12pt,twocolumn]{IEEEtran}
\usepackage[none]{hyphenat}
\usepackage{graphicx}
\usepackage{listings}
\usepackage[english]{babel}
\usepackage{graphicx}
\usepackage{caption} 
\usepackage{hyperref}
\usepackage{booktabs}
\def\inputGnumericTable{}
\usepackage{color}                                            %%
    \usepackage{array}                                            %%
    \usepackage{longtable}                                        %%
    \usepackage{calc}                                             %%
    \usepackage{multirow}                                         %%
    \usepackage{hhline}                                           %%
    \usepackage{ifthen}
\usepackage{array}
\usepackage{amsmath}   % for having text in math mode
\usepackage{listings}
\lstset{
language=tex,
frame=single, 
breaklines=true
}
  
%Following 2 lines were added to remove the blank page at the beginning
\usepackage{atbegshi}% http://ctan.org/pkg/atbegshi
\AtBeginDocument{\AtBeginShipoutNext{\AtBeginShipoutDiscard}}
%


%New macro definitions
\newcommand{\mydet}[1]{\ensuremath{\begin{vmatrix}#1\end{vmatrix}}}
\providecommand{\brak}[1]{\ensuremath{\left(#1\right)}}
\providecommand{\norm}[1]{\left\lVert#1\right\rVert}
\newcommand{\solution}{\noindent \textbf{Solution: }}
\newcommand{\myvec}[1]{\ensuremath{\begin{pmatrix}#1\end{pmatrix}}}
\let\vec\mathbf

\begin{document}

\begin{center}
\title{\textbf{Coordinate Geometry}}
\date{\vspace{-5ex}} %Not to print date automatically
\maketitle
\end{center}

\setcounter{page}{1}



\begin{enumerate}

\item\textbf{Problem statement :} Find the area of a rhombus of its vertices are $\myvec{3 ,0}$, $\myvec{4 ,5}$, $\myvec{-1 ,4}$ and $\myvec{-2 ,-1}$taken in order

\solution \\
\fi
The input vertices for this problem are given as
	\begin{align}
	\vec{A} = \myvec{
		3\\
		0
		},
	\vec{B} = \myvec{
		4\\
		5
		},
        \vec{C} = \myvec{
		-1\\
		4
		},
        \vec{D} = \myvec{
		-2\\
		-1
		}
	\end{align}
Since		
\begin{align}
 \vec{A-D}= \myvec{3 \\ 0} - \myvec{-2 \\-1}= \myvec{5\\1}
 \\
  \vec{B-A}= \myvec{4 \\ 5} - \myvec{3 \\0}= \myvec{1\\5}
\end{align}
the area of the rhombus is
\begin{align}
                \norm{\myvec{\vec{A-D}}\times \myvec{\vec{B-A}}}=\mydet{5 & 1\\1 & 5} = 24
\end{align}
See Fig. 
\ref{fig:chapters/10/7/2/10/gFig1}.
\begin{figure}[!h]
 \begin{center}
  \includegraphics[width=\columnwidth]{chapters/10/7/2/10/figs/fig.pdf}
 \end{center}
\caption{}
\label{fig:chapters/10/7/2/10/gFig1}
\end{figure}

\item Find the position vector of a point R which divides the line joining two points $\vec{P}$
and $\vec{Q}$ whose position vectors are $\hat{i}+2\hat{j}-\hat{k}$ and $-\hat{i}+\hat{j}+\hat{k}$ respectively, in the
ratio 2 : 1
\begin{enumerate}
    \item  internally
    \item  externally
\end{enumerate}
\solution
		\begin{enumerate}[label=\thesection.\arabic*,ref=\thesection.\theenumi]
\numberwithin{equation}{enumi}
\numberwithin{figure}{enumi}
\numberwithin{table}{enumi}

\item Find the coordinates of the point which divides the join of $(-1,7) \text{ and } (4,-3)$ in the ratio 2:3.
	\\
		\solution
	\input{chapters/10/7/2/1/section.tex}
\item Find the coordinates of the points of trisection of the line segment joining $(4,-1) \text{ and } (-2,3)$.
	\\
		\solution
	\input{chapters/10/7/2/2/section.tex}
\item
	\iffalse
\item To conduct Sports Day activities, in your rectangular shaped school                   
ground ABCD, lines have 
drawn with chalk powder at a                 
distance of 1m each. 100 flower pots have been placed at a distance of 1m 
from each other along AD, as shown 
in Fig. 7.12. Niharika runs $ \frac {1}{4} $th the 
distance AD on the 2nd line and 
posts a green flag. Preet runs $ \frac {1}{5} $th 
the distance AD on the eighth line 
and posts a red flag. What is the 
distance between both the flags? If 
Rashmi has to post a blue flag exactly 
halfway between the line segment 
joining the two flags, where should 
she post her flag?
\begin{figure}[h!]
  \centering
  \includegraphics[width=\columnwidth]{sc.png}
  \caption{}
\label{fig:10/7/12Fig1}
\end{figure}               
\fi
      
\item Find the ratio in which the line segment joining the points $(-3,10) \text{ and } (6,-8)$ $\text{ is divided by } (-1,6)$.
	\\
		\solution
	\input{chapters/10/7/2/4/section.tex}
\item Find the ratio in which the line segment joining $A(1,-5) \text{ and } B(-4,5)$ $\text{is divided by the x-axis}$. Also find the coordinates of the point of division.
\item If $(1,2), (4,y), (x,6), (3,5)$ are the vertices of a parallelogram taken in order, find x and y.
	\\
		\solution
	\input{chapters/10/7/2/6/para1.tex}
\item Find the coordinates of a point A, where AB is the diameter of a circle whose centre is $(2,-3) \text{ and }$ B is $(1,4)$.
	\\
		\solution
	\input{chapters/10/7/2/7/section.tex}
\item If A \text{ and } B are $(-2,-2) \text{ and } (2,-4)$, respectively, find the coordinates of P such that AP= $\frac {3}{7}$AB $\text{ and }$ P lies on the line segment AB.
	\\
		\solution
	\input{chapters/10/7/2/8/section.tex}
\item Find the coordinates of the points which divide the line segment joining $A(-2,2) \text{ and } B(2,8)$ into four equal parts.
	\\
		\solution
	\input{chapters/10/7/2/9/section.tex}
\item Find the area of a rhombus if its vertices are $(3,0), (4,5), (-1,4) \text{ and } (-2,-1)$ taken in order. [$\vec{Hint}$ : Area of rhombus =$\frac {1}{2}$(product of its diagonals)]
	\\
		\solution
	\input{chapters/10/7/2/10/cross.tex}
\item Find the position vector of a point R which divides the line joining two points $\vec{P}$
and $\vec{Q}$ whose position vectors are $\hat{i}+2\hat{j}-\hat{k}$ and $-\hat{i}+\hat{j}+\hat{k}$ respectively, in the
ratio 2 : 1
\begin{enumerate}
    \item  internally
    \item  externally
\end{enumerate}
\solution
		\input{chapters/12/10/2/15/section.tex}
\item Find the position vector of the mid point of the vector joining the points $\vec{P}$(2, 3, 4)
and $\vec{Q}$(4, 1, –2).
\\
\solution
		\input{chapters/12/10/2/16/section.tex}
\item Determine the ratio in which the line $2x+y  - 4=0$ divides the line segment joining the points $\vec{A}(2, - 2)$  and  $\vec{B}(3, 7)$.
\\
\solution
	\input{chapters/10/7/4/1/section.tex}
\item Let $\vec{A}(4, 2), \vec{B}(6, 5)$  and $ \vec{C}(1, 4)$ be the vertices of $\triangle ABC$.
\begin{enumerate}
\item The median from $\vec{A}$ meets $BC$ at $\vec{D}$. Find the coordinates of the point $\vec{D}$.
\item Find the coordinates of the point $\vec{P}$ on $AD$ such that $AP : PD = 2 : 1$.
\item Find the coordinates of points $\vec{Q}$ and $\vec{R}$ on medians $BE$ and $CF$ respectively such that $BQ : QE = 2 : 1$  and  $CR : RF = 2 : 1$.
\item What do you observe?
\item If $\vec{A}, \vec{B}$ and $\vec{C}$  are the vertices of $\triangle ABC$, find the coordinates of the centroid of the triangle.
\end{enumerate}
\solution
	\input{chapters/10/7/4/7/section.tex}
\item Find the slope of a line, which passes through the origin and the mid point of the line segment joining the points $\vec{P}$(0,-4) and $\vec{B}$(8,0).
\label{chapters/11/10/1/5}
\input{chapters/11/10/1/5/matrix.tex}
\item Find the position vector of a point R which divides the line joining two points P and Q whose position vectors are $(2\vec{a}+\vec{b})$ and $(\vec{a}-3\vec{b})$
externally in the ratio 1 : 2. Also, show that P is the mid point of the line segment RQ.\\
	\solution
%		\input{chapters/12/10/5/9/section.tex}

\end{enumerate}


\item Find the position vector of the mid point of the vector joining the points $\vec{P}$(2, 3, 4)
and $\vec{Q}$(4, 1, –2).
\\
\solution
		\begin{enumerate}[label=\thesection.\arabic*,ref=\thesection.\theenumi]
\numberwithin{equation}{enumi}
\numberwithin{figure}{enumi}
\numberwithin{table}{enumi}

\item Find the coordinates of the point which divides the join of $(-1,7) \text{ and } (4,-3)$ in the ratio 2:3.
	\\
		\solution
	\input{chapters/10/7/2/1/section.tex}
\item Find the coordinates of the points of trisection of the line segment joining $(4,-1) \text{ and } (-2,3)$.
	\\
		\solution
	\input{chapters/10/7/2/2/section.tex}
\item
	\iffalse
\item To conduct Sports Day activities, in your rectangular shaped school                   
ground ABCD, lines have 
drawn with chalk powder at a                 
distance of 1m each. 100 flower pots have been placed at a distance of 1m 
from each other along AD, as shown 
in Fig. 7.12. Niharika runs $ \frac {1}{4} $th the 
distance AD on the 2nd line and 
posts a green flag. Preet runs $ \frac {1}{5} $th 
the distance AD on the eighth line 
and posts a red flag. What is the 
distance between both the flags? If 
Rashmi has to post a blue flag exactly 
halfway between the line segment 
joining the two flags, where should 
she post her flag?
\begin{figure}[h!]
  \centering
  \includegraphics[width=\columnwidth]{sc.png}
  \caption{}
\label{fig:10/7/12Fig1}
\end{figure}               
\fi
      
\item Find the ratio in which the line segment joining the points $(-3,10) \text{ and } (6,-8)$ $\text{ is divided by } (-1,6)$.
	\\
		\solution
	\input{chapters/10/7/2/4/section.tex}
\item Find the ratio in which the line segment joining $A(1,-5) \text{ and } B(-4,5)$ $\text{is divided by the x-axis}$. Also find the coordinates of the point of division.
\item If $(1,2), (4,y), (x,6), (3,5)$ are the vertices of a parallelogram taken in order, find x and y.
	\\
		\solution
	\input{chapters/10/7/2/6/para1.tex}
\item Find the coordinates of a point A, where AB is the diameter of a circle whose centre is $(2,-3) \text{ and }$ B is $(1,4)$.
	\\
		\solution
	\input{chapters/10/7/2/7/section.tex}
\item If A \text{ and } B are $(-2,-2) \text{ and } (2,-4)$, respectively, find the coordinates of P such that AP= $\frac {3}{7}$AB $\text{ and }$ P lies on the line segment AB.
	\\
		\solution
	\input{chapters/10/7/2/8/section.tex}
\item Find the coordinates of the points which divide the line segment joining $A(-2,2) \text{ and } B(2,8)$ into four equal parts.
	\\
		\solution
	\input{chapters/10/7/2/9/section.tex}
\item Find the area of a rhombus if its vertices are $(3,0), (4,5), (-1,4) \text{ and } (-2,-1)$ taken in order. [$\vec{Hint}$ : Area of rhombus =$\frac {1}{2}$(product of its diagonals)]
	\\
		\solution
	\input{chapters/10/7/2/10/cross.tex}
\item Find the position vector of a point R which divides the line joining two points $\vec{P}$
and $\vec{Q}$ whose position vectors are $\hat{i}+2\hat{j}-\hat{k}$ and $-\hat{i}+\hat{j}+\hat{k}$ respectively, in the
ratio 2 : 1
\begin{enumerate}
    \item  internally
    \item  externally
\end{enumerate}
\solution
		\input{chapters/12/10/2/15/section.tex}
\item Find the position vector of the mid point of the vector joining the points $\vec{P}$(2, 3, 4)
and $\vec{Q}$(4, 1, –2).
\\
\solution
		\input{chapters/12/10/2/16/section.tex}
\item Determine the ratio in which the line $2x+y  - 4=0$ divides the line segment joining the points $\vec{A}(2, - 2)$  and  $\vec{B}(3, 7)$.
\\
\solution
	\input{chapters/10/7/4/1/section.tex}
\item Let $\vec{A}(4, 2), \vec{B}(6, 5)$  and $ \vec{C}(1, 4)$ be the vertices of $\triangle ABC$.
\begin{enumerate}
\item The median from $\vec{A}$ meets $BC$ at $\vec{D}$. Find the coordinates of the point $\vec{D}$.
\item Find the coordinates of the point $\vec{P}$ on $AD$ such that $AP : PD = 2 : 1$.
\item Find the coordinates of points $\vec{Q}$ and $\vec{R}$ on medians $BE$ and $CF$ respectively such that $BQ : QE = 2 : 1$  and  $CR : RF = 2 : 1$.
\item What do you observe?
\item If $\vec{A}, \vec{B}$ and $\vec{C}$  are the vertices of $\triangle ABC$, find the coordinates of the centroid of the triangle.
\end{enumerate}
\solution
	\input{chapters/10/7/4/7/section.tex}
\item Find the slope of a line, which passes through the origin and the mid point of the line segment joining the points $\vec{P}$(0,-4) and $\vec{B}$(8,0).
\label{chapters/11/10/1/5}
\input{chapters/11/10/1/5/matrix.tex}
\item Find the position vector of a point R which divides the line joining two points P and Q whose position vectors are $(2\vec{a}+\vec{b})$ and $(\vec{a}-3\vec{b})$
externally in the ratio 1 : 2. Also, show that P is the mid point of the line segment RQ.\\
	\solution
%		\input{chapters/12/10/5/9/section.tex}

\end{enumerate}


\item Determine the ratio in which the line $2x+y  - 4=0$ divides the line segment joining the points $\vec{A}(2, - 2)$  and  $\vec{B}(3, 7)$.
\\
\solution
	\iffalse
\documentclass[journal,12pt,twocolumn]{IEEEtran}
\usepackage{graphicx}
\graphicspath{{./chapters/10/7/4/1/figs/}}{}
\usepackage{amsmath,amssymb,amsfonts,amsthm}
\newcommand{\myvec}[1]{\ensuremath{\begin{pmatrix}#1\end{pmatrix}}}
\providecommand{\norm}[1]{\lVert#1\rVert}
\usepackage{listings}
\usepackage{watermark}
\usepackage{titlesec}
\usepackage{caption}
\let\vec\mathbf
\lstset{
frame=single, 
breaklines=true,
columns=fullflexible
}
\thiswatermark{\centering \put(0,-105.0){\includegraphics[scale=0.15]{/sdcard/IITH/vector/vectpr-4/chapters/10/7/4/1/figs/logo.png}} }
\title{\mytitle}
\title{
Assignment - Vector-4
}
\author{Surajit Sarkar}
\begin{document}
\maketitle
%\tableofcontents
\bigskip
\section{\textbf{Problem}}
Determine the ratio in which the line 2x+y–4=0 divides the line segment joining the points A(2,–2) and B(3,7).
\section{\textbf{Solution}}
\begin{table}[h]
    \centering
    \begin{tabular}{|c|c|}
       \hline
       \textbf{Symbol}&\textbf{Value}  \\
       \hline
	    $\vec{A}$ & $\myvec{2\\-2}$\\
        \hline
	    $\vec{B}$ & $\myvec{3\\7}$\\
        \hline
	    c&$4$\\
        \hline
       $\vec{n}$ & $\myvec{2\\1}$\\
       \hline
    \end{tabular}
    \caption{Parameters}
    \label{tab:my_label}
\end{table}
Given equation
\fi
The given equation can be expressed as
\begin{align}
    \myvec{2&1}\vec{x}&=4\\
\end{align}
Using section formula, the point of division 
\begin{align}
    \vec{P} = \frac{k\vec{B+A}}{k+1}
\end{align}
which upon substitution in the equation of a line yields
\begin{align}
    \implies\vec{n}^{\top}\myvec{\frac{k\vec{B+A}}{k+1}}&=c\\
    \implies k&=\frac{c-\vec{n}^{\top}\vec{A}}{\vec{n}^{\top}\vec{B}-c}\\
\end{align}
upon simplification.  Substituting numerical values, 
\begin{align}
    k=\frac{2}{9}
\end{align}
See Fig. 
\ref{fig:chapters/10/7/4/1vec}.
\begin{figure}[!h]
\centering
\includegraphics[width=\columnwidth]{chapters/10/7/4/1/figs/vec.pdf}
\caption{}
\label{fig:chapters/10/7/4/1vec}
\end{figure}


\item Let $\vec{A}(4, 2), \vec{B}(6, 5)$  and $ \vec{C}(1, 4)$ be the vertices of $\triangle ABC$.
\begin{enumerate}
\item The median from $\vec{A}$ meets $BC$ at $\vec{D}$. Find the coordinates of the point $\vec{D}$.
\item Find the coordinates of the point $\vec{P}$ on $AD$ such that $AP : PD = 2 : 1$.
\item Find the coordinates of points $\vec{Q}$ and $\vec{R}$ on medians $BE$ and $CF$ respectively such that $BQ : QE = 2 : 1$  and  $CR : RF = 2 : 1$.
\item What do you observe?
\item If $\vec{A}, \vec{B}$ and $\vec{C}$  are the vertices of $\triangle ABC$, find the coordinates of the centroid of the triangle.
\end{enumerate}
\solution
	\iffalse
\documentclass[12pt]{article}
\usepackage{graphicx}
\usepackage[none]{hyphenat}
\usepackage{graphicx}
\usepackage{listings}
\usepackage[english]{babel}
\usepackage{graphicx}
\usepackage{caption} 
\usepackage{booktabs}
\usepackage{array}
\usepackage{amssymb} % for \because
\usepackage{amsmath}   % for having text in math mode
\usepackage{extarrows} % for Row operations arrows
\usepackage{listings}
\usepackage[utf8]{inputenc}
\lstset{
  frame=single,
  breaklines=true
}
\usepackage{hyperref}
  
%Following 2 lines were added to remove the blank page at the beginning
\usepackage{atbegshi}% http://ctan.org/pkg/atbegshi
\AtBeginDocument{\AtBeginShipoutNext{\AtBeginShipoutDiscard}}


%New macro definitions
\newcommand{\mydet}[1]{\ensuremath{\begin{vmatrix}#1\end{vmatrix}}}
\providecommand{\brak}[1]{\ensuremath{\left(#1\right)}}
\newcommand{\solution}{\noindent \textbf{Solution: }}
\newcommand{\myvec}[1]{\ensuremath{\begin{pmatrix}#1\end{pmatrix}}}
\providecommand{\norm}[1]{\left\lVert#1\right\rVert}
\providecommand{\abs}[1]{\left\vert#1\right\vert}
\let\vec\mathbf

\begin{document}

\begin{center}
\title{\textbf{VECTORS}}
\date{\vspace{-5ex}} %Not to print date automatically
\maketitle
\end{center}

\section{10$^{th}$ Maths - EXERCISE-7.4}

Let A(4, 2), B(6, 5) and C(1, 4) be the vertices of $\triangle ABC$
\begin{enumerate}
\item The median from A meets BC at D. Find the coordinates of the point D.
\item Find the coordinates of the point P on AD such that $AP : PD = 2 : 1$
\item Find the coordinates of points Q and R on medians BE and CF respectively such
that $BQ : QE = 2 : 1 \text{and} CR : RF = 2 : 1.$
\item What do yo observe?
\item If $A(x_1, y_1), B(x_2, y_2) \text{and} C(x_3, y_3)$ are the vertices of $\triangle ABC$, find the coordinates of the centroid of the triangle.
\end{enumerate}

Given points are
\begin{align}
\vec{A}=\myvec{4\\ 2} ,
\vec{B}=\myvec{6\\ 5} ,
\vec{C}=\myvec{1\\ 4}
\end{align}
\fi

\begin{enumerate}
\item 
\begin{align}
\vec{D}&=\frac{\vec{B}+\vec{C}}{2}\\
&=\myvec{\frac{7}{2}\\[2pt] \frac{9}{2}}\\
\vec{E}&=\frac{\vec{A}+\vec{C}}{2}\\
&=\myvec{\frac{5}{2}\\ 3}\\
\vec{F}&=\frac{\vec{A}+\vec{B}}{2}\\
&=\myvec{5\\ \frac{7}{2}}
\end{align}

\item 
	For
$n=2$,
\begin{align}
\vec{P}&=\frac{1}{1+n}\brak{\myvec{\vec{A}+n\vec{D}}}\\
&=\frac{1}{3}\myvec{11\\11}
\end{align}

\item 
\begin{align}
\vec{Q}&=\frac{1}{1+n}\brak{\myvec{\vec{B}+n\vec{E}}}\\
&=\frac{1}{3}\myvec{11\\11}\\
\vec{R}&=\frac{1}{1+n}\brak{\myvec{\vec{C}+n\vec{F}}}\\
&=\frac{1}{3}\myvec{11\\11}\\
\end{align}

\item 
 $\vec{P},\vec{Q},\vec{R}$ are the same point.
   
\item 
\begin{align}
\vec{G}&=\frac{\vec{D}+\vec{E}+\vec{F}}{3}\\
&=\frac{1}{3}\myvec{11\\11}\\
\end{align} 
\end{enumerate}
See Fig.  
  \ref{fig:chapters/10/7/4/7/Figure}.
\begin{figure}[h!]
\centering
\includegraphics[width=\columnwidth]{chapters/10/7/4/7/figs/dj.pdf}
\caption{}
  \label{fig:chapters/10/7/4/7/Figure}
\end{figure}

\item Find the slope of a line, which passes through the origin and the mid point of the line segment joining the points $\vec{P}$(0,-4) and $\vec{B}$(8,0).
\label{chapters/11/10/1/5}
\iffalse
\documentclass[journal,12pt,twocolumn]{IEEEtran}
\usepackage{graphicx}
\graphicspath{{./figs/}}{}
\usepackage{amsmath,amssymb,amsfonts,amsthm}
\newcommand{\myvec}[1]{\ensuremath{\begin{pmatrix}#1\end{pmatrix}}}

\let\vec\mathbf

\title{
Matrix-Lines
}
\author{Jyothsna Paluchuri-FWC22059\\}
\begin{document}
\maketitle
\tableofcontents
\bigskip
\section{Problem Statement}
\fi
	\begin{figure}[!ht]
		\centering
 \includegraphics[width=\columnwidth]{chapters/11/10/1/5/figs/line.png}
		\caption{}
		\label{fig:11/10/1/5}
  	\end{figure}
	\\
	\solution
\iffalse
\section{Construction}
\begin{figure}[h]
    \centering
\includegraphics[width=\columnwidth]{line.png}
    \caption{Equation of the slope}
    \label{fig:my_label}
\end{figure}
\vspace{2cm}
\begin{table}[h]
    \centering
    \begin{tabular}{|c|c|c|c|}
       \hline
       \textbf{Symbol}&\textbf{Value}&\textbf{Description}  \\
       \hline
	    $\vec{P}$ & $\myvec{
		    0\\
		    -4}$
	    & Point on Y-axis\\
        \hline
	    $\vec{B}$ & $\myvec{8\\0}$
 & Point on X-axis\\
        \hline
	    $\vec{0}$ & $\myvec{0\\0}$
 & Origin\\
        \hline
    \end{tabular}
    \caption{Parameters}
    \label{tab:my_label}
\end{table}


\section{Solution}
Given that resultant line passes through origin and mid point of the line segment joining point P(0,-4) and B(8,0) \\
\\
\\
given ${\vec{P}}$=$\myvec{
  0\\
  -4}$
 , ${\vec{B}}$=$\myvec{
  8\\
  0}$
  
 \fi 
The mid point of $PB$ is
\begin{align}
\vec{M} &=\frac{1}{2}(\vec{P}+\vec{B})
	= \myvec{4 \\ -2}  
\end{align}
The direction vector of line joining $\vec{O}, \vec{M}$ is 
\begin{align}
\vec{m}&=\vec{O}-\vec{M}
 = -\vec{M}
\end{align}
which can be expressed as
\begin{align}
	\myvec{1 \\ -\frac{1}{2}}
\end{align}
Thus the slope is
\begin{align}
	m = -\frac{1}{2}
\end{align}
\iffalse
\textbf{The direction vector of a line expressed as}
\begin{align}
\implies\vec{m} &= \begin{pmatrix}1 \\ m \\ \end{pmatrix}
\end{align}

\textbf{By solving equation (5) and (6),we get the slope of $\vec{O}$ $\vec{M}$ line}
\begin{align}
        \boxed{m=-0.5}
 \end{align}

\section{Software}
Download the following code using,
\begin{table}[h]
    \centering
    \begin{tabular}{|c|}
    \hline \\
   https://github.com/jyothsna777/jyothsna-fwc.git  \\
         \\
\hline
    \end{tabular}
\end{table}
\\
and execute the code by using command
\begin{center}
\textbf{Python3 lines.py}\\
\end{center}

\section{Conclusion}
Hence the slope of line $\vec{O}$ $\vec{M}$ lineis $\vec{m}$=-0.5

\end{document}
\fi

\item Find the position vector of a point R which divides the line joining two points P and Q whose position vectors are $(2\vec{a}+\vec{b})$ and $(\vec{a}-3\vec{b})$
externally in the ratio 1 : 2. Also, show that P is the mid point of the line segment RQ.\\
	\solution
%		\begin{enumerate}[label=\thesection.\arabic*,ref=\thesection.\theenumi]
\numberwithin{equation}{enumi}
\numberwithin{figure}{enumi}
\numberwithin{table}{enumi}

\item Find the coordinates of the point which divides the join of $(-1,7) \text{ and } (4,-3)$ in the ratio 2:3.
	\\
		\solution
	\input{chapters/10/7/2/1/section.tex}
\item Find the coordinates of the points of trisection of the line segment joining $(4,-1) \text{ and } (-2,3)$.
	\\
		\solution
	\input{chapters/10/7/2/2/section.tex}
\item
	\iffalse
\item To conduct Sports Day activities, in your rectangular shaped school                   
ground ABCD, lines have 
drawn with chalk powder at a                 
distance of 1m each. 100 flower pots have been placed at a distance of 1m 
from each other along AD, as shown 
in Fig. 7.12. Niharika runs $ \frac {1}{4} $th the 
distance AD on the 2nd line and 
posts a green flag. Preet runs $ \frac {1}{5} $th 
the distance AD on the eighth line 
and posts a red flag. What is the 
distance between both the flags? If 
Rashmi has to post a blue flag exactly 
halfway between the line segment 
joining the two flags, where should 
she post her flag?
\begin{figure}[h!]
  \centering
  \includegraphics[width=\columnwidth]{sc.png}
  \caption{}
\label{fig:10/7/12Fig1}
\end{figure}               
\fi
      
\item Find the ratio in which the line segment joining the points $(-3,10) \text{ and } (6,-8)$ $\text{ is divided by } (-1,6)$.
	\\
		\solution
	\input{chapters/10/7/2/4/section.tex}
\item Find the ratio in which the line segment joining $A(1,-5) \text{ and } B(-4,5)$ $\text{is divided by the x-axis}$. Also find the coordinates of the point of division.
\item If $(1,2), (4,y), (x,6), (3,5)$ are the vertices of a parallelogram taken in order, find x and y.
	\\
		\solution
	\input{chapters/10/7/2/6/para1.tex}
\item Find the coordinates of a point A, where AB is the diameter of a circle whose centre is $(2,-3) \text{ and }$ B is $(1,4)$.
	\\
		\solution
	\input{chapters/10/7/2/7/section.tex}
\item If A \text{ and } B are $(-2,-2) \text{ and } (2,-4)$, respectively, find the coordinates of P such that AP= $\frac {3}{7}$AB $\text{ and }$ P lies on the line segment AB.
	\\
		\solution
	\input{chapters/10/7/2/8/section.tex}
\item Find the coordinates of the points which divide the line segment joining $A(-2,2) \text{ and } B(2,8)$ into four equal parts.
	\\
		\solution
	\input{chapters/10/7/2/9/section.tex}
\item Find the area of a rhombus if its vertices are $(3,0), (4,5), (-1,4) \text{ and } (-2,-1)$ taken in order. [$\vec{Hint}$ : Area of rhombus =$\frac {1}{2}$(product of its diagonals)]
	\\
		\solution
	\input{chapters/10/7/2/10/cross.tex}
\item Find the position vector of a point R which divides the line joining two points $\vec{P}$
and $\vec{Q}$ whose position vectors are $\hat{i}+2\hat{j}-\hat{k}$ and $-\hat{i}+\hat{j}+\hat{k}$ respectively, in the
ratio 2 : 1
\begin{enumerate}
    \item  internally
    \item  externally
\end{enumerate}
\solution
		\input{chapters/12/10/2/15/section.tex}
\item Find the position vector of the mid point of the vector joining the points $\vec{P}$(2, 3, 4)
and $\vec{Q}$(4, 1, –2).
\\
\solution
		\input{chapters/12/10/2/16/section.tex}
\item Determine the ratio in which the line $2x+y  - 4=0$ divides the line segment joining the points $\vec{A}(2, - 2)$  and  $\vec{B}(3, 7)$.
\\
\solution
	\input{chapters/10/7/4/1/section.tex}
\item Let $\vec{A}(4, 2), \vec{B}(6, 5)$  and $ \vec{C}(1, 4)$ be the vertices of $\triangle ABC$.
\begin{enumerate}
\item The median from $\vec{A}$ meets $BC$ at $\vec{D}$. Find the coordinates of the point $\vec{D}$.
\item Find the coordinates of the point $\vec{P}$ on $AD$ such that $AP : PD = 2 : 1$.
\item Find the coordinates of points $\vec{Q}$ and $\vec{R}$ on medians $BE$ and $CF$ respectively such that $BQ : QE = 2 : 1$  and  $CR : RF = 2 : 1$.
\item What do you observe?
\item If $\vec{A}, \vec{B}$ and $\vec{C}$  are the vertices of $\triangle ABC$, find the coordinates of the centroid of the triangle.
\end{enumerate}
\solution
	\input{chapters/10/7/4/7/section.tex}
\item Find the slope of a line, which passes through the origin and the mid point of the line segment joining the points $\vec{P}$(0,-4) and $\vec{B}$(8,0).
\label{chapters/11/10/1/5}
\input{chapters/11/10/1/5/matrix.tex}
\item Find the position vector of a point R which divides the line joining two points P and Q whose position vectors are $(2\vec{a}+\vec{b})$ and $(\vec{a}-3\vec{b})$
externally in the ratio 1 : 2. Also, show that P is the mid point of the line segment RQ.\\
	\solution
%		\input{chapters/12/10/5/9/section.tex}

\end{enumerate}



\end{enumerate}


\item Find the area of a rhombus if its vertices are $(3,0), (4,5), (-1,4) \text{ and } (-2,-1)$ taken in order. [$\vec{Hint}$ : Area of rhombus =$\frac {1}{2}$(product of its diagonals)]
	\\
		\solution
	\iffalse
\documentclass[12pt]{article}
\usepackage{graphicx}
%\documentclass[journal,12pt,twocolumn]{IEEEtran}
\usepackage[none]{hyphenat}
\usepackage{graphicx}
\usepackage{listings}
\usepackage[english]{babel}
\usepackage{graphicx}
\usepackage{caption} 
\usepackage{hyperref}
\usepackage{booktabs}
\def\inputGnumericTable{}
\usepackage{color}                                            %%
    \usepackage{array}                                            %%
    \usepackage{longtable}                                        %%
    \usepackage{calc}                                             %%
    \usepackage{multirow}                                         %%
    \usepackage{hhline}                                           %%
    \usepackage{ifthen}
\usepackage{array}
\usepackage{amsmath}   % for having text in math mode
\usepackage{listings}
\lstset{
language=tex,
frame=single, 
breaklines=true
}
  
%Following 2 lines were added to remove the blank page at the beginning
\usepackage{atbegshi}% http://ctan.org/pkg/atbegshi
\AtBeginDocument{\AtBeginShipoutNext{\AtBeginShipoutDiscard}}
%


%New macro definitions
\newcommand{\mydet}[1]{\ensuremath{\begin{vmatrix}#1\end{vmatrix}}}
\providecommand{\brak}[1]{\ensuremath{\left(#1\right)}}
\providecommand{\norm}[1]{\left\lVert#1\right\rVert}
\newcommand{\solution}{\noindent \textbf{Solution: }}
\newcommand{\myvec}[1]{\ensuremath{\begin{pmatrix}#1\end{pmatrix}}}
\let\vec\mathbf

\begin{document}

\begin{center}
\title{\textbf{Coordinate Geometry}}
\date{\vspace{-5ex}} %Not to print date automatically
\maketitle
\end{center}

\setcounter{page}{1}



\begin{enumerate}

\item\textbf{Problem statement :} Find the area of a rhombus of its vertices are $\myvec{3 ,0}$, $\myvec{4 ,5}$, $\myvec{-1 ,4}$ and $\myvec{-2 ,-1}$taken in order

\solution \\
\fi
The input vertices for this problem are given as
	\begin{align}
	\vec{A} = \myvec{
		3\\
		0
		},
	\vec{B} = \myvec{
		4\\
		5
		},
        \vec{C} = \myvec{
		-1\\
		4
		},
        \vec{D} = \myvec{
		-2\\
		-1
		}
	\end{align}
Since		
\begin{align}
 \vec{A-D}= \myvec{3 \\ 0} - \myvec{-2 \\-1}= \myvec{5\\1}
 \\
  \vec{B-A}= \myvec{4 \\ 5} - \myvec{3 \\0}= \myvec{1\\5}
\end{align}
the area of the rhombus is
\begin{align}
                \norm{\myvec{\vec{A-D}}\times \myvec{\vec{B-A}}}=\mydet{5 & 1\\1 & 5} = 24
\end{align}
See Fig. 
\ref{fig:chapters/10/7/2/10/gFig1}.
\begin{figure}[!h]
 \begin{center}
  \includegraphics[width=\columnwidth]{chapters/10/7/2/10/figs/fig.pdf}
 \end{center}
\caption{}
\label{fig:chapters/10/7/2/10/gFig1}
\end{figure}

\item Find the position vector of a point R which divides the line joining two points $\vec{P}$
and $\vec{Q}$ whose position vectors are $\hat{i}+2\hat{j}-\hat{k}$ and $-\hat{i}+\hat{j}+\hat{k}$ respectively, in the
ratio 2 : 1
\begin{enumerate}
    \item  internally
    \item  externally
\end{enumerate}
\solution
		\begin{enumerate}[label=\thesection.\arabic*,ref=\thesection.\theenumi]
\numberwithin{equation}{enumi}
\numberwithin{figure}{enumi}
\numberwithin{table}{enumi}

\item Find the coordinates of the point which divides the join of $(-1,7) \text{ and } (4,-3)$ in the ratio 2:3.
	\\
		\solution
	\iffalse
\documentclass[12pt]{article}
\usepackage{graphicx}
\usepackage{amsmath}
\usepackage{mathtools}
\usepackage{gensymb}

\newcommand{\mydet}[1]{\ensuremath{\begin{vmatrix}#1\end{vmatrix}}}
\providecommand{\brak}[1]{\ensuremath{\left(#1\right)}}
\providecommand{\norm}[1]{\left\lVert#1\right\rVert}
\newcommand{\solution}{\noindent \textbf{Solution: }}
\newcommand{\myvec}[1]{\ensuremath{\begin{pmatrix}#1\end{pmatrix}}}
\let\vec\mathbf

\begin{document}
\begin{center}
\textbf\large{CHAPTER-7 \\ COORDINATE GEOMETRY}
\end{center}
\section*{Excercise 7.2}

1. Find the coordinates of the point which divides the join $\vec(-1,7) \text{ and } \vec(4,-3)$ in the ratio 2:3 :
\\
\\
\solution\\		
\fi
The coordinates and ratio are given as
\begin{align}
\vec{P}=\myvec{-1\\7\\},
\vec{Q}=\myvec{4\\-3\\},
n=\frac{3}{2}
\end{align}
Using section formula
\begin{align}
\vec{R}&=\frac{\vec{Q}+n\vec{P}}{1+n}\\
&=\frac{1}{1+\frac{3}{2}}  \myvec{\myvec{
4\\
-3\\
}
  +
   \frac{3}{2}\myvec{
-1\\
7\\
}}\\
&=\myvec{
1\\
3
}
\end{align}
See Fig. 
\ref{fig:chapters/10/7/2/1/Fig}
\begin{figure}[!h]
\begin{center}
   \includegraphics[width=\columnwidth]{chapters/10/7/2/1/figs/linefig.png}
\end{center}
\caption{}
\label{fig:chapters/10/7/2/1/Fig}
\end{figure}


\item Find the coordinates of the points of trisection of the line segment joining $(4,-1) \text{ and } (-2,3)$.
	\\
		\solution
	\begin{enumerate}[label=\thesection.\arabic*,ref=\thesection.\theenumi]
\numberwithin{equation}{enumi}
\numberwithin{figure}{enumi}
\numberwithin{table}{enumi}

\item Find the coordinates of the point which divides the join of $(-1,7) \text{ and } (4,-3)$ in the ratio 2:3.
	\\
		\solution
	\input{chapters/10/7/2/1/section.tex}
\item Find the coordinates of the points of trisection of the line segment joining $(4,-1) \text{ and } (-2,3)$.
	\\
		\solution
	\input{chapters/10/7/2/2/section.tex}
\item
	\iffalse
\item To conduct Sports Day activities, in your rectangular shaped school                   
ground ABCD, lines have 
drawn with chalk powder at a                 
distance of 1m each. 100 flower pots have been placed at a distance of 1m 
from each other along AD, as shown 
in Fig. 7.12. Niharika runs $ \frac {1}{4} $th the 
distance AD on the 2nd line and 
posts a green flag. Preet runs $ \frac {1}{5} $th 
the distance AD on the eighth line 
and posts a red flag. What is the 
distance between both the flags? If 
Rashmi has to post a blue flag exactly 
halfway between the line segment 
joining the two flags, where should 
she post her flag?
\begin{figure}[h!]
  \centering
  \includegraphics[width=\columnwidth]{sc.png}
  \caption{}
\label{fig:10/7/12Fig1}
\end{figure}               
\fi
      
\item Find the ratio in which the line segment joining the points $(-3,10) \text{ and } (6,-8)$ $\text{ is divided by } (-1,6)$.
	\\
		\solution
	\input{chapters/10/7/2/4/section.tex}
\item Find the ratio in which the line segment joining $A(1,-5) \text{ and } B(-4,5)$ $\text{is divided by the x-axis}$. Also find the coordinates of the point of division.
\item If $(1,2), (4,y), (x,6), (3,5)$ are the vertices of a parallelogram taken in order, find x and y.
	\\
		\solution
	\input{chapters/10/7/2/6/para1.tex}
\item Find the coordinates of a point A, where AB is the diameter of a circle whose centre is $(2,-3) \text{ and }$ B is $(1,4)$.
	\\
		\solution
	\input{chapters/10/7/2/7/section.tex}
\item If A \text{ and } B are $(-2,-2) \text{ and } (2,-4)$, respectively, find the coordinates of P such that AP= $\frac {3}{7}$AB $\text{ and }$ P lies on the line segment AB.
	\\
		\solution
	\input{chapters/10/7/2/8/section.tex}
\item Find the coordinates of the points which divide the line segment joining $A(-2,2) \text{ and } B(2,8)$ into four equal parts.
	\\
		\solution
	\input{chapters/10/7/2/9/section.tex}
\item Find the area of a rhombus if its vertices are $(3,0), (4,5), (-1,4) \text{ and } (-2,-1)$ taken in order. [$\vec{Hint}$ : Area of rhombus =$\frac {1}{2}$(product of its diagonals)]
	\\
		\solution
	\input{chapters/10/7/2/10/cross.tex}
\item Find the position vector of a point R which divides the line joining two points $\vec{P}$
and $\vec{Q}$ whose position vectors are $\hat{i}+2\hat{j}-\hat{k}$ and $-\hat{i}+\hat{j}+\hat{k}$ respectively, in the
ratio 2 : 1
\begin{enumerate}
    \item  internally
    \item  externally
\end{enumerate}
\solution
		\input{chapters/12/10/2/15/section.tex}
\item Find the position vector of the mid point of the vector joining the points $\vec{P}$(2, 3, 4)
and $\vec{Q}$(4, 1, –2).
\\
\solution
		\input{chapters/12/10/2/16/section.tex}
\item Determine the ratio in which the line $2x+y  - 4=0$ divides the line segment joining the points $\vec{A}(2, - 2)$  and  $\vec{B}(3, 7)$.
\\
\solution
	\input{chapters/10/7/4/1/section.tex}
\item Let $\vec{A}(4, 2), \vec{B}(6, 5)$  and $ \vec{C}(1, 4)$ be the vertices of $\triangle ABC$.
\begin{enumerate}
\item The median from $\vec{A}$ meets $BC$ at $\vec{D}$. Find the coordinates of the point $\vec{D}$.
\item Find the coordinates of the point $\vec{P}$ on $AD$ such that $AP : PD = 2 : 1$.
\item Find the coordinates of points $\vec{Q}$ and $\vec{R}$ on medians $BE$ and $CF$ respectively such that $BQ : QE = 2 : 1$  and  $CR : RF = 2 : 1$.
\item What do you observe?
\item If $\vec{A}, \vec{B}$ and $\vec{C}$  are the vertices of $\triangle ABC$, find the coordinates of the centroid of the triangle.
\end{enumerate}
\solution
	\input{chapters/10/7/4/7/section.tex}
\item Find the slope of a line, which passes through the origin and the mid point of the line segment joining the points $\vec{P}$(0,-4) and $\vec{B}$(8,0).
\label{chapters/11/10/1/5}
\input{chapters/11/10/1/5/matrix.tex}
\item Find the position vector of a point R which divides the line joining two points P and Q whose position vectors are $(2\vec{a}+\vec{b})$ and $(\vec{a}-3\vec{b})$
externally in the ratio 1 : 2. Also, show that P is the mid point of the line segment RQ.\\
	\solution
%		\input{chapters/12/10/5/9/section.tex}

\end{enumerate}


\item
	\iffalse
\item To conduct Sports Day activities, in your rectangular shaped school                   
ground ABCD, lines have 
drawn with chalk powder at a                 
distance of 1m each. 100 flower pots have been placed at a distance of 1m 
from each other along AD, as shown 
in Fig. 7.12. Niharika runs $ \frac {1}{4} $th the 
distance AD on the 2nd line and 
posts a green flag. Preet runs $ \frac {1}{5} $th 
the distance AD on the eighth line 
and posts a red flag. What is the 
distance between both the flags? If 
Rashmi has to post a blue flag exactly 
halfway between the line segment 
joining the two flags, where should 
she post her flag?
\begin{figure}[h!]
  \centering
  \includegraphics[width=\columnwidth]{sc.png}
  \caption{}
\label{fig:10/7/12Fig1}
\end{figure}               
\fi
      
\item Find the ratio in which the line segment joining the points $(-3,10) \text{ and } (6,-8)$ $\text{ is divided by } (-1,6)$.
	\\
		\solution
	\iffalse
\documentclass[12pt]{article}
\usepackage{graphicx}
%\documentclass[journal,12pt,twocolumn]{IEEEtran}
\usepackage[none]{hyphenat}
\usepackage{graphicx}
\usepackage{listings}
\usepackage[english]{babel}
\usepackage{graphicx}
\usepackage{caption} 
\usepackage{hyperref}
\usepackage{booktabs}
\def\inputGnumericTable{}
\usepackage{color}                                            %%
    \usepackage{array}                                            %%
    \usepackage{longtable}                                        %%
    \usepackage{calc}                                             %%
    \usepackage{multirow}                                         %%
    \usepackage{hhline}                                           %%
    \usepackage{ifthen}
\usepackage{array}
\usepackage{amsmath}   % for having text in math mode
\usepackage{listings}
\lstset{
language=tex,
frame=single, 
breaklines=true
}
  
%Following 2 lines were added to remove the blank page at the beginning
\usepackage{atbegshi}% http://ctan.org/pkg/atbegshi
\AtBeginDocument{\AtBeginShipoutNext{\AtBeginShipoutDiscard}}
%
%New macro definitions
\newcommand{\mydet}[1]{\ensuremath{\begin{vmatrix}#1\end{vmatrix}}}
\providecommand{\brak}[1]{\ensuremath{\left(#1\right)}}
\providecommand{\norm}[1]{\left\lVert#1\right\rVert}
\newcommand{\solution}{\noindent \textbf{Solution: }}
\newcommand{\myvec}[1]{\ensuremath{\begin{pmatrix}#1\end{pmatrix}}}
\let\vec\mathbf
\begin{document}
\begin{center}
\title{\textbf{Coordinate Geometry}}
\date{\vspace{-5ex}} %Not to print date automatically
\maketitle
\end{center}
\setcounter{page}{1}
\section*{10$^{th}$ Maths - Chapter 7}
This is Problem-4 from Exercise 7.2
\begin{enumerate}
\item Find the ratio in which the line segement joining the points $\myvec{-3 \\ 10}$ and $\myvec{6\\-8}$ is divided by $\myvec{-1\\6}$.\\
\solution \\
\fi
		The input parameters for this problem are available in Table \eqref{tab:10/7/2/4-1}.
\begin{table}[ht!]
\input{chapters/10/7/2/4/tables/table.tex}
\caption{}
\label{tab:10/7/2/4-1} 
\end{table}
Using section formula,
\begin{align}
         \vec{R} &=\frac{\vec{Q}+n\vec{P}}{1+n}\label{eq:chapters/10/7/2/4/1}
\end{align}
Substituting the values of $\vec{P},\vec{Q}$ and $\vec{R}$ in \eqref{eq:chapters/10/7/2/4/1}
\begin{align}
         \myvec{-1\\6} &=\frac{{\myvec{-3\\10}+n\myvec{6\\-8}}}{1+n}\\
 &=\frac{1}{1+n}\brak{{\myvec{-3\\10}+n\myvec{6\\-8}}} \\
 &=\frac{1}{1+n}\myvec{-3+6n\\10-8n} \label{eq:chapters/10/7/2/4/4}
\end{align}
Simplifying \eqref{eq:chapters/10/7/2/4/4} yeilds,
\begin{align}
          -1 &=\frac{-3+6n}{1+n}\\
\implies          n &=\frac{2}{7}
\end{align}
Also,
\begin{align}
          6 &=\frac{10-8n}{1+n}\\
    \implies      n &=\frac{2}{7}
\end{align}
Hence the desired ratio is $\dfrac{2}{7}$.  
\begin{figure}[!h]
 \begin{center}
  \includegraphics[width=\columnwidth]{chapters/10/7/2/4/figs/fig.png}
 \end{center}
\caption{}
\label{fig:10/7/2/4Fig1}
\end{figure}

\item Find the ratio in which the line segment joining $A(1,-5) \text{ and } B(-4,5)$ $\text{is divided by the x-axis}$. Also find the coordinates of the point of division.
\item If $(1,2), (4,y), (x,6), (3,5)$ are the vertices of a parallelogram taken in order, find x and y.
	\\
		\solution
	\iffalse
\documentclass[12pt]{article}
\usepackage{graphicx}
%\documentclass[journal,12pt,twocolumn]{IEEEtran}
\def\inputGnumericTable{}
\usepackage{color}                                            %%
    \usepackage{array}                                            %%
    \usepackage{longtable}                                        %%
    \usepackage{calc}                                             %%
    \usepackage{multirow}                                         %%
    \usepackage{hhline}                                           %%
    \usepackage{ifthen}
\usepackage[none]{hyphenat}
\usepackage{graphicx}
\usepackage{listings}
\usepackage[english]{babel}
\usepackage{graphicx}
\usepackage{caption} 
\usepackage{hyperref}
\usepackage{booktabs}
\usepackage{array}
\usepackage{amsmath}   % for having text in math mode
\usepackage{listings}
\lstset{
  frame=single,
  breaklines=true
}
  
%Following 2 lines were added to remove the blank page at the beginning
\usepackage{atbegshi}% http://ctan.org/pkg/atbegshi
\AtBeginDocument{\AtBeginShipoutNext{\AtBeginShipoutDiscard}}
%


%New macro definitions
\newcommand{\mydet}[1]{\ensuremath{\begin{vmatrix}#1\end{vmatrix}}}
\providecommand{\brak}[1]{\ensuremath{\left(#1\right)}}
\providecommand{\norm}[1]{\left\lVert#1\right\rVert}
\newcommand{\solution}{\noindent \textbf{Solution: }}
\newcommand{\myvec}[1]{\ensuremath{\begin{pmatrix}#1\end{pmatrix}}}
\let\vec\mathbf

\begin{document}

\begin{center}
\title{\textbf{Properties of Parallelegram}}
\date{\vspace{-5ex}} %Not to print date automatically
\maketitle
\end{center}

\setcounter{page}{1}

\section{10$^{th}$ Maths - Chapter 7}

This is Problem-6 from Exercise 7.2

\begin{enumerate}
\item If $\vec{A}(1, 2),\vec{B}(4, x),\vec{C}(y, 6) \text{and } \vec{D}(3, 5)$ are the vertices of a parallelogram taken in order,find x and y.
\end{enumerate}
\fi

The input parameters for this problem are available in
\ref{table:chapters/10/7/2/6/tables/}.	
\begin{table}[!ht]
	\centering
	\input{chapters/10/7/2/6/tables/table.tex}
\caption{}
\label{table:chapters/10/7/2/6/tables/}	
\end{table}
From the given information,
\begin{align}
  \label{eq:chapters/10/7/2/6/tables/det2f}
	\vec{B}-\vec{A} &= \myvec{4 \\y } - \myvec{1 \\2 }  = \myvec{3 \\y-2 }\\
	\vec{C}-\vec{D} &= \myvec{x \\6 } - \myvec{3 \\5 }  = \myvec{x-3 \\1}
\end{align}
Since $ABCD$ is a parallellogram,
\begin{align}
	\myvec{3\\y-2}&=\myvec{x-3\\1}\\
	\implies x&=6 ,y=3
\end{align}
Fig. \ref{fig:chapters/10/7/2/6/Fig3}
provides a verification.
\begin{figure}[h!]
	\begin{center}
  \includegraphics[width=\columnwidth]{chapters/10/7/2/6/figs/para.pdf}
	\end{center}
\caption{}
\label{fig:chapters/10/7/2/6/Fig3}
\end{figure}


\item Find the coordinates of a point A, where AB is the diameter of a circle whose centre is $(2,-3) \text{ and }$ B is $(1,4)$.
	\\
		\solution
	\iffalse
\documentclass[12pt]{article}
\usepackage{graphicx}
\usepackage{amsmath}
\usepackage{mathtools}
\usepackage{gensymb}

\newcommand{\mydet}[1]{\ensuremath{\begin{vmatrix}#1\end{vmatrix}}}
\providecommand{\brak}[1]{\ensuremath{\left(#1\right)}}
\providecommand{\norm}[1]{\left\lVert#1\right\rVert}
\newcommand{\solution}{\noindent \textbf{Solution: }}
\newcommand{\myvec}[1]{\ensuremath{\begin{pmatrix}#1\end{pmatrix}}}
\let\vec\mathbf

\begin{document}
\begin{center}
\section*{CHAPTER 7 - COORDINATE GEOMETRY}

\end{center}
\section*{Excercise 7.2}

Q7.Find the coordinates of point $\vec{A}$, where AB is the diameter of a circle where the center is (2,-3) and $\vec{B}$ is the point (1,4):

\solution
\begin{enumerate}
\item The coordinates $\vec{B}$ and center $\vec{C}$ are given, where:
	\fi
	Let
	\begin{align}
	\vec{B} = \myvec{
		1\\
	    4\\
		},
	\vec{C} = \myvec{
	    2\\
	   -3\\
		}
	\end{align}
	\iffalse
Let us assume the coordinates of $\vec{A}$. Now, $\vec{C}$ is the center which is midpoint of line AB and $\vec{B}$ is one of the coordinate of diameter AB of a circle.
	\fi	
Hence,	
	\begin{align}
	\vec{C} &= \frac{\vec{A+B}}{2} \\
\implies	2\vec{C} &= \vec{A}+\vec{B} \\
		\text{or, }	\vec{A} &= 2\vec{C}-\vec{B} \\
	 &= \myvec{3\\-10\\}	
	\end{align}       
	See Fig. 
\ref{fig:chapters/10/7/2/7Fig}.
\begin{figure}[!h]
\begin{center}	
	\includegraphics[width=\columnwidth]{chapters/10/7/2/7/figs/Vector1.png}
\end{center}
\caption{}
\label{fig:chapters/10/7/2/7Fig}
\end{figure}
	

\item If A \text{ and } B are $(-2,-2) \text{ and } (2,-4)$, respectively, find the coordinates of P such that AP= $\frac {3}{7}$AB $\text{ and }$ P lies on the line segment AB.
	\\
		\solution
	\iffalse
\documentclass[journal,10pt,twocolumn]{article}
\usepackage{graphicx}
\usepackage[none]{hyphenat}
\usepackage{graphicx}
\usepackage{listings}
\usepackage[english]{babel}
\usepackage{graphicx}
\usepackage{caption} 
\usepackage{booktabs}
\usepackage{array}
\usepackage{amssymb} % for \because
\usepackage{amsmath}   % for having text in math mode
\usepackage{extarrows} % for Row operations arrows
\usepackage{listings}
\usepackage[utf8]{inputenc}
\lstset{
  frame=single,
  breaklines=true
}
\usepackage{hyperref}
  
%Following 2 lines were added to remove the blank page at the beginning
\usepackage{atbegshi}% http://ctan.org/pkg/atbegshi
\AtBeginDocument{\AtBeginShipoutNext{\AtBeginShipoutDiscard}}


%New macro definitions
\newcommand{\mydet}[1]{\ensuremath{\begin{vmatrix}#1\end{vmatrix}}}
\providecommand{\brak}[1]{\ensuremath{\left(#1\right)}}
\newcommand{\solution}{\noindent \textbf{Solution: }}
\newcommand{\myvec}[1]{\ensuremath{\begin{pmatrix}#1\end{pmatrix}}}
\providecommand{\norm}[1]{\left\lVert#1\right\rVert}
\providecommand{\abs}[1]{\left\vert#1\right\vert}
\let\vec\mathbf

\begin{document}

\begin{center}
\title{\textbf{VECTORS}}
\date{\vspace{-5ex}} %Not to print date automatically
\maketitle
\end{center}

\section{10$^{th}$ Maths - EXERCISE-7.2}

\begin{enumerate}
\item If A and B are $(– 2, – 2)\text{ and }(2, – 4)$, respectively, find the coordinates of P such that $AP =\frac{3}{7}AB$ and P lies on the line segment AB. 

\section{SOLUTION}
Given points are
\begin{align}
\vec{A}=\myvec{-2\\ -2} ,
\vec{B}=\myvec{2\\ -4}
\end{align}
The equation of the formula is
\fi
Using section formula, 
\begin{align}
\vec{P}&=\frac{\vec{A}+n\vec{B}}{1+n}
\end{align}
where
\begin{align}
	n =\frac{3}{4}
\end{align}
Thus,
\begin{align}
\vec{P}&=\frac{1}{1+\frac{3}{4}}\brak{\myvec{-2\\-2}+\frac{3}{4}\myvec{2\\-4}}\\
&=\myvec{\frac{-2}{7}\\[1pt] \frac{-20}{7}}
\end{align}
See Fig. 
   \ref{fig:chapters/10/7/2/8/vec.png}
\begin{figure}
   \centering 
 \includegraphics[width=\columnwidth]{chapters/10/7/2/8/figs/vec.png}
   \caption{}
   \label{fig:chapters/10/7/2/8/vec.png}
   \end{figure}

\item Find the coordinates of the points which divide the line segment joining $A(-2,2) \text{ and } B(2,8)$ into four equal parts.
	\\
		\solution
	\begin{enumerate}[label=\thesection.\arabic*,ref=\thesection.\theenumi]
\numberwithin{equation}{enumi}
\numberwithin{figure}{enumi}
\numberwithin{table}{enumi}

\item Find the coordinates of the point which divides the join of $(-1,7) \text{ and } (4,-3)$ in the ratio 2:3.
	\\
		\solution
	\input{chapters/10/7/2/1/section.tex}
\item Find the coordinates of the points of trisection of the line segment joining $(4,-1) \text{ and } (-2,3)$.
	\\
		\solution
	\input{chapters/10/7/2/2/section.tex}
\item
	\iffalse
\item To conduct Sports Day activities, in your rectangular shaped school                   
ground ABCD, lines have 
drawn with chalk powder at a                 
distance of 1m each. 100 flower pots have been placed at a distance of 1m 
from each other along AD, as shown 
in Fig. 7.12. Niharika runs $ \frac {1}{4} $th the 
distance AD on the 2nd line and 
posts a green flag. Preet runs $ \frac {1}{5} $th 
the distance AD on the eighth line 
and posts a red flag. What is the 
distance between both the flags? If 
Rashmi has to post a blue flag exactly 
halfway between the line segment 
joining the two flags, where should 
she post her flag?
\begin{figure}[h!]
  \centering
  \includegraphics[width=\columnwidth]{sc.png}
  \caption{}
\label{fig:10/7/12Fig1}
\end{figure}               
\fi
      
\item Find the ratio in which the line segment joining the points $(-3,10) \text{ and } (6,-8)$ $\text{ is divided by } (-1,6)$.
	\\
		\solution
	\input{chapters/10/7/2/4/section.tex}
\item Find the ratio in which the line segment joining $A(1,-5) \text{ and } B(-4,5)$ $\text{is divided by the x-axis}$. Also find the coordinates of the point of division.
\item If $(1,2), (4,y), (x,6), (3,5)$ are the vertices of a parallelogram taken in order, find x and y.
	\\
		\solution
	\input{chapters/10/7/2/6/para1.tex}
\item Find the coordinates of a point A, where AB is the diameter of a circle whose centre is $(2,-3) \text{ and }$ B is $(1,4)$.
	\\
		\solution
	\input{chapters/10/7/2/7/section.tex}
\item If A \text{ and } B are $(-2,-2) \text{ and } (2,-4)$, respectively, find the coordinates of P such that AP= $\frac {3}{7}$AB $\text{ and }$ P lies on the line segment AB.
	\\
		\solution
	\input{chapters/10/7/2/8/section.tex}
\item Find the coordinates of the points which divide the line segment joining $A(-2,2) \text{ and } B(2,8)$ into four equal parts.
	\\
		\solution
	\input{chapters/10/7/2/9/section.tex}
\item Find the area of a rhombus if its vertices are $(3,0), (4,5), (-1,4) \text{ and } (-2,-1)$ taken in order. [$\vec{Hint}$ : Area of rhombus =$\frac {1}{2}$(product of its diagonals)]
	\\
		\solution
	\input{chapters/10/7/2/10/cross.tex}
\item Find the position vector of a point R which divides the line joining two points $\vec{P}$
and $\vec{Q}$ whose position vectors are $\hat{i}+2\hat{j}-\hat{k}$ and $-\hat{i}+\hat{j}+\hat{k}$ respectively, in the
ratio 2 : 1
\begin{enumerate}
    \item  internally
    \item  externally
\end{enumerate}
\solution
		\input{chapters/12/10/2/15/section.tex}
\item Find the position vector of the mid point of the vector joining the points $\vec{P}$(2, 3, 4)
and $\vec{Q}$(4, 1, –2).
\\
\solution
		\input{chapters/12/10/2/16/section.tex}
\item Determine the ratio in which the line $2x+y  - 4=0$ divides the line segment joining the points $\vec{A}(2, - 2)$  and  $\vec{B}(3, 7)$.
\\
\solution
	\input{chapters/10/7/4/1/section.tex}
\item Let $\vec{A}(4, 2), \vec{B}(6, 5)$  and $ \vec{C}(1, 4)$ be the vertices of $\triangle ABC$.
\begin{enumerate}
\item The median from $\vec{A}$ meets $BC$ at $\vec{D}$. Find the coordinates of the point $\vec{D}$.
\item Find the coordinates of the point $\vec{P}$ on $AD$ such that $AP : PD = 2 : 1$.
\item Find the coordinates of points $\vec{Q}$ and $\vec{R}$ on medians $BE$ and $CF$ respectively such that $BQ : QE = 2 : 1$  and  $CR : RF = 2 : 1$.
\item What do you observe?
\item If $\vec{A}, \vec{B}$ and $\vec{C}$  are the vertices of $\triangle ABC$, find the coordinates of the centroid of the triangle.
\end{enumerate}
\solution
	\input{chapters/10/7/4/7/section.tex}
\item Find the slope of a line, which passes through the origin and the mid point of the line segment joining the points $\vec{P}$(0,-4) and $\vec{B}$(8,0).
\label{chapters/11/10/1/5}
\input{chapters/11/10/1/5/matrix.tex}
\item Find the position vector of a point R which divides the line joining two points P and Q whose position vectors are $(2\vec{a}+\vec{b})$ and $(\vec{a}-3\vec{b})$
externally in the ratio 1 : 2. Also, show that P is the mid point of the line segment RQ.\\
	\solution
%		\input{chapters/12/10/5/9/section.tex}

\end{enumerate}


\item Find the area of a rhombus if its vertices are $(3,0), (4,5), (-1,4) \text{ and } (-2,-1)$ taken in order. [$\vec{Hint}$ : Area of rhombus =$\frac {1}{2}$(product of its diagonals)]
	\\
		\solution
	\iffalse
\documentclass[12pt]{article}
\usepackage{graphicx}
%\documentclass[journal,12pt,twocolumn]{IEEEtran}
\usepackage[none]{hyphenat}
\usepackage{graphicx}
\usepackage{listings}
\usepackage[english]{babel}
\usepackage{graphicx}
\usepackage{caption} 
\usepackage{hyperref}
\usepackage{booktabs}
\def\inputGnumericTable{}
\usepackage{color}                                            %%
    \usepackage{array}                                            %%
    \usepackage{longtable}                                        %%
    \usepackage{calc}                                             %%
    \usepackage{multirow}                                         %%
    \usepackage{hhline}                                           %%
    \usepackage{ifthen}
\usepackage{array}
\usepackage{amsmath}   % for having text in math mode
\usepackage{listings}
\lstset{
language=tex,
frame=single, 
breaklines=true
}
  
%Following 2 lines were added to remove the blank page at the beginning
\usepackage{atbegshi}% http://ctan.org/pkg/atbegshi
\AtBeginDocument{\AtBeginShipoutNext{\AtBeginShipoutDiscard}}
%


%New macro definitions
\newcommand{\mydet}[1]{\ensuremath{\begin{vmatrix}#1\end{vmatrix}}}
\providecommand{\brak}[1]{\ensuremath{\left(#1\right)}}
\providecommand{\norm}[1]{\left\lVert#1\right\rVert}
\newcommand{\solution}{\noindent \textbf{Solution: }}
\newcommand{\myvec}[1]{\ensuremath{\begin{pmatrix}#1\end{pmatrix}}}
\let\vec\mathbf

\begin{document}

\begin{center}
\title{\textbf{Coordinate Geometry}}
\date{\vspace{-5ex}} %Not to print date automatically
\maketitle
\end{center}

\setcounter{page}{1}



\begin{enumerate}

\item\textbf{Problem statement :} Find the area of a rhombus of its vertices are $\myvec{3 ,0}$, $\myvec{4 ,5}$, $\myvec{-1 ,4}$ and $\myvec{-2 ,-1}$taken in order

\solution \\
\fi
The input vertices for this problem are given as
	\begin{align}
	\vec{A} = \myvec{
		3\\
		0
		},
	\vec{B} = \myvec{
		4\\
		5
		},
        \vec{C} = \myvec{
		-1\\
		4
		},
        \vec{D} = \myvec{
		-2\\
		-1
		}
	\end{align}
Since		
\begin{align}
 \vec{A-D}= \myvec{3 \\ 0} - \myvec{-2 \\-1}= \myvec{5\\1}
 \\
  \vec{B-A}= \myvec{4 \\ 5} - \myvec{3 \\0}= \myvec{1\\5}
\end{align}
the area of the rhombus is
\begin{align}
                \norm{\myvec{\vec{A-D}}\times \myvec{\vec{B-A}}}=\mydet{5 & 1\\1 & 5} = 24
\end{align}
See Fig. 
\ref{fig:chapters/10/7/2/10/gFig1}.
\begin{figure}[!h]
 \begin{center}
  \includegraphics[width=\columnwidth]{chapters/10/7/2/10/figs/fig.pdf}
 \end{center}
\caption{}
\label{fig:chapters/10/7/2/10/gFig1}
\end{figure}

\item Find the position vector of a point R which divides the line joining two points $\vec{P}$
and $\vec{Q}$ whose position vectors are $\hat{i}+2\hat{j}-\hat{k}$ and $-\hat{i}+\hat{j}+\hat{k}$ respectively, in the
ratio 2 : 1
\begin{enumerate}
    \item  internally
    \item  externally
\end{enumerate}
\solution
		\begin{enumerate}[label=\thesection.\arabic*,ref=\thesection.\theenumi]
\numberwithin{equation}{enumi}
\numberwithin{figure}{enumi}
\numberwithin{table}{enumi}

\item Find the coordinates of the point which divides the join of $(-1,7) \text{ and } (4,-3)$ in the ratio 2:3.
	\\
		\solution
	\input{chapters/10/7/2/1/section.tex}
\item Find the coordinates of the points of trisection of the line segment joining $(4,-1) \text{ and } (-2,3)$.
	\\
		\solution
	\input{chapters/10/7/2/2/section.tex}
\item
	\iffalse
\item To conduct Sports Day activities, in your rectangular shaped school                   
ground ABCD, lines have 
drawn with chalk powder at a                 
distance of 1m each. 100 flower pots have been placed at a distance of 1m 
from each other along AD, as shown 
in Fig. 7.12. Niharika runs $ \frac {1}{4} $th the 
distance AD on the 2nd line and 
posts a green flag. Preet runs $ \frac {1}{5} $th 
the distance AD on the eighth line 
and posts a red flag. What is the 
distance between both the flags? If 
Rashmi has to post a blue flag exactly 
halfway between the line segment 
joining the two flags, where should 
she post her flag?
\begin{figure}[h!]
  \centering
  \includegraphics[width=\columnwidth]{sc.png}
  \caption{}
\label{fig:10/7/12Fig1}
\end{figure}               
\fi
      
\item Find the ratio in which the line segment joining the points $(-3,10) \text{ and } (6,-8)$ $\text{ is divided by } (-1,6)$.
	\\
		\solution
	\input{chapters/10/7/2/4/section.tex}
\item Find the ratio in which the line segment joining $A(1,-5) \text{ and } B(-4,5)$ $\text{is divided by the x-axis}$. Also find the coordinates of the point of division.
\item If $(1,2), (4,y), (x,6), (3,5)$ are the vertices of a parallelogram taken in order, find x and y.
	\\
		\solution
	\input{chapters/10/7/2/6/para1.tex}
\item Find the coordinates of a point A, where AB is the diameter of a circle whose centre is $(2,-3) \text{ and }$ B is $(1,4)$.
	\\
		\solution
	\input{chapters/10/7/2/7/section.tex}
\item If A \text{ and } B are $(-2,-2) \text{ and } (2,-4)$, respectively, find the coordinates of P such that AP= $\frac {3}{7}$AB $\text{ and }$ P lies on the line segment AB.
	\\
		\solution
	\input{chapters/10/7/2/8/section.tex}
\item Find the coordinates of the points which divide the line segment joining $A(-2,2) \text{ and } B(2,8)$ into four equal parts.
	\\
		\solution
	\input{chapters/10/7/2/9/section.tex}
\item Find the area of a rhombus if its vertices are $(3,0), (4,5), (-1,4) \text{ and } (-2,-1)$ taken in order. [$\vec{Hint}$ : Area of rhombus =$\frac {1}{2}$(product of its diagonals)]
	\\
		\solution
	\input{chapters/10/7/2/10/cross.tex}
\item Find the position vector of a point R which divides the line joining two points $\vec{P}$
and $\vec{Q}$ whose position vectors are $\hat{i}+2\hat{j}-\hat{k}$ and $-\hat{i}+\hat{j}+\hat{k}$ respectively, in the
ratio 2 : 1
\begin{enumerate}
    \item  internally
    \item  externally
\end{enumerate}
\solution
		\input{chapters/12/10/2/15/section.tex}
\item Find the position vector of the mid point of the vector joining the points $\vec{P}$(2, 3, 4)
and $\vec{Q}$(4, 1, –2).
\\
\solution
		\input{chapters/12/10/2/16/section.tex}
\item Determine the ratio in which the line $2x+y  - 4=0$ divides the line segment joining the points $\vec{A}(2, - 2)$  and  $\vec{B}(3, 7)$.
\\
\solution
	\input{chapters/10/7/4/1/section.tex}
\item Let $\vec{A}(4, 2), \vec{B}(6, 5)$  and $ \vec{C}(1, 4)$ be the vertices of $\triangle ABC$.
\begin{enumerate}
\item The median from $\vec{A}$ meets $BC$ at $\vec{D}$. Find the coordinates of the point $\vec{D}$.
\item Find the coordinates of the point $\vec{P}$ on $AD$ such that $AP : PD = 2 : 1$.
\item Find the coordinates of points $\vec{Q}$ and $\vec{R}$ on medians $BE$ and $CF$ respectively such that $BQ : QE = 2 : 1$  and  $CR : RF = 2 : 1$.
\item What do you observe?
\item If $\vec{A}, \vec{B}$ and $\vec{C}$  are the vertices of $\triangle ABC$, find the coordinates of the centroid of the triangle.
\end{enumerate}
\solution
	\input{chapters/10/7/4/7/section.tex}
\item Find the slope of a line, which passes through the origin and the mid point of the line segment joining the points $\vec{P}$(0,-4) and $\vec{B}$(8,0).
\label{chapters/11/10/1/5}
\input{chapters/11/10/1/5/matrix.tex}
\item Find the position vector of a point R which divides the line joining two points P and Q whose position vectors are $(2\vec{a}+\vec{b})$ and $(\vec{a}-3\vec{b})$
externally in the ratio 1 : 2. Also, show that P is the mid point of the line segment RQ.\\
	\solution
%		\input{chapters/12/10/5/9/section.tex}

\end{enumerate}


\item Find the position vector of the mid point of the vector joining the points $\vec{P}$(2, 3, 4)
and $\vec{Q}$(4, 1, –2).
\\
\solution
		\begin{enumerate}[label=\thesection.\arabic*,ref=\thesection.\theenumi]
\numberwithin{equation}{enumi}
\numberwithin{figure}{enumi}
\numberwithin{table}{enumi}

\item Find the coordinates of the point which divides the join of $(-1,7) \text{ and } (4,-3)$ in the ratio 2:3.
	\\
		\solution
	\input{chapters/10/7/2/1/section.tex}
\item Find the coordinates of the points of trisection of the line segment joining $(4,-1) \text{ and } (-2,3)$.
	\\
		\solution
	\input{chapters/10/7/2/2/section.tex}
\item
	\iffalse
\item To conduct Sports Day activities, in your rectangular shaped school                   
ground ABCD, lines have 
drawn with chalk powder at a                 
distance of 1m each. 100 flower pots have been placed at a distance of 1m 
from each other along AD, as shown 
in Fig. 7.12. Niharika runs $ \frac {1}{4} $th the 
distance AD on the 2nd line and 
posts a green flag. Preet runs $ \frac {1}{5} $th 
the distance AD on the eighth line 
and posts a red flag. What is the 
distance between both the flags? If 
Rashmi has to post a blue flag exactly 
halfway between the line segment 
joining the two flags, where should 
she post her flag?
\begin{figure}[h!]
  \centering
  \includegraphics[width=\columnwidth]{sc.png}
  \caption{}
\label{fig:10/7/12Fig1}
\end{figure}               
\fi
      
\item Find the ratio in which the line segment joining the points $(-3,10) \text{ and } (6,-8)$ $\text{ is divided by } (-1,6)$.
	\\
		\solution
	\input{chapters/10/7/2/4/section.tex}
\item Find the ratio in which the line segment joining $A(1,-5) \text{ and } B(-4,5)$ $\text{is divided by the x-axis}$. Also find the coordinates of the point of division.
\item If $(1,2), (4,y), (x,6), (3,5)$ are the vertices of a parallelogram taken in order, find x and y.
	\\
		\solution
	\input{chapters/10/7/2/6/para1.tex}
\item Find the coordinates of a point A, where AB is the diameter of a circle whose centre is $(2,-3) \text{ and }$ B is $(1,4)$.
	\\
		\solution
	\input{chapters/10/7/2/7/section.tex}
\item If A \text{ and } B are $(-2,-2) \text{ and } (2,-4)$, respectively, find the coordinates of P such that AP= $\frac {3}{7}$AB $\text{ and }$ P lies on the line segment AB.
	\\
		\solution
	\input{chapters/10/7/2/8/section.tex}
\item Find the coordinates of the points which divide the line segment joining $A(-2,2) \text{ and } B(2,8)$ into four equal parts.
	\\
		\solution
	\input{chapters/10/7/2/9/section.tex}
\item Find the area of a rhombus if its vertices are $(3,0), (4,5), (-1,4) \text{ and } (-2,-1)$ taken in order. [$\vec{Hint}$ : Area of rhombus =$\frac {1}{2}$(product of its diagonals)]
	\\
		\solution
	\input{chapters/10/7/2/10/cross.tex}
\item Find the position vector of a point R which divides the line joining two points $\vec{P}$
and $\vec{Q}$ whose position vectors are $\hat{i}+2\hat{j}-\hat{k}$ and $-\hat{i}+\hat{j}+\hat{k}$ respectively, in the
ratio 2 : 1
\begin{enumerate}
    \item  internally
    \item  externally
\end{enumerate}
\solution
		\input{chapters/12/10/2/15/section.tex}
\item Find the position vector of the mid point of the vector joining the points $\vec{P}$(2, 3, 4)
and $\vec{Q}$(4, 1, –2).
\\
\solution
		\input{chapters/12/10/2/16/section.tex}
\item Determine the ratio in which the line $2x+y  - 4=0$ divides the line segment joining the points $\vec{A}(2, - 2)$  and  $\vec{B}(3, 7)$.
\\
\solution
	\input{chapters/10/7/4/1/section.tex}
\item Let $\vec{A}(4, 2), \vec{B}(6, 5)$  and $ \vec{C}(1, 4)$ be the vertices of $\triangle ABC$.
\begin{enumerate}
\item The median from $\vec{A}$ meets $BC$ at $\vec{D}$. Find the coordinates of the point $\vec{D}$.
\item Find the coordinates of the point $\vec{P}$ on $AD$ such that $AP : PD = 2 : 1$.
\item Find the coordinates of points $\vec{Q}$ and $\vec{R}$ on medians $BE$ and $CF$ respectively such that $BQ : QE = 2 : 1$  and  $CR : RF = 2 : 1$.
\item What do you observe?
\item If $\vec{A}, \vec{B}$ and $\vec{C}$  are the vertices of $\triangle ABC$, find the coordinates of the centroid of the triangle.
\end{enumerate}
\solution
	\input{chapters/10/7/4/7/section.tex}
\item Find the slope of a line, which passes through the origin and the mid point of the line segment joining the points $\vec{P}$(0,-4) and $\vec{B}$(8,0).
\label{chapters/11/10/1/5}
\input{chapters/11/10/1/5/matrix.tex}
\item Find the position vector of a point R which divides the line joining two points P and Q whose position vectors are $(2\vec{a}+\vec{b})$ and $(\vec{a}-3\vec{b})$
externally in the ratio 1 : 2. Also, show that P is the mid point of the line segment RQ.\\
	\solution
%		\input{chapters/12/10/5/9/section.tex}

\end{enumerate}


\item Determine the ratio in which the line $2x+y  - 4=0$ divides the line segment joining the points $\vec{A}(2, - 2)$  and  $\vec{B}(3, 7)$.
\\
\solution
	\iffalse
\documentclass[journal,12pt,twocolumn]{IEEEtran}
\usepackage{graphicx}
\graphicspath{{./chapters/10/7/4/1/figs/}}{}
\usepackage{amsmath,amssymb,amsfonts,amsthm}
\newcommand{\myvec}[1]{\ensuremath{\begin{pmatrix}#1\end{pmatrix}}}
\providecommand{\norm}[1]{\lVert#1\rVert}
\usepackage{listings}
\usepackage{watermark}
\usepackage{titlesec}
\usepackage{caption}
\let\vec\mathbf
\lstset{
frame=single, 
breaklines=true,
columns=fullflexible
}
\thiswatermark{\centering \put(0,-105.0){\includegraphics[scale=0.15]{/sdcard/IITH/vector/vectpr-4/chapters/10/7/4/1/figs/logo.png}} }
\title{\mytitle}
\title{
Assignment - Vector-4
}
\author{Surajit Sarkar}
\begin{document}
\maketitle
%\tableofcontents
\bigskip
\section{\textbf{Problem}}
Determine the ratio in which the line 2x+y–4=0 divides the line segment joining the points A(2,–2) and B(3,7).
\section{\textbf{Solution}}
\begin{table}[h]
    \centering
    \begin{tabular}{|c|c|}
       \hline
       \textbf{Symbol}&\textbf{Value}  \\
       \hline
	    $\vec{A}$ & $\myvec{2\\-2}$\\
        \hline
	    $\vec{B}$ & $\myvec{3\\7}$\\
        \hline
	    c&$4$\\
        \hline
       $\vec{n}$ & $\myvec{2\\1}$\\
       \hline
    \end{tabular}
    \caption{Parameters}
    \label{tab:my_label}
\end{table}
Given equation
\fi
The given equation can be expressed as
\begin{align}
    \myvec{2&1}\vec{x}&=4\\
\end{align}
Using section formula, the point of division 
\begin{align}
    \vec{P} = \frac{k\vec{B+A}}{k+1}
\end{align}
which upon substitution in the equation of a line yields
\begin{align}
    \implies\vec{n}^{\top}\myvec{\frac{k\vec{B+A}}{k+1}}&=c\\
    \implies k&=\frac{c-\vec{n}^{\top}\vec{A}}{\vec{n}^{\top}\vec{B}-c}\\
\end{align}
upon simplification.  Substituting numerical values, 
\begin{align}
    k=\frac{2}{9}
\end{align}
See Fig. 
\ref{fig:chapters/10/7/4/1vec}.
\begin{figure}[!h]
\centering
\includegraphics[width=\columnwidth]{chapters/10/7/4/1/figs/vec.pdf}
\caption{}
\label{fig:chapters/10/7/4/1vec}
\end{figure}


\item Let $\vec{A}(4, 2), \vec{B}(6, 5)$  and $ \vec{C}(1, 4)$ be the vertices of $\triangle ABC$.
\begin{enumerate}
\item The median from $\vec{A}$ meets $BC$ at $\vec{D}$. Find the coordinates of the point $\vec{D}$.
\item Find the coordinates of the point $\vec{P}$ on $AD$ such that $AP : PD = 2 : 1$.
\item Find the coordinates of points $\vec{Q}$ and $\vec{R}$ on medians $BE$ and $CF$ respectively such that $BQ : QE = 2 : 1$  and  $CR : RF = 2 : 1$.
\item What do you observe?
\item If $\vec{A}, \vec{B}$ and $\vec{C}$  are the vertices of $\triangle ABC$, find the coordinates of the centroid of the triangle.
\end{enumerate}
\solution
	\iffalse
\documentclass[12pt]{article}
\usepackage{graphicx}
\usepackage[none]{hyphenat}
\usepackage{graphicx}
\usepackage{listings}
\usepackage[english]{babel}
\usepackage{graphicx}
\usepackage{caption} 
\usepackage{booktabs}
\usepackage{array}
\usepackage{amssymb} % for \because
\usepackage{amsmath}   % for having text in math mode
\usepackage{extarrows} % for Row operations arrows
\usepackage{listings}
\usepackage[utf8]{inputenc}
\lstset{
  frame=single,
  breaklines=true
}
\usepackage{hyperref}
  
%Following 2 lines were added to remove the blank page at the beginning
\usepackage{atbegshi}% http://ctan.org/pkg/atbegshi
\AtBeginDocument{\AtBeginShipoutNext{\AtBeginShipoutDiscard}}


%New macro definitions
\newcommand{\mydet}[1]{\ensuremath{\begin{vmatrix}#1\end{vmatrix}}}
\providecommand{\brak}[1]{\ensuremath{\left(#1\right)}}
\newcommand{\solution}{\noindent \textbf{Solution: }}
\newcommand{\myvec}[1]{\ensuremath{\begin{pmatrix}#1\end{pmatrix}}}
\providecommand{\norm}[1]{\left\lVert#1\right\rVert}
\providecommand{\abs}[1]{\left\vert#1\right\vert}
\let\vec\mathbf

\begin{document}

\begin{center}
\title{\textbf{VECTORS}}
\date{\vspace{-5ex}} %Not to print date automatically
\maketitle
\end{center}

\section{10$^{th}$ Maths - EXERCISE-7.4}

Let A(4, 2), B(6, 5) and C(1, 4) be the vertices of $\triangle ABC$
\begin{enumerate}
\item The median from A meets BC at D. Find the coordinates of the point D.
\item Find the coordinates of the point P on AD such that $AP : PD = 2 : 1$
\item Find the coordinates of points Q and R on medians BE and CF respectively such
that $BQ : QE = 2 : 1 \text{and} CR : RF = 2 : 1.$
\item What do yo observe?
\item If $A(x_1, y_1), B(x_2, y_2) \text{and} C(x_3, y_3)$ are the vertices of $\triangle ABC$, find the coordinates of the centroid of the triangle.
\end{enumerate}

Given points are
\begin{align}
\vec{A}=\myvec{4\\ 2} ,
\vec{B}=\myvec{6\\ 5} ,
\vec{C}=\myvec{1\\ 4}
\end{align}
\fi

\begin{enumerate}
\item 
\begin{align}
\vec{D}&=\frac{\vec{B}+\vec{C}}{2}\\
&=\myvec{\frac{7}{2}\\[2pt] \frac{9}{2}}\\
\vec{E}&=\frac{\vec{A}+\vec{C}}{2}\\
&=\myvec{\frac{5}{2}\\ 3}\\
\vec{F}&=\frac{\vec{A}+\vec{B}}{2}\\
&=\myvec{5\\ \frac{7}{2}}
\end{align}

\item 
	For
$n=2$,
\begin{align}
\vec{P}&=\frac{1}{1+n}\brak{\myvec{\vec{A}+n\vec{D}}}\\
&=\frac{1}{3}\myvec{11\\11}
\end{align}

\item 
\begin{align}
\vec{Q}&=\frac{1}{1+n}\brak{\myvec{\vec{B}+n\vec{E}}}\\
&=\frac{1}{3}\myvec{11\\11}\\
\vec{R}&=\frac{1}{1+n}\brak{\myvec{\vec{C}+n\vec{F}}}\\
&=\frac{1}{3}\myvec{11\\11}\\
\end{align}

\item 
 $\vec{P},\vec{Q},\vec{R}$ are the same point.
   
\item 
\begin{align}
\vec{G}&=\frac{\vec{D}+\vec{E}+\vec{F}}{3}\\
&=\frac{1}{3}\myvec{11\\11}\\
\end{align} 
\end{enumerate}
See Fig.  
  \ref{fig:chapters/10/7/4/7/Figure}.
\begin{figure}[h!]
\centering
\includegraphics[width=\columnwidth]{chapters/10/7/4/7/figs/dj.pdf}
\caption{}
  \label{fig:chapters/10/7/4/7/Figure}
\end{figure}

\item Find the slope of a line, which passes through the origin and the mid point of the line segment joining the points $\vec{P}$(0,-4) and $\vec{B}$(8,0).
\label{chapters/11/10/1/5}
\iffalse
\documentclass[journal,12pt,twocolumn]{IEEEtran}
\usepackage{graphicx}
\graphicspath{{./figs/}}{}
\usepackage{amsmath,amssymb,amsfonts,amsthm}
\newcommand{\myvec}[1]{\ensuremath{\begin{pmatrix}#1\end{pmatrix}}}

\let\vec\mathbf

\title{
Matrix-Lines
}
\author{Jyothsna Paluchuri-FWC22059\\}
\begin{document}
\maketitle
\tableofcontents
\bigskip
\section{Problem Statement}
\fi
	\begin{figure}[!ht]
		\centering
 \includegraphics[width=\columnwidth]{chapters/11/10/1/5/figs/line.png}
		\caption{}
		\label{fig:11/10/1/5}
  	\end{figure}
	\\
	\solution
\iffalse
\section{Construction}
\begin{figure}[h]
    \centering
\includegraphics[width=\columnwidth]{line.png}
    \caption{Equation of the slope}
    \label{fig:my_label}
\end{figure}
\vspace{2cm}
\begin{table}[h]
    \centering
    \begin{tabular}{|c|c|c|c|}
       \hline
       \textbf{Symbol}&\textbf{Value}&\textbf{Description}  \\
       \hline
	    $\vec{P}$ & $\myvec{
		    0\\
		    -4}$
	    & Point on Y-axis\\
        \hline
	    $\vec{B}$ & $\myvec{8\\0}$
 & Point on X-axis\\
        \hline
	    $\vec{0}$ & $\myvec{0\\0}$
 & Origin\\
        \hline
    \end{tabular}
    \caption{Parameters}
    \label{tab:my_label}
\end{table}


\section{Solution}
Given that resultant line passes through origin and mid point of the line segment joining point P(0,-4) and B(8,0) \\
\\
\\
given ${\vec{P}}$=$\myvec{
  0\\
  -4}$
 , ${\vec{B}}$=$\myvec{
  8\\
  0}$
  
 \fi 
The mid point of $PB$ is
\begin{align}
\vec{M} &=\frac{1}{2}(\vec{P}+\vec{B})
	= \myvec{4 \\ -2}  
\end{align}
The direction vector of line joining $\vec{O}, \vec{M}$ is 
\begin{align}
\vec{m}&=\vec{O}-\vec{M}
 = -\vec{M}
\end{align}
which can be expressed as
\begin{align}
	\myvec{1 \\ -\frac{1}{2}}
\end{align}
Thus the slope is
\begin{align}
	m = -\frac{1}{2}
\end{align}
\iffalse
\textbf{The direction vector of a line expressed as}
\begin{align}
\implies\vec{m} &= \begin{pmatrix}1 \\ m \\ \end{pmatrix}
\end{align}

\textbf{By solving equation (5) and (6),we get the slope of $\vec{O}$ $\vec{M}$ line}
\begin{align}
        \boxed{m=-0.5}
 \end{align}

\section{Software}
Download the following code using,
\begin{table}[h]
    \centering
    \begin{tabular}{|c|}
    \hline \\
   https://github.com/jyothsna777/jyothsna-fwc.git  \\
         \\
\hline
    \end{tabular}
\end{table}
\\
and execute the code by using command
\begin{center}
\textbf{Python3 lines.py}\\
\end{center}

\section{Conclusion}
Hence the slope of line $\vec{O}$ $\vec{M}$ lineis $\vec{m}$=-0.5

\end{document}
\fi

\item Find the position vector of a point R which divides the line joining two points P and Q whose position vectors are $(2\vec{a}+\vec{b})$ and $(\vec{a}-3\vec{b})$
externally in the ratio 1 : 2. Also, show that P is the mid point of the line segment RQ.\\
	\solution
%		\begin{enumerate}[label=\thesection.\arabic*,ref=\thesection.\theenumi]
\numberwithin{equation}{enumi}
\numberwithin{figure}{enumi}
\numberwithin{table}{enumi}

\item Find the coordinates of the point which divides the join of $(-1,7) \text{ and } (4,-3)$ in the ratio 2:3.
	\\
		\solution
	\input{chapters/10/7/2/1/section.tex}
\item Find the coordinates of the points of trisection of the line segment joining $(4,-1) \text{ and } (-2,3)$.
	\\
		\solution
	\input{chapters/10/7/2/2/section.tex}
\item
	\iffalse
\item To conduct Sports Day activities, in your rectangular shaped school                   
ground ABCD, lines have 
drawn with chalk powder at a                 
distance of 1m each. 100 flower pots have been placed at a distance of 1m 
from each other along AD, as shown 
in Fig. 7.12. Niharika runs $ \frac {1}{4} $th the 
distance AD on the 2nd line and 
posts a green flag. Preet runs $ \frac {1}{5} $th 
the distance AD on the eighth line 
and posts a red flag. What is the 
distance between both the flags? If 
Rashmi has to post a blue flag exactly 
halfway between the line segment 
joining the two flags, where should 
she post her flag?
\begin{figure}[h!]
  \centering
  \includegraphics[width=\columnwidth]{sc.png}
  \caption{}
\label{fig:10/7/12Fig1}
\end{figure}               
\fi
      
\item Find the ratio in which the line segment joining the points $(-3,10) \text{ and } (6,-8)$ $\text{ is divided by } (-1,6)$.
	\\
		\solution
	\input{chapters/10/7/2/4/section.tex}
\item Find the ratio in which the line segment joining $A(1,-5) \text{ and } B(-4,5)$ $\text{is divided by the x-axis}$. Also find the coordinates of the point of division.
\item If $(1,2), (4,y), (x,6), (3,5)$ are the vertices of a parallelogram taken in order, find x and y.
	\\
		\solution
	\input{chapters/10/7/2/6/para1.tex}
\item Find the coordinates of a point A, where AB is the diameter of a circle whose centre is $(2,-3) \text{ and }$ B is $(1,4)$.
	\\
		\solution
	\input{chapters/10/7/2/7/section.tex}
\item If A \text{ and } B are $(-2,-2) \text{ and } (2,-4)$, respectively, find the coordinates of P such that AP= $\frac {3}{7}$AB $\text{ and }$ P lies on the line segment AB.
	\\
		\solution
	\input{chapters/10/7/2/8/section.tex}
\item Find the coordinates of the points which divide the line segment joining $A(-2,2) \text{ and } B(2,8)$ into four equal parts.
	\\
		\solution
	\input{chapters/10/7/2/9/section.tex}
\item Find the area of a rhombus if its vertices are $(3,0), (4,5), (-1,4) \text{ and } (-2,-1)$ taken in order. [$\vec{Hint}$ : Area of rhombus =$\frac {1}{2}$(product of its diagonals)]
	\\
		\solution
	\input{chapters/10/7/2/10/cross.tex}
\item Find the position vector of a point R which divides the line joining two points $\vec{P}$
and $\vec{Q}$ whose position vectors are $\hat{i}+2\hat{j}-\hat{k}$ and $-\hat{i}+\hat{j}+\hat{k}$ respectively, in the
ratio 2 : 1
\begin{enumerate}
    \item  internally
    \item  externally
\end{enumerate}
\solution
		\input{chapters/12/10/2/15/section.tex}
\item Find the position vector of the mid point of the vector joining the points $\vec{P}$(2, 3, 4)
and $\vec{Q}$(4, 1, –2).
\\
\solution
		\input{chapters/12/10/2/16/section.tex}
\item Determine the ratio in which the line $2x+y  - 4=0$ divides the line segment joining the points $\vec{A}(2, - 2)$  and  $\vec{B}(3, 7)$.
\\
\solution
	\input{chapters/10/7/4/1/section.tex}
\item Let $\vec{A}(4, 2), \vec{B}(6, 5)$  and $ \vec{C}(1, 4)$ be the vertices of $\triangle ABC$.
\begin{enumerate}
\item The median from $\vec{A}$ meets $BC$ at $\vec{D}$. Find the coordinates of the point $\vec{D}$.
\item Find the coordinates of the point $\vec{P}$ on $AD$ such that $AP : PD = 2 : 1$.
\item Find the coordinates of points $\vec{Q}$ and $\vec{R}$ on medians $BE$ and $CF$ respectively such that $BQ : QE = 2 : 1$  and  $CR : RF = 2 : 1$.
\item What do you observe?
\item If $\vec{A}, \vec{B}$ and $\vec{C}$  are the vertices of $\triangle ABC$, find the coordinates of the centroid of the triangle.
\end{enumerate}
\solution
	\input{chapters/10/7/4/7/section.tex}
\item Find the slope of a line, which passes through the origin and the mid point of the line segment joining the points $\vec{P}$(0,-4) and $\vec{B}$(8,0).
\label{chapters/11/10/1/5}
\input{chapters/11/10/1/5/matrix.tex}
\item Find the position vector of a point R which divides the line joining two points P and Q whose position vectors are $(2\vec{a}+\vec{b})$ and $(\vec{a}-3\vec{b})$
externally in the ratio 1 : 2. Also, show that P is the mid point of the line segment RQ.\\
	\solution
%		\input{chapters/12/10/5/9/section.tex}

\end{enumerate}



\end{enumerate}


\item Find the position vector of the mid point of the vector joining the points $\vec{P}$(2, 3, 4)
and $\vec{Q}$(4, 1, –2).
\\
\solution
		\begin{enumerate}[label=\thesection.\arabic*,ref=\thesection.\theenumi]
\numberwithin{equation}{enumi}
\numberwithin{figure}{enumi}
\numberwithin{table}{enumi}

\item Find the coordinates of the point which divides the join of $(-1,7) \text{ and } (4,-3)$ in the ratio 2:3.
	\\
		\solution
	\iffalse
\documentclass[12pt]{article}
\usepackage{graphicx}
\usepackage{amsmath}
\usepackage{mathtools}
\usepackage{gensymb}

\newcommand{\mydet}[1]{\ensuremath{\begin{vmatrix}#1\end{vmatrix}}}
\providecommand{\brak}[1]{\ensuremath{\left(#1\right)}}
\providecommand{\norm}[1]{\left\lVert#1\right\rVert}
\newcommand{\solution}{\noindent \textbf{Solution: }}
\newcommand{\myvec}[1]{\ensuremath{\begin{pmatrix}#1\end{pmatrix}}}
\let\vec\mathbf

\begin{document}
\begin{center}
\textbf\large{CHAPTER-7 \\ COORDINATE GEOMETRY}
\end{center}
\section*{Excercise 7.2}

1. Find the coordinates of the point which divides the join $\vec(-1,7) \text{ and } \vec(4,-3)$ in the ratio 2:3 :
\\
\\
\solution\\		
\fi
The coordinates and ratio are given as
\begin{align}
\vec{P}=\myvec{-1\\7\\},
\vec{Q}=\myvec{4\\-3\\},
n=\frac{3}{2}
\end{align}
Using section formula
\begin{align}
\vec{R}&=\frac{\vec{Q}+n\vec{P}}{1+n}\\
&=\frac{1}{1+\frac{3}{2}}  \myvec{\myvec{
4\\
-3\\
}
  +
   \frac{3}{2}\myvec{
-1\\
7\\
}}\\
&=\myvec{
1\\
3
}
\end{align}
See Fig. 
\ref{fig:chapters/10/7/2/1/Fig}
\begin{figure}[!h]
\begin{center}
   \includegraphics[width=\columnwidth]{chapters/10/7/2/1/figs/linefig.png}
\end{center}
\caption{}
\label{fig:chapters/10/7/2/1/Fig}
\end{figure}


\item Find the coordinates of the points of trisection of the line segment joining $(4,-1) \text{ and } (-2,3)$.
	\\
		\solution
	\begin{enumerate}[label=\thesection.\arabic*,ref=\thesection.\theenumi]
\numberwithin{equation}{enumi}
\numberwithin{figure}{enumi}
\numberwithin{table}{enumi}

\item Find the coordinates of the point which divides the join of $(-1,7) \text{ and } (4,-3)$ in the ratio 2:3.
	\\
		\solution
	\input{chapters/10/7/2/1/section.tex}
\item Find the coordinates of the points of trisection of the line segment joining $(4,-1) \text{ and } (-2,3)$.
	\\
		\solution
	\input{chapters/10/7/2/2/section.tex}
\item
	\iffalse
\item To conduct Sports Day activities, in your rectangular shaped school                   
ground ABCD, lines have 
drawn with chalk powder at a                 
distance of 1m each. 100 flower pots have been placed at a distance of 1m 
from each other along AD, as shown 
in Fig. 7.12. Niharika runs $ \frac {1}{4} $th the 
distance AD on the 2nd line and 
posts a green flag. Preet runs $ \frac {1}{5} $th 
the distance AD on the eighth line 
and posts a red flag. What is the 
distance between both the flags? If 
Rashmi has to post a blue flag exactly 
halfway between the line segment 
joining the two flags, where should 
she post her flag?
\begin{figure}[h!]
  \centering
  \includegraphics[width=\columnwidth]{sc.png}
  \caption{}
\label{fig:10/7/12Fig1}
\end{figure}               
\fi
      
\item Find the ratio in which the line segment joining the points $(-3,10) \text{ and } (6,-8)$ $\text{ is divided by } (-1,6)$.
	\\
		\solution
	\input{chapters/10/7/2/4/section.tex}
\item Find the ratio in which the line segment joining $A(1,-5) \text{ and } B(-4,5)$ $\text{is divided by the x-axis}$. Also find the coordinates of the point of division.
\item If $(1,2), (4,y), (x,6), (3,5)$ are the vertices of a parallelogram taken in order, find x and y.
	\\
		\solution
	\input{chapters/10/7/2/6/para1.tex}
\item Find the coordinates of a point A, where AB is the diameter of a circle whose centre is $(2,-3) \text{ and }$ B is $(1,4)$.
	\\
		\solution
	\input{chapters/10/7/2/7/section.tex}
\item If A \text{ and } B are $(-2,-2) \text{ and } (2,-4)$, respectively, find the coordinates of P such that AP= $\frac {3}{7}$AB $\text{ and }$ P lies on the line segment AB.
	\\
		\solution
	\input{chapters/10/7/2/8/section.tex}
\item Find the coordinates of the points which divide the line segment joining $A(-2,2) \text{ and } B(2,8)$ into four equal parts.
	\\
		\solution
	\input{chapters/10/7/2/9/section.tex}
\item Find the area of a rhombus if its vertices are $(3,0), (4,5), (-1,4) \text{ and } (-2,-1)$ taken in order. [$\vec{Hint}$ : Area of rhombus =$\frac {1}{2}$(product of its diagonals)]
	\\
		\solution
	\input{chapters/10/7/2/10/cross.tex}
\item Find the position vector of a point R which divides the line joining two points $\vec{P}$
and $\vec{Q}$ whose position vectors are $\hat{i}+2\hat{j}-\hat{k}$ and $-\hat{i}+\hat{j}+\hat{k}$ respectively, in the
ratio 2 : 1
\begin{enumerate}
    \item  internally
    \item  externally
\end{enumerate}
\solution
		\input{chapters/12/10/2/15/section.tex}
\item Find the position vector of the mid point of the vector joining the points $\vec{P}$(2, 3, 4)
and $\vec{Q}$(4, 1, –2).
\\
\solution
		\input{chapters/12/10/2/16/section.tex}
\item Determine the ratio in which the line $2x+y  - 4=0$ divides the line segment joining the points $\vec{A}(2, - 2)$  and  $\vec{B}(3, 7)$.
\\
\solution
	\input{chapters/10/7/4/1/section.tex}
\item Let $\vec{A}(4, 2), \vec{B}(6, 5)$  and $ \vec{C}(1, 4)$ be the vertices of $\triangle ABC$.
\begin{enumerate}
\item The median from $\vec{A}$ meets $BC$ at $\vec{D}$. Find the coordinates of the point $\vec{D}$.
\item Find the coordinates of the point $\vec{P}$ on $AD$ such that $AP : PD = 2 : 1$.
\item Find the coordinates of points $\vec{Q}$ and $\vec{R}$ on medians $BE$ and $CF$ respectively such that $BQ : QE = 2 : 1$  and  $CR : RF = 2 : 1$.
\item What do you observe?
\item If $\vec{A}, \vec{B}$ and $\vec{C}$  are the vertices of $\triangle ABC$, find the coordinates of the centroid of the triangle.
\end{enumerate}
\solution
	\input{chapters/10/7/4/7/section.tex}
\item Find the slope of a line, which passes through the origin and the mid point of the line segment joining the points $\vec{P}$(0,-4) and $\vec{B}$(8,0).
\label{chapters/11/10/1/5}
\input{chapters/11/10/1/5/matrix.tex}
\item Find the position vector of a point R which divides the line joining two points P and Q whose position vectors are $(2\vec{a}+\vec{b})$ and $(\vec{a}-3\vec{b})$
externally in the ratio 1 : 2. Also, show that P is the mid point of the line segment RQ.\\
	\solution
%		\input{chapters/12/10/5/9/section.tex}

\end{enumerate}


\item
	\iffalse
\item To conduct Sports Day activities, in your rectangular shaped school                   
ground ABCD, lines have 
drawn with chalk powder at a                 
distance of 1m each. 100 flower pots have been placed at a distance of 1m 
from each other along AD, as shown 
in Fig. 7.12. Niharika runs $ \frac {1}{4} $th the 
distance AD on the 2nd line and 
posts a green flag. Preet runs $ \frac {1}{5} $th 
the distance AD on the eighth line 
and posts a red flag. What is the 
distance between both the flags? If 
Rashmi has to post a blue flag exactly 
halfway between the line segment 
joining the two flags, where should 
she post her flag?
\begin{figure}[h!]
  \centering
  \includegraphics[width=\columnwidth]{sc.png}
  \caption{}
\label{fig:10/7/12Fig1}
\end{figure}               
\fi
      
\item Find the ratio in which the line segment joining the points $(-3,10) \text{ and } (6,-8)$ $\text{ is divided by } (-1,6)$.
	\\
		\solution
	\iffalse
\documentclass[12pt]{article}
\usepackage{graphicx}
%\documentclass[journal,12pt,twocolumn]{IEEEtran}
\usepackage[none]{hyphenat}
\usepackage{graphicx}
\usepackage{listings}
\usepackage[english]{babel}
\usepackage{graphicx}
\usepackage{caption} 
\usepackage{hyperref}
\usepackage{booktabs}
\def\inputGnumericTable{}
\usepackage{color}                                            %%
    \usepackage{array}                                            %%
    \usepackage{longtable}                                        %%
    \usepackage{calc}                                             %%
    \usepackage{multirow}                                         %%
    \usepackage{hhline}                                           %%
    \usepackage{ifthen}
\usepackage{array}
\usepackage{amsmath}   % for having text in math mode
\usepackage{listings}
\lstset{
language=tex,
frame=single, 
breaklines=true
}
  
%Following 2 lines were added to remove the blank page at the beginning
\usepackage{atbegshi}% http://ctan.org/pkg/atbegshi
\AtBeginDocument{\AtBeginShipoutNext{\AtBeginShipoutDiscard}}
%
%New macro definitions
\newcommand{\mydet}[1]{\ensuremath{\begin{vmatrix}#1\end{vmatrix}}}
\providecommand{\brak}[1]{\ensuremath{\left(#1\right)}}
\providecommand{\norm}[1]{\left\lVert#1\right\rVert}
\newcommand{\solution}{\noindent \textbf{Solution: }}
\newcommand{\myvec}[1]{\ensuremath{\begin{pmatrix}#1\end{pmatrix}}}
\let\vec\mathbf
\begin{document}
\begin{center}
\title{\textbf{Coordinate Geometry}}
\date{\vspace{-5ex}} %Not to print date automatically
\maketitle
\end{center}
\setcounter{page}{1}
\section*{10$^{th}$ Maths - Chapter 7}
This is Problem-4 from Exercise 7.2
\begin{enumerate}
\item Find the ratio in which the line segement joining the points $\myvec{-3 \\ 10}$ and $\myvec{6\\-8}$ is divided by $\myvec{-1\\6}$.\\
\solution \\
\fi
		The input parameters for this problem are available in Table \eqref{tab:10/7/2/4-1}.
\begin{table}[ht!]
\input{chapters/10/7/2/4/tables/table.tex}
\caption{}
\label{tab:10/7/2/4-1} 
\end{table}
Using section formula,
\begin{align}
         \vec{R} &=\frac{\vec{Q}+n\vec{P}}{1+n}\label{eq:chapters/10/7/2/4/1}
\end{align}
Substituting the values of $\vec{P},\vec{Q}$ and $\vec{R}$ in \eqref{eq:chapters/10/7/2/4/1}
\begin{align}
         \myvec{-1\\6} &=\frac{{\myvec{-3\\10}+n\myvec{6\\-8}}}{1+n}\\
 &=\frac{1}{1+n}\brak{{\myvec{-3\\10}+n\myvec{6\\-8}}} \\
 &=\frac{1}{1+n}\myvec{-3+6n\\10-8n} \label{eq:chapters/10/7/2/4/4}
\end{align}
Simplifying \eqref{eq:chapters/10/7/2/4/4} yeilds,
\begin{align}
          -1 &=\frac{-3+6n}{1+n}\\
\implies          n &=\frac{2}{7}
\end{align}
Also,
\begin{align}
          6 &=\frac{10-8n}{1+n}\\
    \implies      n &=\frac{2}{7}
\end{align}
Hence the desired ratio is $\dfrac{2}{7}$.  
\begin{figure}[!h]
 \begin{center}
  \includegraphics[width=\columnwidth]{chapters/10/7/2/4/figs/fig.png}
 \end{center}
\caption{}
\label{fig:10/7/2/4Fig1}
\end{figure}

\item Find the ratio in which the line segment joining $A(1,-5) \text{ and } B(-4,5)$ $\text{is divided by the x-axis}$. Also find the coordinates of the point of division.
\item If $(1,2), (4,y), (x,6), (3,5)$ are the vertices of a parallelogram taken in order, find x and y.
	\\
		\solution
	\iffalse
\documentclass[12pt]{article}
\usepackage{graphicx}
%\documentclass[journal,12pt,twocolumn]{IEEEtran}
\def\inputGnumericTable{}
\usepackage{color}                                            %%
    \usepackage{array}                                            %%
    \usepackage{longtable}                                        %%
    \usepackage{calc}                                             %%
    \usepackage{multirow}                                         %%
    \usepackage{hhline}                                           %%
    \usepackage{ifthen}
\usepackage[none]{hyphenat}
\usepackage{graphicx}
\usepackage{listings}
\usepackage[english]{babel}
\usepackage{graphicx}
\usepackage{caption} 
\usepackage{hyperref}
\usepackage{booktabs}
\usepackage{array}
\usepackage{amsmath}   % for having text in math mode
\usepackage{listings}
\lstset{
  frame=single,
  breaklines=true
}
  
%Following 2 lines were added to remove the blank page at the beginning
\usepackage{atbegshi}% http://ctan.org/pkg/atbegshi
\AtBeginDocument{\AtBeginShipoutNext{\AtBeginShipoutDiscard}}
%


%New macro definitions
\newcommand{\mydet}[1]{\ensuremath{\begin{vmatrix}#1\end{vmatrix}}}
\providecommand{\brak}[1]{\ensuremath{\left(#1\right)}}
\providecommand{\norm}[1]{\left\lVert#1\right\rVert}
\newcommand{\solution}{\noindent \textbf{Solution: }}
\newcommand{\myvec}[1]{\ensuremath{\begin{pmatrix}#1\end{pmatrix}}}
\let\vec\mathbf

\begin{document}

\begin{center}
\title{\textbf{Properties of Parallelegram}}
\date{\vspace{-5ex}} %Not to print date automatically
\maketitle
\end{center}

\setcounter{page}{1}

\section{10$^{th}$ Maths - Chapter 7}

This is Problem-6 from Exercise 7.2

\begin{enumerate}
\item If $\vec{A}(1, 2),\vec{B}(4, x),\vec{C}(y, 6) \text{and } \vec{D}(3, 5)$ are the vertices of a parallelogram taken in order,find x and y.
\end{enumerate}
\fi

The input parameters for this problem are available in
\ref{table:chapters/10/7/2/6/tables/}.	
\begin{table}[!ht]
	\centering
	\input{chapters/10/7/2/6/tables/table.tex}
\caption{}
\label{table:chapters/10/7/2/6/tables/}	
\end{table}
From the given information,
\begin{align}
  \label{eq:chapters/10/7/2/6/tables/det2f}
	\vec{B}-\vec{A} &= \myvec{4 \\y } - \myvec{1 \\2 }  = \myvec{3 \\y-2 }\\
	\vec{C}-\vec{D} &= \myvec{x \\6 } - \myvec{3 \\5 }  = \myvec{x-3 \\1}
\end{align}
Since $ABCD$ is a parallellogram,
\begin{align}
	\myvec{3\\y-2}&=\myvec{x-3\\1}\\
	\implies x&=6 ,y=3
\end{align}
Fig. \ref{fig:chapters/10/7/2/6/Fig3}
provides a verification.
\begin{figure}[h!]
	\begin{center}
  \includegraphics[width=\columnwidth]{chapters/10/7/2/6/figs/para.pdf}
	\end{center}
\caption{}
\label{fig:chapters/10/7/2/6/Fig3}
\end{figure}


\item Find the coordinates of a point A, where AB is the diameter of a circle whose centre is $(2,-3) \text{ and }$ B is $(1,4)$.
	\\
		\solution
	\iffalse
\documentclass[12pt]{article}
\usepackage{graphicx}
\usepackage{amsmath}
\usepackage{mathtools}
\usepackage{gensymb}

\newcommand{\mydet}[1]{\ensuremath{\begin{vmatrix}#1\end{vmatrix}}}
\providecommand{\brak}[1]{\ensuremath{\left(#1\right)}}
\providecommand{\norm}[1]{\left\lVert#1\right\rVert}
\newcommand{\solution}{\noindent \textbf{Solution: }}
\newcommand{\myvec}[1]{\ensuremath{\begin{pmatrix}#1\end{pmatrix}}}
\let\vec\mathbf

\begin{document}
\begin{center}
\section*{CHAPTER 7 - COORDINATE GEOMETRY}

\end{center}
\section*{Excercise 7.2}

Q7.Find the coordinates of point $\vec{A}$, where AB is the diameter of a circle where the center is (2,-3) and $\vec{B}$ is the point (1,4):

\solution
\begin{enumerate}
\item The coordinates $\vec{B}$ and center $\vec{C}$ are given, where:
	\fi
	Let
	\begin{align}
	\vec{B} = \myvec{
		1\\
	    4\\
		},
	\vec{C} = \myvec{
	    2\\
	   -3\\
		}
	\end{align}
	\iffalse
Let us assume the coordinates of $\vec{A}$. Now, $\vec{C}$ is the center which is midpoint of line AB and $\vec{B}$ is one of the coordinate of diameter AB of a circle.
	\fi	
Hence,	
	\begin{align}
	\vec{C} &= \frac{\vec{A+B}}{2} \\
\implies	2\vec{C} &= \vec{A}+\vec{B} \\
		\text{or, }	\vec{A} &= 2\vec{C}-\vec{B} \\
	 &= \myvec{3\\-10\\}	
	\end{align}       
	See Fig. 
\ref{fig:chapters/10/7/2/7Fig}.
\begin{figure}[!h]
\begin{center}	
	\includegraphics[width=\columnwidth]{chapters/10/7/2/7/figs/Vector1.png}
\end{center}
\caption{}
\label{fig:chapters/10/7/2/7Fig}
\end{figure}
	

\item If A \text{ and } B are $(-2,-2) \text{ and } (2,-4)$, respectively, find the coordinates of P such that AP= $\frac {3}{7}$AB $\text{ and }$ P lies on the line segment AB.
	\\
		\solution
	\iffalse
\documentclass[journal,10pt,twocolumn]{article}
\usepackage{graphicx}
\usepackage[none]{hyphenat}
\usepackage{graphicx}
\usepackage{listings}
\usepackage[english]{babel}
\usepackage{graphicx}
\usepackage{caption} 
\usepackage{booktabs}
\usepackage{array}
\usepackage{amssymb} % for \because
\usepackage{amsmath}   % for having text in math mode
\usepackage{extarrows} % for Row operations arrows
\usepackage{listings}
\usepackage[utf8]{inputenc}
\lstset{
  frame=single,
  breaklines=true
}
\usepackage{hyperref}
  
%Following 2 lines were added to remove the blank page at the beginning
\usepackage{atbegshi}% http://ctan.org/pkg/atbegshi
\AtBeginDocument{\AtBeginShipoutNext{\AtBeginShipoutDiscard}}


%New macro definitions
\newcommand{\mydet}[1]{\ensuremath{\begin{vmatrix}#1\end{vmatrix}}}
\providecommand{\brak}[1]{\ensuremath{\left(#1\right)}}
\newcommand{\solution}{\noindent \textbf{Solution: }}
\newcommand{\myvec}[1]{\ensuremath{\begin{pmatrix}#1\end{pmatrix}}}
\providecommand{\norm}[1]{\left\lVert#1\right\rVert}
\providecommand{\abs}[1]{\left\vert#1\right\vert}
\let\vec\mathbf

\begin{document}

\begin{center}
\title{\textbf{VECTORS}}
\date{\vspace{-5ex}} %Not to print date automatically
\maketitle
\end{center}

\section{10$^{th}$ Maths - EXERCISE-7.2}

\begin{enumerate}
\item If A and B are $(– 2, – 2)\text{ and }(2, – 4)$, respectively, find the coordinates of P such that $AP =\frac{3}{7}AB$ and P lies on the line segment AB. 

\section{SOLUTION}
Given points are
\begin{align}
\vec{A}=\myvec{-2\\ -2} ,
\vec{B}=\myvec{2\\ -4}
\end{align}
The equation of the formula is
\fi
Using section formula, 
\begin{align}
\vec{P}&=\frac{\vec{A}+n\vec{B}}{1+n}
\end{align}
where
\begin{align}
	n =\frac{3}{4}
\end{align}
Thus,
\begin{align}
\vec{P}&=\frac{1}{1+\frac{3}{4}}\brak{\myvec{-2\\-2}+\frac{3}{4}\myvec{2\\-4}}\\
&=\myvec{\frac{-2}{7}\\[1pt] \frac{-20}{7}}
\end{align}
See Fig. 
   \ref{fig:chapters/10/7/2/8/vec.png}
\begin{figure}
   \centering 
 \includegraphics[width=\columnwidth]{chapters/10/7/2/8/figs/vec.png}
   \caption{}
   \label{fig:chapters/10/7/2/8/vec.png}
   \end{figure}

\item Find the coordinates of the points which divide the line segment joining $A(-2,2) \text{ and } B(2,8)$ into four equal parts.
	\\
		\solution
	\begin{enumerate}[label=\thesection.\arabic*,ref=\thesection.\theenumi]
\numberwithin{equation}{enumi}
\numberwithin{figure}{enumi}
\numberwithin{table}{enumi}

\item Find the coordinates of the point which divides the join of $(-1,7) \text{ and } (4,-3)$ in the ratio 2:3.
	\\
		\solution
	\input{chapters/10/7/2/1/section.tex}
\item Find the coordinates of the points of trisection of the line segment joining $(4,-1) \text{ and } (-2,3)$.
	\\
		\solution
	\input{chapters/10/7/2/2/section.tex}
\item
	\iffalse
\item To conduct Sports Day activities, in your rectangular shaped school                   
ground ABCD, lines have 
drawn with chalk powder at a                 
distance of 1m each. 100 flower pots have been placed at a distance of 1m 
from each other along AD, as shown 
in Fig. 7.12. Niharika runs $ \frac {1}{4} $th the 
distance AD on the 2nd line and 
posts a green flag. Preet runs $ \frac {1}{5} $th 
the distance AD on the eighth line 
and posts a red flag. What is the 
distance between both the flags? If 
Rashmi has to post a blue flag exactly 
halfway between the line segment 
joining the two flags, where should 
she post her flag?
\begin{figure}[h!]
  \centering
  \includegraphics[width=\columnwidth]{sc.png}
  \caption{}
\label{fig:10/7/12Fig1}
\end{figure}               
\fi
      
\item Find the ratio in which the line segment joining the points $(-3,10) \text{ and } (6,-8)$ $\text{ is divided by } (-1,6)$.
	\\
		\solution
	\input{chapters/10/7/2/4/section.tex}
\item Find the ratio in which the line segment joining $A(1,-5) \text{ and } B(-4,5)$ $\text{is divided by the x-axis}$. Also find the coordinates of the point of division.
\item If $(1,2), (4,y), (x,6), (3,5)$ are the vertices of a parallelogram taken in order, find x and y.
	\\
		\solution
	\input{chapters/10/7/2/6/para1.tex}
\item Find the coordinates of a point A, where AB is the diameter of a circle whose centre is $(2,-3) \text{ and }$ B is $(1,4)$.
	\\
		\solution
	\input{chapters/10/7/2/7/section.tex}
\item If A \text{ and } B are $(-2,-2) \text{ and } (2,-4)$, respectively, find the coordinates of P such that AP= $\frac {3}{7}$AB $\text{ and }$ P lies on the line segment AB.
	\\
		\solution
	\input{chapters/10/7/2/8/section.tex}
\item Find the coordinates of the points which divide the line segment joining $A(-2,2) \text{ and } B(2,8)$ into four equal parts.
	\\
		\solution
	\input{chapters/10/7/2/9/section.tex}
\item Find the area of a rhombus if its vertices are $(3,0), (4,5), (-1,4) \text{ and } (-2,-1)$ taken in order. [$\vec{Hint}$ : Area of rhombus =$\frac {1}{2}$(product of its diagonals)]
	\\
		\solution
	\input{chapters/10/7/2/10/cross.tex}
\item Find the position vector of a point R which divides the line joining two points $\vec{P}$
and $\vec{Q}$ whose position vectors are $\hat{i}+2\hat{j}-\hat{k}$ and $-\hat{i}+\hat{j}+\hat{k}$ respectively, in the
ratio 2 : 1
\begin{enumerate}
    \item  internally
    \item  externally
\end{enumerate}
\solution
		\input{chapters/12/10/2/15/section.tex}
\item Find the position vector of the mid point of the vector joining the points $\vec{P}$(2, 3, 4)
and $\vec{Q}$(4, 1, –2).
\\
\solution
		\input{chapters/12/10/2/16/section.tex}
\item Determine the ratio in which the line $2x+y  - 4=0$ divides the line segment joining the points $\vec{A}(2, - 2)$  and  $\vec{B}(3, 7)$.
\\
\solution
	\input{chapters/10/7/4/1/section.tex}
\item Let $\vec{A}(4, 2), \vec{B}(6, 5)$  and $ \vec{C}(1, 4)$ be the vertices of $\triangle ABC$.
\begin{enumerate}
\item The median from $\vec{A}$ meets $BC$ at $\vec{D}$. Find the coordinates of the point $\vec{D}$.
\item Find the coordinates of the point $\vec{P}$ on $AD$ such that $AP : PD = 2 : 1$.
\item Find the coordinates of points $\vec{Q}$ and $\vec{R}$ on medians $BE$ and $CF$ respectively such that $BQ : QE = 2 : 1$  and  $CR : RF = 2 : 1$.
\item What do you observe?
\item If $\vec{A}, \vec{B}$ and $\vec{C}$  are the vertices of $\triangle ABC$, find the coordinates of the centroid of the triangle.
\end{enumerate}
\solution
	\input{chapters/10/7/4/7/section.tex}
\item Find the slope of a line, which passes through the origin and the mid point of the line segment joining the points $\vec{P}$(0,-4) and $\vec{B}$(8,0).
\label{chapters/11/10/1/5}
\input{chapters/11/10/1/5/matrix.tex}
\item Find the position vector of a point R which divides the line joining two points P and Q whose position vectors are $(2\vec{a}+\vec{b})$ and $(\vec{a}-3\vec{b})$
externally in the ratio 1 : 2. Also, show that P is the mid point of the line segment RQ.\\
	\solution
%		\input{chapters/12/10/5/9/section.tex}

\end{enumerate}


\item Find the area of a rhombus if its vertices are $(3,0), (4,5), (-1,4) \text{ and } (-2,-1)$ taken in order. [$\vec{Hint}$ : Area of rhombus =$\frac {1}{2}$(product of its diagonals)]
	\\
		\solution
	\iffalse
\documentclass[12pt]{article}
\usepackage{graphicx}
%\documentclass[journal,12pt,twocolumn]{IEEEtran}
\usepackage[none]{hyphenat}
\usepackage{graphicx}
\usepackage{listings}
\usepackage[english]{babel}
\usepackage{graphicx}
\usepackage{caption} 
\usepackage{hyperref}
\usepackage{booktabs}
\def\inputGnumericTable{}
\usepackage{color}                                            %%
    \usepackage{array}                                            %%
    \usepackage{longtable}                                        %%
    \usepackage{calc}                                             %%
    \usepackage{multirow}                                         %%
    \usepackage{hhline}                                           %%
    \usepackage{ifthen}
\usepackage{array}
\usepackage{amsmath}   % for having text in math mode
\usepackage{listings}
\lstset{
language=tex,
frame=single, 
breaklines=true
}
  
%Following 2 lines were added to remove the blank page at the beginning
\usepackage{atbegshi}% http://ctan.org/pkg/atbegshi
\AtBeginDocument{\AtBeginShipoutNext{\AtBeginShipoutDiscard}}
%


%New macro definitions
\newcommand{\mydet}[1]{\ensuremath{\begin{vmatrix}#1\end{vmatrix}}}
\providecommand{\brak}[1]{\ensuremath{\left(#1\right)}}
\providecommand{\norm}[1]{\left\lVert#1\right\rVert}
\newcommand{\solution}{\noindent \textbf{Solution: }}
\newcommand{\myvec}[1]{\ensuremath{\begin{pmatrix}#1\end{pmatrix}}}
\let\vec\mathbf

\begin{document}

\begin{center}
\title{\textbf{Coordinate Geometry}}
\date{\vspace{-5ex}} %Not to print date automatically
\maketitle
\end{center}

\setcounter{page}{1}



\begin{enumerate}

\item\textbf{Problem statement :} Find the area of a rhombus of its vertices are $\myvec{3 ,0}$, $\myvec{4 ,5}$, $\myvec{-1 ,4}$ and $\myvec{-2 ,-1}$taken in order

\solution \\
\fi
The input vertices for this problem are given as
	\begin{align}
	\vec{A} = \myvec{
		3\\
		0
		},
	\vec{B} = \myvec{
		4\\
		5
		},
        \vec{C} = \myvec{
		-1\\
		4
		},
        \vec{D} = \myvec{
		-2\\
		-1
		}
	\end{align}
Since		
\begin{align}
 \vec{A-D}= \myvec{3 \\ 0} - \myvec{-2 \\-1}= \myvec{5\\1}
 \\
  \vec{B-A}= \myvec{4 \\ 5} - \myvec{3 \\0}= \myvec{1\\5}
\end{align}
the area of the rhombus is
\begin{align}
                \norm{\myvec{\vec{A-D}}\times \myvec{\vec{B-A}}}=\mydet{5 & 1\\1 & 5} = 24
\end{align}
See Fig. 
\ref{fig:chapters/10/7/2/10/gFig1}.
\begin{figure}[!h]
 \begin{center}
  \includegraphics[width=\columnwidth]{chapters/10/7/2/10/figs/fig.pdf}
 \end{center}
\caption{}
\label{fig:chapters/10/7/2/10/gFig1}
\end{figure}

\item Find the position vector of a point R which divides the line joining two points $\vec{P}$
and $\vec{Q}$ whose position vectors are $\hat{i}+2\hat{j}-\hat{k}$ and $-\hat{i}+\hat{j}+\hat{k}$ respectively, in the
ratio 2 : 1
\begin{enumerate}
    \item  internally
    \item  externally
\end{enumerate}
\solution
		\begin{enumerate}[label=\thesection.\arabic*,ref=\thesection.\theenumi]
\numberwithin{equation}{enumi}
\numberwithin{figure}{enumi}
\numberwithin{table}{enumi}

\item Find the coordinates of the point which divides the join of $(-1,7) \text{ and } (4,-3)$ in the ratio 2:3.
	\\
		\solution
	\input{chapters/10/7/2/1/section.tex}
\item Find the coordinates of the points of trisection of the line segment joining $(4,-1) \text{ and } (-2,3)$.
	\\
		\solution
	\input{chapters/10/7/2/2/section.tex}
\item
	\iffalse
\item To conduct Sports Day activities, in your rectangular shaped school                   
ground ABCD, lines have 
drawn with chalk powder at a                 
distance of 1m each. 100 flower pots have been placed at a distance of 1m 
from each other along AD, as shown 
in Fig. 7.12. Niharika runs $ \frac {1}{4} $th the 
distance AD on the 2nd line and 
posts a green flag. Preet runs $ \frac {1}{5} $th 
the distance AD on the eighth line 
and posts a red flag. What is the 
distance between both the flags? If 
Rashmi has to post a blue flag exactly 
halfway between the line segment 
joining the two flags, where should 
she post her flag?
\begin{figure}[h!]
  \centering
  \includegraphics[width=\columnwidth]{sc.png}
  \caption{}
\label{fig:10/7/12Fig1}
\end{figure}               
\fi
      
\item Find the ratio in which the line segment joining the points $(-3,10) \text{ and } (6,-8)$ $\text{ is divided by } (-1,6)$.
	\\
		\solution
	\input{chapters/10/7/2/4/section.tex}
\item Find the ratio in which the line segment joining $A(1,-5) \text{ and } B(-4,5)$ $\text{is divided by the x-axis}$. Also find the coordinates of the point of division.
\item If $(1,2), (4,y), (x,6), (3,5)$ are the vertices of a parallelogram taken in order, find x and y.
	\\
		\solution
	\input{chapters/10/7/2/6/para1.tex}
\item Find the coordinates of a point A, where AB is the diameter of a circle whose centre is $(2,-3) \text{ and }$ B is $(1,4)$.
	\\
		\solution
	\input{chapters/10/7/2/7/section.tex}
\item If A \text{ and } B are $(-2,-2) \text{ and } (2,-4)$, respectively, find the coordinates of P such that AP= $\frac {3}{7}$AB $\text{ and }$ P lies on the line segment AB.
	\\
		\solution
	\input{chapters/10/7/2/8/section.tex}
\item Find the coordinates of the points which divide the line segment joining $A(-2,2) \text{ and } B(2,8)$ into four equal parts.
	\\
		\solution
	\input{chapters/10/7/2/9/section.tex}
\item Find the area of a rhombus if its vertices are $(3,0), (4,5), (-1,4) \text{ and } (-2,-1)$ taken in order. [$\vec{Hint}$ : Area of rhombus =$\frac {1}{2}$(product of its diagonals)]
	\\
		\solution
	\input{chapters/10/7/2/10/cross.tex}
\item Find the position vector of a point R which divides the line joining two points $\vec{P}$
and $\vec{Q}$ whose position vectors are $\hat{i}+2\hat{j}-\hat{k}$ and $-\hat{i}+\hat{j}+\hat{k}$ respectively, in the
ratio 2 : 1
\begin{enumerate}
    \item  internally
    \item  externally
\end{enumerate}
\solution
		\input{chapters/12/10/2/15/section.tex}
\item Find the position vector of the mid point of the vector joining the points $\vec{P}$(2, 3, 4)
and $\vec{Q}$(4, 1, –2).
\\
\solution
		\input{chapters/12/10/2/16/section.tex}
\item Determine the ratio in which the line $2x+y  - 4=0$ divides the line segment joining the points $\vec{A}(2, - 2)$  and  $\vec{B}(3, 7)$.
\\
\solution
	\input{chapters/10/7/4/1/section.tex}
\item Let $\vec{A}(4, 2), \vec{B}(6, 5)$  and $ \vec{C}(1, 4)$ be the vertices of $\triangle ABC$.
\begin{enumerate}
\item The median from $\vec{A}$ meets $BC$ at $\vec{D}$. Find the coordinates of the point $\vec{D}$.
\item Find the coordinates of the point $\vec{P}$ on $AD$ such that $AP : PD = 2 : 1$.
\item Find the coordinates of points $\vec{Q}$ and $\vec{R}$ on medians $BE$ and $CF$ respectively such that $BQ : QE = 2 : 1$  and  $CR : RF = 2 : 1$.
\item What do you observe?
\item If $\vec{A}, \vec{B}$ and $\vec{C}$  are the vertices of $\triangle ABC$, find the coordinates of the centroid of the triangle.
\end{enumerate}
\solution
	\input{chapters/10/7/4/7/section.tex}
\item Find the slope of a line, which passes through the origin and the mid point of the line segment joining the points $\vec{P}$(0,-4) and $\vec{B}$(8,0).
\label{chapters/11/10/1/5}
\input{chapters/11/10/1/5/matrix.tex}
\item Find the position vector of a point R which divides the line joining two points P and Q whose position vectors are $(2\vec{a}+\vec{b})$ and $(\vec{a}-3\vec{b})$
externally in the ratio 1 : 2. Also, show that P is the mid point of the line segment RQ.\\
	\solution
%		\input{chapters/12/10/5/9/section.tex}

\end{enumerate}


\item Find the position vector of the mid point of the vector joining the points $\vec{P}$(2, 3, 4)
and $\vec{Q}$(4, 1, –2).
\\
\solution
		\begin{enumerate}[label=\thesection.\arabic*,ref=\thesection.\theenumi]
\numberwithin{equation}{enumi}
\numberwithin{figure}{enumi}
\numberwithin{table}{enumi}

\item Find the coordinates of the point which divides the join of $(-1,7) \text{ and } (4,-3)$ in the ratio 2:3.
	\\
		\solution
	\input{chapters/10/7/2/1/section.tex}
\item Find the coordinates of the points of trisection of the line segment joining $(4,-1) \text{ and } (-2,3)$.
	\\
		\solution
	\input{chapters/10/7/2/2/section.tex}
\item
	\iffalse
\item To conduct Sports Day activities, in your rectangular shaped school                   
ground ABCD, lines have 
drawn with chalk powder at a                 
distance of 1m each. 100 flower pots have been placed at a distance of 1m 
from each other along AD, as shown 
in Fig. 7.12. Niharika runs $ \frac {1}{4} $th the 
distance AD on the 2nd line and 
posts a green flag. Preet runs $ \frac {1}{5} $th 
the distance AD on the eighth line 
and posts a red flag. What is the 
distance between both the flags? If 
Rashmi has to post a blue flag exactly 
halfway between the line segment 
joining the two flags, where should 
she post her flag?
\begin{figure}[h!]
  \centering
  \includegraphics[width=\columnwidth]{sc.png}
  \caption{}
\label{fig:10/7/12Fig1}
\end{figure}               
\fi
      
\item Find the ratio in which the line segment joining the points $(-3,10) \text{ and } (6,-8)$ $\text{ is divided by } (-1,6)$.
	\\
		\solution
	\input{chapters/10/7/2/4/section.tex}
\item Find the ratio in which the line segment joining $A(1,-5) \text{ and } B(-4,5)$ $\text{is divided by the x-axis}$. Also find the coordinates of the point of division.
\item If $(1,2), (4,y), (x,6), (3,5)$ are the vertices of a parallelogram taken in order, find x and y.
	\\
		\solution
	\input{chapters/10/7/2/6/para1.tex}
\item Find the coordinates of a point A, where AB is the diameter of a circle whose centre is $(2,-3) \text{ and }$ B is $(1,4)$.
	\\
		\solution
	\input{chapters/10/7/2/7/section.tex}
\item If A \text{ and } B are $(-2,-2) \text{ and } (2,-4)$, respectively, find the coordinates of P such that AP= $\frac {3}{7}$AB $\text{ and }$ P lies on the line segment AB.
	\\
		\solution
	\input{chapters/10/7/2/8/section.tex}
\item Find the coordinates of the points which divide the line segment joining $A(-2,2) \text{ and } B(2,8)$ into four equal parts.
	\\
		\solution
	\input{chapters/10/7/2/9/section.tex}
\item Find the area of a rhombus if its vertices are $(3,0), (4,5), (-1,4) \text{ and } (-2,-1)$ taken in order. [$\vec{Hint}$ : Area of rhombus =$\frac {1}{2}$(product of its diagonals)]
	\\
		\solution
	\input{chapters/10/7/2/10/cross.tex}
\item Find the position vector of a point R which divides the line joining two points $\vec{P}$
and $\vec{Q}$ whose position vectors are $\hat{i}+2\hat{j}-\hat{k}$ and $-\hat{i}+\hat{j}+\hat{k}$ respectively, in the
ratio 2 : 1
\begin{enumerate}
    \item  internally
    \item  externally
\end{enumerate}
\solution
		\input{chapters/12/10/2/15/section.tex}
\item Find the position vector of the mid point of the vector joining the points $\vec{P}$(2, 3, 4)
and $\vec{Q}$(4, 1, –2).
\\
\solution
		\input{chapters/12/10/2/16/section.tex}
\item Determine the ratio in which the line $2x+y  - 4=0$ divides the line segment joining the points $\vec{A}(2, - 2)$  and  $\vec{B}(3, 7)$.
\\
\solution
	\input{chapters/10/7/4/1/section.tex}
\item Let $\vec{A}(4, 2), \vec{B}(6, 5)$  and $ \vec{C}(1, 4)$ be the vertices of $\triangle ABC$.
\begin{enumerate}
\item The median from $\vec{A}$ meets $BC$ at $\vec{D}$. Find the coordinates of the point $\vec{D}$.
\item Find the coordinates of the point $\vec{P}$ on $AD$ such that $AP : PD = 2 : 1$.
\item Find the coordinates of points $\vec{Q}$ and $\vec{R}$ on medians $BE$ and $CF$ respectively such that $BQ : QE = 2 : 1$  and  $CR : RF = 2 : 1$.
\item What do you observe?
\item If $\vec{A}, \vec{B}$ and $\vec{C}$  are the vertices of $\triangle ABC$, find the coordinates of the centroid of the triangle.
\end{enumerate}
\solution
	\input{chapters/10/7/4/7/section.tex}
\item Find the slope of a line, which passes through the origin and the mid point of the line segment joining the points $\vec{P}$(0,-4) and $\vec{B}$(8,0).
\label{chapters/11/10/1/5}
\input{chapters/11/10/1/5/matrix.tex}
\item Find the position vector of a point R which divides the line joining two points P and Q whose position vectors are $(2\vec{a}+\vec{b})$ and $(\vec{a}-3\vec{b})$
externally in the ratio 1 : 2. Also, show that P is the mid point of the line segment RQ.\\
	\solution
%		\input{chapters/12/10/5/9/section.tex}

\end{enumerate}


\item Determine the ratio in which the line $2x+y  - 4=0$ divides the line segment joining the points $\vec{A}(2, - 2)$  and  $\vec{B}(3, 7)$.
\\
\solution
	\iffalse
\documentclass[journal,12pt,twocolumn]{IEEEtran}
\usepackage{graphicx}
\graphicspath{{./chapters/10/7/4/1/figs/}}{}
\usepackage{amsmath,amssymb,amsfonts,amsthm}
\newcommand{\myvec}[1]{\ensuremath{\begin{pmatrix}#1\end{pmatrix}}}
\providecommand{\norm}[1]{\lVert#1\rVert}
\usepackage{listings}
\usepackage{watermark}
\usepackage{titlesec}
\usepackage{caption}
\let\vec\mathbf
\lstset{
frame=single, 
breaklines=true,
columns=fullflexible
}
\thiswatermark{\centering \put(0,-105.0){\includegraphics[scale=0.15]{/sdcard/IITH/vector/vectpr-4/chapters/10/7/4/1/figs/logo.png}} }
\title{\mytitle}
\title{
Assignment - Vector-4
}
\author{Surajit Sarkar}
\begin{document}
\maketitle
%\tableofcontents
\bigskip
\section{\textbf{Problem}}
Determine the ratio in which the line 2x+y–4=0 divides the line segment joining the points A(2,–2) and B(3,7).
\section{\textbf{Solution}}
\begin{table}[h]
    \centering
    \begin{tabular}{|c|c|}
       \hline
       \textbf{Symbol}&\textbf{Value}  \\
       \hline
	    $\vec{A}$ & $\myvec{2\\-2}$\\
        \hline
	    $\vec{B}$ & $\myvec{3\\7}$\\
        \hline
	    c&$4$\\
        \hline
       $\vec{n}$ & $\myvec{2\\1}$\\
       \hline
    \end{tabular}
    \caption{Parameters}
    \label{tab:my_label}
\end{table}
Given equation
\fi
The given equation can be expressed as
\begin{align}
    \myvec{2&1}\vec{x}&=4\\
\end{align}
Using section formula, the point of division 
\begin{align}
    \vec{P} = \frac{k\vec{B+A}}{k+1}
\end{align}
which upon substitution in the equation of a line yields
\begin{align}
    \implies\vec{n}^{\top}\myvec{\frac{k\vec{B+A}}{k+1}}&=c\\
    \implies k&=\frac{c-\vec{n}^{\top}\vec{A}}{\vec{n}^{\top}\vec{B}-c}\\
\end{align}
upon simplification.  Substituting numerical values, 
\begin{align}
    k=\frac{2}{9}
\end{align}
See Fig. 
\ref{fig:chapters/10/7/4/1vec}.
\begin{figure}[!h]
\centering
\includegraphics[width=\columnwidth]{chapters/10/7/4/1/figs/vec.pdf}
\caption{}
\label{fig:chapters/10/7/4/1vec}
\end{figure}


\item Let $\vec{A}(4, 2), \vec{B}(6, 5)$  and $ \vec{C}(1, 4)$ be the vertices of $\triangle ABC$.
\begin{enumerate}
\item The median from $\vec{A}$ meets $BC$ at $\vec{D}$. Find the coordinates of the point $\vec{D}$.
\item Find the coordinates of the point $\vec{P}$ on $AD$ such that $AP : PD = 2 : 1$.
\item Find the coordinates of points $\vec{Q}$ and $\vec{R}$ on medians $BE$ and $CF$ respectively such that $BQ : QE = 2 : 1$  and  $CR : RF = 2 : 1$.
\item What do you observe?
\item If $\vec{A}, \vec{B}$ and $\vec{C}$  are the vertices of $\triangle ABC$, find the coordinates of the centroid of the triangle.
\end{enumerate}
\solution
	\iffalse
\documentclass[12pt]{article}
\usepackage{graphicx}
\usepackage[none]{hyphenat}
\usepackage{graphicx}
\usepackage{listings}
\usepackage[english]{babel}
\usepackage{graphicx}
\usepackage{caption} 
\usepackage{booktabs}
\usepackage{array}
\usepackage{amssymb} % for \because
\usepackage{amsmath}   % for having text in math mode
\usepackage{extarrows} % for Row operations arrows
\usepackage{listings}
\usepackage[utf8]{inputenc}
\lstset{
  frame=single,
  breaklines=true
}
\usepackage{hyperref}
  
%Following 2 lines were added to remove the blank page at the beginning
\usepackage{atbegshi}% http://ctan.org/pkg/atbegshi
\AtBeginDocument{\AtBeginShipoutNext{\AtBeginShipoutDiscard}}


%New macro definitions
\newcommand{\mydet}[1]{\ensuremath{\begin{vmatrix}#1\end{vmatrix}}}
\providecommand{\brak}[1]{\ensuremath{\left(#1\right)}}
\newcommand{\solution}{\noindent \textbf{Solution: }}
\newcommand{\myvec}[1]{\ensuremath{\begin{pmatrix}#1\end{pmatrix}}}
\providecommand{\norm}[1]{\left\lVert#1\right\rVert}
\providecommand{\abs}[1]{\left\vert#1\right\vert}
\let\vec\mathbf

\begin{document}

\begin{center}
\title{\textbf{VECTORS}}
\date{\vspace{-5ex}} %Not to print date automatically
\maketitle
\end{center}

\section{10$^{th}$ Maths - EXERCISE-7.4}

Let A(4, 2), B(6, 5) and C(1, 4) be the vertices of $\triangle ABC$
\begin{enumerate}
\item The median from A meets BC at D. Find the coordinates of the point D.
\item Find the coordinates of the point P on AD such that $AP : PD = 2 : 1$
\item Find the coordinates of points Q and R on medians BE and CF respectively such
that $BQ : QE = 2 : 1 \text{and} CR : RF = 2 : 1.$
\item What do yo observe?
\item If $A(x_1, y_1), B(x_2, y_2) \text{and} C(x_3, y_3)$ are the vertices of $\triangle ABC$, find the coordinates of the centroid of the triangle.
\end{enumerate}

Given points are
\begin{align}
\vec{A}=\myvec{4\\ 2} ,
\vec{B}=\myvec{6\\ 5} ,
\vec{C}=\myvec{1\\ 4}
\end{align}
\fi

\begin{enumerate}
\item 
\begin{align}
\vec{D}&=\frac{\vec{B}+\vec{C}}{2}\\
&=\myvec{\frac{7}{2}\\[2pt] \frac{9}{2}}\\
\vec{E}&=\frac{\vec{A}+\vec{C}}{2}\\
&=\myvec{\frac{5}{2}\\ 3}\\
\vec{F}&=\frac{\vec{A}+\vec{B}}{2}\\
&=\myvec{5\\ \frac{7}{2}}
\end{align}

\item 
	For
$n=2$,
\begin{align}
\vec{P}&=\frac{1}{1+n}\brak{\myvec{\vec{A}+n\vec{D}}}\\
&=\frac{1}{3}\myvec{11\\11}
\end{align}

\item 
\begin{align}
\vec{Q}&=\frac{1}{1+n}\brak{\myvec{\vec{B}+n\vec{E}}}\\
&=\frac{1}{3}\myvec{11\\11}\\
\vec{R}&=\frac{1}{1+n}\brak{\myvec{\vec{C}+n\vec{F}}}\\
&=\frac{1}{3}\myvec{11\\11}\\
\end{align}

\item 
 $\vec{P},\vec{Q},\vec{R}$ are the same point.
   
\item 
\begin{align}
\vec{G}&=\frac{\vec{D}+\vec{E}+\vec{F}}{3}\\
&=\frac{1}{3}\myvec{11\\11}\\
\end{align} 
\end{enumerate}
See Fig.  
  \ref{fig:chapters/10/7/4/7/Figure}.
\begin{figure}[h!]
\centering
\includegraphics[width=\columnwidth]{chapters/10/7/4/7/figs/dj.pdf}
\caption{}
  \label{fig:chapters/10/7/4/7/Figure}
\end{figure}

\item Find the slope of a line, which passes through the origin and the mid point of the line segment joining the points $\vec{P}$(0,-4) and $\vec{B}$(8,0).
\label{chapters/11/10/1/5}
\iffalse
\documentclass[journal,12pt,twocolumn]{IEEEtran}
\usepackage{graphicx}
\graphicspath{{./figs/}}{}
\usepackage{amsmath,amssymb,amsfonts,amsthm}
\newcommand{\myvec}[1]{\ensuremath{\begin{pmatrix}#1\end{pmatrix}}}

\let\vec\mathbf

\title{
Matrix-Lines
}
\author{Jyothsna Paluchuri-FWC22059\\}
\begin{document}
\maketitle
\tableofcontents
\bigskip
\section{Problem Statement}
\fi
	\begin{figure}[!ht]
		\centering
 \includegraphics[width=\columnwidth]{chapters/11/10/1/5/figs/line.png}
		\caption{}
		\label{fig:11/10/1/5}
  	\end{figure}
	\\
	\solution
\iffalse
\section{Construction}
\begin{figure}[h]
    \centering
\includegraphics[width=\columnwidth]{line.png}
    \caption{Equation of the slope}
    \label{fig:my_label}
\end{figure}
\vspace{2cm}
\begin{table}[h]
    \centering
    \begin{tabular}{|c|c|c|c|}
       \hline
       \textbf{Symbol}&\textbf{Value}&\textbf{Description}  \\
       \hline
	    $\vec{P}$ & $\myvec{
		    0\\
		    -4}$
	    & Point on Y-axis\\
        \hline
	    $\vec{B}$ & $\myvec{8\\0}$
 & Point on X-axis\\
        \hline
	    $\vec{0}$ & $\myvec{0\\0}$
 & Origin\\
        \hline
    \end{tabular}
    \caption{Parameters}
    \label{tab:my_label}
\end{table}


\section{Solution}
Given that resultant line passes through origin and mid point of the line segment joining point P(0,-4) and B(8,0) \\
\\
\\
given ${\vec{P}}$=$\myvec{
  0\\
  -4}$
 , ${\vec{B}}$=$\myvec{
  8\\
  0}$
  
 \fi 
The mid point of $PB$ is
\begin{align}
\vec{M} &=\frac{1}{2}(\vec{P}+\vec{B})
	= \myvec{4 \\ -2}  
\end{align}
The direction vector of line joining $\vec{O}, \vec{M}$ is 
\begin{align}
\vec{m}&=\vec{O}-\vec{M}
 = -\vec{M}
\end{align}
which can be expressed as
\begin{align}
	\myvec{1 \\ -\frac{1}{2}}
\end{align}
Thus the slope is
\begin{align}
	m = -\frac{1}{2}
\end{align}
\iffalse
\textbf{The direction vector of a line expressed as}
\begin{align}
\implies\vec{m} &= \begin{pmatrix}1 \\ m \\ \end{pmatrix}
\end{align}

\textbf{By solving equation (5) and (6),we get the slope of $\vec{O}$ $\vec{M}$ line}
\begin{align}
        \boxed{m=-0.5}
 \end{align}

\section{Software}
Download the following code using,
\begin{table}[h]
    \centering
    \begin{tabular}{|c|}
    \hline \\
   https://github.com/jyothsna777/jyothsna-fwc.git  \\
         \\
\hline
    \end{tabular}
\end{table}
\\
and execute the code by using command
\begin{center}
\textbf{Python3 lines.py}\\
\end{center}

\section{Conclusion}
Hence the slope of line $\vec{O}$ $\vec{M}$ lineis $\vec{m}$=-0.5

\end{document}
\fi

\item Find the position vector of a point R which divides the line joining two points P and Q whose position vectors are $(2\vec{a}+\vec{b})$ and $(\vec{a}-3\vec{b})$
externally in the ratio 1 : 2. Also, show that P is the mid point of the line segment RQ.\\
	\solution
%		\begin{enumerate}[label=\thesection.\arabic*,ref=\thesection.\theenumi]
\numberwithin{equation}{enumi}
\numberwithin{figure}{enumi}
\numberwithin{table}{enumi}

\item Find the coordinates of the point which divides the join of $(-1,7) \text{ and } (4,-3)$ in the ratio 2:3.
	\\
		\solution
	\input{chapters/10/7/2/1/section.tex}
\item Find the coordinates of the points of trisection of the line segment joining $(4,-1) \text{ and } (-2,3)$.
	\\
		\solution
	\input{chapters/10/7/2/2/section.tex}
\item
	\iffalse
\item To conduct Sports Day activities, in your rectangular shaped school                   
ground ABCD, lines have 
drawn with chalk powder at a                 
distance of 1m each. 100 flower pots have been placed at a distance of 1m 
from each other along AD, as shown 
in Fig. 7.12. Niharika runs $ \frac {1}{4} $th the 
distance AD on the 2nd line and 
posts a green flag. Preet runs $ \frac {1}{5} $th 
the distance AD on the eighth line 
and posts a red flag. What is the 
distance between both the flags? If 
Rashmi has to post a blue flag exactly 
halfway between the line segment 
joining the two flags, where should 
she post her flag?
\begin{figure}[h!]
  \centering
  \includegraphics[width=\columnwidth]{sc.png}
  \caption{}
\label{fig:10/7/12Fig1}
\end{figure}               
\fi
      
\item Find the ratio in which the line segment joining the points $(-3,10) \text{ and } (6,-8)$ $\text{ is divided by } (-1,6)$.
	\\
		\solution
	\input{chapters/10/7/2/4/section.tex}
\item Find the ratio in which the line segment joining $A(1,-5) \text{ and } B(-4,5)$ $\text{is divided by the x-axis}$. Also find the coordinates of the point of division.
\item If $(1,2), (4,y), (x,6), (3,5)$ are the vertices of a parallelogram taken in order, find x and y.
	\\
		\solution
	\input{chapters/10/7/2/6/para1.tex}
\item Find the coordinates of a point A, where AB is the diameter of a circle whose centre is $(2,-3) \text{ and }$ B is $(1,4)$.
	\\
		\solution
	\input{chapters/10/7/2/7/section.tex}
\item If A \text{ and } B are $(-2,-2) \text{ and } (2,-4)$, respectively, find the coordinates of P such that AP= $\frac {3}{7}$AB $\text{ and }$ P lies on the line segment AB.
	\\
		\solution
	\input{chapters/10/7/2/8/section.tex}
\item Find the coordinates of the points which divide the line segment joining $A(-2,2) \text{ and } B(2,8)$ into four equal parts.
	\\
		\solution
	\input{chapters/10/7/2/9/section.tex}
\item Find the area of a rhombus if its vertices are $(3,0), (4,5), (-1,4) \text{ and } (-2,-1)$ taken in order. [$\vec{Hint}$ : Area of rhombus =$\frac {1}{2}$(product of its diagonals)]
	\\
		\solution
	\input{chapters/10/7/2/10/cross.tex}
\item Find the position vector of a point R which divides the line joining two points $\vec{P}$
and $\vec{Q}$ whose position vectors are $\hat{i}+2\hat{j}-\hat{k}$ and $-\hat{i}+\hat{j}+\hat{k}$ respectively, in the
ratio 2 : 1
\begin{enumerate}
    \item  internally
    \item  externally
\end{enumerate}
\solution
		\input{chapters/12/10/2/15/section.tex}
\item Find the position vector of the mid point of the vector joining the points $\vec{P}$(2, 3, 4)
and $\vec{Q}$(4, 1, –2).
\\
\solution
		\input{chapters/12/10/2/16/section.tex}
\item Determine the ratio in which the line $2x+y  - 4=0$ divides the line segment joining the points $\vec{A}(2, - 2)$  and  $\vec{B}(3, 7)$.
\\
\solution
	\input{chapters/10/7/4/1/section.tex}
\item Let $\vec{A}(4, 2), \vec{B}(6, 5)$  and $ \vec{C}(1, 4)$ be the vertices of $\triangle ABC$.
\begin{enumerate}
\item The median from $\vec{A}$ meets $BC$ at $\vec{D}$. Find the coordinates of the point $\vec{D}$.
\item Find the coordinates of the point $\vec{P}$ on $AD$ such that $AP : PD = 2 : 1$.
\item Find the coordinates of points $\vec{Q}$ and $\vec{R}$ on medians $BE$ and $CF$ respectively such that $BQ : QE = 2 : 1$  and  $CR : RF = 2 : 1$.
\item What do you observe?
\item If $\vec{A}, \vec{B}$ and $\vec{C}$  are the vertices of $\triangle ABC$, find the coordinates of the centroid of the triangle.
\end{enumerate}
\solution
	\input{chapters/10/7/4/7/section.tex}
\item Find the slope of a line, which passes through the origin and the mid point of the line segment joining the points $\vec{P}$(0,-4) and $\vec{B}$(8,0).
\label{chapters/11/10/1/5}
\input{chapters/11/10/1/5/matrix.tex}
\item Find the position vector of a point R which divides the line joining two points P and Q whose position vectors are $(2\vec{a}+\vec{b})$ and $(\vec{a}-3\vec{b})$
externally in the ratio 1 : 2. Also, show that P is the mid point of the line segment RQ.\\
	\solution
%		\input{chapters/12/10/5/9/section.tex}

\end{enumerate}



\end{enumerate}


\item Determine the ratio in which the line $2x+y  - 4=0$ divides the line segment joining the points $\vec{A}(2, - 2)$  and  $\vec{B}(3, 7)$.
\\
\solution
	\iffalse
\documentclass[journal,12pt,twocolumn]{IEEEtran}
\usepackage{graphicx}
\graphicspath{{./chapters/10/7/4/1/figs/}}{}
\usepackage{amsmath,amssymb,amsfonts,amsthm}
\newcommand{\myvec}[1]{\ensuremath{\begin{pmatrix}#1\end{pmatrix}}}
\providecommand{\norm}[1]{\lVert#1\rVert}
\usepackage{listings}
\usepackage{watermark}
\usepackage{titlesec}
\usepackage{caption}
\let\vec\mathbf
\lstset{
frame=single, 
breaklines=true,
columns=fullflexible
}
\thiswatermark{\centering \put(0,-105.0){\includegraphics[scale=0.15]{/sdcard/IITH/vector/vectpr-4/chapters/10/7/4/1/figs/logo.png}} }
\title{\mytitle}
\title{
Assignment - Vector-4
}
\author{Surajit Sarkar}
\begin{document}
\maketitle
%\tableofcontents
\bigskip
\section{\textbf{Problem}}
Determine the ratio in which the line 2x+y–4=0 divides the line segment joining the points A(2,–2) and B(3,7).
\section{\textbf{Solution}}
\begin{table}[h]
    \centering
    \begin{tabular}{|c|c|}
       \hline
       \textbf{Symbol}&\textbf{Value}  \\
       \hline
	    $\vec{A}$ & $\myvec{2\\-2}$\\
        \hline
	    $\vec{B}$ & $\myvec{3\\7}$\\
        \hline
	    c&$4$\\
        \hline
       $\vec{n}$ & $\myvec{2\\1}$\\
       \hline
    \end{tabular}
    \caption{Parameters}
    \label{tab:my_label}
\end{table}
Given equation
\fi
The given equation can be expressed as
\begin{align}
    \myvec{2&1}\vec{x}&=4\\
\end{align}
Using section formula, the point of division 
\begin{align}
    \vec{P} = \frac{k\vec{B+A}}{k+1}
\end{align}
which upon substitution in the equation of a line yields
\begin{align}
    \implies\vec{n}^{\top}\myvec{\frac{k\vec{B+A}}{k+1}}&=c\\
    \implies k&=\frac{c-\vec{n}^{\top}\vec{A}}{\vec{n}^{\top}\vec{B}-c}\\
\end{align}
upon simplification.  Substituting numerical values, 
\begin{align}
    k=\frac{2}{9}
\end{align}
See Fig. 
\ref{fig:chapters/10/7/4/1vec}.
\begin{figure}[!h]
\centering
\includegraphics[width=\columnwidth]{chapters/10/7/4/1/figs/vec.pdf}
\caption{}
\label{fig:chapters/10/7/4/1vec}
\end{figure}


\item Let $\vec{A}(4, 2), \vec{B}(6, 5)$  and $ \vec{C}(1, 4)$ be the vertices of $\triangle ABC$.
\begin{enumerate}
\item The median from $\vec{A}$ meets $BC$ at $\vec{D}$. Find the coordinates of the point $\vec{D}$.
\item Find the coordinates of the point $\vec{P}$ on $AD$ such that $AP : PD = 2 : 1$.
\item Find the coordinates of points $\vec{Q}$ and $\vec{R}$ on medians $BE$ and $CF$ respectively such that $BQ : QE = 2 : 1$  and  $CR : RF = 2 : 1$.
\item What do you observe?
\item If $\vec{A}, \vec{B}$ and $\vec{C}$  are the vertices of $\triangle ABC$, find the coordinates of the centroid of the triangle.
\end{enumerate}
\solution
	\iffalse
\documentclass[12pt]{article}
\usepackage{graphicx}
\usepackage[none]{hyphenat}
\usepackage{graphicx}
\usepackage{listings}
\usepackage[english]{babel}
\usepackage{graphicx}
\usepackage{caption} 
\usepackage{booktabs}
\usepackage{array}
\usepackage{amssymb} % for \because
\usepackage{amsmath}   % for having text in math mode
\usepackage{extarrows} % for Row operations arrows
\usepackage{listings}
\usepackage[utf8]{inputenc}
\lstset{
  frame=single,
  breaklines=true
}
\usepackage{hyperref}
  
%Following 2 lines were added to remove the blank page at the beginning
\usepackage{atbegshi}% http://ctan.org/pkg/atbegshi
\AtBeginDocument{\AtBeginShipoutNext{\AtBeginShipoutDiscard}}


%New macro definitions
\newcommand{\mydet}[1]{\ensuremath{\begin{vmatrix}#1\end{vmatrix}}}
\providecommand{\brak}[1]{\ensuremath{\left(#1\right)}}
\newcommand{\solution}{\noindent \textbf{Solution: }}
\newcommand{\myvec}[1]{\ensuremath{\begin{pmatrix}#1\end{pmatrix}}}
\providecommand{\norm}[1]{\left\lVert#1\right\rVert}
\providecommand{\abs}[1]{\left\vert#1\right\vert}
\let\vec\mathbf

\begin{document}

\begin{center}
\title{\textbf{VECTORS}}
\date{\vspace{-5ex}} %Not to print date automatically
\maketitle
\end{center}

\section{10$^{th}$ Maths - EXERCISE-7.4}

Let A(4, 2), B(6, 5) and C(1, 4) be the vertices of $\triangle ABC$
\begin{enumerate}
\item The median from A meets BC at D. Find the coordinates of the point D.
\item Find the coordinates of the point P on AD such that $AP : PD = 2 : 1$
\item Find the coordinates of points Q and R on medians BE and CF respectively such
that $BQ : QE = 2 : 1 \text{and} CR : RF = 2 : 1.$
\item What do yo observe?
\item If $A(x_1, y_1), B(x_2, y_2) \text{and} C(x_3, y_3)$ are the vertices of $\triangle ABC$, find the coordinates of the centroid of the triangle.
\end{enumerate}

Given points are
\begin{align}
\vec{A}=\myvec{4\\ 2} ,
\vec{B}=\myvec{6\\ 5} ,
\vec{C}=\myvec{1\\ 4}
\end{align}
\fi

\begin{enumerate}
\item 
\begin{align}
\vec{D}&=\frac{\vec{B}+\vec{C}}{2}\\
&=\myvec{\frac{7}{2}\\[2pt] \frac{9}{2}}\\
\vec{E}&=\frac{\vec{A}+\vec{C}}{2}\\
&=\myvec{\frac{5}{2}\\ 3}\\
\vec{F}&=\frac{\vec{A}+\vec{B}}{2}\\
&=\myvec{5\\ \frac{7}{2}}
\end{align}

\item 
	For
$n=2$,
\begin{align}
\vec{P}&=\frac{1}{1+n}\brak{\myvec{\vec{A}+n\vec{D}}}\\
&=\frac{1}{3}\myvec{11\\11}
\end{align}

\item 
\begin{align}
\vec{Q}&=\frac{1}{1+n}\brak{\myvec{\vec{B}+n\vec{E}}}\\
&=\frac{1}{3}\myvec{11\\11}\\
\vec{R}&=\frac{1}{1+n}\brak{\myvec{\vec{C}+n\vec{F}}}\\
&=\frac{1}{3}\myvec{11\\11}\\
\end{align}

\item 
 $\vec{P},\vec{Q},\vec{R}$ are the same point.
   
\item 
\begin{align}
\vec{G}&=\frac{\vec{D}+\vec{E}+\vec{F}}{3}\\
&=\frac{1}{3}\myvec{11\\11}\\
\end{align} 
\end{enumerate}
See Fig.  
  \ref{fig:chapters/10/7/4/7/Figure}.
\begin{figure}[h!]
\centering
\includegraphics[width=\columnwidth]{chapters/10/7/4/7/figs/dj.pdf}
\caption{}
  \label{fig:chapters/10/7/4/7/Figure}
\end{figure}

\item Find the slope of a line, which passes through the origin and the mid point of the line segment joining the points $\vec{P}$(0,-4) and $\vec{B}$(8,0).
\label{chapters/11/10/1/5}
\iffalse
\documentclass[journal,12pt,twocolumn]{IEEEtran}
\usepackage{graphicx}
\graphicspath{{./figs/}}{}
\usepackage{amsmath,amssymb,amsfonts,amsthm}
\newcommand{\myvec}[1]{\ensuremath{\begin{pmatrix}#1\end{pmatrix}}}

\let\vec\mathbf

\title{
Matrix-Lines
}
\author{Jyothsna Paluchuri-FWC22059\\}
\begin{document}
\maketitle
\tableofcontents
\bigskip
\section{Problem Statement}
\fi
	\begin{figure}[!ht]
		\centering
 \includegraphics[width=\columnwidth]{chapters/11/10/1/5/figs/line.png}
		\caption{}
		\label{fig:11/10/1/5}
  	\end{figure}
	\\
	\solution
\iffalse
\section{Construction}
\begin{figure}[h]
    \centering
\includegraphics[width=\columnwidth]{line.png}
    \caption{Equation of the slope}
    \label{fig:my_label}
\end{figure}
\vspace{2cm}
\begin{table}[h]
    \centering
    \begin{tabular}{|c|c|c|c|}
       \hline
       \textbf{Symbol}&\textbf{Value}&\textbf{Description}  \\
       \hline
	    $\vec{P}$ & $\myvec{
		    0\\
		    -4}$
	    & Point on Y-axis\\
        \hline
	    $\vec{B}$ & $\myvec{8\\0}$
 & Point on X-axis\\
        \hline
	    $\vec{0}$ & $\myvec{0\\0}$
 & Origin\\
        \hline
    \end{tabular}
    \caption{Parameters}
    \label{tab:my_label}
\end{table}


\section{Solution}
Given that resultant line passes through origin and mid point of the line segment joining point P(0,-4) and B(8,0) \\
\\
\\
given ${\vec{P}}$=$\myvec{
  0\\
  -4}$
 , ${\vec{B}}$=$\myvec{
  8\\
  0}$
  
 \fi 
The mid point of $PB$ is
\begin{align}
\vec{M} &=\frac{1}{2}(\vec{P}+\vec{B})
	= \myvec{4 \\ -2}  
\end{align}
The direction vector of line joining $\vec{O}, \vec{M}$ is 
\begin{align}
\vec{m}&=\vec{O}-\vec{M}
 = -\vec{M}
\end{align}
which can be expressed as
\begin{align}
	\myvec{1 \\ -\frac{1}{2}}
\end{align}
Thus the slope is
\begin{align}
	m = -\frac{1}{2}
\end{align}
\iffalse
\textbf{The direction vector of a line expressed as}
\begin{align}
\implies\vec{m} &= \begin{pmatrix}1 \\ m \\ \end{pmatrix}
\end{align}

\textbf{By solving equation (5) and (6),we get the slope of $\vec{O}$ $\vec{M}$ line}
\begin{align}
        \boxed{m=-0.5}
 \end{align}

\section{Software}
Download the following code using,
\begin{table}[h]
    \centering
    \begin{tabular}{|c|}
    \hline \\
   https://github.com/jyothsna777/jyothsna-fwc.git  \\
         \\
\hline
    \end{tabular}
\end{table}
\\
and execute the code by using command
\begin{center}
\textbf{Python3 lines.py}\\
\end{center}

\section{Conclusion}
Hence the slope of line $\vec{O}$ $\vec{M}$ lineis $\vec{m}$=-0.5

\end{document}
\fi

\item Find the position vector of a point R which divides the line joining two points P and Q whose position vectors are $(2\vec{a}+\vec{b})$ and $(\vec{a}-3\vec{b})$
externally in the ratio 1 : 2. Also, show that P is the mid point of the line segment RQ.\\
	\solution
%		\begin{enumerate}[label=\thesection.\arabic*,ref=\thesection.\theenumi]
\numberwithin{equation}{enumi}
\numberwithin{figure}{enumi}
\numberwithin{table}{enumi}

\item Find the coordinates of the point which divides the join of $(-1,7) \text{ and } (4,-3)$ in the ratio 2:3.
	\\
		\solution
	\iffalse
\documentclass[12pt]{article}
\usepackage{graphicx}
\usepackage{amsmath}
\usepackage{mathtools}
\usepackage{gensymb}

\newcommand{\mydet}[1]{\ensuremath{\begin{vmatrix}#1\end{vmatrix}}}
\providecommand{\brak}[1]{\ensuremath{\left(#1\right)}}
\providecommand{\norm}[1]{\left\lVert#1\right\rVert}
\newcommand{\solution}{\noindent \textbf{Solution: }}
\newcommand{\myvec}[1]{\ensuremath{\begin{pmatrix}#1\end{pmatrix}}}
\let\vec\mathbf

\begin{document}
\begin{center}
\textbf\large{CHAPTER-7 \\ COORDINATE GEOMETRY}
\end{center}
\section*{Excercise 7.2}

1. Find the coordinates of the point which divides the join $\vec(-1,7) \text{ and } \vec(4,-3)$ in the ratio 2:3 :
\\
\\
\solution\\		
\fi
The coordinates and ratio are given as
\begin{align}
\vec{P}=\myvec{-1\\7\\},
\vec{Q}=\myvec{4\\-3\\},
n=\frac{3}{2}
\end{align}
Using section formula
\begin{align}
\vec{R}&=\frac{\vec{Q}+n\vec{P}}{1+n}\\
&=\frac{1}{1+\frac{3}{2}}  \myvec{\myvec{
4\\
-3\\
}
  +
   \frac{3}{2}\myvec{
-1\\
7\\
}}\\
&=\myvec{
1\\
3
}
\end{align}
See Fig. 
\ref{fig:chapters/10/7/2/1/Fig}
\begin{figure}[!h]
\begin{center}
   \includegraphics[width=\columnwidth]{chapters/10/7/2/1/figs/linefig.png}
\end{center}
\caption{}
\label{fig:chapters/10/7/2/1/Fig}
\end{figure}


\item Find the coordinates of the points of trisection of the line segment joining $(4,-1) \text{ and } (-2,3)$.
	\\
		\solution
	\begin{enumerate}[label=\thesection.\arabic*,ref=\thesection.\theenumi]
\numberwithin{equation}{enumi}
\numberwithin{figure}{enumi}
\numberwithin{table}{enumi}

\item Find the coordinates of the point which divides the join of $(-1,7) \text{ and } (4,-3)$ in the ratio 2:3.
	\\
		\solution
	\input{chapters/10/7/2/1/section.tex}
\item Find the coordinates of the points of trisection of the line segment joining $(4,-1) \text{ and } (-2,3)$.
	\\
		\solution
	\input{chapters/10/7/2/2/section.tex}
\item
	\iffalse
\item To conduct Sports Day activities, in your rectangular shaped school                   
ground ABCD, lines have 
drawn with chalk powder at a                 
distance of 1m each. 100 flower pots have been placed at a distance of 1m 
from each other along AD, as shown 
in Fig. 7.12. Niharika runs $ \frac {1}{4} $th the 
distance AD on the 2nd line and 
posts a green flag. Preet runs $ \frac {1}{5} $th 
the distance AD on the eighth line 
and posts a red flag. What is the 
distance between both the flags? If 
Rashmi has to post a blue flag exactly 
halfway between the line segment 
joining the two flags, where should 
she post her flag?
\begin{figure}[h!]
  \centering
  \includegraphics[width=\columnwidth]{sc.png}
  \caption{}
\label{fig:10/7/12Fig1}
\end{figure}               
\fi
      
\item Find the ratio in which the line segment joining the points $(-3,10) \text{ and } (6,-8)$ $\text{ is divided by } (-1,6)$.
	\\
		\solution
	\input{chapters/10/7/2/4/section.tex}
\item Find the ratio in which the line segment joining $A(1,-5) \text{ and } B(-4,5)$ $\text{is divided by the x-axis}$. Also find the coordinates of the point of division.
\item If $(1,2), (4,y), (x,6), (3,5)$ are the vertices of a parallelogram taken in order, find x and y.
	\\
		\solution
	\input{chapters/10/7/2/6/para1.tex}
\item Find the coordinates of a point A, where AB is the diameter of a circle whose centre is $(2,-3) \text{ and }$ B is $(1,4)$.
	\\
		\solution
	\input{chapters/10/7/2/7/section.tex}
\item If A \text{ and } B are $(-2,-2) \text{ and } (2,-4)$, respectively, find the coordinates of P such that AP= $\frac {3}{7}$AB $\text{ and }$ P lies on the line segment AB.
	\\
		\solution
	\input{chapters/10/7/2/8/section.tex}
\item Find the coordinates of the points which divide the line segment joining $A(-2,2) \text{ and } B(2,8)$ into four equal parts.
	\\
		\solution
	\input{chapters/10/7/2/9/section.tex}
\item Find the area of a rhombus if its vertices are $(3,0), (4,5), (-1,4) \text{ and } (-2,-1)$ taken in order. [$\vec{Hint}$ : Area of rhombus =$\frac {1}{2}$(product of its diagonals)]
	\\
		\solution
	\input{chapters/10/7/2/10/cross.tex}
\item Find the position vector of a point R which divides the line joining two points $\vec{P}$
and $\vec{Q}$ whose position vectors are $\hat{i}+2\hat{j}-\hat{k}$ and $-\hat{i}+\hat{j}+\hat{k}$ respectively, in the
ratio 2 : 1
\begin{enumerate}
    \item  internally
    \item  externally
\end{enumerate}
\solution
		\input{chapters/12/10/2/15/section.tex}
\item Find the position vector of the mid point of the vector joining the points $\vec{P}$(2, 3, 4)
and $\vec{Q}$(4, 1, –2).
\\
\solution
		\input{chapters/12/10/2/16/section.tex}
\item Determine the ratio in which the line $2x+y  - 4=0$ divides the line segment joining the points $\vec{A}(2, - 2)$  and  $\vec{B}(3, 7)$.
\\
\solution
	\input{chapters/10/7/4/1/section.tex}
\item Let $\vec{A}(4, 2), \vec{B}(6, 5)$  and $ \vec{C}(1, 4)$ be the vertices of $\triangle ABC$.
\begin{enumerate}
\item The median from $\vec{A}$ meets $BC$ at $\vec{D}$. Find the coordinates of the point $\vec{D}$.
\item Find the coordinates of the point $\vec{P}$ on $AD$ such that $AP : PD = 2 : 1$.
\item Find the coordinates of points $\vec{Q}$ and $\vec{R}$ on medians $BE$ and $CF$ respectively such that $BQ : QE = 2 : 1$  and  $CR : RF = 2 : 1$.
\item What do you observe?
\item If $\vec{A}, \vec{B}$ and $\vec{C}$  are the vertices of $\triangle ABC$, find the coordinates of the centroid of the triangle.
\end{enumerate}
\solution
	\input{chapters/10/7/4/7/section.tex}
\item Find the slope of a line, which passes through the origin and the mid point of the line segment joining the points $\vec{P}$(0,-4) and $\vec{B}$(8,0).
\label{chapters/11/10/1/5}
\input{chapters/11/10/1/5/matrix.tex}
\item Find the position vector of a point R which divides the line joining two points P and Q whose position vectors are $(2\vec{a}+\vec{b})$ and $(\vec{a}-3\vec{b})$
externally in the ratio 1 : 2. Also, show that P is the mid point of the line segment RQ.\\
	\solution
%		\input{chapters/12/10/5/9/section.tex}

\end{enumerate}


\item
	\iffalse
\item To conduct Sports Day activities, in your rectangular shaped school                   
ground ABCD, lines have 
drawn with chalk powder at a                 
distance of 1m each. 100 flower pots have been placed at a distance of 1m 
from each other along AD, as shown 
in Fig. 7.12. Niharika runs $ \frac {1}{4} $th the 
distance AD on the 2nd line and 
posts a green flag. Preet runs $ \frac {1}{5} $th 
the distance AD on the eighth line 
and posts a red flag. What is the 
distance between both the flags? If 
Rashmi has to post a blue flag exactly 
halfway between the line segment 
joining the two flags, where should 
she post her flag?
\begin{figure}[h!]
  \centering
  \includegraphics[width=\columnwidth]{sc.png}
  \caption{}
\label{fig:10/7/12Fig1}
\end{figure}               
\fi
      
\item Find the ratio in which the line segment joining the points $(-3,10) \text{ and } (6,-8)$ $\text{ is divided by } (-1,6)$.
	\\
		\solution
	\iffalse
\documentclass[12pt]{article}
\usepackage{graphicx}
%\documentclass[journal,12pt,twocolumn]{IEEEtran}
\usepackage[none]{hyphenat}
\usepackage{graphicx}
\usepackage{listings}
\usepackage[english]{babel}
\usepackage{graphicx}
\usepackage{caption} 
\usepackage{hyperref}
\usepackage{booktabs}
\def\inputGnumericTable{}
\usepackage{color}                                            %%
    \usepackage{array}                                            %%
    \usepackage{longtable}                                        %%
    \usepackage{calc}                                             %%
    \usepackage{multirow}                                         %%
    \usepackage{hhline}                                           %%
    \usepackage{ifthen}
\usepackage{array}
\usepackage{amsmath}   % for having text in math mode
\usepackage{listings}
\lstset{
language=tex,
frame=single, 
breaklines=true
}
  
%Following 2 lines were added to remove the blank page at the beginning
\usepackage{atbegshi}% http://ctan.org/pkg/atbegshi
\AtBeginDocument{\AtBeginShipoutNext{\AtBeginShipoutDiscard}}
%
%New macro definitions
\newcommand{\mydet}[1]{\ensuremath{\begin{vmatrix}#1\end{vmatrix}}}
\providecommand{\brak}[1]{\ensuremath{\left(#1\right)}}
\providecommand{\norm}[1]{\left\lVert#1\right\rVert}
\newcommand{\solution}{\noindent \textbf{Solution: }}
\newcommand{\myvec}[1]{\ensuremath{\begin{pmatrix}#1\end{pmatrix}}}
\let\vec\mathbf
\begin{document}
\begin{center}
\title{\textbf{Coordinate Geometry}}
\date{\vspace{-5ex}} %Not to print date automatically
\maketitle
\end{center}
\setcounter{page}{1}
\section*{10$^{th}$ Maths - Chapter 7}
This is Problem-4 from Exercise 7.2
\begin{enumerate}
\item Find the ratio in which the line segement joining the points $\myvec{-3 \\ 10}$ and $\myvec{6\\-8}$ is divided by $\myvec{-1\\6}$.\\
\solution \\
\fi
		The input parameters for this problem are available in Table \eqref{tab:10/7/2/4-1}.
\begin{table}[ht!]
\input{chapters/10/7/2/4/tables/table.tex}
\caption{}
\label{tab:10/7/2/4-1} 
\end{table}
Using section formula,
\begin{align}
         \vec{R} &=\frac{\vec{Q}+n\vec{P}}{1+n}\label{eq:chapters/10/7/2/4/1}
\end{align}
Substituting the values of $\vec{P},\vec{Q}$ and $\vec{R}$ in \eqref{eq:chapters/10/7/2/4/1}
\begin{align}
         \myvec{-1\\6} &=\frac{{\myvec{-3\\10}+n\myvec{6\\-8}}}{1+n}\\
 &=\frac{1}{1+n}\brak{{\myvec{-3\\10}+n\myvec{6\\-8}}} \\
 &=\frac{1}{1+n}\myvec{-3+6n\\10-8n} \label{eq:chapters/10/7/2/4/4}
\end{align}
Simplifying \eqref{eq:chapters/10/7/2/4/4} yeilds,
\begin{align}
          -1 &=\frac{-3+6n}{1+n}\\
\implies          n &=\frac{2}{7}
\end{align}
Also,
\begin{align}
          6 &=\frac{10-8n}{1+n}\\
    \implies      n &=\frac{2}{7}
\end{align}
Hence the desired ratio is $\dfrac{2}{7}$.  
\begin{figure}[!h]
 \begin{center}
  \includegraphics[width=\columnwidth]{chapters/10/7/2/4/figs/fig.png}
 \end{center}
\caption{}
\label{fig:10/7/2/4Fig1}
\end{figure}

\item Find the ratio in which the line segment joining $A(1,-5) \text{ and } B(-4,5)$ $\text{is divided by the x-axis}$. Also find the coordinates of the point of division.
\item If $(1,2), (4,y), (x,6), (3,5)$ are the vertices of a parallelogram taken in order, find x and y.
	\\
		\solution
	\iffalse
\documentclass[12pt]{article}
\usepackage{graphicx}
%\documentclass[journal,12pt,twocolumn]{IEEEtran}
\def\inputGnumericTable{}
\usepackage{color}                                            %%
    \usepackage{array}                                            %%
    \usepackage{longtable}                                        %%
    \usepackage{calc}                                             %%
    \usepackage{multirow}                                         %%
    \usepackage{hhline}                                           %%
    \usepackage{ifthen}
\usepackage[none]{hyphenat}
\usepackage{graphicx}
\usepackage{listings}
\usepackage[english]{babel}
\usepackage{graphicx}
\usepackage{caption} 
\usepackage{hyperref}
\usepackage{booktabs}
\usepackage{array}
\usepackage{amsmath}   % for having text in math mode
\usepackage{listings}
\lstset{
  frame=single,
  breaklines=true
}
  
%Following 2 lines were added to remove the blank page at the beginning
\usepackage{atbegshi}% http://ctan.org/pkg/atbegshi
\AtBeginDocument{\AtBeginShipoutNext{\AtBeginShipoutDiscard}}
%


%New macro definitions
\newcommand{\mydet}[1]{\ensuremath{\begin{vmatrix}#1\end{vmatrix}}}
\providecommand{\brak}[1]{\ensuremath{\left(#1\right)}}
\providecommand{\norm}[1]{\left\lVert#1\right\rVert}
\newcommand{\solution}{\noindent \textbf{Solution: }}
\newcommand{\myvec}[1]{\ensuremath{\begin{pmatrix}#1\end{pmatrix}}}
\let\vec\mathbf

\begin{document}

\begin{center}
\title{\textbf{Properties of Parallelegram}}
\date{\vspace{-5ex}} %Not to print date automatically
\maketitle
\end{center}

\setcounter{page}{1}

\section{10$^{th}$ Maths - Chapter 7}

This is Problem-6 from Exercise 7.2

\begin{enumerate}
\item If $\vec{A}(1, 2),\vec{B}(4, x),\vec{C}(y, 6) \text{and } \vec{D}(3, 5)$ are the vertices of a parallelogram taken in order,find x and y.
\end{enumerate}
\fi

The input parameters for this problem are available in
\ref{table:chapters/10/7/2/6/tables/}.	
\begin{table}[!ht]
	\centering
	\input{chapters/10/7/2/6/tables/table.tex}
\caption{}
\label{table:chapters/10/7/2/6/tables/}	
\end{table}
From the given information,
\begin{align}
  \label{eq:chapters/10/7/2/6/tables/det2f}
	\vec{B}-\vec{A} &= \myvec{4 \\y } - \myvec{1 \\2 }  = \myvec{3 \\y-2 }\\
	\vec{C}-\vec{D} &= \myvec{x \\6 } - \myvec{3 \\5 }  = \myvec{x-3 \\1}
\end{align}
Since $ABCD$ is a parallellogram,
\begin{align}
	\myvec{3\\y-2}&=\myvec{x-3\\1}\\
	\implies x&=6 ,y=3
\end{align}
Fig. \ref{fig:chapters/10/7/2/6/Fig3}
provides a verification.
\begin{figure}[h!]
	\begin{center}
  \includegraphics[width=\columnwidth]{chapters/10/7/2/6/figs/para.pdf}
	\end{center}
\caption{}
\label{fig:chapters/10/7/2/6/Fig3}
\end{figure}


\item Find the coordinates of a point A, where AB is the diameter of a circle whose centre is $(2,-3) \text{ and }$ B is $(1,4)$.
	\\
		\solution
	\iffalse
\documentclass[12pt]{article}
\usepackage{graphicx}
\usepackage{amsmath}
\usepackage{mathtools}
\usepackage{gensymb}

\newcommand{\mydet}[1]{\ensuremath{\begin{vmatrix}#1\end{vmatrix}}}
\providecommand{\brak}[1]{\ensuremath{\left(#1\right)}}
\providecommand{\norm}[1]{\left\lVert#1\right\rVert}
\newcommand{\solution}{\noindent \textbf{Solution: }}
\newcommand{\myvec}[1]{\ensuremath{\begin{pmatrix}#1\end{pmatrix}}}
\let\vec\mathbf

\begin{document}
\begin{center}
\section*{CHAPTER 7 - COORDINATE GEOMETRY}

\end{center}
\section*{Excercise 7.2}

Q7.Find the coordinates of point $\vec{A}$, where AB is the diameter of a circle where the center is (2,-3) and $\vec{B}$ is the point (1,4):

\solution
\begin{enumerate}
\item The coordinates $\vec{B}$ and center $\vec{C}$ are given, where:
	\fi
	Let
	\begin{align}
	\vec{B} = \myvec{
		1\\
	    4\\
		},
	\vec{C} = \myvec{
	    2\\
	   -3\\
		}
	\end{align}
	\iffalse
Let us assume the coordinates of $\vec{A}$. Now, $\vec{C}$ is the center which is midpoint of line AB and $\vec{B}$ is one of the coordinate of diameter AB of a circle.
	\fi	
Hence,	
	\begin{align}
	\vec{C} &= \frac{\vec{A+B}}{2} \\
\implies	2\vec{C} &= \vec{A}+\vec{B} \\
		\text{or, }	\vec{A} &= 2\vec{C}-\vec{B} \\
	 &= \myvec{3\\-10\\}	
	\end{align}       
	See Fig. 
\ref{fig:chapters/10/7/2/7Fig}.
\begin{figure}[!h]
\begin{center}	
	\includegraphics[width=\columnwidth]{chapters/10/7/2/7/figs/Vector1.png}
\end{center}
\caption{}
\label{fig:chapters/10/7/2/7Fig}
\end{figure}
	

\item If A \text{ and } B are $(-2,-2) \text{ and } (2,-4)$, respectively, find the coordinates of P such that AP= $\frac {3}{7}$AB $\text{ and }$ P lies on the line segment AB.
	\\
		\solution
	\iffalse
\documentclass[journal,10pt,twocolumn]{article}
\usepackage{graphicx}
\usepackage[none]{hyphenat}
\usepackage{graphicx}
\usepackage{listings}
\usepackage[english]{babel}
\usepackage{graphicx}
\usepackage{caption} 
\usepackage{booktabs}
\usepackage{array}
\usepackage{amssymb} % for \because
\usepackage{amsmath}   % for having text in math mode
\usepackage{extarrows} % for Row operations arrows
\usepackage{listings}
\usepackage[utf8]{inputenc}
\lstset{
  frame=single,
  breaklines=true
}
\usepackage{hyperref}
  
%Following 2 lines were added to remove the blank page at the beginning
\usepackage{atbegshi}% http://ctan.org/pkg/atbegshi
\AtBeginDocument{\AtBeginShipoutNext{\AtBeginShipoutDiscard}}


%New macro definitions
\newcommand{\mydet}[1]{\ensuremath{\begin{vmatrix}#1\end{vmatrix}}}
\providecommand{\brak}[1]{\ensuremath{\left(#1\right)}}
\newcommand{\solution}{\noindent \textbf{Solution: }}
\newcommand{\myvec}[1]{\ensuremath{\begin{pmatrix}#1\end{pmatrix}}}
\providecommand{\norm}[1]{\left\lVert#1\right\rVert}
\providecommand{\abs}[1]{\left\vert#1\right\vert}
\let\vec\mathbf

\begin{document}

\begin{center}
\title{\textbf{VECTORS}}
\date{\vspace{-5ex}} %Not to print date automatically
\maketitle
\end{center}

\section{10$^{th}$ Maths - EXERCISE-7.2}

\begin{enumerate}
\item If A and B are $(– 2, – 2)\text{ and }(2, – 4)$, respectively, find the coordinates of P such that $AP =\frac{3}{7}AB$ and P lies on the line segment AB. 

\section{SOLUTION}
Given points are
\begin{align}
\vec{A}=\myvec{-2\\ -2} ,
\vec{B}=\myvec{2\\ -4}
\end{align}
The equation of the formula is
\fi
Using section formula, 
\begin{align}
\vec{P}&=\frac{\vec{A}+n\vec{B}}{1+n}
\end{align}
where
\begin{align}
	n =\frac{3}{4}
\end{align}
Thus,
\begin{align}
\vec{P}&=\frac{1}{1+\frac{3}{4}}\brak{\myvec{-2\\-2}+\frac{3}{4}\myvec{2\\-4}}\\
&=\myvec{\frac{-2}{7}\\[1pt] \frac{-20}{7}}
\end{align}
See Fig. 
   \ref{fig:chapters/10/7/2/8/vec.png}
\begin{figure}
   \centering 
 \includegraphics[width=\columnwidth]{chapters/10/7/2/8/figs/vec.png}
   \caption{}
   \label{fig:chapters/10/7/2/8/vec.png}
   \end{figure}

\item Find the coordinates of the points which divide the line segment joining $A(-2,2) \text{ and } B(2,8)$ into four equal parts.
	\\
		\solution
	\begin{enumerate}[label=\thesection.\arabic*,ref=\thesection.\theenumi]
\numberwithin{equation}{enumi}
\numberwithin{figure}{enumi}
\numberwithin{table}{enumi}

\item Find the coordinates of the point which divides the join of $(-1,7) \text{ and } (4,-3)$ in the ratio 2:3.
	\\
		\solution
	\input{chapters/10/7/2/1/section.tex}
\item Find the coordinates of the points of trisection of the line segment joining $(4,-1) \text{ and } (-2,3)$.
	\\
		\solution
	\input{chapters/10/7/2/2/section.tex}
\item
	\iffalse
\item To conduct Sports Day activities, in your rectangular shaped school                   
ground ABCD, lines have 
drawn with chalk powder at a                 
distance of 1m each. 100 flower pots have been placed at a distance of 1m 
from each other along AD, as shown 
in Fig. 7.12. Niharika runs $ \frac {1}{4} $th the 
distance AD on the 2nd line and 
posts a green flag. Preet runs $ \frac {1}{5} $th 
the distance AD on the eighth line 
and posts a red flag. What is the 
distance between both the flags? If 
Rashmi has to post a blue flag exactly 
halfway between the line segment 
joining the two flags, where should 
she post her flag?
\begin{figure}[h!]
  \centering
  \includegraphics[width=\columnwidth]{sc.png}
  \caption{}
\label{fig:10/7/12Fig1}
\end{figure}               
\fi
      
\item Find the ratio in which the line segment joining the points $(-3,10) \text{ and } (6,-8)$ $\text{ is divided by } (-1,6)$.
	\\
		\solution
	\input{chapters/10/7/2/4/section.tex}
\item Find the ratio in which the line segment joining $A(1,-5) \text{ and } B(-4,5)$ $\text{is divided by the x-axis}$. Also find the coordinates of the point of division.
\item If $(1,2), (4,y), (x,6), (3,5)$ are the vertices of a parallelogram taken in order, find x and y.
	\\
		\solution
	\input{chapters/10/7/2/6/para1.tex}
\item Find the coordinates of a point A, where AB is the diameter of a circle whose centre is $(2,-3) \text{ and }$ B is $(1,4)$.
	\\
		\solution
	\input{chapters/10/7/2/7/section.tex}
\item If A \text{ and } B are $(-2,-2) \text{ and } (2,-4)$, respectively, find the coordinates of P such that AP= $\frac {3}{7}$AB $\text{ and }$ P lies on the line segment AB.
	\\
		\solution
	\input{chapters/10/7/2/8/section.tex}
\item Find the coordinates of the points which divide the line segment joining $A(-2,2) \text{ and } B(2,8)$ into four equal parts.
	\\
		\solution
	\input{chapters/10/7/2/9/section.tex}
\item Find the area of a rhombus if its vertices are $(3,0), (4,5), (-1,4) \text{ and } (-2,-1)$ taken in order. [$\vec{Hint}$ : Area of rhombus =$\frac {1}{2}$(product of its diagonals)]
	\\
		\solution
	\input{chapters/10/7/2/10/cross.tex}
\item Find the position vector of a point R which divides the line joining two points $\vec{P}$
and $\vec{Q}$ whose position vectors are $\hat{i}+2\hat{j}-\hat{k}$ and $-\hat{i}+\hat{j}+\hat{k}$ respectively, in the
ratio 2 : 1
\begin{enumerate}
    \item  internally
    \item  externally
\end{enumerate}
\solution
		\input{chapters/12/10/2/15/section.tex}
\item Find the position vector of the mid point of the vector joining the points $\vec{P}$(2, 3, 4)
and $\vec{Q}$(4, 1, –2).
\\
\solution
		\input{chapters/12/10/2/16/section.tex}
\item Determine the ratio in which the line $2x+y  - 4=0$ divides the line segment joining the points $\vec{A}(2, - 2)$  and  $\vec{B}(3, 7)$.
\\
\solution
	\input{chapters/10/7/4/1/section.tex}
\item Let $\vec{A}(4, 2), \vec{B}(6, 5)$  and $ \vec{C}(1, 4)$ be the vertices of $\triangle ABC$.
\begin{enumerate}
\item The median from $\vec{A}$ meets $BC$ at $\vec{D}$. Find the coordinates of the point $\vec{D}$.
\item Find the coordinates of the point $\vec{P}$ on $AD$ such that $AP : PD = 2 : 1$.
\item Find the coordinates of points $\vec{Q}$ and $\vec{R}$ on medians $BE$ and $CF$ respectively such that $BQ : QE = 2 : 1$  and  $CR : RF = 2 : 1$.
\item What do you observe?
\item If $\vec{A}, \vec{B}$ and $\vec{C}$  are the vertices of $\triangle ABC$, find the coordinates of the centroid of the triangle.
\end{enumerate}
\solution
	\input{chapters/10/7/4/7/section.tex}
\item Find the slope of a line, which passes through the origin and the mid point of the line segment joining the points $\vec{P}$(0,-4) and $\vec{B}$(8,0).
\label{chapters/11/10/1/5}
\input{chapters/11/10/1/5/matrix.tex}
\item Find the position vector of a point R which divides the line joining two points P and Q whose position vectors are $(2\vec{a}+\vec{b})$ and $(\vec{a}-3\vec{b})$
externally in the ratio 1 : 2. Also, show that P is the mid point of the line segment RQ.\\
	\solution
%		\input{chapters/12/10/5/9/section.tex}

\end{enumerate}


\item Find the area of a rhombus if its vertices are $(3,0), (4,5), (-1,4) \text{ and } (-2,-1)$ taken in order. [$\vec{Hint}$ : Area of rhombus =$\frac {1}{2}$(product of its diagonals)]
	\\
		\solution
	\iffalse
\documentclass[12pt]{article}
\usepackage{graphicx}
%\documentclass[journal,12pt,twocolumn]{IEEEtran}
\usepackage[none]{hyphenat}
\usepackage{graphicx}
\usepackage{listings}
\usepackage[english]{babel}
\usepackage{graphicx}
\usepackage{caption} 
\usepackage{hyperref}
\usepackage{booktabs}
\def\inputGnumericTable{}
\usepackage{color}                                            %%
    \usepackage{array}                                            %%
    \usepackage{longtable}                                        %%
    \usepackage{calc}                                             %%
    \usepackage{multirow}                                         %%
    \usepackage{hhline}                                           %%
    \usepackage{ifthen}
\usepackage{array}
\usepackage{amsmath}   % for having text in math mode
\usepackage{listings}
\lstset{
language=tex,
frame=single, 
breaklines=true
}
  
%Following 2 lines were added to remove the blank page at the beginning
\usepackage{atbegshi}% http://ctan.org/pkg/atbegshi
\AtBeginDocument{\AtBeginShipoutNext{\AtBeginShipoutDiscard}}
%


%New macro definitions
\newcommand{\mydet}[1]{\ensuremath{\begin{vmatrix}#1\end{vmatrix}}}
\providecommand{\brak}[1]{\ensuremath{\left(#1\right)}}
\providecommand{\norm}[1]{\left\lVert#1\right\rVert}
\newcommand{\solution}{\noindent \textbf{Solution: }}
\newcommand{\myvec}[1]{\ensuremath{\begin{pmatrix}#1\end{pmatrix}}}
\let\vec\mathbf

\begin{document}

\begin{center}
\title{\textbf{Coordinate Geometry}}
\date{\vspace{-5ex}} %Not to print date automatically
\maketitle
\end{center}

\setcounter{page}{1}



\begin{enumerate}

\item\textbf{Problem statement :} Find the area of a rhombus of its vertices are $\myvec{3 ,0}$, $\myvec{4 ,5}$, $\myvec{-1 ,4}$ and $\myvec{-2 ,-1}$taken in order

\solution \\
\fi
The input vertices for this problem are given as
	\begin{align}
	\vec{A} = \myvec{
		3\\
		0
		},
	\vec{B} = \myvec{
		4\\
		5
		},
        \vec{C} = \myvec{
		-1\\
		4
		},
        \vec{D} = \myvec{
		-2\\
		-1
		}
	\end{align}
Since		
\begin{align}
 \vec{A-D}= \myvec{3 \\ 0} - \myvec{-2 \\-1}= \myvec{5\\1}
 \\
  \vec{B-A}= \myvec{4 \\ 5} - \myvec{3 \\0}= \myvec{1\\5}
\end{align}
the area of the rhombus is
\begin{align}
                \norm{\myvec{\vec{A-D}}\times \myvec{\vec{B-A}}}=\mydet{5 & 1\\1 & 5} = 24
\end{align}
See Fig. 
\ref{fig:chapters/10/7/2/10/gFig1}.
\begin{figure}[!h]
 \begin{center}
  \includegraphics[width=\columnwidth]{chapters/10/7/2/10/figs/fig.pdf}
 \end{center}
\caption{}
\label{fig:chapters/10/7/2/10/gFig1}
\end{figure}

\item Find the position vector of a point R which divides the line joining two points $\vec{P}$
and $\vec{Q}$ whose position vectors are $\hat{i}+2\hat{j}-\hat{k}$ and $-\hat{i}+\hat{j}+\hat{k}$ respectively, in the
ratio 2 : 1
\begin{enumerate}
    \item  internally
    \item  externally
\end{enumerate}
\solution
		\begin{enumerate}[label=\thesection.\arabic*,ref=\thesection.\theenumi]
\numberwithin{equation}{enumi}
\numberwithin{figure}{enumi}
\numberwithin{table}{enumi}

\item Find the coordinates of the point which divides the join of $(-1,7) \text{ and } (4,-3)$ in the ratio 2:3.
	\\
		\solution
	\input{chapters/10/7/2/1/section.tex}
\item Find the coordinates of the points of trisection of the line segment joining $(4,-1) \text{ and } (-2,3)$.
	\\
		\solution
	\input{chapters/10/7/2/2/section.tex}
\item
	\iffalse
\item To conduct Sports Day activities, in your rectangular shaped school                   
ground ABCD, lines have 
drawn with chalk powder at a                 
distance of 1m each. 100 flower pots have been placed at a distance of 1m 
from each other along AD, as shown 
in Fig. 7.12. Niharika runs $ \frac {1}{4} $th the 
distance AD on the 2nd line and 
posts a green flag. Preet runs $ \frac {1}{5} $th 
the distance AD on the eighth line 
and posts a red flag. What is the 
distance between both the flags? If 
Rashmi has to post a blue flag exactly 
halfway between the line segment 
joining the two flags, where should 
she post her flag?
\begin{figure}[h!]
  \centering
  \includegraphics[width=\columnwidth]{sc.png}
  \caption{}
\label{fig:10/7/12Fig1}
\end{figure}               
\fi
      
\item Find the ratio in which the line segment joining the points $(-3,10) \text{ and } (6,-8)$ $\text{ is divided by } (-1,6)$.
	\\
		\solution
	\input{chapters/10/7/2/4/section.tex}
\item Find the ratio in which the line segment joining $A(1,-5) \text{ and } B(-4,5)$ $\text{is divided by the x-axis}$. Also find the coordinates of the point of division.
\item If $(1,2), (4,y), (x,6), (3,5)$ are the vertices of a parallelogram taken in order, find x and y.
	\\
		\solution
	\input{chapters/10/7/2/6/para1.tex}
\item Find the coordinates of a point A, where AB is the diameter of a circle whose centre is $(2,-3) \text{ and }$ B is $(1,4)$.
	\\
		\solution
	\input{chapters/10/7/2/7/section.tex}
\item If A \text{ and } B are $(-2,-2) \text{ and } (2,-4)$, respectively, find the coordinates of P such that AP= $\frac {3}{7}$AB $\text{ and }$ P lies on the line segment AB.
	\\
		\solution
	\input{chapters/10/7/2/8/section.tex}
\item Find the coordinates of the points which divide the line segment joining $A(-2,2) \text{ and } B(2,8)$ into four equal parts.
	\\
		\solution
	\input{chapters/10/7/2/9/section.tex}
\item Find the area of a rhombus if its vertices are $(3,0), (4,5), (-1,4) \text{ and } (-2,-1)$ taken in order. [$\vec{Hint}$ : Area of rhombus =$\frac {1}{2}$(product of its diagonals)]
	\\
		\solution
	\input{chapters/10/7/2/10/cross.tex}
\item Find the position vector of a point R which divides the line joining two points $\vec{P}$
and $\vec{Q}$ whose position vectors are $\hat{i}+2\hat{j}-\hat{k}$ and $-\hat{i}+\hat{j}+\hat{k}$ respectively, in the
ratio 2 : 1
\begin{enumerate}
    \item  internally
    \item  externally
\end{enumerate}
\solution
		\input{chapters/12/10/2/15/section.tex}
\item Find the position vector of the mid point of the vector joining the points $\vec{P}$(2, 3, 4)
and $\vec{Q}$(4, 1, –2).
\\
\solution
		\input{chapters/12/10/2/16/section.tex}
\item Determine the ratio in which the line $2x+y  - 4=0$ divides the line segment joining the points $\vec{A}(2, - 2)$  and  $\vec{B}(3, 7)$.
\\
\solution
	\input{chapters/10/7/4/1/section.tex}
\item Let $\vec{A}(4, 2), \vec{B}(6, 5)$  and $ \vec{C}(1, 4)$ be the vertices of $\triangle ABC$.
\begin{enumerate}
\item The median from $\vec{A}$ meets $BC$ at $\vec{D}$. Find the coordinates of the point $\vec{D}$.
\item Find the coordinates of the point $\vec{P}$ on $AD$ such that $AP : PD = 2 : 1$.
\item Find the coordinates of points $\vec{Q}$ and $\vec{R}$ on medians $BE$ and $CF$ respectively such that $BQ : QE = 2 : 1$  and  $CR : RF = 2 : 1$.
\item What do you observe?
\item If $\vec{A}, \vec{B}$ and $\vec{C}$  are the vertices of $\triangle ABC$, find the coordinates of the centroid of the triangle.
\end{enumerate}
\solution
	\input{chapters/10/7/4/7/section.tex}
\item Find the slope of a line, which passes through the origin and the mid point of the line segment joining the points $\vec{P}$(0,-4) and $\vec{B}$(8,0).
\label{chapters/11/10/1/5}
\input{chapters/11/10/1/5/matrix.tex}
\item Find the position vector of a point R which divides the line joining two points P and Q whose position vectors are $(2\vec{a}+\vec{b})$ and $(\vec{a}-3\vec{b})$
externally in the ratio 1 : 2. Also, show that P is the mid point of the line segment RQ.\\
	\solution
%		\input{chapters/12/10/5/9/section.tex}

\end{enumerate}


\item Find the position vector of the mid point of the vector joining the points $\vec{P}$(2, 3, 4)
and $\vec{Q}$(4, 1, –2).
\\
\solution
		\begin{enumerate}[label=\thesection.\arabic*,ref=\thesection.\theenumi]
\numberwithin{equation}{enumi}
\numberwithin{figure}{enumi}
\numberwithin{table}{enumi}

\item Find the coordinates of the point which divides the join of $(-1,7) \text{ and } (4,-3)$ in the ratio 2:3.
	\\
		\solution
	\input{chapters/10/7/2/1/section.tex}
\item Find the coordinates of the points of trisection of the line segment joining $(4,-1) \text{ and } (-2,3)$.
	\\
		\solution
	\input{chapters/10/7/2/2/section.tex}
\item
	\iffalse
\item To conduct Sports Day activities, in your rectangular shaped school                   
ground ABCD, lines have 
drawn with chalk powder at a                 
distance of 1m each. 100 flower pots have been placed at a distance of 1m 
from each other along AD, as shown 
in Fig. 7.12. Niharika runs $ \frac {1}{4} $th the 
distance AD on the 2nd line and 
posts a green flag. Preet runs $ \frac {1}{5} $th 
the distance AD on the eighth line 
and posts a red flag. What is the 
distance between both the flags? If 
Rashmi has to post a blue flag exactly 
halfway between the line segment 
joining the two flags, where should 
she post her flag?
\begin{figure}[h!]
  \centering
  \includegraphics[width=\columnwidth]{sc.png}
  \caption{}
\label{fig:10/7/12Fig1}
\end{figure}               
\fi
      
\item Find the ratio in which the line segment joining the points $(-3,10) \text{ and } (6,-8)$ $\text{ is divided by } (-1,6)$.
	\\
		\solution
	\input{chapters/10/7/2/4/section.tex}
\item Find the ratio in which the line segment joining $A(1,-5) \text{ and } B(-4,5)$ $\text{is divided by the x-axis}$. Also find the coordinates of the point of division.
\item If $(1,2), (4,y), (x,6), (3,5)$ are the vertices of a parallelogram taken in order, find x and y.
	\\
		\solution
	\input{chapters/10/7/2/6/para1.tex}
\item Find the coordinates of a point A, where AB is the diameter of a circle whose centre is $(2,-3) \text{ and }$ B is $(1,4)$.
	\\
		\solution
	\input{chapters/10/7/2/7/section.tex}
\item If A \text{ and } B are $(-2,-2) \text{ and } (2,-4)$, respectively, find the coordinates of P such that AP= $\frac {3}{7}$AB $\text{ and }$ P lies on the line segment AB.
	\\
		\solution
	\input{chapters/10/7/2/8/section.tex}
\item Find the coordinates of the points which divide the line segment joining $A(-2,2) \text{ and } B(2,8)$ into four equal parts.
	\\
		\solution
	\input{chapters/10/7/2/9/section.tex}
\item Find the area of a rhombus if its vertices are $(3,0), (4,5), (-1,4) \text{ and } (-2,-1)$ taken in order. [$\vec{Hint}$ : Area of rhombus =$\frac {1}{2}$(product of its diagonals)]
	\\
		\solution
	\input{chapters/10/7/2/10/cross.tex}
\item Find the position vector of a point R which divides the line joining two points $\vec{P}$
and $\vec{Q}$ whose position vectors are $\hat{i}+2\hat{j}-\hat{k}$ and $-\hat{i}+\hat{j}+\hat{k}$ respectively, in the
ratio 2 : 1
\begin{enumerate}
    \item  internally
    \item  externally
\end{enumerate}
\solution
		\input{chapters/12/10/2/15/section.tex}
\item Find the position vector of the mid point of the vector joining the points $\vec{P}$(2, 3, 4)
and $\vec{Q}$(4, 1, –2).
\\
\solution
		\input{chapters/12/10/2/16/section.tex}
\item Determine the ratio in which the line $2x+y  - 4=0$ divides the line segment joining the points $\vec{A}(2, - 2)$  and  $\vec{B}(3, 7)$.
\\
\solution
	\input{chapters/10/7/4/1/section.tex}
\item Let $\vec{A}(4, 2), \vec{B}(6, 5)$  and $ \vec{C}(1, 4)$ be the vertices of $\triangle ABC$.
\begin{enumerate}
\item The median from $\vec{A}$ meets $BC$ at $\vec{D}$. Find the coordinates of the point $\vec{D}$.
\item Find the coordinates of the point $\vec{P}$ on $AD$ such that $AP : PD = 2 : 1$.
\item Find the coordinates of points $\vec{Q}$ and $\vec{R}$ on medians $BE$ and $CF$ respectively such that $BQ : QE = 2 : 1$  and  $CR : RF = 2 : 1$.
\item What do you observe?
\item If $\vec{A}, \vec{B}$ and $\vec{C}$  are the vertices of $\triangle ABC$, find the coordinates of the centroid of the triangle.
\end{enumerate}
\solution
	\input{chapters/10/7/4/7/section.tex}
\item Find the slope of a line, which passes through the origin and the mid point of the line segment joining the points $\vec{P}$(0,-4) and $\vec{B}$(8,0).
\label{chapters/11/10/1/5}
\input{chapters/11/10/1/5/matrix.tex}
\item Find the position vector of a point R which divides the line joining two points P and Q whose position vectors are $(2\vec{a}+\vec{b})$ and $(\vec{a}-3\vec{b})$
externally in the ratio 1 : 2. Also, show that P is the mid point of the line segment RQ.\\
	\solution
%		\input{chapters/12/10/5/9/section.tex}

\end{enumerate}


\item Determine the ratio in which the line $2x+y  - 4=0$ divides the line segment joining the points $\vec{A}(2, - 2)$  and  $\vec{B}(3, 7)$.
\\
\solution
	\iffalse
\documentclass[journal,12pt,twocolumn]{IEEEtran}
\usepackage{graphicx}
\graphicspath{{./chapters/10/7/4/1/figs/}}{}
\usepackage{amsmath,amssymb,amsfonts,amsthm}
\newcommand{\myvec}[1]{\ensuremath{\begin{pmatrix}#1\end{pmatrix}}}
\providecommand{\norm}[1]{\lVert#1\rVert}
\usepackage{listings}
\usepackage{watermark}
\usepackage{titlesec}
\usepackage{caption}
\let\vec\mathbf
\lstset{
frame=single, 
breaklines=true,
columns=fullflexible
}
\thiswatermark{\centering \put(0,-105.0){\includegraphics[scale=0.15]{/sdcard/IITH/vector/vectpr-4/chapters/10/7/4/1/figs/logo.png}} }
\title{\mytitle}
\title{
Assignment - Vector-4
}
\author{Surajit Sarkar}
\begin{document}
\maketitle
%\tableofcontents
\bigskip
\section{\textbf{Problem}}
Determine the ratio in which the line 2x+y–4=0 divides the line segment joining the points A(2,–2) and B(3,7).
\section{\textbf{Solution}}
\begin{table}[h]
    \centering
    \begin{tabular}{|c|c|}
       \hline
       \textbf{Symbol}&\textbf{Value}  \\
       \hline
	    $\vec{A}$ & $\myvec{2\\-2}$\\
        \hline
	    $\vec{B}$ & $\myvec{3\\7}$\\
        \hline
	    c&$4$\\
        \hline
       $\vec{n}$ & $\myvec{2\\1}$\\
       \hline
    \end{tabular}
    \caption{Parameters}
    \label{tab:my_label}
\end{table}
Given equation
\fi
The given equation can be expressed as
\begin{align}
    \myvec{2&1}\vec{x}&=4\\
\end{align}
Using section formula, the point of division 
\begin{align}
    \vec{P} = \frac{k\vec{B+A}}{k+1}
\end{align}
which upon substitution in the equation of a line yields
\begin{align}
    \implies\vec{n}^{\top}\myvec{\frac{k\vec{B+A}}{k+1}}&=c\\
    \implies k&=\frac{c-\vec{n}^{\top}\vec{A}}{\vec{n}^{\top}\vec{B}-c}\\
\end{align}
upon simplification.  Substituting numerical values, 
\begin{align}
    k=\frac{2}{9}
\end{align}
See Fig. 
\ref{fig:chapters/10/7/4/1vec}.
\begin{figure}[!h]
\centering
\includegraphics[width=\columnwidth]{chapters/10/7/4/1/figs/vec.pdf}
\caption{}
\label{fig:chapters/10/7/4/1vec}
\end{figure}


\item Let $\vec{A}(4, 2), \vec{B}(6, 5)$  and $ \vec{C}(1, 4)$ be the vertices of $\triangle ABC$.
\begin{enumerate}
\item The median from $\vec{A}$ meets $BC$ at $\vec{D}$. Find the coordinates of the point $\vec{D}$.
\item Find the coordinates of the point $\vec{P}$ on $AD$ such that $AP : PD = 2 : 1$.
\item Find the coordinates of points $\vec{Q}$ and $\vec{R}$ on medians $BE$ and $CF$ respectively such that $BQ : QE = 2 : 1$  and  $CR : RF = 2 : 1$.
\item What do you observe?
\item If $\vec{A}, \vec{B}$ and $\vec{C}$  are the vertices of $\triangle ABC$, find the coordinates of the centroid of the triangle.
\end{enumerate}
\solution
	\iffalse
\documentclass[12pt]{article}
\usepackage{graphicx}
\usepackage[none]{hyphenat}
\usepackage{graphicx}
\usepackage{listings}
\usepackage[english]{babel}
\usepackage{graphicx}
\usepackage{caption} 
\usepackage{booktabs}
\usepackage{array}
\usepackage{amssymb} % for \because
\usepackage{amsmath}   % for having text in math mode
\usepackage{extarrows} % for Row operations arrows
\usepackage{listings}
\usepackage[utf8]{inputenc}
\lstset{
  frame=single,
  breaklines=true
}
\usepackage{hyperref}
  
%Following 2 lines were added to remove the blank page at the beginning
\usepackage{atbegshi}% http://ctan.org/pkg/atbegshi
\AtBeginDocument{\AtBeginShipoutNext{\AtBeginShipoutDiscard}}


%New macro definitions
\newcommand{\mydet}[1]{\ensuremath{\begin{vmatrix}#1\end{vmatrix}}}
\providecommand{\brak}[1]{\ensuremath{\left(#1\right)}}
\newcommand{\solution}{\noindent \textbf{Solution: }}
\newcommand{\myvec}[1]{\ensuremath{\begin{pmatrix}#1\end{pmatrix}}}
\providecommand{\norm}[1]{\left\lVert#1\right\rVert}
\providecommand{\abs}[1]{\left\vert#1\right\vert}
\let\vec\mathbf

\begin{document}

\begin{center}
\title{\textbf{VECTORS}}
\date{\vspace{-5ex}} %Not to print date automatically
\maketitle
\end{center}

\section{10$^{th}$ Maths - EXERCISE-7.4}

Let A(4, 2), B(6, 5) and C(1, 4) be the vertices of $\triangle ABC$
\begin{enumerate}
\item The median from A meets BC at D. Find the coordinates of the point D.
\item Find the coordinates of the point P on AD such that $AP : PD = 2 : 1$
\item Find the coordinates of points Q and R on medians BE and CF respectively such
that $BQ : QE = 2 : 1 \text{and} CR : RF = 2 : 1.$
\item What do yo observe?
\item If $A(x_1, y_1), B(x_2, y_2) \text{and} C(x_3, y_3)$ are the vertices of $\triangle ABC$, find the coordinates of the centroid of the triangle.
\end{enumerate}

Given points are
\begin{align}
\vec{A}=\myvec{4\\ 2} ,
\vec{B}=\myvec{6\\ 5} ,
\vec{C}=\myvec{1\\ 4}
\end{align}
\fi

\begin{enumerate}
\item 
\begin{align}
\vec{D}&=\frac{\vec{B}+\vec{C}}{2}\\
&=\myvec{\frac{7}{2}\\[2pt] \frac{9}{2}}\\
\vec{E}&=\frac{\vec{A}+\vec{C}}{2}\\
&=\myvec{\frac{5}{2}\\ 3}\\
\vec{F}&=\frac{\vec{A}+\vec{B}}{2}\\
&=\myvec{5\\ \frac{7}{2}}
\end{align}

\item 
	For
$n=2$,
\begin{align}
\vec{P}&=\frac{1}{1+n}\brak{\myvec{\vec{A}+n\vec{D}}}\\
&=\frac{1}{3}\myvec{11\\11}
\end{align}

\item 
\begin{align}
\vec{Q}&=\frac{1}{1+n}\brak{\myvec{\vec{B}+n\vec{E}}}\\
&=\frac{1}{3}\myvec{11\\11}\\
\vec{R}&=\frac{1}{1+n}\brak{\myvec{\vec{C}+n\vec{F}}}\\
&=\frac{1}{3}\myvec{11\\11}\\
\end{align}

\item 
 $\vec{P},\vec{Q},\vec{R}$ are the same point.
   
\item 
\begin{align}
\vec{G}&=\frac{\vec{D}+\vec{E}+\vec{F}}{3}\\
&=\frac{1}{3}\myvec{11\\11}\\
\end{align} 
\end{enumerate}
See Fig.  
  \ref{fig:chapters/10/7/4/7/Figure}.
\begin{figure}[h!]
\centering
\includegraphics[width=\columnwidth]{chapters/10/7/4/7/figs/dj.pdf}
\caption{}
  \label{fig:chapters/10/7/4/7/Figure}
\end{figure}

\item Find the slope of a line, which passes through the origin and the mid point of the line segment joining the points $\vec{P}$(0,-4) and $\vec{B}$(8,0).
\label{chapters/11/10/1/5}
\iffalse
\documentclass[journal,12pt,twocolumn]{IEEEtran}
\usepackage{graphicx}
\graphicspath{{./figs/}}{}
\usepackage{amsmath,amssymb,amsfonts,amsthm}
\newcommand{\myvec}[1]{\ensuremath{\begin{pmatrix}#1\end{pmatrix}}}

\let\vec\mathbf

\title{
Matrix-Lines
}
\author{Jyothsna Paluchuri-FWC22059\\}
\begin{document}
\maketitle
\tableofcontents
\bigskip
\section{Problem Statement}
\fi
	\begin{figure}[!ht]
		\centering
 \includegraphics[width=\columnwidth]{chapters/11/10/1/5/figs/line.png}
		\caption{}
		\label{fig:11/10/1/5}
  	\end{figure}
	\\
	\solution
\iffalse
\section{Construction}
\begin{figure}[h]
    \centering
\includegraphics[width=\columnwidth]{line.png}
    \caption{Equation of the slope}
    \label{fig:my_label}
\end{figure}
\vspace{2cm}
\begin{table}[h]
    \centering
    \begin{tabular}{|c|c|c|c|}
       \hline
       \textbf{Symbol}&\textbf{Value}&\textbf{Description}  \\
       \hline
	    $\vec{P}$ & $\myvec{
		    0\\
		    -4}$
	    & Point on Y-axis\\
        \hline
	    $\vec{B}$ & $\myvec{8\\0}$
 & Point on X-axis\\
        \hline
	    $\vec{0}$ & $\myvec{0\\0}$
 & Origin\\
        \hline
    \end{tabular}
    \caption{Parameters}
    \label{tab:my_label}
\end{table}


\section{Solution}
Given that resultant line passes through origin and mid point of the line segment joining point P(0,-4) and B(8,0) \\
\\
\\
given ${\vec{P}}$=$\myvec{
  0\\
  -4}$
 , ${\vec{B}}$=$\myvec{
  8\\
  0}$
  
 \fi 
The mid point of $PB$ is
\begin{align}
\vec{M} &=\frac{1}{2}(\vec{P}+\vec{B})
	= \myvec{4 \\ -2}  
\end{align}
The direction vector of line joining $\vec{O}, \vec{M}$ is 
\begin{align}
\vec{m}&=\vec{O}-\vec{M}
 = -\vec{M}
\end{align}
which can be expressed as
\begin{align}
	\myvec{1 \\ -\frac{1}{2}}
\end{align}
Thus the slope is
\begin{align}
	m = -\frac{1}{2}
\end{align}
\iffalse
\textbf{The direction vector of a line expressed as}
\begin{align}
\implies\vec{m} &= \begin{pmatrix}1 \\ m \\ \end{pmatrix}
\end{align}

\textbf{By solving equation (5) and (6),we get the slope of $\vec{O}$ $\vec{M}$ line}
\begin{align}
        \boxed{m=-0.5}
 \end{align}

\section{Software}
Download the following code using,
\begin{table}[h]
    \centering
    \begin{tabular}{|c|}
    \hline \\
   https://github.com/jyothsna777/jyothsna-fwc.git  \\
         \\
\hline
    \end{tabular}
\end{table}
\\
and execute the code by using command
\begin{center}
\textbf{Python3 lines.py}\\
\end{center}

\section{Conclusion}
Hence the slope of line $\vec{O}$ $\vec{M}$ lineis $\vec{m}$=-0.5

\end{document}
\fi

\item Find the position vector of a point R which divides the line joining two points P and Q whose position vectors are $(2\vec{a}+\vec{b})$ and $(\vec{a}-3\vec{b})$
externally in the ratio 1 : 2. Also, show that P is the mid point of the line segment RQ.\\
	\solution
%		\begin{enumerate}[label=\thesection.\arabic*,ref=\thesection.\theenumi]
\numberwithin{equation}{enumi}
\numberwithin{figure}{enumi}
\numberwithin{table}{enumi}

\item Find the coordinates of the point which divides the join of $(-1,7) \text{ and } (4,-3)$ in the ratio 2:3.
	\\
		\solution
	\input{chapters/10/7/2/1/section.tex}
\item Find the coordinates of the points of trisection of the line segment joining $(4,-1) \text{ and } (-2,3)$.
	\\
		\solution
	\input{chapters/10/7/2/2/section.tex}
\item
	\iffalse
\item To conduct Sports Day activities, in your rectangular shaped school                   
ground ABCD, lines have 
drawn with chalk powder at a                 
distance of 1m each. 100 flower pots have been placed at a distance of 1m 
from each other along AD, as shown 
in Fig. 7.12. Niharika runs $ \frac {1}{4} $th the 
distance AD on the 2nd line and 
posts a green flag. Preet runs $ \frac {1}{5} $th 
the distance AD on the eighth line 
and posts a red flag. What is the 
distance between both the flags? If 
Rashmi has to post a blue flag exactly 
halfway between the line segment 
joining the two flags, where should 
she post her flag?
\begin{figure}[h!]
  \centering
  \includegraphics[width=\columnwidth]{sc.png}
  \caption{}
\label{fig:10/7/12Fig1}
\end{figure}               
\fi
      
\item Find the ratio in which the line segment joining the points $(-3,10) \text{ and } (6,-8)$ $\text{ is divided by } (-1,6)$.
	\\
		\solution
	\input{chapters/10/7/2/4/section.tex}
\item Find the ratio in which the line segment joining $A(1,-5) \text{ and } B(-4,5)$ $\text{is divided by the x-axis}$. Also find the coordinates of the point of division.
\item If $(1,2), (4,y), (x,6), (3,5)$ are the vertices of a parallelogram taken in order, find x and y.
	\\
		\solution
	\input{chapters/10/7/2/6/para1.tex}
\item Find the coordinates of a point A, where AB is the diameter of a circle whose centre is $(2,-3) \text{ and }$ B is $(1,4)$.
	\\
		\solution
	\input{chapters/10/7/2/7/section.tex}
\item If A \text{ and } B are $(-2,-2) \text{ and } (2,-4)$, respectively, find the coordinates of P such that AP= $\frac {3}{7}$AB $\text{ and }$ P lies on the line segment AB.
	\\
		\solution
	\input{chapters/10/7/2/8/section.tex}
\item Find the coordinates of the points which divide the line segment joining $A(-2,2) \text{ and } B(2,8)$ into four equal parts.
	\\
		\solution
	\input{chapters/10/7/2/9/section.tex}
\item Find the area of a rhombus if its vertices are $(3,0), (4,5), (-1,4) \text{ and } (-2,-1)$ taken in order. [$\vec{Hint}$ : Area of rhombus =$\frac {1}{2}$(product of its diagonals)]
	\\
		\solution
	\input{chapters/10/7/2/10/cross.tex}
\item Find the position vector of a point R which divides the line joining two points $\vec{P}$
and $\vec{Q}$ whose position vectors are $\hat{i}+2\hat{j}-\hat{k}$ and $-\hat{i}+\hat{j}+\hat{k}$ respectively, in the
ratio 2 : 1
\begin{enumerate}
    \item  internally
    \item  externally
\end{enumerate}
\solution
		\input{chapters/12/10/2/15/section.tex}
\item Find the position vector of the mid point of the vector joining the points $\vec{P}$(2, 3, 4)
and $\vec{Q}$(4, 1, –2).
\\
\solution
		\input{chapters/12/10/2/16/section.tex}
\item Determine the ratio in which the line $2x+y  - 4=0$ divides the line segment joining the points $\vec{A}(2, - 2)$  and  $\vec{B}(3, 7)$.
\\
\solution
	\input{chapters/10/7/4/1/section.tex}
\item Let $\vec{A}(4, 2), \vec{B}(6, 5)$  and $ \vec{C}(1, 4)$ be the vertices of $\triangle ABC$.
\begin{enumerate}
\item The median from $\vec{A}$ meets $BC$ at $\vec{D}$. Find the coordinates of the point $\vec{D}$.
\item Find the coordinates of the point $\vec{P}$ on $AD$ such that $AP : PD = 2 : 1$.
\item Find the coordinates of points $\vec{Q}$ and $\vec{R}$ on medians $BE$ and $CF$ respectively such that $BQ : QE = 2 : 1$  and  $CR : RF = 2 : 1$.
\item What do you observe?
\item If $\vec{A}, \vec{B}$ and $\vec{C}$  are the vertices of $\triangle ABC$, find the coordinates of the centroid of the triangle.
\end{enumerate}
\solution
	\input{chapters/10/7/4/7/section.tex}
\item Find the slope of a line, which passes through the origin and the mid point of the line segment joining the points $\vec{P}$(0,-4) and $\vec{B}$(8,0).
\label{chapters/11/10/1/5}
\input{chapters/11/10/1/5/matrix.tex}
\item Find the position vector of a point R which divides the line joining two points P and Q whose position vectors are $(2\vec{a}+\vec{b})$ and $(\vec{a}-3\vec{b})$
externally in the ratio 1 : 2. Also, show that P is the mid point of the line segment RQ.\\
	\solution
%		\input{chapters/12/10/5/9/section.tex}

\end{enumerate}



\end{enumerate}



\end{enumerate}


\item Let $\vec{A}(4, 2), \vec{B}(6, 5)$  and $ \vec{C}(1, 4)$ be the vertices of $\triangle ABC$.

\begin{enumerate}
\item The median from $\vec{A}$ meets $BC$ at $\vec{D}$. Find the coordinates of the point $\vec{D}$.
\item Find the coordinates of the point $\vec{P}$ on $AD$ such that $AP : PD = 2 : 1$.
\item Find the coordinates of points $\vec{Q}$ and $\vec{R}$ on medians $BE$ and $CF$ respectively such that $BQ : QE = 2 : 1$  and  $CR : RF = 2 : 1$.
\item What do you observe?
\item If $\vec{A}, \vec{B}$ and $\vec{C}$  are the vertices of $\triangle ABC$, find the coordinates of the centroid of the triangle.
\end{enumerate}
\solution
	\iffalse
\documentclass[12pt]{article}
\usepackage{graphicx}
\usepackage[none]{hyphenat}
\usepackage{graphicx}
\usepackage{listings}
\usepackage[english]{babel}
\usepackage{graphicx}
\usepackage{caption} 
\usepackage{booktabs}
\usepackage{array}
\usepackage{amssymb} % for \because
\usepackage{amsmath}   % for having text in math mode
\usepackage{extarrows} % for Row operations arrows
\usepackage{listings}
\usepackage[utf8]{inputenc}
\lstset{
  frame=single,
  breaklines=true
}
\usepackage{hyperref}
  
%Following 2 lines were added to remove the blank page at the beginning
\usepackage{atbegshi}% http://ctan.org/pkg/atbegshi
\AtBeginDocument{\AtBeginShipoutNext{\AtBeginShipoutDiscard}}


%New macro definitions
\newcommand{\mydet}[1]{\ensuremath{\begin{vmatrix}#1\end{vmatrix}}}
\providecommand{\brak}[1]{\ensuremath{\left(#1\right)}}
\newcommand{\solution}{\noindent \textbf{Solution: }}
\newcommand{\myvec}[1]{\ensuremath{\begin{pmatrix}#1\end{pmatrix}}}
\providecommand{\norm}[1]{\left\lVert#1\right\rVert}
\providecommand{\abs}[1]{\left\vert#1\right\vert}
\let\vec\mathbf

\begin{document}

\begin{center}
\title{\textbf{VECTORS}}
\date{\vspace{-5ex}} %Not to print date automatically
\maketitle
\end{center}

\section{10$^{th}$ Maths - EXERCISE-7.4}

Let A(4, 2), B(6, 5) and C(1, 4) be the vertices of $\triangle ABC$
\begin{enumerate}
\item The median from A meets BC at D. Find the coordinates of the point D.
\item Find the coordinates of the point P on AD such that $AP : PD = 2 : 1$
\item Find the coordinates of points Q and R on medians BE and CF respectively such
that $BQ : QE = 2 : 1 \text{and} CR : RF = 2 : 1.$
\item What do yo observe?
\item If $A(x_1, y_1), B(x_2, y_2) \text{and} C(x_3, y_3)$ are the vertices of $\triangle ABC$, find the coordinates of the centroid of the triangle.
\end{enumerate}

Given points are
\begin{align}
\vec{A}=\myvec{4\\ 2} ,
\vec{B}=\myvec{6\\ 5} ,
\vec{C}=\myvec{1\\ 4}
\end{align}
\fi

\begin{enumerate}
\item 
\begin{align}
\vec{D}&=\frac{\vec{B}+\vec{C}}{2}\\
&=\myvec{\frac{7}{2}\\[2pt] \frac{9}{2}}\\
\vec{E}&=\frac{\vec{A}+\vec{C}}{2}\\
&=\myvec{\frac{5}{2}\\ 3}\\
\vec{F}&=\frac{\vec{A}+\vec{B}}{2}\\
&=\myvec{5\\ \frac{7}{2}}
\end{align}

\item 
	For
$n=2$,
\begin{align}
\vec{P}&=\frac{1}{1+n}\brak{\myvec{\vec{A}+n\vec{D}}}\\
&=\frac{1}{3}\myvec{11\\11}
\end{align}

\item 
\begin{align}
\vec{Q}&=\frac{1}{1+n}\brak{\myvec{\vec{B}+n\vec{E}}}\\
&=\frac{1}{3}\myvec{11\\11}\\
\vec{R}&=\frac{1}{1+n}\brak{\myvec{\vec{C}+n\vec{F}}}\\
&=\frac{1}{3}\myvec{11\\11}\\
\end{align}

\item 
 $\vec{P},\vec{Q},\vec{R}$ are the same point.
   
\item 
\begin{align}
\vec{G}&=\frac{\vec{D}+\vec{E}+\vec{F}}{3}\\
&=\frac{1}{3}\myvec{11\\11}\\
\end{align} 
\end{enumerate}
See Fig.  
  \ref{fig:chapters/10/7/4/7/Figure}.
\begin{figure}[h!]
\centering
\includegraphics[width=\columnwidth]{chapters/10/7/4/7/figs/dj.pdf}
\caption{}
  \label{fig:chapters/10/7/4/7/Figure}
\end{figure}



\item $ABCD$ is a rectangle formed by the points $\vec{A}(–1, –1), \vec{B}(– 1, 4), \vec{C}(5, 4)$  and  $\vec{D}(5, – 1)$. $\vec{P}, \vec{Q}, \vec{R}$ and $\vec{S}$ are the mid-points of $AB, BC, CD$ and $DA$ respectively. Is the quadrilateral $PQRS$ a square? a rectangle? or a rhombus? Justify your answer.
	\\
	\iffalse
\documentclass[12pt]{article}
\usepackage{graphicx}
\usepackage{amsmath}
\usepackage{mathtools}
\usepackage{gensymb}

\newcommand{\mydet}[1]{\ensuremath{\begin{vmatrix}#1\end{vmatrix}}}
\providecommand{\brak}[1]{\ensuremath{\left(#1\right)}}
\providecommand{\norm}[1]{\left\lVert#1\right\rVert}
\newcommand{\solution}{\noindent \textbf{Solution: }}
\newcommand{\myvec}[1]{\ensuremath{\begin{pmatrix}#1\end{pmatrix}}}
\let\vec\mathbf

\begin{document}
\begin{center}
\textbf\large{CHAPTER-7 \\ COORDINATE GEOMETRY}

\end{center}
\section*{Excercise 7.1}

Q6.Name the type of quadilateral formed,if any, by the following points, and give reasons for your answer:
\begin{enumerate}
	\item $\brak{-1,-2}, \brak{1,0}, \brak{-1,2}, \brak{-3,0}$ 
	\item $\brak{-3,5}, \brak{3,1}, \brak{0,3}, \brak{-1,-4}$
	\item $\brak{4,5}, \brak{7,6}, \brak{4,3}, \brak{1,2}$
\end{enumerate}
\solution
\fi
\begin{enumerate}
\item The coordinates are given as
	\begin{align}
	\vec{A} = \myvec{
		-1\\
		-2\\
		},
	\vec{B} = \myvec{
		1\\
		0\\
		},
	\vec{C} = \myvec{
		-1\\
		2\\
		} \text{ and }
	\vec{D} = \myvec{
		-3\\
		0\\
		}
	\end{align}
	\begin{align}
		\vec{B} - \vec{A} &= \myvec{1\\0} - \myvec{-1\\-2} = \myvec{2\\2}\\
		\vec{C} - \vec{B} &= \myvec{-1\\2} - \myvec{1\\0} = \myvec{-2\\2}\\
		\vec{C} - \vec{D} &= \myvec{-1\\2} - \myvec{-3\\0} = \myvec{2\\2}\\
		\vec{D} - \vec{A} &= \myvec{-3\\0} - \myvec{-1\\-2} = \myvec{-2\\2}
	\end{align}
	\begin{align}	
		\vec{C} - \vec{A} &= \myvec{-1\\2} - \myvec{-1\\-2} = \myvec{0\\4}\\
		\vec{D} - \vec{B} &= \myvec{-3\\0} - \myvec{1\\0} = \myvec{-4\\0}
	\end{align}
	\begin{align}	
		\vec{B}-\vec{A} = \vec{C}-\vec{D} \text{ and } \vec{C}-\vec{B} = \vec{D}-\vec{A}.
	\end{align}
	Hence, $ABCD$ is a parallelogram.
	\begin{enumerate}
		\item Now checking if the adjacent sides are orthogonal to each other
	\begin{align}
		(\vec{B}-\vec{A})^\top (\vec{C}-\vec{B}) = \myvec{2&2} \myvec{-2\\2} = -4+4 = 0
	\end{align}
		\item Now checking if the diagonals are also orthogonal then it is a square else a rectangle.
	\end{enumerate}	
	\begin{align}
		(\vec{C}-\vec{A})^\top (\vec{D}-\vec{B}) = \myvec{0&4} \myvec{-4\\0} = 0
	\end{align}
	Hence the diagonals are orthogonal to each other.

	So, we can conclude that $ABCD$ is a square.

	As shown in Figure \ref{fig:10/7/1/6/Fig1} we can see that $ABCD$ is a square hence we can conclude that our theoritical result is verified.
 
\begin{figure}[!h]
	\begin{center} 
	    \includegraphics[width=\columnwidth]{chapters/10/7/1/6/figs/quad1}
	\end{center}
\caption{}
\label{fig:10/7/1/6/Fig1}
\end{figure}

\item The coordinates are given as
	\begin{align}
	\vec{A} = \myvec{
		-3\\
		5\\
		},
	\vec{B} = \myvec{
		3\\
		1\\
		},
	\vec{C} = \myvec{
		0\\
		3\\
		} \text{ and }
	\vec{D} = \myvec{
		-1\\
		-4\\
		}
	\end{align}
	\begin{align}
		\vec{B} - \vec{A} &= \myvec{3\\1} - \myvec{-3\\5} = \myvec{6\\-4}\\
		\vec{C} - \vec{B} &= \myvec{0\\3} - \myvec{3\\1} = \myvec{-3\\2}\\
		\vec{C} - \vec{D} &= \myvec{0\\3} - \myvec{-1\\-4} = \myvec{1\\7}\\
		\vec{D} - \vec{A} &= \myvec{-1\\-4} - \myvec{-3\\5} = \myvec{2\\-9}
	\end{align}
	\begin{align}
		\vec{C} - \vec{A} &= \myvec{0\\3} - \myvec{-3\\5} = \myvec{3\\-2}\\
		\vec{D} - \vec{B} &= \myvec{-1\\-4} - \myvec{3\\1} = \myvec{-4\\-5}
	\end{align}
	\begin{align}
	\vec{B}-\vec{A} \neq \vec{C}-\vec{D} \text{ and } \vec{C}-\vec{B} \neq \vec{D}-\vec{A},
	\end{align}
	Hence, $ABCD$ is not a parallelogram, it can be a irregular quadilateral.
	\begin{enumerate}
		\item Now to check if any three points are collinear,

	if rank of $\myvec{\vec{B}-\vec{A} & \vec{C}-\vec{B}} = 1$ then points are collinear

	Forming the collinearity matrix
	\begin{align}
		\myvec{6&-3\\-4&2} \xleftrightarrow{R_{2}\rightarrow R_{2}+\frac{2}{3}R_{1}}= \myvec{6&-3\\0&0}
	\end{align}
	\end{enumerate}
	Hence, rank = 1

	Since none of the opposite sides are parallel to each other and three points are collinear so these does not form a quadilateral.

	As shown in Figure \ref{fig:10/7/1/6/Fig2} we can see that $ABCD$ does not form a quadilateral and three points are collinear hence, our theoritical result is verified.
	
\begin{figure}[!h]
	\begin{center} 
	    \includegraphics[width=\columnwidth]{chapters/10/7/1/6/figs/quad2}
	\end{center}
\caption{}
\label{fig:10/7/1/6/Fig2}
\end{figure}
	
\item The coordinates are given as
	\begin{align}
	\vec{A} = \myvec{
		4\\
		5\\
		},
	\vec{B} = \myvec{
		7\\
		6\\
		},
	\vec{C} = \myvec{
		4\\
		3\\
		} \text{ and }
	\vec{D} = \myvec{
		1\\
		2\\
		}
	\end{align}
	\begin{align}
		\vec{B} - \vec{A} &= \myvec{7\\6} - \myvec{4\\5} = \myvec{3\\1}\\
		\vec{C} - \vec{B} &= \myvec{4\\3} - \myvec{7\\6} = \myvec{-3\\-3}\\
		\vec{C} - \vec{D} &= \myvec{4\\3} - \myvec{1\\2} = \myvec{3\\1}\\
		\vec{D} - \vec{A} &= \myvec{1\\2} - \myvec{4\\5} = \myvec{-3\\-3}
	\end{align}
	\begin{align}
		\vec{C} - \vec{A} &= \myvec{4\\3} - \myvec{4\\5} = \myvec{0\\-2}\\
		\vec{D} - \vec{B} &= \myvec{1\\2} - \myvec{7\\6} = \myvec{-6\\-4}
	\end{align}
	\begin{align}
		\vec{B}-\vec{A} = \vec{C}-\vec{D} \text{ and } \vec{C}-\vec{B} = \vec{D}-\vec{A},
	\end{align}
	Hence, $ABCD$ is a parallelogram.
	\begin{enumerate}
		\item Now checking if the adjacent sides are orthogonal to each other
	\begin{align}
		(\vec{B}-\vec{A})^\top (\vec{C}-\vec{B}) = \myvec{3&1} \myvec{-3\\-3} = -9-3 = -12
	\end{align}
	Since inner product is not zero so adjacent sides are not orthogonal.

	Hence, we can say that $ABCD$ is neither a rectangle nor a square.

		\item Now checking if the diagonals are orthogonal then it is a Rhombus.
	\begin{align}
		(\vec{C}- \vec{A})^\top (\vec{D}-\vec{B}) = \myvec{0&-2} \myvec{-6\\-4} = 0+8 = 8
	\end{align}
	\end{enumerate}		
	Hence the diagonals are also not orthogonal so we conclude that $ABCD$ is a parallelogram.

	As shown in Figure \ref{fig:10/7/1/6/Fig3} we can see that $ABCD$ forms a parallelogram hence, our theoritical result is verified.

\begin{figure}[!h]
	\begin{center} 
	    \includegraphics[width=\columnwidth]{chapters/10/7/1/6/figs/quad3}
	\end{center}
\caption{}
\label{fig:10/7/1/6/Fig3}
\end{figure}
\end{enumerate}





\end{enumerate}



