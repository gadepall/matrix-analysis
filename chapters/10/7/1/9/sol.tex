\iffalse
\documentclass[12pt]{article}
\usepackage{graphicx}
%\documentclass[journal,12pt,twocolumn]{IEEEtran}
\usepackage[none]{hyphenat}
\usepackage{graphicx}
\usepackage{listings}
\usepackage[english]{babel}
\usepackage{graphicx}
\usepackage{caption} 
\usepackage{hyperref}
\usepackage{booktabs}
\def\inputGnumericTable{}
\usepackage{color}                                            %%
    \usepackage{array}                                            %%
    \usepackage{longtable}                                        %%
    \usepackage{calc}                                             %%
    \usepackage{multirow}                                         %%
    \usepackage{hhline}                                           %%
    \usepackage{ifthen}
\usepackage{array}
\usepackage{amsmath}   % for having text in math mode
\usepackage{listings}
\lstset{
language=tex,
frame=single, 
breaklines=true
}
  
%Following 2 lines were added to remove the blank page at the beginning
\usepackage{atbegshi}% http://ctan.org/pkg/atbegshi
\AtBeginDocument{\AtBeginShipoutNext{\AtBeginShipoutDiscard}}
%


%New macro definitions
\newcommand{\mydet}[1]{\ensuremath{\begin{vmatrix}#1\end{vmatrix}}}
\providecommand{\brak}[1]{\ensuremath{\left(#1\right)}}
\providecommand{\norm}[1]{\left\lVert#1\right\rVert}
\newcommand{\solution}{\noindent \textbf{Solution: }}
\newcommand{\myvec}[1]{\ensuremath{\begin{pmatrix}#1\end{pmatrix}}}
\let\vec\mathbf

\begin{document}

\begin{center}
\title{\textbf{Coordinate Geometry}}
\date{\vspace{-5ex}} %Not to print date automatically
\maketitle
\end{center}

\setcounter{page}{1}



\section*{10$^{th}$ Maths - Chapter 7}

This is Problem-8 from Exercise 7.1

\begin{enumerate}

\item If $\vec{Q}$= $\myvec{0\\ 1}$  is equidistant from $\vec{P}$=$\myvec{5 \\ -3}$ and $\vec{R}$=$\myvec{x\\6}$. Find the values of $x$.\\

\solution \\
\fi
		The input parameters for this problem are available in Table
\begin{table}[ht!]\centering
\begin{tabular}{|c|c|p{5cm}|}
\hline
\textbf{Symbol} & \textbf{Value} & \textbf{Description} \\
\hline
$\theta$ & $30\degree$ & $\angle{BAP} = \angle{BAQ}$ \\
\hline
$a$ & $9$ & $AB$ \\
\hline
$c$ & $8$ & $AQ$ \\
\hline
$\vec{e}_1$ & $\myvec{1\\0}$ & Basis vector \\
\hline
\end{tabular}

\caption{}
\label{Table-1}	
\end{table}


  If $\vec{Q}$  is  equidistant from the points $\vec{P}$ and $\vec{R}$, 
\begin{align}
 \norm{\vec{P}-\vec{Q}} &=
\norm{\vec{R}-\vec{Q}} 
\\
 \implies \norm{\vec{P}-\vec{Q}}^2 &=
\norm{\vec{R}-\vec{Q}}^2 
\end{align}
which can be expressed as 
\begin{align}
%  \label{eq:norm2d_dist}
 \brak{\vec{P}-\vec{Q}}^{\top} \brak{\vec{P}-\vec{Q}}&=
 \brak{\vec{R}-\vec{Q}}^{\top} 
\brak{\vec{R}-\vec{Q}}
\\ \norm{\vec{Q}}^2-2{\vec{Q}}^{\top}\vec{P} + \norm{\vec{P}}^2
 &= \norm{\vec{Q}}^2-2{\vec{Q}}^{\top}\vec{R} + \norm{\vec{R}}^2
\end{align}
which can be simplified to obtain
  \begin{align}
   \implies \norm{\vec{P}}^2-2{\vec{Q}}^{\top}\vec{P} &= \norm{\vec{R}}^2-2{\vec{Q}}^{\top}\vec{R} 
\label{eq:4} 
  \end{align}
  now substituting the P,Q and R values in \eqref{eq:4}

  \begin{align}
   \norm{\vec{P}}^2 &= 34
  \\ \norm{\vec{R}}^2 &= x^2+36
 \\2{\vec{Q}}^{\top}\vec{P} &= -6
 \\2{\vec{Q}}^{\top}\vec{R} &= 12
 \end{align}
upon   substituting the values in \eqref{eq:4}
\begin{align}
 x^2+36-12 &=34+6 
 \\ \implies  x^2 &=16
\end{align}

Then the value of $x$ = $ 4$ or $-4$.
Hence, the desired point is $\vec{R}$ is$\myvec{ 4 \\ 6}$ or $\myvec{-4\\6}$

\begin{figure}[!h]
 \begin{center}
  \includegraphics[width=\columnwidth]{figs/fig.pdf}
 \end{center}
\caption{}
\label{fig:Fig1}
\end{figure}



\end{enumerate}

\end{document}
