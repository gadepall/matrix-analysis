\let\negmedspace\undefined
\let\negthickspace\undefined
\documentclass[journal,12pt,onecolumn]{IEEEtran}
\usepackage{gvv-book}
\usepackage{gvv}
\begin{document}

\title{
%	\logo{
Matrix Analysis Through Coordinate Geometry
%	}
}
\author{ G V V Sharma$^{*}$% <-this % stops a space
\iffalse
	\thanks{*The author is with the Department
		of Electrical Engineering, Indian Institute of Technology, Hyderabad
		502285 India e-mail:  gadepall@iith.ac.in. All content in this manual is released under GNU GPL.  Free and open source.}
		\fi
	
}	
%\title{
%	\logo{Matrix Analysis through Octave}{\begin{center}\includegraphics[scale=.24]{tlc}\end{center}}{}{HAMDSP}
%}


% paper title
% can use linebreaks \\ within to get better formatting as desired
%\title{Matrix Analysis through Octave}
%
%
% author names and IEEE memberships
% note positions of commas and nonbreaking spaces ( ~ ) LaTeX will not break
% a structure at a ~ so this keeps an author's name from being broken across
% two lines.
% use \thanks{} to gain access to the first footnote area
% a separate \thanks must be used for each paragraph as LaTeX2e's \thanks
% was not built to handle multiple paragraphs
%

%\author{<-this % stops a space
%\thanks{}}
%}
% note the % following the last \IEEEmembership and also \thanks - 
% these prevent an unwanted space from occurring between the last author name
% and the end of the author line. i.e., if you had this:
% 
% \author{....lastname \thanks{...} \thanks{...} }
%                     ^------------^------------^----Do not want these spaces!
%
% a space would be appended to the last name and could cause every name on that
% line to be shifted left slightly. This is one of those "LaTeX things". For
% instance, "\textbf{A} \textbf{B}" will typeset as "A B" not "AB". To get
% "AB" then you have to do: "\textbf{A}\textbf{B}"
% \thanks is no different in this regard, so shield the last } of each \thanks
% that ends a line with a % and do not let a space in before the next \thanks.
% Spaces after \IEEEmembership other than the last one are OK (and needed) as
% you are supposed to have spaces between the names. For what it is worth,
% this is a minor point as most people would not even notice if the said evil
% space somehow managed to creep in.



% The paper headers
%\markboth{Journal of \LaTeX\ Class Files,~Vol.~6, No.~1, January~2007}%
%{Shell \MakeLowercase{\textit{et al.}}: Bare Demo of IEEEtran.cls for Journals}
% The only time the second header will appear is for the odd numbered pages
% after the title page when using the twoside option.
% 
% *** Note that you probably will NOT want to include the author's ***
% *** name in the headers of peer review papers.                   ***
% You can use \ifCLASSOPTIONpeerreview for conditional compilation here if
% you desire.




% If you want to put a publisher's ID mark on the page you can do it like
% this:
%\IEEEpubid{0000--0000/00\$00.00~\copyright~2007 IEEE}
% Remember, if you use this you must call \IEEEpubidadjcol in the second
% column for its text to clear the IEEEpubid mark.



% make the title area
\maketitle

\newpage

\tableofcontents

\bigskip

\renewcommand{\thefigure}{\theenumi}
\renewcommand{\thetable}{\theenumi}
%\renewcommand{\theequation}{\theenumi}

%\begin{abstract}
%%\boldmath
%In this letter, an algorithm for evaluating the exact analytical bit error rate  (BER)  for the piecewise linear (PL) combiner for  multiple relays is presented. Previous results were available only for upto three relays. The algorithm is unique in the sense that  the actual mathematical expressions, that are prohibitively large, need not be explicitly obtained. The diversity gain due to multiple relays is shown through plots of the analytical BER, well supported by simulations. 
%
%\end{abstract}
% IEEEtran.cls defaults to using nonbold math in the Abstract.
% This preserves the distinction between vectors and scalars. However,
% if the journal you are submitting to favors bold math in the abstract,
% then you can use LaTeX's standard command \boldmath at the very start
% of the abstract to achieve this. Many IEEE journals frown on math
% in the abstract anyway.

% Note that keywords are not normally used for peerreview papers.
%\begin{IEEEkeywords}
%Cooperative diversity, decode and forward, piecewise linear
%\end{IEEEkeywords}



% For peer review papers, you can put extra information on the cover
% page as needed:
% \ifCLASSOPTIONpeerreview
% \begin{center} \bfseries EDICS Category: 3-BBND \end{center}
% \fi
%
% For peerreview papers, this IEEEtran command inserts a page break and
% creates the second title. It will be ignored for other modes.
%\IEEEpeerreviewmaketitle

\begin{abstract}
This manual includes \LaTeX figures.
%book provides an introduction to optimization  based on the NCERT textbooks from Class 6-12.  Links to sample Python codes are available in the text.  
\end{abstract}
Download 
\begin{lstlisting}
svn co https://github.com/gadepall/school/trunk/training
\end{lstlisting}
\section{Vectors}
\subsection{Length}
\begin{enumerate}[label=\thesection.\arabic*,ref=\thesection.\theenumi]
\numberwithin{equation}{enumi}
\numberwithin{figure}{enumi}
\numberwithin{table}{enumi}
\item If $\overrightarrow {a}$ is a nonzero vector of magnitude 'a' and $\lambda$ a nonzero scalar , then $\lambda\overrightarrow {a}$ is unit vector if
\begin{enumerate} 
\item $\lambda=1$ 
\item $\lambda=-1$
\item $a=\abs{\lambda}$
\item $a=1/\abs{\lambda}$  
\end{enumerate}
\item Compute the magnitude of the following vectors:
\begin{align*}
\vec{a}=\hat{i}+\hat{j}+k; \vec{b}=2\hat{i}-7\hat{j}-3\hat{k}; \vec{c}=\frac{1}{\sqrt{3}}\hat{i}+\frac{1}{\sqrt{3}}\hat{j}-\frac{1}{3}\hat{k}
\end{align*}
    \solution 
		\iffalse
\documentclass[12pt]{article}
\usepackage{graphicx}
%\documentclass[journal,12pt,twocolumn]{IEEEtran}
\usepackage[none]{hyphenat}
\usepackage{graphicx}
\usepackage{listings}
\usepackage[english]{babel}
\usepackage{graphicx}
\usepackage{caption} 
\usepackage{hyperref}
\usepackage{booktabs}
\usepackage{array}
\usepackage{amsmath}   % for having text in math mode
\usepackage{listings}
\lstset{
  frame=single,
  breaklines=true
}
  
%Following 2 lines were added to remove the blank page at the beginning
\usepackage{atbegshi}% http://ctan.org/pkg/atbegshi
\AtBeginDocument{\AtBeginShipoutNext{\AtBeginShipoutDiscard}}
%


%New macro definitions
\newcommand{\mydet}[1]{\ensuremath{\begin{vmatrix}#1\end{vmatrix}}}
\providecommand{\brak}[1]{\ensuremath{\left(#1\right)}}
\providecommand{\norm}[1]{\left\lVert#1\right\rVert}
\newcommand{\solution}{\noindent \textbf{Solution: }}
\newcommand{\myvec}[1]{\ensuremath{\begin{pmatrix}#1\end{pmatrix}}}
\let\vec\mathbf

\begin{document}

\begin{center}
\title{\textbf{Vector Dot Product}}
\date{\vspace{-5ex}} %Not to print date automatically
\maketitle
\end{center}
\setcounter{page}{1}

\section{12$^{th}$ Maths - Chapter 10}
This is Problem-9 from Exercise 10.3
\begin{enumerate}
\item Find $\norm{\vec{x}}$, if for a unit vector $\vec{a}$, $\brak{\vec{x}-\vec{a}}.\brak{\vec{x}+\vec{a}} = 12$.\\
	\fi
\solution 
From the given information,
\begin{align}
  \label{eq:12/10/3/9det2f}
  \brak{\vec{x}-\vec{a}}^\top\brak{\vec{x}+\vec{a}} &= 12 \\
  \implies \vec{x}^\top\vec{x} - \vec{a}^\top\vec{x} + \vec{x}^\top\vec{a} - \vec{a}^\top\vec{a} &= 12 \\
  \implies \norm{\vec{x}}^{2} - \norm{\vec{a}}^{2} &= 12 \\
\implies   \norm{\vec{x}}^{2} - 1 &= 12  \\
	\text{or, }  
	\norm{\vec{x}} &= \sqrt{13}
\end{align}

\item Find the unit vector in the direction of the vector $\vec{a}=\hat{i}+\hat{j}+2\hat{k}$.
\item Find the unit vector in the direction of vector $\overrightarrow{PQ}$ , where $\vec{P}$ and $\vec{Q}$ are the points
(1, 2, 3) and (4, 5, 6), respectively.
\item For given vectors, $\vec{a}=2\hat{i}-\hat{j}+2\hat{k}$ and $\vec{b}=-\hat{i}+\hat{j}-\hat{k}$ , find the unit vector in the
direction of the vector $\vec{a}+\vec{b}$.
\\
    \solution 
		\iffalse
\documentclass[12pt]{article}
\usepackage{graphicx}
%\documentclass[journal,12pt,twocolumn]{IEEEtran}
\usepackage[none]{hyphenat}
\usepackage{graphicx}
\usepackage{listings}
\usepackage[english]{babel}
\usepackage{graphicx}
\usepackage{caption} 
\usepackage{hyperref}
\usepackage{booktabs}
\usepackage{array}
\usepackage{amsmath}   % for having text in math mode
\usepackage{listings}
\lstset{
  frame=single,
  breaklines=true
}
  
%Following 2 lines were added to remove the blank page at the beginning
\usepackage{atbegshi}% http://ctan.org/pkg/atbegshi
\AtBeginDocument{\AtBeginShipoutNext{\AtBeginShipoutDiscard}}
%


%New macro definitions
\newcommand{\mydet}[1]{\ensuremath{\begin{vmatrix}#1\end{vmatrix}}}
\providecommand{\brak}[1]{\ensuremath{\left(#1\right)}}
\providecommand{\norm}[1]{\left\lVert#1\right\rVert}
\newcommand{\solution}{\noindent \textbf{Solution: }}
\newcommand{\myvec}[1]{\ensuremath{\begin{pmatrix}#1\end{pmatrix}}}
\let\vec\mathbf

\begin{document}

\begin{center}
\title{\textbf{Vector Dot Product}}
\date{\vspace{-5ex}} %Not to print date automatically
\maketitle
\end{center}
\setcounter{page}{1}

\section{12$^{th}$ Maths - Chapter 10}
This is Problem-9 from Exercise 10.3
\begin{enumerate}
\item Find $\norm{\vec{x}}$, if for a unit vector $\vec{a}$, $\brak{\vec{x}-\vec{a}}.\brak{\vec{x}+\vec{a}} = 12$.\\
	\fi
\solution 
From the given information,
\begin{align}
  \label{eq:12/10/3/9det2f}
  \brak{\vec{x}-\vec{a}}^\top\brak{\vec{x}+\vec{a}} &= 12 \\
  \implies \vec{x}^\top\vec{x} - \vec{a}^\top\vec{x} + \vec{x}^\top\vec{a} - \vec{a}^\top\vec{a} &= 12 \\
  \implies \norm{\vec{x}}^{2} - \norm{\vec{a}}^{2} &= 12 \\
\implies   \norm{\vec{x}}^{2} - 1 &= 12  \\
	\text{or, }  
	\norm{\vec{x}} &= \sqrt{13}
\end{align}

\item Find a vector in the direction of vector $5\hat{i}-\hat{j}+2\hat{k}$ which has magnitude 8 units.
    has magnitude 8 units.
   \\ 
    \solution 
		\iffalse
\documentclass[12pt]{article}
\usepackage{graphicx}
%\documentclass[journal,12pt,twocolumn]{IEEEtran}
\usepackage[none]{hyphenat}
\usepackage{graphicx}
\usepackage{listings}
\usepackage[english]{babel}
\usepackage{graphicx}
\usepackage{caption} 
\usepackage{hyperref}
\usepackage{booktabs}
\usepackage{array}
\usepackage{amsmath}   % for having text in math mode
\usepackage{listings}
\lstset{
  frame=single,
  breaklines=true
}
  
%Following 2 lines were added to remove the blank page at the beginning
\usepackage{atbegshi}% http://ctan.org/pkg/atbegshi
\AtBeginDocument{\AtBeginShipoutNext{\AtBeginShipoutDiscard}}
%


%New macro definitions
\newcommand{\mydet}[1]{\ensuremath{\begin{vmatrix}#1\end{vmatrix}}}
\providecommand{\brak}[1]{\ensuremath{\left(#1\right)}}
\providecommand{\norm}[1]{\left\lVert#1\right\rVert}
\newcommand{\solution}{\noindent \textbf{Solution: }}
\newcommand{\myvec}[1]{\ensuremath{\begin{pmatrix}#1\end{pmatrix}}}
\let\vec\mathbf

\begin{document}

\begin{center}
\title{\textbf{Vector Dot Product}}
\date{\vspace{-5ex}} %Not to print date automatically
\maketitle
\end{center}
\setcounter{page}{1}

\section{12$^{th}$ Maths - Chapter 10}
This is Problem-9 from Exercise 10.3
\begin{enumerate}
\item Find $\norm{\vec{x}}$, if for a unit vector $\vec{a}$, $\brak{\vec{x}-\vec{a}}.\brak{\vec{x}+\vec{a}} = 12$.\\
	\fi
\solution 
From the given information,
\begin{align}
  \label{eq:12/10/3/9det2f}
  \brak{\vec{x}-\vec{a}}^\top\brak{\vec{x}+\vec{a}} &= 12 \\
  \implies \vec{x}^\top\vec{x} - \vec{a}^\top\vec{x} + \vec{x}^\top\vec{a} - \vec{a}^\top\vec{a} &= 12 \\
  \implies \norm{\vec{x}}^{2} - \norm{\vec{a}}^{2} &= 12 \\
\implies   \norm{\vec{x}}^{2} - 1 &= 12  \\
	\text{or, }  
	\norm{\vec{x}} &= \sqrt{13}
\end{align}

\item Find the direction cosines of the vector $\hat{i}+2\hat{j}+3\hat{k}$.
	\\
    \solution 
		\iffalse
\documentclass[12pt]{article}
\usepackage{graphicx}
\usepackage{amsmath}
\usepackage{mathtools}
\usepackage{gensymb}

\newcommand{\mydet}[1]{\ensuremath{\begin{vmatrix}#1\end{vmatrix}}}
\providecommand{\brak}[1]{\ensuremath{\left(#1\right)}}
\providecommand{\norm}[1]{\left\lVert#1\right\rVert}
\newcommand{\solution}{\noindent \textbf{Solution: }}
\newcommand{\myvec}[1]{\ensuremath{\begin{pmatrix}#1\end{pmatrix}}}
\let\vec\mathbf

\begin{document}
\begin{center}
\textbf\large{CHAPTER-7 \\ COORDINATE GEOMETRY}

\end{center}
\section*{Excercise 7.1}

Q6.Name the type of quadilateral formed,if any, by the following points, and give reasons for your answer:
\begin{enumerate}
	\item $\brak{-1,-2}, \brak{1,0}, \brak{-1,2}, \brak{-3,0}$ 
	\item $\brak{-3,5}, \brak{3,1}, \brak{0,3}, \brak{-1,-4}$
	\item $\brak{4,5}, \brak{7,6}, \brak{4,3}, \brak{1,2}$
\end{enumerate}
\solution
\fi
\begin{enumerate}
\item The coordinates are given as
	\begin{align}
	\vec{A} = \myvec{
		-1\\
		-2\\
		},
	\vec{B} = \myvec{
		1\\
		0\\
		},
	\vec{C} = \myvec{
		-1\\
		2\\
		} \text{ and }
	\vec{D} = \myvec{
		-3\\
		0\\
		}
	\end{align}
	\begin{align}
		\vec{B} - \vec{A} &= \myvec{1\\0} - \myvec{-1\\-2} = \myvec{2\\2}\\
		\vec{C} - \vec{B} &= \myvec{-1\\2} - \myvec{1\\0} = \myvec{-2\\2}\\
		\vec{C} - \vec{D} &= \myvec{-1\\2} - \myvec{-3\\0} = \myvec{2\\2}\\
		\vec{D} - \vec{A} &= \myvec{-3\\0} - \myvec{-1\\-2} = \myvec{-2\\2}
	\end{align}
	\begin{align}	
		\vec{C} - \vec{A} &= \myvec{-1\\2} - \myvec{-1\\-2} = \myvec{0\\4}\\
		\vec{D} - \vec{B} &= \myvec{-3\\0} - \myvec{1\\0} = \myvec{-4\\0}
	\end{align}
	\begin{align}	
		\vec{B}-\vec{A} = \vec{C}-\vec{D} \text{ and } \vec{C}-\vec{B} = \vec{D}-\vec{A}.
	\end{align}
	Hence, $ABCD$ is a parallelogram.
	\begin{enumerate}
		\item Now checking if the adjacent sides are orthogonal to each other
	\begin{align}
		(\vec{B}-\vec{A})^\top (\vec{C}-\vec{B}) = \myvec{2&2} \myvec{-2\\2} = -4+4 = 0
	\end{align}
		\item Now checking if the diagonals are also orthogonal then it is a square else a rectangle.
	\end{enumerate}	
	\begin{align}
		(\vec{C}-\vec{A})^\top (\vec{D}-\vec{B}) = \myvec{0&4} \myvec{-4\\0} = 0
	\end{align}
	Hence the diagonals are orthogonal to each other.

	So, we can conclude that $ABCD$ is a square.

	As shown in Figure \ref{fig:10/7/1/6/Fig1} we can see that $ABCD$ is a square hence we can conclude that our theoritical result is verified.
 
\begin{figure}[!h]
	\begin{center} 
	    \includegraphics[width=\columnwidth]{chapters/10/7/1/6/figs/quad1}
	\end{center}
\caption{}
\label{fig:10/7/1/6/Fig1}
\end{figure}

\item The coordinates are given as
	\begin{align}
	\vec{A} = \myvec{
		-3\\
		5\\
		},
	\vec{B} = \myvec{
		3\\
		1\\
		},
	\vec{C} = \myvec{
		0\\
		3\\
		} \text{ and }
	\vec{D} = \myvec{
		-1\\
		-4\\
		}
	\end{align}
	\begin{align}
		\vec{B} - \vec{A} &= \myvec{3\\1} - \myvec{-3\\5} = \myvec{6\\-4}\\
		\vec{C} - \vec{B} &= \myvec{0\\3} - \myvec{3\\1} = \myvec{-3\\2}\\
		\vec{C} - \vec{D} &= \myvec{0\\3} - \myvec{-1\\-4} = \myvec{1\\7}\\
		\vec{D} - \vec{A} &= \myvec{-1\\-4} - \myvec{-3\\5} = \myvec{2\\-9}
	\end{align}
	\begin{align}
		\vec{C} - \vec{A} &= \myvec{0\\3} - \myvec{-3\\5} = \myvec{3\\-2}\\
		\vec{D} - \vec{B} &= \myvec{-1\\-4} - \myvec{3\\1} = \myvec{-4\\-5}
	\end{align}
	\begin{align}
	\vec{B}-\vec{A} \neq \vec{C}-\vec{D} \text{ and } \vec{C}-\vec{B} \neq \vec{D}-\vec{A},
	\end{align}
	Hence, $ABCD$ is not a parallelogram, it can be a irregular quadilateral.
	\begin{enumerate}
		\item Now to check if any three points are collinear,

	if rank of $\myvec{\vec{B}-\vec{A} & \vec{C}-\vec{B}} = 1$ then points are collinear

	Forming the collinearity matrix
	\begin{align}
		\myvec{6&-3\\-4&2} \xleftrightarrow{R_{2}\rightarrow R_{2}+\frac{2}{3}R_{1}}= \myvec{6&-3\\0&0}
	\end{align}
	\end{enumerate}
	Hence, rank = 1

	Since none of the opposite sides are parallel to each other and three points are collinear so these does not form a quadilateral.

	As shown in Figure \ref{fig:10/7/1/6/Fig2} we can see that $ABCD$ does not form a quadilateral and three points are collinear hence, our theoritical result is verified.
	
\begin{figure}[!h]
	\begin{center} 
	    \includegraphics[width=\columnwidth]{chapters/10/7/1/6/figs/quad2}
	\end{center}
\caption{}
\label{fig:10/7/1/6/Fig2}
\end{figure}
	
\item The coordinates are given as
	\begin{align}
	\vec{A} = \myvec{
		4\\
		5\\
		},
	\vec{B} = \myvec{
		7\\
		6\\
		},
	\vec{C} = \myvec{
		4\\
		3\\
		} \text{ and }
	\vec{D} = \myvec{
		1\\
		2\\
		}
	\end{align}
	\begin{align}
		\vec{B} - \vec{A} &= \myvec{7\\6} - \myvec{4\\5} = \myvec{3\\1}\\
		\vec{C} - \vec{B} &= \myvec{4\\3} - \myvec{7\\6} = \myvec{-3\\-3}\\
		\vec{C} - \vec{D} &= \myvec{4\\3} - \myvec{1\\2} = \myvec{3\\1}\\
		\vec{D} - \vec{A} &= \myvec{1\\2} - \myvec{4\\5} = \myvec{-3\\-3}
	\end{align}
	\begin{align}
		\vec{C} - \vec{A} &= \myvec{4\\3} - \myvec{4\\5} = \myvec{0\\-2}\\
		\vec{D} - \vec{B} &= \myvec{1\\2} - \myvec{7\\6} = \myvec{-6\\-4}
	\end{align}
	\begin{align}
		\vec{B}-\vec{A} = \vec{C}-\vec{D} \text{ and } \vec{C}-\vec{B} = \vec{D}-\vec{A},
	\end{align}
	Hence, $ABCD$ is a parallelogram.
	\begin{enumerate}
		\item Now checking if the adjacent sides are orthogonal to each other
	\begin{align}
		(\vec{B}-\vec{A})^\top (\vec{C}-\vec{B}) = \myvec{3&1} \myvec{-3\\-3} = -9-3 = -12
	\end{align}
	Since inner product is not zero so adjacent sides are not orthogonal.

	Hence, we can say that $ABCD$ is neither a rectangle nor a square.

		\item Now checking if the diagonals are orthogonal then it is a Rhombus.
	\begin{align}
		(\vec{C}- \vec{A})^\top (\vec{D}-\vec{B}) = \myvec{0&-2} \myvec{-6\\-4} = 0+8 = 8
	\end{align}
	\end{enumerate}		
	Hence the diagonals are also not orthogonal so we conclude that $ABCD$ is a parallelogram.

	As shown in Figure \ref{fig:10/7/1/6/Fig3} we can see that $ABCD$ forms a parallelogram hence, our theoritical result is verified.

\begin{figure}[!h]
	\begin{center} 
	    \includegraphics[width=\columnwidth]{chapters/10/7/1/6/figs/quad3}
	\end{center}
\caption{}
\label{fig:10/7/1/6/Fig3}
\end{figure}
\end{enumerate}



\item Find the direction cosines of the vector joining the points $\vec{A}$ (1, 2, –3) and
$\vec{B}$(–1, –2, 1), directed from $\vec{A}$ to $\vec{B}$.
	\\
    \solution 
		\iffalse
\documentclass[12pt]{article}
\usepackage{graphicx}
%\documentclass[journal,12pt,twocolumn]{IEEEtran}
\usepackage[none]{hyphenat}
\usepackage{graphicx}
\usepackage{listings}
\usepackage[english]{babel}
\usepackage{graphicx}
\usepackage{caption}
\usepackage[parfill]{parskip}
\usepackage{hyperref}
\usepackage{booktabs}
%\usepackage{setspace}\doublespacing\pagestyle{plain}
\def\inputGnumericTable{}
\usepackage{color}                                            %%
    \usepackage{array}                                            %%
    \usepackage{longtable}                                        %%
    \usepackage{calc}                                             %%
    \usepackage{multirow}                                         %%
    \usepackage{hhline}                                           %%
    \usepackage{ifthen}
\usepackage{array}
\usepackage{amsmath}   % for having text in math mode
\usepackage{parallel,enumitem}
\usepackage{listings}
\lstset{
language=tex,
frame=single,
breaklines=true
}
 
%Following 2 lines were added to remove the blank page at the beginning
\usepackage{atbegshi}% http://ctan.org/pkg/atbegshi
\AtBeginDocument{\AtBeginShipoutNext{\AtBeginShipoutDiscard}}
%
%New macro definitions
\newcommand{\mydet}[1]{\ensuremath{\begin{vmatrix}#1\end{vmatrix}}}
\providecommand{\brak}[1]{\ensuremath{\left(#1\right)}}
\providecommand{\norm}[1]{\left\lVert#1\right\rVert}
\newcommand{\solution}{\noindent \textbf{Solution: }}
\newcommand{\myvec}[1]{\ensuremath{\begin{pmatrix}#1\end{pmatrix}}}
\let\vec\mathbf
\begin{document}
\begin{center}
\enlargethispage{-4cm}
\title{\textbf{Vector Algebra}}
\date{\vspace{-5ex}} %Not to print date automatically
\maketitle
\end{center}
\setcounter{page}{1}
\section*{12$^{th}$ Maths - Chapter 10}
 Exercise 10.2 Problem-13
\begin{enumerate}
\item Find the direction cosines of the vector joining the points A (1, 2, –3) and
B(–1, –2, 1), directed from A to B.

\solution 
\fi
The direction vector  is given by 
\begin{align}
	\vec{m}= \myvec{-1\\-2\\1}-\myvec{1\\2\\-3}=\myvec{-2\\-4\\4}
\end{align}
and the corresponding unit vector is
\begin{align}
	\frac{\vec{m}}{\norm{\vec{m}}}=\frac{1}{6}{\myvec{-2\\-4\\4}}
\end{align}

\item Show that the vector $\hat{i}+\hat{j}+\hat{k}$ is equally inclined to the axes OX, OY and OZ.
	\\
\solution
		\iffalse
\documentclass[12pt]{article}
\usepackage{graphicx}
\usepackage{amsmath}
\usepackage{mathtools}
\usepackage{gensymb}

\newcommand{\mydet}[1]{\ensuremath{\begin{vmatrix}#1\end{vmatrix}}}
\providecommand{\brak}[1]{\ensuremath{\left(#1\right)}}
\providecommand{\norm}[1]{\left\lVert#1\right\rVert}
\newcommand{\solution}{\noindent \textbf{Solution: }}
\newcommand{\myvec}[1]{\ensuremath{\begin{pmatrix}#1\end{pmatrix}}}
\let\vec\mathbf

\begin{document}
\begin{center}
\textbf\large{CHAPTER-7 \\ COORDINATE GEOMETRY}

\end{center}
\section*{Excercise 7.1}

Q6.Name the type of quadilateral formed,if any, by the following points, and give reasons for your answer:
\begin{enumerate}
	\item $\brak{-1,-2}, \brak{1,0}, \brak{-1,2}, \brak{-3,0}$ 
	\item $\brak{-3,5}, \brak{3,1}, \brak{0,3}, \brak{-1,-4}$
	\item $\brak{4,5}, \brak{7,6}, \brak{4,3}, \brak{1,2}$
\end{enumerate}
\solution
\fi
\begin{enumerate}
\item The coordinates are given as
	\begin{align}
	\vec{A} = \myvec{
		-1\\
		-2\\
		},
	\vec{B} = \myvec{
		1\\
		0\\
		},
	\vec{C} = \myvec{
		-1\\
		2\\
		} \text{ and }
	\vec{D} = \myvec{
		-3\\
		0\\
		}
	\end{align}
	\begin{align}
		\vec{B} - \vec{A} &= \myvec{1\\0} - \myvec{-1\\-2} = \myvec{2\\2}\\
		\vec{C} - \vec{B} &= \myvec{-1\\2} - \myvec{1\\0} = \myvec{-2\\2}\\
		\vec{C} - \vec{D} &= \myvec{-1\\2} - \myvec{-3\\0} = \myvec{2\\2}\\
		\vec{D} - \vec{A} &= \myvec{-3\\0} - \myvec{-1\\-2} = \myvec{-2\\2}
	\end{align}
	\begin{align}	
		\vec{C} - \vec{A} &= \myvec{-1\\2} - \myvec{-1\\-2} = \myvec{0\\4}\\
		\vec{D} - \vec{B} &= \myvec{-3\\0} - \myvec{1\\0} = \myvec{-4\\0}
	\end{align}
	\begin{align}	
		\vec{B}-\vec{A} = \vec{C}-\vec{D} \text{ and } \vec{C}-\vec{B} = \vec{D}-\vec{A}.
	\end{align}
	Hence, $ABCD$ is a parallelogram.
	\begin{enumerate}
		\item Now checking if the adjacent sides are orthogonal to each other
	\begin{align}
		(\vec{B}-\vec{A})^\top (\vec{C}-\vec{B}) = \myvec{2&2} \myvec{-2\\2} = -4+4 = 0
	\end{align}
		\item Now checking if the diagonals are also orthogonal then it is a square else a rectangle.
	\end{enumerate}	
	\begin{align}
		(\vec{C}-\vec{A})^\top (\vec{D}-\vec{B}) = \myvec{0&4} \myvec{-4\\0} = 0
	\end{align}
	Hence the diagonals are orthogonal to each other.

	So, we can conclude that $ABCD$ is a square.

	As shown in Figure \ref{fig:10/7/1/6/Fig1} we can see that $ABCD$ is a square hence we can conclude that our theoritical result is verified.
 
\begin{figure}[!h]
	\begin{center} 
	    \includegraphics[width=\columnwidth]{chapters/10/7/1/6/figs/quad1}
	\end{center}
\caption{}
\label{fig:10/7/1/6/Fig1}
\end{figure}

\item The coordinates are given as
	\begin{align}
	\vec{A} = \myvec{
		-3\\
		5\\
		},
	\vec{B} = \myvec{
		3\\
		1\\
		},
	\vec{C} = \myvec{
		0\\
		3\\
		} \text{ and }
	\vec{D} = \myvec{
		-1\\
		-4\\
		}
	\end{align}
	\begin{align}
		\vec{B} - \vec{A} &= \myvec{3\\1} - \myvec{-3\\5} = \myvec{6\\-4}\\
		\vec{C} - \vec{B} &= \myvec{0\\3} - \myvec{3\\1} = \myvec{-3\\2}\\
		\vec{C} - \vec{D} &= \myvec{0\\3} - \myvec{-1\\-4} = \myvec{1\\7}\\
		\vec{D} - \vec{A} &= \myvec{-1\\-4} - \myvec{-3\\5} = \myvec{2\\-9}
	\end{align}
	\begin{align}
		\vec{C} - \vec{A} &= \myvec{0\\3} - \myvec{-3\\5} = \myvec{3\\-2}\\
		\vec{D} - \vec{B} &= \myvec{-1\\-4} - \myvec{3\\1} = \myvec{-4\\-5}
	\end{align}
	\begin{align}
	\vec{B}-\vec{A} \neq \vec{C}-\vec{D} \text{ and } \vec{C}-\vec{B} \neq \vec{D}-\vec{A},
	\end{align}
	Hence, $ABCD$ is not a parallelogram, it can be a irregular quadilateral.
	\begin{enumerate}
		\item Now to check if any three points are collinear,

	if rank of $\myvec{\vec{B}-\vec{A} & \vec{C}-\vec{B}} = 1$ then points are collinear

	Forming the collinearity matrix
	\begin{align}
		\myvec{6&-3\\-4&2} \xleftrightarrow{R_{2}\rightarrow R_{2}+\frac{2}{3}R_{1}}= \myvec{6&-3\\0&0}
	\end{align}
	\end{enumerate}
	Hence, rank = 1

	Since none of the opposite sides are parallel to each other and three points are collinear so these does not form a quadilateral.

	As shown in Figure \ref{fig:10/7/1/6/Fig2} we can see that $ABCD$ does not form a quadilateral and three points are collinear hence, our theoritical result is verified.
	
\begin{figure}[!h]
	\begin{center} 
	    \includegraphics[width=\columnwidth]{chapters/10/7/1/6/figs/quad2}
	\end{center}
\caption{}
\label{fig:10/7/1/6/Fig2}
\end{figure}
	
\item The coordinates are given as
	\begin{align}
	\vec{A} = \myvec{
		4\\
		5\\
		},
	\vec{B} = \myvec{
		7\\
		6\\
		},
	\vec{C} = \myvec{
		4\\
		3\\
		} \text{ and }
	\vec{D} = \myvec{
		1\\
		2\\
		}
	\end{align}
	\begin{align}
		\vec{B} - \vec{A} &= \myvec{7\\6} - \myvec{4\\5} = \myvec{3\\1}\\
		\vec{C} - \vec{B} &= \myvec{4\\3} - \myvec{7\\6} = \myvec{-3\\-3}\\
		\vec{C} - \vec{D} &= \myvec{4\\3} - \myvec{1\\2} = \myvec{3\\1}\\
		\vec{D} - \vec{A} &= \myvec{1\\2} - \myvec{4\\5} = \myvec{-3\\-3}
	\end{align}
	\begin{align}
		\vec{C} - \vec{A} &= \myvec{4\\3} - \myvec{4\\5} = \myvec{0\\-2}\\
		\vec{D} - \vec{B} &= \myvec{1\\2} - \myvec{7\\6} = \myvec{-6\\-4}
	\end{align}
	\begin{align}
		\vec{B}-\vec{A} = \vec{C}-\vec{D} \text{ and } \vec{C}-\vec{B} = \vec{D}-\vec{A},
	\end{align}
	Hence, $ABCD$ is a parallelogram.
	\begin{enumerate}
		\item Now checking if the adjacent sides are orthogonal to each other
	\begin{align}
		(\vec{B}-\vec{A})^\top (\vec{C}-\vec{B}) = \myvec{3&1} \myvec{-3\\-3} = -9-3 = -12
	\end{align}
	Since inner product is not zero so adjacent sides are not orthogonal.

	Hence, we can say that $ABCD$ is neither a rectangle nor a square.

		\item Now checking if the diagonals are orthogonal then it is a Rhombus.
	\begin{align}
		(\vec{C}- \vec{A})^\top (\vec{D}-\vec{B}) = \myvec{0&-2} \myvec{-6\\-4} = 0+8 = 8
	\end{align}
	\end{enumerate}		
	Hence the diagonals are also not orthogonal so we conclude that $ABCD$ is a parallelogram.

	As shown in Figure \ref{fig:10/7/1/6/Fig3} we can see that $ABCD$ forms a parallelogram hence, our theoritical result is verified.

\begin{figure}[!h]
	\begin{center} 
	    \includegraphics[width=\columnwidth]{chapters/10/7/1/6/figs/quad3}
	\end{center}
\caption{}
\label{fig:10/7/1/6/Fig3}
\end{figure}
\end{enumerate}



\item If a line has the direction ratios –18, 12, –4, then what are its direction cosines?
		\\
		\solution
		\iffalse
\documentclass[12pt]{article}
\usepackage{graphicx}
%\documentclass[journal,12pt,twocolumn]{IEEEtran}
\usepackage[none]{hyphenat}
\usepackage{graphicx}
\usepackage{listings}
\usepackage[english]{babel}
\usepackage{graphicx}
\usepackage{caption}
\usepackage[parfill]{parskip}
\usepackage{hyperref}
\usepackage{booktabs}
%\usepackage{setspace}\doublespacing\pagestyle{plain}
\def\inputGnumericTable{}
\usepackage{color}                                            %%
    \usepackage{array}                                            %%
    \usepackage{longtable}                                        %%
    \usepackage{calc}                                             %%
    \usepackage{multirow}                                         %%
    \usepackage{hhline}                                           %%
    \usepackage{ifthen}
\usepackage{array}
\usepackage{amsmath}   % for having text in math mode
\usepackage{parallel,enumitem}
\usepackage{listings}
\lstset{
language=tex,
frame=single,
breaklines=true
}
 
%Following 2 lines were added to remove the blank page at the beginning
\usepackage{atbegshi}% http://ctan.org/pkg/atbegshi
\AtBeginDocument{\AtBeginShipoutNext{\AtBeginShipoutDiscard}}
%
%New macro definitions
\newcommand{\mydet}[1]{\ensuremath{\begin{vmatrix}#1\end{vmatrix}}}
\providecommand{\brak}[1]{\ensuremath{\left(#1\right)}}
\providecommand{\norm}[1]{\left\lVert#1\right\rVert}
\newcommand{\solution}{\noindent \textbf{Solution: }}
\newcommand{\myvec}[1]{\ensuremath{\begin{pmatrix}#1\end{pmatrix}}}
\let\vec\mathbf
\begin{document}
\begin{center}
\enlargethispage{-4cm}
\title{\textbf{Three Dimensional Geometry}}
\date{\vspace{-5ex}} %Not to print date automatically
\maketitle
\end{center}
\setcounter{page}{1}
\section*{12$^{th}$ Maths - Chapter 11}
This is Problem-3 from Exercise 11.1
\begin{enumerate}

\solution let $\vec{A}$ be the given vector
\fi
Let
\begin{align}
	\vec{A} =\myvec{-18\\12\\-4}
\end{align}
Then the unit direction vector is
\begin{align}
		\vec{B}=\frac{\vec{A}}{\norm{\vec{A}}} =
\myvec{\frac{-9}{11}\\[2pt] \frac{6}{11}\\[2pt] \frac{-2}{11}}
\end{align}

	\item Find the direction cosines of the sides of a triangle whose vertices are $\myvec{3\\ 5\\-4 }$, $\myvec{ -1\\1 \\2 }$ and $\myvec{-5 \\-5 \\-2 }$.
		\\
		\solution
		\iffalse
\documentclass[journal,12pt,twocolumn]{IEEEtran}
%
\usepackage{setspace}
\usepackage{gensymb}
%\doublespacing
\singlespacing

%\usepackage{graphicx}
%\usepackage{amssymb}
%\usepackage{relsize}
\usepackage[cmex10]{amsmath}
%\usepackage{amsthm}
%\interdisplaylinepenalty=2500
%\savesymbol{iint}
%\usepackage{txfonts}
%\restoresymbol{TXF}{iint}
%\usepackage{wasysym}
\usepackage{amsthm}
%\usepackage{iithtlc}
\usepackage{mathrsfs}
\usepackage{txfonts}
\usepackage{stfloats}
\usepackage{bm}
\usepackage{cite}
\usepackage{cases}
\usepackage{subfig}
%\usepackage{xtab}
\usepackage{longtable}
\usepackage{multirow}
%\usepackage{algorithm}
%\usepackage{algpseudocode}
\usepackage{enumitem}
\usepackage{mathtools}
\usepackage{steinmetz}
\usepackage{tikz}
\usepackage{circuitikz}
\usepackage{verbatim}
\usepackage{tfrupee}
\usepackage[breaklinks=true]{hyperref}
%\usepackage{stmaryrd}
\usepackage{tkz-euclide} % loads  TikZ and tkz-base
%\usetkzobj{all}
\usetikzlibrary{calc,math}
\usepackage{listings}
    \usepackage{color}                                            %%
    \usepackage{array}                                            %%
    \usepackage{longtable}                                        %%
    \usepackage{calc}                                             %%
    \usepackage{multirow}                                         %%
    \usepackage{hhline}                                           %%
    \usepackage{ifthen}                                           %%
  %optionally (for landscape tables embedded in another document): %%
    \usepackage{lscape}     
\usepackage{multicol}
\usepackage{chngcntr}
%\usepackage{enumerate}

%\usepackage{wasysym}
%\newcounter{MYtempeqncnt}
\DeclareMathOperator*{\Res}{Res}
%\renewcommand{\baselinestretch}{2}
\renewcommand\thesection{\arabic{section}}
\renewcommand\thesubsection{\thesection.\arabic{subsection}}
\renewcommand\thesubsubsection{\thesubsection.\arabic{subsubsection}}

\renewcommand\thesectiondis{\arabic{section}}
\renewcommand\thesubsectiondis{\thesectiondis.\arabic{subsection}}
\renewcommand\thesubsubsectiondis{\thesubsectiondis.\arabic{subsubsection}}

% correct bad hyphenation here
\hyphenation{op-tical net-works semi-conduc-tor}
\def\inputGnumericTable{}                                 %%

\lstset{
%language=C,
frame=single, 
breaklines=true,
columns=fullflexible
}
%\lstset{
%language=tex,
%frame=single, 
%breaklines=true
%}

\begin{document}
%


\newtheorem{theorem}{Theorem}[section]
\newtheorem{problem}{Problem}
\newtheorem{proposition}{Proposition}[section]
\newtheorem{lemma}{Lemma}[section]
\newtheorem{corollary}[theorem]{Corollary}
\newtheorem{example}{Example}[section]
\newtheorem{definition}[problem]{Definition}
%\newtheorem{thm}{Theorem}[section] 
%\newtheorem{defn}[thm]{Definition}
%\newtheorem{algorithm}{Algorithm}[section]
%\newtheorem{cor}{Corollary}
\newcommand{\BEQA}{\begin{eqnarray}}
\newcommand{\EEQA}{\end{eqnarray}}
\newcommand{\define}{\stackrel{\triangle}{=}}

\bibliographystyle{IEEEtran}
%\bibliographystyle{ieeetr}


\providecommand{\mbf}{\mathbf}
\providecommand{\pr}[1]{\ensuremath{\Pr\left(#1\right)}}
\providecommand{\qfunc}[1]{\ensuremath{Q\left(#1\right)}}
\providecommand{\sbrak}[1]{\ensuremath{{}\left[#1\right]}}
\providecommand{\lsbrak}[1]{\ensuremath{{}\left[#1\right.}}
\providecommand{\rsbrak}[1]{\ensuremath{{}\left.#1\right]}}
\providecommand{\brak}[1]{\ensuremath{\left(#1\right)}}
\providecommand{\lbrak}[1]{\ensuremath{\left(#1\right.}}
\providecommand{\rbrak}[1]{\ensuremath{\left.#1\right)}}
\providecommand{\cbrak}[1]{\ensuremath{\left\{#1\right\}}}
\providecommand{\lcbrak}[1]{\ensuremath{\left\{#1\right.}}
\providecommand{\rcbrak}[1]{\ensuremath{\left.#1\right\}}}
\theoremstyle{remark}
\newtheorem{rem}{Remark}
\newcommand{\sgn}{\mathop{\mathrm{sgn}}}
\providecommand{\abs}[1]{\left\vert#1\right\vert}
\providecommand{\res}[1]{\Res\displaylimits_{#1}} 
\providecommand{\norm}[1]{\left\lVert#1\right\rVert}
%\providecommand{\norm}[1]{\lVert#1\rVert}
\providecommand{\mtx}[1]{\mathbf{#1}}
\providecommand{\mean}[1]{E\left[ #1 \right]}
\providecommand{\fourier}{\overset{\mathcal{F}}{ \rightleftharpoons}}
%\providecommand{\hilbert}{\overset{\mathcal{H}}{ \rightleftharpoons}}
\providecommand{\system}{\overset{\mathcal{H}}{ \longleftrightarrow}}
	%\newcommand{\solution}[2]{\textbf{Solution:}{#1}}
\newcommand{\solution}{\noindent \textbf{Solution: }}
\newcommand{\cosec}{\,\text{cosec}\,}
\providecommand{\dec}[2]{\ensuremath{\overset{#1}{\underset{#2}{\gtrless}}}}
\newcommand{\myvec}[1]{\ensuremath{\begin{pmatrix}#1\end{pmatrix}}}
\newcommand{\mydet}[1]{\ensuremath{\begin{vmatrix}#1\end{vmatrix}}}
%\numberwithin{equation}{section}
\numberwithin{equation}{subsection}
%\numberwithin{problem}{section}
%\numberwithin{definition}{section}
\makeatletter
\@addtoreset{figure}{problem}
\makeatother

\let\StandardTheFigure\thefigure
\let\vec\mathbf
%\renewcommand{\thefigure}{\theproblem.\arabic{figure}}
\renewcommand{\thefigure}{\theproblem}
%\setlist[enumerate,1]{before=\renewcommand\theequation{\theenumi.\arabic{equation}}
%\counterwithin{equation}{enumi}


%\renewcommand{\theequation}{\arabic{subsection}.\arabic{equation}}

\def\putbox#1#2#3{\makebox[0in][l]{\makebox[#1][l]{}\raisebox{\baselineskip}[0in][0in]{\raisebox{#2}[0in][0in]{#3}}}}
     \def\rightbox#1{\makebox[0in][r]{#1}}
     \def\centbox#1{\makebox[0in]{#1}}
     \def\topbox#1{\raisebox{-\baselineskip}[0in][0in]{#1}}
     \def\midbox#1{\raisebox{-0.5\baselineskip}[0in][0in]{#1}}

\vspace{3cm}


\title{Question: 12.11.1.5}
\author{Nikam Pratik Balasaheb (EE21BTECH11037)}





% make the title area
\maketitle

\newpage

%\tableofcontents

\bigskip

\renewcommand{\thefigure}{\theenumi}
\renewcommand{\thetable}{\theenumi}
%\renewcommand{\theequation}{\theenumi}

\section{Problem}
\section{Solution}
\fi
Let the vertices be
\begin{align}
\vec{A} = \myvec{3\\5\\-4},
\vec{B} = \myvec{-1\\1\\2},
\vec{C} = \myvec{-5\\-5\\-2}
\end{align}
%
The direction vectors of the sides are,
\begin{align}
\vec{A} - \vec{B} = \myvec{4\\4\\-6} = \vec{m_1},
\vec{B} - \vec{C} = \myvec{4\\6\\4} = \vec{m_2}, 
\vec{C} - \vec{A} = \myvec{-8\\-10\\2} =\vec{m_3},
\end{align}
%
The direction vectors of the axes are,
\begin{align}
\vec{e}_1 = \myvec{1\\0\\0},
\vec{e}_2 = \myvec{0\\1\\0},
\vec{e}_3 = \myvec{0\\0\\1}
\end{align}
Direction cosines of a vector $\vec{m}$ are given by
\begin{align}
\myvec{\cos{\theta_1}\\ \cos{\theta_2}\\\cos{\theta_3}} = \myvec{ \frac{\vec{m}^{\top} {\vec{e}_1}}{\norm{\vec{m}}\norm{\vec{e}_1} }  \\[1pt] \frac{\vec{m}^{\top} {\vec{e}_1}}{\norm{\vec{m}}\norm{\vec{e}_1} } \\[1pt] \frac{\vec{m}^{\top} {\vec{e}_1}}{\norm{\vec{m}}\norm{\vec{e}_1} } }
= \frac{1}{\norm{\vec{m}}}\myvec{ \vec{m}^{\top} {\vec{e}_1} \\ \vec{m}^{\top} {\vec{e}_2} \\ \vec{m}^{\top} {\vec{e}_3} }
= \frac{\vec{m}}{\norm{\vec{m}}}
\end{align}
Direction cosines of side $\vec{m_1}$ are
\begin{align}
\myvec{\cos{\theta_1}\\ \cos{\theta_2}\\\cos{\theta_3}} = \frac{\vec{m_1}}{\norm{\vec{m_1}}}
= \myvec{ \frac{2}{\sqrt{17}} \\[1pt] \frac{2}{\sqrt{17}} \\[1pt] \frac{-3}{\sqrt{17}} } 
\end{align}
Direction cosines of side $\vec{m_2}$ are,
\begin{align}
\myvec{\cos{\theta_1}\\ \cos{\theta_2}\\\cos{\theta_3}} = \frac{\vec{m_2}}{\norm{\vec{m_2}}}
= \myvec{ \frac{2}{\sqrt{17}} \\[1pt] \frac{3}{\sqrt{17}} \\[1pt] \frac{2}{\sqrt{17}} } 
\end{align}
Direction cosines of side $\vec{m_3}$ are,
\begin{align}
\myvec{\cos{\theta_1}\\ \cos{\theta_2}\\\cos{\theta_3}} = \frac{\vec{m_3}}{\norm{\vec{m_3}}}
= \myvec{ \frac{-4}{\sqrt{42}} \\[1pt] \frac{-5}{\sqrt{42}} \\[1pt] \frac{1}{\sqrt{42}} } 
\end{align}
\iffalse

\begin{figure}[h!]
  \centering
   \includegraphics[width=\linewidth]{figs/Figure_1.png}
   \caption{Triangle ABC}
   \label{fig:TriangleABC}
\end{figure}

\end{document}


\fi

\item If $\vec{a}=\vec{b}+\vec{c}$, then is it true that $|\vec{a}|=|\vec{b}|+|\vec{c}|$? Justify your answer.\\
	\solution
		\iffalse
\documentclass[12pt]{article}
\usepackage{graphicx}
%\documentclass[journal,12pt,twocolumn]{IEEEtran}
\usepackage[none]{hyphenat}
\usepackage{graphicx}
\usepackage{listings}
\usepackage[english]{babel}
\usepackage{graphicx}
\usepackage{caption} 
\usepackage{hyperref}
\usepackage{booktabs}
\usepackage{array}
\usepackage{amsmath}   % for having text in math mode
\usepackage{listings}
\lstset{
  frame=single,
  breaklines=true
}
  
%Following 2 lines were added to remove the blank page at the beginning
\usepackage{atbegshi}% http://ctan.org/pkg/atbegshi
\AtBeginDocument{\AtBeginShipoutNext{\AtBeginShipoutDiscard}}
%


%New macro definitions
\newcommand{\mydet}[1]{\ensuremath{\begin{vmatrix}#1\end{vmatrix}}}
\providecommand{\brak}[1]{\ensuremath{\left(#1\right)}}
\providecommand{\norm}[1]{\left\lVert#1\right\rVert}
\newcommand{\solution}{\noindent \textbf{Solution: }}
\newcommand{\myvec}[1]{\ensuremath{\begin{pmatrix}#1\end{pmatrix}}}
\let\vec\mathbf

\begin{document}

\begin{center}
\title{\textbf{Vector Dot Product}}
\date{\vspace{-5ex}} %Not to print date automatically
\maketitle
\end{center}
\setcounter{page}{1}

\section{12$^{th}$ Maths - Chapter 10}
This is Problem-9 from Exercise 10.3
\begin{enumerate}
\item Find $\norm{\vec{x}}$, if for a unit vector $\vec{a}$, $\brak{\vec{x}-\vec{a}}.\brak{\vec{x}+\vec{a}} = 12$.\\
	\fi
\solution 
From the given information,
\begin{align}
  \label{eq:12/10/3/9det2f}
  \brak{\vec{x}-\vec{a}}^\top\brak{\vec{x}+\vec{a}} &= 12 \\
  \implies \vec{x}^\top\vec{x} - \vec{a}^\top\vec{x} + \vec{x}^\top\vec{a} - \vec{a}^\top\vec{a} &= 12 \\
  \implies \norm{\vec{x}}^{2} - \norm{\vec{a}}^{2} &= 12 \\
\implies   \norm{\vec{x}}^{2} - 1 &= 12  \\
	\text{or, }  
	\norm{\vec{x}} &= \sqrt{13}
\end{align}

\item Find the value of x for which $x(\hat{i}+\hat{j}+\hat{k})$ is a unit vector.\\
	\solution
		\iffalse
\documentclass[12pt]{article}
\usepackage{graphicx}
%\documentclass[journal,12pt,twocolumn]{IEEEtran}
\usepackage[none]{hyphenat}
\usepackage{graphicx}
\usepackage{listings}
\usepackage[english]{babel}
\usepackage{graphicx}
\usepackage{caption} 
\usepackage{hyperref}
\usepackage{booktabs}
\usepackage{array}
\usepackage{amsmath}   % for having text in math mode
\usepackage{listings}
\lstset{
  frame=single,
  breaklines=true
}
  
%Following 2 lines were added to remove the blank page at the beginning
\usepackage{atbegshi}% http://ctan.org/pkg/atbegshi
\AtBeginDocument{\AtBeginShipoutNext{\AtBeginShipoutDiscard}}
%


%New macro definitions
\newcommand{\mydet}[1]{\ensuremath{\begin{vmatrix}#1\end{vmatrix}}}
\providecommand{\brak}[1]{\ensuremath{\left(#1\right)}}
\providecommand{\norm}[1]{\left\lVert#1\right\rVert}
\newcommand{\solution}{\noindent \textbf{Solution: }}
\newcommand{\myvec}[1]{\ensuremath{\begin{pmatrix}#1\end{pmatrix}}}
\let\vec\mathbf

\begin{document}

\begin{center}
\title{\textbf{Vector Dot Product}}
\date{\vspace{-5ex}} %Not to print date automatically
\maketitle
\end{center}
\setcounter{page}{1}

\section{12$^{th}$ Maths - Chapter 10}
This is Problem-9 from Exercise 10.3
\begin{enumerate}
\item Find $\norm{\vec{x}}$, if for a unit vector $\vec{a}$, $\brak{\vec{x}-\vec{a}}.\brak{\vec{x}+\vec{a}} = 12$.\\
	\fi
\solution 
From the given information,
\begin{align}
  \label{eq:12/10/3/9det2f}
  \brak{\vec{x}-\vec{a}}^\top\brak{\vec{x}+\vec{a}} &= 12 \\
  \implies \vec{x}^\top\vec{x} - \vec{a}^\top\vec{x} + \vec{x}^\top\vec{a} - \vec{a}^\top\vec{a} &= 12 \\
  \implies \norm{\vec{x}}^{2} - \norm{\vec{a}}^{2} &= 12 \\
\implies   \norm{\vec{x}}^{2} - 1 &= 12  \\
	\text{or, }  
	\norm{\vec{x}} &= \sqrt{13}
\end{align}

	\item 
		Find a vector of magnitude 5 units, and parallel to the resultant of the vectors $\vec{a} = 2\hat{i}+3\hat{j}-\hat{k}$ and $\vec{b} = \hat{i}-2\hat{j}+\hat{k}$.

\textbf{Solution :}
\begin{align}
    \Vec{a}&=\myvec{
        2\\3\\-1
    },\Vec{b}=\myvec{
        1\\-2\\1
    }\\
\vec{a+b=c}&=\myvec{
        3\\1\\0
    }\\
    \norm{\vec{a+b}}=\norm{\vec{c}}&=\sqrt{10}\\
\hat{c}&=\frac{\vec{c}}{\norm{\vec{c}}} \\
    \implies\hat{c}&=\frac{1}{\sqrt{10}}\myvec{
        3\\1\\0
    }
\end{align}
So,the unit vector is
\begin{align}
    \pm5\hat{c}&=\pm\frac{\sqrt{10}}{2}\myvec{
        3\\1\\0
}
\end{align}



\end{enumerate}

\subsection{Distance}
\iffalse
\documentclass[12pt]{article}
\usepackage{graphicx}
\usepackage{commath}
\usepackage{gensymb}
\usepackage{float}

\begin{document}
\begin{center}
\textbf\large{CHAPTER-7 \\ COORDINATE GEOMETRY}
\end{center}

\section{EXERCISE - 7.1}
\fi
\begin{enumerate}[label=\thesection.\arabic*,ref=\thesection.\theenumi]
\numberwithin{equation}{enumi}
\numberwithin{figure}{enumi}
\numberwithin{table}{enumi}
\item Find the distances between the following pairs of points:
\begin{enumerate}
\item $(2,3),(4,1)$
\item $(-5,7),(-1,3)$
\item $(a,b),(-a,-b)$
\end{enumerate}
		\renewcommand{\theequation}{\theenumi}
\begin{enumerate}[label=\thesubsection.\arabic*.,ref=\thesubsection.\theenumi]
%\begin{enumerate}[label=\arabic*.,ref=\thesection.\theenumi]
\numberwithin{equation}{enumi}

\item Find the distance between the points 
\begin{align}
\myvec{0\\0}, \myvec{36\\15}
\end{align}
%
\solution The desired distance is 
\begin{align}
\norm{\vec{A} - \vec{B}} 
=\norm{\myvec{0\\0} - \myvec{36\\15}} = \sqrt{36^2+15^2} = 39
\end{align}

\item Find the distance between the following pairs of points
\begin{enumerate}
\item 
\begin{align}
\myvec{2\\3}, \myvec{4\\1}
\end{align}
\item 
\begin{align}
\myvec{-5\\7}, \myvec{-1\\3}
\end{align}
\item 
\begin{align}
\myvec{a\\b}, \myvec{-1\\b}
\end{align}
\end{enumerate}
\solution The distance between two vectors is given by 
\begin{align}
\norm{\vec{A}-\vec{B}}
\label{eq:vectors_3.5.1_norm}
\end{align}
From \eqref{eq:vectors_3.5.1_norm},
\begin{enumerate}
\item 
$\norm{\myvec{2\\3}-\myvec{4\\1}}=\sqrt{\brak{2-4}^2+\brak{3-1}^2 = 2\sqrt{2}}$
\item $\norm{\myvec{-5\\7}-\myvec{-1\\3}}=\sqrt{\brak{-5+1}^2+\brak{7-3}^2 = 4\sqrt{2}}$
\item $\norm{\myvec{a\\b}-\myvec{-1\\b}}=a+1$

\end{enumerate}
\item Name the type of quadrilateral formed, if any, by the following points, and give reasons for your answer.
\begin{enumerate}
\item 
\begin{align}
\vec{P} = \myvec{-1\\-2}, \vec{Q} =\myvec{1\\0},
\vec{R} =\myvec{-1\\2}, \vec{S} =\myvec{-3\\0}
\end{align}
\item 
\begin{align}
\vec{P} = \myvec{-3\\5}, \vec{Q} =\myvec{3\\1},
\vec{R} =\myvec{0\\3}, \vec{S} =\myvec{-1\\-4}
\end{align}
\item 
\begin{align}
\vec{P} = \myvec{4\\5}, \vec{Q} =\myvec{7\\6},
\\
\vec{R} =\myvec{4\\3}, \vec{S} =\myvec{1\\2}
\end{align}
\end{enumerate}
\solution
\begin{enumerate}
\item In Fig. 	\ref{fig:3.5.4_quadrilateral1}
\begin{align}
\label{eq:constr_vectors_quad1_PSQR}
 \vec{P} - \vec{S} &= 
 \vec{Q} - \vec{R} = \myvec{2\\-2}
\\
\vec {R} - \vec {S} &=
 \vec {Q} - \vec {P} = \myvec{2\\2}
\label{eq:constr_vectors_quad1_RSQP}
\end{align}
%
Hence $PQRS$ is a $\parallel$gm $\because$  opposite sides are parallel. Also, 
\begin{align}
\norm{ \vec{P} - \vec{S}} &= 
\norm{ \vec{Q} - \vec{R}} 
\\
=\norm{\vec {R} - \vec {S}} &=
\norm{ \vec {Q} - \vec {P}} = 2\sqrt{2}
\end{align}
%
$\because$ all sides are equal, the $\parallel$gm is a rhombus. The angle between $PS$ and
$RS$ is given by 
\begin{align}
\cos \theta = \frac{\brak{\vec{S}-\vec{P}}^{\top}\brak{\vec{S}-\vec{R}}}{\norm{\vec{S}-\vec{P}}^{\top}\norm{\vec{S}-\vec{R}}}
\end{align}
%
$\because $
\begin{align}
\brak{\vec{S}-\vec{P}}^{\top}\brak{\vec{S}-\vec{R}} = \myvec{2 & -2}\myvec{2\\2} = 0
\end{align}
upon substituting from \eqref{eq:constr_vectors_quad1_PSQR} and \eqref{eq:constr_vectors_quad1_RSQP}, 
\begin{align}
\cos \theta = 0 \implies PS \perp RS
\end{align}
%
Thus, the rhombus is actually a square.

\begin{figure}[!ht]
	\centering
	\includegraphics[width=\columnwidth]{./figs/vectors/quad1.png}
	\caption{quadrilateral1 }
	\label{fig:3.5.4_quadrilateral1}
\end{figure}
\begin{lstlisting}
codes/vectors/quad1.py
\end{lstlisting}

\item In Fig. 	\ref{fig:3.5.4_quadrilateral2}

\begin{align}
\vec{Q} - \vec{P} &= \myvec{6\\-4}
\\
\vec{R} - \vec{P} &= \myvec{3\\-2}
\\
\vec{Q} - \vec{R} &= \myvec{3\\-2}
\\
\left(\vec{Q} - \vec{P}\right) &= \left(\vec{R} - \vec{P}\right) + \left(\vec{Q} - \vec{R}\right) = \myvec{6\\-4}
\end{align}
Hence,  $\vec P,\vec Q$ and $\vec R$ lie on a straight line, so $PQRS$ is not  a quadrilateral.
%
\begin{figure}[!ht]
	\centering
	\includegraphics[width=\columnwidth]{./figs/vectors/quad2.pdf}
	\caption{quadrilateral2 }
	\label{fig:3.5.4_quadrilateral2}
\end{figure}
%
\item See Fig. 	\ref{fig:3.5.4_quadrilateral3}.

\begin{align}
\because \left(\vec{Q} - \vec{P}\right) &= \left(\vec{R} - \vec{S}\right) = \myvec{3\\1}
\\
\left(\vec{P} - \vec{S}\right) &= \left(\vec{Q} - \vec{R}\right) = \myvec{3\\3},
\end{align}
$PQRS$ is a parallelogram.  Also, 
%
\begin{align}
\norm{\vec{Q} - \vec{P}} \ne \norm{\vec{P} - \vec{S}}
\end{align}
Hence, $PQRS$ is neither a rhombus nor a square.
\begin{align}
\because \left(\vec{Q} - \vec{P}\right)^T \left(\vec{Q} - \vec{R}\right) = \myvec{3 & 1}\myvec{3\\3} \ne 0,
\end{align}
$PQRS$ is not a rectangle. 
%
\begin{figure}[!ht]
	\centering
	\includegraphics[width=\columnwidth]{./figs/vectors/quad3.pdf}
	\caption{}
	\label{fig:3.5.4_quadrilateral3}
\end{figure}

\end{enumerate}


\end{enumerate}

\item Find the distance between the points $(0,0)$ and $ (36,15)$.
	\\
		\solution
		\documentclass{article}
\usepackage{amsmath}
\usepackage{xcolor}
\usepackage{gensymb}
\usepackage{ragged2e}
\usepackage{graphicx}
\usepackage{gensymb}
\usepackage{mathtools}
\newcommand{\mydet}[1]{\ensuremath{\begin{vmatrix}#1\end{vmatrix}}}
\providecommand{\brak}[1]{\ensuremath{\left(#1\right)}}
\providecommand{\norm}[1]{\left\lVert#1\right\rVert}
\newcommand{\solution}{\noindent \textbf{Solution: }}
\newcommand{\myvec}[1]{\ensuremath{\begin{pmatrix}#1\end{pmatrix}}}
\let\vec\mathbf 


\begin{document}
\begin{center}
        \textbf\large{CHAPTER-7 \\ TRIANGLES}
\end{center}
\section{Exercise 7.1}
Q3. $AD$ and $BC$ are equal perpendiculars to a line segment $AB$. Show that $CD$ bisects $AB$.\\
\textbf{Construction}\\
\begin{figure}[h]
	\begin{center}
		\includegraphics[width=\columnwidth]{figs/Figure1.png}
	\end{center}
	\label{fig:Fig1}
\end{figure}
The input parameters for construction are shown in \ref{tab:Table1}:\\
\begin{table}[h]
	  \centering
	  \begin{tabular}{|c|c|c|}
  \hline
  \textbf{Symbol}&\textbf{Value}&\textbf{Description}\\
  \hline
  $a$ & 8 & $BC$\\
  \hline
	$\angle{B}$ & 45$\degree{}$ & $\angle{B}$ in $\triangle$$ABC$ \\
  \hline
	$k$ & 3.5 & $AB-AC$ i.e $c-b$ \\
  \hline 
	$\vec{e_2}$ & $\myvec{
			0\\
			1\\
			}$ & Basis vector\\
 \hline			
\end{tabular}

	  \caption{Parameters}
	  \label{tab:Table1}
\end{table}
\pagebreak
\begin{align}
	\vec{A} = a\vec{e_1},\vec{B} = \myvec{a\\b},\vec{C} = \myvec{2a\\b},\vec{D} = \myvec{0\\0}
\end{align}
\solution
Given
\begin{align}
	\vec{D}-\vec{A}&=\vec{B}-\vec{C}\\
	\angle DAB &= \angle CBA=90\degree
\end{align}
\textbf{To Prove:}\\
\begin{align}
	\vec{C}-\vec{O}&=\vec{O}-\vec{D}
\end{align}
\textbf{Proof:}\\
Consider linesegment $DC$\\
Let $\vec{O}$ represent the Midpoint of $DC$
\begin{align}
	\vec{O}&=\frac{1}{2}(\vec{C}+\vec{D})\\
	\implies &= \frac{1}{2}\myvec{6\\8}+\frac{1}{2}\myvec{0\\0}\\
	\implies &= \frac{1}{2}\myvec{6\\8}\\
		 &=\myvec{3\\4}
		 \label{eq:1}\\
\end{align}
\begin{align}
	\text { Since $AB\perp DA$, $AB$ is parallel to $x=0$ }\\
	\text { Equation of $AB$ is defined as $x=3$}\\
	\label{eq:2}\\
\end{align}
from $\eqref{eq:1}$ and $\eqref{eq:2}$
$\vec{O}$ lies on linesegment $CD$ and line $DC$ intersects $BA$ at its midpoint $O$.
\begin{align}
\vec{C}-\vec{O}=\vec{O}-\vec{D}
\end{align}
\end{document}

\item Find the point on the x-axis which is equidistant from $(2,-5)$ and $(-2,9)$.
	\\
\solution
		\iffalse
\documentclass[12pt]{article}
\usepackage{graphicx}
%\documentclass[journal,12pt,twocolumn]{IEEEtran}
\usepackage[none]{hyphenat}
\usepackage{graphicx}
\usepackage{listings}
\usepackage[english]{babel}
\usepackage{graphicx}
\usepackage{caption} 
\usepackage{hyperref}
\usepackage{booktabs}
\def\inputGnumericTable{}
\usepackage{color}                                            %%
    \usepackage{array}                                            %%
    \usepackage{longtable}                                        %%
    \usepackage{calc}                                             %%
    \usepackage{multirow}                                         %%
    \usepackage{hhline}                                           %%
    \usepackage{ifthen}
\usepackage{array}
\usepackage{amsmath}   % for having text in math mode
\usepackage{listings}
\lstset{
language=tex,
frame=single, 
breaklines=true
}
  
%Following 2 lines were added to remove the blank page at the beginning
\usepackage{atbegshi}% http://ctan.org/pkg/atbegshi
\AtBeginDocument{\AtBeginShipoutNext{\AtBeginShipoutDiscard}}
%


%New macro definitions
\newcommand{\mydet}[1]{\ensuremath{\begin{vmatrix}#1\end{vmatrix}}}
\providecommand{\brak}[1]{\ensuremath{\left(#1\right)}}
\providecommand{\norm}[1]{\left\lVert#1\right\rVert}
\newcommand{\solution}{\noindent \textbf{Solution: }}
\newcommand{\myvec}[1]{\ensuremath{\begin{pmatrix}#1\end{pmatrix}}}
\let\vec\mathbf

\begin{document}

\begin{center}
\title{\textbf{Coordinate Geometry}}
\date{\vspace{-5ex}} %Not to print date automatically
\maketitle
\end{center}

\setcounter{page}{1}



\section*{10$^{th}$ Maths - Chapter 7}

This is Problem-7 from Exercise 7.1

\begin{enumerate}

\item The point on the $x$-axis which is equidistant from $\myvec{2 \\ -5}$ and $\myvec{-2\\9}$\\
\solution \\
\fi
		The input parameters for this problem are available in Table \ref{tab:10/7/1/7Table-1}
\begin{table}[ht!]
\begin{tabular}{|c|c|p{5cm}|}
\hline
\textbf{Symbol} & \textbf{Value} & \textbf{Description} \\
\hline
$\theta$ & $30\degree$ & $\angle{BAP} = \angle{BAQ}$ \\
\hline
$a$ & $9$ & $AB$ \\
\hline
$c$ & $8$ & $AQ$ \\
\hline
$\vec{e}_1$ & $\myvec{1\\0}$ & Basis vector \\
\hline
\end{tabular}

\caption{}
\label{tab:10/7/1/7Table-1}	
\end{table}
%
  If $\vec{O}$ lies on the  $x$-axis and is  equidistant from the points $\vec{A}$ and $\vec{B}$, 
\begin{align}
 \norm{\vec{O}-\vec{A}} &=
\norm{\vec{A}-\vec{B}} 
\\
 \implies \norm{\vec{O}-\vec{A}}^2 &=
\norm{\vec{O}-\vec{B}}^2 
\end{align}
which can be expressed as 
\begin{multline}
%  \label{eq:10/7/1/7norm2d_dist}
 \brak{\vec{O}-\vec{A}}^{\top} \brak{\vec{O}-\vec{A}}=
 \brak{\vec{O}-\vec{B}}^{\top} 
\brak{\vec{O}-\vec{B}}
\\
 \implies \norm{\vec{O}}^2-2{\vec{O}}^{\top}\vec{A} + \norm{\vec{A}}^2
 \\= \norm{\vec{O}}^2-2{\vec{O}}^{\top}\vec{B} + \norm{\vec{B}}^2
\end{multline}
which can be simplified to obtain
%  \eqref{eq:10/7/1/7norm2d_equidist}.
  \begin{align}
   \vec{O} &=
    x\vec{e}_1
  \end{align}
  where 
  \begin{align}
   x &=\frac{\norm{\vec{A}}^2 -\norm{\vec{B}}^2 }{2\brak{\vec{A}-\vec{B}}^{\top }\vec{e}_1
}\label{eq:10/7/1/75}  
  \end{align}
  Substituting from Table \eqref{tab:10/7/1/7Table-1} in \eqref{eq:10/7/1/75},
\begin{align}
 \brak{\vec{A}-\vec{B}}^{\top}=
 \brak{\myvec{2 \\ -5}-\myvec{-2\\9}}^{\top}
	&=\myvec{4 & -14}
	\\
	\norm{\vec{A}}^2 = 21,
	\norm{\vec{B}}^2 &= 85
    \end{align}
yielding $x$ = $ -7$.  Thus, 
		\begin{align}
\vec{O} = \myvec{ -7 \\ 0}.
		\end{align}
		See Fig. 
\ref{fig:10/7/1/7Fig1}.

\begin{figure}[!h]
 \begin{center}
  \includegraphics[width=\columnwidth]{chapters/10/7/1/7/figs/fig.pdf}
 \end{center}
\caption{}
\label{fig:10/7/1/7Fig1}
\end{figure}


\item Find the values of $y$ for which the distance between the points                  $\vec{P}(2,-3)$ and $\vec{Q}(10,y)$ is 10 units.
\item  If $\vec{Q}(0, 1)$ is equidistant from $\vec{P}(5, -3)$ and $\vec{R}(x, 6)$, find the values of $x$. Also find the
distances $QR$ and $PR$.
\item  Find a relation between $x$ and $y$ such that the point $(x,y)$ is equidistant from the point
$(3, 6)$ and $(– 3, 4)$.
	\\
\solution
		\iffalse
\documentclass[12pt]{article}
\usepackage{graphicx}
%\documentclass[journal,12pt,twocolumn]{IEEEtran}
\usepackage[none]{hyphenat}
\usepackage{graphicx}
\usepackage{listings}
\usepackage[english]{babel}
\usepackage{graphicx}
\usepackage{caption} 
\usepackage{hyperref}
\usepackage{booktabs}
\def\inputGnumericTable{}
\usepackage{color}                                            %%
    \usepackage{array}                                            %%
    \usepackage{longtable}                                        %%
    \usepackage{calc}                                             %%
    \usepackage{multirow}                                         %%
    \usepackage{hhline}                                           %%
    \usepackage{ifthen}
\usepackage{array}
\usepackage{amsmath}   % for having text in math mode
\usepackage{listings}
\lstset{
language=tex,
frame=single, 
breaklines=true
}
  
%Following 2 lines were added to remove the blank page at the beginning
\usepackage{atbegshi}% http://ctan.org/pkg/atbegshi
\AtBeginDocument{\AtBeginShipoutNext{\AtBeginShipoutDiscard}}
%


%New macro definitions
\newcommand{\mydet}[1]{\ensuremath{\begin{vmatrix}#1\end{vmatrix}}}
\providecommand{\brak}[1]{\ensuremath{\left(#1\right)}}
\providecommand{\norm}[1]{\left\lVert#1\right\rVert}
\newcommand{\solution}{\noindent \textbf{Solution: }}
\newcommand{\myvec}[1]{\ensuremath{\begin{pmatrix}#1\end{pmatrix}}}
\let\vec\mathbf

\begin{document}

\begin{center}
\title{\textbf{Coordinate Geometry}}
\date{\vspace{-5ex}} %Not to print date automatically
\maketitle
\end{center}

\setcounter{page}{1}



\begin{enumerate}

\item\textbf{Problem statement :} Find a relation between x and y such that the point $\myvec{x ,y}$ is equidistant from the point $\myvec{3 ,6}$ and $\myvec{-3 ,4}$

\solution \\
\textbf{\centering{Method I}}
\fi
The input parameters for this problem are given as
	\begin{align}
	\vec{P} = \myvec{
		x\\
		y\\
		},
	\vec{A} = \myvec{
		3\\
		6\\
		},
        \vec{B} = \myvec{
		3\\
		-4\\
		}
	\end{align}
\iffalse

  If $\vec{P}$ equidistant from the points $\vec{A}$ and $\vec{B}$, 
\begin{align}
 \norm{\vec{P}-\vec{A}} &=
\norm{\vec{P}-\vec{B}} 
\\
 \implies \norm{\vec{P}-\vec{A}}^2 &=
\norm{\vec{P}-\vec{B}}^2 
\end{align}
which can be expressed as 
\begin{align}
%  \label{eq:chapters/10/7/1/10/norm2d_dist}
 \brak{\vec{P}-\vec{A}}^{\top} \brak{\vec{P}-\vec{A}}=
 \brak{\vec{P}-\vec{B}}^{\top} 
\brak{\vec{P}-\vec{B}}
\\
 \implies \norm{\vec{P}}^2-2{\vec{P}}^{\top}\vec{A} + \norm{\vec{A}}^2
 \\= \norm{\vec{P}}^2-2{\vec{P}}^{\top}\vec{B} + \norm{\vec{B}}^2
\end{align}
which can be simplified to obtain
%  \eqref{eq:chapters/10/7/1/10/norm2d_equidist}.
  \fi
  \begin{align}
   \vec{P} =
    y\vec{e}_1
  \end{align}
  where 
  \begin{align}
   y &=\frac{\norm{\vec{A}}^2 -\norm{\vec{B}}^2 }{2\brak{\vec{A}-\vec{B}}^{\top }\vec{e}_1
}\label{eq:chapters/10/7/1/10/5}  
  \end{align}
  Substituting the $\vec{A}, \vec{B}$ values in \eqref{eq:chapters/10/7/1/10/5},
\begin{align}
 \brak{\vec{A}-\vec{B}}=
 \myvec{6 \\ 2},
   \norm{\vec{A}}^2 = 45,
   \norm{\vec{B}}^2 = 25
    \end{align}
    yielding
$y =  5$.
Hence, 
\begin{align}
\vec{P} = \myvec{ 0 \\ 5}
\end{align}
%
See Fig. 
\ref{fig:chapters/10/7/1/10/Fig1}.
\begin{figure}[!h]
 \begin{center}
  \includegraphics[width=\columnwidth]{./chapters/10/7/1/10/figs/fig.png}
 \end{center}
\caption{}
\label{fig:chapters/10/7/1/10/Fig1}
\end{figure}


	\item Find a point on the x-axis,which is equidistant from the points$\myvec{
  7 \\
  6 \\
 }$ and $\myvec{
  3 \\
  4 \\
 }$
.
\label{chapters/11/10/1/4}
\iffalse
\documentclass[journal,10pt,twocolumn]{article}
\usepackage{graphicx}
\usepackage[margin=0.5in]{geometry}
\usepackage[cmex10]{amsmath}
\usepackage{array}
\usepackage{booktabs}
\usepackage{makecell}
\title{\textbf{Line Assignment}}
\author{Hari Venkateswarlu}
\date{September 2022}
\usepackage[framemethod=tikz]{mdframed}
\newcommand{\myvec}[1]{\ensuremath{\myvec{#1}}}
\let\vec\mathbf
\newcommand{\mydet}[1]{\ensuremath{\begin{vmatrix}#1\end{vmatrix}}}
\providecommand{\mbf}{\mathbf}
\providecommand{\pr}[1]{\ensuremath{\Pr\left(#1\right)}}
\providecommand{\qfunc}[1]{\ensuremath{Q\left(#1\right)}}
\providecommand{\sbrak}[1]{\ensuremath{{}\left[#1\right]}}
\providecommand{\lsbrak}[1]{\ensuremath{{}\left[#1\right.}}
\providecommand{\rsbrak}[1]{\ensuremath{{}\left.#1\right]}}
\providecommand{\brak}[1]{\ensuremath{\left(#1\right)}}
\providecommand{\lbrak}[1]{\ensuremath{\left(#1\right.}}
\providecommand{\rbrak}[1]{\ensuremath{\left.#1\right)}}
\providecommand{\cbrak}[1]{\ensuremath{\left\{#1\right\}}}
\providecommand{\lcbrak}[1]{\ensuremath{\left\{#1\right.}}
\providecommand{\rcbrak}[1]{\ensuremath{\left.#1\right\}}}

\begin{document}

\maketitle
\paragraph{\textit{Problem Statement} - 
\fi
Find a point on the x-axis,which is equidistant from the points$\myvec{
  7 \\
  6 \\
 }$ and $\myvec{
  3 \\
  4 \\
 }$
.
	\begin{figure}[!ht]
		\centering
 \includegraphics[width=\columnwidth]{chapters/11/10/1/4/figs/line.png}
		\caption{}
		\label{fig:11/10/1/4}
  	\end{figure}
	\\
	\solution 
\iffalse
 }
\begin{center}
    \label{tab:truthtable}
    \setlength{\arrayrulewidth}{0.2mm}
\setlength{\tabcolsep}{5pt}
\renewcommand{\arraystretch}{1.25}
    \begin{tabular}{|c|c|c|}
    \hline % <-- Alignments: 1st column left, 2nd middle and 3rd right, with vertical lines in between
      \large\textbf{Symbol} & \large\textbf{Co-ordinates} & \large\textbf{Description}\\
      \hline
       \large A & $\ \myvec{ 7\\ 6 }$ & co-ordinates of A \\
       \large B & $\ \myvec{ 3\\ 4 }$ & co-ordinates of B \\
	
	
      \hline
   \end{tabular}
 \end{center}\vspace{5mm}

\begin{figure}[h]
\centering
\includegraphics[width=1\columnwidth]{Figure1.png}

\label{fig}
\end{figure}

\section*{Solution}
1. Given points
A=$\myvec{
  7 \\
  6 \\
 }$
 and B=$\myvec{
  3 \\
  4 \\
 }$


\raggedright 2. If the point is lying on x-axis then y-axis will be zero i.e.. y=0

\fi
From the given information

\begin{align}
	\norm{\vec{x}-\vec{A}}^2 &=\norm{\vec{x}-\vec{B}}^2
	\\
	\implies
	\brak{\vec{x}-\vec{A}}^{\top} \brak{\vec{x}-\vec{A}} &= \brak{\vec{x}-\vec{B}}^{\top} \brak{\vec{x}-\vec{B}}
	\\
	\implies     \norm{\vec{x}}^2 - 2\vec{A}^{\top}\vec{x} + \norm{\vec{A}}^2 &= \norm{\vec{x}}^2 - 2\vec{B}^{\top}\vec{x} + \norm{\vec{B}}^2
	\\
	\text{or, }
	\brak{\vec{A}-\vec{B}}^{\top}   \vec{x}&= \frac{\norm{\vec{A}}^2 - \norm{\vec{B}}^2}{2}
		\label{eq:11/10/1/4}
\end{align}  
Since $\vec{x}$ lies on the $x$-axis,
\begin{align}
	\vec{x} &=k\vec{e}_1
\end{align}  
which, upon substituting in 
		\eqref{eq:11/10/1/4}
		yields
\begin{align}
	k = \frac{15}{2}
\end{align}
\iffalse
$\vec{(A-B)^{\top}x} = \frac{\|\vec{A}\|^2 - \|\vec{B}\|^2}{2}$\\ \vspace{2mm}
     $\myvec{0 & 1 \\ 4 & 2}x = $\myvec{0 \\ 30}\\ \vspace{2mm}
      $\myvec{0 & 1 & 0 \\ 4 & 2 & 30}$\\    \vspace{2mm}
      Divide by 2\\
      $\myvec{0 & 1 & 0 \\ 2 & 1 & 15}$\\    \vspace{2mm}
     $\myvec{2 & 1 & 15 \\ 0 & 1 & 0}
    \xleftarrow[]{R_2 \leftarrow R_1}$\\     \vspace{2mm}
    $\myvec{1 & \frac{1}{2} & \frac{15}{2} \\ 0 & 1 & 0}\xleftarrow[]{{R_1}=\frac{R_1}{2}}$\\            \vspace{2mm}
    $\myvec{1 & 0 & \frac{15}{2} \\ 0 & 1 & 0}\xleftarrow[]{{R_1}={R_1}-\frac{R_2}{2}}$\\             \vspace{2mm}
    $\myvec{1 & 0 & 7.5 \\ 0 & 1 & 0}$\\        \vspace{2mm}
on solving we get x = 7.5\\
\vspace{2mm}
  x=$\myvec{
  7.5 \\
  0 \\
 }$               			
\end{document}
\fi


\end{enumerate}


\subsection{Exercises}
\iffalse
\documentclass[12pt]{article}
\usepackage{graphicx}
\usepackage{commath}
\usepackage{gensymb}
\usepackage{float}

\begin{document}
\begin{center}
\textbf\large{CHAPTER-7 \\ COORDINATE GEOMETRY}
\end{center}

\section{EXERCISE - 7.1}
\fi
\begin{enumerate}[label=\thesection.\arabic*,ref=\thesection.\theenumi]
\numberwithin{equation}{enumi}
\numberwithin{figure}{enumi}
\numberwithin{table}{enumi}
\item Find the distances between the following pairs of points:
\begin{enumerate}
\item $(2,3),(4,1)$
\item $(-5,7),(-1,3)$
\item $(a,b),(-a,-b)$
\end{enumerate}
		\renewcommand{\theequation}{\theenumi}
\begin{enumerate}[label=\thesubsection.\arabic*.,ref=\thesubsection.\theenumi]
%\begin{enumerate}[label=\arabic*.,ref=\thesection.\theenumi]
\numberwithin{equation}{enumi}

\item Find the distance between the points 
\begin{align}
\myvec{0\\0}, \myvec{36\\15}
\end{align}
%
\solution The desired distance is 
\begin{align}
\norm{\vec{A} - \vec{B}} 
=\norm{\myvec{0\\0} - \myvec{36\\15}} = \sqrt{36^2+15^2} = 39
\end{align}

\item Find the distance between the following pairs of points
\begin{enumerate}
\item 
\begin{align}
\myvec{2\\3}, \myvec{4\\1}
\end{align}
\item 
\begin{align}
\myvec{-5\\7}, \myvec{-1\\3}
\end{align}
\item 
\begin{align}
\myvec{a\\b}, \myvec{-1\\b}
\end{align}
\end{enumerate}
\solution The distance between two vectors is given by 
\begin{align}
\norm{\vec{A}-\vec{B}}
\label{eq:vectors_3.5.1_norm}
\end{align}
From \eqref{eq:vectors_3.5.1_norm},
\begin{enumerate}
\item 
$\norm{\myvec{2\\3}-\myvec{4\\1}}=\sqrt{\brak{2-4}^2+\brak{3-1}^2 = 2\sqrt{2}}$
\item $\norm{\myvec{-5\\7}-\myvec{-1\\3}}=\sqrt{\brak{-5+1}^2+\brak{7-3}^2 = 4\sqrt{2}}$
\item $\norm{\myvec{a\\b}-\myvec{-1\\b}}=a+1$

\end{enumerate}
\item Name the type of quadrilateral formed, if any, by the following points, and give reasons for your answer.
\begin{enumerate}
\item 
\begin{align}
\vec{P} = \myvec{-1\\-2}, \vec{Q} =\myvec{1\\0},
\vec{R} =\myvec{-1\\2}, \vec{S} =\myvec{-3\\0}
\end{align}
\item 
\begin{align}
\vec{P} = \myvec{-3\\5}, \vec{Q} =\myvec{3\\1},
\vec{R} =\myvec{0\\3}, \vec{S} =\myvec{-1\\-4}
\end{align}
\item 
\begin{align}
\vec{P} = \myvec{4\\5}, \vec{Q} =\myvec{7\\6},
\\
\vec{R} =\myvec{4\\3}, \vec{S} =\myvec{1\\2}
\end{align}
\end{enumerate}
\solution
\begin{enumerate}
\item In Fig. 	\ref{fig:3.5.4_quadrilateral1}
\begin{align}
\label{eq:constr_vectors_quad1_PSQR}
 \vec{P} - \vec{S} &= 
 \vec{Q} - \vec{R} = \myvec{2\\-2}
\\
\vec {R} - \vec {S} &=
 \vec {Q} - \vec {P} = \myvec{2\\2}
\label{eq:constr_vectors_quad1_RSQP}
\end{align}
%
Hence $PQRS$ is a $\parallel$gm $\because$  opposite sides are parallel. Also, 
\begin{align}
\norm{ \vec{P} - \vec{S}} &= 
\norm{ \vec{Q} - \vec{R}} 
\\
=\norm{\vec {R} - \vec {S}} &=
\norm{ \vec {Q} - \vec {P}} = 2\sqrt{2}
\end{align}
%
$\because$ all sides are equal, the $\parallel$gm is a rhombus. The angle between $PS$ and
$RS$ is given by 
\begin{align}
\cos \theta = \frac{\brak{\vec{S}-\vec{P}}^{\top}\brak{\vec{S}-\vec{R}}}{\norm{\vec{S}-\vec{P}}^{\top}\norm{\vec{S}-\vec{R}}}
\end{align}
%
$\because $
\begin{align}
\brak{\vec{S}-\vec{P}}^{\top}\brak{\vec{S}-\vec{R}} = \myvec{2 & -2}\myvec{2\\2} = 0
\end{align}
upon substituting from \eqref{eq:constr_vectors_quad1_PSQR} and \eqref{eq:constr_vectors_quad1_RSQP}, 
\begin{align}
\cos \theta = 0 \implies PS \perp RS
\end{align}
%
Thus, the rhombus is actually a square.

\begin{figure}[!ht]
	\centering
	\includegraphics[width=\columnwidth]{./figs/vectors/quad1.png}
	\caption{quadrilateral1 }
	\label{fig:3.5.4_quadrilateral1}
\end{figure}
\begin{lstlisting}
codes/vectors/quad1.py
\end{lstlisting}

\item In Fig. 	\ref{fig:3.5.4_quadrilateral2}

\begin{align}
\vec{Q} - \vec{P} &= \myvec{6\\-4}
\\
\vec{R} - \vec{P} &= \myvec{3\\-2}
\\
\vec{Q} - \vec{R} &= \myvec{3\\-2}
\\
\left(\vec{Q} - \vec{P}\right) &= \left(\vec{R} - \vec{P}\right) + \left(\vec{Q} - \vec{R}\right) = \myvec{6\\-4}
\end{align}
Hence,  $\vec P,\vec Q$ and $\vec R$ lie on a straight line, so $PQRS$ is not  a quadrilateral.
%
\begin{figure}[!ht]
	\centering
	\includegraphics[width=\columnwidth]{./figs/vectors/quad2.pdf}
	\caption{quadrilateral2 }
	\label{fig:3.5.4_quadrilateral2}
\end{figure}
%
\item See Fig. 	\ref{fig:3.5.4_quadrilateral3}.

\begin{align}
\because \left(\vec{Q} - \vec{P}\right) &= \left(\vec{R} - \vec{S}\right) = \myvec{3\\1}
\\
\left(\vec{P} - \vec{S}\right) &= \left(\vec{Q} - \vec{R}\right) = \myvec{3\\3},
\end{align}
$PQRS$ is a parallelogram.  Also, 
%
\begin{align}
\norm{\vec{Q} - \vec{P}} \ne \norm{\vec{P} - \vec{S}}
\end{align}
Hence, $PQRS$ is neither a rhombus nor a square.
\begin{align}
\because \left(\vec{Q} - \vec{P}\right)^T \left(\vec{Q} - \vec{R}\right) = \myvec{3 & 1}\myvec{3\\3} \ne 0,
\end{align}
$PQRS$ is not a rectangle. 
%
\begin{figure}[!ht]
	\centering
	\includegraphics[width=\columnwidth]{./figs/vectors/quad3.pdf}
	\caption{}
	\label{fig:3.5.4_quadrilateral3}
\end{figure}

\end{enumerate}


\end{enumerate}

\item Find the distance between the points $(0,0)$ and $ (36,15)$.
	\\
		\solution
		\documentclass{article}
\usepackage{amsmath}
\usepackage{xcolor}
\usepackage{gensymb}
\usepackage{ragged2e}
\usepackage{graphicx}
\usepackage{gensymb}
\usepackage{mathtools}
\newcommand{\mydet}[1]{\ensuremath{\begin{vmatrix}#1\end{vmatrix}}}
\providecommand{\brak}[1]{\ensuremath{\left(#1\right)}}
\providecommand{\norm}[1]{\left\lVert#1\right\rVert}
\newcommand{\solution}{\noindent \textbf{Solution: }}
\newcommand{\myvec}[1]{\ensuremath{\begin{pmatrix}#1\end{pmatrix}}}
\let\vec\mathbf 


\begin{document}
\begin{center}
        \textbf\large{CHAPTER-7 \\ TRIANGLES}
\end{center}
\section{Exercise 7.1}
Q3. $AD$ and $BC$ are equal perpendiculars to a line segment $AB$. Show that $CD$ bisects $AB$.\\
\textbf{Construction}\\
\begin{figure}[h]
	\begin{center}
		\includegraphics[width=\columnwidth]{figs/Figure1.png}
	\end{center}
	\label{fig:Fig1}
\end{figure}
The input parameters for construction are shown in \ref{tab:Table1}:\\
\begin{table}[h]
	  \centering
	  \begin{tabular}{|c|c|c|}
  \hline
  \textbf{Symbol}&\textbf{Value}&\textbf{Description}\\
  \hline
  $a$ & 8 & $BC$\\
  \hline
	$\angle{B}$ & 45$\degree{}$ & $\angle{B}$ in $\triangle$$ABC$ \\
  \hline
	$k$ & 3.5 & $AB-AC$ i.e $c-b$ \\
  \hline 
	$\vec{e_2}$ & $\myvec{
			0\\
			1\\
			}$ & Basis vector\\
 \hline			
\end{tabular}

	  \caption{Parameters}
	  \label{tab:Table1}
\end{table}
\pagebreak
\begin{align}
	\vec{A} = a\vec{e_1},\vec{B} = \myvec{a\\b},\vec{C} = \myvec{2a\\b},\vec{D} = \myvec{0\\0}
\end{align}
\solution
Given
\begin{align}
	\vec{D}-\vec{A}&=\vec{B}-\vec{C}\\
	\angle DAB &= \angle CBA=90\degree
\end{align}
\textbf{To Prove:}\\
\begin{align}
	\vec{C}-\vec{O}&=\vec{O}-\vec{D}
\end{align}
\textbf{Proof:}\\
Consider linesegment $DC$\\
Let $\vec{O}$ represent the Midpoint of $DC$
\begin{align}
	\vec{O}&=\frac{1}{2}(\vec{C}+\vec{D})\\
	\implies &= \frac{1}{2}\myvec{6\\8}+\frac{1}{2}\myvec{0\\0}\\
	\implies &= \frac{1}{2}\myvec{6\\8}\\
		 &=\myvec{3\\4}
		 \label{eq:1}\\
\end{align}
\begin{align}
	\text { Since $AB\perp DA$, $AB$ is parallel to $x=0$ }\\
	\text { Equation of $AB$ is defined as $x=3$}\\
	\label{eq:2}\\
\end{align}
from $\eqref{eq:1}$ and $\eqref{eq:2}$
$\vec{O}$ lies on linesegment $CD$ and line $DC$ intersects $BA$ at its midpoint $O$.
\begin{align}
\vec{C}-\vec{O}=\vec{O}-\vec{D}
\end{align}
\end{document}

\item Find the point on the x-axis which is equidistant from $(2,-5)$ and $(-2,9)$.
	\\
\solution
		\iffalse
\documentclass[12pt]{article}
\usepackage{graphicx}
%\documentclass[journal,12pt,twocolumn]{IEEEtran}
\usepackage[none]{hyphenat}
\usepackage{graphicx}
\usepackage{listings}
\usepackage[english]{babel}
\usepackage{graphicx}
\usepackage{caption} 
\usepackage{hyperref}
\usepackage{booktabs}
\def\inputGnumericTable{}
\usepackage{color}                                            %%
    \usepackage{array}                                            %%
    \usepackage{longtable}                                        %%
    \usepackage{calc}                                             %%
    \usepackage{multirow}                                         %%
    \usepackage{hhline}                                           %%
    \usepackage{ifthen}
\usepackage{array}
\usepackage{amsmath}   % for having text in math mode
\usepackage{listings}
\lstset{
language=tex,
frame=single, 
breaklines=true
}
  
%Following 2 lines were added to remove the blank page at the beginning
\usepackage{atbegshi}% http://ctan.org/pkg/atbegshi
\AtBeginDocument{\AtBeginShipoutNext{\AtBeginShipoutDiscard}}
%


%New macro definitions
\newcommand{\mydet}[1]{\ensuremath{\begin{vmatrix}#1\end{vmatrix}}}
\providecommand{\brak}[1]{\ensuremath{\left(#1\right)}}
\providecommand{\norm}[1]{\left\lVert#1\right\rVert}
\newcommand{\solution}{\noindent \textbf{Solution: }}
\newcommand{\myvec}[1]{\ensuremath{\begin{pmatrix}#1\end{pmatrix}}}
\let\vec\mathbf

\begin{document}

\begin{center}
\title{\textbf{Coordinate Geometry}}
\date{\vspace{-5ex}} %Not to print date automatically
\maketitle
\end{center}

\setcounter{page}{1}



\section*{10$^{th}$ Maths - Chapter 7}

This is Problem-7 from Exercise 7.1

\begin{enumerate}

\item The point on the $x$-axis which is equidistant from $\myvec{2 \\ -5}$ and $\myvec{-2\\9}$\\
\solution \\
\fi
		The input parameters for this problem are available in Table \ref{tab:10/7/1/7Table-1}
\begin{table}[ht!]
\begin{tabular}{|c|c|p{5cm}|}
\hline
\textbf{Symbol} & \textbf{Value} & \textbf{Description} \\
\hline
$\theta$ & $30\degree$ & $\angle{BAP} = \angle{BAQ}$ \\
\hline
$a$ & $9$ & $AB$ \\
\hline
$c$ & $8$ & $AQ$ \\
\hline
$\vec{e}_1$ & $\myvec{1\\0}$ & Basis vector \\
\hline
\end{tabular}

\caption{}
\label{tab:10/7/1/7Table-1}	
\end{table}
%
  If $\vec{O}$ lies on the  $x$-axis and is  equidistant from the points $\vec{A}$ and $\vec{B}$, 
\begin{align}
 \norm{\vec{O}-\vec{A}} &=
\norm{\vec{A}-\vec{B}} 
\\
 \implies \norm{\vec{O}-\vec{A}}^2 &=
\norm{\vec{O}-\vec{B}}^2 
\end{align}
which can be expressed as 
\begin{multline}
%  \label{eq:10/7/1/7norm2d_dist}
 \brak{\vec{O}-\vec{A}}^{\top} \brak{\vec{O}-\vec{A}}=
 \brak{\vec{O}-\vec{B}}^{\top} 
\brak{\vec{O}-\vec{B}}
\\
 \implies \norm{\vec{O}}^2-2{\vec{O}}^{\top}\vec{A} + \norm{\vec{A}}^2
 \\= \norm{\vec{O}}^2-2{\vec{O}}^{\top}\vec{B} + \norm{\vec{B}}^2
\end{multline}
which can be simplified to obtain
%  \eqref{eq:10/7/1/7norm2d_equidist}.
  \begin{align}
   \vec{O} &=
    x\vec{e}_1
  \end{align}
  where 
  \begin{align}
   x &=\frac{\norm{\vec{A}}^2 -\norm{\vec{B}}^2 }{2\brak{\vec{A}-\vec{B}}^{\top }\vec{e}_1
}\label{eq:10/7/1/75}  
  \end{align}
  Substituting from Table \eqref{tab:10/7/1/7Table-1} in \eqref{eq:10/7/1/75},
\begin{align}
 \brak{\vec{A}-\vec{B}}^{\top}=
 \brak{\myvec{2 \\ -5}-\myvec{-2\\9}}^{\top}
	&=\myvec{4 & -14}
	\\
	\norm{\vec{A}}^2 = 21,
	\norm{\vec{B}}^2 &= 85
    \end{align}
yielding $x$ = $ -7$.  Thus, 
		\begin{align}
\vec{O} = \myvec{ -7 \\ 0}.
		\end{align}
		See Fig. 
\ref{fig:10/7/1/7Fig1}.

\begin{figure}[!h]
 \begin{center}
  \includegraphics[width=\columnwidth]{chapters/10/7/1/7/figs/fig.pdf}
 \end{center}
\caption{}
\label{fig:10/7/1/7Fig1}
\end{figure}


\item Find the values of $y$ for which the distance between the points                  $\vec{P}(2,-3)$ and $\vec{Q}(10,y)$ is 10 units.
\item  If $\vec{Q}(0, 1)$ is equidistant from $\vec{P}(5, -3)$ and $\vec{R}(x, 6)$, find the values of $x$. Also find the
distances $QR$ and $PR$.
\item  Find a relation between $x$ and $y$ such that the point $(x,y)$ is equidistant from the point
$(3, 6)$ and $(– 3, 4)$.
	\\
\solution
		\iffalse
\documentclass[12pt]{article}
\usepackage{graphicx}
%\documentclass[journal,12pt,twocolumn]{IEEEtran}
\usepackage[none]{hyphenat}
\usepackage{graphicx}
\usepackage{listings}
\usepackage[english]{babel}
\usepackage{graphicx}
\usepackage{caption} 
\usepackage{hyperref}
\usepackage{booktabs}
\def\inputGnumericTable{}
\usepackage{color}                                            %%
    \usepackage{array}                                            %%
    \usepackage{longtable}                                        %%
    \usepackage{calc}                                             %%
    \usepackage{multirow}                                         %%
    \usepackage{hhline}                                           %%
    \usepackage{ifthen}
\usepackage{array}
\usepackage{amsmath}   % for having text in math mode
\usepackage{listings}
\lstset{
language=tex,
frame=single, 
breaklines=true
}
  
%Following 2 lines were added to remove the blank page at the beginning
\usepackage{atbegshi}% http://ctan.org/pkg/atbegshi
\AtBeginDocument{\AtBeginShipoutNext{\AtBeginShipoutDiscard}}
%


%New macro definitions
\newcommand{\mydet}[1]{\ensuremath{\begin{vmatrix}#1\end{vmatrix}}}
\providecommand{\brak}[1]{\ensuremath{\left(#1\right)}}
\providecommand{\norm}[1]{\left\lVert#1\right\rVert}
\newcommand{\solution}{\noindent \textbf{Solution: }}
\newcommand{\myvec}[1]{\ensuremath{\begin{pmatrix}#1\end{pmatrix}}}
\let\vec\mathbf

\begin{document}

\begin{center}
\title{\textbf{Coordinate Geometry}}
\date{\vspace{-5ex}} %Not to print date automatically
\maketitle
\end{center}

\setcounter{page}{1}



\begin{enumerate}

\item\textbf{Problem statement :} Find a relation between x and y such that the point $\myvec{x ,y}$ is equidistant from the point $\myvec{3 ,6}$ and $\myvec{-3 ,4}$

\solution \\
\textbf{\centering{Method I}}
\fi
The input parameters for this problem are given as
	\begin{align}
	\vec{P} = \myvec{
		x\\
		y\\
		},
	\vec{A} = \myvec{
		3\\
		6\\
		},
        \vec{B} = \myvec{
		3\\
		-4\\
		}
	\end{align}
\iffalse

  If $\vec{P}$ equidistant from the points $\vec{A}$ and $\vec{B}$, 
\begin{align}
 \norm{\vec{P}-\vec{A}} &=
\norm{\vec{P}-\vec{B}} 
\\
 \implies \norm{\vec{P}-\vec{A}}^2 &=
\norm{\vec{P}-\vec{B}}^2 
\end{align}
which can be expressed as 
\begin{align}
%  \label{eq:chapters/10/7/1/10/norm2d_dist}
 \brak{\vec{P}-\vec{A}}^{\top} \brak{\vec{P}-\vec{A}}=
 \brak{\vec{P}-\vec{B}}^{\top} 
\brak{\vec{P}-\vec{B}}
\\
 \implies \norm{\vec{P}}^2-2{\vec{P}}^{\top}\vec{A} + \norm{\vec{A}}^2
 \\= \norm{\vec{P}}^2-2{\vec{P}}^{\top}\vec{B} + \norm{\vec{B}}^2
\end{align}
which can be simplified to obtain
%  \eqref{eq:chapters/10/7/1/10/norm2d_equidist}.
  \fi
  \begin{align}
   \vec{P} =
    y\vec{e}_1
  \end{align}
  where 
  \begin{align}
   y &=\frac{\norm{\vec{A}}^2 -\norm{\vec{B}}^2 }{2\brak{\vec{A}-\vec{B}}^{\top }\vec{e}_1
}\label{eq:chapters/10/7/1/10/5}  
  \end{align}
  Substituting the $\vec{A}, \vec{B}$ values in \eqref{eq:chapters/10/7/1/10/5},
\begin{align}
 \brak{\vec{A}-\vec{B}}=
 \myvec{6 \\ 2},
   \norm{\vec{A}}^2 = 45,
   \norm{\vec{B}}^2 = 25
    \end{align}
    yielding
$y =  5$.
Hence, 
\begin{align}
\vec{P} = \myvec{ 0 \\ 5}
\end{align}
%
See Fig. 
\ref{fig:chapters/10/7/1/10/Fig1}.
\begin{figure}[!h]
 \begin{center}
  \includegraphics[width=\columnwidth]{./chapters/10/7/1/10/figs/fig.png}
 \end{center}
\caption{}
\label{fig:chapters/10/7/1/10/Fig1}
\end{figure}


	\item Find a point on the x-axis,which is equidistant from the points$\myvec{
  7 \\
  6 \\
 }$ and $\myvec{
  3 \\
  4 \\
 }$
.
\label{chapters/11/10/1/4}
\iffalse
\documentclass[journal,10pt,twocolumn]{article}
\usepackage{graphicx}
\usepackage[margin=0.5in]{geometry}
\usepackage[cmex10]{amsmath}
\usepackage{array}
\usepackage{booktabs}
\usepackage{makecell}
\title{\textbf{Line Assignment}}
\author{Hari Venkateswarlu}
\date{September 2022}
\usepackage[framemethod=tikz]{mdframed}
\newcommand{\myvec}[1]{\ensuremath{\myvec{#1}}}
\let\vec\mathbf
\newcommand{\mydet}[1]{\ensuremath{\begin{vmatrix}#1\end{vmatrix}}}
\providecommand{\mbf}{\mathbf}
\providecommand{\pr}[1]{\ensuremath{\Pr\left(#1\right)}}
\providecommand{\qfunc}[1]{\ensuremath{Q\left(#1\right)}}
\providecommand{\sbrak}[1]{\ensuremath{{}\left[#1\right]}}
\providecommand{\lsbrak}[1]{\ensuremath{{}\left[#1\right.}}
\providecommand{\rsbrak}[1]{\ensuremath{{}\left.#1\right]}}
\providecommand{\brak}[1]{\ensuremath{\left(#1\right)}}
\providecommand{\lbrak}[1]{\ensuremath{\left(#1\right.}}
\providecommand{\rbrak}[1]{\ensuremath{\left.#1\right)}}
\providecommand{\cbrak}[1]{\ensuremath{\left\{#1\right\}}}
\providecommand{\lcbrak}[1]{\ensuremath{\left\{#1\right.}}
\providecommand{\rcbrak}[1]{\ensuremath{\left.#1\right\}}}

\begin{document}

\maketitle
\paragraph{\textit{Problem Statement} - 
\fi
Find a point on the x-axis,which is equidistant from the points$\myvec{
  7 \\
  6 \\
 }$ and $\myvec{
  3 \\
  4 \\
 }$
.
	\begin{figure}[!ht]
		\centering
 \includegraphics[width=\columnwidth]{chapters/11/10/1/4/figs/line.png}
		\caption{}
		\label{fig:11/10/1/4}
  	\end{figure}
	\\
	\solution 
\iffalse
 }
\begin{center}
    \label{tab:truthtable}
    \setlength{\arrayrulewidth}{0.2mm}
\setlength{\tabcolsep}{5pt}
\renewcommand{\arraystretch}{1.25}
    \begin{tabular}{|c|c|c|}
    \hline % <-- Alignments: 1st column left, 2nd middle and 3rd right, with vertical lines in between
      \large\textbf{Symbol} & \large\textbf{Co-ordinates} & \large\textbf{Description}\\
      \hline
       \large A & $\ \myvec{ 7\\ 6 }$ & co-ordinates of A \\
       \large B & $\ \myvec{ 3\\ 4 }$ & co-ordinates of B \\
	
	
      \hline
   \end{tabular}
 \end{center}\vspace{5mm}

\begin{figure}[h]
\centering
\includegraphics[width=1\columnwidth]{Figure1.png}

\label{fig}
\end{figure}

\section*{Solution}
1. Given points
A=$\myvec{
  7 \\
  6 \\
 }$
 and B=$\myvec{
  3 \\
  4 \\
 }$


\raggedright 2. If the point is lying on x-axis then y-axis will be zero i.e.. y=0

\fi
From the given information

\begin{align}
	\norm{\vec{x}-\vec{A}}^2 &=\norm{\vec{x}-\vec{B}}^2
	\\
	\implies
	\brak{\vec{x}-\vec{A}}^{\top} \brak{\vec{x}-\vec{A}} &= \brak{\vec{x}-\vec{B}}^{\top} \brak{\vec{x}-\vec{B}}
	\\
	\implies     \norm{\vec{x}}^2 - 2\vec{A}^{\top}\vec{x} + \norm{\vec{A}}^2 &= \norm{\vec{x}}^2 - 2\vec{B}^{\top}\vec{x} + \norm{\vec{B}}^2
	\\
	\text{or, }
	\brak{\vec{A}-\vec{B}}^{\top}   \vec{x}&= \frac{\norm{\vec{A}}^2 - \norm{\vec{B}}^2}{2}
		\label{eq:11/10/1/4}
\end{align}  
Since $\vec{x}$ lies on the $x$-axis,
\begin{align}
	\vec{x} &=k\vec{e}_1
\end{align}  
which, upon substituting in 
		\eqref{eq:11/10/1/4}
		yields
\begin{align}
	k = \frac{15}{2}
\end{align}
\iffalse
$\vec{(A-B)^{\top}x} = \frac{\|\vec{A}\|^2 - \|\vec{B}\|^2}{2}$\\ \vspace{2mm}
     $\myvec{0 & 1 \\ 4 & 2}x = $\myvec{0 \\ 30}\\ \vspace{2mm}
      $\myvec{0 & 1 & 0 \\ 4 & 2 & 30}$\\    \vspace{2mm}
      Divide by 2\\
      $\myvec{0 & 1 & 0 \\ 2 & 1 & 15}$\\    \vspace{2mm}
     $\myvec{2 & 1 & 15 \\ 0 & 1 & 0}
    \xleftarrow[]{R_2 \leftarrow R_1}$\\     \vspace{2mm}
    $\myvec{1 & \frac{1}{2} & \frac{15}{2} \\ 0 & 1 & 0}\xleftarrow[]{{R_1}=\frac{R_1}{2}}$\\            \vspace{2mm}
    $\myvec{1 & 0 & \frac{15}{2} \\ 0 & 1 & 0}\xleftarrow[]{{R_1}={R_1}-\frac{R_2}{2}}$\\             \vspace{2mm}
    $\myvec{1 & 0 & 7.5 \\ 0 & 1 & 0}$\\        \vspace{2mm}
on solving we get x = 7.5\\
\vspace{2mm}
  x=$\myvec{
  7.5 \\
  0 \\
 }$               			
\end{document}
\fi


\end{enumerate}


\subsection{subsection Formula}
\begin{enumerate}[label=\thesection.\arabic*,ref=\thesection.\theenumi]
\numberwithin{equation}{enumi}
\numberwithin{figure}{enumi}
\numberwithin{table}{enumi}

\item Find the coordinates of the point which divides the join of $(-1,7) \text{ and } (4,-3)$ in the ratio 2:3.
	\\
		\solution
	\iffalse
\documentclass[12pt]{article}
\usepackage{graphicx}
\usepackage{amsmath}
\usepackage{mathtools}
\usepackage{gensymb}

\newcommand{\mydet}[1]{\ensuremath{\begin{vmatrix}#1\end{vmatrix}}}
\providecommand{\brak}[1]{\ensuremath{\left(#1\right)}}
\providecommand{\norm}[1]{\left\lVert#1\right\rVert}
\newcommand{\solution}{\noindent \textbf{Solution: }}
\newcommand{\myvec}[1]{\ensuremath{\begin{pmatrix}#1\end{pmatrix}}}
\let\vec\mathbf

\begin{document}
\begin{center}
\textbf\large{CHAPTER-7 \\ COORDINATE GEOMETRY}
\end{center}
\section*{Excercise 7.2}

1. Find the coordinates of the point which divides the join $\vec(-1,7) \text{ and } \vec(4,-3)$ in the ratio 2:3 :
\\
\\
\solution\\		
\fi
The coordinates and ratio are given as
\begin{align}
\vec{P}=\myvec{-1\\7\\},
\vec{Q}=\myvec{4\\-3\\},
n=\frac{3}{2}
\end{align}
Using section formula
\begin{align}
\vec{R}&=\frac{\vec{Q}+n\vec{P}}{1+n}\\
&=\frac{1}{1+\frac{3}{2}}  \myvec{\myvec{
4\\
-3\\
}
  +
   \frac{3}{2}\myvec{
-1\\
7\\
}}\\
&=\myvec{
1\\
3
}
\end{align}
See Fig. 
\ref{fig:chapters/10/7/2/1/Fig}
\begin{figure}[!h]
\begin{center}
   \includegraphics[width=\columnwidth]{chapters/10/7/2/1/figs/linefig.png}
\end{center}
\caption{}
\label{fig:chapters/10/7/2/1/Fig}
\end{figure}


\item Find the coordinates of the points of trisection of the line segment joining $(4,-1) \text{ and } (-2,3)$.
	\\
		\solution
	\begin{enumerate}[label=\thesection.\arabic*,ref=\thesection.\theenumi]
\numberwithin{equation}{enumi}
\numberwithin{figure}{enumi}
\numberwithin{table}{enumi}

\item Find the coordinates of the point which divides the join of $(-1,7) \text{ and } (4,-3)$ in the ratio 2:3.
	\\
		\solution
	\iffalse
\documentclass[12pt]{article}
\usepackage{graphicx}
\usepackage{amsmath}
\usepackage{mathtools}
\usepackage{gensymb}

\newcommand{\mydet}[1]{\ensuremath{\begin{vmatrix}#1\end{vmatrix}}}
\providecommand{\brak}[1]{\ensuremath{\left(#1\right)}}
\providecommand{\norm}[1]{\left\lVert#1\right\rVert}
\newcommand{\solution}{\noindent \textbf{Solution: }}
\newcommand{\myvec}[1]{\ensuremath{\begin{pmatrix}#1\end{pmatrix}}}
\let\vec\mathbf

\begin{document}
\begin{center}
\textbf\large{CHAPTER-7 \\ COORDINATE GEOMETRY}
\end{center}
\section*{Excercise 7.2}

1. Find the coordinates of the point which divides the join $\vec(-1,7) \text{ and } \vec(4,-3)$ in the ratio 2:3 :
\\
\\
\solution\\		
\fi
The coordinates and ratio are given as
\begin{align}
\vec{P}=\myvec{-1\\7\\},
\vec{Q}=\myvec{4\\-3\\},
n=\frac{3}{2}
\end{align}
Using section formula
\begin{align}
\vec{R}&=\frac{\vec{Q}+n\vec{P}}{1+n}\\
&=\frac{1}{1+\frac{3}{2}}  \myvec{\myvec{
4\\
-3\\
}
  +
   \frac{3}{2}\myvec{
-1\\
7\\
}}\\
&=\myvec{
1\\
3
}
\end{align}
See Fig. 
\ref{fig:chapters/10/7/2/1/Fig}
\begin{figure}[!h]
\begin{center}
   \includegraphics[width=\columnwidth]{chapters/10/7/2/1/figs/linefig.png}
\end{center}
\caption{}
\label{fig:chapters/10/7/2/1/Fig}
\end{figure}


\item Find the coordinates of the points of trisection of the line segment joining $(4,-1) \text{ and } (-2,3)$.
	\\
		\solution
	\begin{enumerate}[label=\thesection.\arabic*,ref=\thesection.\theenumi]
\numberwithin{equation}{enumi}
\numberwithin{figure}{enumi}
\numberwithin{table}{enumi}

\item Find the coordinates of the point which divides the join of $(-1,7) \text{ and } (4,-3)$ in the ratio 2:3.
	\\
		\solution
	\input{chapters/10/7/2/1/section.tex}
\item Find the coordinates of the points of trisection of the line segment joining $(4,-1) \text{ and } (-2,3)$.
	\\
		\solution
	\input{chapters/10/7/2/2/section.tex}
\item
	\iffalse
\item To conduct Sports Day activities, in your rectangular shaped school                   
ground ABCD, lines have 
drawn with chalk powder at a                 
distance of 1m each. 100 flower pots have been placed at a distance of 1m 
from each other along AD, as shown 
in Fig. 7.12. Niharika runs $ \frac {1}{4} $th the 
distance AD on the 2nd line and 
posts a green flag. Preet runs $ \frac {1}{5} $th 
the distance AD on the eighth line 
and posts a red flag. What is the 
distance between both the flags? If 
Rashmi has to post a blue flag exactly 
halfway between the line segment 
joining the two flags, where should 
she post her flag?
\begin{figure}[h!]
  \centering
  \includegraphics[width=\columnwidth]{sc.png}
  \caption{}
\label{fig:10/7/12Fig1}
\end{figure}               
\fi
      
\item Find the ratio in which the line segment joining the points $(-3,10) \text{ and } (6,-8)$ $\text{ is divided by } (-1,6)$.
	\\
		\solution
	\input{chapters/10/7/2/4/section.tex}
\item Find the ratio in which the line segment joining $A(1,-5) \text{ and } B(-4,5)$ $\text{is divided by the x-axis}$. Also find the coordinates of the point of division.
\item If $(1,2), (4,y), (x,6), (3,5)$ are the vertices of a parallelogram taken in order, find x and y.
	\\
		\solution
	\input{chapters/10/7/2/6/para1.tex}
\item Find the coordinates of a point A, where AB is the diameter of a circle whose centre is $(2,-3) \text{ and }$ B is $(1,4)$.
	\\
		\solution
	\input{chapters/10/7/2/7/section.tex}
\item If A \text{ and } B are $(-2,-2) \text{ and } (2,-4)$, respectively, find the coordinates of P such that AP= $\frac {3}{7}$AB $\text{ and }$ P lies on the line segment AB.
	\\
		\solution
	\input{chapters/10/7/2/8/section.tex}
\item Find the coordinates of the points which divide the line segment joining $A(-2,2) \text{ and } B(2,8)$ into four equal parts.
	\\
		\solution
	\input{chapters/10/7/2/9/section.tex}
\item Find the area of a rhombus if its vertices are $(3,0), (4,5), (-1,4) \text{ and } (-2,-1)$ taken in order. [$\vec{Hint}$ : Area of rhombus =$\frac {1}{2}$(product of its diagonals)]
	\\
		\solution
	\input{chapters/10/7/2/10/cross.tex}
\item Find the position vector of a point R which divides the line joining two points $\vec{P}$
and $\vec{Q}$ whose position vectors are $\hat{i}+2\hat{j}-\hat{k}$ and $-\hat{i}+\hat{j}+\hat{k}$ respectively, in the
ratio 2 : 1
\begin{enumerate}
    \item  internally
    \item  externally
\end{enumerate}
\solution
		\input{chapters/12/10/2/15/section.tex}
\item Find the position vector of the mid point of the vector joining the points $\vec{P}$(2, 3, 4)
and $\vec{Q}$(4, 1, –2).
\\
\solution
		\input{chapters/12/10/2/16/section.tex}
\item Determine the ratio in which the line $2x+y  - 4=0$ divides the line segment joining the points $\vec{A}(2, - 2)$  and  $\vec{B}(3, 7)$.
\\
\solution
	\input{chapters/10/7/4/1/section.tex}
\item Let $\vec{A}(4, 2), \vec{B}(6, 5)$  and $ \vec{C}(1, 4)$ be the vertices of $\triangle ABC$.
\begin{enumerate}
\item The median from $\vec{A}$ meets $BC$ at $\vec{D}$. Find the coordinates of the point $\vec{D}$.
\item Find the coordinates of the point $\vec{P}$ on $AD$ such that $AP : PD = 2 : 1$.
\item Find the coordinates of points $\vec{Q}$ and $\vec{R}$ on medians $BE$ and $CF$ respectively such that $BQ : QE = 2 : 1$  and  $CR : RF = 2 : 1$.
\item What do you observe?
\item If $\vec{A}, \vec{B}$ and $\vec{C}$  are the vertices of $\triangle ABC$, find the coordinates of the centroid of the triangle.
\end{enumerate}
\solution
	\input{chapters/10/7/4/7/section.tex}
\item Find the slope of a line, which passes through the origin and the mid point of the line segment joining the points $\vec{P}$(0,-4) and $\vec{B}$(8,0).
\label{chapters/11/10/1/5}
\input{chapters/11/10/1/5/matrix.tex}
\item Find the position vector of a point R which divides the line joining two points P and Q whose position vectors are $(2\vec{a}+\vec{b})$ and $(\vec{a}-3\vec{b})$
externally in the ratio 1 : 2. Also, show that P is the mid point of the line segment RQ.\\
	\solution
%		\input{chapters/12/10/5/9/section.tex}

\end{enumerate}


\item
	\iffalse
\item To conduct Sports Day activities, in your rectangular shaped school                   
ground ABCD, lines have 
drawn with chalk powder at a                 
distance of 1m each. 100 flower pots have been placed at a distance of 1m 
from each other along AD, as shown 
in Fig. 7.12. Niharika runs $ \frac {1}{4} $th the 
distance AD on the 2nd line and 
posts a green flag. Preet runs $ \frac {1}{5} $th 
the distance AD on the eighth line 
and posts a red flag. What is the 
distance between both the flags? If 
Rashmi has to post a blue flag exactly 
halfway between the line segment 
joining the two flags, where should 
she post her flag?
\begin{figure}[h!]
  \centering
  \includegraphics[width=\columnwidth]{sc.png}
  \caption{}
\label{fig:10/7/12Fig1}
\end{figure}               
\fi
      
\item Find the ratio in which the line segment joining the points $(-3,10) \text{ and } (6,-8)$ $\text{ is divided by } (-1,6)$.
	\\
		\solution
	\iffalse
\documentclass[12pt]{article}
\usepackage{graphicx}
%\documentclass[journal,12pt,twocolumn]{IEEEtran}
\usepackage[none]{hyphenat}
\usepackage{graphicx}
\usepackage{listings}
\usepackage[english]{babel}
\usepackage{graphicx}
\usepackage{caption} 
\usepackage{hyperref}
\usepackage{booktabs}
\def\inputGnumericTable{}
\usepackage{color}                                            %%
    \usepackage{array}                                            %%
    \usepackage{longtable}                                        %%
    \usepackage{calc}                                             %%
    \usepackage{multirow}                                         %%
    \usepackage{hhline}                                           %%
    \usepackage{ifthen}
\usepackage{array}
\usepackage{amsmath}   % for having text in math mode
\usepackage{listings}
\lstset{
language=tex,
frame=single, 
breaklines=true
}
  
%Following 2 lines were added to remove the blank page at the beginning
\usepackage{atbegshi}% http://ctan.org/pkg/atbegshi
\AtBeginDocument{\AtBeginShipoutNext{\AtBeginShipoutDiscard}}
%
%New macro definitions
\newcommand{\mydet}[1]{\ensuremath{\begin{vmatrix}#1\end{vmatrix}}}
\providecommand{\brak}[1]{\ensuremath{\left(#1\right)}}
\providecommand{\norm}[1]{\left\lVert#1\right\rVert}
\newcommand{\solution}{\noindent \textbf{Solution: }}
\newcommand{\myvec}[1]{\ensuremath{\begin{pmatrix}#1\end{pmatrix}}}
\let\vec\mathbf
\begin{document}
\begin{center}
\title{\textbf{Coordinate Geometry}}
\date{\vspace{-5ex}} %Not to print date automatically
\maketitle
\end{center}
\setcounter{page}{1}
\section*{10$^{th}$ Maths - Chapter 7}
This is Problem-4 from Exercise 7.2
\begin{enumerate}
\item Find the ratio in which the line segement joining the points $\myvec{-3 \\ 10}$ and $\myvec{6\\-8}$ is divided by $\myvec{-1\\6}$.\\
\solution \\
\fi
		The input parameters for this problem are available in Table \eqref{tab:10/7/2/4-1}.
\begin{table}[ht!]
\input{chapters/10/7/2/4/tables/table.tex}
\caption{}
\label{tab:10/7/2/4-1} 
\end{table}
Using section formula,
\begin{align}
         \vec{R} &=\frac{\vec{Q}+n\vec{P}}{1+n}\label{eq:chapters/10/7/2/4/1}
\end{align}
Substituting the values of $\vec{P},\vec{Q}$ and $\vec{R}$ in \eqref{eq:chapters/10/7/2/4/1}
\begin{align}
         \myvec{-1\\6} &=\frac{{\myvec{-3\\10}+n\myvec{6\\-8}}}{1+n}\\
 &=\frac{1}{1+n}\brak{{\myvec{-3\\10}+n\myvec{6\\-8}}} \\
 &=\frac{1}{1+n}\myvec{-3+6n\\10-8n} \label{eq:chapters/10/7/2/4/4}
\end{align}
Simplifying \eqref{eq:chapters/10/7/2/4/4} yeilds,
\begin{align}
          -1 &=\frac{-3+6n}{1+n}\\
\implies          n &=\frac{2}{7}
\end{align}
Also,
\begin{align}
          6 &=\frac{10-8n}{1+n}\\
    \implies      n &=\frac{2}{7}
\end{align}
Hence the desired ratio is $\dfrac{2}{7}$.  
\begin{figure}[!h]
 \begin{center}
  \includegraphics[width=\columnwidth]{chapters/10/7/2/4/figs/fig.png}
 \end{center}
\caption{}
\label{fig:10/7/2/4Fig1}
\end{figure}

\item Find the ratio in which the line segment joining $A(1,-5) \text{ and } B(-4,5)$ $\text{is divided by the x-axis}$. Also find the coordinates of the point of division.
\item If $(1,2), (4,y), (x,6), (3,5)$ are the vertices of a parallelogram taken in order, find x and y.
	\\
		\solution
	\iffalse
\documentclass[12pt]{article}
\usepackage{graphicx}
%\documentclass[journal,12pt,twocolumn]{IEEEtran}
\def\inputGnumericTable{}
\usepackage{color}                                            %%
    \usepackage{array}                                            %%
    \usepackage{longtable}                                        %%
    \usepackage{calc}                                             %%
    \usepackage{multirow}                                         %%
    \usepackage{hhline}                                           %%
    \usepackage{ifthen}
\usepackage[none]{hyphenat}
\usepackage{graphicx}
\usepackage{listings}
\usepackage[english]{babel}
\usepackage{graphicx}
\usepackage{caption} 
\usepackage{hyperref}
\usepackage{booktabs}
\usepackage{array}
\usepackage{amsmath}   % for having text in math mode
\usepackage{listings}
\lstset{
  frame=single,
  breaklines=true
}
  
%Following 2 lines were added to remove the blank page at the beginning
\usepackage{atbegshi}% http://ctan.org/pkg/atbegshi
\AtBeginDocument{\AtBeginShipoutNext{\AtBeginShipoutDiscard}}
%


%New macro definitions
\newcommand{\mydet}[1]{\ensuremath{\begin{vmatrix}#1\end{vmatrix}}}
\providecommand{\brak}[1]{\ensuremath{\left(#1\right)}}
\providecommand{\norm}[1]{\left\lVert#1\right\rVert}
\newcommand{\solution}{\noindent \textbf{Solution: }}
\newcommand{\myvec}[1]{\ensuremath{\begin{pmatrix}#1\end{pmatrix}}}
\let\vec\mathbf

\begin{document}

\begin{center}
\title{\textbf{Properties of Parallelegram}}
\date{\vspace{-5ex}} %Not to print date automatically
\maketitle
\end{center}

\setcounter{page}{1}

\section{10$^{th}$ Maths - Chapter 7}

This is Problem-6 from Exercise 7.2

\begin{enumerate}
\item If $\vec{A}(1, 2),\vec{B}(4, x),\vec{C}(y, 6) \text{and } \vec{D}(3, 5)$ are the vertices of a parallelogram taken in order,find x and y.
\end{enumerate}
\fi

The input parameters for this problem are available in
\ref{table:chapters/10/7/2/6/tables/}.	
\begin{table}[!ht]
	\centering
	\input{chapters/10/7/2/6/tables/table.tex}
\caption{}
\label{table:chapters/10/7/2/6/tables/}	
\end{table}
From the given information,
\begin{align}
  \label{eq:chapters/10/7/2/6/tables/det2f}
	\vec{B}-\vec{A} &= \myvec{4 \\y } - \myvec{1 \\2 }  = \myvec{3 \\y-2 }\\
	\vec{C}-\vec{D} &= \myvec{x \\6 } - \myvec{3 \\5 }  = \myvec{x-3 \\1}
\end{align}
Since $ABCD$ is a parallellogram,
\begin{align}
	\myvec{3\\y-2}&=\myvec{x-3\\1}\\
	\implies x&=6 ,y=3
\end{align}
Fig. \ref{fig:chapters/10/7/2/6/Fig3}
provides a verification.
\begin{figure}[h!]
	\begin{center}
  \includegraphics[width=\columnwidth]{chapters/10/7/2/6/figs/para.pdf}
	\end{center}
\caption{}
\label{fig:chapters/10/7/2/6/Fig3}
\end{figure}


\item Find the coordinates of a point A, where AB is the diameter of a circle whose centre is $(2,-3) \text{ and }$ B is $(1,4)$.
	\\
		\solution
	\iffalse
\documentclass[12pt]{article}
\usepackage{graphicx}
\usepackage{amsmath}
\usepackage{mathtools}
\usepackage{gensymb}

\newcommand{\mydet}[1]{\ensuremath{\begin{vmatrix}#1\end{vmatrix}}}
\providecommand{\brak}[1]{\ensuremath{\left(#1\right)}}
\providecommand{\norm}[1]{\left\lVert#1\right\rVert}
\newcommand{\solution}{\noindent \textbf{Solution: }}
\newcommand{\myvec}[1]{\ensuremath{\begin{pmatrix}#1\end{pmatrix}}}
\let\vec\mathbf

\begin{document}
\begin{center}
\section*{CHAPTER 7 - COORDINATE GEOMETRY}

\end{center}
\section*{Excercise 7.2}

Q7.Find the coordinates of point $\vec{A}$, where AB is the diameter of a circle where the center is (2,-3) and $\vec{B}$ is the point (1,4):

\solution
\begin{enumerate}
\item The coordinates $\vec{B}$ and center $\vec{C}$ are given, where:
	\fi
	Let
	\begin{align}
	\vec{B} = \myvec{
		1\\
	    4\\
		},
	\vec{C} = \myvec{
	    2\\
	   -3\\
		}
	\end{align}
	\iffalse
Let us assume the coordinates of $\vec{A}$. Now, $\vec{C}$ is the center which is midpoint of line AB and $\vec{B}$ is one of the coordinate of diameter AB of a circle.
	\fi	
Hence,	
	\begin{align}
	\vec{C} &= \frac{\vec{A+B}}{2} \\
\implies	2\vec{C} &= \vec{A}+\vec{B} \\
		\text{or, }	\vec{A} &= 2\vec{C}-\vec{B} \\
	 &= \myvec{3\\-10\\}	
	\end{align}       
	See Fig. 
\ref{fig:chapters/10/7/2/7Fig}.
\begin{figure}[!h]
\begin{center}	
	\includegraphics[width=\columnwidth]{chapters/10/7/2/7/figs/Vector1.png}
\end{center}
\caption{}
\label{fig:chapters/10/7/2/7Fig}
\end{figure}
	

\item If A \text{ and } B are $(-2,-2) \text{ and } (2,-4)$, respectively, find the coordinates of P such that AP= $\frac {3}{7}$AB $\text{ and }$ P lies on the line segment AB.
	\\
		\solution
	\iffalse
\documentclass[journal,10pt,twocolumn]{article}
\usepackage{graphicx}
\usepackage[none]{hyphenat}
\usepackage{graphicx}
\usepackage{listings}
\usepackage[english]{babel}
\usepackage{graphicx}
\usepackage{caption} 
\usepackage{booktabs}
\usepackage{array}
\usepackage{amssymb} % for \because
\usepackage{amsmath}   % for having text in math mode
\usepackage{extarrows} % for Row operations arrows
\usepackage{listings}
\usepackage[utf8]{inputenc}
\lstset{
  frame=single,
  breaklines=true
}
\usepackage{hyperref}
  
%Following 2 lines were added to remove the blank page at the beginning
\usepackage{atbegshi}% http://ctan.org/pkg/atbegshi
\AtBeginDocument{\AtBeginShipoutNext{\AtBeginShipoutDiscard}}


%New macro definitions
\newcommand{\mydet}[1]{\ensuremath{\begin{vmatrix}#1\end{vmatrix}}}
\providecommand{\brak}[1]{\ensuremath{\left(#1\right)}}
\newcommand{\solution}{\noindent \textbf{Solution: }}
\newcommand{\myvec}[1]{\ensuremath{\begin{pmatrix}#1\end{pmatrix}}}
\providecommand{\norm}[1]{\left\lVert#1\right\rVert}
\providecommand{\abs}[1]{\left\vert#1\right\vert}
\let\vec\mathbf

\begin{document}

\begin{center}
\title{\textbf{VECTORS}}
\date{\vspace{-5ex}} %Not to print date automatically
\maketitle
\end{center}

\section{10$^{th}$ Maths - EXERCISE-7.2}

\begin{enumerate}
\item If A and B are $(– 2, – 2)\text{ and }(2, – 4)$, respectively, find the coordinates of P such that $AP =\frac{3}{7}AB$ and P lies on the line segment AB. 

\section{SOLUTION}
Given points are
\begin{align}
\vec{A}=\myvec{-2\\ -2} ,
\vec{B}=\myvec{2\\ -4}
\end{align}
The equation of the formula is
\fi
Using section formula, 
\begin{align}
\vec{P}&=\frac{\vec{A}+n\vec{B}}{1+n}
\end{align}
where
\begin{align}
	n =\frac{3}{4}
\end{align}
Thus,
\begin{align}
\vec{P}&=\frac{1}{1+\frac{3}{4}}\brak{\myvec{-2\\-2}+\frac{3}{4}\myvec{2\\-4}}\\
&=\myvec{\frac{-2}{7}\\[1pt] \frac{-20}{7}}
\end{align}
See Fig. 
   \ref{fig:chapters/10/7/2/8/vec.png}
\begin{figure}
   \centering 
 \includegraphics[width=\columnwidth]{chapters/10/7/2/8/figs/vec.png}
   \caption{}
   \label{fig:chapters/10/7/2/8/vec.png}
   \end{figure}

\item Find the coordinates of the points which divide the line segment joining $A(-2,2) \text{ and } B(2,8)$ into four equal parts.
	\\
		\solution
	\begin{enumerate}[label=\thesection.\arabic*,ref=\thesection.\theenumi]
\numberwithin{equation}{enumi}
\numberwithin{figure}{enumi}
\numberwithin{table}{enumi}

\item Find the coordinates of the point which divides the join of $(-1,7) \text{ and } (4,-3)$ in the ratio 2:3.
	\\
		\solution
	\input{chapters/10/7/2/1/section.tex}
\item Find the coordinates of the points of trisection of the line segment joining $(4,-1) \text{ and } (-2,3)$.
	\\
		\solution
	\input{chapters/10/7/2/2/section.tex}
\item
	\iffalse
\item To conduct Sports Day activities, in your rectangular shaped school                   
ground ABCD, lines have 
drawn with chalk powder at a                 
distance of 1m each. 100 flower pots have been placed at a distance of 1m 
from each other along AD, as shown 
in Fig. 7.12. Niharika runs $ \frac {1}{4} $th the 
distance AD on the 2nd line and 
posts a green flag. Preet runs $ \frac {1}{5} $th 
the distance AD on the eighth line 
and posts a red flag. What is the 
distance between both the flags? If 
Rashmi has to post a blue flag exactly 
halfway between the line segment 
joining the two flags, where should 
she post her flag?
\begin{figure}[h!]
  \centering
  \includegraphics[width=\columnwidth]{sc.png}
  \caption{}
\label{fig:10/7/12Fig1}
\end{figure}               
\fi
      
\item Find the ratio in which the line segment joining the points $(-3,10) \text{ and } (6,-8)$ $\text{ is divided by } (-1,6)$.
	\\
		\solution
	\input{chapters/10/7/2/4/section.tex}
\item Find the ratio in which the line segment joining $A(1,-5) \text{ and } B(-4,5)$ $\text{is divided by the x-axis}$. Also find the coordinates of the point of division.
\item If $(1,2), (4,y), (x,6), (3,5)$ are the vertices of a parallelogram taken in order, find x and y.
	\\
		\solution
	\input{chapters/10/7/2/6/para1.tex}
\item Find the coordinates of a point A, where AB is the diameter of a circle whose centre is $(2,-3) \text{ and }$ B is $(1,4)$.
	\\
		\solution
	\input{chapters/10/7/2/7/section.tex}
\item If A \text{ and } B are $(-2,-2) \text{ and } (2,-4)$, respectively, find the coordinates of P such that AP= $\frac {3}{7}$AB $\text{ and }$ P lies on the line segment AB.
	\\
		\solution
	\input{chapters/10/7/2/8/section.tex}
\item Find the coordinates of the points which divide the line segment joining $A(-2,2) \text{ and } B(2,8)$ into four equal parts.
	\\
		\solution
	\input{chapters/10/7/2/9/section.tex}
\item Find the area of a rhombus if its vertices are $(3,0), (4,5), (-1,4) \text{ and } (-2,-1)$ taken in order. [$\vec{Hint}$ : Area of rhombus =$\frac {1}{2}$(product of its diagonals)]
	\\
		\solution
	\input{chapters/10/7/2/10/cross.tex}
\item Find the position vector of a point R which divides the line joining two points $\vec{P}$
and $\vec{Q}$ whose position vectors are $\hat{i}+2\hat{j}-\hat{k}$ and $-\hat{i}+\hat{j}+\hat{k}$ respectively, in the
ratio 2 : 1
\begin{enumerate}
    \item  internally
    \item  externally
\end{enumerate}
\solution
		\input{chapters/12/10/2/15/section.tex}
\item Find the position vector of the mid point of the vector joining the points $\vec{P}$(2, 3, 4)
and $\vec{Q}$(4, 1, –2).
\\
\solution
		\input{chapters/12/10/2/16/section.tex}
\item Determine the ratio in which the line $2x+y  - 4=0$ divides the line segment joining the points $\vec{A}(2, - 2)$  and  $\vec{B}(3, 7)$.
\\
\solution
	\input{chapters/10/7/4/1/section.tex}
\item Let $\vec{A}(4, 2), \vec{B}(6, 5)$  and $ \vec{C}(1, 4)$ be the vertices of $\triangle ABC$.
\begin{enumerate}
\item The median from $\vec{A}$ meets $BC$ at $\vec{D}$. Find the coordinates of the point $\vec{D}$.
\item Find the coordinates of the point $\vec{P}$ on $AD$ such that $AP : PD = 2 : 1$.
\item Find the coordinates of points $\vec{Q}$ and $\vec{R}$ on medians $BE$ and $CF$ respectively such that $BQ : QE = 2 : 1$  and  $CR : RF = 2 : 1$.
\item What do you observe?
\item If $\vec{A}, \vec{B}$ and $\vec{C}$  are the vertices of $\triangle ABC$, find the coordinates of the centroid of the triangle.
\end{enumerate}
\solution
	\input{chapters/10/7/4/7/section.tex}
\item Find the slope of a line, which passes through the origin and the mid point of the line segment joining the points $\vec{P}$(0,-4) and $\vec{B}$(8,0).
\label{chapters/11/10/1/5}
\input{chapters/11/10/1/5/matrix.tex}
\item Find the position vector of a point R which divides the line joining two points P and Q whose position vectors are $(2\vec{a}+\vec{b})$ and $(\vec{a}-3\vec{b})$
externally in the ratio 1 : 2. Also, show that P is the mid point of the line segment RQ.\\
	\solution
%		\input{chapters/12/10/5/9/section.tex}

\end{enumerate}


\item Find the area of a rhombus if its vertices are $(3,0), (4,5), (-1,4) \text{ and } (-2,-1)$ taken in order. [$\vec{Hint}$ : Area of rhombus =$\frac {1}{2}$(product of its diagonals)]
	\\
		\solution
	\iffalse
\documentclass[12pt]{article}
\usepackage{graphicx}
%\documentclass[journal,12pt,twocolumn]{IEEEtran}
\usepackage[none]{hyphenat}
\usepackage{graphicx}
\usepackage{listings}
\usepackage[english]{babel}
\usepackage{graphicx}
\usepackage{caption} 
\usepackage{hyperref}
\usepackage{booktabs}
\def\inputGnumericTable{}
\usepackage{color}                                            %%
    \usepackage{array}                                            %%
    \usepackage{longtable}                                        %%
    \usepackage{calc}                                             %%
    \usepackage{multirow}                                         %%
    \usepackage{hhline}                                           %%
    \usepackage{ifthen}
\usepackage{array}
\usepackage{amsmath}   % for having text in math mode
\usepackage{listings}
\lstset{
language=tex,
frame=single, 
breaklines=true
}
  
%Following 2 lines were added to remove the blank page at the beginning
\usepackage{atbegshi}% http://ctan.org/pkg/atbegshi
\AtBeginDocument{\AtBeginShipoutNext{\AtBeginShipoutDiscard}}
%


%New macro definitions
\newcommand{\mydet}[1]{\ensuremath{\begin{vmatrix}#1\end{vmatrix}}}
\providecommand{\brak}[1]{\ensuremath{\left(#1\right)}}
\providecommand{\norm}[1]{\left\lVert#1\right\rVert}
\newcommand{\solution}{\noindent \textbf{Solution: }}
\newcommand{\myvec}[1]{\ensuremath{\begin{pmatrix}#1\end{pmatrix}}}
\let\vec\mathbf

\begin{document}

\begin{center}
\title{\textbf{Coordinate Geometry}}
\date{\vspace{-5ex}} %Not to print date automatically
\maketitle
\end{center}

\setcounter{page}{1}



\begin{enumerate}

\item\textbf{Problem statement :} Find the area of a rhombus of its vertices are $\myvec{3 ,0}$, $\myvec{4 ,5}$, $\myvec{-1 ,4}$ and $\myvec{-2 ,-1}$taken in order

\solution \\
\fi
The input vertices for this problem are given as
	\begin{align}
	\vec{A} = \myvec{
		3\\
		0
		},
	\vec{B} = \myvec{
		4\\
		5
		},
        \vec{C} = \myvec{
		-1\\
		4
		},
        \vec{D} = \myvec{
		-2\\
		-1
		}
	\end{align}
Since		
\begin{align}
 \vec{A-D}= \myvec{3 \\ 0} - \myvec{-2 \\-1}= \myvec{5\\1}
 \\
  \vec{B-A}= \myvec{4 \\ 5} - \myvec{3 \\0}= \myvec{1\\5}
\end{align}
the area of the rhombus is
\begin{align}
                \norm{\myvec{\vec{A-D}}\times \myvec{\vec{B-A}}}=\mydet{5 & 1\\1 & 5} = 24
\end{align}
See Fig. 
\ref{fig:chapters/10/7/2/10/gFig1}.
\begin{figure}[!h]
 \begin{center}
  \includegraphics[width=\columnwidth]{chapters/10/7/2/10/figs/fig.pdf}
 \end{center}
\caption{}
\label{fig:chapters/10/7/2/10/gFig1}
\end{figure}

\item Find the position vector of a point R which divides the line joining two points $\vec{P}$
and $\vec{Q}$ whose position vectors are $\hat{i}+2\hat{j}-\hat{k}$ and $-\hat{i}+\hat{j}+\hat{k}$ respectively, in the
ratio 2 : 1
\begin{enumerate}
    \item  internally
    \item  externally
\end{enumerate}
\solution
		\begin{enumerate}[label=\thesection.\arabic*,ref=\thesection.\theenumi]
\numberwithin{equation}{enumi}
\numberwithin{figure}{enumi}
\numberwithin{table}{enumi}

\item Find the coordinates of the point which divides the join of $(-1,7) \text{ and } (4,-3)$ in the ratio 2:3.
	\\
		\solution
	\input{chapters/10/7/2/1/section.tex}
\item Find the coordinates of the points of trisection of the line segment joining $(4,-1) \text{ and } (-2,3)$.
	\\
		\solution
	\input{chapters/10/7/2/2/section.tex}
\item
	\iffalse
\item To conduct Sports Day activities, in your rectangular shaped school                   
ground ABCD, lines have 
drawn with chalk powder at a                 
distance of 1m each. 100 flower pots have been placed at a distance of 1m 
from each other along AD, as shown 
in Fig. 7.12. Niharika runs $ \frac {1}{4} $th the 
distance AD on the 2nd line and 
posts a green flag. Preet runs $ \frac {1}{5} $th 
the distance AD on the eighth line 
and posts a red flag. What is the 
distance between both the flags? If 
Rashmi has to post a blue flag exactly 
halfway between the line segment 
joining the two flags, where should 
she post her flag?
\begin{figure}[h!]
  \centering
  \includegraphics[width=\columnwidth]{sc.png}
  \caption{}
\label{fig:10/7/12Fig1}
\end{figure}               
\fi
      
\item Find the ratio in which the line segment joining the points $(-3,10) \text{ and } (6,-8)$ $\text{ is divided by } (-1,6)$.
	\\
		\solution
	\input{chapters/10/7/2/4/section.tex}
\item Find the ratio in which the line segment joining $A(1,-5) \text{ and } B(-4,5)$ $\text{is divided by the x-axis}$. Also find the coordinates of the point of division.
\item If $(1,2), (4,y), (x,6), (3,5)$ are the vertices of a parallelogram taken in order, find x and y.
	\\
		\solution
	\input{chapters/10/7/2/6/para1.tex}
\item Find the coordinates of a point A, where AB is the diameter of a circle whose centre is $(2,-3) \text{ and }$ B is $(1,4)$.
	\\
		\solution
	\input{chapters/10/7/2/7/section.tex}
\item If A \text{ and } B are $(-2,-2) \text{ and } (2,-4)$, respectively, find the coordinates of P such that AP= $\frac {3}{7}$AB $\text{ and }$ P lies on the line segment AB.
	\\
		\solution
	\input{chapters/10/7/2/8/section.tex}
\item Find the coordinates of the points which divide the line segment joining $A(-2,2) \text{ and } B(2,8)$ into four equal parts.
	\\
		\solution
	\input{chapters/10/7/2/9/section.tex}
\item Find the area of a rhombus if its vertices are $(3,0), (4,5), (-1,4) \text{ and } (-2,-1)$ taken in order. [$\vec{Hint}$ : Area of rhombus =$\frac {1}{2}$(product of its diagonals)]
	\\
		\solution
	\input{chapters/10/7/2/10/cross.tex}
\item Find the position vector of a point R which divides the line joining two points $\vec{P}$
and $\vec{Q}$ whose position vectors are $\hat{i}+2\hat{j}-\hat{k}$ and $-\hat{i}+\hat{j}+\hat{k}$ respectively, in the
ratio 2 : 1
\begin{enumerate}
    \item  internally
    \item  externally
\end{enumerate}
\solution
		\input{chapters/12/10/2/15/section.tex}
\item Find the position vector of the mid point of the vector joining the points $\vec{P}$(2, 3, 4)
and $\vec{Q}$(4, 1, –2).
\\
\solution
		\input{chapters/12/10/2/16/section.tex}
\item Determine the ratio in which the line $2x+y  - 4=0$ divides the line segment joining the points $\vec{A}(2, - 2)$  and  $\vec{B}(3, 7)$.
\\
\solution
	\input{chapters/10/7/4/1/section.tex}
\item Let $\vec{A}(4, 2), \vec{B}(6, 5)$  and $ \vec{C}(1, 4)$ be the vertices of $\triangle ABC$.
\begin{enumerate}
\item The median from $\vec{A}$ meets $BC$ at $\vec{D}$. Find the coordinates of the point $\vec{D}$.
\item Find the coordinates of the point $\vec{P}$ on $AD$ such that $AP : PD = 2 : 1$.
\item Find the coordinates of points $\vec{Q}$ and $\vec{R}$ on medians $BE$ and $CF$ respectively such that $BQ : QE = 2 : 1$  and  $CR : RF = 2 : 1$.
\item What do you observe?
\item If $\vec{A}, \vec{B}$ and $\vec{C}$  are the vertices of $\triangle ABC$, find the coordinates of the centroid of the triangle.
\end{enumerate}
\solution
	\input{chapters/10/7/4/7/section.tex}
\item Find the slope of a line, which passes through the origin and the mid point of the line segment joining the points $\vec{P}$(0,-4) and $\vec{B}$(8,0).
\label{chapters/11/10/1/5}
\input{chapters/11/10/1/5/matrix.tex}
\item Find the position vector of a point R which divides the line joining two points P and Q whose position vectors are $(2\vec{a}+\vec{b})$ and $(\vec{a}-3\vec{b})$
externally in the ratio 1 : 2. Also, show that P is the mid point of the line segment RQ.\\
	\solution
%		\input{chapters/12/10/5/9/section.tex}

\end{enumerate}


\item Find the position vector of the mid point of the vector joining the points $\vec{P}$(2, 3, 4)
and $\vec{Q}$(4, 1, –2).
\\
\solution
		\begin{enumerate}[label=\thesection.\arabic*,ref=\thesection.\theenumi]
\numberwithin{equation}{enumi}
\numberwithin{figure}{enumi}
\numberwithin{table}{enumi}

\item Find the coordinates of the point which divides the join of $(-1,7) \text{ and } (4,-3)$ in the ratio 2:3.
	\\
		\solution
	\input{chapters/10/7/2/1/section.tex}
\item Find the coordinates of the points of trisection of the line segment joining $(4,-1) \text{ and } (-2,3)$.
	\\
		\solution
	\input{chapters/10/7/2/2/section.tex}
\item
	\iffalse
\item To conduct Sports Day activities, in your rectangular shaped school                   
ground ABCD, lines have 
drawn with chalk powder at a                 
distance of 1m each. 100 flower pots have been placed at a distance of 1m 
from each other along AD, as shown 
in Fig. 7.12. Niharika runs $ \frac {1}{4} $th the 
distance AD on the 2nd line and 
posts a green flag. Preet runs $ \frac {1}{5} $th 
the distance AD on the eighth line 
and posts a red flag. What is the 
distance between both the flags? If 
Rashmi has to post a blue flag exactly 
halfway between the line segment 
joining the two flags, where should 
she post her flag?
\begin{figure}[h!]
  \centering
  \includegraphics[width=\columnwidth]{sc.png}
  \caption{}
\label{fig:10/7/12Fig1}
\end{figure}               
\fi
      
\item Find the ratio in which the line segment joining the points $(-3,10) \text{ and } (6,-8)$ $\text{ is divided by } (-1,6)$.
	\\
		\solution
	\input{chapters/10/7/2/4/section.tex}
\item Find the ratio in which the line segment joining $A(1,-5) \text{ and } B(-4,5)$ $\text{is divided by the x-axis}$. Also find the coordinates of the point of division.
\item If $(1,2), (4,y), (x,6), (3,5)$ are the vertices of a parallelogram taken in order, find x and y.
	\\
		\solution
	\input{chapters/10/7/2/6/para1.tex}
\item Find the coordinates of a point A, where AB is the diameter of a circle whose centre is $(2,-3) \text{ and }$ B is $(1,4)$.
	\\
		\solution
	\input{chapters/10/7/2/7/section.tex}
\item If A \text{ and } B are $(-2,-2) \text{ and } (2,-4)$, respectively, find the coordinates of P such that AP= $\frac {3}{7}$AB $\text{ and }$ P lies on the line segment AB.
	\\
		\solution
	\input{chapters/10/7/2/8/section.tex}
\item Find the coordinates of the points which divide the line segment joining $A(-2,2) \text{ and } B(2,8)$ into four equal parts.
	\\
		\solution
	\input{chapters/10/7/2/9/section.tex}
\item Find the area of a rhombus if its vertices are $(3,0), (4,5), (-1,4) \text{ and } (-2,-1)$ taken in order. [$\vec{Hint}$ : Area of rhombus =$\frac {1}{2}$(product of its diagonals)]
	\\
		\solution
	\input{chapters/10/7/2/10/cross.tex}
\item Find the position vector of a point R which divides the line joining two points $\vec{P}$
and $\vec{Q}$ whose position vectors are $\hat{i}+2\hat{j}-\hat{k}$ and $-\hat{i}+\hat{j}+\hat{k}$ respectively, in the
ratio 2 : 1
\begin{enumerate}
    \item  internally
    \item  externally
\end{enumerate}
\solution
		\input{chapters/12/10/2/15/section.tex}
\item Find the position vector of the mid point of the vector joining the points $\vec{P}$(2, 3, 4)
and $\vec{Q}$(4, 1, –2).
\\
\solution
		\input{chapters/12/10/2/16/section.tex}
\item Determine the ratio in which the line $2x+y  - 4=0$ divides the line segment joining the points $\vec{A}(2, - 2)$  and  $\vec{B}(3, 7)$.
\\
\solution
	\input{chapters/10/7/4/1/section.tex}
\item Let $\vec{A}(4, 2), \vec{B}(6, 5)$  and $ \vec{C}(1, 4)$ be the vertices of $\triangle ABC$.
\begin{enumerate}
\item The median from $\vec{A}$ meets $BC$ at $\vec{D}$. Find the coordinates of the point $\vec{D}$.
\item Find the coordinates of the point $\vec{P}$ on $AD$ such that $AP : PD = 2 : 1$.
\item Find the coordinates of points $\vec{Q}$ and $\vec{R}$ on medians $BE$ and $CF$ respectively such that $BQ : QE = 2 : 1$  and  $CR : RF = 2 : 1$.
\item What do you observe?
\item If $\vec{A}, \vec{B}$ and $\vec{C}$  are the vertices of $\triangle ABC$, find the coordinates of the centroid of the triangle.
\end{enumerate}
\solution
	\input{chapters/10/7/4/7/section.tex}
\item Find the slope of a line, which passes through the origin and the mid point of the line segment joining the points $\vec{P}$(0,-4) and $\vec{B}$(8,0).
\label{chapters/11/10/1/5}
\input{chapters/11/10/1/5/matrix.tex}
\item Find the position vector of a point R which divides the line joining two points P and Q whose position vectors are $(2\vec{a}+\vec{b})$ and $(\vec{a}-3\vec{b})$
externally in the ratio 1 : 2. Also, show that P is the mid point of the line segment RQ.\\
	\solution
%		\input{chapters/12/10/5/9/section.tex}

\end{enumerate}


\item Determine the ratio in which the line $2x+y  - 4=0$ divides the line segment joining the points $\vec{A}(2, - 2)$  and  $\vec{B}(3, 7)$.
\\
\solution
	\iffalse
\documentclass[journal,12pt,twocolumn]{IEEEtran}
\usepackage{graphicx}
\graphicspath{{./chapters/10/7/4/1/figs/}}{}
\usepackage{amsmath,amssymb,amsfonts,amsthm}
\newcommand{\myvec}[1]{\ensuremath{\begin{pmatrix}#1\end{pmatrix}}}
\providecommand{\norm}[1]{\lVert#1\rVert}
\usepackage{listings}
\usepackage{watermark}
\usepackage{titlesec}
\usepackage{caption}
\let\vec\mathbf
\lstset{
frame=single, 
breaklines=true,
columns=fullflexible
}
\thiswatermark{\centering \put(0,-105.0){\includegraphics[scale=0.15]{/sdcard/IITH/vector/vectpr-4/chapters/10/7/4/1/figs/logo.png}} }
\title{\mytitle}
\title{
Assignment - Vector-4
}
\author{Surajit Sarkar}
\begin{document}
\maketitle
%\tableofcontents
\bigskip
\section{\textbf{Problem}}
Determine the ratio in which the line 2x+y–4=0 divides the line segment joining the points A(2,–2) and B(3,7).
\section{\textbf{Solution}}
\begin{table}[h]
    \centering
    \begin{tabular}{|c|c|}
       \hline
       \textbf{Symbol}&\textbf{Value}  \\
       \hline
	    $\vec{A}$ & $\myvec{2\\-2}$\\
        \hline
	    $\vec{B}$ & $\myvec{3\\7}$\\
        \hline
	    c&$4$\\
        \hline
       $\vec{n}$ & $\myvec{2\\1}$\\
       \hline
    \end{tabular}
    \caption{Parameters}
    \label{tab:my_label}
\end{table}
Given equation
\fi
The given equation can be expressed as
\begin{align}
    \myvec{2&1}\vec{x}&=4\\
\end{align}
Using section formula, the point of division 
\begin{align}
    \vec{P} = \frac{k\vec{B+A}}{k+1}
\end{align}
which upon substitution in the equation of a line yields
\begin{align}
    \implies\vec{n}^{\top}\myvec{\frac{k\vec{B+A}}{k+1}}&=c\\
    \implies k&=\frac{c-\vec{n}^{\top}\vec{A}}{\vec{n}^{\top}\vec{B}-c}\\
\end{align}
upon simplification.  Substituting numerical values, 
\begin{align}
    k=\frac{2}{9}
\end{align}
See Fig. 
\ref{fig:chapters/10/7/4/1vec}.
\begin{figure}[!h]
\centering
\includegraphics[width=\columnwidth]{chapters/10/7/4/1/figs/vec.pdf}
\caption{}
\label{fig:chapters/10/7/4/1vec}
\end{figure}


\item Let $\vec{A}(4, 2), \vec{B}(6, 5)$  and $ \vec{C}(1, 4)$ be the vertices of $\triangle ABC$.
\begin{enumerate}
\item The median from $\vec{A}$ meets $BC$ at $\vec{D}$. Find the coordinates of the point $\vec{D}$.
\item Find the coordinates of the point $\vec{P}$ on $AD$ such that $AP : PD = 2 : 1$.
\item Find the coordinates of points $\vec{Q}$ and $\vec{R}$ on medians $BE$ and $CF$ respectively such that $BQ : QE = 2 : 1$  and  $CR : RF = 2 : 1$.
\item What do you observe?
\item If $\vec{A}, \vec{B}$ and $\vec{C}$  are the vertices of $\triangle ABC$, find the coordinates of the centroid of the triangle.
\end{enumerate}
\solution
	\iffalse
\documentclass[12pt]{article}
\usepackage{graphicx}
\usepackage[none]{hyphenat}
\usepackage{graphicx}
\usepackage{listings}
\usepackage[english]{babel}
\usepackage{graphicx}
\usepackage{caption} 
\usepackage{booktabs}
\usepackage{array}
\usepackage{amssymb} % for \because
\usepackage{amsmath}   % for having text in math mode
\usepackage{extarrows} % for Row operations arrows
\usepackage{listings}
\usepackage[utf8]{inputenc}
\lstset{
  frame=single,
  breaklines=true
}
\usepackage{hyperref}
  
%Following 2 lines were added to remove the blank page at the beginning
\usepackage{atbegshi}% http://ctan.org/pkg/atbegshi
\AtBeginDocument{\AtBeginShipoutNext{\AtBeginShipoutDiscard}}


%New macro definitions
\newcommand{\mydet}[1]{\ensuremath{\begin{vmatrix}#1\end{vmatrix}}}
\providecommand{\brak}[1]{\ensuremath{\left(#1\right)}}
\newcommand{\solution}{\noindent \textbf{Solution: }}
\newcommand{\myvec}[1]{\ensuremath{\begin{pmatrix}#1\end{pmatrix}}}
\providecommand{\norm}[1]{\left\lVert#1\right\rVert}
\providecommand{\abs}[1]{\left\vert#1\right\vert}
\let\vec\mathbf

\begin{document}

\begin{center}
\title{\textbf{VECTORS}}
\date{\vspace{-5ex}} %Not to print date automatically
\maketitle
\end{center}

\section{10$^{th}$ Maths - EXERCISE-7.4}

Let A(4, 2), B(6, 5) and C(1, 4) be the vertices of $\triangle ABC$
\begin{enumerate}
\item The median from A meets BC at D. Find the coordinates of the point D.
\item Find the coordinates of the point P on AD such that $AP : PD = 2 : 1$
\item Find the coordinates of points Q and R on medians BE and CF respectively such
that $BQ : QE = 2 : 1 \text{and} CR : RF = 2 : 1.$
\item What do yo observe?
\item If $A(x_1, y_1), B(x_2, y_2) \text{and} C(x_3, y_3)$ are the vertices of $\triangle ABC$, find the coordinates of the centroid of the triangle.
\end{enumerate}

Given points are
\begin{align}
\vec{A}=\myvec{4\\ 2} ,
\vec{B}=\myvec{6\\ 5} ,
\vec{C}=\myvec{1\\ 4}
\end{align}
\fi

\begin{enumerate}
\item 
\begin{align}
\vec{D}&=\frac{\vec{B}+\vec{C}}{2}\\
&=\myvec{\frac{7}{2}\\[2pt] \frac{9}{2}}\\
\vec{E}&=\frac{\vec{A}+\vec{C}}{2}\\
&=\myvec{\frac{5}{2}\\ 3}\\
\vec{F}&=\frac{\vec{A}+\vec{B}}{2}\\
&=\myvec{5\\ \frac{7}{2}}
\end{align}

\item 
	For
$n=2$,
\begin{align}
\vec{P}&=\frac{1}{1+n}\brak{\myvec{\vec{A}+n\vec{D}}}\\
&=\frac{1}{3}\myvec{11\\11}
\end{align}

\item 
\begin{align}
\vec{Q}&=\frac{1}{1+n}\brak{\myvec{\vec{B}+n\vec{E}}}\\
&=\frac{1}{3}\myvec{11\\11}\\
\vec{R}&=\frac{1}{1+n}\brak{\myvec{\vec{C}+n\vec{F}}}\\
&=\frac{1}{3}\myvec{11\\11}\\
\end{align}

\item 
 $\vec{P},\vec{Q},\vec{R}$ are the same point.
   
\item 
\begin{align}
\vec{G}&=\frac{\vec{D}+\vec{E}+\vec{F}}{3}\\
&=\frac{1}{3}\myvec{11\\11}\\
\end{align} 
\end{enumerate}
See Fig.  
  \ref{fig:chapters/10/7/4/7/Figure}.
\begin{figure}[h!]
\centering
\includegraphics[width=\columnwidth]{chapters/10/7/4/7/figs/dj.pdf}
\caption{}
  \label{fig:chapters/10/7/4/7/Figure}
\end{figure}

\item Find the slope of a line, which passes through the origin and the mid point of the line segment joining the points $\vec{P}$(0,-4) and $\vec{B}$(8,0).
\label{chapters/11/10/1/5}
\iffalse
\documentclass[journal,12pt,twocolumn]{IEEEtran}
\usepackage{graphicx}
\graphicspath{{./figs/}}{}
\usepackage{amsmath,amssymb,amsfonts,amsthm}
\newcommand{\myvec}[1]{\ensuremath{\begin{pmatrix}#1\end{pmatrix}}}

\let\vec\mathbf

\title{
Matrix-Lines
}
\author{Jyothsna Paluchuri-FWC22059\\}
\begin{document}
\maketitle
\tableofcontents
\bigskip
\section{Problem Statement}
\fi
	\begin{figure}[!ht]
		\centering
 \includegraphics[width=\columnwidth]{chapters/11/10/1/5/figs/line.png}
		\caption{}
		\label{fig:11/10/1/5}
  	\end{figure}
	\\
	\solution
\iffalse
\section{Construction}
\begin{figure}[h]
    \centering
\includegraphics[width=\columnwidth]{line.png}
    \caption{Equation of the slope}
    \label{fig:my_label}
\end{figure}
\vspace{2cm}
\begin{table}[h]
    \centering
    \begin{tabular}{|c|c|c|c|}
       \hline
       \textbf{Symbol}&\textbf{Value}&\textbf{Description}  \\
       \hline
	    $\vec{P}$ & $\myvec{
		    0\\
		    -4}$
	    & Point on Y-axis\\
        \hline
	    $\vec{B}$ & $\myvec{8\\0}$
 & Point on X-axis\\
        \hline
	    $\vec{0}$ & $\myvec{0\\0}$
 & Origin\\
        \hline
    \end{tabular}
    \caption{Parameters}
    \label{tab:my_label}
\end{table}


\section{Solution}
Given that resultant line passes through origin and mid point of the line segment joining point P(0,-4) and B(8,0) \\
\\
\\
given ${\vec{P}}$=$\myvec{
  0\\
  -4}$
 , ${\vec{B}}$=$\myvec{
  8\\
  0}$
  
 \fi 
The mid point of $PB$ is
\begin{align}
\vec{M} &=\frac{1}{2}(\vec{P}+\vec{B})
	= \myvec{4 \\ -2}  
\end{align}
The direction vector of line joining $\vec{O}, \vec{M}$ is 
\begin{align}
\vec{m}&=\vec{O}-\vec{M}
 = -\vec{M}
\end{align}
which can be expressed as
\begin{align}
	\myvec{1 \\ -\frac{1}{2}}
\end{align}
Thus the slope is
\begin{align}
	m = -\frac{1}{2}
\end{align}
\iffalse
\textbf{The direction vector of a line expressed as}
\begin{align}
\implies\vec{m} &= \begin{pmatrix}1 \\ m \\ \end{pmatrix}
\end{align}

\textbf{By solving equation (5) and (6),we get the slope of $\vec{O}$ $\vec{M}$ line}
\begin{align}
        \boxed{m=-0.5}
 \end{align}

\section{Software}
Download the following code using,
\begin{table}[h]
    \centering
    \begin{tabular}{|c|}
    \hline \\
   https://github.com/jyothsna777/jyothsna-fwc.git  \\
         \\
\hline
    \end{tabular}
\end{table}
\\
and execute the code by using command
\begin{center}
\textbf{Python3 lines.py}\\
\end{center}

\section{Conclusion}
Hence the slope of line $\vec{O}$ $\vec{M}$ lineis $\vec{m}$=-0.5

\end{document}
\fi

\item Find the position vector of a point R which divides the line joining two points P and Q whose position vectors are $(2\vec{a}+\vec{b})$ and $(\vec{a}-3\vec{b})$
externally in the ratio 1 : 2. Also, show that P is the mid point of the line segment RQ.\\
	\solution
%		\begin{enumerate}[label=\thesection.\arabic*,ref=\thesection.\theenumi]
\numberwithin{equation}{enumi}
\numberwithin{figure}{enumi}
\numberwithin{table}{enumi}

\item Find the coordinates of the point which divides the join of $(-1,7) \text{ and } (4,-3)$ in the ratio 2:3.
	\\
		\solution
	\input{chapters/10/7/2/1/section.tex}
\item Find the coordinates of the points of trisection of the line segment joining $(4,-1) \text{ and } (-2,3)$.
	\\
		\solution
	\input{chapters/10/7/2/2/section.tex}
\item
	\iffalse
\item To conduct Sports Day activities, in your rectangular shaped school                   
ground ABCD, lines have 
drawn with chalk powder at a                 
distance of 1m each. 100 flower pots have been placed at a distance of 1m 
from each other along AD, as shown 
in Fig. 7.12. Niharika runs $ \frac {1}{4} $th the 
distance AD on the 2nd line and 
posts a green flag. Preet runs $ \frac {1}{5} $th 
the distance AD on the eighth line 
and posts a red flag. What is the 
distance between both the flags? If 
Rashmi has to post a blue flag exactly 
halfway between the line segment 
joining the two flags, where should 
she post her flag?
\begin{figure}[h!]
  \centering
  \includegraphics[width=\columnwidth]{sc.png}
  \caption{}
\label{fig:10/7/12Fig1}
\end{figure}               
\fi
      
\item Find the ratio in which the line segment joining the points $(-3,10) \text{ and } (6,-8)$ $\text{ is divided by } (-1,6)$.
	\\
		\solution
	\input{chapters/10/7/2/4/section.tex}
\item Find the ratio in which the line segment joining $A(1,-5) \text{ and } B(-4,5)$ $\text{is divided by the x-axis}$. Also find the coordinates of the point of division.
\item If $(1,2), (4,y), (x,6), (3,5)$ are the vertices of a parallelogram taken in order, find x and y.
	\\
		\solution
	\input{chapters/10/7/2/6/para1.tex}
\item Find the coordinates of a point A, where AB is the diameter of a circle whose centre is $(2,-3) \text{ and }$ B is $(1,4)$.
	\\
		\solution
	\input{chapters/10/7/2/7/section.tex}
\item If A \text{ and } B are $(-2,-2) \text{ and } (2,-4)$, respectively, find the coordinates of P such that AP= $\frac {3}{7}$AB $\text{ and }$ P lies on the line segment AB.
	\\
		\solution
	\input{chapters/10/7/2/8/section.tex}
\item Find the coordinates of the points which divide the line segment joining $A(-2,2) \text{ and } B(2,8)$ into four equal parts.
	\\
		\solution
	\input{chapters/10/7/2/9/section.tex}
\item Find the area of a rhombus if its vertices are $(3,0), (4,5), (-1,4) \text{ and } (-2,-1)$ taken in order. [$\vec{Hint}$ : Area of rhombus =$\frac {1}{2}$(product of its diagonals)]
	\\
		\solution
	\input{chapters/10/7/2/10/cross.tex}
\item Find the position vector of a point R which divides the line joining two points $\vec{P}$
and $\vec{Q}$ whose position vectors are $\hat{i}+2\hat{j}-\hat{k}$ and $-\hat{i}+\hat{j}+\hat{k}$ respectively, in the
ratio 2 : 1
\begin{enumerate}
    \item  internally
    \item  externally
\end{enumerate}
\solution
		\input{chapters/12/10/2/15/section.tex}
\item Find the position vector of the mid point of the vector joining the points $\vec{P}$(2, 3, 4)
and $\vec{Q}$(4, 1, –2).
\\
\solution
		\input{chapters/12/10/2/16/section.tex}
\item Determine the ratio in which the line $2x+y  - 4=0$ divides the line segment joining the points $\vec{A}(2, - 2)$  and  $\vec{B}(3, 7)$.
\\
\solution
	\input{chapters/10/7/4/1/section.tex}
\item Let $\vec{A}(4, 2), \vec{B}(6, 5)$  and $ \vec{C}(1, 4)$ be the vertices of $\triangle ABC$.
\begin{enumerate}
\item The median from $\vec{A}$ meets $BC$ at $\vec{D}$. Find the coordinates of the point $\vec{D}$.
\item Find the coordinates of the point $\vec{P}$ on $AD$ such that $AP : PD = 2 : 1$.
\item Find the coordinates of points $\vec{Q}$ and $\vec{R}$ on medians $BE$ and $CF$ respectively such that $BQ : QE = 2 : 1$  and  $CR : RF = 2 : 1$.
\item What do you observe?
\item If $\vec{A}, \vec{B}$ and $\vec{C}$  are the vertices of $\triangle ABC$, find the coordinates of the centroid of the triangle.
\end{enumerate}
\solution
	\input{chapters/10/7/4/7/section.tex}
\item Find the slope of a line, which passes through the origin and the mid point of the line segment joining the points $\vec{P}$(0,-4) and $\vec{B}$(8,0).
\label{chapters/11/10/1/5}
\input{chapters/11/10/1/5/matrix.tex}
\item Find the position vector of a point R which divides the line joining two points P and Q whose position vectors are $(2\vec{a}+\vec{b})$ and $(\vec{a}-3\vec{b})$
externally in the ratio 1 : 2. Also, show that P is the mid point of the line segment RQ.\\
	\solution
%		\input{chapters/12/10/5/9/section.tex}

\end{enumerate}



\end{enumerate}


\item
	\iffalse
\item To conduct Sports Day activities, in your rectangular shaped school                   
ground ABCD, lines have 
drawn with chalk powder at a                 
distance of 1m each. 100 flower pots have been placed at a distance of 1m 
from each other along AD, as shown 
in Fig. 7.12. Niharika runs $ \frac {1}{4} $th the 
distance AD on the 2nd line and 
posts a green flag. Preet runs $ \frac {1}{5} $th 
the distance AD on the eighth line 
and posts a red flag. What is the 
distance between both the flags? If 
Rashmi has to post a blue flag exactly 
halfway between the line segment 
joining the two flags, where should 
she post her flag?
\begin{figure}[h!]
  \centering
  \includegraphics[width=\columnwidth]{sc.png}
  \caption{}
\label{fig:10/7/12Fig1}
\end{figure}               
\fi
      
\item Find the ratio in which the line segment joining the points $(-3,10) \text{ and } (6,-8)$ $\text{ is divided by } (-1,6)$.
	\\
		\solution
	\iffalse
\documentclass[12pt]{article}
\usepackage{graphicx}
%\documentclass[journal,12pt,twocolumn]{IEEEtran}
\usepackage[none]{hyphenat}
\usepackage{graphicx}
\usepackage{listings}
\usepackage[english]{babel}
\usepackage{graphicx}
\usepackage{caption} 
\usepackage{hyperref}
\usepackage{booktabs}
\def\inputGnumericTable{}
\usepackage{color}                                            %%
    \usepackage{array}                                            %%
    \usepackage{longtable}                                        %%
    \usepackage{calc}                                             %%
    \usepackage{multirow}                                         %%
    \usepackage{hhline}                                           %%
    \usepackage{ifthen}
\usepackage{array}
\usepackage{amsmath}   % for having text in math mode
\usepackage{listings}
\lstset{
language=tex,
frame=single, 
breaklines=true
}
  
%Following 2 lines were added to remove the blank page at the beginning
\usepackage{atbegshi}% http://ctan.org/pkg/atbegshi
\AtBeginDocument{\AtBeginShipoutNext{\AtBeginShipoutDiscard}}
%
%New macro definitions
\newcommand{\mydet}[1]{\ensuremath{\begin{vmatrix}#1\end{vmatrix}}}
\providecommand{\brak}[1]{\ensuremath{\left(#1\right)}}
\providecommand{\norm}[1]{\left\lVert#1\right\rVert}
\newcommand{\solution}{\noindent \textbf{Solution: }}
\newcommand{\myvec}[1]{\ensuremath{\begin{pmatrix}#1\end{pmatrix}}}
\let\vec\mathbf
\begin{document}
\begin{center}
\title{\textbf{Coordinate Geometry}}
\date{\vspace{-5ex}} %Not to print date automatically
\maketitle
\end{center}
\setcounter{page}{1}
\section*{10$^{th}$ Maths - Chapter 7}
This is Problem-4 from Exercise 7.2
\begin{enumerate}
\item Find the ratio in which the line segement joining the points $\myvec{-3 \\ 10}$ and $\myvec{6\\-8}$ is divided by $\myvec{-1\\6}$.\\
\solution \\
\fi
		The input parameters for this problem are available in Table \eqref{tab:10/7/2/4-1}.
\begin{table}[ht!]
\begin{tabular}{|c|c|p{5cm}|}
\hline
\textbf{Symbol} & \textbf{Value} & \textbf{Description} \\
\hline
$\theta$ & $30\degree$ & $\angle{BAP} = \angle{BAQ}$ \\
\hline
$a$ & $9$ & $AB$ \\
\hline
$c$ & $8$ & $AQ$ \\
\hline
$\vec{e}_1$ & $\myvec{1\\0}$ & Basis vector \\
\hline
\end{tabular}

\caption{}
\label{tab:10/7/2/4-1} 
\end{table}
Using section formula,
\begin{align}
         \vec{R} &=\frac{\vec{Q}+n\vec{P}}{1+n}\label{eq:chapters/10/7/2/4/1}
\end{align}
Substituting the values of $\vec{P},\vec{Q}$ and $\vec{R}$ in \eqref{eq:chapters/10/7/2/4/1}
\begin{align}
         \myvec{-1\\6} &=\frac{{\myvec{-3\\10}+n\myvec{6\\-8}}}{1+n}\\
 &=\frac{1}{1+n}\brak{{\myvec{-3\\10}+n\myvec{6\\-8}}} \\
 &=\frac{1}{1+n}\myvec{-3+6n\\10-8n} \label{eq:chapters/10/7/2/4/4}
\end{align}
Simplifying \eqref{eq:chapters/10/7/2/4/4} yeilds,
\begin{align}
          -1 &=\frac{-3+6n}{1+n}\\
\implies          n &=\frac{2}{7}
\end{align}
Also,
\begin{align}
          6 &=\frac{10-8n}{1+n}\\
    \implies      n &=\frac{2}{7}
\end{align}
Hence the desired ratio is $\dfrac{2}{7}$.  
\begin{figure}[!h]
 \begin{center}
  \includegraphics[width=\columnwidth]{chapters/10/7/2/4/figs/fig.png}
 \end{center}
\caption{}
\label{fig:10/7/2/4Fig1}
\end{figure}

\item Find the ratio in which the line segment joining $A(1,-5) \text{ and } B(-4,5)$ $\text{is divided by the x-axis}$. Also find the coordinates of the point of division.
\item If $(1,2), (4,y), (x,6), (3,5)$ are the vertices of a parallelogram taken in order, find x and y.
	\\
		\solution
	\iffalse
\documentclass[12pt]{article}
\usepackage{graphicx}
%\documentclass[journal,12pt,twocolumn]{IEEEtran}
\def\inputGnumericTable{}
\usepackage{color}                                            %%
    \usepackage{array}                                            %%
    \usepackage{longtable}                                        %%
    \usepackage{calc}                                             %%
    \usepackage{multirow}                                         %%
    \usepackage{hhline}                                           %%
    \usepackage{ifthen}
\usepackage[none]{hyphenat}
\usepackage{graphicx}
\usepackage{listings}
\usepackage[english]{babel}
\usepackage{graphicx}
\usepackage{caption} 
\usepackage{hyperref}
\usepackage{booktabs}
\usepackage{array}
\usepackage{amsmath}   % for having text in math mode
\usepackage{listings}
\lstset{
  frame=single,
  breaklines=true
}
  
%Following 2 lines were added to remove the blank page at the beginning
\usepackage{atbegshi}% http://ctan.org/pkg/atbegshi
\AtBeginDocument{\AtBeginShipoutNext{\AtBeginShipoutDiscard}}
%


%New macro definitions
\newcommand{\mydet}[1]{\ensuremath{\begin{vmatrix}#1\end{vmatrix}}}
\providecommand{\brak}[1]{\ensuremath{\left(#1\right)}}
\providecommand{\norm}[1]{\left\lVert#1\right\rVert}
\newcommand{\solution}{\noindent \textbf{Solution: }}
\newcommand{\myvec}[1]{\ensuremath{\begin{pmatrix}#1\end{pmatrix}}}
\let\vec\mathbf

\begin{document}

\begin{center}
\title{\textbf{Properties of Parallelegram}}
\date{\vspace{-5ex}} %Not to print date automatically
\maketitle
\end{center}

\setcounter{page}{1}

\section{10$^{th}$ Maths - Chapter 7}

This is Problem-6 from Exercise 7.2

\begin{enumerate}
\item If $\vec{A}(1, 2),\vec{B}(4, x),\vec{C}(y, 6) \text{and } \vec{D}(3, 5)$ are the vertices of a parallelogram taken in order,find x and y.
\end{enumerate}
\fi

The input parameters for this problem are available in
\ref{table:chapters/10/7/2/6/tables/}.	
\begin{table}[!ht]
	\centering
	\begin{tabular}{|c|c|p{5cm}|}
\hline
\textbf{Symbol} & \textbf{Value} & \textbf{Description} \\
\hline
$\theta$ & $30\degree$ & $\angle{BAP} = \angle{BAQ}$ \\
\hline
$a$ & $9$ & $AB$ \\
\hline
$c$ & $8$ & $AQ$ \\
\hline
$\vec{e}_1$ & $\myvec{1\\0}$ & Basis vector \\
\hline
\end{tabular}

\caption{}
\label{table:chapters/10/7/2/6/tables/}	
\end{table}
From the given information,
\begin{align}
  \label{eq:chapters/10/7/2/6/tables/det2f}
	\vec{B}-\vec{A} &= \myvec{4 \\y } - \myvec{1 \\2 }  = \myvec{3 \\y-2 }\\
	\vec{C}-\vec{D} &= \myvec{x \\6 } - \myvec{3 \\5 }  = \myvec{x-3 \\1}
\end{align}
Since $ABCD$ is a parallellogram,
\begin{align}
	\myvec{3\\y-2}&=\myvec{x-3\\1}\\
	\implies x&=6 ,y=3
\end{align}
Fig. \ref{fig:chapters/10/7/2/6/Fig3}
provides a verification.
\begin{figure}[h!]
	\begin{center}
  \includegraphics[width=\columnwidth]{chapters/10/7/2/6/figs/para.pdf}
	\end{center}
\caption{}
\label{fig:chapters/10/7/2/6/Fig3}
\end{figure}


\item Find the coordinates of a point A, where AB is the diameter of a circle whose centre is $(2,-3) \text{ and }$ B is $(1,4)$.
	\\
		\solution
	\iffalse
\documentclass[12pt]{article}
\usepackage{graphicx}
\usepackage{amsmath}
\usepackage{mathtools}
\usepackage{gensymb}

\newcommand{\mydet}[1]{\ensuremath{\begin{vmatrix}#1\end{vmatrix}}}
\providecommand{\brak}[1]{\ensuremath{\left(#1\right)}}
\providecommand{\norm}[1]{\left\lVert#1\right\rVert}
\newcommand{\solution}{\noindent \textbf{Solution: }}
\newcommand{\myvec}[1]{\ensuremath{\begin{pmatrix}#1\end{pmatrix}}}
\let\vec\mathbf

\begin{document}
\begin{center}
\section*{CHAPTER 7 - COORDINATE GEOMETRY}

\end{center}
\section*{Excercise 7.2}

Q7.Find the coordinates of point $\vec{A}$, where AB is the diameter of a circle where the center is (2,-3) and $\vec{B}$ is the point (1,4):

\solution
\begin{enumerate}
\item The coordinates $\vec{B}$ and center $\vec{C}$ are given, where:
	\fi
	Let
	\begin{align}
	\vec{B} = \myvec{
		1\\
	    4\\
		},
	\vec{C} = \myvec{
	    2\\
	   -3\\
		}
	\end{align}
	\iffalse
Let us assume the coordinates of $\vec{A}$. Now, $\vec{C}$ is the center which is midpoint of line AB and $\vec{B}$ is one of the coordinate of diameter AB of a circle.
	\fi	
Hence,	
	\begin{align}
	\vec{C} &= \frac{\vec{A+B}}{2} \\
\implies	2\vec{C} &= \vec{A}+\vec{B} \\
		\text{or, }	\vec{A} &= 2\vec{C}-\vec{B} \\
	 &= \myvec{3\\-10\\}	
	\end{align}       
	See Fig. 
\ref{fig:chapters/10/7/2/7Fig}.
\begin{figure}[!h]
\begin{center}	
	\includegraphics[width=\columnwidth]{chapters/10/7/2/7/figs/Vector1.png}
\end{center}
\caption{}
\label{fig:chapters/10/7/2/7Fig}
\end{figure}
	

\item If A \text{ and } B are $(-2,-2) \text{ and } (2,-4)$, respectively, find the coordinates of P such that AP= $\frac {3}{7}$AB $\text{ and }$ P lies on the line segment AB.
	\\
		\solution
	\iffalse
\documentclass[journal,10pt,twocolumn]{article}
\usepackage{graphicx}
\usepackage[none]{hyphenat}
\usepackage{graphicx}
\usepackage{listings}
\usepackage[english]{babel}
\usepackage{graphicx}
\usepackage{caption} 
\usepackage{booktabs}
\usepackage{array}
\usepackage{amssymb} % for \because
\usepackage{amsmath}   % for having text in math mode
\usepackage{extarrows} % for Row operations arrows
\usepackage{listings}
\usepackage[utf8]{inputenc}
\lstset{
  frame=single,
  breaklines=true
}
\usepackage{hyperref}
  
%Following 2 lines were added to remove the blank page at the beginning
\usepackage{atbegshi}% http://ctan.org/pkg/atbegshi
\AtBeginDocument{\AtBeginShipoutNext{\AtBeginShipoutDiscard}}


%New macro definitions
\newcommand{\mydet}[1]{\ensuremath{\begin{vmatrix}#1\end{vmatrix}}}
\providecommand{\brak}[1]{\ensuremath{\left(#1\right)}}
\newcommand{\solution}{\noindent \textbf{Solution: }}
\newcommand{\myvec}[1]{\ensuremath{\begin{pmatrix}#1\end{pmatrix}}}
\providecommand{\norm}[1]{\left\lVert#1\right\rVert}
\providecommand{\abs}[1]{\left\vert#1\right\vert}
\let\vec\mathbf

\begin{document}

\begin{center}
\title{\textbf{VECTORS}}
\date{\vspace{-5ex}} %Not to print date automatically
\maketitle
\end{center}

\section{10$^{th}$ Maths - EXERCISE-7.2}

\begin{enumerate}
\item If A and B are $(– 2, – 2)\text{ and }(2, – 4)$, respectively, find the coordinates of P such that $AP =\frac{3}{7}AB$ and P lies on the line segment AB. 

\section{SOLUTION}
Given points are
\begin{align}
\vec{A}=\myvec{-2\\ -2} ,
\vec{B}=\myvec{2\\ -4}
\end{align}
The equation of the formula is
\fi
Using section formula, 
\begin{align}
\vec{P}&=\frac{\vec{A}+n\vec{B}}{1+n}
\end{align}
where
\begin{align}
	n =\frac{3}{4}
\end{align}
Thus,
\begin{align}
\vec{P}&=\frac{1}{1+\frac{3}{4}}\brak{\myvec{-2\\-2}+\frac{3}{4}\myvec{2\\-4}}\\
&=\myvec{\frac{-2}{7}\\[1pt] \frac{-20}{7}}
\end{align}
See Fig. 
   \ref{fig:chapters/10/7/2/8/vec.png}
\begin{figure}
   \centering 
 \includegraphics[width=\columnwidth]{chapters/10/7/2/8/figs/vec.png}
   \caption{}
   \label{fig:chapters/10/7/2/8/vec.png}
   \end{figure}

\item Find the coordinates of the points which divide the line segment joining $A(-2,2) \text{ and } B(2,8)$ into four equal parts.
	\\
		\solution
	\begin{enumerate}[label=\thesection.\arabic*,ref=\thesection.\theenumi]
\numberwithin{equation}{enumi}
\numberwithin{figure}{enumi}
\numberwithin{table}{enumi}

\item Find the coordinates of the point which divides the join of $(-1,7) \text{ and } (4,-3)$ in the ratio 2:3.
	\\
		\solution
	\iffalse
\documentclass[12pt]{article}
\usepackage{graphicx}
\usepackage{amsmath}
\usepackage{mathtools}
\usepackage{gensymb}

\newcommand{\mydet}[1]{\ensuremath{\begin{vmatrix}#1\end{vmatrix}}}
\providecommand{\brak}[1]{\ensuremath{\left(#1\right)}}
\providecommand{\norm}[1]{\left\lVert#1\right\rVert}
\newcommand{\solution}{\noindent \textbf{Solution: }}
\newcommand{\myvec}[1]{\ensuremath{\begin{pmatrix}#1\end{pmatrix}}}
\let\vec\mathbf

\begin{document}
\begin{center}
\textbf\large{CHAPTER-7 \\ COORDINATE GEOMETRY}
\end{center}
\section*{Excercise 7.2}

1. Find the coordinates of the point which divides the join $\vec(-1,7) \text{ and } \vec(4,-3)$ in the ratio 2:3 :
\\
\\
\solution\\		
\fi
The coordinates and ratio are given as
\begin{align}
\vec{P}=\myvec{-1\\7\\},
\vec{Q}=\myvec{4\\-3\\},
n=\frac{3}{2}
\end{align}
Using section formula
\begin{align}
\vec{R}&=\frac{\vec{Q}+n\vec{P}}{1+n}\\
&=\frac{1}{1+\frac{3}{2}}  \myvec{\myvec{
4\\
-3\\
}
  +
   \frac{3}{2}\myvec{
-1\\
7\\
}}\\
&=\myvec{
1\\
3
}
\end{align}
See Fig. 
\ref{fig:chapters/10/7/2/1/Fig}
\begin{figure}[!h]
\begin{center}
   \includegraphics[width=\columnwidth]{chapters/10/7/2/1/figs/linefig.png}
\end{center}
\caption{}
\label{fig:chapters/10/7/2/1/Fig}
\end{figure}


\item Find the coordinates of the points of trisection of the line segment joining $(4,-1) \text{ and } (-2,3)$.
	\\
		\solution
	\begin{enumerate}[label=\thesection.\arabic*,ref=\thesection.\theenumi]
\numberwithin{equation}{enumi}
\numberwithin{figure}{enumi}
\numberwithin{table}{enumi}

\item Find the coordinates of the point which divides the join of $(-1,7) \text{ and } (4,-3)$ in the ratio 2:3.
	\\
		\solution
	\input{chapters/10/7/2/1/section.tex}
\item Find the coordinates of the points of trisection of the line segment joining $(4,-1) \text{ and } (-2,3)$.
	\\
		\solution
	\input{chapters/10/7/2/2/section.tex}
\item
	\iffalse
\item To conduct Sports Day activities, in your rectangular shaped school                   
ground ABCD, lines have 
drawn with chalk powder at a                 
distance of 1m each. 100 flower pots have been placed at a distance of 1m 
from each other along AD, as shown 
in Fig. 7.12. Niharika runs $ \frac {1}{4} $th the 
distance AD on the 2nd line and 
posts a green flag. Preet runs $ \frac {1}{5} $th 
the distance AD on the eighth line 
and posts a red flag. What is the 
distance between both the flags? If 
Rashmi has to post a blue flag exactly 
halfway between the line segment 
joining the two flags, where should 
she post her flag?
\begin{figure}[h!]
  \centering
  \includegraphics[width=\columnwidth]{sc.png}
  \caption{}
\label{fig:10/7/12Fig1}
\end{figure}               
\fi
      
\item Find the ratio in which the line segment joining the points $(-3,10) \text{ and } (6,-8)$ $\text{ is divided by } (-1,6)$.
	\\
		\solution
	\input{chapters/10/7/2/4/section.tex}
\item Find the ratio in which the line segment joining $A(1,-5) \text{ and } B(-4,5)$ $\text{is divided by the x-axis}$. Also find the coordinates of the point of division.
\item If $(1,2), (4,y), (x,6), (3,5)$ are the vertices of a parallelogram taken in order, find x and y.
	\\
		\solution
	\input{chapters/10/7/2/6/para1.tex}
\item Find the coordinates of a point A, where AB is the diameter of a circle whose centre is $(2,-3) \text{ and }$ B is $(1,4)$.
	\\
		\solution
	\input{chapters/10/7/2/7/section.tex}
\item If A \text{ and } B are $(-2,-2) \text{ and } (2,-4)$, respectively, find the coordinates of P such that AP= $\frac {3}{7}$AB $\text{ and }$ P lies on the line segment AB.
	\\
		\solution
	\input{chapters/10/7/2/8/section.tex}
\item Find the coordinates of the points which divide the line segment joining $A(-2,2) \text{ and } B(2,8)$ into four equal parts.
	\\
		\solution
	\input{chapters/10/7/2/9/section.tex}
\item Find the area of a rhombus if its vertices are $(3,0), (4,5), (-1,4) \text{ and } (-2,-1)$ taken in order. [$\vec{Hint}$ : Area of rhombus =$\frac {1}{2}$(product of its diagonals)]
	\\
		\solution
	\input{chapters/10/7/2/10/cross.tex}
\item Find the position vector of a point R which divides the line joining two points $\vec{P}$
and $\vec{Q}$ whose position vectors are $\hat{i}+2\hat{j}-\hat{k}$ and $-\hat{i}+\hat{j}+\hat{k}$ respectively, in the
ratio 2 : 1
\begin{enumerate}
    \item  internally
    \item  externally
\end{enumerate}
\solution
		\input{chapters/12/10/2/15/section.tex}
\item Find the position vector of the mid point of the vector joining the points $\vec{P}$(2, 3, 4)
and $\vec{Q}$(4, 1, –2).
\\
\solution
		\input{chapters/12/10/2/16/section.tex}
\item Determine the ratio in which the line $2x+y  - 4=0$ divides the line segment joining the points $\vec{A}(2, - 2)$  and  $\vec{B}(3, 7)$.
\\
\solution
	\input{chapters/10/7/4/1/section.tex}
\item Let $\vec{A}(4, 2), \vec{B}(6, 5)$  and $ \vec{C}(1, 4)$ be the vertices of $\triangle ABC$.
\begin{enumerate}
\item The median from $\vec{A}$ meets $BC$ at $\vec{D}$. Find the coordinates of the point $\vec{D}$.
\item Find the coordinates of the point $\vec{P}$ on $AD$ such that $AP : PD = 2 : 1$.
\item Find the coordinates of points $\vec{Q}$ and $\vec{R}$ on medians $BE$ and $CF$ respectively such that $BQ : QE = 2 : 1$  and  $CR : RF = 2 : 1$.
\item What do you observe?
\item If $\vec{A}, \vec{B}$ and $\vec{C}$  are the vertices of $\triangle ABC$, find the coordinates of the centroid of the triangle.
\end{enumerate}
\solution
	\input{chapters/10/7/4/7/section.tex}
\item Find the slope of a line, which passes through the origin and the mid point of the line segment joining the points $\vec{P}$(0,-4) and $\vec{B}$(8,0).
\label{chapters/11/10/1/5}
\input{chapters/11/10/1/5/matrix.tex}
\item Find the position vector of a point R which divides the line joining two points P and Q whose position vectors are $(2\vec{a}+\vec{b})$ and $(\vec{a}-3\vec{b})$
externally in the ratio 1 : 2. Also, show that P is the mid point of the line segment RQ.\\
	\solution
%		\input{chapters/12/10/5/9/section.tex}

\end{enumerate}


\item
	\iffalse
\item To conduct Sports Day activities, in your rectangular shaped school                   
ground ABCD, lines have 
drawn with chalk powder at a                 
distance of 1m each. 100 flower pots have been placed at a distance of 1m 
from each other along AD, as shown 
in Fig. 7.12. Niharika runs $ \frac {1}{4} $th the 
distance AD on the 2nd line and 
posts a green flag. Preet runs $ \frac {1}{5} $th 
the distance AD on the eighth line 
and posts a red flag. What is the 
distance between both the flags? If 
Rashmi has to post a blue flag exactly 
halfway between the line segment 
joining the two flags, where should 
she post her flag?
\begin{figure}[h!]
  \centering
  \includegraphics[width=\columnwidth]{sc.png}
  \caption{}
\label{fig:10/7/12Fig1}
\end{figure}               
\fi
      
\item Find the ratio in which the line segment joining the points $(-3,10) \text{ and } (6,-8)$ $\text{ is divided by } (-1,6)$.
	\\
		\solution
	\iffalse
\documentclass[12pt]{article}
\usepackage{graphicx}
%\documentclass[journal,12pt,twocolumn]{IEEEtran}
\usepackage[none]{hyphenat}
\usepackage{graphicx}
\usepackage{listings}
\usepackage[english]{babel}
\usepackage{graphicx}
\usepackage{caption} 
\usepackage{hyperref}
\usepackage{booktabs}
\def\inputGnumericTable{}
\usepackage{color}                                            %%
    \usepackage{array}                                            %%
    \usepackage{longtable}                                        %%
    \usepackage{calc}                                             %%
    \usepackage{multirow}                                         %%
    \usepackage{hhline}                                           %%
    \usepackage{ifthen}
\usepackage{array}
\usepackage{amsmath}   % for having text in math mode
\usepackage{listings}
\lstset{
language=tex,
frame=single, 
breaklines=true
}
  
%Following 2 lines were added to remove the blank page at the beginning
\usepackage{atbegshi}% http://ctan.org/pkg/atbegshi
\AtBeginDocument{\AtBeginShipoutNext{\AtBeginShipoutDiscard}}
%
%New macro definitions
\newcommand{\mydet}[1]{\ensuremath{\begin{vmatrix}#1\end{vmatrix}}}
\providecommand{\brak}[1]{\ensuremath{\left(#1\right)}}
\providecommand{\norm}[1]{\left\lVert#1\right\rVert}
\newcommand{\solution}{\noindent \textbf{Solution: }}
\newcommand{\myvec}[1]{\ensuremath{\begin{pmatrix}#1\end{pmatrix}}}
\let\vec\mathbf
\begin{document}
\begin{center}
\title{\textbf{Coordinate Geometry}}
\date{\vspace{-5ex}} %Not to print date automatically
\maketitle
\end{center}
\setcounter{page}{1}
\section*{10$^{th}$ Maths - Chapter 7}
This is Problem-4 from Exercise 7.2
\begin{enumerate}
\item Find the ratio in which the line segement joining the points $\myvec{-3 \\ 10}$ and $\myvec{6\\-8}$ is divided by $\myvec{-1\\6}$.\\
\solution \\
\fi
		The input parameters for this problem are available in Table \eqref{tab:10/7/2/4-1}.
\begin{table}[ht!]
\input{chapters/10/7/2/4/tables/table.tex}
\caption{}
\label{tab:10/7/2/4-1} 
\end{table}
Using section formula,
\begin{align}
         \vec{R} &=\frac{\vec{Q}+n\vec{P}}{1+n}\label{eq:chapters/10/7/2/4/1}
\end{align}
Substituting the values of $\vec{P},\vec{Q}$ and $\vec{R}$ in \eqref{eq:chapters/10/7/2/4/1}
\begin{align}
         \myvec{-1\\6} &=\frac{{\myvec{-3\\10}+n\myvec{6\\-8}}}{1+n}\\
 &=\frac{1}{1+n}\brak{{\myvec{-3\\10}+n\myvec{6\\-8}}} \\
 &=\frac{1}{1+n}\myvec{-3+6n\\10-8n} \label{eq:chapters/10/7/2/4/4}
\end{align}
Simplifying \eqref{eq:chapters/10/7/2/4/4} yeilds,
\begin{align}
          -1 &=\frac{-3+6n}{1+n}\\
\implies          n &=\frac{2}{7}
\end{align}
Also,
\begin{align}
          6 &=\frac{10-8n}{1+n}\\
    \implies      n &=\frac{2}{7}
\end{align}
Hence the desired ratio is $\dfrac{2}{7}$.  
\begin{figure}[!h]
 \begin{center}
  \includegraphics[width=\columnwidth]{chapters/10/7/2/4/figs/fig.png}
 \end{center}
\caption{}
\label{fig:10/7/2/4Fig1}
\end{figure}

\item Find the ratio in which the line segment joining $A(1,-5) \text{ and } B(-4,5)$ $\text{is divided by the x-axis}$. Also find the coordinates of the point of division.
\item If $(1,2), (4,y), (x,6), (3,5)$ are the vertices of a parallelogram taken in order, find x and y.
	\\
		\solution
	\iffalse
\documentclass[12pt]{article}
\usepackage{graphicx}
%\documentclass[journal,12pt,twocolumn]{IEEEtran}
\def\inputGnumericTable{}
\usepackage{color}                                            %%
    \usepackage{array}                                            %%
    \usepackage{longtable}                                        %%
    \usepackage{calc}                                             %%
    \usepackage{multirow}                                         %%
    \usepackage{hhline}                                           %%
    \usepackage{ifthen}
\usepackage[none]{hyphenat}
\usepackage{graphicx}
\usepackage{listings}
\usepackage[english]{babel}
\usepackage{graphicx}
\usepackage{caption} 
\usepackage{hyperref}
\usepackage{booktabs}
\usepackage{array}
\usepackage{amsmath}   % for having text in math mode
\usepackage{listings}
\lstset{
  frame=single,
  breaklines=true
}
  
%Following 2 lines were added to remove the blank page at the beginning
\usepackage{atbegshi}% http://ctan.org/pkg/atbegshi
\AtBeginDocument{\AtBeginShipoutNext{\AtBeginShipoutDiscard}}
%


%New macro definitions
\newcommand{\mydet}[1]{\ensuremath{\begin{vmatrix}#1\end{vmatrix}}}
\providecommand{\brak}[1]{\ensuremath{\left(#1\right)}}
\providecommand{\norm}[1]{\left\lVert#1\right\rVert}
\newcommand{\solution}{\noindent \textbf{Solution: }}
\newcommand{\myvec}[1]{\ensuremath{\begin{pmatrix}#1\end{pmatrix}}}
\let\vec\mathbf

\begin{document}

\begin{center}
\title{\textbf{Properties of Parallelegram}}
\date{\vspace{-5ex}} %Not to print date automatically
\maketitle
\end{center}

\setcounter{page}{1}

\section{10$^{th}$ Maths - Chapter 7}

This is Problem-6 from Exercise 7.2

\begin{enumerate}
\item If $\vec{A}(1, 2),\vec{B}(4, x),\vec{C}(y, 6) \text{and } \vec{D}(3, 5)$ are the vertices of a parallelogram taken in order,find x and y.
\end{enumerate}
\fi

The input parameters for this problem are available in
\ref{table:chapters/10/7/2/6/tables/}.	
\begin{table}[!ht]
	\centering
	\input{chapters/10/7/2/6/tables/table.tex}
\caption{}
\label{table:chapters/10/7/2/6/tables/}	
\end{table}
From the given information,
\begin{align}
  \label{eq:chapters/10/7/2/6/tables/det2f}
	\vec{B}-\vec{A} &= \myvec{4 \\y } - \myvec{1 \\2 }  = \myvec{3 \\y-2 }\\
	\vec{C}-\vec{D} &= \myvec{x \\6 } - \myvec{3 \\5 }  = \myvec{x-3 \\1}
\end{align}
Since $ABCD$ is a parallellogram,
\begin{align}
	\myvec{3\\y-2}&=\myvec{x-3\\1}\\
	\implies x&=6 ,y=3
\end{align}
Fig. \ref{fig:chapters/10/7/2/6/Fig3}
provides a verification.
\begin{figure}[h!]
	\begin{center}
  \includegraphics[width=\columnwidth]{chapters/10/7/2/6/figs/para.pdf}
	\end{center}
\caption{}
\label{fig:chapters/10/7/2/6/Fig3}
\end{figure}


\item Find the coordinates of a point A, where AB is the diameter of a circle whose centre is $(2,-3) \text{ and }$ B is $(1,4)$.
	\\
		\solution
	\iffalse
\documentclass[12pt]{article}
\usepackage{graphicx}
\usepackage{amsmath}
\usepackage{mathtools}
\usepackage{gensymb}

\newcommand{\mydet}[1]{\ensuremath{\begin{vmatrix}#1\end{vmatrix}}}
\providecommand{\brak}[1]{\ensuremath{\left(#1\right)}}
\providecommand{\norm}[1]{\left\lVert#1\right\rVert}
\newcommand{\solution}{\noindent \textbf{Solution: }}
\newcommand{\myvec}[1]{\ensuremath{\begin{pmatrix}#1\end{pmatrix}}}
\let\vec\mathbf

\begin{document}
\begin{center}
\section*{CHAPTER 7 - COORDINATE GEOMETRY}

\end{center}
\section*{Excercise 7.2}

Q7.Find the coordinates of point $\vec{A}$, where AB is the diameter of a circle where the center is (2,-3) and $\vec{B}$ is the point (1,4):

\solution
\begin{enumerate}
\item The coordinates $\vec{B}$ and center $\vec{C}$ are given, where:
	\fi
	Let
	\begin{align}
	\vec{B} = \myvec{
		1\\
	    4\\
		},
	\vec{C} = \myvec{
	    2\\
	   -3\\
		}
	\end{align}
	\iffalse
Let us assume the coordinates of $\vec{A}$. Now, $\vec{C}$ is the center which is midpoint of line AB and $\vec{B}$ is one of the coordinate of diameter AB of a circle.
	\fi	
Hence,	
	\begin{align}
	\vec{C} &= \frac{\vec{A+B}}{2} \\
\implies	2\vec{C} &= \vec{A}+\vec{B} \\
		\text{or, }	\vec{A} &= 2\vec{C}-\vec{B} \\
	 &= \myvec{3\\-10\\}	
	\end{align}       
	See Fig. 
\ref{fig:chapters/10/7/2/7Fig}.
\begin{figure}[!h]
\begin{center}	
	\includegraphics[width=\columnwidth]{chapters/10/7/2/7/figs/Vector1.png}
\end{center}
\caption{}
\label{fig:chapters/10/7/2/7Fig}
\end{figure}
	

\item If A \text{ and } B are $(-2,-2) \text{ and } (2,-4)$, respectively, find the coordinates of P such that AP= $\frac {3}{7}$AB $\text{ and }$ P lies on the line segment AB.
	\\
		\solution
	\iffalse
\documentclass[journal,10pt,twocolumn]{article}
\usepackage{graphicx}
\usepackage[none]{hyphenat}
\usepackage{graphicx}
\usepackage{listings}
\usepackage[english]{babel}
\usepackage{graphicx}
\usepackage{caption} 
\usepackage{booktabs}
\usepackage{array}
\usepackage{amssymb} % for \because
\usepackage{amsmath}   % for having text in math mode
\usepackage{extarrows} % for Row operations arrows
\usepackage{listings}
\usepackage[utf8]{inputenc}
\lstset{
  frame=single,
  breaklines=true
}
\usepackage{hyperref}
  
%Following 2 lines were added to remove the blank page at the beginning
\usepackage{atbegshi}% http://ctan.org/pkg/atbegshi
\AtBeginDocument{\AtBeginShipoutNext{\AtBeginShipoutDiscard}}


%New macro definitions
\newcommand{\mydet}[1]{\ensuremath{\begin{vmatrix}#1\end{vmatrix}}}
\providecommand{\brak}[1]{\ensuremath{\left(#1\right)}}
\newcommand{\solution}{\noindent \textbf{Solution: }}
\newcommand{\myvec}[1]{\ensuremath{\begin{pmatrix}#1\end{pmatrix}}}
\providecommand{\norm}[1]{\left\lVert#1\right\rVert}
\providecommand{\abs}[1]{\left\vert#1\right\vert}
\let\vec\mathbf

\begin{document}

\begin{center}
\title{\textbf{VECTORS}}
\date{\vspace{-5ex}} %Not to print date automatically
\maketitle
\end{center}

\section{10$^{th}$ Maths - EXERCISE-7.2}

\begin{enumerate}
\item If A and B are $(– 2, – 2)\text{ and }(2, – 4)$, respectively, find the coordinates of P such that $AP =\frac{3}{7}AB$ and P lies on the line segment AB. 

\section{SOLUTION}
Given points are
\begin{align}
\vec{A}=\myvec{-2\\ -2} ,
\vec{B}=\myvec{2\\ -4}
\end{align}
The equation of the formula is
\fi
Using section formula, 
\begin{align}
\vec{P}&=\frac{\vec{A}+n\vec{B}}{1+n}
\end{align}
where
\begin{align}
	n =\frac{3}{4}
\end{align}
Thus,
\begin{align}
\vec{P}&=\frac{1}{1+\frac{3}{4}}\brak{\myvec{-2\\-2}+\frac{3}{4}\myvec{2\\-4}}\\
&=\myvec{\frac{-2}{7}\\[1pt] \frac{-20}{7}}
\end{align}
See Fig. 
   \ref{fig:chapters/10/7/2/8/vec.png}
\begin{figure}
   \centering 
 \includegraphics[width=\columnwidth]{chapters/10/7/2/8/figs/vec.png}
   \caption{}
   \label{fig:chapters/10/7/2/8/vec.png}
   \end{figure}

\item Find the coordinates of the points which divide the line segment joining $A(-2,2) \text{ and } B(2,8)$ into four equal parts.
	\\
		\solution
	\begin{enumerate}[label=\thesection.\arabic*,ref=\thesection.\theenumi]
\numberwithin{equation}{enumi}
\numberwithin{figure}{enumi}
\numberwithin{table}{enumi}

\item Find the coordinates of the point which divides the join of $(-1,7) \text{ and } (4,-3)$ in the ratio 2:3.
	\\
		\solution
	\input{chapters/10/7/2/1/section.tex}
\item Find the coordinates of the points of trisection of the line segment joining $(4,-1) \text{ and } (-2,3)$.
	\\
		\solution
	\input{chapters/10/7/2/2/section.tex}
\item
	\iffalse
\item To conduct Sports Day activities, in your rectangular shaped school                   
ground ABCD, lines have 
drawn with chalk powder at a                 
distance of 1m each. 100 flower pots have been placed at a distance of 1m 
from each other along AD, as shown 
in Fig. 7.12. Niharika runs $ \frac {1}{4} $th the 
distance AD on the 2nd line and 
posts a green flag. Preet runs $ \frac {1}{5} $th 
the distance AD on the eighth line 
and posts a red flag. What is the 
distance between both the flags? If 
Rashmi has to post a blue flag exactly 
halfway between the line segment 
joining the two flags, where should 
she post her flag?
\begin{figure}[h!]
  \centering
  \includegraphics[width=\columnwidth]{sc.png}
  \caption{}
\label{fig:10/7/12Fig1}
\end{figure}               
\fi
      
\item Find the ratio in which the line segment joining the points $(-3,10) \text{ and } (6,-8)$ $\text{ is divided by } (-1,6)$.
	\\
		\solution
	\input{chapters/10/7/2/4/section.tex}
\item Find the ratio in which the line segment joining $A(1,-5) \text{ and } B(-4,5)$ $\text{is divided by the x-axis}$. Also find the coordinates of the point of division.
\item If $(1,2), (4,y), (x,6), (3,5)$ are the vertices of a parallelogram taken in order, find x and y.
	\\
		\solution
	\input{chapters/10/7/2/6/para1.tex}
\item Find the coordinates of a point A, where AB is the diameter of a circle whose centre is $(2,-3) \text{ and }$ B is $(1,4)$.
	\\
		\solution
	\input{chapters/10/7/2/7/section.tex}
\item If A \text{ and } B are $(-2,-2) \text{ and } (2,-4)$, respectively, find the coordinates of P such that AP= $\frac {3}{7}$AB $\text{ and }$ P lies on the line segment AB.
	\\
		\solution
	\input{chapters/10/7/2/8/section.tex}
\item Find the coordinates of the points which divide the line segment joining $A(-2,2) \text{ and } B(2,8)$ into four equal parts.
	\\
		\solution
	\input{chapters/10/7/2/9/section.tex}
\item Find the area of a rhombus if its vertices are $(3,0), (4,5), (-1,4) \text{ and } (-2,-1)$ taken in order. [$\vec{Hint}$ : Area of rhombus =$\frac {1}{2}$(product of its diagonals)]
	\\
		\solution
	\input{chapters/10/7/2/10/cross.tex}
\item Find the position vector of a point R which divides the line joining two points $\vec{P}$
and $\vec{Q}$ whose position vectors are $\hat{i}+2\hat{j}-\hat{k}$ and $-\hat{i}+\hat{j}+\hat{k}$ respectively, in the
ratio 2 : 1
\begin{enumerate}
    \item  internally
    \item  externally
\end{enumerate}
\solution
		\input{chapters/12/10/2/15/section.tex}
\item Find the position vector of the mid point of the vector joining the points $\vec{P}$(2, 3, 4)
and $\vec{Q}$(4, 1, –2).
\\
\solution
		\input{chapters/12/10/2/16/section.tex}
\item Determine the ratio in which the line $2x+y  - 4=0$ divides the line segment joining the points $\vec{A}(2, - 2)$  and  $\vec{B}(3, 7)$.
\\
\solution
	\input{chapters/10/7/4/1/section.tex}
\item Let $\vec{A}(4, 2), \vec{B}(6, 5)$  and $ \vec{C}(1, 4)$ be the vertices of $\triangle ABC$.
\begin{enumerate}
\item The median from $\vec{A}$ meets $BC$ at $\vec{D}$. Find the coordinates of the point $\vec{D}$.
\item Find the coordinates of the point $\vec{P}$ on $AD$ such that $AP : PD = 2 : 1$.
\item Find the coordinates of points $\vec{Q}$ and $\vec{R}$ on medians $BE$ and $CF$ respectively such that $BQ : QE = 2 : 1$  and  $CR : RF = 2 : 1$.
\item What do you observe?
\item If $\vec{A}, \vec{B}$ and $\vec{C}$  are the vertices of $\triangle ABC$, find the coordinates of the centroid of the triangle.
\end{enumerate}
\solution
	\input{chapters/10/7/4/7/section.tex}
\item Find the slope of a line, which passes through the origin and the mid point of the line segment joining the points $\vec{P}$(0,-4) and $\vec{B}$(8,0).
\label{chapters/11/10/1/5}
\input{chapters/11/10/1/5/matrix.tex}
\item Find the position vector of a point R which divides the line joining two points P and Q whose position vectors are $(2\vec{a}+\vec{b})$ and $(\vec{a}-3\vec{b})$
externally in the ratio 1 : 2. Also, show that P is the mid point of the line segment RQ.\\
	\solution
%		\input{chapters/12/10/5/9/section.tex}

\end{enumerate}


\item Find the area of a rhombus if its vertices are $(3,0), (4,5), (-1,4) \text{ and } (-2,-1)$ taken in order. [$\vec{Hint}$ : Area of rhombus =$\frac {1}{2}$(product of its diagonals)]
	\\
		\solution
	\iffalse
\documentclass[12pt]{article}
\usepackage{graphicx}
%\documentclass[journal,12pt,twocolumn]{IEEEtran}
\usepackage[none]{hyphenat}
\usepackage{graphicx}
\usepackage{listings}
\usepackage[english]{babel}
\usepackage{graphicx}
\usepackage{caption} 
\usepackage{hyperref}
\usepackage{booktabs}
\def\inputGnumericTable{}
\usepackage{color}                                            %%
    \usepackage{array}                                            %%
    \usepackage{longtable}                                        %%
    \usepackage{calc}                                             %%
    \usepackage{multirow}                                         %%
    \usepackage{hhline}                                           %%
    \usepackage{ifthen}
\usepackage{array}
\usepackage{amsmath}   % for having text in math mode
\usepackage{listings}
\lstset{
language=tex,
frame=single, 
breaklines=true
}
  
%Following 2 lines were added to remove the blank page at the beginning
\usepackage{atbegshi}% http://ctan.org/pkg/atbegshi
\AtBeginDocument{\AtBeginShipoutNext{\AtBeginShipoutDiscard}}
%


%New macro definitions
\newcommand{\mydet}[1]{\ensuremath{\begin{vmatrix}#1\end{vmatrix}}}
\providecommand{\brak}[1]{\ensuremath{\left(#1\right)}}
\providecommand{\norm}[1]{\left\lVert#1\right\rVert}
\newcommand{\solution}{\noindent \textbf{Solution: }}
\newcommand{\myvec}[1]{\ensuremath{\begin{pmatrix}#1\end{pmatrix}}}
\let\vec\mathbf

\begin{document}

\begin{center}
\title{\textbf{Coordinate Geometry}}
\date{\vspace{-5ex}} %Not to print date automatically
\maketitle
\end{center}

\setcounter{page}{1}



\begin{enumerate}

\item\textbf{Problem statement :} Find the area of a rhombus of its vertices are $\myvec{3 ,0}$, $\myvec{4 ,5}$, $\myvec{-1 ,4}$ and $\myvec{-2 ,-1}$taken in order

\solution \\
\fi
The input vertices for this problem are given as
	\begin{align}
	\vec{A} = \myvec{
		3\\
		0
		},
	\vec{B} = \myvec{
		4\\
		5
		},
        \vec{C} = \myvec{
		-1\\
		4
		},
        \vec{D} = \myvec{
		-2\\
		-1
		}
	\end{align}
Since		
\begin{align}
 \vec{A-D}= \myvec{3 \\ 0} - \myvec{-2 \\-1}= \myvec{5\\1}
 \\
  \vec{B-A}= \myvec{4 \\ 5} - \myvec{3 \\0}= \myvec{1\\5}
\end{align}
the area of the rhombus is
\begin{align}
                \norm{\myvec{\vec{A-D}}\times \myvec{\vec{B-A}}}=\mydet{5 & 1\\1 & 5} = 24
\end{align}
See Fig. 
\ref{fig:chapters/10/7/2/10/gFig1}.
\begin{figure}[!h]
 \begin{center}
  \includegraphics[width=\columnwidth]{chapters/10/7/2/10/figs/fig.pdf}
 \end{center}
\caption{}
\label{fig:chapters/10/7/2/10/gFig1}
\end{figure}

\item Find the position vector of a point R which divides the line joining two points $\vec{P}$
and $\vec{Q}$ whose position vectors are $\hat{i}+2\hat{j}-\hat{k}$ and $-\hat{i}+\hat{j}+\hat{k}$ respectively, in the
ratio 2 : 1
\begin{enumerate}
    \item  internally
    \item  externally
\end{enumerate}
\solution
		\begin{enumerate}[label=\thesection.\arabic*,ref=\thesection.\theenumi]
\numberwithin{equation}{enumi}
\numberwithin{figure}{enumi}
\numberwithin{table}{enumi}

\item Find the coordinates of the point which divides the join of $(-1,7) \text{ and } (4,-3)$ in the ratio 2:3.
	\\
		\solution
	\input{chapters/10/7/2/1/section.tex}
\item Find the coordinates of the points of trisection of the line segment joining $(4,-1) \text{ and } (-2,3)$.
	\\
		\solution
	\input{chapters/10/7/2/2/section.tex}
\item
	\iffalse
\item To conduct Sports Day activities, in your rectangular shaped school                   
ground ABCD, lines have 
drawn with chalk powder at a                 
distance of 1m each. 100 flower pots have been placed at a distance of 1m 
from each other along AD, as shown 
in Fig. 7.12. Niharika runs $ \frac {1}{4} $th the 
distance AD on the 2nd line and 
posts a green flag. Preet runs $ \frac {1}{5} $th 
the distance AD on the eighth line 
and posts a red flag. What is the 
distance between both the flags? If 
Rashmi has to post a blue flag exactly 
halfway between the line segment 
joining the two flags, where should 
she post her flag?
\begin{figure}[h!]
  \centering
  \includegraphics[width=\columnwidth]{sc.png}
  \caption{}
\label{fig:10/7/12Fig1}
\end{figure}               
\fi
      
\item Find the ratio in which the line segment joining the points $(-3,10) \text{ and } (6,-8)$ $\text{ is divided by } (-1,6)$.
	\\
		\solution
	\input{chapters/10/7/2/4/section.tex}
\item Find the ratio in which the line segment joining $A(1,-5) \text{ and } B(-4,5)$ $\text{is divided by the x-axis}$. Also find the coordinates of the point of division.
\item If $(1,2), (4,y), (x,6), (3,5)$ are the vertices of a parallelogram taken in order, find x and y.
	\\
		\solution
	\input{chapters/10/7/2/6/para1.tex}
\item Find the coordinates of a point A, where AB is the diameter of a circle whose centre is $(2,-3) \text{ and }$ B is $(1,4)$.
	\\
		\solution
	\input{chapters/10/7/2/7/section.tex}
\item If A \text{ and } B are $(-2,-2) \text{ and } (2,-4)$, respectively, find the coordinates of P such that AP= $\frac {3}{7}$AB $\text{ and }$ P lies on the line segment AB.
	\\
		\solution
	\input{chapters/10/7/2/8/section.tex}
\item Find the coordinates of the points which divide the line segment joining $A(-2,2) \text{ and } B(2,8)$ into four equal parts.
	\\
		\solution
	\input{chapters/10/7/2/9/section.tex}
\item Find the area of a rhombus if its vertices are $(3,0), (4,5), (-1,4) \text{ and } (-2,-1)$ taken in order. [$\vec{Hint}$ : Area of rhombus =$\frac {1}{2}$(product of its diagonals)]
	\\
		\solution
	\input{chapters/10/7/2/10/cross.tex}
\item Find the position vector of a point R which divides the line joining two points $\vec{P}$
and $\vec{Q}$ whose position vectors are $\hat{i}+2\hat{j}-\hat{k}$ and $-\hat{i}+\hat{j}+\hat{k}$ respectively, in the
ratio 2 : 1
\begin{enumerate}
    \item  internally
    \item  externally
\end{enumerate}
\solution
		\input{chapters/12/10/2/15/section.tex}
\item Find the position vector of the mid point of the vector joining the points $\vec{P}$(2, 3, 4)
and $\vec{Q}$(4, 1, –2).
\\
\solution
		\input{chapters/12/10/2/16/section.tex}
\item Determine the ratio in which the line $2x+y  - 4=0$ divides the line segment joining the points $\vec{A}(2, - 2)$  and  $\vec{B}(3, 7)$.
\\
\solution
	\input{chapters/10/7/4/1/section.tex}
\item Let $\vec{A}(4, 2), \vec{B}(6, 5)$  and $ \vec{C}(1, 4)$ be the vertices of $\triangle ABC$.
\begin{enumerate}
\item The median from $\vec{A}$ meets $BC$ at $\vec{D}$. Find the coordinates of the point $\vec{D}$.
\item Find the coordinates of the point $\vec{P}$ on $AD$ such that $AP : PD = 2 : 1$.
\item Find the coordinates of points $\vec{Q}$ and $\vec{R}$ on medians $BE$ and $CF$ respectively such that $BQ : QE = 2 : 1$  and  $CR : RF = 2 : 1$.
\item What do you observe?
\item If $\vec{A}, \vec{B}$ and $\vec{C}$  are the vertices of $\triangle ABC$, find the coordinates of the centroid of the triangle.
\end{enumerate}
\solution
	\input{chapters/10/7/4/7/section.tex}
\item Find the slope of a line, which passes through the origin and the mid point of the line segment joining the points $\vec{P}$(0,-4) and $\vec{B}$(8,0).
\label{chapters/11/10/1/5}
\input{chapters/11/10/1/5/matrix.tex}
\item Find the position vector of a point R which divides the line joining two points P and Q whose position vectors are $(2\vec{a}+\vec{b})$ and $(\vec{a}-3\vec{b})$
externally in the ratio 1 : 2. Also, show that P is the mid point of the line segment RQ.\\
	\solution
%		\input{chapters/12/10/5/9/section.tex}

\end{enumerate}


\item Find the position vector of the mid point of the vector joining the points $\vec{P}$(2, 3, 4)
and $\vec{Q}$(4, 1, –2).
\\
\solution
		\begin{enumerate}[label=\thesection.\arabic*,ref=\thesection.\theenumi]
\numberwithin{equation}{enumi}
\numberwithin{figure}{enumi}
\numberwithin{table}{enumi}

\item Find the coordinates of the point which divides the join of $(-1,7) \text{ and } (4,-3)$ in the ratio 2:3.
	\\
		\solution
	\input{chapters/10/7/2/1/section.tex}
\item Find the coordinates of the points of trisection of the line segment joining $(4,-1) \text{ and } (-2,3)$.
	\\
		\solution
	\input{chapters/10/7/2/2/section.tex}
\item
	\iffalse
\item To conduct Sports Day activities, in your rectangular shaped school                   
ground ABCD, lines have 
drawn with chalk powder at a                 
distance of 1m each. 100 flower pots have been placed at a distance of 1m 
from each other along AD, as shown 
in Fig. 7.12. Niharika runs $ \frac {1}{4} $th the 
distance AD on the 2nd line and 
posts a green flag. Preet runs $ \frac {1}{5} $th 
the distance AD on the eighth line 
and posts a red flag. What is the 
distance between both the flags? If 
Rashmi has to post a blue flag exactly 
halfway between the line segment 
joining the two flags, where should 
she post her flag?
\begin{figure}[h!]
  \centering
  \includegraphics[width=\columnwidth]{sc.png}
  \caption{}
\label{fig:10/7/12Fig1}
\end{figure}               
\fi
      
\item Find the ratio in which the line segment joining the points $(-3,10) \text{ and } (6,-8)$ $\text{ is divided by } (-1,6)$.
	\\
		\solution
	\input{chapters/10/7/2/4/section.tex}
\item Find the ratio in which the line segment joining $A(1,-5) \text{ and } B(-4,5)$ $\text{is divided by the x-axis}$. Also find the coordinates of the point of division.
\item If $(1,2), (4,y), (x,6), (3,5)$ are the vertices of a parallelogram taken in order, find x and y.
	\\
		\solution
	\input{chapters/10/7/2/6/para1.tex}
\item Find the coordinates of a point A, where AB is the diameter of a circle whose centre is $(2,-3) \text{ and }$ B is $(1,4)$.
	\\
		\solution
	\input{chapters/10/7/2/7/section.tex}
\item If A \text{ and } B are $(-2,-2) \text{ and } (2,-4)$, respectively, find the coordinates of P such that AP= $\frac {3}{7}$AB $\text{ and }$ P lies on the line segment AB.
	\\
		\solution
	\input{chapters/10/7/2/8/section.tex}
\item Find the coordinates of the points which divide the line segment joining $A(-2,2) \text{ and } B(2,8)$ into four equal parts.
	\\
		\solution
	\input{chapters/10/7/2/9/section.tex}
\item Find the area of a rhombus if its vertices are $(3,0), (4,5), (-1,4) \text{ and } (-2,-1)$ taken in order. [$\vec{Hint}$ : Area of rhombus =$\frac {1}{2}$(product of its diagonals)]
	\\
		\solution
	\input{chapters/10/7/2/10/cross.tex}
\item Find the position vector of a point R which divides the line joining two points $\vec{P}$
and $\vec{Q}$ whose position vectors are $\hat{i}+2\hat{j}-\hat{k}$ and $-\hat{i}+\hat{j}+\hat{k}$ respectively, in the
ratio 2 : 1
\begin{enumerate}
    \item  internally
    \item  externally
\end{enumerate}
\solution
		\input{chapters/12/10/2/15/section.tex}
\item Find the position vector of the mid point of the vector joining the points $\vec{P}$(2, 3, 4)
and $\vec{Q}$(4, 1, –2).
\\
\solution
		\input{chapters/12/10/2/16/section.tex}
\item Determine the ratio in which the line $2x+y  - 4=0$ divides the line segment joining the points $\vec{A}(2, - 2)$  and  $\vec{B}(3, 7)$.
\\
\solution
	\input{chapters/10/7/4/1/section.tex}
\item Let $\vec{A}(4, 2), \vec{B}(6, 5)$  and $ \vec{C}(1, 4)$ be the vertices of $\triangle ABC$.
\begin{enumerate}
\item The median from $\vec{A}$ meets $BC$ at $\vec{D}$. Find the coordinates of the point $\vec{D}$.
\item Find the coordinates of the point $\vec{P}$ on $AD$ such that $AP : PD = 2 : 1$.
\item Find the coordinates of points $\vec{Q}$ and $\vec{R}$ on medians $BE$ and $CF$ respectively such that $BQ : QE = 2 : 1$  and  $CR : RF = 2 : 1$.
\item What do you observe?
\item If $\vec{A}, \vec{B}$ and $\vec{C}$  are the vertices of $\triangle ABC$, find the coordinates of the centroid of the triangle.
\end{enumerate}
\solution
	\input{chapters/10/7/4/7/section.tex}
\item Find the slope of a line, which passes through the origin and the mid point of the line segment joining the points $\vec{P}$(0,-4) and $\vec{B}$(8,0).
\label{chapters/11/10/1/5}
\input{chapters/11/10/1/5/matrix.tex}
\item Find the position vector of a point R which divides the line joining two points P and Q whose position vectors are $(2\vec{a}+\vec{b})$ and $(\vec{a}-3\vec{b})$
externally in the ratio 1 : 2. Also, show that P is the mid point of the line segment RQ.\\
	\solution
%		\input{chapters/12/10/5/9/section.tex}

\end{enumerate}


\item Determine the ratio in which the line $2x+y  - 4=0$ divides the line segment joining the points $\vec{A}(2, - 2)$  and  $\vec{B}(3, 7)$.
\\
\solution
	\iffalse
\documentclass[journal,12pt,twocolumn]{IEEEtran}
\usepackage{graphicx}
\graphicspath{{./chapters/10/7/4/1/figs/}}{}
\usepackage{amsmath,amssymb,amsfonts,amsthm}
\newcommand{\myvec}[1]{\ensuremath{\begin{pmatrix}#1\end{pmatrix}}}
\providecommand{\norm}[1]{\lVert#1\rVert}
\usepackage{listings}
\usepackage{watermark}
\usepackage{titlesec}
\usepackage{caption}
\let\vec\mathbf
\lstset{
frame=single, 
breaklines=true,
columns=fullflexible
}
\thiswatermark{\centering \put(0,-105.0){\includegraphics[scale=0.15]{/sdcard/IITH/vector/vectpr-4/chapters/10/7/4/1/figs/logo.png}} }
\title{\mytitle}
\title{
Assignment - Vector-4
}
\author{Surajit Sarkar}
\begin{document}
\maketitle
%\tableofcontents
\bigskip
\section{\textbf{Problem}}
Determine the ratio in which the line 2x+y–4=0 divides the line segment joining the points A(2,–2) and B(3,7).
\section{\textbf{Solution}}
\begin{table}[h]
    \centering
    \begin{tabular}{|c|c|}
       \hline
       \textbf{Symbol}&\textbf{Value}  \\
       \hline
	    $\vec{A}$ & $\myvec{2\\-2}$\\
        \hline
	    $\vec{B}$ & $\myvec{3\\7}$\\
        \hline
	    c&$4$\\
        \hline
       $\vec{n}$ & $\myvec{2\\1}$\\
       \hline
    \end{tabular}
    \caption{Parameters}
    \label{tab:my_label}
\end{table}
Given equation
\fi
The given equation can be expressed as
\begin{align}
    \myvec{2&1}\vec{x}&=4\\
\end{align}
Using section formula, the point of division 
\begin{align}
    \vec{P} = \frac{k\vec{B+A}}{k+1}
\end{align}
which upon substitution in the equation of a line yields
\begin{align}
    \implies\vec{n}^{\top}\myvec{\frac{k\vec{B+A}}{k+1}}&=c\\
    \implies k&=\frac{c-\vec{n}^{\top}\vec{A}}{\vec{n}^{\top}\vec{B}-c}\\
\end{align}
upon simplification.  Substituting numerical values, 
\begin{align}
    k=\frac{2}{9}
\end{align}
See Fig. 
\ref{fig:chapters/10/7/4/1vec}.
\begin{figure}[!h]
\centering
\includegraphics[width=\columnwidth]{chapters/10/7/4/1/figs/vec.pdf}
\caption{}
\label{fig:chapters/10/7/4/1vec}
\end{figure}


\item Let $\vec{A}(4, 2), \vec{B}(6, 5)$  and $ \vec{C}(1, 4)$ be the vertices of $\triangle ABC$.
\begin{enumerate}
\item The median from $\vec{A}$ meets $BC$ at $\vec{D}$. Find the coordinates of the point $\vec{D}$.
\item Find the coordinates of the point $\vec{P}$ on $AD$ such that $AP : PD = 2 : 1$.
\item Find the coordinates of points $\vec{Q}$ and $\vec{R}$ on medians $BE$ and $CF$ respectively such that $BQ : QE = 2 : 1$  and  $CR : RF = 2 : 1$.
\item What do you observe?
\item If $\vec{A}, \vec{B}$ and $\vec{C}$  are the vertices of $\triangle ABC$, find the coordinates of the centroid of the triangle.
\end{enumerate}
\solution
	\iffalse
\documentclass[12pt]{article}
\usepackage{graphicx}
\usepackage[none]{hyphenat}
\usepackage{graphicx}
\usepackage{listings}
\usepackage[english]{babel}
\usepackage{graphicx}
\usepackage{caption} 
\usepackage{booktabs}
\usepackage{array}
\usepackage{amssymb} % for \because
\usepackage{amsmath}   % for having text in math mode
\usepackage{extarrows} % for Row operations arrows
\usepackage{listings}
\usepackage[utf8]{inputenc}
\lstset{
  frame=single,
  breaklines=true
}
\usepackage{hyperref}
  
%Following 2 lines were added to remove the blank page at the beginning
\usepackage{atbegshi}% http://ctan.org/pkg/atbegshi
\AtBeginDocument{\AtBeginShipoutNext{\AtBeginShipoutDiscard}}


%New macro definitions
\newcommand{\mydet}[1]{\ensuremath{\begin{vmatrix}#1\end{vmatrix}}}
\providecommand{\brak}[1]{\ensuremath{\left(#1\right)}}
\newcommand{\solution}{\noindent \textbf{Solution: }}
\newcommand{\myvec}[1]{\ensuremath{\begin{pmatrix}#1\end{pmatrix}}}
\providecommand{\norm}[1]{\left\lVert#1\right\rVert}
\providecommand{\abs}[1]{\left\vert#1\right\vert}
\let\vec\mathbf

\begin{document}

\begin{center}
\title{\textbf{VECTORS}}
\date{\vspace{-5ex}} %Not to print date automatically
\maketitle
\end{center}

\section{10$^{th}$ Maths - EXERCISE-7.4}

Let A(4, 2), B(6, 5) and C(1, 4) be the vertices of $\triangle ABC$
\begin{enumerate}
\item The median from A meets BC at D. Find the coordinates of the point D.
\item Find the coordinates of the point P on AD such that $AP : PD = 2 : 1$
\item Find the coordinates of points Q and R on medians BE and CF respectively such
that $BQ : QE = 2 : 1 \text{and} CR : RF = 2 : 1.$
\item What do yo observe?
\item If $A(x_1, y_1), B(x_2, y_2) \text{and} C(x_3, y_3)$ are the vertices of $\triangle ABC$, find the coordinates of the centroid of the triangle.
\end{enumerate}

Given points are
\begin{align}
\vec{A}=\myvec{4\\ 2} ,
\vec{B}=\myvec{6\\ 5} ,
\vec{C}=\myvec{1\\ 4}
\end{align}
\fi

\begin{enumerate}
\item 
\begin{align}
\vec{D}&=\frac{\vec{B}+\vec{C}}{2}\\
&=\myvec{\frac{7}{2}\\[2pt] \frac{9}{2}}\\
\vec{E}&=\frac{\vec{A}+\vec{C}}{2}\\
&=\myvec{\frac{5}{2}\\ 3}\\
\vec{F}&=\frac{\vec{A}+\vec{B}}{2}\\
&=\myvec{5\\ \frac{7}{2}}
\end{align}

\item 
	For
$n=2$,
\begin{align}
\vec{P}&=\frac{1}{1+n}\brak{\myvec{\vec{A}+n\vec{D}}}\\
&=\frac{1}{3}\myvec{11\\11}
\end{align}

\item 
\begin{align}
\vec{Q}&=\frac{1}{1+n}\brak{\myvec{\vec{B}+n\vec{E}}}\\
&=\frac{1}{3}\myvec{11\\11}\\
\vec{R}&=\frac{1}{1+n}\brak{\myvec{\vec{C}+n\vec{F}}}\\
&=\frac{1}{3}\myvec{11\\11}\\
\end{align}

\item 
 $\vec{P},\vec{Q},\vec{R}$ are the same point.
   
\item 
\begin{align}
\vec{G}&=\frac{\vec{D}+\vec{E}+\vec{F}}{3}\\
&=\frac{1}{3}\myvec{11\\11}\\
\end{align} 
\end{enumerate}
See Fig.  
  \ref{fig:chapters/10/7/4/7/Figure}.
\begin{figure}[h!]
\centering
\includegraphics[width=\columnwidth]{chapters/10/7/4/7/figs/dj.pdf}
\caption{}
  \label{fig:chapters/10/7/4/7/Figure}
\end{figure}

\item Find the slope of a line, which passes through the origin and the mid point of the line segment joining the points $\vec{P}$(0,-4) and $\vec{B}$(8,0).
\label{chapters/11/10/1/5}
\iffalse
\documentclass[journal,12pt,twocolumn]{IEEEtran}
\usepackage{graphicx}
\graphicspath{{./figs/}}{}
\usepackage{amsmath,amssymb,amsfonts,amsthm}
\newcommand{\myvec}[1]{\ensuremath{\begin{pmatrix}#1\end{pmatrix}}}

\let\vec\mathbf

\title{
Matrix-Lines
}
\author{Jyothsna Paluchuri-FWC22059\\}
\begin{document}
\maketitle
\tableofcontents
\bigskip
\section{Problem Statement}
\fi
	\begin{figure}[!ht]
		\centering
 \includegraphics[width=\columnwidth]{chapters/11/10/1/5/figs/line.png}
		\caption{}
		\label{fig:11/10/1/5}
  	\end{figure}
	\\
	\solution
\iffalse
\section{Construction}
\begin{figure}[h]
    \centering
\includegraphics[width=\columnwidth]{line.png}
    \caption{Equation of the slope}
    \label{fig:my_label}
\end{figure}
\vspace{2cm}
\begin{table}[h]
    \centering
    \begin{tabular}{|c|c|c|c|}
       \hline
       \textbf{Symbol}&\textbf{Value}&\textbf{Description}  \\
       \hline
	    $\vec{P}$ & $\myvec{
		    0\\
		    -4}$
	    & Point on Y-axis\\
        \hline
	    $\vec{B}$ & $\myvec{8\\0}$
 & Point on X-axis\\
        \hline
	    $\vec{0}$ & $\myvec{0\\0}$
 & Origin\\
        \hline
    \end{tabular}
    \caption{Parameters}
    \label{tab:my_label}
\end{table}


\section{Solution}
Given that resultant line passes through origin and mid point of the line segment joining point P(0,-4) and B(8,0) \\
\\
\\
given ${\vec{P}}$=$\myvec{
  0\\
  -4}$
 , ${\vec{B}}$=$\myvec{
  8\\
  0}$
  
 \fi 
The mid point of $PB$ is
\begin{align}
\vec{M} &=\frac{1}{2}(\vec{P}+\vec{B})
	= \myvec{4 \\ -2}  
\end{align}
The direction vector of line joining $\vec{O}, \vec{M}$ is 
\begin{align}
\vec{m}&=\vec{O}-\vec{M}
 = -\vec{M}
\end{align}
which can be expressed as
\begin{align}
	\myvec{1 \\ -\frac{1}{2}}
\end{align}
Thus the slope is
\begin{align}
	m = -\frac{1}{2}
\end{align}
\iffalse
\textbf{The direction vector of a line expressed as}
\begin{align}
\implies\vec{m} &= \begin{pmatrix}1 \\ m \\ \end{pmatrix}
\end{align}

\textbf{By solving equation (5) and (6),we get the slope of $\vec{O}$ $\vec{M}$ line}
\begin{align}
        \boxed{m=-0.5}
 \end{align}

\section{Software}
Download the following code using,
\begin{table}[h]
    \centering
    \begin{tabular}{|c|}
    \hline \\
   https://github.com/jyothsna777/jyothsna-fwc.git  \\
         \\
\hline
    \end{tabular}
\end{table}
\\
and execute the code by using command
\begin{center}
\textbf{Python3 lines.py}\\
\end{center}

\section{Conclusion}
Hence the slope of line $\vec{O}$ $\vec{M}$ lineis $\vec{m}$=-0.5

\end{document}
\fi

\item Find the position vector of a point R which divides the line joining two points P and Q whose position vectors are $(2\vec{a}+\vec{b})$ and $(\vec{a}-3\vec{b})$
externally in the ratio 1 : 2. Also, show that P is the mid point of the line segment RQ.\\
	\solution
%		\begin{enumerate}[label=\thesection.\arabic*,ref=\thesection.\theenumi]
\numberwithin{equation}{enumi}
\numberwithin{figure}{enumi}
\numberwithin{table}{enumi}

\item Find the coordinates of the point which divides the join of $(-1,7) \text{ and } (4,-3)$ in the ratio 2:3.
	\\
		\solution
	\input{chapters/10/7/2/1/section.tex}
\item Find the coordinates of the points of trisection of the line segment joining $(4,-1) \text{ and } (-2,3)$.
	\\
		\solution
	\input{chapters/10/7/2/2/section.tex}
\item
	\iffalse
\item To conduct Sports Day activities, in your rectangular shaped school                   
ground ABCD, lines have 
drawn with chalk powder at a                 
distance of 1m each. 100 flower pots have been placed at a distance of 1m 
from each other along AD, as shown 
in Fig. 7.12. Niharika runs $ \frac {1}{4} $th the 
distance AD on the 2nd line and 
posts a green flag. Preet runs $ \frac {1}{5} $th 
the distance AD on the eighth line 
and posts a red flag. What is the 
distance between both the flags? If 
Rashmi has to post a blue flag exactly 
halfway between the line segment 
joining the two flags, where should 
she post her flag?
\begin{figure}[h!]
  \centering
  \includegraphics[width=\columnwidth]{sc.png}
  \caption{}
\label{fig:10/7/12Fig1}
\end{figure}               
\fi
      
\item Find the ratio in which the line segment joining the points $(-3,10) \text{ and } (6,-8)$ $\text{ is divided by } (-1,6)$.
	\\
		\solution
	\input{chapters/10/7/2/4/section.tex}
\item Find the ratio in which the line segment joining $A(1,-5) \text{ and } B(-4,5)$ $\text{is divided by the x-axis}$. Also find the coordinates of the point of division.
\item If $(1,2), (4,y), (x,6), (3,5)$ are the vertices of a parallelogram taken in order, find x and y.
	\\
		\solution
	\input{chapters/10/7/2/6/para1.tex}
\item Find the coordinates of a point A, where AB is the diameter of a circle whose centre is $(2,-3) \text{ and }$ B is $(1,4)$.
	\\
		\solution
	\input{chapters/10/7/2/7/section.tex}
\item If A \text{ and } B are $(-2,-2) \text{ and } (2,-4)$, respectively, find the coordinates of P such that AP= $\frac {3}{7}$AB $\text{ and }$ P lies on the line segment AB.
	\\
		\solution
	\input{chapters/10/7/2/8/section.tex}
\item Find the coordinates of the points which divide the line segment joining $A(-2,2) \text{ and } B(2,8)$ into four equal parts.
	\\
		\solution
	\input{chapters/10/7/2/9/section.tex}
\item Find the area of a rhombus if its vertices are $(3,0), (4,5), (-1,4) \text{ and } (-2,-1)$ taken in order. [$\vec{Hint}$ : Area of rhombus =$\frac {1}{2}$(product of its diagonals)]
	\\
		\solution
	\input{chapters/10/7/2/10/cross.tex}
\item Find the position vector of a point R which divides the line joining two points $\vec{P}$
and $\vec{Q}$ whose position vectors are $\hat{i}+2\hat{j}-\hat{k}$ and $-\hat{i}+\hat{j}+\hat{k}$ respectively, in the
ratio 2 : 1
\begin{enumerate}
    \item  internally
    \item  externally
\end{enumerate}
\solution
		\input{chapters/12/10/2/15/section.tex}
\item Find the position vector of the mid point of the vector joining the points $\vec{P}$(2, 3, 4)
and $\vec{Q}$(4, 1, –2).
\\
\solution
		\input{chapters/12/10/2/16/section.tex}
\item Determine the ratio in which the line $2x+y  - 4=0$ divides the line segment joining the points $\vec{A}(2, - 2)$  and  $\vec{B}(3, 7)$.
\\
\solution
	\input{chapters/10/7/4/1/section.tex}
\item Let $\vec{A}(4, 2), \vec{B}(6, 5)$  and $ \vec{C}(1, 4)$ be the vertices of $\triangle ABC$.
\begin{enumerate}
\item The median from $\vec{A}$ meets $BC$ at $\vec{D}$. Find the coordinates of the point $\vec{D}$.
\item Find the coordinates of the point $\vec{P}$ on $AD$ such that $AP : PD = 2 : 1$.
\item Find the coordinates of points $\vec{Q}$ and $\vec{R}$ on medians $BE$ and $CF$ respectively such that $BQ : QE = 2 : 1$  and  $CR : RF = 2 : 1$.
\item What do you observe?
\item If $\vec{A}, \vec{B}$ and $\vec{C}$  are the vertices of $\triangle ABC$, find the coordinates of the centroid of the triangle.
\end{enumerate}
\solution
	\input{chapters/10/7/4/7/section.tex}
\item Find the slope of a line, which passes through the origin and the mid point of the line segment joining the points $\vec{P}$(0,-4) and $\vec{B}$(8,0).
\label{chapters/11/10/1/5}
\input{chapters/11/10/1/5/matrix.tex}
\item Find the position vector of a point R which divides the line joining two points P and Q whose position vectors are $(2\vec{a}+\vec{b})$ and $(\vec{a}-3\vec{b})$
externally in the ratio 1 : 2. Also, show that P is the mid point of the line segment RQ.\\
	\solution
%		\input{chapters/12/10/5/9/section.tex}

\end{enumerate}



\end{enumerate}


\item Find the area of a rhombus if its vertices are $(3,0), (4,5), (-1,4) \text{ and } (-2,-1)$ taken in order. [$\vec{Hint}$ : Area of rhombus =$\frac {1}{2}$(product of its diagonals)]
	\\
		\solution
	\iffalse
\documentclass[12pt]{article}
\usepackage{graphicx}
%\documentclass[journal,12pt,twocolumn]{IEEEtran}
\usepackage[none]{hyphenat}
\usepackage{graphicx}
\usepackage{listings}
\usepackage[english]{babel}
\usepackage{graphicx}
\usepackage{caption} 
\usepackage{hyperref}
\usepackage{booktabs}
\def\inputGnumericTable{}
\usepackage{color}                                            %%
    \usepackage{array}                                            %%
    \usepackage{longtable}                                        %%
    \usepackage{calc}                                             %%
    \usepackage{multirow}                                         %%
    \usepackage{hhline}                                           %%
    \usepackage{ifthen}
\usepackage{array}
\usepackage{amsmath}   % for having text in math mode
\usepackage{listings}
\lstset{
language=tex,
frame=single, 
breaklines=true
}
  
%Following 2 lines were added to remove the blank page at the beginning
\usepackage{atbegshi}% http://ctan.org/pkg/atbegshi
\AtBeginDocument{\AtBeginShipoutNext{\AtBeginShipoutDiscard}}
%


%New macro definitions
\newcommand{\mydet}[1]{\ensuremath{\begin{vmatrix}#1\end{vmatrix}}}
\providecommand{\brak}[1]{\ensuremath{\left(#1\right)}}
\providecommand{\norm}[1]{\left\lVert#1\right\rVert}
\newcommand{\solution}{\noindent \textbf{Solution: }}
\newcommand{\myvec}[1]{\ensuremath{\begin{pmatrix}#1\end{pmatrix}}}
\let\vec\mathbf

\begin{document}

\begin{center}
\title{\textbf{Coordinate Geometry}}
\date{\vspace{-5ex}} %Not to print date automatically
\maketitle
\end{center}

\setcounter{page}{1}



\begin{enumerate}

\item\textbf{Problem statement :} Find the area of a rhombus of its vertices are $\myvec{3 ,0}$, $\myvec{4 ,5}$, $\myvec{-1 ,4}$ and $\myvec{-2 ,-1}$taken in order

\solution \\
\fi
The input vertices for this problem are given as
	\begin{align}
	\vec{A} = \myvec{
		3\\
		0
		},
	\vec{B} = \myvec{
		4\\
		5
		},
        \vec{C} = \myvec{
		-1\\
		4
		},
        \vec{D} = \myvec{
		-2\\
		-1
		}
	\end{align}
Since		
\begin{align}
 \vec{A-D}= \myvec{3 \\ 0} - \myvec{-2 \\-1}= \myvec{5\\1}
 \\
  \vec{B-A}= \myvec{4 \\ 5} - \myvec{3 \\0}= \myvec{1\\5}
\end{align}
the area of the rhombus is
\begin{align}
                \norm{\myvec{\vec{A-D}}\times \myvec{\vec{B-A}}}=\mydet{5 & 1\\1 & 5} = 24
\end{align}
See Fig. 
\ref{fig:chapters/10/7/2/10/gFig1}.
\begin{figure}[!h]
 \begin{center}
  \includegraphics[width=\columnwidth]{chapters/10/7/2/10/figs/fig.pdf}
 \end{center}
\caption{}
\label{fig:chapters/10/7/2/10/gFig1}
\end{figure}

\item Find the position vector of a point R which divides the line joining two points $\vec{P}$
and $\vec{Q}$ whose position vectors are $\hat{i}+2\hat{j}-\hat{k}$ and $-\hat{i}+\hat{j}+\hat{k}$ respectively, in the
ratio 2 : 1
\begin{enumerate}
    \item  internally
    \item  externally
\end{enumerate}
\solution
		\begin{enumerate}[label=\thesection.\arabic*,ref=\thesection.\theenumi]
\numberwithin{equation}{enumi}
\numberwithin{figure}{enumi}
\numberwithin{table}{enumi}

\item Find the coordinates of the point which divides the join of $(-1,7) \text{ and } (4,-3)$ in the ratio 2:3.
	\\
		\solution
	\iffalse
\documentclass[12pt]{article}
\usepackage{graphicx}
\usepackage{amsmath}
\usepackage{mathtools}
\usepackage{gensymb}

\newcommand{\mydet}[1]{\ensuremath{\begin{vmatrix}#1\end{vmatrix}}}
\providecommand{\brak}[1]{\ensuremath{\left(#1\right)}}
\providecommand{\norm}[1]{\left\lVert#1\right\rVert}
\newcommand{\solution}{\noindent \textbf{Solution: }}
\newcommand{\myvec}[1]{\ensuremath{\begin{pmatrix}#1\end{pmatrix}}}
\let\vec\mathbf

\begin{document}
\begin{center}
\textbf\large{CHAPTER-7 \\ COORDINATE GEOMETRY}
\end{center}
\section*{Excercise 7.2}

1. Find the coordinates of the point which divides the join $\vec(-1,7) \text{ and } \vec(4,-3)$ in the ratio 2:3 :
\\
\\
\solution\\		
\fi
The coordinates and ratio are given as
\begin{align}
\vec{P}=\myvec{-1\\7\\},
\vec{Q}=\myvec{4\\-3\\},
n=\frac{3}{2}
\end{align}
Using section formula
\begin{align}
\vec{R}&=\frac{\vec{Q}+n\vec{P}}{1+n}\\
&=\frac{1}{1+\frac{3}{2}}  \myvec{\myvec{
4\\
-3\\
}
  +
   \frac{3}{2}\myvec{
-1\\
7\\
}}\\
&=\myvec{
1\\
3
}
\end{align}
See Fig. 
\ref{fig:chapters/10/7/2/1/Fig}
\begin{figure}[!h]
\begin{center}
   \includegraphics[width=\columnwidth]{chapters/10/7/2/1/figs/linefig.png}
\end{center}
\caption{}
\label{fig:chapters/10/7/2/1/Fig}
\end{figure}


\item Find the coordinates of the points of trisection of the line segment joining $(4,-1) \text{ and } (-2,3)$.
	\\
		\solution
	\begin{enumerate}[label=\thesection.\arabic*,ref=\thesection.\theenumi]
\numberwithin{equation}{enumi}
\numberwithin{figure}{enumi}
\numberwithin{table}{enumi}

\item Find the coordinates of the point which divides the join of $(-1,7) \text{ and } (4,-3)$ in the ratio 2:3.
	\\
		\solution
	\input{chapters/10/7/2/1/section.tex}
\item Find the coordinates of the points of trisection of the line segment joining $(4,-1) \text{ and } (-2,3)$.
	\\
		\solution
	\input{chapters/10/7/2/2/section.tex}
\item
	\iffalse
\item To conduct Sports Day activities, in your rectangular shaped school                   
ground ABCD, lines have 
drawn with chalk powder at a                 
distance of 1m each. 100 flower pots have been placed at a distance of 1m 
from each other along AD, as shown 
in Fig. 7.12. Niharika runs $ \frac {1}{4} $th the 
distance AD on the 2nd line and 
posts a green flag. Preet runs $ \frac {1}{5} $th 
the distance AD on the eighth line 
and posts a red flag. What is the 
distance between both the flags? If 
Rashmi has to post a blue flag exactly 
halfway between the line segment 
joining the two flags, where should 
she post her flag?
\begin{figure}[h!]
  \centering
  \includegraphics[width=\columnwidth]{sc.png}
  \caption{}
\label{fig:10/7/12Fig1}
\end{figure}               
\fi
      
\item Find the ratio in which the line segment joining the points $(-3,10) \text{ and } (6,-8)$ $\text{ is divided by } (-1,6)$.
	\\
		\solution
	\input{chapters/10/7/2/4/section.tex}
\item Find the ratio in which the line segment joining $A(1,-5) \text{ and } B(-4,5)$ $\text{is divided by the x-axis}$. Also find the coordinates of the point of division.
\item If $(1,2), (4,y), (x,6), (3,5)$ are the vertices of a parallelogram taken in order, find x and y.
	\\
		\solution
	\input{chapters/10/7/2/6/para1.tex}
\item Find the coordinates of a point A, where AB is the diameter of a circle whose centre is $(2,-3) \text{ and }$ B is $(1,4)$.
	\\
		\solution
	\input{chapters/10/7/2/7/section.tex}
\item If A \text{ and } B are $(-2,-2) \text{ and } (2,-4)$, respectively, find the coordinates of P such that AP= $\frac {3}{7}$AB $\text{ and }$ P lies on the line segment AB.
	\\
		\solution
	\input{chapters/10/7/2/8/section.tex}
\item Find the coordinates of the points which divide the line segment joining $A(-2,2) \text{ and } B(2,8)$ into four equal parts.
	\\
		\solution
	\input{chapters/10/7/2/9/section.tex}
\item Find the area of a rhombus if its vertices are $(3,0), (4,5), (-1,4) \text{ and } (-2,-1)$ taken in order. [$\vec{Hint}$ : Area of rhombus =$\frac {1}{2}$(product of its diagonals)]
	\\
		\solution
	\input{chapters/10/7/2/10/cross.tex}
\item Find the position vector of a point R which divides the line joining two points $\vec{P}$
and $\vec{Q}$ whose position vectors are $\hat{i}+2\hat{j}-\hat{k}$ and $-\hat{i}+\hat{j}+\hat{k}$ respectively, in the
ratio 2 : 1
\begin{enumerate}
    \item  internally
    \item  externally
\end{enumerate}
\solution
		\input{chapters/12/10/2/15/section.tex}
\item Find the position vector of the mid point of the vector joining the points $\vec{P}$(2, 3, 4)
and $\vec{Q}$(4, 1, –2).
\\
\solution
		\input{chapters/12/10/2/16/section.tex}
\item Determine the ratio in which the line $2x+y  - 4=0$ divides the line segment joining the points $\vec{A}(2, - 2)$  and  $\vec{B}(3, 7)$.
\\
\solution
	\input{chapters/10/7/4/1/section.tex}
\item Let $\vec{A}(4, 2), \vec{B}(6, 5)$  and $ \vec{C}(1, 4)$ be the vertices of $\triangle ABC$.
\begin{enumerate}
\item The median from $\vec{A}$ meets $BC$ at $\vec{D}$. Find the coordinates of the point $\vec{D}$.
\item Find the coordinates of the point $\vec{P}$ on $AD$ such that $AP : PD = 2 : 1$.
\item Find the coordinates of points $\vec{Q}$ and $\vec{R}$ on medians $BE$ and $CF$ respectively such that $BQ : QE = 2 : 1$  and  $CR : RF = 2 : 1$.
\item What do you observe?
\item If $\vec{A}, \vec{B}$ and $\vec{C}$  are the vertices of $\triangle ABC$, find the coordinates of the centroid of the triangle.
\end{enumerate}
\solution
	\input{chapters/10/7/4/7/section.tex}
\item Find the slope of a line, which passes through the origin and the mid point of the line segment joining the points $\vec{P}$(0,-4) and $\vec{B}$(8,0).
\label{chapters/11/10/1/5}
\input{chapters/11/10/1/5/matrix.tex}
\item Find the position vector of a point R which divides the line joining two points P and Q whose position vectors are $(2\vec{a}+\vec{b})$ and $(\vec{a}-3\vec{b})$
externally in the ratio 1 : 2. Also, show that P is the mid point of the line segment RQ.\\
	\solution
%		\input{chapters/12/10/5/9/section.tex}

\end{enumerate}


\item
	\iffalse
\item To conduct Sports Day activities, in your rectangular shaped school                   
ground ABCD, lines have 
drawn with chalk powder at a                 
distance of 1m each. 100 flower pots have been placed at a distance of 1m 
from each other along AD, as shown 
in Fig. 7.12. Niharika runs $ \frac {1}{4} $th the 
distance AD on the 2nd line and 
posts a green flag. Preet runs $ \frac {1}{5} $th 
the distance AD on the eighth line 
and posts a red flag. What is the 
distance between both the flags? If 
Rashmi has to post a blue flag exactly 
halfway between the line segment 
joining the two flags, where should 
she post her flag?
\begin{figure}[h!]
  \centering
  \includegraphics[width=\columnwidth]{sc.png}
  \caption{}
\label{fig:10/7/12Fig1}
\end{figure}               
\fi
      
\item Find the ratio in which the line segment joining the points $(-3,10) \text{ and } (6,-8)$ $\text{ is divided by } (-1,6)$.
	\\
		\solution
	\iffalse
\documentclass[12pt]{article}
\usepackage{graphicx}
%\documentclass[journal,12pt,twocolumn]{IEEEtran}
\usepackage[none]{hyphenat}
\usepackage{graphicx}
\usepackage{listings}
\usepackage[english]{babel}
\usepackage{graphicx}
\usepackage{caption} 
\usepackage{hyperref}
\usepackage{booktabs}
\def\inputGnumericTable{}
\usepackage{color}                                            %%
    \usepackage{array}                                            %%
    \usepackage{longtable}                                        %%
    \usepackage{calc}                                             %%
    \usepackage{multirow}                                         %%
    \usepackage{hhline}                                           %%
    \usepackage{ifthen}
\usepackage{array}
\usepackage{amsmath}   % for having text in math mode
\usepackage{listings}
\lstset{
language=tex,
frame=single, 
breaklines=true
}
  
%Following 2 lines were added to remove the blank page at the beginning
\usepackage{atbegshi}% http://ctan.org/pkg/atbegshi
\AtBeginDocument{\AtBeginShipoutNext{\AtBeginShipoutDiscard}}
%
%New macro definitions
\newcommand{\mydet}[1]{\ensuremath{\begin{vmatrix}#1\end{vmatrix}}}
\providecommand{\brak}[1]{\ensuremath{\left(#1\right)}}
\providecommand{\norm}[1]{\left\lVert#1\right\rVert}
\newcommand{\solution}{\noindent \textbf{Solution: }}
\newcommand{\myvec}[1]{\ensuremath{\begin{pmatrix}#1\end{pmatrix}}}
\let\vec\mathbf
\begin{document}
\begin{center}
\title{\textbf{Coordinate Geometry}}
\date{\vspace{-5ex}} %Not to print date automatically
\maketitle
\end{center}
\setcounter{page}{1}
\section*{10$^{th}$ Maths - Chapter 7}
This is Problem-4 from Exercise 7.2
\begin{enumerate}
\item Find the ratio in which the line segement joining the points $\myvec{-3 \\ 10}$ and $\myvec{6\\-8}$ is divided by $\myvec{-1\\6}$.\\
\solution \\
\fi
		The input parameters for this problem are available in Table \eqref{tab:10/7/2/4-1}.
\begin{table}[ht!]
\input{chapters/10/7/2/4/tables/table.tex}
\caption{}
\label{tab:10/7/2/4-1} 
\end{table}
Using section formula,
\begin{align}
         \vec{R} &=\frac{\vec{Q}+n\vec{P}}{1+n}\label{eq:chapters/10/7/2/4/1}
\end{align}
Substituting the values of $\vec{P},\vec{Q}$ and $\vec{R}$ in \eqref{eq:chapters/10/7/2/4/1}
\begin{align}
         \myvec{-1\\6} &=\frac{{\myvec{-3\\10}+n\myvec{6\\-8}}}{1+n}\\
 &=\frac{1}{1+n}\brak{{\myvec{-3\\10}+n\myvec{6\\-8}}} \\
 &=\frac{1}{1+n}\myvec{-3+6n\\10-8n} \label{eq:chapters/10/7/2/4/4}
\end{align}
Simplifying \eqref{eq:chapters/10/7/2/4/4} yeilds,
\begin{align}
          -1 &=\frac{-3+6n}{1+n}\\
\implies          n &=\frac{2}{7}
\end{align}
Also,
\begin{align}
          6 &=\frac{10-8n}{1+n}\\
    \implies      n &=\frac{2}{7}
\end{align}
Hence the desired ratio is $\dfrac{2}{7}$.  
\begin{figure}[!h]
 \begin{center}
  \includegraphics[width=\columnwidth]{chapters/10/7/2/4/figs/fig.png}
 \end{center}
\caption{}
\label{fig:10/7/2/4Fig1}
\end{figure}

\item Find the ratio in which the line segment joining $A(1,-5) \text{ and } B(-4,5)$ $\text{is divided by the x-axis}$. Also find the coordinates of the point of division.
\item If $(1,2), (4,y), (x,6), (3,5)$ are the vertices of a parallelogram taken in order, find x and y.
	\\
		\solution
	\iffalse
\documentclass[12pt]{article}
\usepackage{graphicx}
%\documentclass[journal,12pt,twocolumn]{IEEEtran}
\def\inputGnumericTable{}
\usepackage{color}                                            %%
    \usepackage{array}                                            %%
    \usepackage{longtable}                                        %%
    \usepackage{calc}                                             %%
    \usepackage{multirow}                                         %%
    \usepackage{hhline}                                           %%
    \usepackage{ifthen}
\usepackage[none]{hyphenat}
\usepackage{graphicx}
\usepackage{listings}
\usepackage[english]{babel}
\usepackage{graphicx}
\usepackage{caption} 
\usepackage{hyperref}
\usepackage{booktabs}
\usepackage{array}
\usepackage{amsmath}   % for having text in math mode
\usepackage{listings}
\lstset{
  frame=single,
  breaklines=true
}
  
%Following 2 lines were added to remove the blank page at the beginning
\usepackage{atbegshi}% http://ctan.org/pkg/atbegshi
\AtBeginDocument{\AtBeginShipoutNext{\AtBeginShipoutDiscard}}
%


%New macro definitions
\newcommand{\mydet}[1]{\ensuremath{\begin{vmatrix}#1\end{vmatrix}}}
\providecommand{\brak}[1]{\ensuremath{\left(#1\right)}}
\providecommand{\norm}[1]{\left\lVert#1\right\rVert}
\newcommand{\solution}{\noindent \textbf{Solution: }}
\newcommand{\myvec}[1]{\ensuremath{\begin{pmatrix}#1\end{pmatrix}}}
\let\vec\mathbf

\begin{document}

\begin{center}
\title{\textbf{Properties of Parallelegram}}
\date{\vspace{-5ex}} %Not to print date automatically
\maketitle
\end{center}

\setcounter{page}{1}

\section{10$^{th}$ Maths - Chapter 7}

This is Problem-6 from Exercise 7.2

\begin{enumerate}
\item If $\vec{A}(1, 2),\vec{B}(4, x),\vec{C}(y, 6) \text{and } \vec{D}(3, 5)$ are the vertices of a parallelogram taken in order,find x and y.
\end{enumerate}
\fi

The input parameters for this problem are available in
\ref{table:chapters/10/7/2/6/tables/}.	
\begin{table}[!ht]
	\centering
	\input{chapters/10/7/2/6/tables/table.tex}
\caption{}
\label{table:chapters/10/7/2/6/tables/}	
\end{table}
From the given information,
\begin{align}
  \label{eq:chapters/10/7/2/6/tables/det2f}
	\vec{B}-\vec{A} &= \myvec{4 \\y } - \myvec{1 \\2 }  = \myvec{3 \\y-2 }\\
	\vec{C}-\vec{D} &= \myvec{x \\6 } - \myvec{3 \\5 }  = \myvec{x-3 \\1}
\end{align}
Since $ABCD$ is a parallellogram,
\begin{align}
	\myvec{3\\y-2}&=\myvec{x-3\\1}\\
	\implies x&=6 ,y=3
\end{align}
Fig. \ref{fig:chapters/10/7/2/6/Fig3}
provides a verification.
\begin{figure}[h!]
	\begin{center}
  \includegraphics[width=\columnwidth]{chapters/10/7/2/6/figs/para.pdf}
	\end{center}
\caption{}
\label{fig:chapters/10/7/2/6/Fig3}
\end{figure}


\item Find the coordinates of a point A, where AB is the diameter of a circle whose centre is $(2,-3) \text{ and }$ B is $(1,4)$.
	\\
		\solution
	\iffalse
\documentclass[12pt]{article}
\usepackage{graphicx}
\usepackage{amsmath}
\usepackage{mathtools}
\usepackage{gensymb}

\newcommand{\mydet}[1]{\ensuremath{\begin{vmatrix}#1\end{vmatrix}}}
\providecommand{\brak}[1]{\ensuremath{\left(#1\right)}}
\providecommand{\norm}[1]{\left\lVert#1\right\rVert}
\newcommand{\solution}{\noindent \textbf{Solution: }}
\newcommand{\myvec}[1]{\ensuremath{\begin{pmatrix}#1\end{pmatrix}}}
\let\vec\mathbf

\begin{document}
\begin{center}
\section*{CHAPTER 7 - COORDINATE GEOMETRY}

\end{center}
\section*{Excercise 7.2}

Q7.Find the coordinates of point $\vec{A}$, where AB is the diameter of a circle where the center is (2,-3) and $\vec{B}$ is the point (1,4):

\solution
\begin{enumerate}
\item The coordinates $\vec{B}$ and center $\vec{C}$ are given, where:
	\fi
	Let
	\begin{align}
	\vec{B} = \myvec{
		1\\
	    4\\
		},
	\vec{C} = \myvec{
	    2\\
	   -3\\
		}
	\end{align}
	\iffalse
Let us assume the coordinates of $\vec{A}$. Now, $\vec{C}$ is the center which is midpoint of line AB and $\vec{B}$ is one of the coordinate of diameter AB of a circle.
	\fi	
Hence,	
	\begin{align}
	\vec{C} &= \frac{\vec{A+B}}{2} \\
\implies	2\vec{C} &= \vec{A}+\vec{B} \\
		\text{or, }	\vec{A} &= 2\vec{C}-\vec{B} \\
	 &= \myvec{3\\-10\\}	
	\end{align}       
	See Fig. 
\ref{fig:chapters/10/7/2/7Fig}.
\begin{figure}[!h]
\begin{center}	
	\includegraphics[width=\columnwidth]{chapters/10/7/2/7/figs/Vector1.png}
\end{center}
\caption{}
\label{fig:chapters/10/7/2/7Fig}
\end{figure}
	

\item If A \text{ and } B are $(-2,-2) \text{ and } (2,-4)$, respectively, find the coordinates of P such that AP= $\frac {3}{7}$AB $\text{ and }$ P lies on the line segment AB.
	\\
		\solution
	\iffalse
\documentclass[journal,10pt,twocolumn]{article}
\usepackage{graphicx}
\usepackage[none]{hyphenat}
\usepackage{graphicx}
\usepackage{listings}
\usepackage[english]{babel}
\usepackage{graphicx}
\usepackage{caption} 
\usepackage{booktabs}
\usepackage{array}
\usepackage{amssymb} % for \because
\usepackage{amsmath}   % for having text in math mode
\usepackage{extarrows} % for Row operations arrows
\usepackage{listings}
\usepackage[utf8]{inputenc}
\lstset{
  frame=single,
  breaklines=true
}
\usepackage{hyperref}
  
%Following 2 lines were added to remove the blank page at the beginning
\usepackage{atbegshi}% http://ctan.org/pkg/atbegshi
\AtBeginDocument{\AtBeginShipoutNext{\AtBeginShipoutDiscard}}


%New macro definitions
\newcommand{\mydet}[1]{\ensuremath{\begin{vmatrix}#1\end{vmatrix}}}
\providecommand{\brak}[1]{\ensuremath{\left(#1\right)}}
\newcommand{\solution}{\noindent \textbf{Solution: }}
\newcommand{\myvec}[1]{\ensuremath{\begin{pmatrix}#1\end{pmatrix}}}
\providecommand{\norm}[1]{\left\lVert#1\right\rVert}
\providecommand{\abs}[1]{\left\vert#1\right\vert}
\let\vec\mathbf

\begin{document}

\begin{center}
\title{\textbf{VECTORS}}
\date{\vspace{-5ex}} %Not to print date automatically
\maketitle
\end{center}

\section{10$^{th}$ Maths - EXERCISE-7.2}

\begin{enumerate}
\item If A and B are $(– 2, – 2)\text{ and }(2, – 4)$, respectively, find the coordinates of P such that $AP =\frac{3}{7}AB$ and P lies on the line segment AB. 

\section{SOLUTION}
Given points are
\begin{align}
\vec{A}=\myvec{-2\\ -2} ,
\vec{B}=\myvec{2\\ -4}
\end{align}
The equation of the formula is
\fi
Using section formula, 
\begin{align}
\vec{P}&=\frac{\vec{A}+n\vec{B}}{1+n}
\end{align}
where
\begin{align}
	n =\frac{3}{4}
\end{align}
Thus,
\begin{align}
\vec{P}&=\frac{1}{1+\frac{3}{4}}\brak{\myvec{-2\\-2}+\frac{3}{4}\myvec{2\\-4}}\\
&=\myvec{\frac{-2}{7}\\[1pt] \frac{-20}{7}}
\end{align}
See Fig. 
   \ref{fig:chapters/10/7/2/8/vec.png}
\begin{figure}
   \centering 
 \includegraphics[width=\columnwidth]{chapters/10/7/2/8/figs/vec.png}
   \caption{}
   \label{fig:chapters/10/7/2/8/vec.png}
   \end{figure}

\item Find the coordinates of the points which divide the line segment joining $A(-2,2) \text{ and } B(2,8)$ into four equal parts.
	\\
		\solution
	\begin{enumerate}[label=\thesection.\arabic*,ref=\thesection.\theenumi]
\numberwithin{equation}{enumi}
\numberwithin{figure}{enumi}
\numberwithin{table}{enumi}

\item Find the coordinates of the point which divides the join of $(-1,7) \text{ and } (4,-3)$ in the ratio 2:3.
	\\
		\solution
	\input{chapters/10/7/2/1/section.tex}
\item Find the coordinates of the points of trisection of the line segment joining $(4,-1) \text{ and } (-2,3)$.
	\\
		\solution
	\input{chapters/10/7/2/2/section.tex}
\item
	\iffalse
\item To conduct Sports Day activities, in your rectangular shaped school                   
ground ABCD, lines have 
drawn with chalk powder at a                 
distance of 1m each. 100 flower pots have been placed at a distance of 1m 
from each other along AD, as shown 
in Fig. 7.12. Niharika runs $ \frac {1}{4} $th the 
distance AD on the 2nd line and 
posts a green flag. Preet runs $ \frac {1}{5} $th 
the distance AD on the eighth line 
and posts a red flag. What is the 
distance between both the flags? If 
Rashmi has to post a blue flag exactly 
halfway between the line segment 
joining the two flags, where should 
she post her flag?
\begin{figure}[h!]
  \centering
  \includegraphics[width=\columnwidth]{sc.png}
  \caption{}
\label{fig:10/7/12Fig1}
\end{figure}               
\fi
      
\item Find the ratio in which the line segment joining the points $(-3,10) \text{ and } (6,-8)$ $\text{ is divided by } (-1,6)$.
	\\
		\solution
	\input{chapters/10/7/2/4/section.tex}
\item Find the ratio in which the line segment joining $A(1,-5) \text{ and } B(-4,5)$ $\text{is divided by the x-axis}$. Also find the coordinates of the point of division.
\item If $(1,2), (4,y), (x,6), (3,5)$ are the vertices of a parallelogram taken in order, find x and y.
	\\
		\solution
	\input{chapters/10/7/2/6/para1.tex}
\item Find the coordinates of a point A, where AB is the diameter of a circle whose centre is $(2,-3) \text{ and }$ B is $(1,4)$.
	\\
		\solution
	\input{chapters/10/7/2/7/section.tex}
\item If A \text{ and } B are $(-2,-2) \text{ and } (2,-4)$, respectively, find the coordinates of P such that AP= $\frac {3}{7}$AB $\text{ and }$ P lies on the line segment AB.
	\\
		\solution
	\input{chapters/10/7/2/8/section.tex}
\item Find the coordinates of the points which divide the line segment joining $A(-2,2) \text{ and } B(2,8)$ into four equal parts.
	\\
		\solution
	\input{chapters/10/7/2/9/section.tex}
\item Find the area of a rhombus if its vertices are $(3,0), (4,5), (-1,4) \text{ and } (-2,-1)$ taken in order. [$\vec{Hint}$ : Area of rhombus =$\frac {1}{2}$(product of its diagonals)]
	\\
		\solution
	\input{chapters/10/7/2/10/cross.tex}
\item Find the position vector of a point R which divides the line joining two points $\vec{P}$
and $\vec{Q}$ whose position vectors are $\hat{i}+2\hat{j}-\hat{k}$ and $-\hat{i}+\hat{j}+\hat{k}$ respectively, in the
ratio 2 : 1
\begin{enumerate}
    \item  internally
    \item  externally
\end{enumerate}
\solution
		\input{chapters/12/10/2/15/section.tex}
\item Find the position vector of the mid point of the vector joining the points $\vec{P}$(2, 3, 4)
and $\vec{Q}$(4, 1, –2).
\\
\solution
		\input{chapters/12/10/2/16/section.tex}
\item Determine the ratio in which the line $2x+y  - 4=0$ divides the line segment joining the points $\vec{A}(2, - 2)$  and  $\vec{B}(3, 7)$.
\\
\solution
	\input{chapters/10/7/4/1/section.tex}
\item Let $\vec{A}(4, 2), \vec{B}(6, 5)$  and $ \vec{C}(1, 4)$ be the vertices of $\triangle ABC$.
\begin{enumerate}
\item The median from $\vec{A}$ meets $BC$ at $\vec{D}$. Find the coordinates of the point $\vec{D}$.
\item Find the coordinates of the point $\vec{P}$ on $AD$ such that $AP : PD = 2 : 1$.
\item Find the coordinates of points $\vec{Q}$ and $\vec{R}$ on medians $BE$ and $CF$ respectively such that $BQ : QE = 2 : 1$  and  $CR : RF = 2 : 1$.
\item What do you observe?
\item If $\vec{A}, \vec{B}$ and $\vec{C}$  are the vertices of $\triangle ABC$, find the coordinates of the centroid of the triangle.
\end{enumerate}
\solution
	\input{chapters/10/7/4/7/section.tex}
\item Find the slope of a line, which passes through the origin and the mid point of the line segment joining the points $\vec{P}$(0,-4) and $\vec{B}$(8,0).
\label{chapters/11/10/1/5}
\input{chapters/11/10/1/5/matrix.tex}
\item Find the position vector of a point R which divides the line joining two points P and Q whose position vectors are $(2\vec{a}+\vec{b})$ and $(\vec{a}-3\vec{b})$
externally in the ratio 1 : 2. Also, show that P is the mid point of the line segment RQ.\\
	\solution
%		\input{chapters/12/10/5/9/section.tex}

\end{enumerate}


\item Find the area of a rhombus if its vertices are $(3,0), (4,5), (-1,4) \text{ and } (-2,-1)$ taken in order. [$\vec{Hint}$ : Area of rhombus =$\frac {1}{2}$(product of its diagonals)]
	\\
		\solution
	\iffalse
\documentclass[12pt]{article}
\usepackage{graphicx}
%\documentclass[journal,12pt,twocolumn]{IEEEtran}
\usepackage[none]{hyphenat}
\usepackage{graphicx}
\usepackage{listings}
\usepackage[english]{babel}
\usepackage{graphicx}
\usepackage{caption} 
\usepackage{hyperref}
\usepackage{booktabs}
\def\inputGnumericTable{}
\usepackage{color}                                            %%
    \usepackage{array}                                            %%
    \usepackage{longtable}                                        %%
    \usepackage{calc}                                             %%
    \usepackage{multirow}                                         %%
    \usepackage{hhline}                                           %%
    \usepackage{ifthen}
\usepackage{array}
\usepackage{amsmath}   % for having text in math mode
\usepackage{listings}
\lstset{
language=tex,
frame=single, 
breaklines=true
}
  
%Following 2 lines were added to remove the blank page at the beginning
\usepackage{atbegshi}% http://ctan.org/pkg/atbegshi
\AtBeginDocument{\AtBeginShipoutNext{\AtBeginShipoutDiscard}}
%


%New macro definitions
\newcommand{\mydet}[1]{\ensuremath{\begin{vmatrix}#1\end{vmatrix}}}
\providecommand{\brak}[1]{\ensuremath{\left(#1\right)}}
\providecommand{\norm}[1]{\left\lVert#1\right\rVert}
\newcommand{\solution}{\noindent \textbf{Solution: }}
\newcommand{\myvec}[1]{\ensuremath{\begin{pmatrix}#1\end{pmatrix}}}
\let\vec\mathbf

\begin{document}

\begin{center}
\title{\textbf{Coordinate Geometry}}
\date{\vspace{-5ex}} %Not to print date automatically
\maketitle
\end{center}

\setcounter{page}{1}



\begin{enumerate}

\item\textbf{Problem statement :} Find the area of a rhombus of its vertices are $\myvec{3 ,0}$, $\myvec{4 ,5}$, $\myvec{-1 ,4}$ and $\myvec{-2 ,-1}$taken in order

\solution \\
\fi
The input vertices for this problem are given as
	\begin{align}
	\vec{A} = \myvec{
		3\\
		0
		},
	\vec{B} = \myvec{
		4\\
		5
		},
        \vec{C} = \myvec{
		-1\\
		4
		},
        \vec{D} = \myvec{
		-2\\
		-1
		}
	\end{align}
Since		
\begin{align}
 \vec{A-D}= \myvec{3 \\ 0} - \myvec{-2 \\-1}= \myvec{5\\1}
 \\
  \vec{B-A}= \myvec{4 \\ 5} - \myvec{3 \\0}= \myvec{1\\5}
\end{align}
the area of the rhombus is
\begin{align}
                \norm{\myvec{\vec{A-D}}\times \myvec{\vec{B-A}}}=\mydet{5 & 1\\1 & 5} = 24
\end{align}
See Fig. 
\ref{fig:chapters/10/7/2/10/gFig1}.
\begin{figure}[!h]
 \begin{center}
  \includegraphics[width=\columnwidth]{chapters/10/7/2/10/figs/fig.pdf}
 \end{center}
\caption{}
\label{fig:chapters/10/7/2/10/gFig1}
\end{figure}

\item Find the position vector of a point R which divides the line joining two points $\vec{P}$
and $\vec{Q}$ whose position vectors are $\hat{i}+2\hat{j}-\hat{k}$ and $-\hat{i}+\hat{j}+\hat{k}$ respectively, in the
ratio 2 : 1
\begin{enumerate}
    \item  internally
    \item  externally
\end{enumerate}
\solution
		\begin{enumerate}[label=\thesection.\arabic*,ref=\thesection.\theenumi]
\numberwithin{equation}{enumi}
\numberwithin{figure}{enumi}
\numberwithin{table}{enumi}

\item Find the coordinates of the point which divides the join of $(-1,7) \text{ and } (4,-3)$ in the ratio 2:3.
	\\
		\solution
	\input{chapters/10/7/2/1/section.tex}
\item Find the coordinates of the points of trisection of the line segment joining $(4,-1) \text{ and } (-2,3)$.
	\\
		\solution
	\input{chapters/10/7/2/2/section.tex}
\item
	\iffalse
\item To conduct Sports Day activities, in your rectangular shaped school                   
ground ABCD, lines have 
drawn with chalk powder at a                 
distance of 1m each. 100 flower pots have been placed at a distance of 1m 
from each other along AD, as shown 
in Fig. 7.12. Niharika runs $ \frac {1}{4} $th the 
distance AD on the 2nd line and 
posts a green flag. Preet runs $ \frac {1}{5} $th 
the distance AD on the eighth line 
and posts a red flag. What is the 
distance between both the flags? If 
Rashmi has to post a blue flag exactly 
halfway between the line segment 
joining the two flags, where should 
she post her flag?
\begin{figure}[h!]
  \centering
  \includegraphics[width=\columnwidth]{sc.png}
  \caption{}
\label{fig:10/7/12Fig1}
\end{figure}               
\fi
      
\item Find the ratio in which the line segment joining the points $(-3,10) \text{ and } (6,-8)$ $\text{ is divided by } (-1,6)$.
	\\
		\solution
	\input{chapters/10/7/2/4/section.tex}
\item Find the ratio in which the line segment joining $A(1,-5) \text{ and } B(-4,5)$ $\text{is divided by the x-axis}$. Also find the coordinates of the point of division.
\item If $(1,2), (4,y), (x,6), (3,5)$ are the vertices of a parallelogram taken in order, find x and y.
	\\
		\solution
	\input{chapters/10/7/2/6/para1.tex}
\item Find the coordinates of a point A, where AB is the diameter of a circle whose centre is $(2,-3) \text{ and }$ B is $(1,4)$.
	\\
		\solution
	\input{chapters/10/7/2/7/section.tex}
\item If A \text{ and } B are $(-2,-2) \text{ and } (2,-4)$, respectively, find the coordinates of P such that AP= $\frac {3}{7}$AB $\text{ and }$ P lies on the line segment AB.
	\\
		\solution
	\input{chapters/10/7/2/8/section.tex}
\item Find the coordinates of the points which divide the line segment joining $A(-2,2) \text{ and } B(2,8)$ into four equal parts.
	\\
		\solution
	\input{chapters/10/7/2/9/section.tex}
\item Find the area of a rhombus if its vertices are $(3,0), (4,5), (-1,4) \text{ and } (-2,-1)$ taken in order. [$\vec{Hint}$ : Area of rhombus =$\frac {1}{2}$(product of its diagonals)]
	\\
		\solution
	\input{chapters/10/7/2/10/cross.tex}
\item Find the position vector of a point R which divides the line joining two points $\vec{P}$
and $\vec{Q}$ whose position vectors are $\hat{i}+2\hat{j}-\hat{k}$ and $-\hat{i}+\hat{j}+\hat{k}$ respectively, in the
ratio 2 : 1
\begin{enumerate}
    \item  internally
    \item  externally
\end{enumerate}
\solution
		\input{chapters/12/10/2/15/section.tex}
\item Find the position vector of the mid point of the vector joining the points $\vec{P}$(2, 3, 4)
and $\vec{Q}$(4, 1, –2).
\\
\solution
		\input{chapters/12/10/2/16/section.tex}
\item Determine the ratio in which the line $2x+y  - 4=0$ divides the line segment joining the points $\vec{A}(2, - 2)$  and  $\vec{B}(3, 7)$.
\\
\solution
	\input{chapters/10/7/4/1/section.tex}
\item Let $\vec{A}(4, 2), \vec{B}(6, 5)$  and $ \vec{C}(1, 4)$ be the vertices of $\triangle ABC$.
\begin{enumerate}
\item The median from $\vec{A}$ meets $BC$ at $\vec{D}$. Find the coordinates of the point $\vec{D}$.
\item Find the coordinates of the point $\vec{P}$ on $AD$ such that $AP : PD = 2 : 1$.
\item Find the coordinates of points $\vec{Q}$ and $\vec{R}$ on medians $BE$ and $CF$ respectively such that $BQ : QE = 2 : 1$  and  $CR : RF = 2 : 1$.
\item What do you observe?
\item If $\vec{A}, \vec{B}$ and $\vec{C}$  are the vertices of $\triangle ABC$, find the coordinates of the centroid of the triangle.
\end{enumerate}
\solution
	\input{chapters/10/7/4/7/section.tex}
\item Find the slope of a line, which passes through the origin and the mid point of the line segment joining the points $\vec{P}$(0,-4) and $\vec{B}$(8,0).
\label{chapters/11/10/1/5}
\input{chapters/11/10/1/5/matrix.tex}
\item Find the position vector of a point R which divides the line joining two points P and Q whose position vectors are $(2\vec{a}+\vec{b})$ and $(\vec{a}-3\vec{b})$
externally in the ratio 1 : 2. Also, show that P is the mid point of the line segment RQ.\\
	\solution
%		\input{chapters/12/10/5/9/section.tex}

\end{enumerate}


\item Find the position vector of the mid point of the vector joining the points $\vec{P}$(2, 3, 4)
and $\vec{Q}$(4, 1, –2).
\\
\solution
		\begin{enumerate}[label=\thesection.\arabic*,ref=\thesection.\theenumi]
\numberwithin{equation}{enumi}
\numberwithin{figure}{enumi}
\numberwithin{table}{enumi}

\item Find the coordinates of the point which divides the join of $(-1,7) \text{ and } (4,-3)$ in the ratio 2:3.
	\\
		\solution
	\input{chapters/10/7/2/1/section.tex}
\item Find the coordinates of the points of trisection of the line segment joining $(4,-1) \text{ and } (-2,3)$.
	\\
		\solution
	\input{chapters/10/7/2/2/section.tex}
\item
	\iffalse
\item To conduct Sports Day activities, in your rectangular shaped school                   
ground ABCD, lines have 
drawn with chalk powder at a                 
distance of 1m each. 100 flower pots have been placed at a distance of 1m 
from each other along AD, as shown 
in Fig. 7.12. Niharika runs $ \frac {1}{4} $th the 
distance AD on the 2nd line and 
posts a green flag. Preet runs $ \frac {1}{5} $th 
the distance AD on the eighth line 
and posts a red flag. What is the 
distance between both the flags? If 
Rashmi has to post a blue flag exactly 
halfway between the line segment 
joining the two flags, where should 
she post her flag?
\begin{figure}[h!]
  \centering
  \includegraphics[width=\columnwidth]{sc.png}
  \caption{}
\label{fig:10/7/12Fig1}
\end{figure}               
\fi
      
\item Find the ratio in which the line segment joining the points $(-3,10) \text{ and } (6,-8)$ $\text{ is divided by } (-1,6)$.
	\\
		\solution
	\input{chapters/10/7/2/4/section.tex}
\item Find the ratio in which the line segment joining $A(1,-5) \text{ and } B(-4,5)$ $\text{is divided by the x-axis}$. Also find the coordinates of the point of division.
\item If $(1,2), (4,y), (x,6), (3,5)$ are the vertices of a parallelogram taken in order, find x and y.
	\\
		\solution
	\input{chapters/10/7/2/6/para1.tex}
\item Find the coordinates of a point A, where AB is the diameter of a circle whose centre is $(2,-3) \text{ and }$ B is $(1,4)$.
	\\
		\solution
	\input{chapters/10/7/2/7/section.tex}
\item If A \text{ and } B are $(-2,-2) \text{ and } (2,-4)$, respectively, find the coordinates of P such that AP= $\frac {3}{7}$AB $\text{ and }$ P lies on the line segment AB.
	\\
		\solution
	\input{chapters/10/7/2/8/section.tex}
\item Find the coordinates of the points which divide the line segment joining $A(-2,2) \text{ and } B(2,8)$ into four equal parts.
	\\
		\solution
	\input{chapters/10/7/2/9/section.tex}
\item Find the area of a rhombus if its vertices are $(3,0), (4,5), (-1,4) \text{ and } (-2,-1)$ taken in order. [$\vec{Hint}$ : Area of rhombus =$\frac {1}{2}$(product of its diagonals)]
	\\
		\solution
	\input{chapters/10/7/2/10/cross.tex}
\item Find the position vector of a point R which divides the line joining two points $\vec{P}$
and $\vec{Q}$ whose position vectors are $\hat{i}+2\hat{j}-\hat{k}$ and $-\hat{i}+\hat{j}+\hat{k}$ respectively, in the
ratio 2 : 1
\begin{enumerate}
    \item  internally
    \item  externally
\end{enumerate}
\solution
		\input{chapters/12/10/2/15/section.tex}
\item Find the position vector of the mid point of the vector joining the points $\vec{P}$(2, 3, 4)
and $\vec{Q}$(4, 1, –2).
\\
\solution
		\input{chapters/12/10/2/16/section.tex}
\item Determine the ratio in which the line $2x+y  - 4=0$ divides the line segment joining the points $\vec{A}(2, - 2)$  and  $\vec{B}(3, 7)$.
\\
\solution
	\input{chapters/10/7/4/1/section.tex}
\item Let $\vec{A}(4, 2), \vec{B}(6, 5)$  and $ \vec{C}(1, 4)$ be the vertices of $\triangle ABC$.
\begin{enumerate}
\item The median from $\vec{A}$ meets $BC$ at $\vec{D}$. Find the coordinates of the point $\vec{D}$.
\item Find the coordinates of the point $\vec{P}$ on $AD$ such that $AP : PD = 2 : 1$.
\item Find the coordinates of points $\vec{Q}$ and $\vec{R}$ on medians $BE$ and $CF$ respectively such that $BQ : QE = 2 : 1$  and  $CR : RF = 2 : 1$.
\item What do you observe?
\item If $\vec{A}, \vec{B}$ and $\vec{C}$  are the vertices of $\triangle ABC$, find the coordinates of the centroid of the triangle.
\end{enumerate}
\solution
	\input{chapters/10/7/4/7/section.tex}
\item Find the slope of a line, which passes through the origin and the mid point of the line segment joining the points $\vec{P}$(0,-4) and $\vec{B}$(8,0).
\label{chapters/11/10/1/5}
\input{chapters/11/10/1/5/matrix.tex}
\item Find the position vector of a point R which divides the line joining two points P and Q whose position vectors are $(2\vec{a}+\vec{b})$ and $(\vec{a}-3\vec{b})$
externally in the ratio 1 : 2. Also, show that P is the mid point of the line segment RQ.\\
	\solution
%		\input{chapters/12/10/5/9/section.tex}

\end{enumerate}


\item Determine the ratio in which the line $2x+y  - 4=0$ divides the line segment joining the points $\vec{A}(2, - 2)$  and  $\vec{B}(3, 7)$.
\\
\solution
	\iffalse
\documentclass[journal,12pt,twocolumn]{IEEEtran}
\usepackage{graphicx}
\graphicspath{{./chapters/10/7/4/1/figs/}}{}
\usepackage{amsmath,amssymb,amsfonts,amsthm}
\newcommand{\myvec}[1]{\ensuremath{\begin{pmatrix}#1\end{pmatrix}}}
\providecommand{\norm}[1]{\lVert#1\rVert}
\usepackage{listings}
\usepackage{watermark}
\usepackage{titlesec}
\usepackage{caption}
\let\vec\mathbf
\lstset{
frame=single, 
breaklines=true,
columns=fullflexible
}
\thiswatermark{\centering \put(0,-105.0){\includegraphics[scale=0.15]{/sdcard/IITH/vector/vectpr-4/chapters/10/7/4/1/figs/logo.png}} }
\title{\mytitle}
\title{
Assignment - Vector-4
}
\author{Surajit Sarkar}
\begin{document}
\maketitle
%\tableofcontents
\bigskip
\section{\textbf{Problem}}
Determine the ratio in which the line 2x+y–4=0 divides the line segment joining the points A(2,–2) and B(3,7).
\section{\textbf{Solution}}
\begin{table}[h]
    \centering
    \begin{tabular}{|c|c|}
       \hline
       \textbf{Symbol}&\textbf{Value}  \\
       \hline
	    $\vec{A}$ & $\myvec{2\\-2}$\\
        \hline
	    $\vec{B}$ & $\myvec{3\\7}$\\
        \hline
	    c&$4$\\
        \hline
       $\vec{n}$ & $\myvec{2\\1}$\\
       \hline
    \end{tabular}
    \caption{Parameters}
    \label{tab:my_label}
\end{table}
Given equation
\fi
The given equation can be expressed as
\begin{align}
    \myvec{2&1}\vec{x}&=4\\
\end{align}
Using section formula, the point of division 
\begin{align}
    \vec{P} = \frac{k\vec{B+A}}{k+1}
\end{align}
which upon substitution in the equation of a line yields
\begin{align}
    \implies\vec{n}^{\top}\myvec{\frac{k\vec{B+A}}{k+1}}&=c\\
    \implies k&=\frac{c-\vec{n}^{\top}\vec{A}}{\vec{n}^{\top}\vec{B}-c}\\
\end{align}
upon simplification.  Substituting numerical values, 
\begin{align}
    k=\frac{2}{9}
\end{align}
See Fig. 
\ref{fig:chapters/10/7/4/1vec}.
\begin{figure}[!h]
\centering
\includegraphics[width=\columnwidth]{chapters/10/7/4/1/figs/vec.pdf}
\caption{}
\label{fig:chapters/10/7/4/1vec}
\end{figure}


\item Let $\vec{A}(4, 2), \vec{B}(6, 5)$  and $ \vec{C}(1, 4)$ be the vertices of $\triangle ABC$.
\begin{enumerate}
\item The median from $\vec{A}$ meets $BC$ at $\vec{D}$. Find the coordinates of the point $\vec{D}$.
\item Find the coordinates of the point $\vec{P}$ on $AD$ such that $AP : PD = 2 : 1$.
\item Find the coordinates of points $\vec{Q}$ and $\vec{R}$ on medians $BE$ and $CF$ respectively such that $BQ : QE = 2 : 1$  and  $CR : RF = 2 : 1$.
\item What do you observe?
\item If $\vec{A}, \vec{B}$ and $\vec{C}$  are the vertices of $\triangle ABC$, find the coordinates of the centroid of the triangle.
\end{enumerate}
\solution
	\iffalse
\documentclass[12pt]{article}
\usepackage{graphicx}
\usepackage[none]{hyphenat}
\usepackage{graphicx}
\usepackage{listings}
\usepackage[english]{babel}
\usepackage{graphicx}
\usepackage{caption} 
\usepackage{booktabs}
\usepackage{array}
\usepackage{amssymb} % for \because
\usepackage{amsmath}   % for having text in math mode
\usepackage{extarrows} % for Row operations arrows
\usepackage{listings}
\usepackage[utf8]{inputenc}
\lstset{
  frame=single,
  breaklines=true
}
\usepackage{hyperref}
  
%Following 2 lines were added to remove the blank page at the beginning
\usepackage{atbegshi}% http://ctan.org/pkg/atbegshi
\AtBeginDocument{\AtBeginShipoutNext{\AtBeginShipoutDiscard}}


%New macro definitions
\newcommand{\mydet}[1]{\ensuremath{\begin{vmatrix}#1\end{vmatrix}}}
\providecommand{\brak}[1]{\ensuremath{\left(#1\right)}}
\newcommand{\solution}{\noindent \textbf{Solution: }}
\newcommand{\myvec}[1]{\ensuremath{\begin{pmatrix}#1\end{pmatrix}}}
\providecommand{\norm}[1]{\left\lVert#1\right\rVert}
\providecommand{\abs}[1]{\left\vert#1\right\vert}
\let\vec\mathbf

\begin{document}

\begin{center}
\title{\textbf{VECTORS}}
\date{\vspace{-5ex}} %Not to print date automatically
\maketitle
\end{center}

\section{10$^{th}$ Maths - EXERCISE-7.4}

Let A(4, 2), B(6, 5) and C(1, 4) be the vertices of $\triangle ABC$
\begin{enumerate}
\item The median from A meets BC at D. Find the coordinates of the point D.
\item Find the coordinates of the point P on AD such that $AP : PD = 2 : 1$
\item Find the coordinates of points Q and R on medians BE and CF respectively such
that $BQ : QE = 2 : 1 \text{and} CR : RF = 2 : 1.$
\item What do yo observe?
\item If $A(x_1, y_1), B(x_2, y_2) \text{and} C(x_3, y_3)$ are the vertices of $\triangle ABC$, find the coordinates of the centroid of the triangle.
\end{enumerate}

Given points are
\begin{align}
\vec{A}=\myvec{4\\ 2} ,
\vec{B}=\myvec{6\\ 5} ,
\vec{C}=\myvec{1\\ 4}
\end{align}
\fi

\begin{enumerate}
\item 
\begin{align}
\vec{D}&=\frac{\vec{B}+\vec{C}}{2}\\
&=\myvec{\frac{7}{2}\\[2pt] \frac{9}{2}}\\
\vec{E}&=\frac{\vec{A}+\vec{C}}{2}\\
&=\myvec{\frac{5}{2}\\ 3}\\
\vec{F}&=\frac{\vec{A}+\vec{B}}{2}\\
&=\myvec{5\\ \frac{7}{2}}
\end{align}

\item 
	For
$n=2$,
\begin{align}
\vec{P}&=\frac{1}{1+n}\brak{\myvec{\vec{A}+n\vec{D}}}\\
&=\frac{1}{3}\myvec{11\\11}
\end{align}

\item 
\begin{align}
\vec{Q}&=\frac{1}{1+n}\brak{\myvec{\vec{B}+n\vec{E}}}\\
&=\frac{1}{3}\myvec{11\\11}\\
\vec{R}&=\frac{1}{1+n}\brak{\myvec{\vec{C}+n\vec{F}}}\\
&=\frac{1}{3}\myvec{11\\11}\\
\end{align}

\item 
 $\vec{P},\vec{Q},\vec{R}$ are the same point.
   
\item 
\begin{align}
\vec{G}&=\frac{\vec{D}+\vec{E}+\vec{F}}{3}\\
&=\frac{1}{3}\myvec{11\\11}\\
\end{align} 
\end{enumerate}
See Fig.  
  \ref{fig:chapters/10/7/4/7/Figure}.
\begin{figure}[h!]
\centering
\includegraphics[width=\columnwidth]{chapters/10/7/4/7/figs/dj.pdf}
\caption{}
  \label{fig:chapters/10/7/4/7/Figure}
\end{figure}

\item Find the slope of a line, which passes through the origin and the mid point of the line segment joining the points $\vec{P}$(0,-4) and $\vec{B}$(8,0).
\label{chapters/11/10/1/5}
\iffalse
\documentclass[journal,12pt,twocolumn]{IEEEtran}
\usepackage{graphicx}
\graphicspath{{./figs/}}{}
\usepackage{amsmath,amssymb,amsfonts,amsthm}
\newcommand{\myvec}[1]{\ensuremath{\begin{pmatrix}#1\end{pmatrix}}}

\let\vec\mathbf

\title{
Matrix-Lines
}
\author{Jyothsna Paluchuri-FWC22059\\}
\begin{document}
\maketitle
\tableofcontents
\bigskip
\section{Problem Statement}
\fi
	\begin{figure}[!ht]
		\centering
 \includegraphics[width=\columnwidth]{chapters/11/10/1/5/figs/line.png}
		\caption{}
		\label{fig:11/10/1/5}
  	\end{figure}
	\\
	\solution
\iffalse
\section{Construction}
\begin{figure}[h]
    \centering
\includegraphics[width=\columnwidth]{line.png}
    \caption{Equation of the slope}
    \label{fig:my_label}
\end{figure}
\vspace{2cm}
\begin{table}[h]
    \centering
    \begin{tabular}{|c|c|c|c|}
       \hline
       \textbf{Symbol}&\textbf{Value}&\textbf{Description}  \\
       \hline
	    $\vec{P}$ & $\myvec{
		    0\\
		    -4}$
	    & Point on Y-axis\\
        \hline
	    $\vec{B}$ & $\myvec{8\\0}$
 & Point on X-axis\\
        \hline
	    $\vec{0}$ & $\myvec{0\\0}$
 & Origin\\
        \hline
    \end{tabular}
    \caption{Parameters}
    \label{tab:my_label}
\end{table}


\section{Solution}
Given that resultant line passes through origin and mid point of the line segment joining point P(0,-4) and B(8,0) \\
\\
\\
given ${\vec{P}}$=$\myvec{
  0\\
  -4}$
 , ${\vec{B}}$=$\myvec{
  8\\
  0}$
  
 \fi 
The mid point of $PB$ is
\begin{align}
\vec{M} &=\frac{1}{2}(\vec{P}+\vec{B})
	= \myvec{4 \\ -2}  
\end{align}
The direction vector of line joining $\vec{O}, \vec{M}$ is 
\begin{align}
\vec{m}&=\vec{O}-\vec{M}
 = -\vec{M}
\end{align}
which can be expressed as
\begin{align}
	\myvec{1 \\ -\frac{1}{2}}
\end{align}
Thus the slope is
\begin{align}
	m = -\frac{1}{2}
\end{align}
\iffalse
\textbf{The direction vector of a line expressed as}
\begin{align}
\implies\vec{m} &= \begin{pmatrix}1 \\ m \\ \end{pmatrix}
\end{align}

\textbf{By solving equation (5) and (6),we get the slope of $\vec{O}$ $\vec{M}$ line}
\begin{align}
        \boxed{m=-0.5}
 \end{align}

\section{Software}
Download the following code using,
\begin{table}[h]
    \centering
    \begin{tabular}{|c|}
    \hline \\
   https://github.com/jyothsna777/jyothsna-fwc.git  \\
         \\
\hline
    \end{tabular}
\end{table}
\\
and execute the code by using command
\begin{center}
\textbf{Python3 lines.py}\\
\end{center}

\section{Conclusion}
Hence the slope of line $\vec{O}$ $\vec{M}$ lineis $\vec{m}$=-0.5

\end{document}
\fi

\item Find the position vector of a point R which divides the line joining two points P and Q whose position vectors are $(2\vec{a}+\vec{b})$ and $(\vec{a}-3\vec{b})$
externally in the ratio 1 : 2. Also, show that P is the mid point of the line segment RQ.\\
	\solution
%		\begin{enumerate}[label=\thesection.\arabic*,ref=\thesection.\theenumi]
\numberwithin{equation}{enumi}
\numberwithin{figure}{enumi}
\numberwithin{table}{enumi}

\item Find the coordinates of the point which divides the join of $(-1,7) \text{ and } (4,-3)$ in the ratio 2:3.
	\\
		\solution
	\input{chapters/10/7/2/1/section.tex}
\item Find the coordinates of the points of trisection of the line segment joining $(4,-1) \text{ and } (-2,3)$.
	\\
		\solution
	\input{chapters/10/7/2/2/section.tex}
\item
	\iffalse
\item To conduct Sports Day activities, in your rectangular shaped school                   
ground ABCD, lines have 
drawn with chalk powder at a                 
distance of 1m each. 100 flower pots have been placed at a distance of 1m 
from each other along AD, as shown 
in Fig. 7.12. Niharika runs $ \frac {1}{4} $th the 
distance AD on the 2nd line and 
posts a green flag. Preet runs $ \frac {1}{5} $th 
the distance AD on the eighth line 
and posts a red flag. What is the 
distance between both the flags? If 
Rashmi has to post a blue flag exactly 
halfway between the line segment 
joining the two flags, where should 
she post her flag?
\begin{figure}[h!]
  \centering
  \includegraphics[width=\columnwidth]{sc.png}
  \caption{}
\label{fig:10/7/12Fig1}
\end{figure}               
\fi
      
\item Find the ratio in which the line segment joining the points $(-3,10) \text{ and } (6,-8)$ $\text{ is divided by } (-1,6)$.
	\\
		\solution
	\input{chapters/10/7/2/4/section.tex}
\item Find the ratio in which the line segment joining $A(1,-5) \text{ and } B(-4,5)$ $\text{is divided by the x-axis}$. Also find the coordinates of the point of division.
\item If $(1,2), (4,y), (x,6), (3,5)$ are the vertices of a parallelogram taken in order, find x and y.
	\\
		\solution
	\input{chapters/10/7/2/6/para1.tex}
\item Find the coordinates of a point A, where AB is the diameter of a circle whose centre is $(2,-3) \text{ and }$ B is $(1,4)$.
	\\
		\solution
	\input{chapters/10/7/2/7/section.tex}
\item If A \text{ and } B are $(-2,-2) \text{ and } (2,-4)$, respectively, find the coordinates of P such that AP= $\frac {3}{7}$AB $\text{ and }$ P lies on the line segment AB.
	\\
		\solution
	\input{chapters/10/7/2/8/section.tex}
\item Find the coordinates of the points which divide the line segment joining $A(-2,2) \text{ and } B(2,8)$ into four equal parts.
	\\
		\solution
	\input{chapters/10/7/2/9/section.tex}
\item Find the area of a rhombus if its vertices are $(3,0), (4,5), (-1,4) \text{ and } (-2,-1)$ taken in order. [$\vec{Hint}$ : Area of rhombus =$\frac {1}{2}$(product of its diagonals)]
	\\
		\solution
	\input{chapters/10/7/2/10/cross.tex}
\item Find the position vector of a point R which divides the line joining two points $\vec{P}$
and $\vec{Q}$ whose position vectors are $\hat{i}+2\hat{j}-\hat{k}$ and $-\hat{i}+\hat{j}+\hat{k}$ respectively, in the
ratio 2 : 1
\begin{enumerate}
    \item  internally
    \item  externally
\end{enumerate}
\solution
		\input{chapters/12/10/2/15/section.tex}
\item Find the position vector of the mid point of the vector joining the points $\vec{P}$(2, 3, 4)
and $\vec{Q}$(4, 1, –2).
\\
\solution
		\input{chapters/12/10/2/16/section.tex}
\item Determine the ratio in which the line $2x+y  - 4=0$ divides the line segment joining the points $\vec{A}(2, - 2)$  and  $\vec{B}(3, 7)$.
\\
\solution
	\input{chapters/10/7/4/1/section.tex}
\item Let $\vec{A}(4, 2), \vec{B}(6, 5)$  and $ \vec{C}(1, 4)$ be the vertices of $\triangle ABC$.
\begin{enumerate}
\item The median from $\vec{A}$ meets $BC$ at $\vec{D}$. Find the coordinates of the point $\vec{D}$.
\item Find the coordinates of the point $\vec{P}$ on $AD$ such that $AP : PD = 2 : 1$.
\item Find the coordinates of points $\vec{Q}$ and $\vec{R}$ on medians $BE$ and $CF$ respectively such that $BQ : QE = 2 : 1$  and  $CR : RF = 2 : 1$.
\item What do you observe?
\item If $\vec{A}, \vec{B}$ and $\vec{C}$  are the vertices of $\triangle ABC$, find the coordinates of the centroid of the triangle.
\end{enumerate}
\solution
	\input{chapters/10/7/4/7/section.tex}
\item Find the slope of a line, which passes through the origin and the mid point of the line segment joining the points $\vec{P}$(0,-4) and $\vec{B}$(8,0).
\label{chapters/11/10/1/5}
\input{chapters/11/10/1/5/matrix.tex}
\item Find the position vector of a point R which divides the line joining two points P and Q whose position vectors are $(2\vec{a}+\vec{b})$ and $(\vec{a}-3\vec{b})$
externally in the ratio 1 : 2. Also, show that P is the mid point of the line segment RQ.\\
	\solution
%		\input{chapters/12/10/5/9/section.tex}

\end{enumerate}



\end{enumerate}


\item Find the position vector of the mid point of the vector joining the points $\vec{P}$(2, 3, 4)
and $\vec{Q}$(4, 1, –2).
\\
\solution
		\begin{enumerate}[label=\thesection.\arabic*,ref=\thesection.\theenumi]
\numberwithin{equation}{enumi}
\numberwithin{figure}{enumi}
\numberwithin{table}{enumi}

\item Find the coordinates of the point which divides the join of $(-1,7) \text{ and } (4,-3)$ in the ratio 2:3.
	\\
		\solution
	\iffalse
\documentclass[12pt]{article}
\usepackage{graphicx}
\usepackage{amsmath}
\usepackage{mathtools}
\usepackage{gensymb}

\newcommand{\mydet}[1]{\ensuremath{\begin{vmatrix}#1\end{vmatrix}}}
\providecommand{\brak}[1]{\ensuremath{\left(#1\right)}}
\providecommand{\norm}[1]{\left\lVert#1\right\rVert}
\newcommand{\solution}{\noindent \textbf{Solution: }}
\newcommand{\myvec}[1]{\ensuremath{\begin{pmatrix}#1\end{pmatrix}}}
\let\vec\mathbf

\begin{document}
\begin{center}
\textbf\large{CHAPTER-7 \\ COORDINATE GEOMETRY}
\end{center}
\section*{Excercise 7.2}

1. Find the coordinates of the point which divides the join $\vec(-1,7) \text{ and } \vec(4,-3)$ in the ratio 2:3 :
\\
\\
\solution\\		
\fi
The coordinates and ratio are given as
\begin{align}
\vec{P}=\myvec{-1\\7\\},
\vec{Q}=\myvec{4\\-3\\},
n=\frac{3}{2}
\end{align}
Using section formula
\begin{align}
\vec{R}&=\frac{\vec{Q}+n\vec{P}}{1+n}\\
&=\frac{1}{1+\frac{3}{2}}  \myvec{\myvec{
4\\
-3\\
}
  +
   \frac{3}{2}\myvec{
-1\\
7\\
}}\\
&=\myvec{
1\\
3
}
\end{align}
See Fig. 
\ref{fig:chapters/10/7/2/1/Fig}
\begin{figure}[!h]
\begin{center}
   \includegraphics[width=\columnwidth]{chapters/10/7/2/1/figs/linefig.png}
\end{center}
\caption{}
\label{fig:chapters/10/7/2/1/Fig}
\end{figure}


\item Find the coordinates of the points of trisection of the line segment joining $(4,-1) \text{ and } (-2,3)$.
	\\
		\solution
	\begin{enumerate}[label=\thesection.\arabic*,ref=\thesection.\theenumi]
\numberwithin{equation}{enumi}
\numberwithin{figure}{enumi}
\numberwithin{table}{enumi}

\item Find the coordinates of the point which divides the join of $(-1,7) \text{ and } (4,-3)$ in the ratio 2:3.
	\\
		\solution
	\input{chapters/10/7/2/1/section.tex}
\item Find the coordinates of the points of trisection of the line segment joining $(4,-1) \text{ and } (-2,3)$.
	\\
		\solution
	\input{chapters/10/7/2/2/section.tex}
\item
	\iffalse
\item To conduct Sports Day activities, in your rectangular shaped school                   
ground ABCD, lines have 
drawn with chalk powder at a                 
distance of 1m each. 100 flower pots have been placed at a distance of 1m 
from each other along AD, as shown 
in Fig. 7.12. Niharika runs $ \frac {1}{4} $th the 
distance AD on the 2nd line and 
posts a green flag. Preet runs $ \frac {1}{5} $th 
the distance AD on the eighth line 
and posts a red flag. What is the 
distance between both the flags? If 
Rashmi has to post a blue flag exactly 
halfway between the line segment 
joining the two flags, where should 
she post her flag?
\begin{figure}[h!]
  \centering
  \includegraphics[width=\columnwidth]{sc.png}
  \caption{}
\label{fig:10/7/12Fig1}
\end{figure}               
\fi
      
\item Find the ratio in which the line segment joining the points $(-3,10) \text{ and } (6,-8)$ $\text{ is divided by } (-1,6)$.
	\\
		\solution
	\input{chapters/10/7/2/4/section.tex}
\item Find the ratio in which the line segment joining $A(1,-5) \text{ and } B(-4,5)$ $\text{is divided by the x-axis}$. Also find the coordinates of the point of division.
\item If $(1,2), (4,y), (x,6), (3,5)$ are the vertices of a parallelogram taken in order, find x and y.
	\\
		\solution
	\input{chapters/10/7/2/6/para1.tex}
\item Find the coordinates of a point A, where AB is the diameter of a circle whose centre is $(2,-3) \text{ and }$ B is $(1,4)$.
	\\
		\solution
	\input{chapters/10/7/2/7/section.tex}
\item If A \text{ and } B are $(-2,-2) \text{ and } (2,-4)$, respectively, find the coordinates of P such that AP= $\frac {3}{7}$AB $\text{ and }$ P lies on the line segment AB.
	\\
		\solution
	\input{chapters/10/7/2/8/section.tex}
\item Find the coordinates of the points which divide the line segment joining $A(-2,2) \text{ and } B(2,8)$ into four equal parts.
	\\
		\solution
	\input{chapters/10/7/2/9/section.tex}
\item Find the area of a rhombus if its vertices are $(3,0), (4,5), (-1,4) \text{ and } (-2,-1)$ taken in order. [$\vec{Hint}$ : Area of rhombus =$\frac {1}{2}$(product of its diagonals)]
	\\
		\solution
	\input{chapters/10/7/2/10/cross.tex}
\item Find the position vector of a point R which divides the line joining two points $\vec{P}$
and $\vec{Q}$ whose position vectors are $\hat{i}+2\hat{j}-\hat{k}$ and $-\hat{i}+\hat{j}+\hat{k}$ respectively, in the
ratio 2 : 1
\begin{enumerate}
    \item  internally
    \item  externally
\end{enumerate}
\solution
		\input{chapters/12/10/2/15/section.tex}
\item Find the position vector of the mid point of the vector joining the points $\vec{P}$(2, 3, 4)
and $\vec{Q}$(4, 1, –2).
\\
\solution
		\input{chapters/12/10/2/16/section.tex}
\item Determine the ratio in which the line $2x+y  - 4=0$ divides the line segment joining the points $\vec{A}(2, - 2)$  and  $\vec{B}(3, 7)$.
\\
\solution
	\input{chapters/10/7/4/1/section.tex}
\item Let $\vec{A}(4, 2), \vec{B}(6, 5)$  and $ \vec{C}(1, 4)$ be the vertices of $\triangle ABC$.
\begin{enumerate}
\item The median from $\vec{A}$ meets $BC$ at $\vec{D}$. Find the coordinates of the point $\vec{D}$.
\item Find the coordinates of the point $\vec{P}$ on $AD$ such that $AP : PD = 2 : 1$.
\item Find the coordinates of points $\vec{Q}$ and $\vec{R}$ on medians $BE$ and $CF$ respectively such that $BQ : QE = 2 : 1$  and  $CR : RF = 2 : 1$.
\item What do you observe?
\item If $\vec{A}, \vec{B}$ and $\vec{C}$  are the vertices of $\triangle ABC$, find the coordinates of the centroid of the triangle.
\end{enumerate}
\solution
	\input{chapters/10/7/4/7/section.tex}
\item Find the slope of a line, which passes through the origin and the mid point of the line segment joining the points $\vec{P}$(0,-4) and $\vec{B}$(8,0).
\label{chapters/11/10/1/5}
\input{chapters/11/10/1/5/matrix.tex}
\item Find the position vector of a point R which divides the line joining two points P and Q whose position vectors are $(2\vec{a}+\vec{b})$ and $(\vec{a}-3\vec{b})$
externally in the ratio 1 : 2. Also, show that P is the mid point of the line segment RQ.\\
	\solution
%		\input{chapters/12/10/5/9/section.tex}

\end{enumerate}


\item
	\iffalse
\item To conduct Sports Day activities, in your rectangular shaped school                   
ground ABCD, lines have 
drawn with chalk powder at a                 
distance of 1m each. 100 flower pots have been placed at a distance of 1m 
from each other along AD, as shown 
in Fig. 7.12. Niharika runs $ \frac {1}{4} $th the 
distance AD on the 2nd line and 
posts a green flag. Preet runs $ \frac {1}{5} $th 
the distance AD on the eighth line 
and posts a red flag. What is the 
distance between both the flags? If 
Rashmi has to post a blue flag exactly 
halfway between the line segment 
joining the two flags, where should 
she post her flag?
\begin{figure}[h!]
  \centering
  \includegraphics[width=\columnwidth]{sc.png}
  \caption{}
\label{fig:10/7/12Fig1}
\end{figure}               
\fi
      
\item Find the ratio in which the line segment joining the points $(-3,10) \text{ and } (6,-8)$ $\text{ is divided by } (-1,6)$.
	\\
		\solution
	\iffalse
\documentclass[12pt]{article}
\usepackage{graphicx}
%\documentclass[journal,12pt,twocolumn]{IEEEtran}
\usepackage[none]{hyphenat}
\usepackage{graphicx}
\usepackage{listings}
\usepackage[english]{babel}
\usepackage{graphicx}
\usepackage{caption} 
\usepackage{hyperref}
\usepackage{booktabs}
\def\inputGnumericTable{}
\usepackage{color}                                            %%
    \usepackage{array}                                            %%
    \usepackage{longtable}                                        %%
    \usepackage{calc}                                             %%
    \usepackage{multirow}                                         %%
    \usepackage{hhline}                                           %%
    \usepackage{ifthen}
\usepackage{array}
\usepackage{amsmath}   % for having text in math mode
\usepackage{listings}
\lstset{
language=tex,
frame=single, 
breaklines=true
}
  
%Following 2 lines were added to remove the blank page at the beginning
\usepackage{atbegshi}% http://ctan.org/pkg/atbegshi
\AtBeginDocument{\AtBeginShipoutNext{\AtBeginShipoutDiscard}}
%
%New macro definitions
\newcommand{\mydet}[1]{\ensuremath{\begin{vmatrix}#1\end{vmatrix}}}
\providecommand{\brak}[1]{\ensuremath{\left(#1\right)}}
\providecommand{\norm}[1]{\left\lVert#1\right\rVert}
\newcommand{\solution}{\noindent \textbf{Solution: }}
\newcommand{\myvec}[1]{\ensuremath{\begin{pmatrix}#1\end{pmatrix}}}
\let\vec\mathbf
\begin{document}
\begin{center}
\title{\textbf{Coordinate Geometry}}
\date{\vspace{-5ex}} %Not to print date automatically
\maketitle
\end{center}
\setcounter{page}{1}
\section*{10$^{th}$ Maths - Chapter 7}
This is Problem-4 from Exercise 7.2
\begin{enumerate}
\item Find the ratio in which the line segement joining the points $\myvec{-3 \\ 10}$ and $\myvec{6\\-8}$ is divided by $\myvec{-1\\6}$.\\
\solution \\
\fi
		The input parameters for this problem are available in Table \eqref{tab:10/7/2/4-1}.
\begin{table}[ht!]
\input{chapters/10/7/2/4/tables/table.tex}
\caption{}
\label{tab:10/7/2/4-1} 
\end{table}
Using section formula,
\begin{align}
         \vec{R} &=\frac{\vec{Q}+n\vec{P}}{1+n}\label{eq:chapters/10/7/2/4/1}
\end{align}
Substituting the values of $\vec{P},\vec{Q}$ and $\vec{R}$ in \eqref{eq:chapters/10/7/2/4/1}
\begin{align}
         \myvec{-1\\6} &=\frac{{\myvec{-3\\10}+n\myvec{6\\-8}}}{1+n}\\
 &=\frac{1}{1+n}\brak{{\myvec{-3\\10}+n\myvec{6\\-8}}} \\
 &=\frac{1}{1+n}\myvec{-3+6n\\10-8n} \label{eq:chapters/10/7/2/4/4}
\end{align}
Simplifying \eqref{eq:chapters/10/7/2/4/4} yeilds,
\begin{align}
          -1 &=\frac{-3+6n}{1+n}\\
\implies          n &=\frac{2}{7}
\end{align}
Also,
\begin{align}
          6 &=\frac{10-8n}{1+n}\\
    \implies      n &=\frac{2}{7}
\end{align}
Hence the desired ratio is $\dfrac{2}{7}$.  
\begin{figure}[!h]
 \begin{center}
  \includegraphics[width=\columnwidth]{chapters/10/7/2/4/figs/fig.png}
 \end{center}
\caption{}
\label{fig:10/7/2/4Fig1}
\end{figure}

\item Find the ratio in which the line segment joining $A(1,-5) \text{ and } B(-4,5)$ $\text{is divided by the x-axis}$. Also find the coordinates of the point of division.
\item If $(1,2), (4,y), (x,6), (3,5)$ are the vertices of a parallelogram taken in order, find x and y.
	\\
		\solution
	\iffalse
\documentclass[12pt]{article}
\usepackage{graphicx}
%\documentclass[journal,12pt,twocolumn]{IEEEtran}
\def\inputGnumericTable{}
\usepackage{color}                                            %%
    \usepackage{array}                                            %%
    \usepackage{longtable}                                        %%
    \usepackage{calc}                                             %%
    \usepackage{multirow}                                         %%
    \usepackage{hhline}                                           %%
    \usepackage{ifthen}
\usepackage[none]{hyphenat}
\usepackage{graphicx}
\usepackage{listings}
\usepackage[english]{babel}
\usepackage{graphicx}
\usepackage{caption} 
\usepackage{hyperref}
\usepackage{booktabs}
\usepackage{array}
\usepackage{amsmath}   % for having text in math mode
\usepackage{listings}
\lstset{
  frame=single,
  breaklines=true
}
  
%Following 2 lines were added to remove the blank page at the beginning
\usepackage{atbegshi}% http://ctan.org/pkg/atbegshi
\AtBeginDocument{\AtBeginShipoutNext{\AtBeginShipoutDiscard}}
%


%New macro definitions
\newcommand{\mydet}[1]{\ensuremath{\begin{vmatrix}#1\end{vmatrix}}}
\providecommand{\brak}[1]{\ensuremath{\left(#1\right)}}
\providecommand{\norm}[1]{\left\lVert#1\right\rVert}
\newcommand{\solution}{\noindent \textbf{Solution: }}
\newcommand{\myvec}[1]{\ensuremath{\begin{pmatrix}#1\end{pmatrix}}}
\let\vec\mathbf

\begin{document}

\begin{center}
\title{\textbf{Properties of Parallelegram}}
\date{\vspace{-5ex}} %Not to print date automatically
\maketitle
\end{center}

\setcounter{page}{1}

\section{10$^{th}$ Maths - Chapter 7}

This is Problem-6 from Exercise 7.2

\begin{enumerate}
\item If $\vec{A}(1, 2),\vec{B}(4, x),\vec{C}(y, 6) \text{and } \vec{D}(3, 5)$ are the vertices of a parallelogram taken in order,find x and y.
\end{enumerate}
\fi

The input parameters for this problem are available in
\ref{table:chapters/10/7/2/6/tables/}.	
\begin{table}[!ht]
	\centering
	\input{chapters/10/7/2/6/tables/table.tex}
\caption{}
\label{table:chapters/10/7/2/6/tables/}	
\end{table}
From the given information,
\begin{align}
  \label{eq:chapters/10/7/2/6/tables/det2f}
	\vec{B}-\vec{A} &= \myvec{4 \\y } - \myvec{1 \\2 }  = \myvec{3 \\y-2 }\\
	\vec{C}-\vec{D} &= \myvec{x \\6 } - \myvec{3 \\5 }  = \myvec{x-3 \\1}
\end{align}
Since $ABCD$ is a parallellogram,
\begin{align}
	\myvec{3\\y-2}&=\myvec{x-3\\1}\\
	\implies x&=6 ,y=3
\end{align}
Fig. \ref{fig:chapters/10/7/2/6/Fig3}
provides a verification.
\begin{figure}[h!]
	\begin{center}
  \includegraphics[width=\columnwidth]{chapters/10/7/2/6/figs/para.pdf}
	\end{center}
\caption{}
\label{fig:chapters/10/7/2/6/Fig3}
\end{figure}


\item Find the coordinates of a point A, where AB is the diameter of a circle whose centre is $(2,-3) \text{ and }$ B is $(1,4)$.
	\\
		\solution
	\iffalse
\documentclass[12pt]{article}
\usepackage{graphicx}
\usepackage{amsmath}
\usepackage{mathtools}
\usepackage{gensymb}

\newcommand{\mydet}[1]{\ensuremath{\begin{vmatrix}#1\end{vmatrix}}}
\providecommand{\brak}[1]{\ensuremath{\left(#1\right)}}
\providecommand{\norm}[1]{\left\lVert#1\right\rVert}
\newcommand{\solution}{\noindent \textbf{Solution: }}
\newcommand{\myvec}[1]{\ensuremath{\begin{pmatrix}#1\end{pmatrix}}}
\let\vec\mathbf

\begin{document}
\begin{center}
\section*{CHAPTER 7 - COORDINATE GEOMETRY}

\end{center}
\section*{Excercise 7.2}

Q7.Find the coordinates of point $\vec{A}$, where AB is the diameter of a circle where the center is (2,-3) and $\vec{B}$ is the point (1,4):

\solution
\begin{enumerate}
\item The coordinates $\vec{B}$ and center $\vec{C}$ are given, where:
	\fi
	Let
	\begin{align}
	\vec{B} = \myvec{
		1\\
	    4\\
		},
	\vec{C} = \myvec{
	    2\\
	   -3\\
		}
	\end{align}
	\iffalse
Let us assume the coordinates of $\vec{A}$. Now, $\vec{C}$ is the center which is midpoint of line AB and $\vec{B}$ is one of the coordinate of diameter AB of a circle.
	\fi	
Hence,	
	\begin{align}
	\vec{C} &= \frac{\vec{A+B}}{2} \\
\implies	2\vec{C} &= \vec{A}+\vec{B} \\
		\text{or, }	\vec{A} &= 2\vec{C}-\vec{B} \\
	 &= \myvec{3\\-10\\}	
	\end{align}       
	See Fig. 
\ref{fig:chapters/10/7/2/7Fig}.
\begin{figure}[!h]
\begin{center}	
	\includegraphics[width=\columnwidth]{chapters/10/7/2/7/figs/Vector1.png}
\end{center}
\caption{}
\label{fig:chapters/10/7/2/7Fig}
\end{figure}
	

\item If A \text{ and } B are $(-2,-2) \text{ and } (2,-4)$, respectively, find the coordinates of P such that AP= $\frac {3}{7}$AB $\text{ and }$ P lies on the line segment AB.
	\\
		\solution
	\iffalse
\documentclass[journal,10pt,twocolumn]{article}
\usepackage{graphicx}
\usepackage[none]{hyphenat}
\usepackage{graphicx}
\usepackage{listings}
\usepackage[english]{babel}
\usepackage{graphicx}
\usepackage{caption} 
\usepackage{booktabs}
\usepackage{array}
\usepackage{amssymb} % for \because
\usepackage{amsmath}   % for having text in math mode
\usepackage{extarrows} % for Row operations arrows
\usepackage{listings}
\usepackage[utf8]{inputenc}
\lstset{
  frame=single,
  breaklines=true
}
\usepackage{hyperref}
  
%Following 2 lines were added to remove the blank page at the beginning
\usepackage{atbegshi}% http://ctan.org/pkg/atbegshi
\AtBeginDocument{\AtBeginShipoutNext{\AtBeginShipoutDiscard}}


%New macro definitions
\newcommand{\mydet}[1]{\ensuremath{\begin{vmatrix}#1\end{vmatrix}}}
\providecommand{\brak}[1]{\ensuremath{\left(#1\right)}}
\newcommand{\solution}{\noindent \textbf{Solution: }}
\newcommand{\myvec}[1]{\ensuremath{\begin{pmatrix}#1\end{pmatrix}}}
\providecommand{\norm}[1]{\left\lVert#1\right\rVert}
\providecommand{\abs}[1]{\left\vert#1\right\vert}
\let\vec\mathbf

\begin{document}

\begin{center}
\title{\textbf{VECTORS}}
\date{\vspace{-5ex}} %Not to print date automatically
\maketitle
\end{center}

\section{10$^{th}$ Maths - EXERCISE-7.2}

\begin{enumerate}
\item If A and B are $(– 2, – 2)\text{ and }(2, – 4)$, respectively, find the coordinates of P such that $AP =\frac{3}{7}AB$ and P lies on the line segment AB. 

\section{SOLUTION}
Given points are
\begin{align}
\vec{A}=\myvec{-2\\ -2} ,
\vec{B}=\myvec{2\\ -4}
\end{align}
The equation of the formula is
\fi
Using section formula, 
\begin{align}
\vec{P}&=\frac{\vec{A}+n\vec{B}}{1+n}
\end{align}
where
\begin{align}
	n =\frac{3}{4}
\end{align}
Thus,
\begin{align}
\vec{P}&=\frac{1}{1+\frac{3}{4}}\brak{\myvec{-2\\-2}+\frac{3}{4}\myvec{2\\-4}}\\
&=\myvec{\frac{-2}{7}\\[1pt] \frac{-20}{7}}
\end{align}
See Fig. 
   \ref{fig:chapters/10/7/2/8/vec.png}
\begin{figure}
   \centering 
 \includegraphics[width=\columnwidth]{chapters/10/7/2/8/figs/vec.png}
   \caption{}
   \label{fig:chapters/10/7/2/8/vec.png}
   \end{figure}

\item Find the coordinates of the points which divide the line segment joining $A(-2,2) \text{ and } B(2,8)$ into four equal parts.
	\\
		\solution
	\begin{enumerate}[label=\thesection.\arabic*,ref=\thesection.\theenumi]
\numberwithin{equation}{enumi}
\numberwithin{figure}{enumi}
\numberwithin{table}{enumi}

\item Find the coordinates of the point which divides the join of $(-1,7) \text{ and } (4,-3)$ in the ratio 2:3.
	\\
		\solution
	\input{chapters/10/7/2/1/section.tex}
\item Find the coordinates of the points of trisection of the line segment joining $(4,-1) \text{ and } (-2,3)$.
	\\
		\solution
	\input{chapters/10/7/2/2/section.tex}
\item
	\iffalse
\item To conduct Sports Day activities, in your rectangular shaped school                   
ground ABCD, lines have 
drawn with chalk powder at a                 
distance of 1m each. 100 flower pots have been placed at a distance of 1m 
from each other along AD, as shown 
in Fig. 7.12. Niharika runs $ \frac {1}{4} $th the 
distance AD on the 2nd line and 
posts a green flag. Preet runs $ \frac {1}{5} $th 
the distance AD on the eighth line 
and posts a red flag. What is the 
distance between both the flags? If 
Rashmi has to post a blue flag exactly 
halfway between the line segment 
joining the two flags, where should 
she post her flag?
\begin{figure}[h!]
  \centering
  \includegraphics[width=\columnwidth]{sc.png}
  \caption{}
\label{fig:10/7/12Fig1}
\end{figure}               
\fi
      
\item Find the ratio in which the line segment joining the points $(-3,10) \text{ and } (6,-8)$ $\text{ is divided by } (-1,6)$.
	\\
		\solution
	\input{chapters/10/7/2/4/section.tex}
\item Find the ratio in which the line segment joining $A(1,-5) \text{ and } B(-4,5)$ $\text{is divided by the x-axis}$. Also find the coordinates of the point of division.
\item If $(1,2), (4,y), (x,6), (3,5)$ are the vertices of a parallelogram taken in order, find x and y.
	\\
		\solution
	\input{chapters/10/7/2/6/para1.tex}
\item Find the coordinates of a point A, where AB is the diameter of a circle whose centre is $(2,-3) \text{ and }$ B is $(1,4)$.
	\\
		\solution
	\input{chapters/10/7/2/7/section.tex}
\item If A \text{ and } B are $(-2,-2) \text{ and } (2,-4)$, respectively, find the coordinates of P such that AP= $\frac {3}{7}$AB $\text{ and }$ P lies on the line segment AB.
	\\
		\solution
	\input{chapters/10/7/2/8/section.tex}
\item Find the coordinates of the points which divide the line segment joining $A(-2,2) \text{ and } B(2,8)$ into four equal parts.
	\\
		\solution
	\input{chapters/10/7/2/9/section.tex}
\item Find the area of a rhombus if its vertices are $(3,0), (4,5), (-1,4) \text{ and } (-2,-1)$ taken in order. [$\vec{Hint}$ : Area of rhombus =$\frac {1}{2}$(product of its diagonals)]
	\\
		\solution
	\input{chapters/10/7/2/10/cross.tex}
\item Find the position vector of a point R which divides the line joining two points $\vec{P}$
and $\vec{Q}$ whose position vectors are $\hat{i}+2\hat{j}-\hat{k}$ and $-\hat{i}+\hat{j}+\hat{k}$ respectively, in the
ratio 2 : 1
\begin{enumerate}
    \item  internally
    \item  externally
\end{enumerate}
\solution
		\input{chapters/12/10/2/15/section.tex}
\item Find the position vector of the mid point of the vector joining the points $\vec{P}$(2, 3, 4)
and $\vec{Q}$(4, 1, –2).
\\
\solution
		\input{chapters/12/10/2/16/section.tex}
\item Determine the ratio in which the line $2x+y  - 4=0$ divides the line segment joining the points $\vec{A}(2, - 2)$  and  $\vec{B}(3, 7)$.
\\
\solution
	\input{chapters/10/7/4/1/section.tex}
\item Let $\vec{A}(4, 2), \vec{B}(6, 5)$  and $ \vec{C}(1, 4)$ be the vertices of $\triangle ABC$.
\begin{enumerate}
\item The median from $\vec{A}$ meets $BC$ at $\vec{D}$. Find the coordinates of the point $\vec{D}$.
\item Find the coordinates of the point $\vec{P}$ on $AD$ such that $AP : PD = 2 : 1$.
\item Find the coordinates of points $\vec{Q}$ and $\vec{R}$ on medians $BE$ and $CF$ respectively such that $BQ : QE = 2 : 1$  and  $CR : RF = 2 : 1$.
\item What do you observe?
\item If $\vec{A}, \vec{B}$ and $\vec{C}$  are the vertices of $\triangle ABC$, find the coordinates of the centroid of the triangle.
\end{enumerate}
\solution
	\input{chapters/10/7/4/7/section.tex}
\item Find the slope of a line, which passes through the origin and the mid point of the line segment joining the points $\vec{P}$(0,-4) and $\vec{B}$(8,0).
\label{chapters/11/10/1/5}
\input{chapters/11/10/1/5/matrix.tex}
\item Find the position vector of a point R which divides the line joining two points P and Q whose position vectors are $(2\vec{a}+\vec{b})$ and $(\vec{a}-3\vec{b})$
externally in the ratio 1 : 2. Also, show that P is the mid point of the line segment RQ.\\
	\solution
%		\input{chapters/12/10/5/9/section.tex}

\end{enumerate}


\item Find the area of a rhombus if its vertices are $(3,0), (4,5), (-1,4) \text{ and } (-2,-1)$ taken in order. [$\vec{Hint}$ : Area of rhombus =$\frac {1}{2}$(product of its diagonals)]
	\\
		\solution
	\iffalse
\documentclass[12pt]{article}
\usepackage{graphicx}
%\documentclass[journal,12pt,twocolumn]{IEEEtran}
\usepackage[none]{hyphenat}
\usepackage{graphicx}
\usepackage{listings}
\usepackage[english]{babel}
\usepackage{graphicx}
\usepackage{caption} 
\usepackage{hyperref}
\usepackage{booktabs}
\def\inputGnumericTable{}
\usepackage{color}                                            %%
    \usepackage{array}                                            %%
    \usepackage{longtable}                                        %%
    \usepackage{calc}                                             %%
    \usepackage{multirow}                                         %%
    \usepackage{hhline}                                           %%
    \usepackage{ifthen}
\usepackage{array}
\usepackage{amsmath}   % for having text in math mode
\usepackage{listings}
\lstset{
language=tex,
frame=single, 
breaklines=true
}
  
%Following 2 lines were added to remove the blank page at the beginning
\usepackage{atbegshi}% http://ctan.org/pkg/atbegshi
\AtBeginDocument{\AtBeginShipoutNext{\AtBeginShipoutDiscard}}
%


%New macro definitions
\newcommand{\mydet}[1]{\ensuremath{\begin{vmatrix}#1\end{vmatrix}}}
\providecommand{\brak}[1]{\ensuremath{\left(#1\right)}}
\providecommand{\norm}[1]{\left\lVert#1\right\rVert}
\newcommand{\solution}{\noindent \textbf{Solution: }}
\newcommand{\myvec}[1]{\ensuremath{\begin{pmatrix}#1\end{pmatrix}}}
\let\vec\mathbf

\begin{document}

\begin{center}
\title{\textbf{Coordinate Geometry}}
\date{\vspace{-5ex}} %Not to print date automatically
\maketitle
\end{center}

\setcounter{page}{1}



\begin{enumerate}

\item\textbf{Problem statement :} Find the area of a rhombus of its vertices are $\myvec{3 ,0}$, $\myvec{4 ,5}$, $\myvec{-1 ,4}$ and $\myvec{-2 ,-1}$taken in order

\solution \\
\fi
The input vertices for this problem are given as
	\begin{align}
	\vec{A} = \myvec{
		3\\
		0
		},
	\vec{B} = \myvec{
		4\\
		5
		},
        \vec{C} = \myvec{
		-1\\
		4
		},
        \vec{D} = \myvec{
		-2\\
		-1
		}
	\end{align}
Since		
\begin{align}
 \vec{A-D}= \myvec{3 \\ 0} - \myvec{-2 \\-1}= \myvec{5\\1}
 \\
  \vec{B-A}= \myvec{4 \\ 5} - \myvec{3 \\0}= \myvec{1\\5}
\end{align}
the area of the rhombus is
\begin{align}
                \norm{\myvec{\vec{A-D}}\times \myvec{\vec{B-A}}}=\mydet{5 & 1\\1 & 5} = 24
\end{align}
See Fig. 
\ref{fig:chapters/10/7/2/10/gFig1}.
\begin{figure}[!h]
 \begin{center}
  \includegraphics[width=\columnwidth]{chapters/10/7/2/10/figs/fig.pdf}
 \end{center}
\caption{}
\label{fig:chapters/10/7/2/10/gFig1}
\end{figure}

\item Find the position vector of a point R which divides the line joining two points $\vec{P}$
and $\vec{Q}$ whose position vectors are $\hat{i}+2\hat{j}-\hat{k}$ and $-\hat{i}+\hat{j}+\hat{k}$ respectively, in the
ratio 2 : 1
\begin{enumerate}
    \item  internally
    \item  externally
\end{enumerate}
\solution
		\begin{enumerate}[label=\thesection.\arabic*,ref=\thesection.\theenumi]
\numberwithin{equation}{enumi}
\numberwithin{figure}{enumi}
\numberwithin{table}{enumi}

\item Find the coordinates of the point which divides the join of $(-1,7) \text{ and } (4,-3)$ in the ratio 2:3.
	\\
		\solution
	\input{chapters/10/7/2/1/section.tex}
\item Find the coordinates of the points of trisection of the line segment joining $(4,-1) \text{ and } (-2,3)$.
	\\
		\solution
	\input{chapters/10/7/2/2/section.tex}
\item
	\iffalse
\item To conduct Sports Day activities, in your rectangular shaped school                   
ground ABCD, lines have 
drawn with chalk powder at a                 
distance of 1m each. 100 flower pots have been placed at a distance of 1m 
from each other along AD, as shown 
in Fig. 7.12. Niharika runs $ \frac {1}{4} $th the 
distance AD on the 2nd line and 
posts a green flag. Preet runs $ \frac {1}{5} $th 
the distance AD on the eighth line 
and posts a red flag. What is the 
distance between both the flags? If 
Rashmi has to post a blue flag exactly 
halfway between the line segment 
joining the two flags, where should 
she post her flag?
\begin{figure}[h!]
  \centering
  \includegraphics[width=\columnwidth]{sc.png}
  \caption{}
\label{fig:10/7/12Fig1}
\end{figure}               
\fi
      
\item Find the ratio in which the line segment joining the points $(-3,10) \text{ and } (6,-8)$ $\text{ is divided by } (-1,6)$.
	\\
		\solution
	\input{chapters/10/7/2/4/section.tex}
\item Find the ratio in which the line segment joining $A(1,-5) \text{ and } B(-4,5)$ $\text{is divided by the x-axis}$. Also find the coordinates of the point of division.
\item If $(1,2), (4,y), (x,6), (3,5)$ are the vertices of a parallelogram taken in order, find x and y.
	\\
		\solution
	\input{chapters/10/7/2/6/para1.tex}
\item Find the coordinates of a point A, where AB is the diameter of a circle whose centre is $(2,-3) \text{ and }$ B is $(1,4)$.
	\\
		\solution
	\input{chapters/10/7/2/7/section.tex}
\item If A \text{ and } B are $(-2,-2) \text{ and } (2,-4)$, respectively, find the coordinates of P such that AP= $\frac {3}{7}$AB $\text{ and }$ P lies on the line segment AB.
	\\
		\solution
	\input{chapters/10/7/2/8/section.tex}
\item Find the coordinates of the points which divide the line segment joining $A(-2,2) \text{ and } B(2,8)$ into four equal parts.
	\\
		\solution
	\input{chapters/10/7/2/9/section.tex}
\item Find the area of a rhombus if its vertices are $(3,0), (4,5), (-1,4) \text{ and } (-2,-1)$ taken in order. [$\vec{Hint}$ : Area of rhombus =$\frac {1}{2}$(product of its diagonals)]
	\\
		\solution
	\input{chapters/10/7/2/10/cross.tex}
\item Find the position vector of a point R which divides the line joining two points $\vec{P}$
and $\vec{Q}$ whose position vectors are $\hat{i}+2\hat{j}-\hat{k}$ and $-\hat{i}+\hat{j}+\hat{k}$ respectively, in the
ratio 2 : 1
\begin{enumerate}
    \item  internally
    \item  externally
\end{enumerate}
\solution
		\input{chapters/12/10/2/15/section.tex}
\item Find the position vector of the mid point of the vector joining the points $\vec{P}$(2, 3, 4)
and $\vec{Q}$(4, 1, –2).
\\
\solution
		\input{chapters/12/10/2/16/section.tex}
\item Determine the ratio in which the line $2x+y  - 4=0$ divides the line segment joining the points $\vec{A}(2, - 2)$  and  $\vec{B}(3, 7)$.
\\
\solution
	\input{chapters/10/7/4/1/section.tex}
\item Let $\vec{A}(4, 2), \vec{B}(6, 5)$  and $ \vec{C}(1, 4)$ be the vertices of $\triangle ABC$.
\begin{enumerate}
\item The median from $\vec{A}$ meets $BC$ at $\vec{D}$. Find the coordinates of the point $\vec{D}$.
\item Find the coordinates of the point $\vec{P}$ on $AD$ such that $AP : PD = 2 : 1$.
\item Find the coordinates of points $\vec{Q}$ and $\vec{R}$ on medians $BE$ and $CF$ respectively such that $BQ : QE = 2 : 1$  and  $CR : RF = 2 : 1$.
\item What do you observe?
\item If $\vec{A}, \vec{B}$ and $\vec{C}$  are the vertices of $\triangle ABC$, find the coordinates of the centroid of the triangle.
\end{enumerate}
\solution
	\input{chapters/10/7/4/7/section.tex}
\item Find the slope of a line, which passes through the origin and the mid point of the line segment joining the points $\vec{P}$(0,-4) and $\vec{B}$(8,0).
\label{chapters/11/10/1/5}
\input{chapters/11/10/1/5/matrix.tex}
\item Find the position vector of a point R which divides the line joining two points P and Q whose position vectors are $(2\vec{a}+\vec{b})$ and $(\vec{a}-3\vec{b})$
externally in the ratio 1 : 2. Also, show that P is the mid point of the line segment RQ.\\
	\solution
%		\input{chapters/12/10/5/9/section.tex}

\end{enumerate}


\item Find the position vector of the mid point of the vector joining the points $\vec{P}$(2, 3, 4)
and $\vec{Q}$(4, 1, –2).
\\
\solution
		\begin{enumerate}[label=\thesection.\arabic*,ref=\thesection.\theenumi]
\numberwithin{equation}{enumi}
\numberwithin{figure}{enumi}
\numberwithin{table}{enumi}

\item Find the coordinates of the point which divides the join of $(-1,7) \text{ and } (4,-3)$ in the ratio 2:3.
	\\
		\solution
	\input{chapters/10/7/2/1/section.tex}
\item Find the coordinates of the points of trisection of the line segment joining $(4,-1) \text{ and } (-2,3)$.
	\\
		\solution
	\input{chapters/10/7/2/2/section.tex}
\item
	\iffalse
\item To conduct Sports Day activities, in your rectangular shaped school                   
ground ABCD, lines have 
drawn with chalk powder at a                 
distance of 1m each. 100 flower pots have been placed at a distance of 1m 
from each other along AD, as shown 
in Fig. 7.12. Niharika runs $ \frac {1}{4} $th the 
distance AD on the 2nd line and 
posts a green flag. Preet runs $ \frac {1}{5} $th 
the distance AD on the eighth line 
and posts a red flag. What is the 
distance between both the flags? If 
Rashmi has to post a blue flag exactly 
halfway between the line segment 
joining the two flags, where should 
she post her flag?
\begin{figure}[h!]
  \centering
  \includegraphics[width=\columnwidth]{sc.png}
  \caption{}
\label{fig:10/7/12Fig1}
\end{figure}               
\fi
      
\item Find the ratio in which the line segment joining the points $(-3,10) \text{ and } (6,-8)$ $\text{ is divided by } (-1,6)$.
	\\
		\solution
	\input{chapters/10/7/2/4/section.tex}
\item Find the ratio in which the line segment joining $A(1,-5) \text{ and } B(-4,5)$ $\text{is divided by the x-axis}$. Also find the coordinates of the point of division.
\item If $(1,2), (4,y), (x,6), (3,5)$ are the vertices of a parallelogram taken in order, find x and y.
	\\
		\solution
	\input{chapters/10/7/2/6/para1.tex}
\item Find the coordinates of a point A, where AB is the diameter of a circle whose centre is $(2,-3) \text{ and }$ B is $(1,4)$.
	\\
		\solution
	\input{chapters/10/7/2/7/section.tex}
\item If A \text{ and } B are $(-2,-2) \text{ and } (2,-4)$, respectively, find the coordinates of P such that AP= $\frac {3}{7}$AB $\text{ and }$ P lies on the line segment AB.
	\\
		\solution
	\input{chapters/10/7/2/8/section.tex}
\item Find the coordinates of the points which divide the line segment joining $A(-2,2) \text{ and } B(2,8)$ into four equal parts.
	\\
		\solution
	\input{chapters/10/7/2/9/section.tex}
\item Find the area of a rhombus if its vertices are $(3,0), (4,5), (-1,4) \text{ and } (-2,-1)$ taken in order. [$\vec{Hint}$ : Area of rhombus =$\frac {1}{2}$(product of its diagonals)]
	\\
		\solution
	\input{chapters/10/7/2/10/cross.tex}
\item Find the position vector of a point R which divides the line joining two points $\vec{P}$
and $\vec{Q}$ whose position vectors are $\hat{i}+2\hat{j}-\hat{k}$ and $-\hat{i}+\hat{j}+\hat{k}$ respectively, in the
ratio 2 : 1
\begin{enumerate}
    \item  internally
    \item  externally
\end{enumerate}
\solution
		\input{chapters/12/10/2/15/section.tex}
\item Find the position vector of the mid point of the vector joining the points $\vec{P}$(2, 3, 4)
and $\vec{Q}$(4, 1, –2).
\\
\solution
		\input{chapters/12/10/2/16/section.tex}
\item Determine the ratio in which the line $2x+y  - 4=0$ divides the line segment joining the points $\vec{A}(2, - 2)$  and  $\vec{B}(3, 7)$.
\\
\solution
	\input{chapters/10/7/4/1/section.tex}
\item Let $\vec{A}(4, 2), \vec{B}(6, 5)$  and $ \vec{C}(1, 4)$ be the vertices of $\triangle ABC$.
\begin{enumerate}
\item The median from $\vec{A}$ meets $BC$ at $\vec{D}$. Find the coordinates of the point $\vec{D}$.
\item Find the coordinates of the point $\vec{P}$ on $AD$ such that $AP : PD = 2 : 1$.
\item Find the coordinates of points $\vec{Q}$ and $\vec{R}$ on medians $BE$ and $CF$ respectively such that $BQ : QE = 2 : 1$  and  $CR : RF = 2 : 1$.
\item What do you observe?
\item If $\vec{A}, \vec{B}$ and $\vec{C}$  are the vertices of $\triangle ABC$, find the coordinates of the centroid of the triangle.
\end{enumerate}
\solution
	\input{chapters/10/7/4/7/section.tex}
\item Find the slope of a line, which passes through the origin and the mid point of the line segment joining the points $\vec{P}$(0,-4) and $\vec{B}$(8,0).
\label{chapters/11/10/1/5}
\input{chapters/11/10/1/5/matrix.tex}
\item Find the position vector of a point R which divides the line joining two points P and Q whose position vectors are $(2\vec{a}+\vec{b})$ and $(\vec{a}-3\vec{b})$
externally in the ratio 1 : 2. Also, show that P is the mid point of the line segment RQ.\\
	\solution
%		\input{chapters/12/10/5/9/section.tex}

\end{enumerate}


\item Determine the ratio in which the line $2x+y  - 4=0$ divides the line segment joining the points $\vec{A}(2, - 2)$  and  $\vec{B}(3, 7)$.
\\
\solution
	\iffalse
\documentclass[journal,12pt,twocolumn]{IEEEtran}
\usepackage{graphicx}
\graphicspath{{./chapters/10/7/4/1/figs/}}{}
\usepackage{amsmath,amssymb,amsfonts,amsthm}
\newcommand{\myvec}[1]{\ensuremath{\begin{pmatrix}#1\end{pmatrix}}}
\providecommand{\norm}[1]{\lVert#1\rVert}
\usepackage{listings}
\usepackage{watermark}
\usepackage{titlesec}
\usepackage{caption}
\let\vec\mathbf
\lstset{
frame=single, 
breaklines=true,
columns=fullflexible
}
\thiswatermark{\centering \put(0,-105.0){\includegraphics[scale=0.15]{/sdcard/IITH/vector/vectpr-4/chapters/10/7/4/1/figs/logo.png}} }
\title{\mytitle}
\title{
Assignment - Vector-4
}
\author{Surajit Sarkar}
\begin{document}
\maketitle
%\tableofcontents
\bigskip
\section{\textbf{Problem}}
Determine the ratio in which the line 2x+y–4=0 divides the line segment joining the points A(2,–2) and B(3,7).
\section{\textbf{Solution}}
\begin{table}[h]
    \centering
    \begin{tabular}{|c|c|}
       \hline
       \textbf{Symbol}&\textbf{Value}  \\
       \hline
	    $\vec{A}$ & $\myvec{2\\-2}$\\
        \hline
	    $\vec{B}$ & $\myvec{3\\7}$\\
        \hline
	    c&$4$\\
        \hline
       $\vec{n}$ & $\myvec{2\\1}$\\
       \hline
    \end{tabular}
    \caption{Parameters}
    \label{tab:my_label}
\end{table}
Given equation
\fi
The given equation can be expressed as
\begin{align}
    \myvec{2&1}\vec{x}&=4\\
\end{align}
Using section formula, the point of division 
\begin{align}
    \vec{P} = \frac{k\vec{B+A}}{k+1}
\end{align}
which upon substitution in the equation of a line yields
\begin{align}
    \implies\vec{n}^{\top}\myvec{\frac{k\vec{B+A}}{k+1}}&=c\\
    \implies k&=\frac{c-\vec{n}^{\top}\vec{A}}{\vec{n}^{\top}\vec{B}-c}\\
\end{align}
upon simplification.  Substituting numerical values, 
\begin{align}
    k=\frac{2}{9}
\end{align}
See Fig. 
\ref{fig:chapters/10/7/4/1vec}.
\begin{figure}[!h]
\centering
\includegraphics[width=\columnwidth]{chapters/10/7/4/1/figs/vec.pdf}
\caption{}
\label{fig:chapters/10/7/4/1vec}
\end{figure}


\item Let $\vec{A}(4, 2), \vec{B}(6, 5)$  and $ \vec{C}(1, 4)$ be the vertices of $\triangle ABC$.
\begin{enumerate}
\item The median from $\vec{A}$ meets $BC$ at $\vec{D}$. Find the coordinates of the point $\vec{D}$.
\item Find the coordinates of the point $\vec{P}$ on $AD$ such that $AP : PD = 2 : 1$.
\item Find the coordinates of points $\vec{Q}$ and $\vec{R}$ on medians $BE$ and $CF$ respectively such that $BQ : QE = 2 : 1$  and  $CR : RF = 2 : 1$.
\item What do you observe?
\item If $\vec{A}, \vec{B}$ and $\vec{C}$  are the vertices of $\triangle ABC$, find the coordinates of the centroid of the triangle.
\end{enumerate}
\solution
	\iffalse
\documentclass[12pt]{article}
\usepackage{graphicx}
\usepackage[none]{hyphenat}
\usepackage{graphicx}
\usepackage{listings}
\usepackage[english]{babel}
\usepackage{graphicx}
\usepackage{caption} 
\usepackage{booktabs}
\usepackage{array}
\usepackage{amssymb} % for \because
\usepackage{amsmath}   % for having text in math mode
\usepackage{extarrows} % for Row operations arrows
\usepackage{listings}
\usepackage[utf8]{inputenc}
\lstset{
  frame=single,
  breaklines=true
}
\usepackage{hyperref}
  
%Following 2 lines were added to remove the blank page at the beginning
\usepackage{atbegshi}% http://ctan.org/pkg/atbegshi
\AtBeginDocument{\AtBeginShipoutNext{\AtBeginShipoutDiscard}}


%New macro definitions
\newcommand{\mydet}[1]{\ensuremath{\begin{vmatrix}#1\end{vmatrix}}}
\providecommand{\brak}[1]{\ensuremath{\left(#1\right)}}
\newcommand{\solution}{\noindent \textbf{Solution: }}
\newcommand{\myvec}[1]{\ensuremath{\begin{pmatrix}#1\end{pmatrix}}}
\providecommand{\norm}[1]{\left\lVert#1\right\rVert}
\providecommand{\abs}[1]{\left\vert#1\right\vert}
\let\vec\mathbf

\begin{document}

\begin{center}
\title{\textbf{VECTORS}}
\date{\vspace{-5ex}} %Not to print date automatically
\maketitle
\end{center}

\section{10$^{th}$ Maths - EXERCISE-7.4}

Let A(4, 2), B(6, 5) and C(1, 4) be the vertices of $\triangle ABC$
\begin{enumerate}
\item The median from A meets BC at D. Find the coordinates of the point D.
\item Find the coordinates of the point P on AD such that $AP : PD = 2 : 1$
\item Find the coordinates of points Q and R on medians BE and CF respectively such
that $BQ : QE = 2 : 1 \text{and} CR : RF = 2 : 1.$
\item What do yo observe?
\item If $A(x_1, y_1), B(x_2, y_2) \text{and} C(x_3, y_3)$ are the vertices of $\triangle ABC$, find the coordinates of the centroid of the triangle.
\end{enumerate}

Given points are
\begin{align}
\vec{A}=\myvec{4\\ 2} ,
\vec{B}=\myvec{6\\ 5} ,
\vec{C}=\myvec{1\\ 4}
\end{align}
\fi

\begin{enumerate}
\item 
\begin{align}
\vec{D}&=\frac{\vec{B}+\vec{C}}{2}\\
&=\myvec{\frac{7}{2}\\[2pt] \frac{9}{2}}\\
\vec{E}&=\frac{\vec{A}+\vec{C}}{2}\\
&=\myvec{\frac{5}{2}\\ 3}\\
\vec{F}&=\frac{\vec{A}+\vec{B}}{2}\\
&=\myvec{5\\ \frac{7}{2}}
\end{align}

\item 
	For
$n=2$,
\begin{align}
\vec{P}&=\frac{1}{1+n}\brak{\myvec{\vec{A}+n\vec{D}}}\\
&=\frac{1}{3}\myvec{11\\11}
\end{align}

\item 
\begin{align}
\vec{Q}&=\frac{1}{1+n}\brak{\myvec{\vec{B}+n\vec{E}}}\\
&=\frac{1}{3}\myvec{11\\11}\\
\vec{R}&=\frac{1}{1+n}\brak{\myvec{\vec{C}+n\vec{F}}}\\
&=\frac{1}{3}\myvec{11\\11}\\
\end{align}

\item 
 $\vec{P},\vec{Q},\vec{R}$ are the same point.
   
\item 
\begin{align}
\vec{G}&=\frac{\vec{D}+\vec{E}+\vec{F}}{3}\\
&=\frac{1}{3}\myvec{11\\11}\\
\end{align} 
\end{enumerate}
See Fig.  
  \ref{fig:chapters/10/7/4/7/Figure}.
\begin{figure}[h!]
\centering
\includegraphics[width=\columnwidth]{chapters/10/7/4/7/figs/dj.pdf}
\caption{}
  \label{fig:chapters/10/7/4/7/Figure}
\end{figure}

\item Find the slope of a line, which passes through the origin and the mid point of the line segment joining the points $\vec{P}$(0,-4) and $\vec{B}$(8,0).
\label{chapters/11/10/1/5}
\iffalse
\documentclass[journal,12pt,twocolumn]{IEEEtran}
\usepackage{graphicx}
\graphicspath{{./figs/}}{}
\usepackage{amsmath,amssymb,amsfonts,amsthm}
\newcommand{\myvec}[1]{\ensuremath{\begin{pmatrix}#1\end{pmatrix}}}

\let\vec\mathbf

\title{
Matrix-Lines
}
\author{Jyothsna Paluchuri-FWC22059\\}
\begin{document}
\maketitle
\tableofcontents
\bigskip
\section{Problem Statement}
\fi
	\begin{figure}[!ht]
		\centering
 \includegraphics[width=\columnwidth]{chapters/11/10/1/5/figs/line.png}
		\caption{}
		\label{fig:11/10/1/5}
  	\end{figure}
	\\
	\solution
\iffalse
\section{Construction}
\begin{figure}[h]
    \centering
\includegraphics[width=\columnwidth]{line.png}
    \caption{Equation of the slope}
    \label{fig:my_label}
\end{figure}
\vspace{2cm}
\begin{table}[h]
    \centering
    \begin{tabular}{|c|c|c|c|}
       \hline
       \textbf{Symbol}&\textbf{Value}&\textbf{Description}  \\
       \hline
	    $\vec{P}$ & $\myvec{
		    0\\
		    -4}$
	    & Point on Y-axis\\
        \hline
	    $\vec{B}$ & $\myvec{8\\0}$
 & Point on X-axis\\
        \hline
	    $\vec{0}$ & $\myvec{0\\0}$
 & Origin\\
        \hline
    \end{tabular}
    \caption{Parameters}
    \label{tab:my_label}
\end{table}


\section{Solution}
Given that resultant line passes through origin and mid point of the line segment joining point P(0,-4) and B(8,0) \\
\\
\\
given ${\vec{P}}$=$\myvec{
  0\\
  -4}$
 , ${\vec{B}}$=$\myvec{
  8\\
  0}$
  
 \fi 
The mid point of $PB$ is
\begin{align}
\vec{M} &=\frac{1}{2}(\vec{P}+\vec{B})
	= \myvec{4 \\ -2}  
\end{align}
The direction vector of line joining $\vec{O}, \vec{M}$ is 
\begin{align}
\vec{m}&=\vec{O}-\vec{M}
 = -\vec{M}
\end{align}
which can be expressed as
\begin{align}
	\myvec{1 \\ -\frac{1}{2}}
\end{align}
Thus the slope is
\begin{align}
	m = -\frac{1}{2}
\end{align}
\iffalse
\textbf{The direction vector of a line expressed as}
\begin{align}
\implies\vec{m} &= \begin{pmatrix}1 \\ m \\ \end{pmatrix}
\end{align}

\textbf{By solving equation (5) and (6),we get the slope of $\vec{O}$ $\vec{M}$ line}
\begin{align}
        \boxed{m=-0.5}
 \end{align}

\section{Software}
Download the following code using,
\begin{table}[h]
    \centering
    \begin{tabular}{|c|}
    \hline \\
   https://github.com/jyothsna777/jyothsna-fwc.git  \\
         \\
\hline
    \end{tabular}
\end{table}
\\
and execute the code by using command
\begin{center}
\textbf{Python3 lines.py}\\
\end{center}

\section{Conclusion}
Hence the slope of line $\vec{O}$ $\vec{M}$ lineis $\vec{m}$=-0.5

\end{document}
\fi

\item Find the position vector of a point R which divides the line joining two points P and Q whose position vectors are $(2\vec{a}+\vec{b})$ and $(\vec{a}-3\vec{b})$
externally in the ratio 1 : 2. Also, show that P is the mid point of the line segment RQ.\\
	\solution
%		\begin{enumerate}[label=\thesection.\arabic*,ref=\thesection.\theenumi]
\numberwithin{equation}{enumi}
\numberwithin{figure}{enumi}
\numberwithin{table}{enumi}

\item Find the coordinates of the point which divides the join of $(-1,7) \text{ and } (4,-3)$ in the ratio 2:3.
	\\
		\solution
	\input{chapters/10/7/2/1/section.tex}
\item Find the coordinates of the points of trisection of the line segment joining $(4,-1) \text{ and } (-2,3)$.
	\\
		\solution
	\input{chapters/10/7/2/2/section.tex}
\item
	\iffalse
\item To conduct Sports Day activities, in your rectangular shaped school                   
ground ABCD, lines have 
drawn with chalk powder at a                 
distance of 1m each. 100 flower pots have been placed at a distance of 1m 
from each other along AD, as shown 
in Fig. 7.12. Niharika runs $ \frac {1}{4} $th the 
distance AD on the 2nd line and 
posts a green flag. Preet runs $ \frac {1}{5} $th 
the distance AD on the eighth line 
and posts a red flag. What is the 
distance between both the flags? If 
Rashmi has to post a blue flag exactly 
halfway between the line segment 
joining the two flags, where should 
she post her flag?
\begin{figure}[h!]
  \centering
  \includegraphics[width=\columnwidth]{sc.png}
  \caption{}
\label{fig:10/7/12Fig1}
\end{figure}               
\fi
      
\item Find the ratio in which the line segment joining the points $(-3,10) \text{ and } (6,-8)$ $\text{ is divided by } (-1,6)$.
	\\
		\solution
	\input{chapters/10/7/2/4/section.tex}
\item Find the ratio in which the line segment joining $A(1,-5) \text{ and } B(-4,5)$ $\text{is divided by the x-axis}$. Also find the coordinates of the point of division.
\item If $(1,2), (4,y), (x,6), (3,5)$ are the vertices of a parallelogram taken in order, find x and y.
	\\
		\solution
	\input{chapters/10/7/2/6/para1.tex}
\item Find the coordinates of a point A, where AB is the diameter of a circle whose centre is $(2,-3) \text{ and }$ B is $(1,4)$.
	\\
		\solution
	\input{chapters/10/7/2/7/section.tex}
\item If A \text{ and } B are $(-2,-2) \text{ and } (2,-4)$, respectively, find the coordinates of P such that AP= $\frac {3}{7}$AB $\text{ and }$ P lies on the line segment AB.
	\\
		\solution
	\input{chapters/10/7/2/8/section.tex}
\item Find the coordinates of the points which divide the line segment joining $A(-2,2) \text{ and } B(2,8)$ into four equal parts.
	\\
		\solution
	\input{chapters/10/7/2/9/section.tex}
\item Find the area of a rhombus if its vertices are $(3,0), (4,5), (-1,4) \text{ and } (-2,-1)$ taken in order. [$\vec{Hint}$ : Area of rhombus =$\frac {1}{2}$(product of its diagonals)]
	\\
		\solution
	\input{chapters/10/7/2/10/cross.tex}
\item Find the position vector of a point R which divides the line joining two points $\vec{P}$
and $\vec{Q}$ whose position vectors are $\hat{i}+2\hat{j}-\hat{k}$ and $-\hat{i}+\hat{j}+\hat{k}$ respectively, in the
ratio 2 : 1
\begin{enumerate}
    \item  internally
    \item  externally
\end{enumerate}
\solution
		\input{chapters/12/10/2/15/section.tex}
\item Find the position vector of the mid point of the vector joining the points $\vec{P}$(2, 3, 4)
and $\vec{Q}$(4, 1, –2).
\\
\solution
		\input{chapters/12/10/2/16/section.tex}
\item Determine the ratio in which the line $2x+y  - 4=0$ divides the line segment joining the points $\vec{A}(2, - 2)$  and  $\vec{B}(3, 7)$.
\\
\solution
	\input{chapters/10/7/4/1/section.tex}
\item Let $\vec{A}(4, 2), \vec{B}(6, 5)$  and $ \vec{C}(1, 4)$ be the vertices of $\triangle ABC$.
\begin{enumerate}
\item The median from $\vec{A}$ meets $BC$ at $\vec{D}$. Find the coordinates of the point $\vec{D}$.
\item Find the coordinates of the point $\vec{P}$ on $AD$ such that $AP : PD = 2 : 1$.
\item Find the coordinates of points $\vec{Q}$ and $\vec{R}$ on medians $BE$ and $CF$ respectively such that $BQ : QE = 2 : 1$  and  $CR : RF = 2 : 1$.
\item What do you observe?
\item If $\vec{A}, \vec{B}$ and $\vec{C}$  are the vertices of $\triangle ABC$, find the coordinates of the centroid of the triangle.
\end{enumerate}
\solution
	\input{chapters/10/7/4/7/section.tex}
\item Find the slope of a line, which passes through the origin and the mid point of the line segment joining the points $\vec{P}$(0,-4) and $\vec{B}$(8,0).
\label{chapters/11/10/1/5}
\input{chapters/11/10/1/5/matrix.tex}
\item Find the position vector of a point R which divides the line joining two points P and Q whose position vectors are $(2\vec{a}+\vec{b})$ and $(\vec{a}-3\vec{b})$
externally in the ratio 1 : 2. Also, show that P is the mid point of the line segment RQ.\\
	\solution
%		\input{chapters/12/10/5/9/section.tex}

\end{enumerate}



\end{enumerate}


\item Determine the ratio in which the line $2x+y  - 4=0$ divides the line segment joining the points $\vec{A}(2, - 2)$  and  $\vec{B}(3, 7)$.
\\
\solution
	\iffalse
\documentclass[journal,12pt,twocolumn]{IEEEtran}
\usepackage{graphicx}
\graphicspath{{./chapters/10/7/4/1/figs/}}{}
\usepackage{amsmath,amssymb,amsfonts,amsthm}
\newcommand{\myvec}[1]{\ensuremath{\begin{pmatrix}#1\end{pmatrix}}}
\providecommand{\norm}[1]{\lVert#1\rVert}
\usepackage{listings}
\usepackage{watermark}
\usepackage{titlesec}
\usepackage{caption}
\let\vec\mathbf
\lstset{
frame=single, 
breaklines=true,
columns=fullflexible
}
\thiswatermark{\centering \put(0,-105.0){\includegraphics[scale=0.15]{/sdcard/IITH/vector/vectpr-4/chapters/10/7/4/1/figs/logo.png}} }
\title{\mytitle}
\title{
Assignment - Vector-4
}
\author{Surajit Sarkar}
\begin{document}
\maketitle
%\tableofcontents
\bigskip
\section{\textbf{Problem}}
Determine the ratio in which the line 2x+y–4=0 divides the line segment joining the points A(2,–2) and B(3,7).
\section{\textbf{Solution}}
\begin{table}[h]
    \centering
    \begin{tabular}{|c|c|}
       \hline
       \textbf{Symbol}&\textbf{Value}  \\
       \hline
	    $\vec{A}$ & $\myvec{2\\-2}$\\
        \hline
	    $\vec{B}$ & $\myvec{3\\7}$\\
        \hline
	    c&$4$\\
        \hline
       $\vec{n}$ & $\myvec{2\\1}$\\
       \hline
    \end{tabular}
    \caption{Parameters}
    \label{tab:my_label}
\end{table}
Given equation
\fi
The given equation can be expressed as
\begin{align}
    \myvec{2&1}\vec{x}&=4\\
\end{align}
Using section formula, the point of division 
\begin{align}
    \vec{P} = \frac{k\vec{B+A}}{k+1}
\end{align}
which upon substitution in the equation of a line yields
\begin{align}
    \implies\vec{n}^{\top}\myvec{\frac{k\vec{B+A}}{k+1}}&=c\\
    \implies k&=\frac{c-\vec{n}^{\top}\vec{A}}{\vec{n}^{\top}\vec{B}-c}\\
\end{align}
upon simplification.  Substituting numerical values, 
\begin{align}
    k=\frac{2}{9}
\end{align}
See Fig. 
\ref{fig:chapters/10/7/4/1vec}.
\begin{figure}[!h]
\centering
\includegraphics[width=\columnwidth]{chapters/10/7/4/1/figs/vec.pdf}
\caption{}
\label{fig:chapters/10/7/4/1vec}
\end{figure}


\item Let $\vec{A}(4, 2), \vec{B}(6, 5)$  and $ \vec{C}(1, 4)$ be the vertices of $\triangle ABC$.
\begin{enumerate}
\item The median from $\vec{A}$ meets $BC$ at $\vec{D}$. Find the coordinates of the point $\vec{D}$.
\item Find the coordinates of the point $\vec{P}$ on $AD$ such that $AP : PD = 2 : 1$.
\item Find the coordinates of points $\vec{Q}$ and $\vec{R}$ on medians $BE$ and $CF$ respectively such that $BQ : QE = 2 : 1$  and  $CR : RF = 2 : 1$.
\item What do you observe?
\item If $\vec{A}, \vec{B}$ and $\vec{C}$  are the vertices of $\triangle ABC$, find the coordinates of the centroid of the triangle.
\end{enumerate}
\solution
	\iffalse
\documentclass[12pt]{article}
\usepackage{graphicx}
\usepackage[none]{hyphenat}
\usepackage{graphicx}
\usepackage{listings}
\usepackage[english]{babel}
\usepackage{graphicx}
\usepackage{caption} 
\usepackage{booktabs}
\usepackage{array}
\usepackage{amssymb} % for \because
\usepackage{amsmath}   % for having text in math mode
\usepackage{extarrows} % for Row operations arrows
\usepackage{listings}
\usepackage[utf8]{inputenc}
\lstset{
  frame=single,
  breaklines=true
}
\usepackage{hyperref}
  
%Following 2 lines were added to remove the blank page at the beginning
\usepackage{atbegshi}% http://ctan.org/pkg/atbegshi
\AtBeginDocument{\AtBeginShipoutNext{\AtBeginShipoutDiscard}}


%New macro definitions
\newcommand{\mydet}[1]{\ensuremath{\begin{vmatrix}#1\end{vmatrix}}}
\providecommand{\brak}[1]{\ensuremath{\left(#1\right)}}
\newcommand{\solution}{\noindent \textbf{Solution: }}
\newcommand{\myvec}[1]{\ensuremath{\begin{pmatrix}#1\end{pmatrix}}}
\providecommand{\norm}[1]{\left\lVert#1\right\rVert}
\providecommand{\abs}[1]{\left\vert#1\right\vert}
\let\vec\mathbf

\begin{document}

\begin{center}
\title{\textbf{VECTORS}}
\date{\vspace{-5ex}} %Not to print date automatically
\maketitle
\end{center}

\section{10$^{th}$ Maths - EXERCISE-7.4}

Let A(4, 2), B(6, 5) and C(1, 4) be the vertices of $\triangle ABC$
\begin{enumerate}
\item The median from A meets BC at D. Find the coordinates of the point D.
\item Find the coordinates of the point P on AD such that $AP : PD = 2 : 1$
\item Find the coordinates of points Q and R on medians BE and CF respectively such
that $BQ : QE = 2 : 1 \text{and} CR : RF = 2 : 1.$
\item What do yo observe?
\item If $A(x_1, y_1), B(x_2, y_2) \text{and} C(x_3, y_3)$ are the vertices of $\triangle ABC$, find the coordinates of the centroid of the triangle.
\end{enumerate}

Given points are
\begin{align}
\vec{A}=\myvec{4\\ 2} ,
\vec{B}=\myvec{6\\ 5} ,
\vec{C}=\myvec{1\\ 4}
\end{align}
\fi

\begin{enumerate}
\item 
\begin{align}
\vec{D}&=\frac{\vec{B}+\vec{C}}{2}\\
&=\myvec{\frac{7}{2}\\[2pt] \frac{9}{2}}\\
\vec{E}&=\frac{\vec{A}+\vec{C}}{2}\\
&=\myvec{\frac{5}{2}\\ 3}\\
\vec{F}&=\frac{\vec{A}+\vec{B}}{2}\\
&=\myvec{5\\ \frac{7}{2}}
\end{align}

\item 
	For
$n=2$,
\begin{align}
\vec{P}&=\frac{1}{1+n}\brak{\myvec{\vec{A}+n\vec{D}}}\\
&=\frac{1}{3}\myvec{11\\11}
\end{align}

\item 
\begin{align}
\vec{Q}&=\frac{1}{1+n}\brak{\myvec{\vec{B}+n\vec{E}}}\\
&=\frac{1}{3}\myvec{11\\11}\\
\vec{R}&=\frac{1}{1+n}\brak{\myvec{\vec{C}+n\vec{F}}}\\
&=\frac{1}{3}\myvec{11\\11}\\
\end{align}

\item 
 $\vec{P},\vec{Q},\vec{R}$ are the same point.
   
\item 
\begin{align}
\vec{G}&=\frac{\vec{D}+\vec{E}+\vec{F}}{3}\\
&=\frac{1}{3}\myvec{11\\11}\\
\end{align} 
\end{enumerate}
See Fig.  
  \ref{fig:chapters/10/7/4/7/Figure}.
\begin{figure}[h!]
\centering
\includegraphics[width=\columnwidth]{chapters/10/7/4/7/figs/dj.pdf}
\caption{}
  \label{fig:chapters/10/7/4/7/Figure}
\end{figure}

\item Find the slope of a line, which passes through the origin and the mid point of the line segment joining the points $\vec{P}$(0,-4) and $\vec{B}$(8,0).
\label{chapters/11/10/1/5}
\iffalse
\documentclass[journal,12pt,twocolumn]{IEEEtran}
\usepackage{graphicx}
\graphicspath{{./figs/}}{}
\usepackage{amsmath,amssymb,amsfonts,amsthm}
\newcommand{\myvec}[1]{\ensuremath{\begin{pmatrix}#1\end{pmatrix}}}

\let\vec\mathbf

\title{
Matrix-Lines
}
\author{Jyothsna Paluchuri-FWC22059\\}
\begin{document}
\maketitle
\tableofcontents
\bigskip
\section{Problem Statement}
\fi
	\begin{figure}[!ht]
		\centering
 \includegraphics[width=\columnwidth]{chapters/11/10/1/5/figs/line.png}
		\caption{}
		\label{fig:11/10/1/5}
  	\end{figure}
	\\
	\solution
\iffalse
\section{Construction}
\begin{figure}[h]
    \centering
\includegraphics[width=\columnwidth]{line.png}
    \caption{Equation of the slope}
    \label{fig:my_label}
\end{figure}
\vspace{2cm}
\begin{table}[h]
    \centering
    \begin{tabular}{|c|c|c|c|}
       \hline
       \textbf{Symbol}&\textbf{Value}&\textbf{Description}  \\
       \hline
	    $\vec{P}$ & $\myvec{
		    0\\
		    -4}$
	    & Point on Y-axis\\
        \hline
	    $\vec{B}$ & $\myvec{8\\0}$
 & Point on X-axis\\
        \hline
	    $\vec{0}$ & $\myvec{0\\0}$
 & Origin\\
        \hline
    \end{tabular}
    \caption{Parameters}
    \label{tab:my_label}
\end{table}


\section{Solution}
Given that resultant line passes through origin and mid point of the line segment joining point P(0,-4) and B(8,0) \\
\\
\\
given ${\vec{P}}$=$\myvec{
  0\\
  -4}$
 , ${\vec{B}}$=$\myvec{
  8\\
  0}$
  
 \fi 
The mid point of $PB$ is
\begin{align}
\vec{M} &=\frac{1}{2}(\vec{P}+\vec{B})
	= \myvec{4 \\ -2}  
\end{align}
The direction vector of line joining $\vec{O}, \vec{M}$ is 
\begin{align}
\vec{m}&=\vec{O}-\vec{M}
 = -\vec{M}
\end{align}
which can be expressed as
\begin{align}
	\myvec{1 \\ -\frac{1}{2}}
\end{align}
Thus the slope is
\begin{align}
	m = -\frac{1}{2}
\end{align}
\iffalse
\textbf{The direction vector of a line expressed as}
\begin{align}
\implies\vec{m} &= \begin{pmatrix}1 \\ m \\ \end{pmatrix}
\end{align}

\textbf{By solving equation (5) and (6),we get the slope of $\vec{O}$ $\vec{M}$ line}
\begin{align}
        \boxed{m=-0.5}
 \end{align}

\section{Software}
Download the following code using,
\begin{table}[h]
    \centering
    \begin{tabular}{|c|}
    \hline \\
   https://github.com/jyothsna777/jyothsna-fwc.git  \\
         \\
\hline
    \end{tabular}
\end{table}
\\
and execute the code by using command
\begin{center}
\textbf{Python3 lines.py}\\
\end{center}

\section{Conclusion}
Hence the slope of line $\vec{O}$ $\vec{M}$ lineis $\vec{m}$=-0.5

\end{document}
\fi

\item Find the position vector of a point R which divides the line joining two points P and Q whose position vectors are $(2\vec{a}+\vec{b})$ and $(\vec{a}-3\vec{b})$
externally in the ratio 1 : 2. Also, show that P is the mid point of the line segment RQ.\\
	\solution
%		\begin{enumerate}[label=\thesection.\arabic*,ref=\thesection.\theenumi]
\numberwithin{equation}{enumi}
\numberwithin{figure}{enumi}
\numberwithin{table}{enumi}

\item Find the coordinates of the point which divides the join of $(-1,7) \text{ and } (4,-3)$ in the ratio 2:3.
	\\
		\solution
	\iffalse
\documentclass[12pt]{article}
\usepackage{graphicx}
\usepackage{amsmath}
\usepackage{mathtools}
\usepackage{gensymb}

\newcommand{\mydet}[1]{\ensuremath{\begin{vmatrix}#1\end{vmatrix}}}
\providecommand{\brak}[1]{\ensuremath{\left(#1\right)}}
\providecommand{\norm}[1]{\left\lVert#1\right\rVert}
\newcommand{\solution}{\noindent \textbf{Solution: }}
\newcommand{\myvec}[1]{\ensuremath{\begin{pmatrix}#1\end{pmatrix}}}
\let\vec\mathbf

\begin{document}
\begin{center}
\textbf\large{CHAPTER-7 \\ COORDINATE GEOMETRY}
\end{center}
\section*{Excercise 7.2}

1. Find the coordinates of the point which divides the join $\vec(-1,7) \text{ and } \vec(4,-3)$ in the ratio 2:3 :
\\
\\
\solution\\		
\fi
The coordinates and ratio are given as
\begin{align}
\vec{P}=\myvec{-1\\7\\},
\vec{Q}=\myvec{4\\-3\\},
n=\frac{3}{2}
\end{align}
Using section formula
\begin{align}
\vec{R}&=\frac{\vec{Q}+n\vec{P}}{1+n}\\
&=\frac{1}{1+\frac{3}{2}}  \myvec{\myvec{
4\\
-3\\
}
  +
   \frac{3}{2}\myvec{
-1\\
7\\
}}\\
&=\myvec{
1\\
3
}
\end{align}
See Fig. 
\ref{fig:chapters/10/7/2/1/Fig}
\begin{figure}[!h]
\begin{center}
   \includegraphics[width=\columnwidth]{chapters/10/7/2/1/figs/linefig.png}
\end{center}
\caption{}
\label{fig:chapters/10/7/2/1/Fig}
\end{figure}


\item Find the coordinates of the points of trisection of the line segment joining $(4,-1) \text{ and } (-2,3)$.
	\\
		\solution
	\begin{enumerate}[label=\thesection.\arabic*,ref=\thesection.\theenumi]
\numberwithin{equation}{enumi}
\numberwithin{figure}{enumi}
\numberwithin{table}{enumi}

\item Find the coordinates of the point which divides the join of $(-1,7) \text{ and } (4,-3)$ in the ratio 2:3.
	\\
		\solution
	\input{chapters/10/7/2/1/section.tex}
\item Find the coordinates of the points of trisection of the line segment joining $(4,-1) \text{ and } (-2,3)$.
	\\
		\solution
	\input{chapters/10/7/2/2/section.tex}
\item
	\iffalse
\item To conduct Sports Day activities, in your rectangular shaped school                   
ground ABCD, lines have 
drawn with chalk powder at a                 
distance of 1m each. 100 flower pots have been placed at a distance of 1m 
from each other along AD, as shown 
in Fig. 7.12. Niharika runs $ \frac {1}{4} $th the 
distance AD on the 2nd line and 
posts a green flag. Preet runs $ \frac {1}{5} $th 
the distance AD on the eighth line 
and posts a red flag. What is the 
distance between both the flags? If 
Rashmi has to post a blue flag exactly 
halfway between the line segment 
joining the two flags, where should 
she post her flag?
\begin{figure}[h!]
  \centering
  \includegraphics[width=\columnwidth]{sc.png}
  \caption{}
\label{fig:10/7/12Fig1}
\end{figure}               
\fi
      
\item Find the ratio in which the line segment joining the points $(-3,10) \text{ and } (6,-8)$ $\text{ is divided by } (-1,6)$.
	\\
		\solution
	\input{chapters/10/7/2/4/section.tex}
\item Find the ratio in which the line segment joining $A(1,-5) \text{ and } B(-4,5)$ $\text{is divided by the x-axis}$. Also find the coordinates of the point of division.
\item If $(1,2), (4,y), (x,6), (3,5)$ are the vertices of a parallelogram taken in order, find x and y.
	\\
		\solution
	\input{chapters/10/7/2/6/para1.tex}
\item Find the coordinates of a point A, where AB is the diameter of a circle whose centre is $(2,-3) \text{ and }$ B is $(1,4)$.
	\\
		\solution
	\input{chapters/10/7/2/7/section.tex}
\item If A \text{ and } B are $(-2,-2) \text{ and } (2,-4)$, respectively, find the coordinates of P such that AP= $\frac {3}{7}$AB $\text{ and }$ P lies on the line segment AB.
	\\
		\solution
	\input{chapters/10/7/2/8/section.tex}
\item Find the coordinates of the points which divide the line segment joining $A(-2,2) \text{ and } B(2,8)$ into four equal parts.
	\\
		\solution
	\input{chapters/10/7/2/9/section.tex}
\item Find the area of a rhombus if its vertices are $(3,0), (4,5), (-1,4) \text{ and } (-2,-1)$ taken in order. [$\vec{Hint}$ : Area of rhombus =$\frac {1}{2}$(product of its diagonals)]
	\\
		\solution
	\input{chapters/10/7/2/10/cross.tex}
\item Find the position vector of a point R which divides the line joining two points $\vec{P}$
and $\vec{Q}$ whose position vectors are $\hat{i}+2\hat{j}-\hat{k}$ and $-\hat{i}+\hat{j}+\hat{k}$ respectively, in the
ratio 2 : 1
\begin{enumerate}
    \item  internally
    \item  externally
\end{enumerate}
\solution
		\input{chapters/12/10/2/15/section.tex}
\item Find the position vector of the mid point of the vector joining the points $\vec{P}$(2, 3, 4)
and $\vec{Q}$(4, 1, –2).
\\
\solution
		\input{chapters/12/10/2/16/section.tex}
\item Determine the ratio in which the line $2x+y  - 4=0$ divides the line segment joining the points $\vec{A}(2, - 2)$  and  $\vec{B}(3, 7)$.
\\
\solution
	\input{chapters/10/7/4/1/section.tex}
\item Let $\vec{A}(4, 2), \vec{B}(6, 5)$  and $ \vec{C}(1, 4)$ be the vertices of $\triangle ABC$.
\begin{enumerate}
\item The median from $\vec{A}$ meets $BC$ at $\vec{D}$. Find the coordinates of the point $\vec{D}$.
\item Find the coordinates of the point $\vec{P}$ on $AD$ such that $AP : PD = 2 : 1$.
\item Find the coordinates of points $\vec{Q}$ and $\vec{R}$ on medians $BE$ and $CF$ respectively such that $BQ : QE = 2 : 1$  and  $CR : RF = 2 : 1$.
\item What do you observe?
\item If $\vec{A}, \vec{B}$ and $\vec{C}$  are the vertices of $\triangle ABC$, find the coordinates of the centroid of the triangle.
\end{enumerate}
\solution
	\input{chapters/10/7/4/7/section.tex}
\item Find the slope of a line, which passes through the origin and the mid point of the line segment joining the points $\vec{P}$(0,-4) and $\vec{B}$(8,0).
\label{chapters/11/10/1/5}
\input{chapters/11/10/1/5/matrix.tex}
\item Find the position vector of a point R which divides the line joining two points P and Q whose position vectors are $(2\vec{a}+\vec{b})$ and $(\vec{a}-3\vec{b})$
externally in the ratio 1 : 2. Also, show that P is the mid point of the line segment RQ.\\
	\solution
%		\input{chapters/12/10/5/9/section.tex}

\end{enumerate}


\item
	\iffalse
\item To conduct Sports Day activities, in your rectangular shaped school                   
ground ABCD, lines have 
drawn with chalk powder at a                 
distance of 1m each. 100 flower pots have been placed at a distance of 1m 
from each other along AD, as shown 
in Fig. 7.12. Niharika runs $ \frac {1}{4} $th the 
distance AD on the 2nd line and 
posts a green flag. Preet runs $ \frac {1}{5} $th 
the distance AD on the eighth line 
and posts a red flag. What is the 
distance between both the flags? If 
Rashmi has to post a blue flag exactly 
halfway between the line segment 
joining the two flags, where should 
she post her flag?
\begin{figure}[h!]
  \centering
  \includegraphics[width=\columnwidth]{sc.png}
  \caption{}
\label{fig:10/7/12Fig1}
\end{figure}               
\fi
      
\item Find the ratio in which the line segment joining the points $(-3,10) \text{ and } (6,-8)$ $\text{ is divided by } (-1,6)$.
	\\
		\solution
	\iffalse
\documentclass[12pt]{article}
\usepackage{graphicx}
%\documentclass[journal,12pt,twocolumn]{IEEEtran}
\usepackage[none]{hyphenat}
\usepackage{graphicx}
\usepackage{listings}
\usepackage[english]{babel}
\usepackage{graphicx}
\usepackage{caption} 
\usepackage{hyperref}
\usepackage{booktabs}
\def\inputGnumericTable{}
\usepackage{color}                                            %%
    \usepackage{array}                                            %%
    \usepackage{longtable}                                        %%
    \usepackage{calc}                                             %%
    \usepackage{multirow}                                         %%
    \usepackage{hhline}                                           %%
    \usepackage{ifthen}
\usepackage{array}
\usepackage{amsmath}   % for having text in math mode
\usepackage{listings}
\lstset{
language=tex,
frame=single, 
breaklines=true
}
  
%Following 2 lines were added to remove the blank page at the beginning
\usepackage{atbegshi}% http://ctan.org/pkg/atbegshi
\AtBeginDocument{\AtBeginShipoutNext{\AtBeginShipoutDiscard}}
%
%New macro definitions
\newcommand{\mydet}[1]{\ensuremath{\begin{vmatrix}#1\end{vmatrix}}}
\providecommand{\brak}[1]{\ensuremath{\left(#1\right)}}
\providecommand{\norm}[1]{\left\lVert#1\right\rVert}
\newcommand{\solution}{\noindent \textbf{Solution: }}
\newcommand{\myvec}[1]{\ensuremath{\begin{pmatrix}#1\end{pmatrix}}}
\let\vec\mathbf
\begin{document}
\begin{center}
\title{\textbf{Coordinate Geometry}}
\date{\vspace{-5ex}} %Not to print date automatically
\maketitle
\end{center}
\setcounter{page}{1}
\section*{10$^{th}$ Maths - Chapter 7}
This is Problem-4 from Exercise 7.2
\begin{enumerate}
\item Find the ratio in which the line segement joining the points $\myvec{-3 \\ 10}$ and $\myvec{6\\-8}$ is divided by $\myvec{-1\\6}$.\\
\solution \\
\fi
		The input parameters for this problem are available in Table \eqref{tab:10/7/2/4-1}.
\begin{table}[ht!]
\input{chapters/10/7/2/4/tables/table.tex}
\caption{}
\label{tab:10/7/2/4-1} 
\end{table}
Using section formula,
\begin{align}
         \vec{R} &=\frac{\vec{Q}+n\vec{P}}{1+n}\label{eq:chapters/10/7/2/4/1}
\end{align}
Substituting the values of $\vec{P},\vec{Q}$ and $\vec{R}$ in \eqref{eq:chapters/10/7/2/4/1}
\begin{align}
         \myvec{-1\\6} &=\frac{{\myvec{-3\\10}+n\myvec{6\\-8}}}{1+n}\\
 &=\frac{1}{1+n}\brak{{\myvec{-3\\10}+n\myvec{6\\-8}}} \\
 &=\frac{1}{1+n}\myvec{-3+6n\\10-8n} \label{eq:chapters/10/7/2/4/4}
\end{align}
Simplifying \eqref{eq:chapters/10/7/2/4/4} yeilds,
\begin{align}
          -1 &=\frac{-3+6n}{1+n}\\
\implies          n &=\frac{2}{7}
\end{align}
Also,
\begin{align}
          6 &=\frac{10-8n}{1+n}\\
    \implies      n &=\frac{2}{7}
\end{align}
Hence the desired ratio is $\dfrac{2}{7}$.  
\begin{figure}[!h]
 \begin{center}
  \includegraphics[width=\columnwidth]{chapters/10/7/2/4/figs/fig.png}
 \end{center}
\caption{}
\label{fig:10/7/2/4Fig1}
\end{figure}

\item Find the ratio in which the line segment joining $A(1,-5) \text{ and } B(-4,5)$ $\text{is divided by the x-axis}$. Also find the coordinates of the point of division.
\item If $(1,2), (4,y), (x,6), (3,5)$ are the vertices of a parallelogram taken in order, find x and y.
	\\
		\solution
	\iffalse
\documentclass[12pt]{article}
\usepackage{graphicx}
%\documentclass[journal,12pt,twocolumn]{IEEEtran}
\def\inputGnumericTable{}
\usepackage{color}                                            %%
    \usepackage{array}                                            %%
    \usepackage{longtable}                                        %%
    \usepackage{calc}                                             %%
    \usepackage{multirow}                                         %%
    \usepackage{hhline}                                           %%
    \usepackage{ifthen}
\usepackage[none]{hyphenat}
\usepackage{graphicx}
\usepackage{listings}
\usepackage[english]{babel}
\usepackage{graphicx}
\usepackage{caption} 
\usepackage{hyperref}
\usepackage{booktabs}
\usepackage{array}
\usepackage{amsmath}   % for having text in math mode
\usepackage{listings}
\lstset{
  frame=single,
  breaklines=true
}
  
%Following 2 lines were added to remove the blank page at the beginning
\usepackage{atbegshi}% http://ctan.org/pkg/atbegshi
\AtBeginDocument{\AtBeginShipoutNext{\AtBeginShipoutDiscard}}
%


%New macro definitions
\newcommand{\mydet}[1]{\ensuremath{\begin{vmatrix}#1\end{vmatrix}}}
\providecommand{\brak}[1]{\ensuremath{\left(#1\right)}}
\providecommand{\norm}[1]{\left\lVert#1\right\rVert}
\newcommand{\solution}{\noindent \textbf{Solution: }}
\newcommand{\myvec}[1]{\ensuremath{\begin{pmatrix}#1\end{pmatrix}}}
\let\vec\mathbf

\begin{document}

\begin{center}
\title{\textbf{Properties of Parallelegram}}
\date{\vspace{-5ex}} %Not to print date automatically
\maketitle
\end{center}

\setcounter{page}{1}

\section{10$^{th}$ Maths - Chapter 7}

This is Problem-6 from Exercise 7.2

\begin{enumerate}
\item If $\vec{A}(1, 2),\vec{B}(4, x),\vec{C}(y, 6) \text{and } \vec{D}(3, 5)$ are the vertices of a parallelogram taken in order,find x and y.
\end{enumerate}
\fi

The input parameters for this problem are available in
\ref{table:chapters/10/7/2/6/tables/}.	
\begin{table}[!ht]
	\centering
	\input{chapters/10/7/2/6/tables/table.tex}
\caption{}
\label{table:chapters/10/7/2/6/tables/}	
\end{table}
From the given information,
\begin{align}
  \label{eq:chapters/10/7/2/6/tables/det2f}
	\vec{B}-\vec{A} &= \myvec{4 \\y } - \myvec{1 \\2 }  = \myvec{3 \\y-2 }\\
	\vec{C}-\vec{D} &= \myvec{x \\6 } - \myvec{3 \\5 }  = \myvec{x-3 \\1}
\end{align}
Since $ABCD$ is a parallellogram,
\begin{align}
	\myvec{3\\y-2}&=\myvec{x-3\\1}\\
	\implies x&=6 ,y=3
\end{align}
Fig. \ref{fig:chapters/10/7/2/6/Fig3}
provides a verification.
\begin{figure}[h!]
	\begin{center}
  \includegraphics[width=\columnwidth]{chapters/10/7/2/6/figs/para.pdf}
	\end{center}
\caption{}
\label{fig:chapters/10/7/2/6/Fig3}
\end{figure}


\item Find the coordinates of a point A, where AB is the diameter of a circle whose centre is $(2,-3) \text{ and }$ B is $(1,4)$.
	\\
		\solution
	\iffalse
\documentclass[12pt]{article}
\usepackage{graphicx}
\usepackage{amsmath}
\usepackage{mathtools}
\usepackage{gensymb}

\newcommand{\mydet}[1]{\ensuremath{\begin{vmatrix}#1\end{vmatrix}}}
\providecommand{\brak}[1]{\ensuremath{\left(#1\right)}}
\providecommand{\norm}[1]{\left\lVert#1\right\rVert}
\newcommand{\solution}{\noindent \textbf{Solution: }}
\newcommand{\myvec}[1]{\ensuremath{\begin{pmatrix}#1\end{pmatrix}}}
\let\vec\mathbf

\begin{document}
\begin{center}
\section*{CHAPTER 7 - COORDINATE GEOMETRY}

\end{center}
\section*{Excercise 7.2}

Q7.Find the coordinates of point $\vec{A}$, where AB is the diameter of a circle where the center is (2,-3) and $\vec{B}$ is the point (1,4):

\solution
\begin{enumerate}
\item The coordinates $\vec{B}$ and center $\vec{C}$ are given, where:
	\fi
	Let
	\begin{align}
	\vec{B} = \myvec{
		1\\
	    4\\
		},
	\vec{C} = \myvec{
	    2\\
	   -3\\
		}
	\end{align}
	\iffalse
Let us assume the coordinates of $\vec{A}$. Now, $\vec{C}$ is the center which is midpoint of line AB and $\vec{B}$ is one of the coordinate of diameter AB of a circle.
	\fi	
Hence,	
	\begin{align}
	\vec{C} &= \frac{\vec{A+B}}{2} \\
\implies	2\vec{C} &= \vec{A}+\vec{B} \\
		\text{or, }	\vec{A} &= 2\vec{C}-\vec{B} \\
	 &= \myvec{3\\-10\\}	
	\end{align}       
	See Fig. 
\ref{fig:chapters/10/7/2/7Fig}.
\begin{figure}[!h]
\begin{center}	
	\includegraphics[width=\columnwidth]{chapters/10/7/2/7/figs/Vector1.png}
\end{center}
\caption{}
\label{fig:chapters/10/7/2/7Fig}
\end{figure}
	

\item If A \text{ and } B are $(-2,-2) \text{ and } (2,-4)$, respectively, find the coordinates of P such that AP= $\frac {3}{7}$AB $\text{ and }$ P lies on the line segment AB.
	\\
		\solution
	\iffalse
\documentclass[journal,10pt,twocolumn]{article}
\usepackage{graphicx}
\usepackage[none]{hyphenat}
\usepackage{graphicx}
\usepackage{listings}
\usepackage[english]{babel}
\usepackage{graphicx}
\usepackage{caption} 
\usepackage{booktabs}
\usepackage{array}
\usepackage{amssymb} % for \because
\usepackage{amsmath}   % for having text in math mode
\usepackage{extarrows} % for Row operations arrows
\usepackage{listings}
\usepackage[utf8]{inputenc}
\lstset{
  frame=single,
  breaklines=true
}
\usepackage{hyperref}
  
%Following 2 lines were added to remove the blank page at the beginning
\usepackage{atbegshi}% http://ctan.org/pkg/atbegshi
\AtBeginDocument{\AtBeginShipoutNext{\AtBeginShipoutDiscard}}


%New macro definitions
\newcommand{\mydet}[1]{\ensuremath{\begin{vmatrix}#1\end{vmatrix}}}
\providecommand{\brak}[1]{\ensuremath{\left(#1\right)}}
\newcommand{\solution}{\noindent \textbf{Solution: }}
\newcommand{\myvec}[1]{\ensuremath{\begin{pmatrix}#1\end{pmatrix}}}
\providecommand{\norm}[1]{\left\lVert#1\right\rVert}
\providecommand{\abs}[1]{\left\vert#1\right\vert}
\let\vec\mathbf

\begin{document}

\begin{center}
\title{\textbf{VECTORS}}
\date{\vspace{-5ex}} %Not to print date automatically
\maketitle
\end{center}

\section{10$^{th}$ Maths - EXERCISE-7.2}

\begin{enumerate}
\item If A and B are $(– 2, – 2)\text{ and }(2, – 4)$, respectively, find the coordinates of P such that $AP =\frac{3}{7}AB$ and P lies on the line segment AB. 

\section{SOLUTION}
Given points are
\begin{align}
\vec{A}=\myvec{-2\\ -2} ,
\vec{B}=\myvec{2\\ -4}
\end{align}
The equation of the formula is
\fi
Using section formula, 
\begin{align}
\vec{P}&=\frac{\vec{A}+n\vec{B}}{1+n}
\end{align}
where
\begin{align}
	n =\frac{3}{4}
\end{align}
Thus,
\begin{align}
\vec{P}&=\frac{1}{1+\frac{3}{4}}\brak{\myvec{-2\\-2}+\frac{3}{4}\myvec{2\\-4}}\\
&=\myvec{\frac{-2}{7}\\[1pt] \frac{-20}{7}}
\end{align}
See Fig. 
   \ref{fig:chapters/10/7/2/8/vec.png}
\begin{figure}
   \centering 
 \includegraphics[width=\columnwidth]{chapters/10/7/2/8/figs/vec.png}
   \caption{}
   \label{fig:chapters/10/7/2/8/vec.png}
   \end{figure}

\item Find the coordinates of the points which divide the line segment joining $A(-2,2) \text{ and } B(2,8)$ into four equal parts.
	\\
		\solution
	\begin{enumerate}[label=\thesection.\arabic*,ref=\thesection.\theenumi]
\numberwithin{equation}{enumi}
\numberwithin{figure}{enumi}
\numberwithin{table}{enumi}

\item Find the coordinates of the point which divides the join of $(-1,7) \text{ and } (4,-3)$ in the ratio 2:3.
	\\
		\solution
	\input{chapters/10/7/2/1/section.tex}
\item Find the coordinates of the points of trisection of the line segment joining $(4,-1) \text{ and } (-2,3)$.
	\\
		\solution
	\input{chapters/10/7/2/2/section.tex}
\item
	\iffalse
\item To conduct Sports Day activities, in your rectangular shaped school                   
ground ABCD, lines have 
drawn with chalk powder at a                 
distance of 1m each. 100 flower pots have been placed at a distance of 1m 
from each other along AD, as shown 
in Fig. 7.12. Niharika runs $ \frac {1}{4} $th the 
distance AD on the 2nd line and 
posts a green flag. Preet runs $ \frac {1}{5} $th 
the distance AD on the eighth line 
and posts a red flag. What is the 
distance between both the flags? If 
Rashmi has to post a blue flag exactly 
halfway between the line segment 
joining the two flags, where should 
she post her flag?
\begin{figure}[h!]
  \centering
  \includegraphics[width=\columnwidth]{sc.png}
  \caption{}
\label{fig:10/7/12Fig1}
\end{figure}               
\fi
      
\item Find the ratio in which the line segment joining the points $(-3,10) \text{ and } (6,-8)$ $\text{ is divided by } (-1,6)$.
	\\
		\solution
	\input{chapters/10/7/2/4/section.tex}
\item Find the ratio in which the line segment joining $A(1,-5) \text{ and } B(-4,5)$ $\text{is divided by the x-axis}$. Also find the coordinates of the point of division.
\item If $(1,2), (4,y), (x,6), (3,5)$ are the vertices of a parallelogram taken in order, find x and y.
	\\
		\solution
	\input{chapters/10/7/2/6/para1.tex}
\item Find the coordinates of a point A, where AB is the diameter of a circle whose centre is $(2,-3) \text{ and }$ B is $(1,4)$.
	\\
		\solution
	\input{chapters/10/7/2/7/section.tex}
\item If A \text{ and } B are $(-2,-2) \text{ and } (2,-4)$, respectively, find the coordinates of P such that AP= $\frac {3}{7}$AB $\text{ and }$ P lies on the line segment AB.
	\\
		\solution
	\input{chapters/10/7/2/8/section.tex}
\item Find the coordinates of the points which divide the line segment joining $A(-2,2) \text{ and } B(2,8)$ into four equal parts.
	\\
		\solution
	\input{chapters/10/7/2/9/section.tex}
\item Find the area of a rhombus if its vertices are $(3,0), (4,5), (-1,4) \text{ and } (-2,-1)$ taken in order. [$\vec{Hint}$ : Area of rhombus =$\frac {1}{2}$(product of its diagonals)]
	\\
		\solution
	\input{chapters/10/7/2/10/cross.tex}
\item Find the position vector of a point R which divides the line joining two points $\vec{P}$
and $\vec{Q}$ whose position vectors are $\hat{i}+2\hat{j}-\hat{k}$ and $-\hat{i}+\hat{j}+\hat{k}$ respectively, in the
ratio 2 : 1
\begin{enumerate}
    \item  internally
    \item  externally
\end{enumerate}
\solution
		\input{chapters/12/10/2/15/section.tex}
\item Find the position vector of the mid point of the vector joining the points $\vec{P}$(2, 3, 4)
and $\vec{Q}$(4, 1, –2).
\\
\solution
		\input{chapters/12/10/2/16/section.tex}
\item Determine the ratio in which the line $2x+y  - 4=0$ divides the line segment joining the points $\vec{A}(2, - 2)$  and  $\vec{B}(3, 7)$.
\\
\solution
	\input{chapters/10/7/4/1/section.tex}
\item Let $\vec{A}(4, 2), \vec{B}(6, 5)$  and $ \vec{C}(1, 4)$ be the vertices of $\triangle ABC$.
\begin{enumerate}
\item The median from $\vec{A}$ meets $BC$ at $\vec{D}$. Find the coordinates of the point $\vec{D}$.
\item Find the coordinates of the point $\vec{P}$ on $AD$ such that $AP : PD = 2 : 1$.
\item Find the coordinates of points $\vec{Q}$ and $\vec{R}$ on medians $BE$ and $CF$ respectively such that $BQ : QE = 2 : 1$  and  $CR : RF = 2 : 1$.
\item What do you observe?
\item If $\vec{A}, \vec{B}$ and $\vec{C}$  are the vertices of $\triangle ABC$, find the coordinates of the centroid of the triangle.
\end{enumerate}
\solution
	\input{chapters/10/7/4/7/section.tex}
\item Find the slope of a line, which passes through the origin and the mid point of the line segment joining the points $\vec{P}$(0,-4) and $\vec{B}$(8,0).
\label{chapters/11/10/1/5}
\input{chapters/11/10/1/5/matrix.tex}
\item Find the position vector of a point R which divides the line joining two points P and Q whose position vectors are $(2\vec{a}+\vec{b})$ and $(\vec{a}-3\vec{b})$
externally in the ratio 1 : 2. Also, show that P is the mid point of the line segment RQ.\\
	\solution
%		\input{chapters/12/10/5/9/section.tex}

\end{enumerate}


\item Find the area of a rhombus if its vertices are $(3,0), (4,5), (-1,4) \text{ and } (-2,-1)$ taken in order. [$\vec{Hint}$ : Area of rhombus =$\frac {1}{2}$(product of its diagonals)]
	\\
		\solution
	\iffalse
\documentclass[12pt]{article}
\usepackage{graphicx}
%\documentclass[journal,12pt,twocolumn]{IEEEtran}
\usepackage[none]{hyphenat}
\usepackage{graphicx}
\usepackage{listings}
\usepackage[english]{babel}
\usepackage{graphicx}
\usepackage{caption} 
\usepackage{hyperref}
\usepackage{booktabs}
\def\inputGnumericTable{}
\usepackage{color}                                            %%
    \usepackage{array}                                            %%
    \usepackage{longtable}                                        %%
    \usepackage{calc}                                             %%
    \usepackage{multirow}                                         %%
    \usepackage{hhline}                                           %%
    \usepackage{ifthen}
\usepackage{array}
\usepackage{amsmath}   % for having text in math mode
\usepackage{listings}
\lstset{
language=tex,
frame=single, 
breaklines=true
}
  
%Following 2 lines were added to remove the blank page at the beginning
\usepackage{atbegshi}% http://ctan.org/pkg/atbegshi
\AtBeginDocument{\AtBeginShipoutNext{\AtBeginShipoutDiscard}}
%


%New macro definitions
\newcommand{\mydet}[1]{\ensuremath{\begin{vmatrix}#1\end{vmatrix}}}
\providecommand{\brak}[1]{\ensuremath{\left(#1\right)}}
\providecommand{\norm}[1]{\left\lVert#1\right\rVert}
\newcommand{\solution}{\noindent \textbf{Solution: }}
\newcommand{\myvec}[1]{\ensuremath{\begin{pmatrix}#1\end{pmatrix}}}
\let\vec\mathbf

\begin{document}

\begin{center}
\title{\textbf{Coordinate Geometry}}
\date{\vspace{-5ex}} %Not to print date automatically
\maketitle
\end{center}

\setcounter{page}{1}



\begin{enumerate}

\item\textbf{Problem statement :} Find the area of a rhombus of its vertices are $\myvec{3 ,0}$, $\myvec{4 ,5}$, $\myvec{-1 ,4}$ and $\myvec{-2 ,-1}$taken in order

\solution \\
\fi
The input vertices for this problem are given as
	\begin{align}
	\vec{A} = \myvec{
		3\\
		0
		},
	\vec{B} = \myvec{
		4\\
		5
		},
        \vec{C} = \myvec{
		-1\\
		4
		},
        \vec{D} = \myvec{
		-2\\
		-1
		}
	\end{align}
Since		
\begin{align}
 \vec{A-D}= \myvec{3 \\ 0} - \myvec{-2 \\-1}= \myvec{5\\1}
 \\
  \vec{B-A}= \myvec{4 \\ 5} - \myvec{3 \\0}= \myvec{1\\5}
\end{align}
the area of the rhombus is
\begin{align}
                \norm{\myvec{\vec{A-D}}\times \myvec{\vec{B-A}}}=\mydet{5 & 1\\1 & 5} = 24
\end{align}
See Fig. 
\ref{fig:chapters/10/7/2/10/gFig1}.
\begin{figure}[!h]
 \begin{center}
  \includegraphics[width=\columnwidth]{chapters/10/7/2/10/figs/fig.pdf}
 \end{center}
\caption{}
\label{fig:chapters/10/7/2/10/gFig1}
\end{figure}

\item Find the position vector of a point R which divides the line joining two points $\vec{P}$
and $\vec{Q}$ whose position vectors are $\hat{i}+2\hat{j}-\hat{k}$ and $-\hat{i}+\hat{j}+\hat{k}$ respectively, in the
ratio 2 : 1
\begin{enumerate}
    \item  internally
    \item  externally
\end{enumerate}
\solution
		\begin{enumerate}[label=\thesection.\arabic*,ref=\thesection.\theenumi]
\numberwithin{equation}{enumi}
\numberwithin{figure}{enumi}
\numberwithin{table}{enumi}

\item Find the coordinates of the point which divides the join of $(-1,7) \text{ and } (4,-3)$ in the ratio 2:3.
	\\
		\solution
	\input{chapters/10/7/2/1/section.tex}
\item Find the coordinates of the points of trisection of the line segment joining $(4,-1) \text{ and } (-2,3)$.
	\\
		\solution
	\input{chapters/10/7/2/2/section.tex}
\item
	\iffalse
\item To conduct Sports Day activities, in your rectangular shaped school                   
ground ABCD, lines have 
drawn with chalk powder at a                 
distance of 1m each. 100 flower pots have been placed at a distance of 1m 
from each other along AD, as shown 
in Fig. 7.12. Niharika runs $ \frac {1}{4} $th the 
distance AD on the 2nd line and 
posts a green flag. Preet runs $ \frac {1}{5} $th 
the distance AD on the eighth line 
and posts a red flag. What is the 
distance between both the flags? If 
Rashmi has to post a blue flag exactly 
halfway between the line segment 
joining the two flags, where should 
she post her flag?
\begin{figure}[h!]
  \centering
  \includegraphics[width=\columnwidth]{sc.png}
  \caption{}
\label{fig:10/7/12Fig1}
\end{figure}               
\fi
      
\item Find the ratio in which the line segment joining the points $(-3,10) \text{ and } (6,-8)$ $\text{ is divided by } (-1,6)$.
	\\
		\solution
	\input{chapters/10/7/2/4/section.tex}
\item Find the ratio in which the line segment joining $A(1,-5) \text{ and } B(-4,5)$ $\text{is divided by the x-axis}$. Also find the coordinates of the point of division.
\item If $(1,2), (4,y), (x,6), (3,5)$ are the vertices of a parallelogram taken in order, find x and y.
	\\
		\solution
	\input{chapters/10/7/2/6/para1.tex}
\item Find the coordinates of a point A, where AB is the diameter of a circle whose centre is $(2,-3) \text{ and }$ B is $(1,4)$.
	\\
		\solution
	\input{chapters/10/7/2/7/section.tex}
\item If A \text{ and } B are $(-2,-2) \text{ and } (2,-4)$, respectively, find the coordinates of P such that AP= $\frac {3}{7}$AB $\text{ and }$ P lies on the line segment AB.
	\\
		\solution
	\input{chapters/10/7/2/8/section.tex}
\item Find the coordinates of the points which divide the line segment joining $A(-2,2) \text{ and } B(2,8)$ into four equal parts.
	\\
		\solution
	\input{chapters/10/7/2/9/section.tex}
\item Find the area of a rhombus if its vertices are $(3,0), (4,5), (-1,4) \text{ and } (-2,-1)$ taken in order. [$\vec{Hint}$ : Area of rhombus =$\frac {1}{2}$(product of its diagonals)]
	\\
		\solution
	\input{chapters/10/7/2/10/cross.tex}
\item Find the position vector of a point R which divides the line joining two points $\vec{P}$
and $\vec{Q}$ whose position vectors are $\hat{i}+2\hat{j}-\hat{k}$ and $-\hat{i}+\hat{j}+\hat{k}$ respectively, in the
ratio 2 : 1
\begin{enumerate}
    \item  internally
    \item  externally
\end{enumerate}
\solution
		\input{chapters/12/10/2/15/section.tex}
\item Find the position vector of the mid point of the vector joining the points $\vec{P}$(2, 3, 4)
and $\vec{Q}$(4, 1, –2).
\\
\solution
		\input{chapters/12/10/2/16/section.tex}
\item Determine the ratio in which the line $2x+y  - 4=0$ divides the line segment joining the points $\vec{A}(2, - 2)$  and  $\vec{B}(3, 7)$.
\\
\solution
	\input{chapters/10/7/4/1/section.tex}
\item Let $\vec{A}(4, 2), \vec{B}(6, 5)$  and $ \vec{C}(1, 4)$ be the vertices of $\triangle ABC$.
\begin{enumerate}
\item The median from $\vec{A}$ meets $BC$ at $\vec{D}$. Find the coordinates of the point $\vec{D}$.
\item Find the coordinates of the point $\vec{P}$ on $AD$ such that $AP : PD = 2 : 1$.
\item Find the coordinates of points $\vec{Q}$ and $\vec{R}$ on medians $BE$ and $CF$ respectively such that $BQ : QE = 2 : 1$  and  $CR : RF = 2 : 1$.
\item What do you observe?
\item If $\vec{A}, \vec{B}$ and $\vec{C}$  are the vertices of $\triangle ABC$, find the coordinates of the centroid of the triangle.
\end{enumerate}
\solution
	\input{chapters/10/7/4/7/section.tex}
\item Find the slope of a line, which passes through the origin and the mid point of the line segment joining the points $\vec{P}$(0,-4) and $\vec{B}$(8,0).
\label{chapters/11/10/1/5}
\input{chapters/11/10/1/5/matrix.tex}
\item Find the position vector of a point R which divides the line joining two points P and Q whose position vectors are $(2\vec{a}+\vec{b})$ and $(\vec{a}-3\vec{b})$
externally in the ratio 1 : 2. Also, show that P is the mid point of the line segment RQ.\\
	\solution
%		\input{chapters/12/10/5/9/section.tex}

\end{enumerate}


\item Find the position vector of the mid point of the vector joining the points $\vec{P}$(2, 3, 4)
and $\vec{Q}$(4, 1, –2).
\\
\solution
		\begin{enumerate}[label=\thesection.\arabic*,ref=\thesection.\theenumi]
\numberwithin{equation}{enumi}
\numberwithin{figure}{enumi}
\numberwithin{table}{enumi}

\item Find the coordinates of the point which divides the join of $(-1,7) \text{ and } (4,-3)$ in the ratio 2:3.
	\\
		\solution
	\input{chapters/10/7/2/1/section.tex}
\item Find the coordinates of the points of trisection of the line segment joining $(4,-1) \text{ and } (-2,3)$.
	\\
		\solution
	\input{chapters/10/7/2/2/section.tex}
\item
	\iffalse
\item To conduct Sports Day activities, in your rectangular shaped school                   
ground ABCD, lines have 
drawn with chalk powder at a                 
distance of 1m each. 100 flower pots have been placed at a distance of 1m 
from each other along AD, as shown 
in Fig. 7.12. Niharika runs $ \frac {1}{4} $th the 
distance AD on the 2nd line and 
posts a green flag. Preet runs $ \frac {1}{5} $th 
the distance AD on the eighth line 
and posts a red flag. What is the 
distance between both the flags? If 
Rashmi has to post a blue flag exactly 
halfway between the line segment 
joining the two flags, where should 
she post her flag?
\begin{figure}[h!]
  \centering
  \includegraphics[width=\columnwidth]{sc.png}
  \caption{}
\label{fig:10/7/12Fig1}
\end{figure}               
\fi
      
\item Find the ratio in which the line segment joining the points $(-3,10) \text{ and } (6,-8)$ $\text{ is divided by } (-1,6)$.
	\\
		\solution
	\input{chapters/10/7/2/4/section.tex}
\item Find the ratio in which the line segment joining $A(1,-5) \text{ and } B(-4,5)$ $\text{is divided by the x-axis}$. Also find the coordinates of the point of division.
\item If $(1,2), (4,y), (x,6), (3,5)$ are the vertices of a parallelogram taken in order, find x and y.
	\\
		\solution
	\input{chapters/10/7/2/6/para1.tex}
\item Find the coordinates of a point A, where AB is the diameter of a circle whose centre is $(2,-3) \text{ and }$ B is $(1,4)$.
	\\
		\solution
	\input{chapters/10/7/2/7/section.tex}
\item If A \text{ and } B are $(-2,-2) \text{ and } (2,-4)$, respectively, find the coordinates of P such that AP= $\frac {3}{7}$AB $\text{ and }$ P lies on the line segment AB.
	\\
		\solution
	\input{chapters/10/7/2/8/section.tex}
\item Find the coordinates of the points which divide the line segment joining $A(-2,2) \text{ and } B(2,8)$ into four equal parts.
	\\
		\solution
	\input{chapters/10/7/2/9/section.tex}
\item Find the area of a rhombus if its vertices are $(3,0), (4,5), (-1,4) \text{ and } (-2,-1)$ taken in order. [$\vec{Hint}$ : Area of rhombus =$\frac {1}{2}$(product of its diagonals)]
	\\
		\solution
	\input{chapters/10/7/2/10/cross.tex}
\item Find the position vector of a point R which divides the line joining two points $\vec{P}$
and $\vec{Q}$ whose position vectors are $\hat{i}+2\hat{j}-\hat{k}$ and $-\hat{i}+\hat{j}+\hat{k}$ respectively, in the
ratio 2 : 1
\begin{enumerate}
    \item  internally
    \item  externally
\end{enumerate}
\solution
		\input{chapters/12/10/2/15/section.tex}
\item Find the position vector of the mid point of the vector joining the points $\vec{P}$(2, 3, 4)
and $\vec{Q}$(4, 1, –2).
\\
\solution
		\input{chapters/12/10/2/16/section.tex}
\item Determine the ratio in which the line $2x+y  - 4=0$ divides the line segment joining the points $\vec{A}(2, - 2)$  and  $\vec{B}(3, 7)$.
\\
\solution
	\input{chapters/10/7/4/1/section.tex}
\item Let $\vec{A}(4, 2), \vec{B}(6, 5)$  and $ \vec{C}(1, 4)$ be the vertices of $\triangle ABC$.
\begin{enumerate}
\item The median from $\vec{A}$ meets $BC$ at $\vec{D}$. Find the coordinates of the point $\vec{D}$.
\item Find the coordinates of the point $\vec{P}$ on $AD$ such that $AP : PD = 2 : 1$.
\item Find the coordinates of points $\vec{Q}$ and $\vec{R}$ on medians $BE$ and $CF$ respectively such that $BQ : QE = 2 : 1$  and  $CR : RF = 2 : 1$.
\item What do you observe?
\item If $\vec{A}, \vec{B}$ and $\vec{C}$  are the vertices of $\triangle ABC$, find the coordinates of the centroid of the triangle.
\end{enumerate}
\solution
	\input{chapters/10/7/4/7/section.tex}
\item Find the slope of a line, which passes through the origin and the mid point of the line segment joining the points $\vec{P}$(0,-4) and $\vec{B}$(8,0).
\label{chapters/11/10/1/5}
\input{chapters/11/10/1/5/matrix.tex}
\item Find the position vector of a point R which divides the line joining two points P and Q whose position vectors are $(2\vec{a}+\vec{b})$ and $(\vec{a}-3\vec{b})$
externally in the ratio 1 : 2. Also, show that P is the mid point of the line segment RQ.\\
	\solution
%		\input{chapters/12/10/5/9/section.tex}

\end{enumerate}


\item Determine the ratio in which the line $2x+y  - 4=0$ divides the line segment joining the points $\vec{A}(2, - 2)$  and  $\vec{B}(3, 7)$.
\\
\solution
	\iffalse
\documentclass[journal,12pt,twocolumn]{IEEEtran}
\usepackage{graphicx}
\graphicspath{{./chapters/10/7/4/1/figs/}}{}
\usepackage{amsmath,amssymb,amsfonts,amsthm}
\newcommand{\myvec}[1]{\ensuremath{\begin{pmatrix}#1\end{pmatrix}}}
\providecommand{\norm}[1]{\lVert#1\rVert}
\usepackage{listings}
\usepackage{watermark}
\usepackage{titlesec}
\usepackage{caption}
\let\vec\mathbf
\lstset{
frame=single, 
breaklines=true,
columns=fullflexible
}
\thiswatermark{\centering \put(0,-105.0){\includegraphics[scale=0.15]{/sdcard/IITH/vector/vectpr-4/chapters/10/7/4/1/figs/logo.png}} }
\title{\mytitle}
\title{
Assignment - Vector-4
}
\author{Surajit Sarkar}
\begin{document}
\maketitle
%\tableofcontents
\bigskip
\section{\textbf{Problem}}
Determine the ratio in which the line 2x+y–4=0 divides the line segment joining the points A(2,–2) and B(3,7).
\section{\textbf{Solution}}
\begin{table}[h]
    \centering
    \begin{tabular}{|c|c|}
       \hline
       \textbf{Symbol}&\textbf{Value}  \\
       \hline
	    $\vec{A}$ & $\myvec{2\\-2}$\\
        \hline
	    $\vec{B}$ & $\myvec{3\\7}$\\
        \hline
	    c&$4$\\
        \hline
       $\vec{n}$ & $\myvec{2\\1}$\\
       \hline
    \end{tabular}
    \caption{Parameters}
    \label{tab:my_label}
\end{table}
Given equation
\fi
The given equation can be expressed as
\begin{align}
    \myvec{2&1}\vec{x}&=4\\
\end{align}
Using section formula, the point of division 
\begin{align}
    \vec{P} = \frac{k\vec{B+A}}{k+1}
\end{align}
which upon substitution in the equation of a line yields
\begin{align}
    \implies\vec{n}^{\top}\myvec{\frac{k\vec{B+A}}{k+1}}&=c\\
    \implies k&=\frac{c-\vec{n}^{\top}\vec{A}}{\vec{n}^{\top}\vec{B}-c}\\
\end{align}
upon simplification.  Substituting numerical values, 
\begin{align}
    k=\frac{2}{9}
\end{align}
See Fig. 
\ref{fig:chapters/10/7/4/1vec}.
\begin{figure}[!h]
\centering
\includegraphics[width=\columnwidth]{chapters/10/7/4/1/figs/vec.pdf}
\caption{}
\label{fig:chapters/10/7/4/1vec}
\end{figure}


\item Let $\vec{A}(4, 2), \vec{B}(6, 5)$  and $ \vec{C}(1, 4)$ be the vertices of $\triangle ABC$.
\begin{enumerate}
\item The median from $\vec{A}$ meets $BC$ at $\vec{D}$. Find the coordinates of the point $\vec{D}$.
\item Find the coordinates of the point $\vec{P}$ on $AD$ such that $AP : PD = 2 : 1$.
\item Find the coordinates of points $\vec{Q}$ and $\vec{R}$ on medians $BE$ and $CF$ respectively such that $BQ : QE = 2 : 1$  and  $CR : RF = 2 : 1$.
\item What do you observe?
\item If $\vec{A}, \vec{B}$ and $\vec{C}$  are the vertices of $\triangle ABC$, find the coordinates of the centroid of the triangle.
\end{enumerate}
\solution
	\iffalse
\documentclass[12pt]{article}
\usepackage{graphicx}
\usepackage[none]{hyphenat}
\usepackage{graphicx}
\usepackage{listings}
\usepackage[english]{babel}
\usepackage{graphicx}
\usepackage{caption} 
\usepackage{booktabs}
\usepackage{array}
\usepackage{amssymb} % for \because
\usepackage{amsmath}   % for having text in math mode
\usepackage{extarrows} % for Row operations arrows
\usepackage{listings}
\usepackage[utf8]{inputenc}
\lstset{
  frame=single,
  breaklines=true
}
\usepackage{hyperref}
  
%Following 2 lines were added to remove the blank page at the beginning
\usepackage{atbegshi}% http://ctan.org/pkg/atbegshi
\AtBeginDocument{\AtBeginShipoutNext{\AtBeginShipoutDiscard}}


%New macro definitions
\newcommand{\mydet}[1]{\ensuremath{\begin{vmatrix}#1\end{vmatrix}}}
\providecommand{\brak}[1]{\ensuremath{\left(#1\right)}}
\newcommand{\solution}{\noindent \textbf{Solution: }}
\newcommand{\myvec}[1]{\ensuremath{\begin{pmatrix}#1\end{pmatrix}}}
\providecommand{\norm}[1]{\left\lVert#1\right\rVert}
\providecommand{\abs}[1]{\left\vert#1\right\vert}
\let\vec\mathbf

\begin{document}

\begin{center}
\title{\textbf{VECTORS}}
\date{\vspace{-5ex}} %Not to print date automatically
\maketitle
\end{center}

\section{10$^{th}$ Maths - EXERCISE-7.4}

Let A(4, 2), B(6, 5) and C(1, 4) be the vertices of $\triangle ABC$
\begin{enumerate}
\item The median from A meets BC at D. Find the coordinates of the point D.
\item Find the coordinates of the point P on AD such that $AP : PD = 2 : 1$
\item Find the coordinates of points Q and R on medians BE and CF respectively such
that $BQ : QE = 2 : 1 \text{and} CR : RF = 2 : 1.$
\item What do yo observe?
\item If $A(x_1, y_1), B(x_2, y_2) \text{and} C(x_3, y_3)$ are the vertices of $\triangle ABC$, find the coordinates of the centroid of the triangle.
\end{enumerate}

Given points are
\begin{align}
\vec{A}=\myvec{4\\ 2} ,
\vec{B}=\myvec{6\\ 5} ,
\vec{C}=\myvec{1\\ 4}
\end{align}
\fi

\begin{enumerate}
\item 
\begin{align}
\vec{D}&=\frac{\vec{B}+\vec{C}}{2}\\
&=\myvec{\frac{7}{2}\\[2pt] \frac{9}{2}}\\
\vec{E}&=\frac{\vec{A}+\vec{C}}{2}\\
&=\myvec{\frac{5}{2}\\ 3}\\
\vec{F}&=\frac{\vec{A}+\vec{B}}{2}\\
&=\myvec{5\\ \frac{7}{2}}
\end{align}

\item 
	For
$n=2$,
\begin{align}
\vec{P}&=\frac{1}{1+n}\brak{\myvec{\vec{A}+n\vec{D}}}\\
&=\frac{1}{3}\myvec{11\\11}
\end{align}

\item 
\begin{align}
\vec{Q}&=\frac{1}{1+n}\brak{\myvec{\vec{B}+n\vec{E}}}\\
&=\frac{1}{3}\myvec{11\\11}\\
\vec{R}&=\frac{1}{1+n}\brak{\myvec{\vec{C}+n\vec{F}}}\\
&=\frac{1}{3}\myvec{11\\11}\\
\end{align}

\item 
 $\vec{P},\vec{Q},\vec{R}$ are the same point.
   
\item 
\begin{align}
\vec{G}&=\frac{\vec{D}+\vec{E}+\vec{F}}{3}\\
&=\frac{1}{3}\myvec{11\\11}\\
\end{align} 
\end{enumerate}
See Fig.  
  \ref{fig:chapters/10/7/4/7/Figure}.
\begin{figure}[h!]
\centering
\includegraphics[width=\columnwidth]{chapters/10/7/4/7/figs/dj.pdf}
\caption{}
  \label{fig:chapters/10/7/4/7/Figure}
\end{figure}

\item Find the slope of a line, which passes through the origin and the mid point of the line segment joining the points $\vec{P}$(0,-4) and $\vec{B}$(8,0).
\label{chapters/11/10/1/5}
\iffalse
\documentclass[journal,12pt,twocolumn]{IEEEtran}
\usepackage{graphicx}
\graphicspath{{./figs/}}{}
\usepackage{amsmath,amssymb,amsfonts,amsthm}
\newcommand{\myvec}[1]{\ensuremath{\begin{pmatrix}#1\end{pmatrix}}}

\let\vec\mathbf

\title{
Matrix-Lines
}
\author{Jyothsna Paluchuri-FWC22059\\}
\begin{document}
\maketitle
\tableofcontents
\bigskip
\section{Problem Statement}
\fi
	\begin{figure}[!ht]
		\centering
 \includegraphics[width=\columnwidth]{chapters/11/10/1/5/figs/line.png}
		\caption{}
		\label{fig:11/10/1/5}
  	\end{figure}
	\\
	\solution
\iffalse
\section{Construction}
\begin{figure}[h]
    \centering
\includegraphics[width=\columnwidth]{line.png}
    \caption{Equation of the slope}
    \label{fig:my_label}
\end{figure}
\vspace{2cm}
\begin{table}[h]
    \centering
    \begin{tabular}{|c|c|c|c|}
       \hline
       \textbf{Symbol}&\textbf{Value}&\textbf{Description}  \\
       \hline
	    $\vec{P}$ & $\myvec{
		    0\\
		    -4}$
	    & Point on Y-axis\\
        \hline
	    $\vec{B}$ & $\myvec{8\\0}$
 & Point on X-axis\\
        \hline
	    $\vec{0}$ & $\myvec{0\\0}$
 & Origin\\
        \hline
    \end{tabular}
    \caption{Parameters}
    \label{tab:my_label}
\end{table}


\section{Solution}
Given that resultant line passes through origin and mid point of the line segment joining point P(0,-4) and B(8,0) \\
\\
\\
given ${\vec{P}}$=$\myvec{
  0\\
  -4}$
 , ${\vec{B}}$=$\myvec{
  8\\
  0}$
  
 \fi 
The mid point of $PB$ is
\begin{align}
\vec{M} &=\frac{1}{2}(\vec{P}+\vec{B})
	= \myvec{4 \\ -2}  
\end{align}
The direction vector of line joining $\vec{O}, \vec{M}$ is 
\begin{align}
\vec{m}&=\vec{O}-\vec{M}
 = -\vec{M}
\end{align}
which can be expressed as
\begin{align}
	\myvec{1 \\ -\frac{1}{2}}
\end{align}
Thus the slope is
\begin{align}
	m = -\frac{1}{2}
\end{align}
\iffalse
\textbf{The direction vector of a line expressed as}
\begin{align}
\implies\vec{m} &= \begin{pmatrix}1 \\ m \\ \end{pmatrix}
\end{align}

\textbf{By solving equation (5) and (6),we get the slope of $\vec{O}$ $\vec{M}$ line}
\begin{align}
        \boxed{m=-0.5}
 \end{align}

\section{Software}
Download the following code using,
\begin{table}[h]
    \centering
    \begin{tabular}{|c|}
    \hline \\
   https://github.com/jyothsna777/jyothsna-fwc.git  \\
         \\
\hline
    \end{tabular}
\end{table}
\\
and execute the code by using command
\begin{center}
\textbf{Python3 lines.py}\\
\end{center}

\section{Conclusion}
Hence the slope of line $\vec{O}$ $\vec{M}$ lineis $\vec{m}$=-0.5

\end{document}
\fi

\item Find the position vector of a point R which divides the line joining two points P and Q whose position vectors are $(2\vec{a}+\vec{b})$ and $(\vec{a}-3\vec{b})$
externally in the ratio 1 : 2. Also, show that P is the mid point of the line segment RQ.\\
	\solution
%		\begin{enumerate}[label=\thesection.\arabic*,ref=\thesection.\theenumi]
\numberwithin{equation}{enumi}
\numberwithin{figure}{enumi}
\numberwithin{table}{enumi}

\item Find the coordinates of the point which divides the join of $(-1,7) \text{ and } (4,-3)$ in the ratio 2:3.
	\\
		\solution
	\input{chapters/10/7/2/1/section.tex}
\item Find the coordinates of the points of trisection of the line segment joining $(4,-1) \text{ and } (-2,3)$.
	\\
		\solution
	\input{chapters/10/7/2/2/section.tex}
\item
	\iffalse
\item To conduct Sports Day activities, in your rectangular shaped school                   
ground ABCD, lines have 
drawn with chalk powder at a                 
distance of 1m each. 100 flower pots have been placed at a distance of 1m 
from each other along AD, as shown 
in Fig. 7.12. Niharika runs $ \frac {1}{4} $th the 
distance AD on the 2nd line and 
posts a green flag. Preet runs $ \frac {1}{5} $th 
the distance AD on the eighth line 
and posts a red flag. What is the 
distance between both the flags? If 
Rashmi has to post a blue flag exactly 
halfway between the line segment 
joining the two flags, where should 
she post her flag?
\begin{figure}[h!]
  \centering
  \includegraphics[width=\columnwidth]{sc.png}
  \caption{}
\label{fig:10/7/12Fig1}
\end{figure}               
\fi
      
\item Find the ratio in which the line segment joining the points $(-3,10) \text{ and } (6,-8)$ $\text{ is divided by } (-1,6)$.
	\\
		\solution
	\input{chapters/10/7/2/4/section.tex}
\item Find the ratio in which the line segment joining $A(1,-5) \text{ and } B(-4,5)$ $\text{is divided by the x-axis}$. Also find the coordinates of the point of division.
\item If $(1,2), (4,y), (x,6), (3,5)$ are the vertices of a parallelogram taken in order, find x and y.
	\\
		\solution
	\input{chapters/10/7/2/6/para1.tex}
\item Find the coordinates of a point A, where AB is the diameter of a circle whose centre is $(2,-3) \text{ and }$ B is $(1,4)$.
	\\
		\solution
	\input{chapters/10/7/2/7/section.tex}
\item If A \text{ and } B are $(-2,-2) \text{ and } (2,-4)$, respectively, find the coordinates of P such that AP= $\frac {3}{7}$AB $\text{ and }$ P lies on the line segment AB.
	\\
		\solution
	\input{chapters/10/7/2/8/section.tex}
\item Find the coordinates of the points which divide the line segment joining $A(-2,2) \text{ and } B(2,8)$ into four equal parts.
	\\
		\solution
	\input{chapters/10/7/2/9/section.tex}
\item Find the area of a rhombus if its vertices are $(3,0), (4,5), (-1,4) \text{ and } (-2,-1)$ taken in order. [$\vec{Hint}$ : Area of rhombus =$\frac {1}{2}$(product of its diagonals)]
	\\
		\solution
	\input{chapters/10/7/2/10/cross.tex}
\item Find the position vector of a point R which divides the line joining two points $\vec{P}$
and $\vec{Q}$ whose position vectors are $\hat{i}+2\hat{j}-\hat{k}$ and $-\hat{i}+\hat{j}+\hat{k}$ respectively, in the
ratio 2 : 1
\begin{enumerate}
    \item  internally
    \item  externally
\end{enumerate}
\solution
		\input{chapters/12/10/2/15/section.tex}
\item Find the position vector of the mid point of the vector joining the points $\vec{P}$(2, 3, 4)
and $\vec{Q}$(4, 1, –2).
\\
\solution
		\input{chapters/12/10/2/16/section.tex}
\item Determine the ratio in which the line $2x+y  - 4=0$ divides the line segment joining the points $\vec{A}(2, - 2)$  and  $\vec{B}(3, 7)$.
\\
\solution
	\input{chapters/10/7/4/1/section.tex}
\item Let $\vec{A}(4, 2), \vec{B}(6, 5)$  and $ \vec{C}(1, 4)$ be the vertices of $\triangle ABC$.
\begin{enumerate}
\item The median from $\vec{A}$ meets $BC$ at $\vec{D}$. Find the coordinates of the point $\vec{D}$.
\item Find the coordinates of the point $\vec{P}$ on $AD$ such that $AP : PD = 2 : 1$.
\item Find the coordinates of points $\vec{Q}$ and $\vec{R}$ on medians $BE$ and $CF$ respectively such that $BQ : QE = 2 : 1$  and  $CR : RF = 2 : 1$.
\item What do you observe?
\item If $\vec{A}, \vec{B}$ and $\vec{C}$  are the vertices of $\triangle ABC$, find the coordinates of the centroid of the triangle.
\end{enumerate}
\solution
	\input{chapters/10/7/4/7/section.tex}
\item Find the slope of a line, which passes through the origin and the mid point of the line segment joining the points $\vec{P}$(0,-4) and $\vec{B}$(8,0).
\label{chapters/11/10/1/5}
\input{chapters/11/10/1/5/matrix.tex}
\item Find the position vector of a point R which divides the line joining two points P and Q whose position vectors are $(2\vec{a}+\vec{b})$ and $(\vec{a}-3\vec{b})$
externally in the ratio 1 : 2. Also, show that P is the mid point of the line segment RQ.\\
	\solution
%		\input{chapters/12/10/5/9/section.tex}

\end{enumerate}



\end{enumerate}



\end{enumerate}


\subsection{Exercises}
\begin{enumerate}[label=\thesection.\arabic*,ref=\thesection.\theenumi]
\numberwithin{equation}{enumi}
\numberwithin{figure}{enumi}
\numberwithin{table}{enumi}

\item The point which divides the line segment joining the points $\vec{P} (7, –6) \text{ and } (3, 4)$ in
ratio 1 : 2 internally lies in the
\begin{enumerate}

\item I quadrant

\item  II quadrant

\item  III quadrant

\item  IV quadrant
\end{enumerate}

\item If the point $\vec{P} (2, 1)$ lies on the line segment joining points$\vec{A} (4, 2) \text{ and } \vec{B} (8, 4)$,
then
\begin{enumerate}
	\item $\vec{AP} =\frac{1}{3}\vec{AB}$ 
\item $\vec{AP}=\vec{PE}$
\item $\vec{PB}=\frac{1}{3}\vec{AB}$
\item$\vec{AP}=\frac{1}{2}\vec{AB}$
 \end{enumerate}
 \item If P $\frac{a}{3}$ is the mid-point of the line segment joining the points $\vec{Q} (– 6, 5) \text{ and }(– 2, 3),$ then the value of a is
\begin{enumerate}
\item – 4
\item – 12
\item 12
\item – 6
\end{enumerate}
\item A line interects the y-axis and x-axis of the points $\vec{P} \text{ and }\vec{Q}$, respectiveiy.lf $(2,5)$is the mid-point of $\vec{PQ}$.then the coordinates of $\vec{P} \text{ and } \vec{Q}$ are,respectively
\begin{enumerate}
	\item$(0,-5)\text{ and }(2,0)$
	\item$(0,-10)\text{ and }(-4,0)$
	\item$(0,4)\text{ and } (-10,0)$
	\item$(0,-10)\text{ and }(4,0)$
\end{enumerate}
\item Point $\vec{P}(5,-3)$ is one of the two points of trisection of line segment joining the points $\vec{A}(7,-2)\text{ and }\vec{B}(1,-5)$
\item Points $\vec{A}(-6,10),\vec{B}(-4,6) \text{ and } \vec{C}(3,-8)$ are collinear such that $\vec{A}\vec{B}=  \frac{2}{9}\vec{A}\vec{C}$
\item In what ratio does the $x$-axis divide the line segment joining the points $(-4,-6)\text{ and }(-1,7)$? Find the coordinates of the point of division.
\item Find the ratio in which the point $\vec{P}\brak{\frac{3}{4},\frac{5}{12}}$ divides the line segment joining the points $\vec{A}\brak{\frac{1}{2},\frac{3}{2}}\text{ and }{B}(2,-5)$.
\item If $\vec{P}(9a-2,-b)$ divides line segment joining $\vec{A}(3a+1,-3)\text{ and }\vec{B}(8a,5)$ in the ratio 3:1,find the values of $a$ and $b$.
\item The line segment joining the points $\vec{A}(3,2)\text{ and }\vec{B}(5,1)$ is divided at the point $\vec{P}$ in the ratio 1:2 and it lies $3x-18y+k=0$, Find the value of k  
\item Find the coordinates of the point $\vec{R}$ on the line segment joining the points $\vec{P}(-1,3)\text{ and }\vec{Q}(2,5)$ such that $\vec{PR}={\frac{3}{5}}\vec{PQ}$.
\item Find the ratio in which the line 2x+3y-5=0 divides the line segment joining the points $(8,-9)\text{ and }(2,1)$. Also find the coordinates of the point of division,
\item If $\vec{a}$ $\text{and}$ $\vec{b}$ are the postion vectors of A and B, respectively, find the position vector of a point C in BA produced such that BC=1.5BA.
\item The position vector of the point which divides the join of points 2$\vec{a}$-3$\vec{b}$ $\text{and}$ $\vec{a}+\vec{b}$ in the ratio 3:1 is
	\begin{enumerate}
\item $\frac{3\vec{a}-2\vec{b}}{2}$
\item $\frac{7\vec{a}-8\vec{b}}{4}$
\item $\frac{\vec{3a}}{4}$
\item $\frac{\vec{5a}}{4}$
\end{enumerate}
\end{enumerate}


\subsection{Rank}
\iffalse
\documentclass[12pt]{article}
\usepackage{graphicx}
\usepackage{amsmath}
\usepackage{mathtools}
\usepackage{gensymb}

\newcommand{\mydet}[1]{\ensuremath{\begin{vmatrix}#1\end{vmatrix}}}
\providecommand{\brak}[1]{\ensuremath{\left(#1\right)}}
\providecommand{\norm}[1]{\left\lVert#1\right\rVert}
\newcommand{\solution}{\noindent \textbf{Solution: }}
\newcommand{\myvec}[1]{\ensuremath{\begin{pmatrix}#1\end{pmatrix}}}
\let\vec\mathbf

\begin{document}
\begin{center}
\textbf\large{CHAPTER-7 \\ COORDINATE GEOMETRY}
\end{center}
\section*{Excercise 7.4}

Q2. Find a relation between x and y if the points $\vec(x, y), \vec(1, 2) \text{ and } \vec(7, 0)$ are collinear.
\\
\solution
\\
The coordinates are given as
\fi
Let
	\begin{align}
	\vec{A} = \myvec{
		x\\
		y\\
		},
	\vec{B} = \myvec{
		1\\
		2\\
		},
	\vec{C} = \myvec{
		7\\
		0\\
		}
	\end{align}
	Then
	\begin{align}
\vec{D} &=\brak{\vec{A}-\vec{B}} = \brak{\myvec{x \\y } - \myvec{1 \\2 } } = \myvec{x-1 \\ y-2 }\\
\vec{E} &= \brak{\vec{A}-\vec{C}} = \brak{\myvec{x \\ y } - \myvec{7 \\0} } = \myvec{x-7 \\y}
\end{align}
Forming the collinearity matrix
\begin{align}
	\vec{F} &={\myvec{\vec{D}^{\top}\\ \vec{E}^{\top}}}
\end{align}
and performing row reduction,
\begin{align}
\label{eq:chapters/10/7/4/2chem_balance_mat_row}
\myvec{
x-1 & y-2
\\
x-7 & y
}
\xleftrightarrow[]{R_2 = R_2-R_1}
\myvec{
  x-1 & y-2
  \\
	  -6 & 2                 
	  }
	  \\
	\xleftrightarrow[]{R_2 = \frac{R_2}{-6}(x-1)-R_1}
\myvec{
x-1 & y-2
\\
	0 & -\frac{1}{3}(x-1)-(y-2)
}
\end{align}
For the rank of the matrix to be 1,
\begin{align}
	-\frac{1}{3}(x-1)-(y-2)&=0\\
	\implies \myvec{1 & 3}\vec{x} &=7	
\end{align}
For $x=-2, y=3$, see Fig. \ref{fig:chapters/10/7/4/2Fig} verifying that the points are collinear.
\begin{figure}[!h]
	\begin{center} 
	    \includegraphics[width=\columnwidth]{chapters/10/7/4/2/figs/sc1.png}
	\end{center}
\caption{}
\label{fig:chapters/10/7/4/2Fig}
\end{figure}

\subsection{Exercises}
\iffalse
\documentclass[12pt]{article}
\usepackage{graphicx}
\usepackage{amsmath}
\usepackage{mathtools}
\usepackage{gensymb}

\newcommand{\mydet}[1]{\ensuremath{\begin{vmatrix}#1\end{vmatrix}}}
\providecommand{\brak}[1]{\ensuremath{\left(#1\right)}}
\providecommand{\norm}[1]{\left\lVert#1\right\rVert}
\newcommand{\solution}{\noindent \textbf{Solution: }}
\newcommand{\myvec}[1]{\ensuremath{\begin{pmatrix}#1\end{pmatrix}}}
\let\vec\mathbf

\begin{document}
\begin{center}
\textbf\large{CHAPTER-7 \\ COORDINATE GEOMETRY}
\end{center}
\section*{Excercise 7.4}

Q2. Find a relation between x and y if the points $\vec(x, y), \vec(1, 2) \text{ and } \vec(7, 0)$ are collinear.
\\
\solution
\\
The coordinates are given as
\fi
Let
	\begin{align}
	\vec{A} = \myvec{
		x\\
		y\\
		},
	\vec{B} = \myvec{
		1\\
		2\\
		},
	\vec{C} = \myvec{
		7\\
		0\\
		}
	\end{align}
	Then
	\begin{align}
\vec{D} &=\brak{\vec{A}-\vec{B}} = \brak{\myvec{x \\y } - \myvec{1 \\2 } } = \myvec{x-1 \\ y-2 }\\
\vec{E} &= \brak{\vec{A}-\vec{C}} = \brak{\myvec{x \\ y } - \myvec{7 \\0} } = \myvec{x-7 \\y}
\end{align}
Forming the collinearity matrix
\begin{align}
	\vec{F} &={\myvec{\vec{D}^{\top}\\ \vec{E}^{\top}}}
\end{align}
and performing row reduction,
\begin{align}
\label{eq:chapters/10/7/4/2chem_balance_mat_row}
\myvec{
x-1 & y-2
\\
x-7 & y
}
\xleftrightarrow[]{R_2 = R_2-R_1}
\myvec{
  x-1 & y-2
  \\
	  -6 & 2                 
	  }
	  \\
	\xleftrightarrow[]{R_2 = \frac{R_2}{-6}(x-1)-R_1}
\myvec{
x-1 & y-2
\\
	0 & -\frac{1}{3}(x-1)-(y-2)
}
\end{align}
For the rank of the matrix to be 1,
\begin{align}
	-\frac{1}{3}(x-1)-(y-2)&=0\\
	\implies \myvec{1 & 3}\vec{x} &=7	
\end{align}
For $x=-2, y=3$, see Fig. \ref{fig:chapters/10/7/4/2Fig} verifying that the points are collinear.
\begin{figure}[!h]
	\begin{center} 
	    \includegraphics[width=\columnwidth]{chapters/10/7/4/2/figs/sc1.png}
	\end{center}
\caption{}
\label{fig:chapters/10/7/4/2Fig}
\end{figure}

\subsection{Scalar Product}
\begin{enumerate}[label=\thesection.\arabic*,ref=\thesection.\theenumi]
\numberwithin{equation}{enumi}
\numberwithin{figure}{enumi}
\numberwithin{table}{enumi}
\item Find the angle between two vectors $\overrightarrow{a}$ and $\overrightarrow {b} $ with magnitudes $\sqrt{3}$ and 2 respectively having $\overrightarrow {a}.\overrightarrow {b}=\sqrt{6}$.
	\\
	\solution
		\iffalse
\documentclass[10pt]{article}
\usepackage{graphicx}
\usepackage[none]{hyphenat}
\usepackage{graphicx}
\usepackage{listings}
\usepackage[english]{babel}
\usepackage{siunitx}
\usepackage{graphicx}
\usepackage{caption} 
\usepackage{booktabs}
\usepackage{array}
\usepackage{amssymb} % for \because
\usepackage{amsmath}   % for having text in math mode
\usepackage{extarrows} % for Row operations arrows
\usepackage{listings}
\usepackage[utf8]{inputenc}
\lstset{
  frame=single,
  breaklines=true
}
\usepackage{hyperref}
  
%Following 2 lines were added to remove the blank page at the beginning
\usepackage{atbegshi}% http://ctan.org/pkg/atbegshi
\AtBeginDocument{\AtBeginShipoutNext{\AtBeginShipoutDiscard}}


%New macro definitions
\newcommand{\mydet}[1]{\ensuremath{\begin{vmatrix}#1\end{vmatrix}}}
\providecommand{\brak}[1]{\ensuremath{\left(#1\right)}}
\newcommand{\solution}{\noindent \textbf{Solution: }}
\newcommand{\myvec}[1]{\ensuremath{\begin{pmatrix}#1\end{pmatrix}}}
\providecommand{\norm}[1]{\left\lVert#1\right\rVert}
\providecommand{\abs}[1]{\left\vert#1\right\vert}
\let\vec\mathbf{}
\begin{document}

\begin{center}
\title{\textbf{VECTORS}}
\date{\vspace{-5ex}} %Not to print date automatically
\maketitle
\end{center}

\section{12$^{th}$ Maths - EXERCISE-10.3}

\begin{enumerate}
\item Find the angle between two vectors $\overrightarrow{a}$ and $\overrightarrow {b} $ with magnitudes $\sqrt{3}$ and 2 respectively having $\overrightarrow {a}.\overrightarrow {b}=\sqrt{6}$.\\  

\solution
\fi
From the given information,
\begin{align}
\norm{\vec{a}}&=\sqrt{3}\\
\norm{\vec{b}}&= 2\\
{\vec{a}^{\top}}{\vec{b}}&=\sqrt{6}  
\end{align}
Thus,
\begin{align}
\cos\theta&=\frac{{\vec{a}^{\top}}{\vec{b}}}{\norm{\vec{a}}\norm{\vec{b}}}\\
&=\frac{1}{\sqrt{2}}\\
\implies\theta&={45}\degree
\end{align}

\item Find the angle between the the vectors $\hat{i}-2\hat{j}+3\hat{k}$ and $3\hat{i}-2\hat{j}+\hat{k}$.
	\\
	\solution
		\iffalse
\documentclass[journal,12pt]{article}
\usepackage{graphicx}
\usepackage[none]{hyphenat}
\usepackage{graphicx}
\usepackage{listings}
\usepackage[english]{babel}
\usepackage{siunitx}
\usepackage{graphicx}
\usepackage{caption} 
\usepackage{booktabs}
\usepackage{array}
\usepackage{amssymb} % for \because
\usepackage{amsmath}   % for having text in math mode
\usepackage{extarrows} % for Row operations arrows
\usepackage{listings}
\usepackage[utf8]{inputenc}
\lstset{
  frame=single,
  breaklines=true
}
\usepackage{hyperref}
  
%Following 2 lines were added to remove the blank page at the beginning
\usepackage{atbegshi}% http://ctan.org/pkg/atbegshi
\AtBeginDocument{\AtBeginShipoutNext{\AtBeginShipoutDiscard}}


%New macro definitions
\newcommand{\mydet}[1]{\ensuremath{\begin{vmatrix}#1\end{vmatrix}}}
\providecommand{\brak}[1]{\ensuremath{\left(#1\right)}}
\newcommand{\solution}{\noindent \textbf{Solution: }}
\newcommand{\myvec}[1]{\ensuremath{\begin{pmatrix}#1\end{pmatrix}}}
\providecommand{\norm}[1]{\left\lVert#1\right\rVert}
\providecommand{\abs}[1]{\left\vert#1\right\vert}
\let\vec\mathbf

\begin{document}

\begin{center}
\title{\textbf{VECTORS}}
\date{\vspace{-5ex}} %Not to print date automatically
\maketitle
\end{center}

\section*{$12^{th}$ Maths - EXERCISE-10.3}

\begin{enumerate}
\item Find the angle between the vectors $\\ \overrightarrow{a}=\hat{i}-2\hat{j}+3\hat{k}$ and $\overrightarrow{b}=3\hat{i}-2\hat{j}+\hat{k}$  
\section*{solution}
\fi
Let the given points be
\begin{align}
	\vec{a} &= \myvec{1\\-2\\3} , \vec{b} = \myvec{3\\ -2 \\ 1},
\end{align}
Since 
\begin{align}
\vec{a}^{\top}\vec{b}=10, \,
\norm{\vec{a}}=\sqrt{14}, \, 
	\norm{\vec{b}}&=\sqrt{14}, 
\label{eq:chapters/12/10/3/2/5}
\\
\cos\theta=\frac{\vec{a}^{\top}\vec{b}}{\norm{\vec{a}}\norm{\vec{b}}}
	&= \frac{5}{7}
	\\
	\implies \theta&= \cos^{-1}\frac{5}{7}
\end{align}

\item Find $\abs{\overrightarrow {a}}$ and $\abs{\overrightarrow {b}}$,if ($\overrightarrow {a}+\overrightarrow {b}).(\overrightarrow {a}-\overrightarrow {b})=8$ and $\abs{\overrightarrow {a}}=8\abs{\overrightarrow {b}}$.
	\\
	\solution
		\iffalse
\documentclass[12pt]{article}
\usepackage{graphicx}
%\documentclass[journal,12pt,twocolumn]{IEEEtran}
\usepackage[none]{hyphenat}
\usepackage{graphicx}
\usepackage{listings}
\usepackage[english]{babel}
\usepackage{graphicx}
\usepackage{caption} 
\usepackage{hyperref}
\usepackage{booktabs}
\usepackage{array}
\usepackage{amsmath}   % for having text in math mode
\usepackage{listings}
\lstset{
  frame=single,
  breaklines=true
}
  
%Following 2 lines were added to remove the blank page at the beginning
\usepackage{atbegshi}% http://ctan.org/pkg/atbegshi
\AtBeginDocument{\AtBeginShipoutNext{\AtBeginShipoutDiscard}}
%


%New macro definitions
\newcommand{\mydet}[1]{\ensuremath{\begin{vmatrix}#1\end{vmatrix}}}
\providecommand{\brak}[1]{\ensuremath{\left(#1\right)}}
\providecommand{\norm}[1]{\left\lVert#1\right\rVert}
\newcommand{\solution}{\noindent \textbf{Solution: }}
\newcommand{\myvec}[1]{\ensuremath{\begin{pmatrix}#1\end{pmatrix}}}
\let\vec\mathbf

\begin{document}

\begin{center}
\title{\textbf{Vector Dot Product}}
\date{\vspace{-5ex}} %Not to print date automatically
\maketitle
\end{center}
\setcounter{page}{1}

\section{12$^{th}$ Maths - Chapter 10}
This is Problem-9 from Exercise 10.3
\begin{enumerate}
\item Find $\norm{\vec{x}}$, if for a unit vector $\vec{a}$, $\brak{\vec{x}-\vec{a}}.\brak{\vec{x}+\vec{a}} = 12$.\\
	\fi
\solution 
From the given information,
\begin{align}
  \label{eq:12/10/3/9det2f}
  \brak{\vec{x}-\vec{a}}^\top\brak{\vec{x}+\vec{a}} &= 12 \\
  \implies \vec{x}^\top\vec{x} - \vec{a}^\top\vec{x} + \vec{x}^\top\vec{a} - \vec{a}^\top\vec{a} &= 12 \\
  \implies \norm{\vec{x}}^{2} - \norm{\vec{a}}^{2} &= 12 \\
\implies   \norm{\vec{x}}^{2} - 1 &= 12  \\
	\text{or, }  
	\norm{\vec{x}} &= \sqrt{13}
\end{align}

\item Evaluate the product(3$\overrightarrow {a}-5\overrightarrow {b}).(2\overrightarrow {a}+7\overrightarrow {b}$).
	\\
	\solution
		\iffalse
\documentclass[12pt]{article}\usepackage{graphicx}
\graphicspath{{./figs/}}{}

\usepackage{amsmath,amssymb,amsfonts,amsthm}
\newcommand{\myvec}[1]{\ensuremath{\begin{pmatrix}#1\end{pmatrix}}}
%\providecommand{\norm}[1]{\lVert#1\rVert}3
\usepackage{listings}
\usepackage{watermark}
\usepackage{titlesec}
\usepackage{caption}
\let\vec\mathbf
\lstset{
frame=single, 
breaklines=true,
columns=fullflexible
}
\usepackage{atbegshi}% http://ctan.org/pkg/atbegshi
\AtBeginDocument{\AtBeginShipoutNext{\AtBeginShipoutDiscard}}
\let\vec\mathbf
\providecommand{\norm}[1]{\left\lVert#1\right\rVert}
\providecommand{\qfunc}[1]{\ensuremath{Q\left(#1\right)}}
\providecommand{\sbrak}[1]{\ensuremath{{}\left[#1\right]}}
\providecommand{\lsbrak}[1]{\ensuremath{{}\left[#1\right.}}
\providecommand{\rsbrak}[1]{\ensuremath{{}\left.#1\right]}}
\providecommand{\brak}[1]{\ensuremath{\left(#1\right)}}
\providecommand{\lbrak}[1]{\ensuremath{\left(#1\right.}}
\providecommand{\rbrak}[1]{\ensuremath{\left.#1\right)}}
\providecommand{\cbrak}[1]{\ensuremath{\left\{#1\right\}}}
\providecommand{\lcbrak}[1]{\ensuremath{\left\{#1\right.}}
\providecommand{\rcbrak}[1]{\ensuremath{\left.#1\right\}}}
\newcommand{\solution}{\noindent \textbf{Solution: }}
\newcommand{\mydet}[1]{\ensuremath{\begin{vmatrix}#1\end{vmatrix}}}
\title{\mytitle}
\begin{document}
\begin{center}
\title{\textbf{VECTOR ALGEBRA}}
\maketitle
\end{center}
\begin{enumerate}
\item\textbf{Problem statement :} Evaluate the product $\brak{3\overrightarrow{a}-5\overrightarrow{b}}\cdot\brak{2\overrightarrow{a}+7\overrightarrow{b}}$
\solution
\fi
\begin{multline}
    \brak{3\vec{a}-5\vec{b}}^{\top}\brak{2\vec{a}+7\vec{b}}= \brak{3\vec{a}^{\top}}\brak{2\vec{a}}+\brak{3\vec{a}^{\top}}\brak{7\vec{b}}-\brak{5\vec{b}^{\top}}\brak{2\vec{a}}-\brak{5\vec{b}^{\top}}\brak{7\vec{b}}
    \\
     = 6\norm{\vec{a}}^2 +21\vec{a}^{\top}\vec{b}-10\vec{b}^{\top}\vec{a}-35\norm{\vec{b}}^2 
     =6\norm{\vec{a}}^2-35\norm{\vec{b}}^2+11\vec{a}^{\top}\vec{b}
\end{multline}

\item Find the magnitude of two vectors $\overrightarrow {a}$ and $\overrightarrow {b}$, having the same magnitude and such that the angle between them is $60\degree$ and their scalar product is $\frac{1}{2}$
	\\
	\solution
		\iffalse
\documentclass[10pt]{article}
\usepackage{graphicx}
\usepackage[none]{hyphenat}
\usepackage{listings}
\usepackage[english]{babel}
\usepackage{siunitx}
\usepackage{caption}
\usepackage{booktabs}
\usepackage{array}
\usepackage{extarrows}
\usepackage{enumerate}
\usepackage{enumitem}
\usepackage{amsmath}
\usepackage{commath}
\usepackage{gensymb}
\usepackage{amssymb}
\usepackage{multicol}
\usepackage[utf8]{inputenc}
\lstset{
	frame=single,
	breaklines=true
}
\usepackage{hyperref}
%\usepackage[margin=0.8in]{geometry}
%\usepackage{exsheets}% also loads the `tasks' package
\usepackage{atbegshi}
\AtBeginDocument{\AtBeginShipoutNext{\AtBeginShipoutDiscard}}

%new macro definitions
\newcommand{\mydet}[1]{\ensuremath{\begin{vmatrix}#1\end{vmatrix}}}
\providecommand{\brak}[1]{\ensuremath{\left(#1\right)}}
\newcommand{\solution}{\noindent \textbf{Solution: }}
\newcommand{\myvet}[1]{\ensuremath{\begin{pmatrix}#1\end{pmatrix}}}
\providecommand{\norm}[1]{\left\1Vert#1\right\rVert}
\let\vec\mathbf{}


%\SetEnumitemKey{twocol}{
%	before=\raggedcolumns\begin{multicols}{2},
%	after=\end{multicols}}
%\SetEnumitemKey{fourcol}{
%	before=\raggedcolumns\begin{multicols}{4},
%	after=\end{multicols}}	


\begin{document}
\begin{center}
\title{\textbf{VECTORS}}
\date{\vspace{-5ex}}
\maketitle
\end{center}
\section*{12$^{th}$Math - Chapter 10}
This is Problem-8 from Exercise 10.3\\\\
Find the magnitude of two vectors $\overrightarrow{a}$ and $\overrightarrow{b}$, having the same magnitude and such that the angle between them is 60 $\degree$ and their scalar product is $\dfrac{1}{2}$.

\solution
\fi
Given 
\begin{align}
\norm{\vec{a}}\norm{\vec{b}}&=\frac{{{\vec{a}}^{\top}}{\vec{b}}}{\cos\theta}\\
\implies 
\norm{\vec{a}}&=\sqrt{\frac{{\vec{a}}^{\top}{\vec{b}}}{\cos\theta}}\\
\end{align}
since 
\begin{align}
\norm{\vec{a}}=\norm{\vec{b}}
\end{align}
Substituting numerical values,
\begin{align}
\norm{\vec{a}}
= \norm{\vec{b}}=1
\end{align}

\item Find $\abs{\overrightarrow {x}}$,if for a unit vector $\overrightarrow {a},(\overrightarrow {x}-\overrightarrow {a}).(\overrightarrow {x}+\overrightarrow {a}$)=12.
	\\
		\iffalse
\documentclass[12pt]{article}
\usepackage{graphicx}
%\documentclass[journal,12pt,twocolumn]{IEEEtran}
\usepackage[none]{hyphenat}
\usepackage{graphicx}
\usepackage{listings}
\usepackage[english]{babel}
\usepackage{graphicx}
\usepackage{caption} 
\usepackage{hyperref}
\usepackage{booktabs}
\usepackage{array}
\usepackage{amsmath}   % for having text in math mode
\usepackage{listings}
\lstset{
  frame=single,
  breaklines=true
}
  
%Following 2 lines were added to remove the blank page at the beginning
\usepackage{atbegshi}% http://ctan.org/pkg/atbegshi
\AtBeginDocument{\AtBeginShipoutNext{\AtBeginShipoutDiscard}}
%


%New macro definitions
\newcommand{\mydet}[1]{\ensuremath{\begin{vmatrix}#1\end{vmatrix}}}
\providecommand{\brak}[1]{\ensuremath{\left(#1\right)}}
\providecommand{\norm}[1]{\left\lVert#1\right\rVert}
\newcommand{\solution}{\noindent \textbf{Solution: }}
\newcommand{\myvec}[1]{\ensuremath{\begin{pmatrix}#1\end{pmatrix}}}
\let\vec\mathbf

\begin{document}

\begin{center}
\title{\textbf{Vector Dot Product}}
\date{\vspace{-5ex}} %Not to print date automatically
\maketitle
\end{center}
\setcounter{page}{1}

\section{12$^{th}$ Maths - Chapter 10}
This is Problem-9 from Exercise 10.3
\begin{enumerate}
\item Find $\norm{\vec{x}}$, if for a unit vector $\vec{a}$, $\brak{\vec{x}-\vec{a}}.\brak{\vec{x}+\vec{a}} = 12$.\\
	\fi
\solution 
From the given information,
\begin{align}
  \label{eq:12/10/3/9det2f}
  \brak{\vec{x}-\vec{a}}^\top\brak{\vec{x}+\vec{a}} &= 12 \\
  \implies \vec{x}^\top\vec{x} - \vec{a}^\top\vec{x} + \vec{x}^\top\vec{a} - \vec{a}^\top\vec{a} &= 12 \\
  \implies \norm{\vec{x}}^{2} - \norm{\vec{a}}^{2} &= 12 \\
\implies   \norm{\vec{x}}^{2} - 1 &= 12  \\
	\text{or, }  
	\norm{\vec{x}} &= \sqrt{13}
\end{align}

\item If the vertices A,B,C of a triangle ABC are (1,2,3),(-1,0,0)(0,1,2), respectively , then find  $\angle{ABC}. [\angle{ABC}$ is the angle between the vectors $\overrightarrow{BA}$ and $\overrightarrow{BC}$].
	\\
	\solution
		\iffalse
\documentclass[12pt]{article}
\usepackage{graphicx}
\usepackage{amsmath}
\usepackage{mathtools}
\usepackage{gensymb}

\newcommand{\mydet}[1]{\ensuremath{\begin{vmatrix}#1\end{vmatrix}}}
\providecommand{\brak}[1]{\ensuremath{\left(#1\right)}}
\providecommand{\norm}[1]{\left\lVert#1\right\rVert}
\newcommand{\solution}{\noindent \textbf{Solution: }}
\newcommand{\myvec}[1]{\ensuremath{\begin{pmatrix}#1\end{pmatrix}}}
\let\vec\mathbf

\begin{document}
\begin{center}
\textbf\large{CHAPTER-7 \\ COORDINATE GEOMETRY}

\end{center}
\section*{Excercise 7.1}

Q6.Name the type of quadilateral formed,if any, by the following points, and give reasons for your answer:
\begin{enumerate}
	\item $\brak{-1,-2}, \brak{1,0}, \brak{-1,2}, \brak{-3,0}$ 
	\item $\brak{-3,5}, \brak{3,1}, \brak{0,3}, \brak{-1,-4}$
	\item $\brak{4,5}, \brak{7,6}, \brak{4,3}, \brak{1,2}$
\end{enumerate}
\solution
\fi
\begin{enumerate}
\item The coordinates are given as
	\begin{align}
	\vec{A} = \myvec{
		-1\\
		-2\\
		},
	\vec{B} = \myvec{
		1\\
		0\\
		},
	\vec{C} = \myvec{
		-1\\
		2\\
		} \text{ and }
	\vec{D} = \myvec{
		-3\\
		0\\
		}
	\end{align}
	\begin{align}
		\vec{B} - \vec{A} &= \myvec{1\\0} - \myvec{-1\\-2} = \myvec{2\\2}\\
		\vec{C} - \vec{B} &= \myvec{-1\\2} - \myvec{1\\0} = \myvec{-2\\2}\\
		\vec{C} - \vec{D} &= \myvec{-1\\2} - \myvec{-3\\0} = \myvec{2\\2}\\
		\vec{D} - \vec{A} &= \myvec{-3\\0} - \myvec{-1\\-2} = \myvec{-2\\2}
	\end{align}
	\begin{align}	
		\vec{C} - \vec{A} &= \myvec{-1\\2} - \myvec{-1\\-2} = \myvec{0\\4}\\
		\vec{D} - \vec{B} &= \myvec{-3\\0} - \myvec{1\\0} = \myvec{-4\\0}
	\end{align}
	\begin{align}	
		\vec{B}-\vec{A} = \vec{C}-\vec{D} \text{ and } \vec{C}-\vec{B} = \vec{D}-\vec{A}.
	\end{align}
	Hence, $ABCD$ is a parallelogram.
	\begin{enumerate}
		\item Now checking if the adjacent sides are orthogonal to each other
	\begin{align}
		(\vec{B}-\vec{A})^\top (\vec{C}-\vec{B}) = \myvec{2&2} \myvec{-2\\2} = -4+4 = 0
	\end{align}
		\item Now checking if the diagonals are also orthogonal then it is a square else a rectangle.
	\end{enumerate}	
	\begin{align}
		(\vec{C}-\vec{A})^\top (\vec{D}-\vec{B}) = \myvec{0&4} \myvec{-4\\0} = 0
	\end{align}
	Hence the diagonals are orthogonal to each other.

	So, we can conclude that $ABCD$ is a square.

	As shown in Figure \ref{fig:10/7/1/6/Fig1} we can see that $ABCD$ is a square hence we can conclude that our theoritical result is verified.
 
\begin{figure}[!h]
	\begin{center} 
	    \includegraphics[width=\columnwidth]{chapters/10/7/1/6/figs/quad1}
	\end{center}
\caption{}
\label{fig:10/7/1/6/Fig1}
\end{figure}

\item The coordinates are given as
	\begin{align}
	\vec{A} = \myvec{
		-3\\
		5\\
		},
	\vec{B} = \myvec{
		3\\
		1\\
		},
	\vec{C} = \myvec{
		0\\
		3\\
		} \text{ and }
	\vec{D} = \myvec{
		-1\\
		-4\\
		}
	\end{align}
	\begin{align}
		\vec{B} - \vec{A} &= \myvec{3\\1} - \myvec{-3\\5} = \myvec{6\\-4}\\
		\vec{C} - \vec{B} &= \myvec{0\\3} - \myvec{3\\1} = \myvec{-3\\2}\\
		\vec{C} - \vec{D} &= \myvec{0\\3} - \myvec{-1\\-4} = \myvec{1\\7}\\
		\vec{D} - \vec{A} &= \myvec{-1\\-4} - \myvec{-3\\5} = \myvec{2\\-9}
	\end{align}
	\begin{align}
		\vec{C} - \vec{A} &= \myvec{0\\3} - \myvec{-3\\5} = \myvec{3\\-2}\\
		\vec{D} - \vec{B} &= \myvec{-1\\-4} - \myvec{3\\1} = \myvec{-4\\-5}
	\end{align}
	\begin{align}
	\vec{B}-\vec{A} \neq \vec{C}-\vec{D} \text{ and } \vec{C}-\vec{B} \neq \vec{D}-\vec{A},
	\end{align}
	Hence, $ABCD$ is not a parallelogram, it can be a irregular quadilateral.
	\begin{enumerate}
		\item Now to check if any three points are collinear,

	if rank of $\myvec{\vec{B}-\vec{A} & \vec{C}-\vec{B}} = 1$ then points are collinear

	Forming the collinearity matrix
	\begin{align}
		\myvec{6&-3\\-4&2} \xleftrightarrow{R_{2}\rightarrow R_{2}+\frac{2}{3}R_{1}}= \myvec{6&-3\\0&0}
	\end{align}
	\end{enumerate}
	Hence, rank = 1

	Since none of the opposite sides are parallel to each other and three points are collinear so these does not form a quadilateral.

	As shown in Figure \ref{fig:10/7/1/6/Fig2} we can see that $ABCD$ does not form a quadilateral and three points are collinear hence, our theoritical result is verified.
	
\begin{figure}[!h]
	\begin{center} 
	    \includegraphics[width=\columnwidth]{chapters/10/7/1/6/figs/quad2}
	\end{center}
\caption{}
\label{fig:10/7/1/6/Fig2}
\end{figure}
	
\item The coordinates are given as
	\begin{align}
	\vec{A} = \myvec{
		4\\
		5\\
		},
	\vec{B} = \myvec{
		7\\
		6\\
		},
	\vec{C} = \myvec{
		4\\
		3\\
		} \text{ and }
	\vec{D} = \myvec{
		1\\
		2\\
		}
	\end{align}
	\begin{align}
		\vec{B} - \vec{A} &= \myvec{7\\6} - \myvec{4\\5} = \myvec{3\\1}\\
		\vec{C} - \vec{B} &= \myvec{4\\3} - \myvec{7\\6} = \myvec{-3\\-3}\\
		\vec{C} - \vec{D} &= \myvec{4\\3} - \myvec{1\\2} = \myvec{3\\1}\\
		\vec{D} - \vec{A} &= \myvec{1\\2} - \myvec{4\\5} = \myvec{-3\\-3}
	\end{align}
	\begin{align}
		\vec{C} - \vec{A} &= \myvec{4\\3} - \myvec{4\\5} = \myvec{0\\-2}\\
		\vec{D} - \vec{B} &= \myvec{1\\2} - \myvec{7\\6} = \myvec{-6\\-4}
	\end{align}
	\begin{align}
		\vec{B}-\vec{A} = \vec{C}-\vec{D} \text{ and } \vec{C}-\vec{B} = \vec{D}-\vec{A},
	\end{align}
	Hence, $ABCD$ is a parallelogram.
	\begin{enumerate}
		\item Now checking if the adjacent sides are orthogonal to each other
	\begin{align}
		(\vec{B}-\vec{A})^\top (\vec{C}-\vec{B}) = \myvec{3&1} \myvec{-3\\-3} = -9-3 = -12
	\end{align}
	Since inner product is not zero so adjacent sides are not orthogonal.

	Hence, we can say that $ABCD$ is neither a rectangle nor a square.

		\item Now checking if the diagonals are orthogonal then it is a Rhombus.
	\begin{align}
		(\vec{C}- \vec{A})^\top (\vec{D}-\vec{B}) = \myvec{0&-2} \myvec{-6\\-4} = 0+8 = 8
	\end{align}
	\end{enumerate}		
	Hence the diagonals are also not orthogonal so we conclude that $ABCD$ is a parallelogram.

	As shown in Figure \ref{fig:10/7/1/6/Fig3} we can see that $ABCD$ forms a parallelogram hence, our theoritical result is verified.

\begin{figure}[!h]
	\begin{center} 
	    \includegraphics[width=\columnwidth]{chapters/10/7/1/6/figs/quad3}
	\end{center}
\caption{}
\label{fig:10/7/1/6/Fig3}
\end{figure}
\end{enumerate}



    \item Find the direction cosines of a line which makes equal angles with the coordinate
    axes.
		\\
		\solution
		\iffalse
\documentclass[12pt]{article}
\usepackage{graphicx}
%\documentclass[journal,12pt,twocolumn]{IEEEtran}
\usepackage[none]{hyphenat}
\usepackage{graphicx}
\usepackage{listings}
\usepackage[english]{babel}
\usepackage{graphicx}
\usepackage{caption} 
\usepackage{hyperref}
\usepackage{booktabs}
\usepackage{array}
\usepackage{amsmath}   % for having text in math mode
\usepackage{listings}
\lstset{
  frame=single,
  breaklines=true
}
  
%Following 2 lines were added to remove the blank page at the beginning
\usepackage{atbegshi}% http://ctan.org/pkg/atbegshi
\AtBeginDocument{\AtBeginShipoutNext{\AtBeginShipoutDiscard}}
%


%New macro definitions
\newcommand{\mydet}[1]{\ensuremath{\begin{vmatrix}#1\end{vmatrix}}}
\providecommand{\brak}[1]{\ensuremath{\left(#1\right)}}
\providecommand{\norm}[1]{\left\lVert#1\right\rVert}
\newcommand{\solution}{\noindent \textbf{Solution: }}
\newcommand{\myvec}[1]{\ensuremath{\begin{pmatrix}#1\end{pmatrix}}}
\let\vec\mathbf

\begin{document}

\begin{center}
\title{\textbf{Vector Dot Product}}
\date{\vspace{-5ex}} %Not to print date automatically
\maketitle
\end{center}
\setcounter{page}{1}

\section{12$^{th}$ Maths - Chapter 10}
This is Problem-9 from Exercise 10.3
\begin{enumerate}
\item Find $\norm{\vec{x}}$, if for a unit vector $\vec{a}$, $\brak{\vec{x}-\vec{a}}.\brak{\vec{x}+\vec{a}} = 12$.\\
	\fi
\solution 
From the given information,
\begin{align}
  \label{eq:12/10/3/9det2f}
  \brak{\vec{x}-\vec{a}}^\top\brak{\vec{x}+\vec{a}} &= 12 \\
  \implies \vec{x}^\top\vec{x} - \vec{a}^\top\vec{x} + \vec{x}^\top\vec{a} - \vec{a}^\top\vec{a} &= 12 \\
  \implies \norm{\vec{x}}^{2} - \norm{\vec{a}}^{2} &= 12 \\
\implies   \norm{\vec{x}}^{2} - 1 &= 12  \\
	\text{or, }  
	\norm{\vec{x}} &= \sqrt{13}
\end{align}

\item Find a unit vector perpendicular to each of the vector $\overrightarrow{a}+\overrightarrow{b}\text{ and }\overrightarrow{a}-\overrightarrow{b},\text{ where } \overrightarrow{a}=3\hat{i}+2\hat{j}+2\hat{k}\text{ and } \overrightarrow{b}=\hat{i}+2\hat{j}-2\hat{k}$. 
	\\
		\solution
		\iffalse
\documentclass[12pt]{article}
\usepackage{graphicx}
\usepackage[none]{hyphenat}
\usepackage{graphicx}
\usepackage{listings}
\usepackage[english]{babel}
\usepackage{graphicx}
\usepackage{caption} 
\usepackage{booktabs}
\usepackage{array}
\usepackage{amssymb} % for \because
\usepackage{amsmath}   % for having text in math mode
\usepackage{extarrows} % for Row operations arrows
\usepackage{listings}
\usepackage[utf8]{inputenc}
\lstset{
  frame=single,
  breaklines=true
}
\usepackage{hyperref}
  
%Following 2 lines were added to remove the blank page at the beginning
\usepackage{atbegshi}% http://ctan.org/pkg/atbegshi
\AtBeginDocument{\AtBeginShipoutNext{\AtBeginShipoutDiscard}}


%New macro definitions
\newcommand{\mydet}[1]{\ensuremath{\begin{vmatrix}#1\end{vmatrix}}}
\providecommand{\brak}[1]{\ensuremath{\left(#1\right)}}
\newcommand{\solution}{\noindent \textbf{Solution: }}
\newcommand{\myvec}[1]{\ensuremath{\begin{pmatrix}#1\end{pmatrix}}}
\providecommand{\norm}[1]{\left\lVert#1\right\rVert}
\providecommand{\abs}[1]{\left\vert#1\right\vert}
\let\vec\mathbf
\begin{document}
\begin{center}
\title{\textbf{  Unit Vector Perpendicular}}
\date{\vspace{-5ex}} %Not to print date automatically
\maketitle
\end{center}
\setcounter{page}{1}
\section{12$^{th}$ Maths - Chapter 10}
\textbf{This is Problem-2 from Exercise 10.4}
\begin{enumerate}

\item Find a unit vector  perpendicular to each of a vector $\bar{a}+\bar{b} \text{ and }\bar{a}-\bar{b}$ where  $\overrightarrow{a}=3\hat{i}+2\hat{j}+2\hat{k}\text{ and }\overrightarrow{b}=\hat{i}+2\hat{j}-2\hat{k}$
\section{Solution}
\fi
Since
\begin{align}
	\vec{a}+\vec{b}=\myvec{4\\4\\0},\,
	\vec{a}-\vec{b}=\myvec{2\\0\\4}
\end{align}
the desired vector is obtained as
\begin{align} 
\myvec{\vec{a}+\vec{b}& \vec{a}-\vec{b}}^\top\vec{x}=0\\
\implies
\myvec{
4&4&0\\
2&0&4
}
\xleftrightarrow[]{R_1=\frac{R_1}{4}}
\myvec{
1&1&0\\
2&0&4
}\xleftrightarrow[]{R_2=\frac{R_2}{2}}
\myvec{
1&1&0\\
1&0&2
}\\
\xleftrightarrow[]{R_2={R_1}-{R_2}}
\myvec{
1&1&0\\
0&-1&2
}
\xleftrightarrow[]{R_2=\frac{R_2}{-1}}
\myvec{
1&1&0\\
0&1&-2
}
\xleftrightarrow[]{R_1={R_1}-{R_2}}
\myvec{
1&0&2\\
0&1&-2
}
\end{align}
yielding
\begin{align}
\begin{split}
x_1+2x_3=0\\
x_2-2x_3=0
\end{split}
\implies 
\vec{x}
=x_3\myvec{-2\\2\\1}
\end{align}

\item If a unit vector $\overrightarrow{a}$ makes angles $\dfrac{\pi}{3}\text{ with }\hat{i}, \dfrac{\pi}{4}\text{ with }\hat{j}$ and an acute angle $\theta \text{ with }\hat{k},\text{ then find } \theta$ and hence, the components of $\overrightarrow{a}$.
	\\
		\solution
		\iffalse
\documentclass[12pt]{article}
\usepackage{graphicx}
%\documentclass[journal,12pt,twocolumn]{IEEEtran}
\usepackage[none]{hyphenat}
\usepackage{graphicx}
\usepackage{listings}
\usepackage[english]{babel}
\usepackage{graphicx}
\usepackage{caption}
\usepackage[parfill]{parskip}
\usepackage{hyperref}
\usepackage{booktabs}
\usepackage{gensymb}
%\usepackage{setspace}\doublespacing\pagestyle{plain}
\def\inputGnumericTable{}
\usepackage{color}                                            %%
    \usepackage{array}                                            %%
    \usepackage{longtable}                                        %%
    \usepackage{calc}                                             %%
    \usepackage{multirow}                                         %%
    \usepackage{hhline}                                           %%
    \usepackage{ifthen}
\usepackage{array}
\usepackage{amsmath}   % for having text in math mode
\usepackage{parallel,enumitem}
\usepackage{listings}
\lstset{
language=tex,
frame=single,
breaklines=true
}
 
%Following 2 lines were added to remove the blank page at the beginning
\usepackage{atbegshi}% http://ctan.org/pkg/atbegshi
\AtBeginDocument{\AtBeginShipoutNext{\AtBeginShipoutDiscard}}
%
%New macro definitions
\newcommand{\mydet}[1]{\ensuremath{\begin{vmatrix}#1\end{vmatrix}}}
\providecommand{\brak}[1]{\ensuremath{\left(#1\right)}}
\providecommand{\norm}[1]{\left\lVert#1\right\rVert}
\newcommand{\solution}{\noindent \textbf{Solution: }}
\newcommand{\myvec}[1]{\ensuremath{\begin{pmatrix}#1\end{pmatrix}}}
\let\vec\mathbf
\begin{document}
\begin{center}
\enlargethispage{-4cm}
\title{\textbf{Vector Algebra}}
\date{\vspace{-5ex}} %Not to print date automatically
\maketitle
\end{center}
\setcounter{page}{1}
\section*{12$^{th}$ Maths - Chapter 10}
This is Problem-3 from Exercise 10.4
\begin{enumerate}
\item If unit vector $\overrightarrow{a}$ makes angles $\frac{\pi}{3}$ with $\hat{i}$, $\frac{\pi}{4}$ with $\hat{j}$ and an acute angle $\theta$ with $\hat{k}$, then find $\theta$ and hence, the components of $\overrightarrow{a}$.

\solution
\fi
		Let 
		\begin{align}
			\vec{A}=\myvec{\cos\theta_1\\\cos\theta_2\\\cos\theta_3}
		\end{align}
		where
		\begin{align}
		\cos\theta_1 &=\cos\frac{\pi}{3}
			=\frac{1}{2}\\
			\cos\theta_2 &=\cos\frac{\pi}{4}\\
			=\frac{1}{\sqrt{2}}
		\end{align}
		Since
\begin{align}
    \norm{\vec{A}}&=1,
\sqrt{\cos^2\theta_1+\cos^2\theta_2+\cos^2\theta_3}&=1
    \implies\sqrt{\frac{1}{2}^2+\frac{1}{\sqrt{2}}^2+\cos^2\theta_3 }&=1\\
    \implies\cos\theta_3 &=\pm\frac{1}{2}
\end{align}
Since $\theta_3$ is an acute angle
\begin{align}
 \cos\theta_3=\frac{1}{2}
\end{align}
    Hence 
\begin{align}
		\vec{A}=\myvec{\frac{1}{2}\\[2pt] \frac{1}{\sqrt{2}}\\[2pt] \frac{1}{2}}
\end{align}

\item Show that the direction cosines of a vector equally inclined to the axes OX, OY and OZ are \textpm $\sbrak{\frac{1}{\sqrt{3}},\frac{1}{\sqrt{3}},\frac{1}{\sqrt{3}}}$.\\
	\solution
		\iffalse
\documentclass[12pt]{article}
\usepackage{graphicx}
%\documentclass[journal,12pt,twocolumn]{IEEEtran}
\usepackage[none]{hyphenat}
\usepackage{graphicx}
\usepackage{listings}
\usepackage[english]{babel}
\usepackage{graphicx}
\usepackage{caption} 
\usepackage{hyperref}
\usepackage{booktabs}
\usepackage{array}
\usepackage{amsmath}   % for having text in math mode
\usepackage{listings}
\lstset{
  frame=single,
  breaklines=true
}
  
%Following 2 lines were added to remove the blank page at the beginning
\usepackage{atbegshi}% http://ctan.org/pkg/atbegshi
\AtBeginDocument{\AtBeginShipoutNext{\AtBeginShipoutDiscard}}
%


%New macro definitions
\newcommand{\mydet}[1]{\ensuremath{\begin{vmatrix}#1\end{vmatrix}}}
\providecommand{\brak}[1]{\ensuremath{\left(#1\right)}}
\providecommand{\norm}[1]{\left\lVert#1\right\rVert}
\newcommand{\solution}{\noindent \textbf{Solution: }}
\newcommand{\myvec}[1]{\ensuremath{\begin{pmatrix}#1\end{pmatrix}}}
\let\vec\mathbf

\begin{document}

\begin{center}
\title{\textbf{Vector Dot Product}}
\date{\vspace{-5ex}} %Not to print date automatically
\maketitle
\end{center}
\setcounter{page}{1}

\section{12$^{th}$ Maths - Chapter 10}
This is Problem-9 from Exercise 10.3
\begin{enumerate}
\item Find $\norm{\vec{x}}$, if for a unit vector $\vec{a}$, $\brak{\vec{x}-\vec{a}}.\brak{\vec{x}+\vec{a}} = 12$.\\
	\fi
\solution 
From the given information,
\begin{align}
  \label{eq:12/10/3/9det2f}
  \brak{\vec{x}-\vec{a}}^\top\brak{\vec{x}+\vec{a}} &= 12 \\
  \implies \vec{x}^\top\vec{x} - \vec{a}^\top\vec{x} + \vec{x}^\top\vec{a} - \vec{a}^\top\vec{a} &= 12 \\
  \implies \norm{\vec{x}}^{2} - \norm{\vec{a}}^{2} &= 12 \\
\implies   \norm{\vec{x}}^{2} - 1 &= 12  \\
	\text{or, }  
	\norm{\vec{x}} &= \sqrt{13}
\end{align}

\item Write down a unit vector in XY-plane, making an angle of 30$^{\circ}$ with the positive direction of x-axis.\\
\item The scalar product of the vector $\hat{i}+\hat{j}+\hat{k}$ with a unit vector along the sum of vectors $2\hat{i}+4\hat{j}-5\hat{k}$ and $\lambda\hat{i}+2\hat{j}+3\hat{k}$ is equal to one. Find the value of $\lambda$.
\item If $\theta$ is the angle between two vectors $\vec{a}$ and $\vec{b}$,then $\vec{a}.\vec{b}\geq0$ only when 
\begin{enumerate}
\item \label{itm:chapters/12/10/5/161} $0<\theta<\frac{\pi}{2}$
\item \label{itm:chapters/12/10/5/162} $0\le\theta\le\frac{\pi}{2}$
\item \label{itm:chapters/12/10/5/163} $0<\theta<\pi$
\item \label{itm:chapters/12/10/5/164} $0\le\theta\le\pi$
\end{enumerate}
	\solution
		\iffalse
\documentclass[12pt]{article}
\usepackage{graphicx}
\usepackage{amsmath}
\usepackage{mathtools}
\usepackage{gensymb}

\newcommand{\mydet}[1]{\ensuremath{\begin{vmatrix}#1\end{vmatrix}}}
\providecommand{\brak}[1]{\ensuremath{\left(#1\right)}}
\providecommand{\norm}[1]{\left\lVert#1\right\rVert}
\newcommand{\solution}{\noindent \textbf{Solution: }}
\newcommand{\myvec}[1]{\ensuremath{\begin{pmatrix}#1\end{pmatrix}}}
\let\vec\mathbf

\begin{document}
\begin{center}
\textbf\large{CHAPTER-7 \\ COORDINATE GEOMETRY}

\end{center}
\section*{Excercise 7.1}

Q6.Name the type of quadilateral formed,if any, by the following points, and give reasons for your answer:
\begin{enumerate}
	\item $\brak{-1,-2}, \brak{1,0}, \brak{-1,2}, \brak{-3,0}$ 
	\item $\brak{-3,5}, \brak{3,1}, \brak{0,3}, \brak{-1,-4}$
	\item $\brak{4,5}, \brak{7,6}, \brak{4,3}, \brak{1,2}$
\end{enumerate}
\solution
\fi
\begin{enumerate}
\item The coordinates are given as
	\begin{align}
	\vec{A} = \myvec{
		-1\\
		-2\\
		},
	\vec{B} = \myvec{
		1\\
		0\\
		},
	\vec{C} = \myvec{
		-1\\
		2\\
		} \text{ and }
	\vec{D} = \myvec{
		-3\\
		0\\
		}
	\end{align}
	\begin{align}
		\vec{B} - \vec{A} &= \myvec{1\\0} - \myvec{-1\\-2} = \myvec{2\\2}\\
		\vec{C} - \vec{B} &= \myvec{-1\\2} - \myvec{1\\0} = \myvec{-2\\2}\\
		\vec{C} - \vec{D} &= \myvec{-1\\2} - \myvec{-3\\0} = \myvec{2\\2}\\
		\vec{D} - \vec{A} &= \myvec{-3\\0} - \myvec{-1\\-2} = \myvec{-2\\2}
	\end{align}
	\begin{align}	
		\vec{C} - \vec{A} &= \myvec{-1\\2} - \myvec{-1\\-2} = \myvec{0\\4}\\
		\vec{D} - \vec{B} &= \myvec{-3\\0} - \myvec{1\\0} = \myvec{-4\\0}
	\end{align}
	\begin{align}	
		\vec{B}-\vec{A} = \vec{C}-\vec{D} \text{ and } \vec{C}-\vec{B} = \vec{D}-\vec{A}.
	\end{align}
	Hence, $ABCD$ is a parallelogram.
	\begin{enumerate}
		\item Now checking if the adjacent sides are orthogonal to each other
	\begin{align}
		(\vec{B}-\vec{A})^\top (\vec{C}-\vec{B}) = \myvec{2&2} \myvec{-2\\2} = -4+4 = 0
	\end{align}
		\item Now checking if the diagonals are also orthogonal then it is a square else a rectangle.
	\end{enumerate}	
	\begin{align}
		(\vec{C}-\vec{A})^\top (\vec{D}-\vec{B}) = \myvec{0&4} \myvec{-4\\0} = 0
	\end{align}
	Hence the diagonals are orthogonal to each other.

	So, we can conclude that $ABCD$ is a square.

	As shown in Figure \ref{fig:10/7/1/6/Fig1} we can see that $ABCD$ is a square hence we can conclude that our theoritical result is verified.
 
\begin{figure}[!h]
	\begin{center} 
	    \includegraphics[width=\columnwidth]{chapters/10/7/1/6/figs/quad1}
	\end{center}
\caption{}
\label{fig:10/7/1/6/Fig1}
\end{figure}

\item The coordinates are given as
	\begin{align}
	\vec{A} = \myvec{
		-3\\
		5\\
		},
	\vec{B} = \myvec{
		3\\
		1\\
		},
	\vec{C} = \myvec{
		0\\
		3\\
		} \text{ and }
	\vec{D} = \myvec{
		-1\\
		-4\\
		}
	\end{align}
	\begin{align}
		\vec{B} - \vec{A} &= \myvec{3\\1} - \myvec{-3\\5} = \myvec{6\\-4}\\
		\vec{C} - \vec{B} &= \myvec{0\\3} - \myvec{3\\1} = \myvec{-3\\2}\\
		\vec{C} - \vec{D} &= \myvec{0\\3} - \myvec{-1\\-4} = \myvec{1\\7}\\
		\vec{D} - \vec{A} &= \myvec{-1\\-4} - \myvec{-3\\5} = \myvec{2\\-9}
	\end{align}
	\begin{align}
		\vec{C} - \vec{A} &= \myvec{0\\3} - \myvec{-3\\5} = \myvec{3\\-2}\\
		\vec{D} - \vec{B} &= \myvec{-1\\-4} - \myvec{3\\1} = \myvec{-4\\-5}
	\end{align}
	\begin{align}
	\vec{B}-\vec{A} \neq \vec{C}-\vec{D} \text{ and } \vec{C}-\vec{B} \neq \vec{D}-\vec{A},
	\end{align}
	Hence, $ABCD$ is not a parallelogram, it can be a irregular quadilateral.
	\begin{enumerate}
		\item Now to check if any three points are collinear,

	if rank of $\myvec{\vec{B}-\vec{A} & \vec{C}-\vec{B}} = 1$ then points are collinear

	Forming the collinearity matrix
	\begin{align}
		\myvec{6&-3\\-4&2} \xleftrightarrow{R_{2}\rightarrow R_{2}+\frac{2}{3}R_{1}}= \myvec{6&-3\\0&0}
	\end{align}
	\end{enumerate}
	Hence, rank = 1

	Since none of the opposite sides are parallel to each other and three points are collinear so these does not form a quadilateral.

	As shown in Figure \ref{fig:10/7/1/6/Fig2} we can see that $ABCD$ does not form a quadilateral and three points are collinear hence, our theoritical result is verified.
	
\begin{figure}[!h]
	\begin{center} 
	    \includegraphics[width=\columnwidth]{chapters/10/7/1/6/figs/quad2}
	\end{center}
\caption{}
\label{fig:10/7/1/6/Fig2}
\end{figure}
	
\item The coordinates are given as
	\begin{align}
	\vec{A} = \myvec{
		4\\
		5\\
		},
	\vec{B} = \myvec{
		7\\
		6\\
		},
	\vec{C} = \myvec{
		4\\
		3\\
		} \text{ and }
	\vec{D} = \myvec{
		1\\
		2\\
		}
	\end{align}
	\begin{align}
		\vec{B} - \vec{A} &= \myvec{7\\6} - \myvec{4\\5} = \myvec{3\\1}\\
		\vec{C} - \vec{B} &= \myvec{4\\3} - \myvec{7\\6} = \myvec{-3\\-3}\\
		\vec{C} - \vec{D} &= \myvec{4\\3} - \myvec{1\\2} = \myvec{3\\1}\\
		\vec{D} - \vec{A} &= \myvec{1\\2} - \myvec{4\\5} = \myvec{-3\\-3}
	\end{align}
	\begin{align}
		\vec{C} - \vec{A} &= \myvec{4\\3} - \myvec{4\\5} = \myvec{0\\-2}\\
		\vec{D} - \vec{B} &= \myvec{1\\2} - \myvec{7\\6} = \myvec{-6\\-4}
	\end{align}
	\begin{align}
		\vec{B}-\vec{A} = \vec{C}-\vec{D} \text{ and } \vec{C}-\vec{B} = \vec{D}-\vec{A},
	\end{align}
	Hence, $ABCD$ is a parallelogram.
	\begin{enumerate}
		\item Now checking if the adjacent sides are orthogonal to each other
	\begin{align}
		(\vec{B}-\vec{A})^\top (\vec{C}-\vec{B}) = \myvec{3&1} \myvec{-3\\-3} = -9-3 = -12
	\end{align}
	Since inner product is not zero so adjacent sides are not orthogonal.

	Hence, we can say that $ABCD$ is neither a rectangle nor a square.

		\item Now checking if the diagonals are orthogonal then it is a Rhombus.
	\begin{align}
		(\vec{C}- \vec{A})^\top (\vec{D}-\vec{B}) = \myvec{0&-2} \myvec{-6\\-4} = 0+8 = 8
	\end{align}
	\end{enumerate}		
	Hence the diagonals are also not orthogonal so we conclude that $ABCD$ is a parallelogram.

	As shown in Figure \ref{fig:10/7/1/6/Fig3} we can see that $ABCD$ forms a parallelogram hence, our theoritical result is verified.

\begin{figure}[!h]
	\begin{center} 
	    \includegraphics[width=\columnwidth]{chapters/10/7/1/6/figs/quad3}
	\end{center}
\caption{}
\label{fig:10/7/1/6/Fig3}
\end{figure}
\end{enumerate}



\item Find the slope of the line, which makes an angle of 30 degrees with the positive direction of y-axis measures anticlockwise.
\label{chapters/11/10/1/7}\\
\solution
\iffalse
\documentclass[12pt]{article}
\usepackage{graphicx}
\usepackage{amsmath}
\usepackage{mathtools}
\usepackage{gensymb}

\newcommand{\mydet}[1]{\ensuremath{\begin{vmatrix}#1\end{vmatrix}}}
\providecommand{\brak}[1]{\ensuremath{\left(#1\right)}}
\providecommand{\norm}[1]{\left\lVert#1\right\rVert}
\newcommand{\solution}{\noindent \textbf{Solution: }}
\newcommand{\myvec}[1]{\ensuremath{\begin{pmatrix}#1\end{pmatrix}}}
\let\vec\mathbf

\begin{document}
\begin{center}
\textbf\large{CHAPTER-7 \\ COORDINATE GEOMETRY}

\end{center}
\section*{Excercise 7.1}

Q6.Name the type of quadilateral formed,if any, by the following points, and give reasons for your answer:
\begin{enumerate}
	\item $\brak{-1,-2}, \brak{1,0}, \brak{-1,2}, \brak{-3,0}$ 
	\item $\brak{-3,5}, \brak{3,1}, \brak{0,3}, \brak{-1,-4}$
	\item $\brak{4,5}, \brak{7,6}, \brak{4,3}, \brak{1,2}$
\end{enumerate}
\solution
\fi
\begin{enumerate}
\item The coordinates are given as
	\begin{align}
	\vec{A} = \myvec{
		-1\\
		-2\\
		},
	\vec{B} = \myvec{
		1\\
		0\\
		},
	\vec{C} = \myvec{
		-1\\
		2\\
		} \text{ and }
	\vec{D} = \myvec{
		-3\\
		0\\
		}
	\end{align}
	\begin{align}
		\vec{B} - \vec{A} &= \myvec{1\\0} - \myvec{-1\\-2} = \myvec{2\\2}\\
		\vec{C} - \vec{B} &= \myvec{-1\\2} - \myvec{1\\0} = \myvec{-2\\2}\\
		\vec{C} - \vec{D} &= \myvec{-1\\2} - \myvec{-3\\0} = \myvec{2\\2}\\
		\vec{D} - \vec{A} &= \myvec{-3\\0} - \myvec{-1\\-2} = \myvec{-2\\2}
	\end{align}
	\begin{align}	
		\vec{C} - \vec{A} &= \myvec{-1\\2} - \myvec{-1\\-2} = \myvec{0\\4}\\
		\vec{D} - \vec{B} &= \myvec{-3\\0} - \myvec{1\\0} = \myvec{-4\\0}
	\end{align}
	\begin{align}	
		\vec{B}-\vec{A} = \vec{C}-\vec{D} \text{ and } \vec{C}-\vec{B} = \vec{D}-\vec{A}.
	\end{align}
	Hence, $ABCD$ is a parallelogram.
	\begin{enumerate}
		\item Now checking if the adjacent sides are orthogonal to each other
	\begin{align}
		(\vec{B}-\vec{A})^\top (\vec{C}-\vec{B}) = \myvec{2&2} \myvec{-2\\2} = -4+4 = 0
	\end{align}
		\item Now checking if the diagonals are also orthogonal then it is a square else a rectangle.
	\end{enumerate}	
	\begin{align}
		(\vec{C}-\vec{A})^\top (\vec{D}-\vec{B}) = \myvec{0&4} \myvec{-4\\0} = 0
	\end{align}
	Hence the diagonals are orthogonal to each other.

	So, we can conclude that $ABCD$ is a square.

	As shown in Figure \ref{fig:10/7/1/6/Fig1} we can see that $ABCD$ is a square hence we can conclude that our theoritical result is verified.
 
\begin{figure}[!h]
	\begin{center} 
	    \includegraphics[width=\columnwidth]{chapters/10/7/1/6/figs/quad1}
	\end{center}
\caption{}
\label{fig:10/7/1/6/Fig1}
\end{figure}

\item The coordinates are given as
	\begin{align}
	\vec{A} = \myvec{
		-3\\
		5\\
		},
	\vec{B} = \myvec{
		3\\
		1\\
		},
	\vec{C} = \myvec{
		0\\
		3\\
		} \text{ and }
	\vec{D} = \myvec{
		-1\\
		-4\\
		}
	\end{align}
	\begin{align}
		\vec{B} - \vec{A} &= \myvec{3\\1} - \myvec{-3\\5} = \myvec{6\\-4}\\
		\vec{C} - \vec{B} &= \myvec{0\\3} - \myvec{3\\1} = \myvec{-3\\2}\\
		\vec{C} - \vec{D} &= \myvec{0\\3} - \myvec{-1\\-4} = \myvec{1\\7}\\
		\vec{D} - \vec{A} &= \myvec{-1\\-4} - \myvec{-3\\5} = \myvec{2\\-9}
	\end{align}
	\begin{align}
		\vec{C} - \vec{A} &= \myvec{0\\3} - \myvec{-3\\5} = \myvec{3\\-2}\\
		\vec{D} - \vec{B} &= \myvec{-1\\-4} - \myvec{3\\1} = \myvec{-4\\-5}
	\end{align}
	\begin{align}
	\vec{B}-\vec{A} \neq \vec{C}-\vec{D} \text{ and } \vec{C}-\vec{B} \neq \vec{D}-\vec{A},
	\end{align}
	Hence, $ABCD$ is not a parallelogram, it can be a irregular quadilateral.
	\begin{enumerate}
		\item Now to check if any three points are collinear,

	if rank of $\myvec{\vec{B}-\vec{A} & \vec{C}-\vec{B}} = 1$ then points are collinear

	Forming the collinearity matrix
	\begin{align}
		\myvec{6&-3\\-4&2} \xleftrightarrow{R_{2}\rightarrow R_{2}+\frac{2}{3}R_{1}}= \myvec{6&-3\\0&0}
	\end{align}
	\end{enumerate}
	Hence, rank = 1

	Since none of the opposite sides are parallel to each other and three points are collinear so these does not form a quadilateral.

	As shown in Figure \ref{fig:10/7/1/6/Fig2} we can see that $ABCD$ does not form a quadilateral and three points are collinear hence, our theoritical result is verified.
	
\begin{figure}[!h]
	\begin{center} 
	    \includegraphics[width=\columnwidth]{chapters/10/7/1/6/figs/quad2}
	\end{center}
\caption{}
\label{fig:10/7/1/6/Fig2}
\end{figure}
	
\item The coordinates are given as
	\begin{align}
	\vec{A} = \myvec{
		4\\
		5\\
		},
	\vec{B} = \myvec{
		7\\
		6\\
		},
	\vec{C} = \myvec{
		4\\
		3\\
		} \text{ and }
	\vec{D} = \myvec{
		1\\
		2\\
		}
	\end{align}
	\begin{align}
		\vec{B} - \vec{A} &= \myvec{7\\6} - \myvec{4\\5} = \myvec{3\\1}\\
		\vec{C} - \vec{B} &= \myvec{4\\3} - \myvec{7\\6} = \myvec{-3\\-3}\\
		\vec{C} - \vec{D} &= \myvec{4\\3} - \myvec{1\\2} = \myvec{3\\1}\\
		\vec{D} - \vec{A} &= \myvec{1\\2} - \myvec{4\\5} = \myvec{-3\\-3}
	\end{align}
	\begin{align}
		\vec{C} - \vec{A} &= \myvec{4\\3} - \myvec{4\\5} = \myvec{0\\-2}\\
		\vec{D} - \vec{B} &= \myvec{1\\2} - \myvec{7\\6} = \myvec{-6\\-4}
	\end{align}
	\begin{align}
		\vec{B}-\vec{A} = \vec{C}-\vec{D} \text{ and } \vec{C}-\vec{B} = \vec{D}-\vec{A},
	\end{align}
	Hence, $ABCD$ is a parallelogram.
	\begin{enumerate}
		\item Now checking if the adjacent sides are orthogonal to each other
	\begin{align}
		(\vec{B}-\vec{A})^\top (\vec{C}-\vec{B}) = \myvec{3&1} \myvec{-3\\-3} = -9-3 = -12
	\end{align}
	Since inner product is not zero so adjacent sides are not orthogonal.

	Hence, we can say that $ABCD$ is neither a rectangle nor a square.

		\item Now checking if the diagonals are orthogonal then it is a Rhombus.
	\begin{align}
		(\vec{C}- \vec{A})^\top (\vec{D}-\vec{B}) = \myvec{0&-2} \myvec{-6\\-4} = 0+8 = 8
	\end{align}
	\end{enumerate}		
	Hence the diagonals are also not orthogonal so we conclude that $ABCD$ is a parallelogram.

	As shown in Figure \ref{fig:10/7/1/6/Fig3} we can see that $ABCD$ forms a parallelogram hence, our theoritical result is verified.

\begin{figure}[!h]
	\begin{center} 
	    \includegraphics[width=\columnwidth]{chapters/10/7/1/6/figs/quad3}
	\end{center}
\caption{}
\label{fig:10/7/1/6/Fig3}
\end{figure}
\end{enumerate}



\item Find the angle between x-axis and the line joining points (3,-1) and (4,-2).
\label{chapters/11/10/1/10}
\iffalse
\documentclass[journal,12pt,twocolumn]{IEEEtran}
\usepackage{graphicx}
\graphicspath{{./figs/}}{}
\usepackage{amsmath,amssymb,amsfonts,amsthm}
\newcommand{\myvec}[1]{\ensuremath{\begin{pmatrix}#1\end{pmatrix}}}

\let\vec\mathbf

\title{
Matrix-Lines
}
\author{Jyothsna Paluchuri-FWC22059\\}
\begin{document}
\maketitle
\tableofcontents
\bigskip
\section{Problem Statement}
\fi
	\begin{figure}[!ht]
		\centering
 \includegraphics[width=\columnwidth]{chapters/11/10/1/5/figs/line.png}
		\caption{}
		\label{fig:11/10/1/5}
  	\end{figure}
	\\
	\solution
\iffalse
\section{Construction}
\begin{figure}[h]
    \centering
\includegraphics[width=\columnwidth]{line.png}
    \caption{Equation of the slope}
    \label{fig:my_label}
\end{figure}
\vspace{2cm}
\begin{table}[h]
    \centering
    \begin{tabular}{|c|c|c|c|}
       \hline
       \textbf{Symbol}&\textbf{Value}&\textbf{Description}  \\
       \hline
	    $\vec{P}$ & $\myvec{
		    0\\
		    -4}$
	    & Point on Y-axis\\
        \hline
	    $\vec{B}$ & $\myvec{8\\0}$
 & Point on X-axis\\
        \hline
	    $\vec{0}$ & $\myvec{0\\0}$
 & Origin\\
        \hline
    \end{tabular}
    \caption{Parameters}
    \label{tab:my_label}
\end{table}


\section{Solution}
Given that resultant line passes through origin and mid point of the line segment joining point P(0,-4) and B(8,0) \\
\\
\\
given ${\vec{P}}$=$\myvec{
  0\\
  -4}$
 , ${\vec{B}}$=$\myvec{
  8\\
  0}$
  
 \fi 
The mid point of $PB$ is
\begin{align}
\vec{M} &=\frac{1}{2}(\vec{P}+\vec{B})
	= \myvec{4 \\ -2}  
\end{align}
The direction vector of line joining $\vec{O}, \vec{M}$ is 
\begin{align}
\vec{m}&=\vec{O}-\vec{M}
 = -\vec{M}
\end{align}
which can be expressed as
\begin{align}
	\myvec{1 \\ -\frac{1}{2}}
\end{align}
Thus the slope is
\begin{align}
	m = -\frac{1}{2}
\end{align}
\iffalse
\textbf{The direction vector of a line expressed as}
\begin{align}
\implies\vec{m} &= \begin{pmatrix}1 \\ m \\ \end{pmatrix}
\end{align}

\textbf{By solving equation (5) and (6),we get the slope of $\vec{O}$ $\vec{M}$ line}
\begin{align}
        \boxed{m=-0.5}
 \end{align}

\section{Software}
Download the following code using,
\begin{table}[h]
    \centering
    \begin{tabular}{|c|}
    \hline \\
   https://github.com/jyothsna777/jyothsna-fwc.git  \\
         \\
\hline
    \end{tabular}
\end{table}
\\
and execute the code by using command
\begin{center}
\textbf{Python3 lines.py}\\
\end{center}

\section{Conclusion}
Hence the slope of line $\vec{O}$ $\vec{M}$ lineis $\vec{m}$=-0.5

\end{document}
\fi

	\item The slope of a line is double of the slope of another line. If tangent of the angle between them is 1/3, find the slopes of the lines.
\label{chapters/11/10/1/11}
\iffalse
\documentclass[journal,12pt,twocolumn]{IEEEtran}
\usepackage{graphicx}
\graphicspath{{./figs/}}{}
\usepackage{amsmath,amssymb,amsfonts,amsthm}
\newcommand{\myvec}[1]{\ensuremath{\begin{pmatrix}#1\end{pmatrix}}}

\let\vec\mathbf

\title{
Matrix-Lines
}
\author{Jyothsna Paluchuri-FWC22059\\}
\begin{document}
\maketitle
\tableofcontents
\bigskip
\section{Problem Statement}
\fi
	\begin{figure}[!ht]
		\centering
 \includegraphics[width=\columnwidth]{chapters/11/10/1/5/figs/line.png}
		\caption{}
		\label{fig:11/10/1/5}
  	\end{figure}
	\\
	\solution
\iffalse
\section{Construction}
\begin{figure}[h]
    \centering
\includegraphics[width=\columnwidth]{line.png}
    \caption{Equation of the slope}
    \label{fig:my_label}
\end{figure}
\vspace{2cm}
\begin{table}[h]
    \centering
    \begin{tabular}{|c|c|c|c|}
       \hline
       \textbf{Symbol}&\textbf{Value}&\textbf{Description}  \\
       \hline
	    $\vec{P}$ & $\myvec{
		    0\\
		    -4}$
	    & Point on Y-axis\\
        \hline
	    $\vec{B}$ & $\myvec{8\\0}$
 & Point on X-axis\\
        \hline
	    $\vec{0}$ & $\myvec{0\\0}$
 & Origin\\
        \hline
    \end{tabular}
    \caption{Parameters}
    \label{tab:my_label}
\end{table}


\section{Solution}
Given that resultant line passes through origin and mid point of the line segment joining point P(0,-4) and B(8,0) \\
\\
\\
given ${\vec{P}}$=$\myvec{
  0\\
  -4}$
 , ${\vec{B}}$=$\myvec{
  8\\
  0}$
  
 \fi 
The mid point of $PB$ is
\begin{align}
\vec{M} &=\frac{1}{2}(\vec{P}+\vec{B})
	= \myvec{4 \\ -2}  
\end{align}
The direction vector of line joining $\vec{O}, \vec{M}$ is 
\begin{align}
\vec{m}&=\vec{O}-\vec{M}
 = -\vec{M}
\end{align}
which can be expressed as
\begin{align}
	\myvec{1 \\ -\frac{1}{2}}
\end{align}
Thus the slope is
\begin{align}
	m = -\frac{1}{2}
\end{align}
\iffalse
\textbf{The direction vector of a line expressed as}
\begin{align}
\implies\vec{m} &= \begin{pmatrix}1 \\ m \\ \end{pmatrix}
\end{align}

\textbf{By solving equation (5) and (6),we get the slope of $\vec{O}$ $\vec{M}$ line}
\begin{align}
        \boxed{m=-0.5}
 \end{align}

\section{Software}
Download the following code using,
\begin{table}[h]
    \centering
    \begin{tabular}{|c|}
    \hline \\
   https://github.com/jyothsna777/jyothsna-fwc.git  \\
         \\
\hline
    \end{tabular}
\end{table}
\\
and execute the code by using command
\begin{center}
\textbf{Python3 lines.py}\\
\end{center}

\section{Conclusion}
Hence the slope of line $\vec{O}$ $\vec{M}$ lineis $\vec{m}$=-0.5

\end{document}
\fi

\item    Find angle between the lines,$\sqrt{3}x+y=1$ and $x+\sqrt{3}y$=1.
\label{chapters/11/10/3/9}
\def\mytitle{LINE ASSIGNMENT}
\def\myauthor{G.Kumar}
\def\contact{kumargandhamaneni20016@gmail.com}
\def\mymodule{Future Wireless Communication (FWC)}
\documentclass[10pt, a4paper]{article}
\usepackage[a4paper,outer=1.5cm,inner=1.5cm,top=1.75cm,bottom=1.5cm]{geometry}
\twocolumn
\usepackage{graphicx}
\graphicspath{{./images/}}
\usepackage[colorlinks,linkcolor={black},citecolor={blue!80!black},urlcolor={blue!80!black}]{hyperref}
\usepackage[parfill]{parskip}
\usepackage{lmodern}
\usepackage{tikz}
\usepackage{physics}
\usepackage{tabularx}
\usetikzlibrary{calc}
\usepackage{amsmath}
\usepackage{amssymb}
\renewcommand*\familydefault{\sfdefault}
\usepackage{watermark}
\usepackage{lipsum}
\usepackage{xcolor}
\usepackage{listings}
\usepackage{float}
\usepackage{titlesec}
\providecommand{\mtx}[1]{\mathbf{#1}}
\titlespacing{\subsection}{1pt}{\parskip}{3pt}
\titlespacing{\subsubsection}{0pt}{\parskip}{-\parskip}
\titlespacing{\paragraph}{0pt}{\parskip}{\parskip}
\newcommand{\figuremacro}[5]{
    \begin{figure}[#1]
        \centering
        \includegraphics[width=#5\columnwidth]{#2}
        \caption[#3]{\textbf{#3}#4}
        \label{fig:#2}
    \end{figure}
}
\newcommand{\myvec}[1]{\ensuremath{\begin{pmatrix}#1\end{pmatrix}}}
\let\vec\mathbf
\lstset{
frame=single, 
breaklines=true,
columns=fullflexible
}
\thiswatermark{\centering \put(0,-105.0){\includegraphics[scale=0.35]{iith}} }
\title{\mytitle}
\author{\myauthor\hspace{1em}\\\contact\\IITH\hspace{0.5em}-\hspace{0.5em}\mymodule}
\date{}
\begin{document}
	\maketitle
\section*{Problem}
   Find angle between the lines,$\sqrt{3}$x+y=1 and x+$\sqrt{3}$y=1.
   \section*{Solution}
   \includegraphics[scale=0.55]{line.png}
   The input parameters for this construction are :
   \begin{center}
\begin{tabular}{|c|c|c|}
	\hline
	\textbf{Symbol}&\textbf{Value}&\textbf{Description}\\
	\hline
	P&$\
	\begin{pmatrix}
		0.57736 \\
		0 \\
	\end{pmatrix}$%
	&Point P\\ 
	\hline
	X&$\
	\begin{pmatrix}
		0.36603 \\
		0.36603 \\
	\end{pmatrix}$%
	&Point X\\
	\hline
	Q&$\
	\begin{pmatrix}
		1 \\
		0 \\
	\end{pmatrix}$%
	&Point Q\\
	
	\hline
\end{tabular}
\end{center}
   \subsection*{Step 1}
   Given two equations are, \\
   \begin{equation}
   \sqrt{3}x+y=1 
   \end{equation}
   \begin{equation}
   x+\sqrt{3}y=1 
   \end{equation}
   Equation(1) in vector form is given as,
   \begin{align}
   \myvec{\sqrt{3}&1}\vec{x}=1
   \end{align}
   From this, Normal vector to the line is given as,
   \begin{align*}
   \vec{n_1}=\myvec{\sqrt{3}\\1}
   \end{align*}
   So, the direction vector of the line is given as,
\begin{eqnarray*}
   \vec{m1}=\myvec{-1\\\sqrt{3}}
\end{eqnarray*} 
Similarly, Normal vector to the line(2) is given as,
   \begin{align*}
   \vec{n_2}=\myvec{1\\\sqrt{3}}
   \end{align*}
   So, the direction vector of the line is given as,
\begin{eqnarray*}
   \vec{m2}=\myvec{-\sqrt{3}\\1}
\end{eqnarray*}     

\subsection*{Step 2}
Now, Angle between any two lines,using their direction vectors, is given by, \\
\begin{eqnarray*}
 cos\theta=\frac{(\vec{m1})^T(\vec{m2})}{\norm{\vec{m1}}\norm{\vec{m2}}}
\end{eqnarray*}
So, Angle between the two lines is given by,
\begin{eqnarray}
 cos\angle{x}=\frac{(\vec{m1})^T(\vec{m2})}{\norm{\vec{m1}}\norm{\vec{m2}}}\
\end{eqnarray}
\begin{eqnarray}
 cos\angle{x}=\frac{\myvec{-1\\\sqrt{3}}^T\myvec{-\sqrt{3}\\1}}{\norm{\myvec{-1\\\sqrt{3}}}\norm{\myvec{-\sqrt{3}\\1}}}
\end{eqnarray}
By solving the above equation, we get, \\
\begin{equation}
cos\angle{x}=\frac{\sqrt{3}}{2} \\
\end{equation}
This Implies,
\begin{equation*}
\angle{x}=30^\circ
\end{equation*}
Therefore, the angle between given two lines is $30^\circ$. \\
\bibliographystyle{ieeetr}
\end{document}
\item Find the equation of the lines through the point (3, 2) which make an angle of 45\degree  with the line $x – 2y$ = 3.
\label{chapters/11/10/4/11}\\
\solution
%\documentclass{article}
\documentclass[10pt,a4paper]{report}
\usepackage{amsmath}
\usepackage{amssymb}
\usepackage{gensymb}
\usepackage{amsfonts}
\usepackage{setspace}
\usepackage{tasks}
\usepackage{graphicx}
\usepackage{float}
\usepackage{listings}
\newcommand{\myvec}[1]{\ensuremath{\begin{pmatrix}#1\end{pmatrix}}}
\let\vec\mathbf
\providecommand{\sbrak}[1]{\ensuremath{{}\left[#1\right]}}
\providecommand{\lsbrak}[1]{\ensuremath{{}\left[#1\right.}}
\providecommand{\rsbrak}[1]{\ensuremath{{}\left.#1\right]}}
\providecommand{\brak}[1]{\ensuremath{\left(#1\right)}}
\providecommand{\lbrak}[1]{\ensuremath{\left(#1\right.}}
\providecommand{\rbrak}[1]{\ensuremath{\left.#1\right)}}
\providecommand{\cbrak}[1]{\ensuremath{\left\{#1\right\}}}
\providecommand{\lcbrak}[1]{\ensuremath{\left\{#1\right.}}
\providecommand{\rcbrak}[1]{\ensuremath{\left.#1\right\}}}
\providecommand{\norm}[1]{\left\lVert#1\right\rVert}
\providecommand{\abs}[1]{\left\vert#1\right\vert}
\let\vec\mathbf
%\newcommand{\norm}[1]{\lVert#1\rVert}
\renewcommand{\vec}[1]{\textbf{#1}}
\begin{document}
\onehalfspacing
\begin{center}
	\section*{\textbf{Class 11}}
	\subsection*{Chapter 10 - STRAIGHT LINES}
\end{center}
The following problem is question 11 from exercise 10.4
\begin{enumerate}
    \item Find the equation of the lines through the point (3, 2) which make an angle of 45\degree  with the line x – 2y = 3.
\end{enumerate}
\textbf{Solution:}\\
The given line parameters are
\begin{align}
   \vec{n}=\myvec{1\\-2},c=-5\\
	\vec{P}=\myvec{3\\2}
\end{align}
yielding
\begin{align}
\vec{m}_1=\myvec{2\\1}\\
\vec{m}_2=\myvec{1\\m}
\end{align}
where  $m$ is defined to be the slope of the line. If the angle between the lines be $\theta$,

\begin{align}
\cos \theta = \frac{\vec{m}_1^\top \vec{m}_2}{\norm{\vec{m}_1}\norm{\vec{m}_2}}\\
	\text{given, } \theta = 45\degree\\
\implies \cos45\degree =  \frac{\vec{m}_1^\top \vec{m}_2}{\norm{\vec{m}_1}\norm{\vec{m}_2}}\\
\implies \frac{1}{\sqrt{2}} = \frac{\myvec{2 & 1} \myvec{1\\m}}{\norm{\myvec{2\\1}}\norm{\myvec{1\\m}}}
\end{align}
\begin{align}
\implies \frac{1}{\sqrt{2}}=\frac{2+m}{\sqrt{2^2 + 1}\sqrt{m^2 + 1}}\\
\implies \frac{1}{2}=\frac{m^2 + 4m +4}{5m^2 +5}\\
\text{or, } 3m^2 - 8m -3 = 0
\end{align}
yielding
\begin{align}
m= - \frac{1}{3}, 3
\end{align} 
when m=3,the equation of line passing through $\vec{P}$  is then obtained as
\begin{align}
\vec{n}^{\top} ({\vec{x}-\vec{P}}) = 0\\
\text{where,}{\vec{n}}=\myvec{m\\-1} \\
{\vec{n}}=\myvec{3\\-1} \\
\implies 
	\myvec{3&-1}\cbrak{\vec{x}-\myvec{3\\2}}&=0\\
	&=7 \\
 \implies 	\myvec{3 & -1}\vec{x} &= 7
\end{align}
And, when $m=-\frac{1}{3}$,the equation of the line passing through $\vec{P}$  and having a slope of $-\frac{1}{3}$is
\begin{align}
\vec{n}^{\top} ({\vec{x}-\vec{P}}) = 0\\
{\vec{n}}=\myvec{-\frac{1}{3}\\-1} \\
\implies {\vec{n}}=\myvec{1\\3} \\
\implies 
	\myvec{1&3}\cbrak{\vec{x}-\myvec{3\\2}}&=0\\
	&=9 \\
		\implies 	\myvec{1 & 3}\vec{x} &= 9
\end{align}
Therefore,the equations of the lines are 
\begin{align}
	\myvec{3 & -1}\vec{x} = 7  \text{ and }   \myvec{1 & 3}\vec{x} = 9 .
\end{align}
\begin{figure}[H]
\centering
\includegraphics[width=\columnwidth]{figs/stline.jpg}
\caption{STRAIGHT LINES}
\label{fig:strline.jpg}
\end{figure}




\end{document}

\begin{figure}[H]
\centering
\includegraphics[width=\columnwidth]{chapters/11/10/4/11/figs/strline.jpg}
\caption{STRAIGHT LINES}
\label{fig:chapters/11/10/4/11/figs/strline.jpg}
\end{figure}
\item\textbf{}The scalar product of the vector $\hat{i}+\hat{j}+\hat{k}$ with a unit vector along the sum of vectors $2\hat{i}+4\hat{j}-5\hat{k}$ and $\lambda\hat{i}+2\hat{j}+3\hat{k}$ is equal to one, Find the value of $\lambda$.
\\\\
\textbf{Generalized Construction:}\\
We now that \\
\begin{align}
   &\implies \vec{A}^\top = \frac{\brak{\vec{B}+\vec{C}}}{\norm{\vec{B}+\vec{C}}}\\
       &\implies \vec{A}^\top \brak{\vec{B}+\vec{C}}=\norm{\vec{B}+\vec{C}} \label{eq:Eqat2}\\
       &\implies \vec{C}=\lambda\vec{e}_1+\vec{D}\label{eq:EQT-C}
    \end{align}
    were,
    \begin{align}
       &\implies \norm{\vec{B}+\vec{C}}= \sqrt{\brak{\vec{B}+\vec{C}}^\top\brak{\vec{B}+\vec{C}}}
    \end{align}
From the Equation\eqref{eq:Eqat2},We can do
\begin{align}
   &\implies \vec{A}^\top \brak{\vec{B}+\vec{C}}=\sqrt{\brak{\vec{B}+\vec{C}}^\top\brak{\vec{B}+\vec{C}}}\\
&\implies \vec{A}^\top \brak{\vec{B}+\vec{C}}=\sqrt{\norm{\vec{B}}^2+2\sbrak{\vec{B}^{\top}\vec{C}}+\norm{\vec{C}}^2}\\
&\implies \vec{A}^\top \brak{\vec{B}+\vec{C}}=\sqrt{{\vec{B}^{\top}\vec{B}}+2\sbrak{\vec{B}^{\top}\vec{C}}+{\vec{C}^\top\vec{C}}}\label{eq:Eqt6}
\end{align}
Substitute the $\vec{C}$ Value in the Equation\eqref{eq:Eqt6},We get
\begin{align}
&\implies\vec{A}^{\top}\brak{\vec{B}+\lambda\vec{e}_1+\vec{D}}=\sqrt{\vec{B}^{\top}\vec{B}+2\vec{B}^{\top}\brak{\lambda\vec{e}_1+\vec{D}}+\brak{\lambda\vec{e}_1+\vec{D}}^{\top}\brak{\lambda\vec{e}_1+\vec{D}}}
\end{align}
S.O.B.S,we get
\begin{align}
&\implies\brak{\vec{A}^{\top}\brak{\vec{B}+\lambda\vec{e}_1+\vec{D}}}^{2}=\vec{B}^{\top}\vec{B}+2\vec{B}^{\top}\brak{\lambda\vec{e}_1+\vec{D}}+\brak{\brak{\lambda\vec{e}_1+\vec{D}}^{\top}\brak{\lambda\vec{e}_1+\vec{D}}} \\
&\implies\brak{\vec{A}^{\top}\lambda\vec{e}_1}^{2}+\brak{\vec{A}^{\top}\vec{B}+\vec{D}}^{2}+2\brak{\vec{A}^{\top}\lambda\vec{e}_1}\brak{\vec{A}^{\top}\brak{\vec{B}+\vec{D}}}=\vec{B}^{\top}\vec{B}+2\vec{B}^{\top}\brak{\lambda\vec{e}_1+\vec{D}}+\lambda^{2}+2\lambda\vec{e}_1^{\top}\vec{D}+\vec{D}^{\top}\vec{D}\\
&\implies\brak{\lambda^{2}}+\brak{\vec{A}^{\top}\brak{\vec{B}+\vec{D}}}^{2}+2\brak{\vec{A}^{\top}\lambda\vec{e}_1}\brak{\vec{A}^{\top}\brak{\vec{B}+\vec{D}}}=\vec{B}^{\top}\vec{B}+2\lambda\brak{\vec{B}^{\top}\vec{e}_1+\vec{e}_1^{\top}\vec{D}}+\vec{D}^{\top}\vec{D}+\lambda^{2}\\
&\implies2\lambda\sbrak{\vec{A}^\top\vec{e}_1\vec{A}^\top\brak{\vec{B}+\vec{D}}-\brak{\vec{B}^{\top}\vec{e}_1+\vec{e}_1^{\top}\vec{D}}}=\vec{B}^{\top}\vec{B}+2\lambda\brak{\vec{B}^{\top}\vec{e}_1+\vec{e}_1^{\top}\vec{D}}+\vec{D}^{\top}\vec{D}-\brak{\vec{A}^{\top}\brak{\vec{B}+\vec{D}}}^{2}\\
&\implies2\lambda=\frac{\vec{B}^{\top}\vec{B}+2\vec{B}^{\top}\vec{D}+\vec{D}^{\top}\vec{D}-\brak{\vec{A}^{\top}\brak{\vec{B}+\vec{D}}}^{2}}{\sbrak{\vec{A}^\top\vec{e}_1\vec{A}^\top\brak{\vec{B}+\vec{D}}-\brak{\vec{B}^{\top}\vec{e}_1+\vec{e}_1^{\top}\vec{D}}}}\\
&\implies\lambda=\frac{\vec{B}^{\top}\vec{B}+2\vec{B}^{\top}\vec{D}+\vec{D}^{\top}\vec{D}-\brak{\vec{A}^{\top}\brak{\vec{B}+\vec{D}}}^{2}}{2\sbrak{\vec{A}^\top\vec{e}_1\vec{A}^\top\brak{\vec{B}+\vec{D}}-\brak{\vec{B}^{\top}\vec{e}_1+\vec{e}_1^{\top}\vec{D}}}} \label{eq:EWQ77}
\end{align}
Substitute the Given Data in Equation\eqref{eq:EWQ77},
\begin{align*}
\vec{A}=\myvec{1\\1\\1};\vec{B}=\myvec{2\\4\\-5};\vec{C}=\myvec{\lambda\\2\\3}
\end{align*}
we get,
\begin{align}   
&\implies\lambda=\frac{45-14+13-36}{2\brak{1\brak{6}-2}}\\
&\implies\lambda=\frac{44-36}{8}\\
&\impliedby\lambda=\frac{8}{8}\\
 &\implies \lambda = 1
\end{align}
\end{enumerate}

\subsection{Exercises}
\begin{enumerate}[label=\thesection.\arabic*,ref=\thesection.\theenumi]
\numberwithin{equation}{enumi}
\numberwithin{figure}{enumi}
\numberwithin{table}{enumi}
\item Let $\vec{a}$ and $\vec{b}$ be two unit vectors and $\theta$ is the angle between them.Then $\vec{a}+\vec{b}$ is a unit vector if
\begin{enumerate}
\item $\theta=\frac{\pi}{4}$
\item $\theta=\frac{\pi}{3}$
\item $\theta=\frac{\pi}{2}$
\item $\theta=\frac{2\pi}{3}$
\end{enumerate}
\item The value of $\hat{i}.(\hat{j}\times\hat{k})+\hat{j}.(\hat{i}\times\hat{k})+\hat{k}.(\hat{i}\times\hat{j})$ is
\begin{enumerate}
\item 0
\item -1
\item 1
\item 3
\end{enumerate}
\item If $\theta$ is the angle between any two vectors $\vec{a}$ and $\vec{b}$,then $|\vec{a}.\vec{b}|=|\vec{a}\times\vec{b}|$ when $\theta$ is equal to
\begin{enumerate}
\item 0
\item $\frac{\pi}{4}$
\item $\frac{\pi}{2}$
\item $\pi$
\end{enumerate}
\item A vector $\vec{r}$ has a magnitude 14 and direction ratios 2,3,-6. Find the direction cosines and components of $\vec{r}$, given that $\vec{r}$ makes an acute angle with x-axis.
\item Find the angle between the vectors $2\hat{i}-\hat{j}+\hat{k}$ $\text{and}$ $3\hat{i}+4\hat{j}-\hat{k}$.
\item If $\vec{a},\vec{b},\vec{c}$ are the three vectors such that $\vec{a}+\vec{b}+\vec{c}=0$ $\text{ and }$ $|\vec{a}|=2$, $|\vec{b}|$=3, $|\vec{c}|$=5, the value of $\vec{a}.\vec{b}+\vec{b}.\vec{c}+\vec{c}.\vec{a}$ is
	\begin{enumerate}
\item 0
\item 1	
\item -19
\item 38
\end{enumerate}
\item If $\vec{a}$, $\vec{b}$, $\vec{c}$ are unit vectors such that $\vec{a}$+$\vec{b}$+$\vec{c}$=0, then the value of $\vec{a}.\vec{b}+\vec{b}.\vec{c}+\vec{c}.\vec{a}$ is
	\begin{enumerate}
\item 1
\item 3
\item $\frac{-3}{2}$
\item None of these
\end{enumerate}
\item The angles between two vectors $\vec{a}$ $\text{and}$ $\vec{b}$ with magnitude $\sqrt{3}$ $\text{ and }$ 4, respectively, and $\vec{a}$, $\vec{b}$= $2\sqrt{3}$ is
	\begin{enumerate}
\item $\frac{\pi}{6}$
\item $\frac{\pi}{3}$
\item $\frac{\pi}{2}$ 
\item $\frac{5\pi}{2}$
\end{enumerate}

\item The vector $\vec{a}+\vec{b}$ bisects the angle between the non-collinear vectors $\vec{a}$ $\text{ and }$ $\vec{b}$ if \rule{1cm}{0.15mm}.
\item The vectors $\vec{a}=3\hat{i}-2\hat{j}+2\hat{k}$ $\text{ and }$ $\vec{b}=\hat{i}-2\hat{k}$ are the adjancent sides of a parallelogram. The acute angle between its diagonals is \rule{1cm}{0.15mm}.
\item If $\vec{a}$ is  any non-zero vector, then $(\vec{a}.\hat{i})\hat{i}$+$(\vec{a}.\hat{j})\hat{j}$+$(\vec{a}.\hat{k})$ $\hat{k}$ equals \rule{1cm}{0.15mm}.
\item If $\vec{a}$ $\text{ and }$ $\vec{b}$ are adjacent sides of a rhombus, then $\vec{a}.\vec{b}$.=0.
\item Find the angle between the lines $$\overrightarrow{r}=3\hat{i}-2\hat{j}+6\hat{k}+\lambda(2\hat{i}+\hat{j}+2\hat{k})\text{ and } \overrightarrow{r}=(2\hat{j}-5\hat{k})+\mu(6\hat{i}+3\hat{j}+2\hat{k})$$
\item Find the angle between the lines whose direction cosines are given by the equations $l+m+n=0$, $l^2+m^2-n^2=0$.
\item If a variable line in two adjacent positions has directions cosines $l, m, n$ and $l+\delta l, m+\delta m, n+\delta n$, show that the small angle $\delta\theta$ between the two positions is given by $$\delta\theta^2=\delta l^2+\delta m^2+\delta n^2$$ 
\item The sine of the angle between the straight line $\dfrac{x-2}{3}=\dfrac{y-3}{4}=\dfrac{z-4}{5}$ and the plane $2x-2y+z=5$ is
\begin{enumerate}
	\item $\dfrac{10}{6\sqrt{5}}$ 
	\item $\dfrac{4}{5\sqrt{2}}$
	\item $\dfrac{2\sqrt{3}}{5}$
	\item $\dfrac{\sqrt{2}}{10}$
\end{enumerate}
\item The plane $2x-3y+6z-11=0$ makes an angle $\sin^{-1}(\alpha)$ with x-axis. The value of $\alpha$ is equal to 
\begin{enumerate}
	\item  $\dfrac{\sqrt{3}}{2}$
	\item  $\dfrac{\sqrt{2}}{3}$
	\item  $\dfrac{2}{7}$
	\item  $\dfrac{3}{7}$
\end{enumerate}
\item The angle between the line $\overrightarrow{r}=(5\hat{i}-\hat{j}-4\hat{k})+\lambda(2\hat{i}-\hat{j}+\hat{k})$ and the plane $\overrightarrow{r} \cdot (3\hat{i}-4\hat{j}-\hat{k})+5=0$ is $\sin^{-1}\brak{\dfrac{5}{2\sqrt{91}}}$.
\item The angle between the planes $\overrightarrow{r} \cdot (2\hat{i}-3\hat{j}+\hat{k})=1$ and $\overrightarrow{r} \cdot (\hat{i}-\hat{j})=4$ is $\cos^{-1} \brak{\dfrac{-5}{\sqrt{58}}}$.
\item Let $\vec{a}$ and $\vec{b}$ be two unit vectors and $\theta$ is the angle between them.Then $\vec{a}+\vec{b}$ is a unit vector if
\begin{enumerate}
\item $\theta=\frac{\pi}{4}$
\item $\theta=\frac{\pi}{3}$
\item $\theta=\frac{\pi}{2}$
\item $\theta=\frac{2\pi}{3}$
\end{enumerate}
\item The value of $\hat{i}.(\hat{j}\times\hat{k})+\hat{j}.(\hat{i}\times\hat{k})+\hat{k}.(\hat{i}\times\hat{j})$ is
\begin{enumerate}
\item 0
\item -1
\item 1
\item 3
\end{enumerate}
\item If $\theta$ is the angle between any two vectors $\vec{a}$ and $\vec{b}$,then $|\vec{a}.\vec{b}|=|\vec{a}\times\vec{b}|$ when $\theta$ is equal to
\begin{enumerate}
\item 0
\item $\frac{\pi}{4}$
\item $\frac{\pi}{2}$
\item $\pi$
\end{enumerate}

\end{enumerate}

\subsection{Orthogonality}
\begin{enumerate}[label=\thesection.\arabic*,ref=\thesection.\theenumi]
\numberwithin{equation}{enumi}
\numberwithin{figure}{enumi}
\numberwithin{table}{enumi}
\item Check whether$(5,-2),(6,4)$ and $(7,-2)$ are the vertices of an isosceles triangle.
\item Name the type of quadrilateral formed, if any, by the following points,and give reasons for your answer
\begin{enumerate}
\item $(-1,-2),(1,0),(-1,2),(-3,0)$
\item $(-3,5),(-3,1),(0,3),(-1,-4)$
\item $(4,5),(7,6),(4,3),(1,2)$
\end{enumerate}
\solution
		\iffalse
\documentclass[12pt]{article}
\usepackage{graphicx}
\usepackage{amsmath}
\usepackage{mathtools}
\usepackage{gensymb}

\newcommand{\mydet}[1]{\ensuremath{\begin{vmatrix}#1\end{vmatrix}}}
\providecommand{\brak}[1]{\ensuremath{\left(#1\right)}}
\providecommand{\norm}[1]{\left\lVert#1\right\rVert}
\newcommand{\solution}{\noindent \textbf{Solution: }}
\newcommand{\myvec}[1]{\ensuremath{\begin{pmatrix}#1\end{pmatrix}}}
\let\vec\mathbf

\begin{document}
\begin{center}
\textbf\large{CHAPTER-7 \\ COORDINATE GEOMETRY}

\end{center}
\section*{Excercise 7.1}

Q6.Name the type of quadilateral formed,if any, by the following points, and give reasons for your answer:
\begin{enumerate}
	\item $\brak{-1,-2}, \brak{1,0}, \brak{-1,2}, \brak{-3,0}$ 
	\item $\brak{-3,5}, \brak{3,1}, \brak{0,3}, \brak{-1,-4}$
	\item $\brak{4,5}, \brak{7,6}, \brak{4,3}, \brak{1,2}$
\end{enumerate}
\solution
\fi
\begin{enumerate}
\item The coordinates are given as
	\begin{align}
	\vec{A} = \myvec{
		-1\\
		-2\\
		},
	\vec{B} = \myvec{
		1\\
		0\\
		},
	\vec{C} = \myvec{
		-1\\
		2\\
		} \text{ and }
	\vec{D} = \myvec{
		-3\\
		0\\
		}
	\end{align}
	\begin{align}
		\vec{B} - \vec{A} &= \myvec{1\\0} - \myvec{-1\\-2} = \myvec{2\\2}\\
		\vec{C} - \vec{B} &= \myvec{-1\\2} - \myvec{1\\0} = \myvec{-2\\2}\\
		\vec{C} - \vec{D} &= \myvec{-1\\2} - \myvec{-3\\0} = \myvec{2\\2}\\
		\vec{D} - \vec{A} &= \myvec{-3\\0} - \myvec{-1\\-2} = \myvec{-2\\2}
	\end{align}
	\begin{align}	
		\vec{C} - \vec{A} &= \myvec{-1\\2} - \myvec{-1\\-2} = \myvec{0\\4}\\
		\vec{D} - \vec{B} &= \myvec{-3\\0} - \myvec{1\\0} = \myvec{-4\\0}
	\end{align}
	\begin{align}	
		\vec{B}-\vec{A} = \vec{C}-\vec{D} \text{ and } \vec{C}-\vec{B} = \vec{D}-\vec{A}.
	\end{align}
	Hence, $ABCD$ is a parallelogram.
	\begin{enumerate}
		\item Now checking if the adjacent sides are orthogonal to each other
	\begin{align}
		(\vec{B}-\vec{A})^\top (\vec{C}-\vec{B}) = \myvec{2&2} \myvec{-2\\2} = -4+4 = 0
	\end{align}
		\item Now checking if the diagonals are also orthogonal then it is a square else a rectangle.
	\end{enumerate}	
	\begin{align}
		(\vec{C}-\vec{A})^\top (\vec{D}-\vec{B}) = \myvec{0&4} \myvec{-4\\0} = 0
	\end{align}
	Hence the diagonals are orthogonal to each other.

	So, we can conclude that $ABCD$ is a square.

	As shown in Figure \ref{fig:10/7/1/6/Fig1} we can see that $ABCD$ is a square hence we can conclude that our theoritical result is verified.
 
\begin{figure}[!h]
	\begin{center} 
	    \includegraphics[width=\columnwidth]{chapters/10/7/1/6/figs/quad1}
	\end{center}
\caption{}
\label{fig:10/7/1/6/Fig1}
\end{figure}

\item The coordinates are given as
	\begin{align}
	\vec{A} = \myvec{
		-3\\
		5\\
		},
	\vec{B} = \myvec{
		3\\
		1\\
		},
	\vec{C} = \myvec{
		0\\
		3\\
		} \text{ and }
	\vec{D} = \myvec{
		-1\\
		-4\\
		}
	\end{align}
	\begin{align}
		\vec{B} - \vec{A} &= \myvec{3\\1} - \myvec{-3\\5} = \myvec{6\\-4}\\
		\vec{C} - \vec{B} &= \myvec{0\\3} - \myvec{3\\1} = \myvec{-3\\2}\\
		\vec{C} - \vec{D} &= \myvec{0\\3} - \myvec{-1\\-4} = \myvec{1\\7}\\
		\vec{D} - \vec{A} &= \myvec{-1\\-4} - \myvec{-3\\5} = \myvec{2\\-9}
	\end{align}
	\begin{align}
		\vec{C} - \vec{A} &= \myvec{0\\3} - \myvec{-3\\5} = \myvec{3\\-2}\\
		\vec{D} - \vec{B} &= \myvec{-1\\-4} - \myvec{3\\1} = \myvec{-4\\-5}
	\end{align}
	\begin{align}
	\vec{B}-\vec{A} \neq \vec{C}-\vec{D} \text{ and } \vec{C}-\vec{B} \neq \vec{D}-\vec{A},
	\end{align}
	Hence, $ABCD$ is not a parallelogram, it can be a irregular quadilateral.
	\begin{enumerate}
		\item Now to check if any three points are collinear,

	if rank of $\myvec{\vec{B}-\vec{A} & \vec{C}-\vec{B}} = 1$ then points are collinear

	Forming the collinearity matrix
	\begin{align}
		\myvec{6&-3\\-4&2} \xleftrightarrow{R_{2}\rightarrow R_{2}+\frac{2}{3}R_{1}}= \myvec{6&-3\\0&0}
	\end{align}
	\end{enumerate}
	Hence, rank = 1

	Since none of the opposite sides are parallel to each other and three points are collinear so these does not form a quadilateral.

	As shown in Figure \ref{fig:10/7/1/6/Fig2} we can see that $ABCD$ does not form a quadilateral and three points are collinear hence, our theoritical result is verified.
	
\begin{figure}[!h]
	\begin{center} 
	    \includegraphics[width=\columnwidth]{chapters/10/7/1/6/figs/quad2}
	\end{center}
\caption{}
\label{fig:10/7/1/6/Fig2}
\end{figure}
	
\item The coordinates are given as
	\begin{align}
	\vec{A} = \myvec{
		4\\
		5\\
		},
	\vec{B} = \myvec{
		7\\
		6\\
		},
	\vec{C} = \myvec{
		4\\
		3\\
		} \text{ and }
	\vec{D} = \myvec{
		1\\
		2\\
		}
	\end{align}
	\begin{align}
		\vec{B} - \vec{A} &= \myvec{7\\6} - \myvec{4\\5} = \myvec{3\\1}\\
		\vec{C} - \vec{B} &= \myvec{4\\3} - \myvec{7\\6} = \myvec{-3\\-3}\\
		\vec{C} - \vec{D} &= \myvec{4\\3} - \myvec{1\\2} = \myvec{3\\1}\\
		\vec{D} - \vec{A} &= \myvec{1\\2} - \myvec{4\\5} = \myvec{-3\\-3}
	\end{align}
	\begin{align}
		\vec{C} - \vec{A} &= \myvec{4\\3} - \myvec{4\\5} = \myvec{0\\-2}\\
		\vec{D} - \vec{B} &= \myvec{1\\2} - \myvec{7\\6} = \myvec{-6\\-4}
	\end{align}
	\begin{align}
		\vec{B}-\vec{A} = \vec{C}-\vec{D} \text{ and } \vec{C}-\vec{B} = \vec{D}-\vec{A},
	\end{align}
	Hence, $ABCD$ is a parallelogram.
	\begin{enumerate}
		\item Now checking if the adjacent sides are orthogonal to each other
	\begin{align}
		(\vec{B}-\vec{A})^\top (\vec{C}-\vec{B}) = \myvec{3&1} \myvec{-3\\-3} = -9-3 = -12
	\end{align}
	Since inner product is not zero so adjacent sides are not orthogonal.

	Hence, we can say that $ABCD$ is neither a rectangle nor a square.

		\item Now checking if the diagonals are orthogonal then it is a Rhombus.
	\begin{align}
		(\vec{C}- \vec{A})^\top (\vec{D}-\vec{B}) = \myvec{0&-2} \myvec{-6\\-4} = 0+8 = 8
	\end{align}
	\end{enumerate}		
	Hence the diagonals are also not orthogonal so we conclude that $ABCD$ is a parallelogram.

	As shown in Figure \ref{fig:10/7/1/6/Fig3} we can see that $ABCD$ forms a parallelogram hence, our theoritical result is verified.

\begin{figure}[!h]
	\begin{center} 
	    \includegraphics[width=\columnwidth]{chapters/10/7/1/6/figs/quad3}
	\end{center}
\caption{}
\label{fig:10/7/1/6/Fig3}
\end{figure}
\end{enumerate}



\item Find the projection of the vector $\hat{i}-\hat{j}$ on the vector $\hat{i}+\hat{j}$.
	\\
		\iffalse
\documentclass[12pt]{chapters/10/7/4/3/figsarticle}
\usepackage{graphicx}
\usepackage[none]{chapters/10/7/4/3/figshyphenat}
\usepackage{graphicx}
\usepackage{listings}
\usepackage[english]{chapters/10/7/4/3/figsbabel}
\usepackage{graphicx}
\usepackage{caption} 
\usepackage{booktabs}
\usepackage{array}
\usepackage{amssymb} % for \because
\usepackage{amsmath}   % for having text in math mode
\usepackage{extarrows} % for Row operations arrows
\usepackage{listings}
\lstset{
  frame=single,
  breaklines=true
}
\usepackage{hyperref}
  
%Following 2 lines were added to remove the blank page at the beginning
\usepackage{atbegshi}% http://ctan.org/pkg/atbegshi
\AtBeginDocument{\AtBeginShipoutNext{\AtBeginShipoutDiscard}}


%New macro definitions
\newcommand{\mydet}[1]{chapters/10/7/4/3/figs\ensuremath{\begin{vmatrix}#1\end{vmatrix}}}
\providecommand{\brak}[1]{chapters/10/7/4/3/figs\ensuremath{\left(#1\right)}}
\newcommand{\solution}{\noindent \textbf{Solution: }}
\newcommand{\myvec}[1]{chapters/10/7/4/3/figs\ensuremath{\begin{pmatrix}#1\end{pmatrix}}}
\providecommand{\norm}[1]{chapters/10/7/4/3/figs\left\lVert#1\right\rVert}
\providecommand{\abs}[1]{chapters/10/7/4/3/figs\left\vert#1\right\vert}
\let\vec\mathbf


\begin{document}

\begin{center}
\title{\textbf{VECTORS}}
\date{\vspace{-5ex}} %Not to print date automatically
\maketitle
\end{center}

\setcounter{page}{1}

\section{10$^{th}$ Maths - Chapter 10}

This is Problem-3 from Exercise 10.3

\begin{enumerate}
\item Find the projection of the vector $\hat{i}-\hat{j}$ on the vector $\hat{i}+\hat{j}$  
\end{enumerate}
\section{SOLUTION}
\fi
\solution
The given points are
\begin{align}
 \vec{A}=\myvec{1\\ -1},
 \vec{B}=\myvec{1\\ 1}
\end{align}
Since
\begin{align}
	\vec{A}^\top \vec{B} &= \myvec{1 &-1} \myvec{1\\ 1}=\myvec{1 \times 1}+\myvec{-1 \times  1}=0
	\\
	\norm {\vec {B}}^2 &= (\vec{B}^\top  \vec{B})=\myvec{1 & 1} \myvec{1\\ 1}= (1 \times  1)+(1 \times  1)=2,
\end{align}
and the project vector is given by 
\begin{align}
	\vec{C} &= 
	\frac{\vec{A}^\top  \vec{B}}{\norm {\vec{B}}}^2 \vec{B}
	&=\frac{0}{2} \myvec{1\\ 1}
	=\myvec{0\\ 0}
\end{align}
This is verfied in Fig.
		\ref{fig:12/10/3/3Figure}.
\begin{figure}[h]
\includegraphics[width=\columnwidth]{chapters/12/10/3/3/figs/vector.png}
\caption{}
		\label{fig:12/10/3/3Figure}
\end{figure}

\item Find the projection of the vector $\hat{i}+3\hat{j}+7\hat{k}$ on the vector $7\hat{i}-\hat{j}+8\hat{k}$.
	\\
	\solution
		\iffalse
\documentclass[12pt]{article}
\usepackage{graphicx}
\usepackage{amsmath}
\usepackage{mathtools}
\usepackage{gensymb}

\newcommand{\mydet}[1]{\ensuremath{\begin{vmatrix}#1\end{vmatrix}}}
\providecommand{\brak}[1]{\ensuremath{\left(#1\right)}}
\providecommand{\norm}[1]{\left\lVert#1\right\rVert}
\newcommand{\solution}{\noindent \textbf{Solution: }}
\newcommand{\myvec}[1]{\ensuremath{\begin{pmatrix}#1\end{pmatrix}}}
\let\vec\mathbf

\begin{document}
\begin{center}
\textbf\large{CHAPTER-7 \\ COORDINATE GEOMETRY}

\end{center}
\section*{Excercise 7.1}

Q6.Name the type of quadilateral formed,if any, by the following points, and give reasons for your answer:
\begin{enumerate}
	\item $\brak{-1,-2}, \brak{1,0}, \brak{-1,2}, \brak{-3,0}$ 
	\item $\brak{-3,5}, \brak{3,1}, \brak{0,3}, \brak{-1,-4}$
	\item $\brak{4,5}, \brak{7,6}, \brak{4,3}, \brak{1,2}$
\end{enumerate}
\solution
\fi
\begin{enumerate}
\item The coordinates are given as
	\begin{align}
	\vec{A} = \myvec{
		-1\\
		-2\\
		},
	\vec{B} = \myvec{
		1\\
		0\\
		},
	\vec{C} = \myvec{
		-1\\
		2\\
		} \text{ and }
	\vec{D} = \myvec{
		-3\\
		0\\
		}
	\end{align}
	\begin{align}
		\vec{B} - \vec{A} &= \myvec{1\\0} - \myvec{-1\\-2} = \myvec{2\\2}\\
		\vec{C} - \vec{B} &= \myvec{-1\\2} - \myvec{1\\0} = \myvec{-2\\2}\\
		\vec{C} - \vec{D} &= \myvec{-1\\2} - \myvec{-3\\0} = \myvec{2\\2}\\
		\vec{D} - \vec{A} &= \myvec{-3\\0} - \myvec{-1\\-2} = \myvec{-2\\2}
	\end{align}
	\begin{align}	
		\vec{C} - \vec{A} &= \myvec{-1\\2} - \myvec{-1\\-2} = \myvec{0\\4}\\
		\vec{D} - \vec{B} &= \myvec{-3\\0} - \myvec{1\\0} = \myvec{-4\\0}
	\end{align}
	\begin{align}	
		\vec{B}-\vec{A} = \vec{C}-\vec{D} \text{ and } \vec{C}-\vec{B} = \vec{D}-\vec{A}.
	\end{align}
	Hence, $ABCD$ is a parallelogram.
	\begin{enumerate}
		\item Now checking if the adjacent sides are orthogonal to each other
	\begin{align}
		(\vec{B}-\vec{A})^\top (\vec{C}-\vec{B}) = \myvec{2&2} \myvec{-2\\2} = -4+4 = 0
	\end{align}
		\item Now checking if the diagonals are also orthogonal then it is a square else a rectangle.
	\end{enumerate}	
	\begin{align}
		(\vec{C}-\vec{A})^\top (\vec{D}-\vec{B}) = \myvec{0&4} \myvec{-4\\0} = 0
	\end{align}
	Hence the diagonals are orthogonal to each other.

	So, we can conclude that $ABCD$ is a square.

	As shown in Figure \ref{fig:10/7/1/6/Fig1} we can see that $ABCD$ is a square hence we can conclude that our theoritical result is verified.
 
\begin{figure}[!h]
	\begin{center} 
	    \includegraphics[width=\columnwidth]{chapters/10/7/1/6/figs/quad1}
	\end{center}
\caption{}
\label{fig:10/7/1/6/Fig1}
\end{figure}

\item The coordinates are given as
	\begin{align}
	\vec{A} = \myvec{
		-3\\
		5\\
		},
	\vec{B} = \myvec{
		3\\
		1\\
		},
	\vec{C} = \myvec{
		0\\
		3\\
		} \text{ and }
	\vec{D} = \myvec{
		-1\\
		-4\\
		}
	\end{align}
	\begin{align}
		\vec{B} - \vec{A} &= \myvec{3\\1} - \myvec{-3\\5} = \myvec{6\\-4}\\
		\vec{C} - \vec{B} &= \myvec{0\\3} - \myvec{3\\1} = \myvec{-3\\2}\\
		\vec{C} - \vec{D} &= \myvec{0\\3} - \myvec{-1\\-4} = \myvec{1\\7}\\
		\vec{D} - \vec{A} &= \myvec{-1\\-4} - \myvec{-3\\5} = \myvec{2\\-9}
	\end{align}
	\begin{align}
		\vec{C} - \vec{A} &= \myvec{0\\3} - \myvec{-3\\5} = \myvec{3\\-2}\\
		\vec{D} - \vec{B} &= \myvec{-1\\-4} - \myvec{3\\1} = \myvec{-4\\-5}
	\end{align}
	\begin{align}
	\vec{B}-\vec{A} \neq \vec{C}-\vec{D} \text{ and } \vec{C}-\vec{B} \neq \vec{D}-\vec{A},
	\end{align}
	Hence, $ABCD$ is not a parallelogram, it can be a irregular quadilateral.
	\begin{enumerate}
		\item Now to check if any three points are collinear,

	if rank of $\myvec{\vec{B}-\vec{A} & \vec{C}-\vec{B}} = 1$ then points are collinear

	Forming the collinearity matrix
	\begin{align}
		\myvec{6&-3\\-4&2} \xleftrightarrow{R_{2}\rightarrow R_{2}+\frac{2}{3}R_{1}}= \myvec{6&-3\\0&0}
	\end{align}
	\end{enumerate}
	Hence, rank = 1

	Since none of the opposite sides are parallel to each other and three points are collinear so these does not form a quadilateral.

	As shown in Figure \ref{fig:10/7/1/6/Fig2} we can see that $ABCD$ does not form a quadilateral and three points are collinear hence, our theoritical result is verified.
	
\begin{figure}[!h]
	\begin{center} 
	    \includegraphics[width=\columnwidth]{chapters/10/7/1/6/figs/quad2}
	\end{center}
\caption{}
\label{fig:10/7/1/6/Fig2}
\end{figure}
	
\item The coordinates are given as
	\begin{align}
	\vec{A} = \myvec{
		4\\
		5\\
		},
	\vec{B} = \myvec{
		7\\
		6\\
		},
	\vec{C} = \myvec{
		4\\
		3\\
		} \text{ and }
	\vec{D} = \myvec{
		1\\
		2\\
		}
	\end{align}
	\begin{align}
		\vec{B} - \vec{A} &= \myvec{7\\6} - \myvec{4\\5} = \myvec{3\\1}\\
		\vec{C} - \vec{B} &= \myvec{4\\3} - \myvec{7\\6} = \myvec{-3\\-3}\\
		\vec{C} - \vec{D} &= \myvec{4\\3} - \myvec{1\\2} = \myvec{3\\1}\\
		\vec{D} - \vec{A} &= \myvec{1\\2} - \myvec{4\\5} = \myvec{-3\\-3}
	\end{align}
	\begin{align}
		\vec{C} - \vec{A} &= \myvec{4\\3} - \myvec{4\\5} = \myvec{0\\-2}\\
		\vec{D} - \vec{B} &= \myvec{1\\2} - \myvec{7\\6} = \myvec{-6\\-4}
	\end{align}
	\begin{align}
		\vec{B}-\vec{A} = \vec{C}-\vec{D} \text{ and } \vec{C}-\vec{B} = \vec{D}-\vec{A},
	\end{align}
	Hence, $ABCD$ is a parallelogram.
	\begin{enumerate}
		\item Now checking if the adjacent sides are orthogonal to each other
	\begin{align}
		(\vec{B}-\vec{A})^\top (\vec{C}-\vec{B}) = \myvec{3&1} \myvec{-3\\-3} = -9-3 = -12
	\end{align}
	Since inner product is not zero so adjacent sides are not orthogonal.

	Hence, we can say that $ABCD$ is neither a rectangle nor a square.

		\item Now checking if the diagonals are orthogonal then it is a Rhombus.
	\begin{align}
		(\vec{C}- \vec{A})^\top (\vec{D}-\vec{B}) = \myvec{0&-2} \myvec{-6\\-4} = 0+8 = 8
	\end{align}
	\end{enumerate}		
	Hence the diagonals are also not orthogonal so we conclude that $ABCD$ is a parallelogram.

	As shown in Figure \ref{fig:10/7/1/6/Fig3} we can see that $ABCD$ forms a parallelogram hence, our theoritical result is verified.

\begin{figure}[!h]
	\begin{center} 
	    \includegraphics[width=\columnwidth]{chapters/10/7/1/6/figs/quad3}
	\end{center}
\caption{}
\label{fig:10/7/1/6/Fig3}
\end{figure}
\end{enumerate}



\item Show that each of the given three vectors is a unit vector: 
 $\frac{1}{7}(2\hat{i}+3\hat{j}+6\hat{k}),\frac{1}{7}(3\hat{i}-6\hat{j}+2\hat{k}),\frac{1}{7}(6\hat{i}+2\hat{j}-3\hat{k}$)
Also,show that they are mutually perpendicular to each other.
	\\
	\solution
		\iffalse
\documentclass[12pt]{article}
\usepackage{graphicx}
\usepackage{amsmath}
\usepackage{mathtools}
\usepackage{gensymb}

\newcommand{\mydet}[1]{\ensuremath{\begin{vmatrix}#1\end{vmatrix}}}
\providecommand{\brak}[1]{\ensuremath{\left(#1\right)}}
\providecommand{\norm}[1]{\left\lVert#1\right\rVert}
\newcommand{\solution}{\noindent \textbf{Solution: }}
\newcommand{\myvec}[1]{\ensuremath{\begin{pmatrix}#1\end{pmatrix}}}
\let\vec\mathbf

\begin{document}
\begin{center}
\textbf\large{CHAPTER-7 \\ COORDINATE GEOMETRY}

\end{center}
\section*{Excercise 7.1}

Q6.Name the type of quadilateral formed,if any, by the following points, and give reasons for your answer:
\begin{enumerate}
	\item $\brak{-1,-2}, \brak{1,0}, \brak{-1,2}, \brak{-3,0}$ 
	\item $\brak{-3,5}, \brak{3,1}, \brak{0,3}, \brak{-1,-4}$
	\item $\brak{4,5}, \brak{7,6}, \brak{4,3}, \brak{1,2}$
\end{enumerate}
\solution
\fi
\begin{enumerate}
\item The coordinates are given as
	\begin{align}
	\vec{A} = \myvec{
		-1\\
		-2\\
		},
	\vec{B} = \myvec{
		1\\
		0\\
		},
	\vec{C} = \myvec{
		-1\\
		2\\
		} \text{ and }
	\vec{D} = \myvec{
		-3\\
		0\\
		}
	\end{align}
	\begin{align}
		\vec{B} - \vec{A} &= \myvec{1\\0} - \myvec{-1\\-2} = \myvec{2\\2}\\
		\vec{C} - \vec{B} &= \myvec{-1\\2} - \myvec{1\\0} = \myvec{-2\\2}\\
		\vec{C} - \vec{D} &= \myvec{-1\\2} - \myvec{-3\\0} = \myvec{2\\2}\\
		\vec{D} - \vec{A} &= \myvec{-3\\0} - \myvec{-1\\-2} = \myvec{-2\\2}
	\end{align}
	\begin{align}	
		\vec{C} - \vec{A} &= \myvec{-1\\2} - \myvec{-1\\-2} = \myvec{0\\4}\\
		\vec{D} - \vec{B} &= \myvec{-3\\0} - \myvec{1\\0} = \myvec{-4\\0}
	\end{align}
	\begin{align}	
		\vec{B}-\vec{A} = \vec{C}-\vec{D} \text{ and } \vec{C}-\vec{B} = \vec{D}-\vec{A}.
	\end{align}
	Hence, $ABCD$ is a parallelogram.
	\begin{enumerate}
		\item Now checking if the adjacent sides are orthogonal to each other
	\begin{align}
		(\vec{B}-\vec{A})^\top (\vec{C}-\vec{B}) = \myvec{2&2} \myvec{-2\\2} = -4+4 = 0
	\end{align}
		\item Now checking if the diagonals are also orthogonal then it is a square else a rectangle.
	\end{enumerate}	
	\begin{align}
		(\vec{C}-\vec{A})^\top (\vec{D}-\vec{B}) = \myvec{0&4} \myvec{-4\\0} = 0
	\end{align}
	Hence the diagonals are orthogonal to each other.

	So, we can conclude that $ABCD$ is a square.

	As shown in Figure \ref{fig:10/7/1/6/Fig1} we can see that $ABCD$ is a square hence we can conclude that our theoritical result is verified.
 
\begin{figure}[!h]
	\begin{center} 
	    \includegraphics[width=\columnwidth]{chapters/10/7/1/6/figs/quad1}
	\end{center}
\caption{}
\label{fig:10/7/1/6/Fig1}
\end{figure}

\item The coordinates are given as
	\begin{align}
	\vec{A} = \myvec{
		-3\\
		5\\
		},
	\vec{B} = \myvec{
		3\\
		1\\
		},
	\vec{C} = \myvec{
		0\\
		3\\
		} \text{ and }
	\vec{D} = \myvec{
		-1\\
		-4\\
		}
	\end{align}
	\begin{align}
		\vec{B} - \vec{A} &= \myvec{3\\1} - \myvec{-3\\5} = \myvec{6\\-4}\\
		\vec{C} - \vec{B} &= \myvec{0\\3} - \myvec{3\\1} = \myvec{-3\\2}\\
		\vec{C} - \vec{D} &= \myvec{0\\3} - \myvec{-1\\-4} = \myvec{1\\7}\\
		\vec{D} - \vec{A} &= \myvec{-1\\-4} - \myvec{-3\\5} = \myvec{2\\-9}
	\end{align}
	\begin{align}
		\vec{C} - \vec{A} &= \myvec{0\\3} - \myvec{-3\\5} = \myvec{3\\-2}\\
		\vec{D} - \vec{B} &= \myvec{-1\\-4} - \myvec{3\\1} = \myvec{-4\\-5}
	\end{align}
	\begin{align}
	\vec{B}-\vec{A} \neq \vec{C}-\vec{D} \text{ and } \vec{C}-\vec{B} \neq \vec{D}-\vec{A},
	\end{align}
	Hence, $ABCD$ is not a parallelogram, it can be a irregular quadilateral.
	\begin{enumerate}
		\item Now to check if any three points are collinear,

	if rank of $\myvec{\vec{B}-\vec{A} & \vec{C}-\vec{B}} = 1$ then points are collinear

	Forming the collinearity matrix
	\begin{align}
		\myvec{6&-3\\-4&2} \xleftrightarrow{R_{2}\rightarrow R_{2}+\frac{2}{3}R_{1}}= \myvec{6&-3\\0&0}
	\end{align}
	\end{enumerate}
	Hence, rank = 1

	Since none of the opposite sides are parallel to each other and three points are collinear so these does not form a quadilateral.

	As shown in Figure \ref{fig:10/7/1/6/Fig2} we can see that $ABCD$ does not form a quadilateral and three points are collinear hence, our theoritical result is verified.
	
\begin{figure}[!h]
	\begin{center} 
	    \includegraphics[width=\columnwidth]{chapters/10/7/1/6/figs/quad2}
	\end{center}
\caption{}
\label{fig:10/7/1/6/Fig2}
\end{figure}
	
\item The coordinates are given as
	\begin{align}
	\vec{A} = \myvec{
		4\\
		5\\
		},
	\vec{B} = \myvec{
		7\\
		6\\
		},
	\vec{C} = \myvec{
		4\\
		3\\
		} \text{ and }
	\vec{D} = \myvec{
		1\\
		2\\
		}
	\end{align}
	\begin{align}
		\vec{B} - \vec{A} &= \myvec{7\\6} - \myvec{4\\5} = \myvec{3\\1}\\
		\vec{C} - \vec{B} &= \myvec{4\\3} - \myvec{7\\6} = \myvec{-3\\-3}\\
		\vec{C} - \vec{D} &= \myvec{4\\3} - \myvec{1\\2} = \myvec{3\\1}\\
		\vec{D} - \vec{A} &= \myvec{1\\2} - \myvec{4\\5} = \myvec{-3\\-3}
	\end{align}
	\begin{align}
		\vec{C} - \vec{A} &= \myvec{4\\3} - \myvec{4\\5} = \myvec{0\\-2}\\
		\vec{D} - \vec{B} &= \myvec{1\\2} - \myvec{7\\6} = \myvec{-6\\-4}
	\end{align}
	\begin{align}
		\vec{B}-\vec{A} = \vec{C}-\vec{D} \text{ and } \vec{C}-\vec{B} = \vec{D}-\vec{A},
	\end{align}
	Hence, $ABCD$ is a parallelogram.
	\begin{enumerate}
		\item Now checking if the adjacent sides are orthogonal to each other
	\begin{align}
		(\vec{B}-\vec{A})^\top (\vec{C}-\vec{B}) = \myvec{3&1} \myvec{-3\\-3} = -9-3 = -12
	\end{align}
	Since inner product is not zero so adjacent sides are not orthogonal.

	Hence, we can say that $ABCD$ is neither a rectangle nor a square.

		\item Now checking if the diagonals are orthogonal then it is a Rhombus.
	\begin{align}
		(\vec{C}- \vec{A})^\top (\vec{D}-\vec{B}) = \myvec{0&-2} \myvec{-6\\-4} = 0+8 = 8
	\end{align}
	\end{enumerate}		
	Hence the diagonals are also not orthogonal so we conclude that $ABCD$ is a parallelogram.

	As shown in Figure \ref{fig:10/7/1/6/Fig3} we can see that $ABCD$ forms a parallelogram hence, our theoritical result is verified.

\begin{figure}[!h]
	\begin{center} 
	    \includegraphics[width=\columnwidth]{chapters/10/7/1/6/figs/quad3}
	\end{center}
\caption{}
\label{fig:10/7/1/6/Fig3}
\end{figure}
\end{enumerate}



\item If $\overrightarrow {a}=2\hat{i}+2\hat{j}3\hat{k},\overrightarrow {b}=\hat{-i}+2\hat{j}+\hat{k}$ and $\overrightarrow {c}=3\hat{i}+\hat{j}$ are such that $\overrightarrow {a}+\lambda\overrightarrow {b}$ is perpendicular to $\overrightarrow {c}$,then find the value of $\lambda$.
	\\
		\iffalse
\documentclass[12pt]{article}
\usepackage{graphicx}
\usepackage{amsmath}
\usepackage{mathtools}
\usepackage{gensymb}

\newcommand{\mydet}[1]{\ensuremath{\begin{vmatrix}#1\end{vmatrix}}}
\providecommand{\brak}[1]{\ensuremath{\left(#1\right)}}
\providecommand{\norm}[1]{\left\lVert#1\right\rVert}
\newcommand{\solution}{\noindent \textbf{Solution: }}
\newcommand{\myvec}[1]{\ensuremath{\begin{pmatrix}#1\end{pmatrix}}}
\let\vec\mathbf

\begin{document}
\begin{center}
\textbf\large{CHAPTER-10 \\ VECTOR ALGEBRA}

\end{center}
\section*{Excercise 10.3}

Q10.If $\vec{a} = 2\hat{i}+2\hat{j}+3\hat{k}, \vec{b} = -\hat{i}+2\hat{j}+\hat{k} \text{ and } \vec{c} = 3\hat{i}+\hat{j}$ are such that $\vec{a}+\lambda \vec{b}$ is perpendicular to $\vec{c}$, then find the value of $\lambda$.
\fi
\solution
Given that
\begin{align}
	(\vec{a}+\lambda \vec{b})^{\top} \vec{c} &= 0\\
\implies \vec{a}^{\top}\vec{c}+\lambda \vec{b}^{\top}\vec{c}&=0\\
\implies 	\lambda \vec{b}^{\top}\vec{c}&=-\vec{a}^{\top}\vec{c}\\
\implies 	\lambda(\vec{b}^{\top}\vec{c})(\vec{b}^{\top}\vec{c})^{-1}&=-(\vec{a}^{\top}\vec{c})(\vec{b}^{\top}\vec{c})^{-1}\\
\implies 	\lambda&=-(\vec{a}^{\top}\vec{c})(\vec{b}^{\top}\vec{c})^{-1}
\end{align}
Now substituting the values
\begin{align}
	\vec{a}^{\top}\vec{c}&=\myvec{2&2&3} \myvec{3\\1\\0} = 8\\
	\vec{b}^{\top}\vec{c}&=\myvec{-1&2&1} \myvec{3\\1\\0} = -1,
\end{align}
\begin{align}
	\lambda&=-(\vec{a}^{\top}\vec{c})(\vec{b}^{\top}\vec{c})^{-1}\\
	&=-(8)(-1)^{-1}\\
	&=8
\end{align}



\item Show that $\abs {\overrightarrow {a}}\overrightarrow {b}+\abs{\overrightarrow {b}}\overrightarrow {a}$ is perpendicular to $\abs{\overrightarrow {a}} \overrightarrow {b}-\abs{\overrightarrow {b}} \overrightarrow {a}$, for any two nonzero vectors $\overrightarrow {a}$ and $\overrightarrow {b}$.
	\\
	\solution
		\iffalse
\documentclass[12pt]{article}
\usepackage{graphicx}
\usepackage{amsmath}
\usepackage{mathtools}
\usepackage{gensymb}

\newcommand{\mydet}[1]{\ensuremath{\begin{vmatrix}#1\end{vmatrix}}}
\providecommand{\brak}[1]{\ensuremath{\left(#1\right)}}
\providecommand{\norm}[1]{\left\lVert#1\right\rVert}
\newcommand{\solution}{\noindent \textbf{Solution: }}
\newcommand{\myvec}[1]{\ensuremath{\begin{pmatrix}#1\end{pmatrix}}}
\let\vec\mathbf

\begin{document}
\begin{center}
\textbf\large{CHAPTER-7 \\ COORDINATE GEOMETRY}

\end{center}
\section*{Excercise 7.1}

Q6.Name the type of quadilateral formed,if any, by the following points, and give reasons for your answer:
\begin{enumerate}
	\item $\brak{-1,-2}, \brak{1,0}, \brak{-1,2}, \brak{-3,0}$ 
	\item $\brak{-3,5}, \brak{3,1}, \brak{0,3}, \brak{-1,-4}$
	\item $\brak{4,5}, \brak{7,6}, \brak{4,3}, \brak{1,2}$
\end{enumerate}
\solution
\fi
\begin{enumerate}
\item The coordinates are given as
	\begin{align}
	\vec{A} = \myvec{
		-1\\
		-2\\
		},
	\vec{B} = \myvec{
		1\\
		0\\
		},
	\vec{C} = \myvec{
		-1\\
		2\\
		} \text{ and }
	\vec{D} = \myvec{
		-3\\
		0\\
		}
	\end{align}
	\begin{align}
		\vec{B} - \vec{A} &= \myvec{1\\0} - \myvec{-1\\-2} = \myvec{2\\2}\\
		\vec{C} - \vec{B} &= \myvec{-1\\2} - \myvec{1\\0} = \myvec{-2\\2}\\
		\vec{C} - \vec{D} &= \myvec{-1\\2} - \myvec{-3\\0} = \myvec{2\\2}\\
		\vec{D} - \vec{A} &= \myvec{-3\\0} - \myvec{-1\\-2} = \myvec{-2\\2}
	\end{align}
	\begin{align}	
		\vec{C} - \vec{A} &= \myvec{-1\\2} - \myvec{-1\\-2} = \myvec{0\\4}\\
		\vec{D} - \vec{B} &= \myvec{-3\\0} - \myvec{1\\0} = \myvec{-4\\0}
	\end{align}
	\begin{align}	
		\vec{B}-\vec{A} = \vec{C}-\vec{D} \text{ and } \vec{C}-\vec{B} = \vec{D}-\vec{A}.
	\end{align}
	Hence, $ABCD$ is a parallelogram.
	\begin{enumerate}
		\item Now checking if the adjacent sides are orthogonal to each other
	\begin{align}
		(\vec{B}-\vec{A})^\top (\vec{C}-\vec{B}) = \myvec{2&2} \myvec{-2\\2} = -4+4 = 0
	\end{align}
		\item Now checking if the diagonals are also orthogonal then it is a square else a rectangle.
	\end{enumerate}	
	\begin{align}
		(\vec{C}-\vec{A})^\top (\vec{D}-\vec{B}) = \myvec{0&4} \myvec{-4\\0} = 0
	\end{align}
	Hence the diagonals are orthogonal to each other.

	So, we can conclude that $ABCD$ is a square.

	As shown in Figure \ref{fig:10/7/1/6/Fig1} we can see that $ABCD$ is a square hence we can conclude that our theoritical result is verified.
 
\begin{figure}[!h]
	\begin{center} 
	    \includegraphics[width=\columnwidth]{chapters/10/7/1/6/figs/quad1}
	\end{center}
\caption{}
\label{fig:10/7/1/6/Fig1}
\end{figure}

\item The coordinates are given as
	\begin{align}
	\vec{A} = \myvec{
		-3\\
		5\\
		},
	\vec{B} = \myvec{
		3\\
		1\\
		},
	\vec{C} = \myvec{
		0\\
		3\\
		} \text{ and }
	\vec{D} = \myvec{
		-1\\
		-4\\
		}
	\end{align}
	\begin{align}
		\vec{B} - \vec{A} &= \myvec{3\\1} - \myvec{-3\\5} = \myvec{6\\-4}\\
		\vec{C} - \vec{B} &= \myvec{0\\3} - \myvec{3\\1} = \myvec{-3\\2}\\
		\vec{C} - \vec{D} &= \myvec{0\\3} - \myvec{-1\\-4} = \myvec{1\\7}\\
		\vec{D} - \vec{A} &= \myvec{-1\\-4} - \myvec{-3\\5} = \myvec{2\\-9}
	\end{align}
	\begin{align}
		\vec{C} - \vec{A} &= \myvec{0\\3} - \myvec{-3\\5} = \myvec{3\\-2}\\
		\vec{D} - \vec{B} &= \myvec{-1\\-4} - \myvec{3\\1} = \myvec{-4\\-5}
	\end{align}
	\begin{align}
	\vec{B}-\vec{A} \neq \vec{C}-\vec{D} \text{ and } \vec{C}-\vec{B} \neq \vec{D}-\vec{A},
	\end{align}
	Hence, $ABCD$ is not a parallelogram, it can be a irregular quadilateral.
	\begin{enumerate}
		\item Now to check if any three points are collinear,

	if rank of $\myvec{\vec{B}-\vec{A} & \vec{C}-\vec{B}} = 1$ then points are collinear

	Forming the collinearity matrix
	\begin{align}
		\myvec{6&-3\\-4&2} \xleftrightarrow{R_{2}\rightarrow R_{2}+\frac{2}{3}R_{1}}= \myvec{6&-3\\0&0}
	\end{align}
	\end{enumerate}
	Hence, rank = 1

	Since none of the opposite sides are parallel to each other and three points are collinear so these does not form a quadilateral.

	As shown in Figure \ref{fig:10/7/1/6/Fig2} we can see that $ABCD$ does not form a quadilateral and three points are collinear hence, our theoritical result is verified.
	
\begin{figure}[!h]
	\begin{center} 
	    \includegraphics[width=\columnwidth]{chapters/10/7/1/6/figs/quad2}
	\end{center}
\caption{}
\label{fig:10/7/1/6/Fig2}
\end{figure}
	
\item The coordinates are given as
	\begin{align}
	\vec{A} = \myvec{
		4\\
		5\\
		},
	\vec{B} = \myvec{
		7\\
		6\\
		},
	\vec{C} = \myvec{
		4\\
		3\\
		} \text{ and }
	\vec{D} = \myvec{
		1\\
		2\\
		}
	\end{align}
	\begin{align}
		\vec{B} - \vec{A} &= \myvec{7\\6} - \myvec{4\\5} = \myvec{3\\1}\\
		\vec{C} - \vec{B} &= \myvec{4\\3} - \myvec{7\\6} = \myvec{-3\\-3}\\
		\vec{C} - \vec{D} &= \myvec{4\\3} - \myvec{1\\2} = \myvec{3\\1}\\
		\vec{D} - \vec{A} &= \myvec{1\\2} - \myvec{4\\5} = \myvec{-3\\-3}
	\end{align}
	\begin{align}
		\vec{C} - \vec{A} &= \myvec{4\\3} - \myvec{4\\5} = \myvec{0\\-2}\\
		\vec{D} - \vec{B} &= \myvec{1\\2} - \myvec{7\\6} = \myvec{-6\\-4}
	\end{align}
	\begin{align}
		\vec{B}-\vec{A} = \vec{C}-\vec{D} \text{ and } \vec{C}-\vec{B} = \vec{D}-\vec{A},
	\end{align}
	Hence, $ABCD$ is a parallelogram.
	\begin{enumerate}
		\item Now checking if the adjacent sides are orthogonal to each other
	\begin{align}
		(\vec{B}-\vec{A})^\top (\vec{C}-\vec{B}) = \myvec{3&1} \myvec{-3\\-3} = -9-3 = -12
	\end{align}
	Since inner product is not zero so adjacent sides are not orthogonal.

	Hence, we can say that $ABCD$ is neither a rectangle nor a square.

		\item Now checking if the diagonals are orthogonal then it is a Rhombus.
	\begin{align}
		(\vec{C}- \vec{A})^\top (\vec{D}-\vec{B}) = \myvec{0&-2} \myvec{-6\\-4} = 0+8 = 8
	\end{align}
	\end{enumerate}		
	Hence the diagonals are also not orthogonal so we conclude that $ABCD$ is a parallelogram.

	As shown in Figure \ref{fig:10/7/1/6/Fig3} we can see that $ABCD$ forms a parallelogram hence, our theoritical result is verified.

\begin{figure}[!h]
	\begin{center} 
	    \includegraphics[width=\columnwidth]{chapters/10/7/1/6/figs/quad3}
	\end{center}
\caption{}
\label{fig:10/7/1/6/Fig3}
\end{figure}
\end{enumerate}



\item If $\overrightarrow {a}.\overrightarrow {a}$=0 and $\overrightarrow {a}.\overrightarrow {b}$=0, then what can be conculded about the vector $\overrightarrow {b}$?
\item If $\overrightarrow {a},\overrightarrow {b},\overrightarrow {c}$ are unit vectors such that $\overrightarrow {a}+\overrightarrow {b}+\overrightarrow {c}=\overrightarrow {0}$, find the value of $\overrightarrow {a}.\overrightarrow {b}+\overrightarrow {b}.\overrightarrow {c}+\overrightarrow {c}.\overrightarrow {a}$.
	\\
	\solution
		\iffalse
\documentclass[12pt]{article}
\usepackage{graphicx}
%\documentclass[journal,12pt,twocolumn]{IEEEtran}
\usepackage[none]{hyphenat}
\usepackage{graphicx}
\usepackage{listings}
\usepackage[english]{babel}
\usepackage{graphicx}
\usepackage{caption} 
\usepackage{hyperref}
\usepackage{booktabs}
\usepackage{commath}
\usepackage{gensymb}
\usepackage{array}
\usepackage{amsmath}   % for having text in math mode
\usepackage{listings}
\let\vec\mathbf
\lstset{
  frame=single,
  breaklines=true
}
  
%Following 2 lines were added to remove the blank page at the beginning
\usepackage{atbegshi}% http://ctan.org/pkg/atbegshi
\AtBeginDocument{\AtBeginShipoutNext{\AtBeginShipoutDiscard}}
%
%New macro definitions
\newcommand{\mydet}[1]{\ensuremath{\begin{vmatrix}#1\end{vmatrix}}}
\providecommand{\brak}[1]{\ensuremath{\left(#1\right)}}
\providecommand{\norm}[1]{\left\lVert#1\right\rVert}
\newcommand{\solution}{\noindent \textbf{Solution: }}
\newcommand{\myvec}[1]{\ensuremath{\begin{pmatrix}#1\end{pmatrix}}}
\let\vec\mathbf
\begin{document}
\begin{center}
\title{\textbf{Vector Algebra}}
\date{\vspace{-5ex}} %Not to print date automatically
\maketitle
\end{center}
\setcounter{page}{1}
\section*{CHAPTER 10 - VECTOR ALGEBRA}
\section*{Excercise 10.3}
\solution 
\begin{enumerate}
\item If $\overrightarrow{a},\overrightarrow{b},\overrightarrow{c}$ are unit vectors such that $\overrightarrow{a}+\overrightarrow{b}+\overrightarrow{c}=0$, find the value of $\overrightarrow{a}.\overrightarrow{b}+\overrightarrow{b}.\overrightarrow{c}+\overrightarrow{c}.\overrightarrow{a}$.  
\section{Solution}
The given vectors $\vec{a},\vec{b}$ and $\vec{c}$ are unit vectors. Since the given vectors $\vec{a},\vec{b},\vec{c}$ are unit vector hence $\vec{a}=\vec{b}=\vec{c}$ which is equal to 1.
        \begin{align}
\norm{\vec{a}} &=\sqrt{1^2}=1\\ \norm{\vec{b}}&=\sqrt{1^2}=1\\ \norm{\vec{c}}&=\sqrt{1^2}=1
        \end{align}
The Given equation is 
        \begin{align}
\vec{a}+\vec{b}+\vec{c}=0
\end{align}      
Squaring on both sides,
\fi
\begin{align}
	\norm{{\vec{a}}+{\vec{b}}+{\vec{c}}}^2&=0
	\\
	\implies{\norm{\vec{a}}}^2+{\norm{\vec{b}}}^2+{\norm{\vec{c}}}^2+2({{\vec{a}^\top}{\vec{b}}+{\vec{b}^\top}{\vec{c}}+{\vec{c}^\top}{\vec{a}}})&=0\\
	\implies3+2({{\vec{a}^\top}{\vec{b}}+{\vec{b}^\top}{\vec{c}}+{\vec{c}^\top}{\vec{a}}})&=0\\
	\implies{\vec{a}^\top}{\vec{b}}+{\vec{b}^\top}{\vec{c}}+{\vec{c}^\top}\vec{a}&=-\frac{3}{2}
\end{align}

\item If either vector $\overrightarrow {a}=0$ or $\overrightarrow {b}=0$, then $\overrightarrow {a}.\overrightarrow {b}$=0. But the converse need not be true. Justify your answer with an example.
	\\
	\solution
		\iffalse
\documentclass[12pt]{article}
\usepackage{graphicx}
\usepackage{amsmath}
\usepackage{mathtools}
\usepackage{gensymb}

\newcommand{\mydet}[1]{\ensuremath{\begin{vmatrix}#1\end{vmatrix}}}
\providecommand{\brak}[1]{\ensuremath{\left(#1\right)}}
\providecommand{\norm}[1]{\left\lVert#1\right\rVert}
\newcommand{\solution}{\noindent \textbf{Solution: }}
\newcommand{\myvec}[1]{\ensuremath{\begin{pmatrix}#1\end{pmatrix}}}
\let\vec\mathbf

\begin{document}
\begin{center}
\textbf\large{CHAPTER-7 \\ COORDINATE GEOMETRY}

\end{center}
\section*{Excercise 7.1}

Q6.Name the type of quadilateral formed,if any, by the following points, and give reasons for your answer:
\begin{enumerate}
	\item $\brak{-1,-2}, \brak{1,0}, \brak{-1,2}, \brak{-3,0}$ 
	\item $\brak{-3,5}, \brak{3,1}, \brak{0,3}, \brak{-1,-4}$
	\item $\brak{4,5}, \brak{7,6}, \brak{4,3}, \brak{1,2}$
\end{enumerate}
\solution
\fi
\begin{enumerate}
\item The coordinates are given as
	\begin{align}
	\vec{A} = \myvec{
		-1\\
		-2\\
		},
	\vec{B} = \myvec{
		1\\
		0\\
		},
	\vec{C} = \myvec{
		-1\\
		2\\
		} \text{ and }
	\vec{D} = \myvec{
		-3\\
		0\\
		}
	\end{align}
	\begin{align}
		\vec{B} - \vec{A} &= \myvec{1\\0} - \myvec{-1\\-2} = \myvec{2\\2}\\
		\vec{C} - \vec{B} &= \myvec{-1\\2} - \myvec{1\\0} = \myvec{-2\\2}\\
		\vec{C} - \vec{D} &= \myvec{-1\\2} - \myvec{-3\\0} = \myvec{2\\2}\\
		\vec{D} - \vec{A} &= \myvec{-3\\0} - \myvec{-1\\-2} = \myvec{-2\\2}
	\end{align}
	\begin{align}	
		\vec{C} - \vec{A} &= \myvec{-1\\2} - \myvec{-1\\-2} = \myvec{0\\4}\\
		\vec{D} - \vec{B} &= \myvec{-3\\0} - \myvec{1\\0} = \myvec{-4\\0}
	\end{align}
	\begin{align}	
		\vec{B}-\vec{A} = \vec{C}-\vec{D} \text{ and } \vec{C}-\vec{B} = \vec{D}-\vec{A}.
	\end{align}
	Hence, $ABCD$ is a parallelogram.
	\begin{enumerate}
		\item Now checking if the adjacent sides are orthogonal to each other
	\begin{align}
		(\vec{B}-\vec{A})^\top (\vec{C}-\vec{B}) = \myvec{2&2} \myvec{-2\\2} = -4+4 = 0
	\end{align}
		\item Now checking if the diagonals are also orthogonal then it is a square else a rectangle.
	\end{enumerate}	
	\begin{align}
		(\vec{C}-\vec{A})^\top (\vec{D}-\vec{B}) = \myvec{0&4} \myvec{-4\\0} = 0
	\end{align}
	Hence the diagonals are orthogonal to each other.

	So, we can conclude that $ABCD$ is a square.

	As shown in Figure \ref{fig:10/7/1/6/Fig1} we can see that $ABCD$ is a square hence we can conclude that our theoritical result is verified.
 
\begin{figure}[!h]
	\begin{center} 
	    \includegraphics[width=\columnwidth]{chapters/10/7/1/6/figs/quad1}
	\end{center}
\caption{}
\label{fig:10/7/1/6/Fig1}
\end{figure}

\item The coordinates are given as
	\begin{align}
	\vec{A} = \myvec{
		-3\\
		5\\
		},
	\vec{B} = \myvec{
		3\\
		1\\
		},
	\vec{C} = \myvec{
		0\\
		3\\
		} \text{ and }
	\vec{D} = \myvec{
		-1\\
		-4\\
		}
	\end{align}
	\begin{align}
		\vec{B} - \vec{A} &= \myvec{3\\1} - \myvec{-3\\5} = \myvec{6\\-4}\\
		\vec{C} - \vec{B} &= \myvec{0\\3} - \myvec{3\\1} = \myvec{-3\\2}\\
		\vec{C} - \vec{D} &= \myvec{0\\3} - \myvec{-1\\-4} = \myvec{1\\7}\\
		\vec{D} - \vec{A} &= \myvec{-1\\-4} - \myvec{-3\\5} = \myvec{2\\-9}
	\end{align}
	\begin{align}
		\vec{C} - \vec{A} &= \myvec{0\\3} - \myvec{-3\\5} = \myvec{3\\-2}\\
		\vec{D} - \vec{B} &= \myvec{-1\\-4} - \myvec{3\\1} = \myvec{-4\\-5}
	\end{align}
	\begin{align}
	\vec{B}-\vec{A} \neq \vec{C}-\vec{D} \text{ and } \vec{C}-\vec{B} \neq \vec{D}-\vec{A},
	\end{align}
	Hence, $ABCD$ is not a parallelogram, it can be a irregular quadilateral.
	\begin{enumerate}
		\item Now to check if any three points are collinear,

	if rank of $\myvec{\vec{B}-\vec{A} & \vec{C}-\vec{B}} = 1$ then points are collinear

	Forming the collinearity matrix
	\begin{align}
		\myvec{6&-3\\-4&2} \xleftrightarrow{R_{2}\rightarrow R_{2}+\frac{2}{3}R_{1}}= \myvec{6&-3\\0&0}
	\end{align}
	\end{enumerate}
	Hence, rank = 1

	Since none of the opposite sides are parallel to each other and three points are collinear so these does not form a quadilateral.

	As shown in Figure \ref{fig:10/7/1/6/Fig2} we can see that $ABCD$ does not form a quadilateral and three points are collinear hence, our theoritical result is verified.
	
\begin{figure}[!h]
	\begin{center} 
	    \includegraphics[width=\columnwidth]{chapters/10/7/1/6/figs/quad2}
	\end{center}
\caption{}
\label{fig:10/7/1/6/Fig2}
\end{figure}
	
\item The coordinates are given as
	\begin{align}
	\vec{A} = \myvec{
		4\\
		5\\
		},
	\vec{B} = \myvec{
		7\\
		6\\
		},
	\vec{C} = \myvec{
		4\\
		3\\
		} \text{ and }
	\vec{D} = \myvec{
		1\\
		2\\
		}
	\end{align}
	\begin{align}
		\vec{B} - \vec{A} &= \myvec{7\\6} - \myvec{4\\5} = \myvec{3\\1}\\
		\vec{C} - \vec{B} &= \myvec{4\\3} - \myvec{7\\6} = \myvec{-3\\-3}\\
		\vec{C} - \vec{D} &= \myvec{4\\3} - \myvec{1\\2} = \myvec{3\\1}\\
		\vec{D} - \vec{A} &= \myvec{1\\2} - \myvec{4\\5} = \myvec{-3\\-3}
	\end{align}
	\begin{align}
		\vec{C} - \vec{A} &= \myvec{4\\3} - \myvec{4\\5} = \myvec{0\\-2}\\
		\vec{D} - \vec{B} &= \myvec{1\\2} - \myvec{7\\6} = \myvec{-6\\-4}
	\end{align}
	\begin{align}
		\vec{B}-\vec{A} = \vec{C}-\vec{D} \text{ and } \vec{C}-\vec{B} = \vec{D}-\vec{A},
	\end{align}
	Hence, $ABCD$ is a parallelogram.
	\begin{enumerate}
		\item Now checking if the adjacent sides are orthogonal to each other
	\begin{align}
		(\vec{B}-\vec{A})^\top (\vec{C}-\vec{B}) = \myvec{3&1} \myvec{-3\\-3} = -9-3 = -12
	\end{align}
	Since inner product is not zero so adjacent sides are not orthogonal.

	Hence, we can say that $ABCD$ is neither a rectangle nor a square.

		\item Now checking if the diagonals are orthogonal then it is a Rhombus.
	\begin{align}
		(\vec{C}- \vec{A})^\top (\vec{D}-\vec{B}) = \myvec{0&-2} \myvec{-6\\-4} = 0+8 = 8
	\end{align}
	\end{enumerate}		
	Hence the diagonals are also not orthogonal so we conclude that $ABCD$ is a parallelogram.

	As shown in Figure \ref{fig:10/7/1/6/Fig3} we can see that $ABCD$ forms a parallelogram hence, our theoritical result is verified.

\begin{figure}[!h]
	\begin{center} 
	    \includegraphics[width=\columnwidth]{chapters/10/7/1/6/figs/quad3}
	\end{center}
\caption{}
\label{fig:10/7/1/6/Fig3}
\end{figure}
\end{enumerate}



\item Show that the vectors $2\hat{i}-\hat{j}+\hat{k},\hat{i}-3\hat{j}-5\hat{k}$ and  $3\hat{i}-4\hat{j}-4\hat{k}$ from the vertices of a right angled triangle.
	\\
	\solution
		\iffalse
\documentclass[journal,12pt,twocolumn]{IEEEtran}
\usepackage{setspace}
\usepackage{gensymb}
\singlespacing
\usepackage[cmex10]{amsmath}
\usepackage{amsthm}
\usepackage{mathrsfs}
\usepackage{txfonts}
\usepackage{stfloats}
\usepackage{bm}
\usepackage{cite}
\usepackage{cases}
\usepackage{subfig}
\usepackage{longtable}
\usepackage{multirow}
\usepackage{enumitem}
\usepackage{mathtools}
\usepackage{steinmetz}
\usepackage{tikz}
\usepackage{circuitikz}
\usepackage{verbatim}
\usepackage{tfrupee}
\usepackage[breaklinks=true]{hyperref}
\usepackage{tkz-euclide}
\usetikzlibrary{calc,math}
\usepackage{listings}
    \usepackage{color}                                            %%
    \usepackage{array}                                            %%
    \usepackage{longtable}                                        %%
    \usepackage{calc}                                             %%
    \usepackage{multirow}                                         %%
    \usepackage{hhline}                                           %%
    \usepackage{ifthen}                                           %%
  %optionally (for landscape tables embedded in another document): %%
    \usepackage{lscape}     
\usepackage{multicol}
\usepackage{chngcntr}
\DeclareMathOperator*{\Res}{Res}
\renewcommand\thesection{\arabic{section}}
\renewcommand\thesubsection{\thesection.\arabic{subsection}}
\renewcommand\thesubsubsection{\thesubsection.\arabic{subsubsection}}

\renewcommand\thesectiondis{\arabic{section}}
\renewcommand\thesubsectiondis{\thesectiondis.\arabic{subsection}}
\renewcommand\thesubsubsectiondis{\thesubsectiondis.\arabic{subsubsection}}

% correct bad hyphenation here
\hyphenation{op-tical net-works semi-conduc-tor}
\def\inputGnumericTable{}                                 %%

\lstset{
frame=single, 
breaklines=true,
columns=fullflexible
}

\begin{document}


\newtheorem{theorem}{Theorem}[section]
\newtheorem{problem}{Problem}
\newtheorem{proposition}{Proposition}[section]
\newtheorem{lemma}{Lemma}[section]
\newtheorem{corollary}[theorem]{Corollary}
\newtheorem{example}{Example}[section]
\newtheorem{definition}[problem]{Definition}
\newcommand{\BEQA}{\begin{eqnarray}}
\newcommand{\EEQA}{\end{eqnarray}}
\newcommand{\define}{\stackrel{\triangle}{=}}

\bibliographystyle{IEEEtran}
\providecommand{\mbf}{\mathbf}
\providecommand{\pr}[1]{\ensuremath{\Pr\left(#1\right)}}
\providecommand{\qfunc}[1]{\ensuremath{Q\left(#1\right)}}
\providecommand{\sbrak}[1]{\ensuremath{{}\left[#1\right]}}
\providecommand{\lsbrak}[1]{\ensuremath{{}\left[#1\right.}}
\providecommand{\rsbrak}[1]{\ensuremath{{}\left.#1\right]}}
\providecommand{\brak}[1]{\ensuremath{\left(#1\right)}}
\providecommand{\lbrak}[1]{\ensuremath{\left(#1\right.}}
\providecommand{\rbrak}[1]{\ensuremath{\left.#1\right)}}
\providecommand{\cbrak}[1]{\ensuremath{\left\{#1\right\}}}
\providecommand{\lcbrak}[1]{\ensuremath{\left\{#1\right.}}
\providecommand{\rcbrak}[1]{\ensuremath{\left.#1\right\}}}
\theoremstyle{remark}
\newtheorem{rem}{Remark}
\newcommand{\sgn}{\mathop{\mathrm{sgn}}}
\providecommand{\abs}[1]{\left\vert#1\right\vert}
\providecommand{\res}[1]{\Res\displaylimits_{#1}} 
\providecommand{\norm}[1]{\left\lVert#1\right\rVert}
\providecommand{\mtx}[1]{\mathbf{#1}}
\providecommand{\mean}[1]{E\left[ #1 \right]}
\providecommand{\fourier}{\overset{\mathcal{F}}{ \rightleftharpoons}}
\providecommand{\system}{\overset{\mathcal{H}}{ \longleftrightarrow}}
\newcommand{\solution}{\noindent \textbf{Solution: }}
\newcommand{\cosec}{\,\text{cosec}\,}
\providecommand{\dec}[2]{\ensuremath{\overset{#1}{\underset{#2}{\gtrless}}}}
\newcommand{\myvec}[1]{\ensuremath{\begin{pmatrix}#1\end{pmatrix}}}
\newcommand{\mydet}[1]{\ensuremath{\begin{vmatrix}#1\end{vmatrix}}}
\numberwithin{equation}{subsection}
\makeatletter
\@addtoreset{figure}{problem}
\makeatother

\let\StandardTheFigure\thefigure
\let\vec\mathbf
\renewcommand{\thefigure}{\theproblem}



\def\putbox#1#2#3{\makebox[0in][l]{\makebox[#1][l]{}\raisebox{\baselineskip}[0in][0in]{\raisebox{#2}[0in][0in]{#3}}}}
     \def\rightbox#1{\makebox[0in][r]{#1}}
     \def\centbox#1{\makebox[0in]{#1}}
     \def\topbox#1{\raisebox{-\baselineskip}[0in][0in]{#1}}
     \def\midbox#1{\raisebox{-0.5\baselineskip}[0in][0in]{#1}}

\vspace{3cm}


\title{Assignment 1}
\author{Jaswanth Chowdary Madala}





% make the title area
\maketitle

\newpage

%\tableofcontents

\bigskip

\renewcommand{\thefigure}{\theenumi}
\renewcommand{\thetable}{\theenumi}


\begin{enumerate}

\item Show that the vectors $2\hat{i}-\hat{j}+\hat{k}$, $\hat{i}-3\hat{j}-5\hat{k}$ and $3\hat{i}-4\hat{j}-4\hat{k}$ form the vertices of a right angled triangle.
\fi
Let
\begin{align}
\vec{A} = \myvec{2\\-1\\1}, \, \vec{B} = \myvec{1\\-3\\-5}, \, \vec{C}=\myvec{3\\-4\\-4} 
\end{align}
 Form the matrix 
\begin{align}
\myvec{\vec{A}&\vec{B}&\vec{C}} = \myvec{2&1&3\\-1&-3&-4\\1&-5&-4}\\
\xleftrightarrow [R_2 \leftarrow R_2+\frac{1}{2}R_1]{R_3 \leftarrow R_3-\frac{1}{2}R_1} \\
\myvec{2&1&3\\ \\0&-\dfrac{5}{2}&-\dfrac{5}{2}\\\\0&-\dfrac{11}{2}&-\dfrac{11}{2}}\\
\xleftrightarrow[]{R_3 \leftarrow R_3-\frac{11}{5}R_2}\\
\myvec{2&1&3\\ \\0&-\dfrac{5}{2}&-\dfrac{5}{2}\\\\0&0&0},
\end{align}
the rank of the matrix is 2 and the points are in 3-Dimensional space, so the points $\vec{A},\vec{B},\vec{C}$ form a triangle.
\begin{enumerate}
\item checking whether the triangle is right angled at $\vec{A}$
\begin{align}
\vec{B}-\vec{A} &= \myvec{-1\\-2\\-6} \\
\vec{C}-\vec{A} &= \myvec{1\\-3\\-5} \\
\brak{\vec{B}-\vec{A}}^{\top}\brak{\vec{C}-\vec{A}} &= \myvec{-1&-2&-6}\myvec{1\\-3\\-5} = 35
\neq 0
\end{align}
The triangle is not right angled at $\vec{A}$.
%
\item checking whether the triangle is right angled at $\vec{B}$
\begin{align}
\vec{A}-\vec{B} &= \myvec{1\\2\\6} \\
\vec{C}-\vec{B} &= \myvec{2\\-1\\1} 
\end{align}
\begin{align}
\brak{\vec{A}-\vec{B}}^{\top}\brak{\vec{C}-\vec{B}} &= \myvec{1&2&6}\myvec{2\\-1\\1} = 6
\neq 0
\end{align}
The triangle is not right angled at $\vec{B}$.
%
\item checking whether the triangle is right angled at $\vec{C}$
\begin{align}
\vec{A}-\vec{C} &= \myvec{-1\\3\\5} \\
\vec{B}-\vec{C} &= \myvec{-2\\1\\-1} \\
\brak{\vec{A}-\vec{C}}^{\top}\brak{\vec{B}-\vec{C}} &= \myvec{-1&3&5}\myvec{-2\\1\\-1} = 0\\
\end{align}
Hence the triangle is right angled at $\vec{C}$.
\end{enumerate}





\item Show that the points A, B and C with position vectors,$\vec{a}=3\hat{i}-4\hat{j}-4\hat{k}$,$\vec{b}=2\hat{i}-\hat{j}+\hat{k}$ and $\vec{c}=\hat{i}-3\hat{j}-5\hat{k}$, respectively form the vertices of a right angled
triangle.
\\
\solution
		\iffalse
\documentclass[12pt]{article}
\usepackage{graphicx}
\usepackage{amsmath}
\usepackage{mathtools}
\usepackage{gensymb}

\newcommand{\mydet}[1]{\ensuremath{\begin{vmatrix}#1\end{vmatrix}}}
\providecommand{\brak}[1]{\ensuremath{\left(#1\right)}}
\providecommand{\norm}[1]{\left\lVert#1\right\rVert}
\newcommand{\solution}{\noindent \textbf{Solution: }}
\newcommand{\myvec}[1]{\ensuremath{\begin{pmatrix}#1\end{pmatrix}}}
\let\vec\mathbf

\begin{document}
\begin{center}
\textbf\large{CHAPTER-7 \\ COORDINATE GEOMETRY}

\end{center}
\section*{Excercise 7.1}

Q6.Name the type of quadilateral formed,if any, by the following points, and give reasons for your answer:
\begin{enumerate}
	\item $\brak{-1,-2}, \brak{1,0}, \brak{-1,2}, \brak{-3,0}$ 
	\item $\brak{-3,5}, \brak{3,1}, \brak{0,3}, \brak{-1,-4}$
	\item $\brak{4,5}, \brak{7,6}, \brak{4,3}, \brak{1,2}$
\end{enumerate}
\solution
\fi
\begin{enumerate}
\item The coordinates are given as
	\begin{align}
	\vec{A} = \myvec{
		-1\\
		-2\\
		},
	\vec{B} = \myvec{
		1\\
		0\\
		},
	\vec{C} = \myvec{
		-1\\
		2\\
		} \text{ and }
	\vec{D} = \myvec{
		-3\\
		0\\
		}
	\end{align}
	\begin{align}
		\vec{B} - \vec{A} &= \myvec{1\\0} - \myvec{-1\\-2} = \myvec{2\\2}\\
		\vec{C} - \vec{B} &= \myvec{-1\\2} - \myvec{1\\0} = \myvec{-2\\2}\\
		\vec{C} - \vec{D} &= \myvec{-1\\2} - \myvec{-3\\0} = \myvec{2\\2}\\
		\vec{D} - \vec{A} &= \myvec{-3\\0} - \myvec{-1\\-2} = \myvec{-2\\2}
	\end{align}
	\begin{align}	
		\vec{C} - \vec{A} &= \myvec{-1\\2} - \myvec{-1\\-2} = \myvec{0\\4}\\
		\vec{D} - \vec{B} &= \myvec{-3\\0} - \myvec{1\\0} = \myvec{-4\\0}
	\end{align}
	\begin{align}	
		\vec{B}-\vec{A} = \vec{C}-\vec{D} \text{ and } \vec{C}-\vec{B} = \vec{D}-\vec{A}.
	\end{align}
	Hence, $ABCD$ is a parallelogram.
	\begin{enumerate}
		\item Now checking if the adjacent sides are orthogonal to each other
	\begin{align}
		(\vec{B}-\vec{A})^\top (\vec{C}-\vec{B}) = \myvec{2&2} \myvec{-2\\2} = -4+4 = 0
	\end{align}
		\item Now checking if the diagonals are also orthogonal then it is a square else a rectangle.
	\end{enumerate}	
	\begin{align}
		(\vec{C}-\vec{A})^\top (\vec{D}-\vec{B}) = \myvec{0&4} \myvec{-4\\0} = 0
	\end{align}
	Hence the diagonals are orthogonal to each other.

	So, we can conclude that $ABCD$ is a square.

	As shown in Figure \ref{fig:10/7/1/6/Fig1} we can see that $ABCD$ is a square hence we can conclude that our theoritical result is verified.
 
\begin{figure}[!h]
	\begin{center} 
	    \includegraphics[width=\columnwidth]{chapters/10/7/1/6/figs/quad1}
	\end{center}
\caption{}
\label{fig:10/7/1/6/Fig1}
\end{figure}

\item The coordinates are given as
	\begin{align}
	\vec{A} = \myvec{
		-3\\
		5\\
		},
	\vec{B} = \myvec{
		3\\
		1\\
		},
	\vec{C} = \myvec{
		0\\
		3\\
		} \text{ and }
	\vec{D} = \myvec{
		-1\\
		-4\\
		}
	\end{align}
	\begin{align}
		\vec{B} - \vec{A} &= \myvec{3\\1} - \myvec{-3\\5} = \myvec{6\\-4}\\
		\vec{C} - \vec{B} &= \myvec{0\\3} - \myvec{3\\1} = \myvec{-3\\2}\\
		\vec{C} - \vec{D} &= \myvec{0\\3} - \myvec{-1\\-4} = \myvec{1\\7}\\
		\vec{D} - \vec{A} &= \myvec{-1\\-4} - \myvec{-3\\5} = \myvec{2\\-9}
	\end{align}
	\begin{align}
		\vec{C} - \vec{A} &= \myvec{0\\3} - \myvec{-3\\5} = \myvec{3\\-2}\\
		\vec{D} - \vec{B} &= \myvec{-1\\-4} - \myvec{3\\1} = \myvec{-4\\-5}
	\end{align}
	\begin{align}
	\vec{B}-\vec{A} \neq \vec{C}-\vec{D} \text{ and } \vec{C}-\vec{B} \neq \vec{D}-\vec{A},
	\end{align}
	Hence, $ABCD$ is not a parallelogram, it can be a irregular quadilateral.
	\begin{enumerate}
		\item Now to check if any three points are collinear,

	if rank of $\myvec{\vec{B}-\vec{A} & \vec{C}-\vec{B}} = 1$ then points are collinear

	Forming the collinearity matrix
	\begin{align}
		\myvec{6&-3\\-4&2} \xleftrightarrow{R_{2}\rightarrow R_{2}+\frac{2}{3}R_{1}}= \myvec{6&-3\\0&0}
	\end{align}
	\end{enumerate}
	Hence, rank = 1

	Since none of the opposite sides are parallel to each other and three points are collinear so these does not form a quadilateral.

	As shown in Figure \ref{fig:10/7/1/6/Fig2} we can see that $ABCD$ does not form a quadilateral and three points are collinear hence, our theoritical result is verified.
	
\begin{figure}[!h]
	\begin{center} 
	    \includegraphics[width=\columnwidth]{chapters/10/7/1/6/figs/quad2}
	\end{center}
\caption{}
\label{fig:10/7/1/6/Fig2}
\end{figure}
	
\item The coordinates are given as
	\begin{align}
	\vec{A} = \myvec{
		4\\
		5\\
		},
	\vec{B} = \myvec{
		7\\
		6\\
		},
	\vec{C} = \myvec{
		4\\
		3\\
		} \text{ and }
	\vec{D} = \myvec{
		1\\
		2\\
		}
	\end{align}
	\begin{align}
		\vec{B} - \vec{A} &= \myvec{7\\6} - \myvec{4\\5} = \myvec{3\\1}\\
		\vec{C} - \vec{B} &= \myvec{4\\3} - \myvec{7\\6} = \myvec{-3\\-3}\\
		\vec{C} - \vec{D} &= \myvec{4\\3} - \myvec{1\\2} = \myvec{3\\1}\\
		\vec{D} - \vec{A} &= \myvec{1\\2} - \myvec{4\\5} = \myvec{-3\\-3}
	\end{align}
	\begin{align}
		\vec{C} - \vec{A} &= \myvec{4\\3} - \myvec{4\\5} = \myvec{0\\-2}\\
		\vec{D} - \vec{B} &= \myvec{1\\2} - \myvec{7\\6} = \myvec{-6\\-4}
	\end{align}
	\begin{align}
		\vec{B}-\vec{A} = \vec{C}-\vec{D} \text{ and } \vec{C}-\vec{B} = \vec{D}-\vec{A},
	\end{align}
	Hence, $ABCD$ is a parallelogram.
	\begin{enumerate}
		\item Now checking if the adjacent sides are orthogonal to each other
	\begin{align}
		(\vec{B}-\vec{A})^\top (\vec{C}-\vec{B}) = \myvec{3&1} \myvec{-3\\-3} = -9-3 = -12
	\end{align}
	Since inner product is not zero so adjacent sides are not orthogonal.

	Hence, we can say that $ABCD$ is neither a rectangle nor a square.

		\item Now checking if the diagonals are orthogonal then it is a Rhombus.
	\begin{align}
		(\vec{C}- \vec{A})^\top (\vec{D}-\vec{B}) = \myvec{0&-2} \myvec{-6\\-4} = 0+8 = 8
	\end{align}
	\end{enumerate}		
	Hence the diagonals are also not orthogonal so we conclude that $ABCD$ is a parallelogram.

	As shown in Figure \ref{fig:10/7/1/6/Fig3} we can see that $ABCD$ forms a parallelogram hence, our theoritical result is verified.

\begin{figure}[!h]
	\begin{center} 
	    \includegraphics[width=\columnwidth]{chapters/10/7/1/6/figs/quad3}
	\end{center}
\caption{}
\label{fig:10/7/1/6/Fig3}
\end{figure}
\end{enumerate}



\item Let $\vec{a}=\hat{i}+4\hat{j}+2\hat{k}$,$\vec{b}=3\hat{i}-2\hat{j}+7\hat{k}$ and $\vec{c}=2\hat{i}-\hat{j}+4\hat{k}$.Find a vector $\vec{d}$ which is perpendicular to both $\vec{a}$ and $\vec{b}$,and $\vec{c}.\vec{d}$=15.\\
	\solution
		\begin{enumerate}[label=\thesection.\arabic*,ref=\thesection.\theenumi]
\numberwithin{equation}{enumi}
\numberwithin{figure}{enumi}
\numberwithin{table}{enumi}
\item  Find the vector equation of the line which is parallel to the vector $3\hat{i}-2\hat{j}+6\hat{k}$ and which passes through the point $(1,-2,3)$.
\item Find the equations of the two lines through the origin which intersect the line $ \dfrac{x-3}{2}=\dfrac{y-3}{1}=\dfrac{z}{1}$ at angles of  $\dfrac{\pi}{3}$each.
\item Find the equations of the line passing through the point $(3,0,1)$ and parallel to the planes $x+2y=0$ and $3y-z=0.$
\item The vector equation of the line $\dfrac{x-5}{3}=\dfrac{y+4}{7}=\dfrac{z-6}{2}$ is \noindent\rule{2cm}{0.4pt}. 
\item The vector equation of the line through the points $(3,4,-7)$ and $(1,-1,6)$ is \noindent\rule{2cm}{0.4pt}.
\item the unit vector normal to the plane $x+2y+3z-6=0$ is $\dfrac{1}{\sqrt{14}}\hat{i} + \dfrac{2}{\sqrt{14}}\hat{j} + \dfrac{3}{\sqrt{14}}\hat{k}$.
\item The vector equation of the line $\dfrac{x-5}{3}=\dfrac{y+4}{7}=\dfrac{z-6}{2}$ is
$$\overrightarrow{r}=5\hat{i}-4\hat{j}+6\hat{k}+\lambda(3\hat{i}+7\hat{j}+2\hat{k}).$$
\item The equation of a line, which is parallel to $2\hat{i}+\hat{j}+3\hat{k}$ and which passes through the point $(5,-2,4)$ is $\dfrac{x-5}{2}=\dfrac{y+2}{-1}=\dfrac{z-4}{3}$.
\end{enumerate}

\item Prove that $(\vec{a}+\vec{b}).(\vec{a}+\vec{b})$=$|{\vec{a}}|^2+|{\vec{b}}|^2$,if and only if $\vec{a},\vec{b}$ are perpendicular, given $\vec{a}\neq\vec{0}$,$\vec{b}\neq\vec{0}$.\\
	\solution
		\iffalse
\documentclass[12pt]{article}
\usepackage{graphicx}
\usepackage{amsmath}
\usepackage{mathtools}
\usepackage{gensymb}

\newcommand{\mydet}[1]{\ensuremath{\begin{vmatrix}#1\end{vmatrix}}}
\providecommand{\brak}[1]{\ensuremath{\left(#1\right)}}
\providecommand{\norm}[1]{\left\lVert#1\right\rVert}
\newcommand{\solution}{\noindent \textbf{Solution: }}
\newcommand{\myvec}[1]{\ensuremath{\begin{pmatrix}#1\end{pmatrix}}}
\let\vec\mathbf

\begin{document}
\begin{center}
\textbf\large{CHAPTER-7 \\ COORDINATE GEOMETRY}

\end{center}
\section*{Excercise 7.1}

Q6.Name the type of quadilateral formed,if any, by the following points, and give reasons for your answer:
\begin{enumerate}
	\item $\brak{-1,-2}, \brak{1,0}, \brak{-1,2}, \brak{-3,0}$ 
	\item $\brak{-3,5}, \brak{3,1}, \brak{0,3}, \brak{-1,-4}$
	\item $\brak{4,5}, \brak{7,6}, \brak{4,3}, \brak{1,2}$
\end{enumerate}
\solution
\fi
\begin{enumerate}
\item The coordinates are given as
	\begin{align}
	\vec{A} = \myvec{
		-1\\
		-2\\
		},
	\vec{B} = \myvec{
		1\\
		0\\
		},
	\vec{C} = \myvec{
		-1\\
		2\\
		} \text{ and }
	\vec{D} = \myvec{
		-3\\
		0\\
		}
	\end{align}
	\begin{align}
		\vec{B} - \vec{A} &= \myvec{1\\0} - \myvec{-1\\-2} = \myvec{2\\2}\\
		\vec{C} - \vec{B} &= \myvec{-1\\2} - \myvec{1\\0} = \myvec{-2\\2}\\
		\vec{C} - \vec{D} &= \myvec{-1\\2} - \myvec{-3\\0} = \myvec{2\\2}\\
		\vec{D} - \vec{A} &= \myvec{-3\\0} - \myvec{-1\\-2} = \myvec{-2\\2}
	\end{align}
	\begin{align}	
		\vec{C} - \vec{A} &= \myvec{-1\\2} - \myvec{-1\\-2} = \myvec{0\\4}\\
		\vec{D} - \vec{B} &= \myvec{-3\\0} - \myvec{1\\0} = \myvec{-4\\0}
	\end{align}
	\begin{align}	
		\vec{B}-\vec{A} = \vec{C}-\vec{D} \text{ and } \vec{C}-\vec{B} = \vec{D}-\vec{A}.
	\end{align}
	Hence, $ABCD$ is a parallelogram.
	\begin{enumerate}
		\item Now checking if the adjacent sides are orthogonal to each other
	\begin{align}
		(\vec{B}-\vec{A})^\top (\vec{C}-\vec{B}) = \myvec{2&2} \myvec{-2\\2} = -4+4 = 0
	\end{align}
		\item Now checking if the diagonals are also orthogonal then it is a square else a rectangle.
	\end{enumerate}	
	\begin{align}
		(\vec{C}-\vec{A})^\top (\vec{D}-\vec{B}) = \myvec{0&4} \myvec{-4\\0} = 0
	\end{align}
	Hence the diagonals are orthogonal to each other.

	So, we can conclude that $ABCD$ is a square.

	As shown in Figure \ref{fig:10/7/1/6/Fig1} we can see that $ABCD$ is a square hence we can conclude that our theoritical result is verified.
 
\begin{figure}[!h]
	\begin{center} 
	    \includegraphics[width=\columnwidth]{chapters/10/7/1/6/figs/quad1}
	\end{center}
\caption{}
\label{fig:10/7/1/6/Fig1}
\end{figure}

\item The coordinates are given as
	\begin{align}
	\vec{A} = \myvec{
		-3\\
		5\\
		},
	\vec{B} = \myvec{
		3\\
		1\\
		},
	\vec{C} = \myvec{
		0\\
		3\\
		} \text{ and }
	\vec{D} = \myvec{
		-1\\
		-4\\
		}
	\end{align}
	\begin{align}
		\vec{B} - \vec{A} &= \myvec{3\\1} - \myvec{-3\\5} = \myvec{6\\-4}\\
		\vec{C} - \vec{B} &= \myvec{0\\3} - \myvec{3\\1} = \myvec{-3\\2}\\
		\vec{C} - \vec{D} &= \myvec{0\\3} - \myvec{-1\\-4} = \myvec{1\\7}\\
		\vec{D} - \vec{A} &= \myvec{-1\\-4} - \myvec{-3\\5} = \myvec{2\\-9}
	\end{align}
	\begin{align}
		\vec{C} - \vec{A} &= \myvec{0\\3} - \myvec{-3\\5} = \myvec{3\\-2}\\
		\vec{D} - \vec{B} &= \myvec{-1\\-4} - \myvec{3\\1} = \myvec{-4\\-5}
	\end{align}
	\begin{align}
	\vec{B}-\vec{A} \neq \vec{C}-\vec{D} \text{ and } \vec{C}-\vec{B} \neq \vec{D}-\vec{A},
	\end{align}
	Hence, $ABCD$ is not a parallelogram, it can be a irregular quadilateral.
	\begin{enumerate}
		\item Now to check if any three points are collinear,

	if rank of $\myvec{\vec{B}-\vec{A} & \vec{C}-\vec{B}} = 1$ then points are collinear

	Forming the collinearity matrix
	\begin{align}
		\myvec{6&-3\\-4&2} \xleftrightarrow{R_{2}\rightarrow R_{2}+\frac{2}{3}R_{1}}= \myvec{6&-3\\0&0}
	\end{align}
	\end{enumerate}
	Hence, rank = 1

	Since none of the opposite sides are parallel to each other and three points are collinear so these does not form a quadilateral.

	As shown in Figure \ref{fig:10/7/1/6/Fig2} we can see that $ABCD$ does not form a quadilateral and three points are collinear hence, our theoritical result is verified.
	
\begin{figure}[!h]
	\begin{center} 
	    \includegraphics[width=\columnwidth]{chapters/10/7/1/6/figs/quad2}
	\end{center}
\caption{}
\label{fig:10/7/1/6/Fig2}
\end{figure}
	
\item The coordinates are given as
	\begin{align}
	\vec{A} = \myvec{
		4\\
		5\\
		},
	\vec{B} = \myvec{
		7\\
		6\\
		},
	\vec{C} = \myvec{
		4\\
		3\\
		} \text{ and }
	\vec{D} = \myvec{
		1\\
		2\\
		}
	\end{align}
	\begin{align}
		\vec{B} - \vec{A} &= \myvec{7\\6} - \myvec{4\\5} = \myvec{3\\1}\\
		\vec{C} - \vec{B} &= \myvec{4\\3} - \myvec{7\\6} = \myvec{-3\\-3}\\
		\vec{C} - \vec{D} &= \myvec{4\\3} - \myvec{1\\2} = \myvec{3\\1}\\
		\vec{D} - \vec{A} &= \myvec{1\\2} - \myvec{4\\5} = \myvec{-3\\-3}
	\end{align}
	\begin{align}
		\vec{C} - \vec{A} &= \myvec{4\\3} - \myvec{4\\5} = \myvec{0\\-2}\\
		\vec{D} - \vec{B} &= \myvec{1\\2} - \myvec{7\\6} = \myvec{-6\\-4}
	\end{align}
	\begin{align}
		\vec{B}-\vec{A} = \vec{C}-\vec{D} \text{ and } \vec{C}-\vec{B} = \vec{D}-\vec{A},
	\end{align}
	Hence, $ABCD$ is a parallelogram.
	\begin{enumerate}
		\item Now checking if the adjacent sides are orthogonal to each other
	\begin{align}
		(\vec{B}-\vec{A})^\top (\vec{C}-\vec{B}) = \myvec{3&1} \myvec{-3\\-3} = -9-3 = -12
	\end{align}
	Since inner product is not zero so adjacent sides are not orthogonal.

	Hence, we can say that $ABCD$ is neither a rectangle nor a square.

		\item Now checking if the diagonals are orthogonal then it is a Rhombus.
	\begin{align}
		(\vec{C}- \vec{A})^\top (\vec{D}-\vec{B}) = \myvec{0&-2} \myvec{-6\\-4} = 0+8 = 8
	\end{align}
	\end{enumerate}		
	Hence the diagonals are also not orthogonal so we conclude that $ABCD$ is a parallelogram.

	As shown in Figure \ref{fig:10/7/1/6/Fig3} we can see that $ABCD$ forms a parallelogram hence, our theoritical result is verified.

\begin{figure}[!h]
	\begin{center} 
	    \includegraphics[width=\columnwidth]{chapters/10/7/1/6/figs/quad3}
	\end{center}
\caption{}
\label{fig:10/7/1/6/Fig3}
\end{figure}
\end{enumerate}



\item $ABCD$ is a rectangle formed by the points $\vec{A}(–1, –1), \vec{B}(– 1, 4), \vec{C}(5, 4)$  and  $\vec{D}(5, – 1)$. $\vec{P}, \vec{Q}, \vec{R}$ and $\vec{S}$ are the mid-points of $AB, BC, CD$ and $DA$ respectively. Is the quadrilateral $PQRS$ a square? a rectangle? or a rhombus? Justify your answer.
	\\
	\iffalse
\documentclass[12pt]{article}
\usepackage{graphicx}
\usepackage{amsmath}
\usepackage{mathtools}
\usepackage{gensymb}

\newcommand{\mydet}[1]{\ensuremath{\begin{vmatrix}#1\end{vmatrix}}}
\providecommand{\brak}[1]{\ensuremath{\left(#1\right)}}
\providecommand{\norm}[1]{\left\lVert#1\right\rVert}
\newcommand{\solution}{\noindent \textbf{Solution: }}
\newcommand{\myvec}[1]{\ensuremath{\begin{pmatrix}#1\end{pmatrix}}}
\let\vec\mathbf

\begin{document}
\begin{center}
\textbf\large{CHAPTER-7 \\ COORDINATE GEOMETRY}

\end{center}
\section*{Excercise 7.1}

Q6.Name the type of quadilateral formed,if any, by the following points, and give reasons for your answer:
\begin{enumerate}
	\item $\brak{-1,-2}, \brak{1,0}, \brak{-1,2}, \brak{-3,0}$ 
	\item $\brak{-3,5}, \brak{3,1}, \brak{0,3}, \brak{-1,-4}$
	\item $\brak{4,5}, \brak{7,6}, \brak{4,3}, \brak{1,2}$
\end{enumerate}
\solution
\fi
\begin{enumerate}
\item The coordinates are given as
	\begin{align}
	\vec{A} = \myvec{
		-1\\
		-2\\
		},
	\vec{B} = \myvec{
		1\\
		0\\
		},
	\vec{C} = \myvec{
		-1\\
		2\\
		} \text{ and }
	\vec{D} = \myvec{
		-3\\
		0\\
		}
	\end{align}
	\begin{align}
		\vec{B} - \vec{A} &= \myvec{1\\0} - \myvec{-1\\-2} = \myvec{2\\2}\\
		\vec{C} - \vec{B} &= \myvec{-1\\2} - \myvec{1\\0} = \myvec{-2\\2}\\
		\vec{C} - \vec{D} &= \myvec{-1\\2} - \myvec{-3\\0} = \myvec{2\\2}\\
		\vec{D} - \vec{A} &= \myvec{-3\\0} - \myvec{-1\\-2} = \myvec{-2\\2}
	\end{align}
	\begin{align}	
		\vec{C} - \vec{A} &= \myvec{-1\\2} - \myvec{-1\\-2} = \myvec{0\\4}\\
		\vec{D} - \vec{B} &= \myvec{-3\\0} - \myvec{1\\0} = \myvec{-4\\0}
	\end{align}
	\begin{align}	
		\vec{B}-\vec{A} = \vec{C}-\vec{D} \text{ and } \vec{C}-\vec{B} = \vec{D}-\vec{A}.
	\end{align}
	Hence, $ABCD$ is a parallelogram.
	\begin{enumerate}
		\item Now checking if the adjacent sides are orthogonal to each other
	\begin{align}
		(\vec{B}-\vec{A})^\top (\vec{C}-\vec{B}) = \myvec{2&2} \myvec{-2\\2} = -4+4 = 0
	\end{align}
		\item Now checking if the diagonals are also orthogonal then it is a square else a rectangle.
	\end{enumerate}	
	\begin{align}
		(\vec{C}-\vec{A})^\top (\vec{D}-\vec{B}) = \myvec{0&4} \myvec{-4\\0} = 0
	\end{align}
	Hence the diagonals are orthogonal to each other.

	So, we can conclude that $ABCD$ is a square.

	As shown in Figure \ref{fig:10/7/1/6/Fig1} we can see that $ABCD$ is a square hence we can conclude that our theoritical result is verified.
 
\begin{figure}[!h]
	\begin{center} 
	    \includegraphics[width=\columnwidth]{chapters/10/7/1/6/figs/quad1}
	\end{center}
\caption{}
\label{fig:10/7/1/6/Fig1}
\end{figure}

\item The coordinates are given as
	\begin{align}
	\vec{A} = \myvec{
		-3\\
		5\\
		},
	\vec{B} = \myvec{
		3\\
		1\\
		},
	\vec{C} = \myvec{
		0\\
		3\\
		} \text{ and }
	\vec{D} = \myvec{
		-1\\
		-4\\
		}
	\end{align}
	\begin{align}
		\vec{B} - \vec{A} &= \myvec{3\\1} - \myvec{-3\\5} = \myvec{6\\-4}\\
		\vec{C} - \vec{B} &= \myvec{0\\3} - \myvec{3\\1} = \myvec{-3\\2}\\
		\vec{C} - \vec{D} &= \myvec{0\\3} - \myvec{-1\\-4} = \myvec{1\\7}\\
		\vec{D} - \vec{A} &= \myvec{-1\\-4} - \myvec{-3\\5} = \myvec{2\\-9}
	\end{align}
	\begin{align}
		\vec{C} - \vec{A} &= \myvec{0\\3} - \myvec{-3\\5} = \myvec{3\\-2}\\
		\vec{D} - \vec{B} &= \myvec{-1\\-4} - \myvec{3\\1} = \myvec{-4\\-5}
	\end{align}
	\begin{align}
	\vec{B}-\vec{A} \neq \vec{C}-\vec{D} \text{ and } \vec{C}-\vec{B} \neq \vec{D}-\vec{A},
	\end{align}
	Hence, $ABCD$ is not a parallelogram, it can be a irregular quadilateral.
	\begin{enumerate}
		\item Now to check if any three points are collinear,

	if rank of $\myvec{\vec{B}-\vec{A} & \vec{C}-\vec{B}} = 1$ then points are collinear

	Forming the collinearity matrix
	\begin{align}
		\myvec{6&-3\\-4&2} \xleftrightarrow{R_{2}\rightarrow R_{2}+\frac{2}{3}R_{1}}= \myvec{6&-3\\0&0}
	\end{align}
	\end{enumerate}
	Hence, rank = 1

	Since none of the opposite sides are parallel to each other and three points are collinear so these does not form a quadilateral.

	As shown in Figure \ref{fig:10/7/1/6/Fig2} we can see that $ABCD$ does not form a quadilateral and three points are collinear hence, our theoritical result is verified.
	
\begin{figure}[!h]
	\begin{center} 
	    \includegraphics[width=\columnwidth]{chapters/10/7/1/6/figs/quad2}
	\end{center}
\caption{}
\label{fig:10/7/1/6/Fig2}
\end{figure}
	
\item The coordinates are given as
	\begin{align}
	\vec{A} = \myvec{
		4\\
		5\\
		},
	\vec{B} = \myvec{
		7\\
		6\\
		},
	\vec{C} = \myvec{
		4\\
		3\\
		} \text{ and }
	\vec{D} = \myvec{
		1\\
		2\\
		}
	\end{align}
	\begin{align}
		\vec{B} - \vec{A} &= \myvec{7\\6} - \myvec{4\\5} = \myvec{3\\1}\\
		\vec{C} - \vec{B} &= \myvec{4\\3} - \myvec{7\\6} = \myvec{-3\\-3}\\
		\vec{C} - \vec{D} &= \myvec{4\\3} - \myvec{1\\2} = \myvec{3\\1}\\
		\vec{D} - \vec{A} &= \myvec{1\\2} - \myvec{4\\5} = \myvec{-3\\-3}
	\end{align}
	\begin{align}
		\vec{C} - \vec{A} &= \myvec{4\\3} - \myvec{4\\5} = \myvec{0\\-2}\\
		\vec{D} - \vec{B} &= \myvec{1\\2} - \myvec{7\\6} = \myvec{-6\\-4}
	\end{align}
	\begin{align}
		\vec{B}-\vec{A} = \vec{C}-\vec{D} \text{ and } \vec{C}-\vec{B} = \vec{D}-\vec{A},
	\end{align}
	Hence, $ABCD$ is a parallelogram.
	\begin{enumerate}
		\item Now checking if the adjacent sides are orthogonal to each other
	\begin{align}
		(\vec{B}-\vec{A})^\top (\vec{C}-\vec{B}) = \myvec{3&1} \myvec{-3\\-3} = -9-3 = -12
	\end{align}
	Since inner product is not zero so adjacent sides are not orthogonal.

	Hence, we can say that $ABCD$ is neither a rectangle nor a square.

		\item Now checking if the diagonals are orthogonal then it is a Rhombus.
	\begin{align}
		(\vec{C}- \vec{A})^\top (\vec{D}-\vec{B}) = \myvec{0&-2} \myvec{-6\\-4} = 0+8 = 8
	\end{align}
	\end{enumerate}		
	Hence the diagonals are also not orthogonal so we conclude that $ABCD$ is a parallelogram.

	As shown in Figure \ref{fig:10/7/1/6/Fig3} we can see that $ABCD$ forms a parallelogram hence, our theoritical result is verified.

\begin{figure}[!h]
	\begin{center} 
	    \includegraphics[width=\columnwidth]{chapters/10/7/1/6/figs/quad3}
	\end{center}
\caption{}
\label{fig:10/7/1/6/Fig3}
\end{figure}
\end{enumerate}



\item Without using the Baudhayana theorem, show that the points $(4,4),(3,5)$ and $(-1,-1)$ are the vertices of a right angled triangle.
\label{chapters/11/10/1/6}
\iffalse
\documentclass[journal,12pt,twocolumn]{IEEEtran}
\usepackage[none]{hyphenat}
\usepackage{graphicx}
\usepackage{listings}
\usepackage[english]{babel}
\usepackage{graphicx}
\usepackage{caption} 
\usepackage{amsmath}
\usepackage{hyperref}
\usepackage{booktabs}
\usepackage{array}


\title{\textbf{\\Assignment on line}}
\author{Sireesha Abbavaram - FWC22060}
\begin{document}
\maketitle


\section{Question}
\textbf{\textit{Class 11, Exercise 10.1, Q(6):}
\fi
	\begin{figure}[!ht]
		\centering
 \includegraphics[width=\columnwidth]{chapters/11/10/1/6/figs/triangle.png}
		\caption{}
		\label{fig:11/10/1/6}
  	\end{figure}
\iffalse
}

\begin{figure}[h!]
\centering
\includegraphics[scale=0.35]{triangle.png}
\centering
\caption{Traingle ABC}
\end{figure}


\section{Solution}
\raggedright 
\vspace{0.25cm}
Let A,B and C be the vertices of a given traingle with coordinates $\myvec{
4 \\
4
}
, \myvec{
3 \\
5
}
 and \myvec{
-1 \\
-1
} $
\raggedright
. we have verify whther the given vertices are of right angled triangle or not.\\
\begin{center}
\raggedright
Let The directional vector of two vectors A and B is given as AB (m1)=A-B.
\end{center}
\vspace{0.25cm}
\begin{center}
The directional vector of the vectors B and C is given as BC (m2)=B-C.
\end{center}
\vspace{0.25cm}
\begin{center}
The directinal vector of the vectors C and A is given as CA (m3)=C-A.
\end{center}
\vspace{0.25cm}
The angle between any two vectors is given by
\boldmath
\\ $ cos b =\frac{(m1)^T(m2)}{||m1|| ||m2||}  equation-1$
\unboldmath
\vspace{0.5cm}\raggedright\\
Where b is the angle between the two vectors .
when the angle b=90 ,we get cos 90=0.
\vspace{0.5cm}\raggedright\\
It implies that the numerator of the equation 1 should be zero.

\vspace{0.25cm}
 In order to prove that the triangle is right angled we have to show any two vectors should be orthogonal to each other.
 
\vspace{0.25cm}\raggedright
So we need to show $(A-B)^T(B-C) or (B-C)^T(C-A) or (C-A)^T(A-B) $ is equal to zero.
\fi

\vspace{0.5cm}\raggedright
\begin{align}
	\vec{C}-\vec{A}&=\myvec{
-5 \\
	-5},
\\
	\vec{A}-\vec{B}&=\myvec{
1 \\
-1 
}
\\
	\implies \brak{\vec{C}-\vec{A}}^{\top}
	\brak{\vec{A}-\vec{B}}&=0
\end{align}
Thus, $AB \perp AC$.
\iffalse
\vspace{0.25cm}\raggedright
Thus we have right angle at the vertex A.


\vspace{0.2cm}
\section*{Construction}
\centering
\vspace{0.2cm}
{
\setlength\extrarowheight{2pt}
\begin{tabular}{|c|c|c|}
	\hline
	\textbf{Symbol}&\textbf{Value}&\textbf{Description}\\
	\hline
	A & (4,4) & Vertex A\\
	\hline
	B & (3,5) & Vertex B\\
	\hline
	C & (-1,-1) & Vertex C\\
	\hline
	
\end{tabular}
}

\vspace{0.6cm}
Get the python code of the figures from
\begin{table}[h]
\large
\centering
\begin{tabular}{|l|}
\hline
https://github.com/Sireesha1602/sireesha/
\\blob/main/line assignment \\
\hline
\end{tabular}

\end{table}




\end{document}
\fi

\item The line through the points $(h, 3)$ and $(4, 1)$ intersects the line $7x- 9y- 19= 0$ at right angle. Find the value of $h$.
\label{chapters/11/10/3/10}
\\
\solution
\iffalse
\documentclass[12pt]{article}
\usepackage{graphicx}
\usepackage[none]{hyphenat}
\usepackage{graphicx}
\usepackage{listings}
\usepackage[english]{babel}
\usepackage{graphicx}
\usepackage{caption} 
\usepackage{booktabs}
\usepackage{array}
\usepackage{amssymb} % for \because
\usepackage{amsmath}   % for having text in math mode
\usepackage{extarrows} % for Row operations arrows
\usepackage{listings}
\usepackage[utf8]{inputenc}
\lstset{
  frame=single,
  breaklines=true
}
\usepackage{hyperref}
  
%Following 2 lines were added to remove the blank page at the beginning
\usepackage{atbegshi}% http://ctan.org/pkg/atbegshi
\AtBeginDocument{\AtBeginShipoutNext{\AtBeginShipoutDiscard}}


%New macro definitions
\newcommand{\mydet}[1]{\ensuremath{\begin{vmatrix}#1\end{vmatrix}}}
\providecommand{\brak}[1]{\ensuremath{\left(#1\right)}}
\newcommand{\solution}{\noindent \textbf{Solution: }}
\newcommand{\myvec}[1]{\ensuremath{\begin{pmatrix}#1\end{pmatrix}}}
\providecommand{\norm}[1]{\left\lVert#1\right\rVert}
\providecommand{\abs}[1]{\left\vert#1\right\vert}
\let\vec\mathbf

\begin{document}

\begin{center}
\title{\textbf{LINE}}
\date{\vspace{-5ex}} %Not to print date automatically
\maketitle
\end{center}

\section{11$^{th}$ Maths - EXERCISE-10.3}
\begin{enumerate}
\end{enumerate}
\section{SOLUTION}
\fi
Let
Given points are 
\begin{align}
\vec{A}=\myvec{h\\ 3},\vec{B}=\myvec{4\\ 1} 
\implies \vec{B}-\vec{A}=
\myvec{4-h\\ -2}
\end{align}
The given line equation  can be expressed as
\begin{align}
\myvec{7& -9}\vec{x}&=19
\end{align}
yielding 
\begin{align}
\vec{n}=\myvec{7\\ -9},
\vec{m}=\myvec{9\\ 7}
\end{align}
Thus, 
\begin{align}
	\vec{m}^\top\brak{\vec{B}- \vec{A}}&=0\\
\implies\myvec{9& 7}\myvec{4-h\\ -2}&=0\\
\implies h&=\frac{22}{9}
\end{align}
See Fig. 
		\ref{fig:chapters/11/10/3/10/Figure}.
\begin{figure}[h]
\centering
\includegraphics[width=\columnwidth]{chapters/11/10/3/10/figs/fig.pdf}
\caption{}
		\label{fig:chapters/11/10/3/10/Figure}
\end{figure}

\item In the following cases, determine whether the given planes are parallel or perpendicular, and in case they are neither, find the angles between them.
\begin{enumerate}
\item $7x + 5y + 6z + 30 = 0$ and $3x – y – 10z + 4 = 0$
\item $2x + y + 3z – 2 = 0$ and $x – 2y + 5 = 0$
\item $2x – 2y + 4z + 5 = 0$ and $3x – 3y + 6z – 1 = 0$
\item $2x – y + 3z – 1 = 0$ and $2x – y + 3z + 3 = 0$
\item $4x + 8y + z – 8 = 0$ and $y + z – 4 = 0$
\end{enumerate}
    \solution
		\iffalse
\documentclass[12pt]{article}
\usepackage{graphicx}
\usepackage{amsmath}
\usepackage{mathtools}
\usepackage{gensymb}

\newcommand{\mydet}[1]{\ensuremath{\begin{vmatrix}#1\end{vmatrix}}}
\providecommand{\brak}[1]{\ensuremath{\left(#1\right)}}
\providecommand{\norm}[1]{\left\lVert#1\right\rVert}
\newcommand{\solution}{\noindent \textbf{Solution: }}
\newcommand{\myvec}[1]{\ensuremath{\begin{pmatrix}#1\end{pmatrix}}}
\let\vec\mathbf

\begin{document}
\begin{center}
\textbf\large{CHAPTER-7 \\ COORDINATE GEOMETRY}

\end{center}
\section*{Excercise 7.1}

Q6.Name the type of quadilateral formed,if any, by the following points, and give reasons for your answer:
\begin{enumerate}
	\item $\brak{-1,-2}, \brak{1,0}, \brak{-1,2}, \brak{-3,0}$ 
	\item $\brak{-3,5}, \brak{3,1}, \brak{0,3}, \brak{-1,-4}$
	\item $\brak{4,5}, \brak{7,6}, \brak{4,3}, \brak{1,2}$
\end{enumerate}
\solution
\fi
\begin{enumerate}
\item The coordinates are given as
	\begin{align}
	\vec{A} = \myvec{
		-1\\
		-2\\
		},
	\vec{B} = \myvec{
		1\\
		0\\
		},
	\vec{C} = \myvec{
		-1\\
		2\\
		} \text{ and }
	\vec{D} = \myvec{
		-3\\
		0\\
		}
	\end{align}
	\begin{align}
		\vec{B} - \vec{A} &= \myvec{1\\0} - \myvec{-1\\-2} = \myvec{2\\2}\\
		\vec{C} - \vec{B} &= \myvec{-1\\2} - \myvec{1\\0} = \myvec{-2\\2}\\
		\vec{C} - \vec{D} &= \myvec{-1\\2} - \myvec{-3\\0} = \myvec{2\\2}\\
		\vec{D} - \vec{A} &= \myvec{-3\\0} - \myvec{-1\\-2} = \myvec{-2\\2}
	\end{align}
	\begin{align}	
		\vec{C} - \vec{A} &= \myvec{-1\\2} - \myvec{-1\\-2} = \myvec{0\\4}\\
		\vec{D} - \vec{B} &= \myvec{-3\\0} - \myvec{1\\0} = \myvec{-4\\0}
	\end{align}
	\begin{align}	
		\vec{B}-\vec{A} = \vec{C}-\vec{D} \text{ and } \vec{C}-\vec{B} = \vec{D}-\vec{A}.
	\end{align}
	Hence, $ABCD$ is a parallelogram.
	\begin{enumerate}
		\item Now checking if the adjacent sides are orthogonal to each other
	\begin{align}
		(\vec{B}-\vec{A})^\top (\vec{C}-\vec{B}) = \myvec{2&2} \myvec{-2\\2} = -4+4 = 0
	\end{align}
		\item Now checking if the diagonals are also orthogonal then it is a square else a rectangle.
	\end{enumerate}	
	\begin{align}
		(\vec{C}-\vec{A})^\top (\vec{D}-\vec{B}) = \myvec{0&4} \myvec{-4\\0} = 0
	\end{align}
	Hence the diagonals are orthogonal to each other.

	So, we can conclude that $ABCD$ is a square.

	As shown in Figure \ref{fig:10/7/1/6/Fig1} we can see that $ABCD$ is a square hence we can conclude that our theoritical result is verified.
 
\begin{figure}[!h]
	\begin{center} 
	    \includegraphics[width=\columnwidth]{chapters/10/7/1/6/figs/quad1}
	\end{center}
\caption{}
\label{fig:10/7/1/6/Fig1}
\end{figure}

\item The coordinates are given as
	\begin{align}
	\vec{A} = \myvec{
		-3\\
		5\\
		},
	\vec{B} = \myvec{
		3\\
		1\\
		},
	\vec{C} = \myvec{
		0\\
		3\\
		} \text{ and }
	\vec{D} = \myvec{
		-1\\
		-4\\
		}
	\end{align}
	\begin{align}
		\vec{B} - \vec{A} &= \myvec{3\\1} - \myvec{-3\\5} = \myvec{6\\-4}\\
		\vec{C} - \vec{B} &= \myvec{0\\3} - \myvec{3\\1} = \myvec{-3\\2}\\
		\vec{C} - \vec{D} &= \myvec{0\\3} - \myvec{-1\\-4} = \myvec{1\\7}\\
		\vec{D} - \vec{A} &= \myvec{-1\\-4} - \myvec{-3\\5} = \myvec{2\\-9}
	\end{align}
	\begin{align}
		\vec{C} - \vec{A} &= \myvec{0\\3} - \myvec{-3\\5} = \myvec{3\\-2}\\
		\vec{D} - \vec{B} &= \myvec{-1\\-4} - \myvec{3\\1} = \myvec{-4\\-5}
	\end{align}
	\begin{align}
	\vec{B}-\vec{A} \neq \vec{C}-\vec{D} \text{ and } \vec{C}-\vec{B} \neq \vec{D}-\vec{A},
	\end{align}
	Hence, $ABCD$ is not a parallelogram, it can be a irregular quadilateral.
	\begin{enumerate}
		\item Now to check if any three points are collinear,

	if rank of $\myvec{\vec{B}-\vec{A} & \vec{C}-\vec{B}} = 1$ then points are collinear

	Forming the collinearity matrix
	\begin{align}
		\myvec{6&-3\\-4&2} \xleftrightarrow{R_{2}\rightarrow R_{2}+\frac{2}{3}R_{1}}= \myvec{6&-3\\0&0}
	\end{align}
	\end{enumerate}
	Hence, rank = 1

	Since none of the opposite sides are parallel to each other and three points are collinear so these does not form a quadilateral.

	As shown in Figure \ref{fig:10/7/1/6/Fig2} we can see that $ABCD$ does not form a quadilateral and three points are collinear hence, our theoritical result is verified.
	
\begin{figure}[!h]
	\begin{center} 
	    \includegraphics[width=\columnwidth]{chapters/10/7/1/6/figs/quad2}
	\end{center}
\caption{}
\label{fig:10/7/1/6/Fig2}
\end{figure}
	
\item The coordinates are given as
	\begin{align}
	\vec{A} = \myvec{
		4\\
		5\\
		},
	\vec{B} = \myvec{
		7\\
		6\\
		},
	\vec{C} = \myvec{
		4\\
		3\\
		} \text{ and }
	\vec{D} = \myvec{
		1\\
		2\\
		}
	\end{align}
	\begin{align}
		\vec{B} - \vec{A} &= \myvec{7\\6} - \myvec{4\\5} = \myvec{3\\1}\\
		\vec{C} - \vec{B} &= \myvec{4\\3} - \myvec{7\\6} = \myvec{-3\\-3}\\
		\vec{C} - \vec{D} &= \myvec{4\\3} - \myvec{1\\2} = \myvec{3\\1}\\
		\vec{D} - \vec{A} &= \myvec{1\\2} - \myvec{4\\5} = \myvec{-3\\-3}
	\end{align}
	\begin{align}
		\vec{C} - \vec{A} &= \myvec{4\\3} - \myvec{4\\5} = \myvec{0\\-2}\\
		\vec{D} - \vec{B} &= \myvec{1\\2} - \myvec{7\\6} = \myvec{-6\\-4}
	\end{align}
	\begin{align}
		\vec{B}-\vec{A} = \vec{C}-\vec{D} \text{ and } \vec{C}-\vec{B} = \vec{D}-\vec{A},
	\end{align}
	Hence, $ABCD$ is a parallelogram.
	\begin{enumerate}
		\item Now checking if the adjacent sides are orthogonal to each other
	\begin{align}
		(\vec{B}-\vec{A})^\top (\vec{C}-\vec{B}) = \myvec{3&1} \myvec{-3\\-3} = -9-3 = -12
	\end{align}
	Since inner product is not zero so adjacent sides are not orthogonal.

	Hence, we can say that $ABCD$ is neither a rectangle nor a square.

		\item Now checking if the diagonals are orthogonal then it is a Rhombus.
	\begin{align}
		(\vec{C}- \vec{A})^\top (\vec{D}-\vec{B}) = \myvec{0&-2} \myvec{-6\\-4} = 0+8 = 8
	\end{align}
	\end{enumerate}		
	Hence the diagonals are also not orthogonal so we conclude that $ABCD$ is a parallelogram.

	As shown in Figure \ref{fig:10/7/1/6/Fig3} we can see that $ABCD$ forms a parallelogram hence, our theoritical result is verified.

\begin{figure}[!h]
	\begin{center} 
	    \includegraphics[width=\columnwidth]{chapters/10/7/1/6/figs/quad3}
	\end{center}
\caption{}
\label{fig:10/7/1/6/Fig3}
\end{figure}
\end{enumerate}



		\item 
 Show that the line joining the origin to the point $(2, 1, 1)$ is perpendicular to the
line determined by the points $(3, 5, – 1), (4, 3, – 1)$.
\\
    \solution
		\iffalse
\documentclass[12pt]{article}
\usepackage{graphicx}
\usepackage{amsmath}
\usepackage{mathtools}
\usepackage{gensymb}
\usepackage[utf8]{inputenc}
\usepackage{float}
\newcommand{\mydet}[1]{\ensuremath{\begin{vmatrix}#1\end{vmatrix}}}
\providecommand{\brak}[1]{\ensuremath{\left(#1\right)}}
\providecommand{\norm}[1]{\left\lVert#1\right\rVert}
\newcommand{\solution}{\noindent \textbf{Solution: }}
\newcommand{\myvec}[1]{\ensuremath{\begin{pmatrix}#1\end{pmatrix}}}
\let\vec\mathbf

\begin{document}
\begin{center}
\textbf\large{CLASS-12 \\ CHAPTER-11 \\ THREE DIMENSIONAL GEOMETRY}
\end{center}
\section*{Excercise 11.4}

\\
\solution
\\
\fi
Let
\begin{align}
  \vec{P}=\myvec{2\\1\\1},\vec{A}=\myvec{3\\5\\-1},\vec{B}=\myvec{4\\3\\-1}
\end{align}
Then
		\begin{align}
	\vec{m}=\vec{A}-\vec{B}=\myvec{-1\\2\\0}
		\end{align}
		and
		\begin{align}
			\vec{m}^\top\vec{P}=
			\myvec{-1&2&0}\myvec{2\\1\\1}=0
		\end{align}
		This proves the result.


	\item  If $l_1, m_1,n_1 \text{ and } l_2,m_2,n_2$ are the direction cosines of two mutually perpendicular lines, show that the direction cosines of the line perpendicular to both these are  $m_1n_2-m_2n_1,n_1l_2-n_2l_1,l_1m_2-l_2m_1$.
\\
    \solution
		\iffalse
\documentclass[12pt]{article}
\usepackage{graphicx}
\usepackage{amsmath}
\usepackage{mathtools}
\usepackage{gensymb}

\newcommand{\mydet}[1]{\ensuremath{\begin{vmatrix}#1\end{vmatrix}}}
\providecommand{\brak}[1]{\ensuremath{\left(#1\right)}}
\providecommand{\norm}[1]{\left\lVert#1\right\rVert}
\newcommand{\solution}{\noindent \textbf{Solution: }}
\newcommand{\myvec}[1]{\ensuremath{\begin{pmatrix}#1\end{pmatrix}}}
\let\vec\mathbf

\begin{document}
\begin{center}
\textbf\large{CHAPTER-7 \\ COORDINATE GEOMETRY}

\end{center}
\section*{Excercise 7.1}

Q6.Name the type of quadilateral formed,if any, by the following points, and give reasons for your answer:
\begin{enumerate}
	\item $\brak{-1,-2}, \brak{1,0}, \brak{-1,2}, \brak{-3,0}$ 
	\item $\brak{-3,5}, \brak{3,1}, \brak{0,3}, \brak{-1,-4}$
	\item $\brak{4,5}, \brak{7,6}, \brak{4,3}, \brak{1,2}$
\end{enumerate}
\solution
\fi
\begin{enumerate}
\item The coordinates are given as
	\begin{align}
	\vec{A} = \myvec{
		-1\\
		-2\\
		},
	\vec{B} = \myvec{
		1\\
		0\\
		},
	\vec{C} = \myvec{
		-1\\
		2\\
		} \text{ and }
	\vec{D} = \myvec{
		-3\\
		0\\
		}
	\end{align}
	\begin{align}
		\vec{B} - \vec{A} &= \myvec{1\\0} - \myvec{-1\\-2} = \myvec{2\\2}\\
		\vec{C} - \vec{B} &= \myvec{-1\\2} - \myvec{1\\0} = \myvec{-2\\2}\\
		\vec{C} - \vec{D} &= \myvec{-1\\2} - \myvec{-3\\0} = \myvec{2\\2}\\
		\vec{D} - \vec{A} &= \myvec{-3\\0} - \myvec{-1\\-2} = \myvec{-2\\2}
	\end{align}
	\begin{align}	
		\vec{C} - \vec{A} &= \myvec{-1\\2} - \myvec{-1\\-2} = \myvec{0\\4}\\
		\vec{D} - \vec{B} &= \myvec{-3\\0} - \myvec{1\\0} = \myvec{-4\\0}
	\end{align}
	\begin{align}	
		\vec{B}-\vec{A} = \vec{C}-\vec{D} \text{ and } \vec{C}-\vec{B} = \vec{D}-\vec{A}.
	\end{align}
	Hence, $ABCD$ is a parallelogram.
	\begin{enumerate}
		\item Now checking if the adjacent sides are orthogonal to each other
	\begin{align}
		(\vec{B}-\vec{A})^\top (\vec{C}-\vec{B}) = \myvec{2&2} \myvec{-2\\2} = -4+4 = 0
	\end{align}
		\item Now checking if the diagonals are also orthogonal then it is a square else a rectangle.
	\end{enumerate}	
	\begin{align}
		(\vec{C}-\vec{A})^\top (\vec{D}-\vec{B}) = \myvec{0&4} \myvec{-4\\0} = 0
	\end{align}
	Hence the diagonals are orthogonal to each other.

	So, we can conclude that $ABCD$ is a square.

	As shown in Figure \ref{fig:10/7/1/6/Fig1} we can see that $ABCD$ is a square hence we can conclude that our theoritical result is verified.
 
\begin{figure}[!h]
	\begin{center} 
	    \includegraphics[width=\columnwidth]{chapters/10/7/1/6/figs/quad1}
	\end{center}
\caption{}
\label{fig:10/7/1/6/Fig1}
\end{figure}

\item The coordinates are given as
	\begin{align}
	\vec{A} = \myvec{
		-3\\
		5\\
		},
	\vec{B} = \myvec{
		3\\
		1\\
		},
	\vec{C} = \myvec{
		0\\
		3\\
		} \text{ and }
	\vec{D} = \myvec{
		-1\\
		-4\\
		}
	\end{align}
	\begin{align}
		\vec{B} - \vec{A} &= \myvec{3\\1} - \myvec{-3\\5} = \myvec{6\\-4}\\
		\vec{C} - \vec{B} &= \myvec{0\\3} - \myvec{3\\1} = \myvec{-3\\2}\\
		\vec{C} - \vec{D} &= \myvec{0\\3} - \myvec{-1\\-4} = \myvec{1\\7}\\
		\vec{D} - \vec{A} &= \myvec{-1\\-4} - \myvec{-3\\5} = \myvec{2\\-9}
	\end{align}
	\begin{align}
		\vec{C} - \vec{A} &= \myvec{0\\3} - \myvec{-3\\5} = \myvec{3\\-2}\\
		\vec{D} - \vec{B} &= \myvec{-1\\-4} - \myvec{3\\1} = \myvec{-4\\-5}
	\end{align}
	\begin{align}
	\vec{B}-\vec{A} \neq \vec{C}-\vec{D} \text{ and } \vec{C}-\vec{B} \neq \vec{D}-\vec{A},
	\end{align}
	Hence, $ABCD$ is not a parallelogram, it can be a irregular quadilateral.
	\begin{enumerate}
		\item Now to check if any three points are collinear,

	if rank of $\myvec{\vec{B}-\vec{A} & \vec{C}-\vec{B}} = 1$ then points are collinear

	Forming the collinearity matrix
	\begin{align}
		\myvec{6&-3\\-4&2} \xleftrightarrow{R_{2}\rightarrow R_{2}+\frac{2}{3}R_{1}}= \myvec{6&-3\\0&0}
	\end{align}
	\end{enumerate}
	Hence, rank = 1

	Since none of the opposite sides are parallel to each other and three points are collinear so these does not form a quadilateral.

	As shown in Figure \ref{fig:10/7/1/6/Fig2} we can see that $ABCD$ does not form a quadilateral and three points are collinear hence, our theoritical result is verified.
	
\begin{figure}[!h]
	\begin{center} 
	    \includegraphics[width=\columnwidth]{chapters/10/7/1/6/figs/quad2}
	\end{center}
\caption{}
\label{fig:10/7/1/6/Fig2}
\end{figure}
	
\item The coordinates are given as
	\begin{align}
	\vec{A} = \myvec{
		4\\
		5\\
		},
	\vec{B} = \myvec{
		7\\
		6\\
		},
	\vec{C} = \myvec{
		4\\
		3\\
		} \text{ and }
	\vec{D} = \myvec{
		1\\
		2\\
		}
	\end{align}
	\begin{align}
		\vec{B} - \vec{A} &= \myvec{7\\6} - \myvec{4\\5} = \myvec{3\\1}\\
		\vec{C} - \vec{B} &= \myvec{4\\3} - \myvec{7\\6} = \myvec{-3\\-3}\\
		\vec{C} - \vec{D} &= \myvec{4\\3} - \myvec{1\\2} = \myvec{3\\1}\\
		\vec{D} - \vec{A} &= \myvec{1\\2} - \myvec{4\\5} = \myvec{-3\\-3}
	\end{align}
	\begin{align}
		\vec{C} - \vec{A} &= \myvec{4\\3} - \myvec{4\\5} = \myvec{0\\-2}\\
		\vec{D} - \vec{B} &= \myvec{1\\2} - \myvec{7\\6} = \myvec{-6\\-4}
	\end{align}
	\begin{align}
		\vec{B}-\vec{A} = \vec{C}-\vec{D} \text{ and } \vec{C}-\vec{B} = \vec{D}-\vec{A},
	\end{align}
	Hence, $ABCD$ is a parallelogram.
	\begin{enumerate}
		\item Now checking if the adjacent sides are orthogonal to each other
	\begin{align}
		(\vec{B}-\vec{A})^\top (\vec{C}-\vec{B}) = \myvec{3&1} \myvec{-3\\-3} = -9-3 = -12
	\end{align}
	Since inner product is not zero so adjacent sides are not orthogonal.

	Hence, we can say that $ABCD$ is neither a rectangle nor a square.

		\item Now checking if the diagonals are orthogonal then it is a Rhombus.
	\begin{align}
		(\vec{C}- \vec{A})^\top (\vec{D}-\vec{B}) = \myvec{0&-2} \myvec{-6\\-4} = 0+8 = 8
	\end{align}
	\end{enumerate}		
	Hence the diagonals are also not orthogonal so we conclude that $ABCD$ is a parallelogram.

	As shown in Figure \ref{fig:10/7/1/6/Fig3} we can see that $ABCD$ forms a parallelogram hence, our theoritical result is verified.

\begin{figure}[!h]
	\begin{center} 
	    \includegraphics[width=\columnwidth]{chapters/10/7/1/6/figs/quad3}
	\end{center}
\caption{}
\label{fig:10/7/1/6/Fig3}
\end{figure}
\end{enumerate}



	\item If the lines $\frac{x-1}{-3} = \frac{y-2}{2k} = \frac{z-3}{2}$ and  $\frac{x-1}{3k} = \frac{y-1}{1} = \frac{z-6}{-5}$ are perpendicular, find the value of k.\\
    \solution
		\iffalse
\documentclass[12pt]{article}
\usepackage{graphicx}
\usepackage{amsmath}
\usepackage{mathtools}
\usepackage{gensymb}

\newcommand{\mydet}[1]{\ensuremath{\begin{vmatrix}#1\end{vmatrix}}}
\providecommand{\brak}[1]{\ensuremath{\left(#1\right)}}
\providecommand{\norm}[1]{\left\lVert#1\right\rVert}
\newcommand{\solution}{\noindent \textbf{Solution: }}
\newcommand{\myvec}[1]{\ensuremath{\begin{pmatrix}#1\end{pmatrix}}}
\let\vec\mathbf

\begin{document}
\begin{center}
\textbf\large{CHAPTER-7 \\ COORDINATE GEOMETRY}

\end{center}
\section*{Excercise 7.1}

Q6.Name the type of quadilateral formed,if any, by the following points, and give reasons for your answer:
\begin{enumerate}
	\item $\brak{-1,-2}, \brak{1,0}, \brak{-1,2}, \brak{-3,0}$ 
	\item $\brak{-3,5}, \brak{3,1}, \brak{0,3}, \brak{-1,-4}$
	\item $\brak{4,5}, \brak{7,6}, \brak{4,3}, \brak{1,2}$
\end{enumerate}
\solution
\fi
\begin{enumerate}
\item The coordinates are given as
	\begin{align}
	\vec{A} = \myvec{
		-1\\
		-2\\
		},
	\vec{B} = \myvec{
		1\\
		0\\
		},
	\vec{C} = \myvec{
		-1\\
		2\\
		} \text{ and }
	\vec{D} = \myvec{
		-3\\
		0\\
		}
	\end{align}
	\begin{align}
		\vec{B} - \vec{A} &= \myvec{1\\0} - \myvec{-1\\-2} = \myvec{2\\2}\\
		\vec{C} - \vec{B} &= \myvec{-1\\2} - \myvec{1\\0} = \myvec{-2\\2}\\
		\vec{C} - \vec{D} &= \myvec{-1\\2} - \myvec{-3\\0} = \myvec{2\\2}\\
		\vec{D} - \vec{A} &= \myvec{-3\\0} - \myvec{-1\\-2} = \myvec{-2\\2}
	\end{align}
	\begin{align}	
		\vec{C} - \vec{A} &= \myvec{-1\\2} - \myvec{-1\\-2} = \myvec{0\\4}\\
		\vec{D} - \vec{B} &= \myvec{-3\\0} - \myvec{1\\0} = \myvec{-4\\0}
	\end{align}
	\begin{align}	
		\vec{B}-\vec{A} = \vec{C}-\vec{D} \text{ and } \vec{C}-\vec{B} = \vec{D}-\vec{A}.
	\end{align}
	Hence, $ABCD$ is a parallelogram.
	\begin{enumerate}
		\item Now checking if the adjacent sides are orthogonal to each other
	\begin{align}
		(\vec{B}-\vec{A})^\top (\vec{C}-\vec{B}) = \myvec{2&2} \myvec{-2\\2} = -4+4 = 0
	\end{align}
		\item Now checking if the diagonals are also orthogonal then it is a square else a rectangle.
	\end{enumerate}	
	\begin{align}
		(\vec{C}-\vec{A})^\top (\vec{D}-\vec{B}) = \myvec{0&4} \myvec{-4\\0} = 0
	\end{align}
	Hence the diagonals are orthogonal to each other.

	So, we can conclude that $ABCD$ is a square.

	As shown in Figure \ref{fig:10/7/1/6/Fig1} we can see that $ABCD$ is a square hence we can conclude that our theoritical result is verified.
 
\begin{figure}[!h]
	\begin{center} 
	    \includegraphics[width=\columnwidth]{chapters/10/7/1/6/figs/quad1}
	\end{center}
\caption{}
\label{fig:10/7/1/6/Fig1}
\end{figure}

\item The coordinates are given as
	\begin{align}
	\vec{A} = \myvec{
		-3\\
		5\\
		},
	\vec{B} = \myvec{
		3\\
		1\\
		},
	\vec{C} = \myvec{
		0\\
		3\\
		} \text{ and }
	\vec{D} = \myvec{
		-1\\
		-4\\
		}
	\end{align}
	\begin{align}
		\vec{B} - \vec{A} &= \myvec{3\\1} - \myvec{-3\\5} = \myvec{6\\-4}\\
		\vec{C} - \vec{B} &= \myvec{0\\3} - \myvec{3\\1} = \myvec{-3\\2}\\
		\vec{C} - \vec{D} &= \myvec{0\\3} - \myvec{-1\\-4} = \myvec{1\\7}\\
		\vec{D} - \vec{A} &= \myvec{-1\\-4} - \myvec{-3\\5} = \myvec{2\\-9}
	\end{align}
	\begin{align}
		\vec{C} - \vec{A} &= \myvec{0\\3} - \myvec{-3\\5} = \myvec{3\\-2}\\
		\vec{D} - \vec{B} &= \myvec{-1\\-4} - \myvec{3\\1} = \myvec{-4\\-5}
	\end{align}
	\begin{align}
	\vec{B}-\vec{A} \neq \vec{C}-\vec{D} \text{ and } \vec{C}-\vec{B} \neq \vec{D}-\vec{A},
	\end{align}
	Hence, $ABCD$ is not a parallelogram, it can be a irregular quadilateral.
	\begin{enumerate}
		\item Now to check if any three points are collinear,

	if rank of $\myvec{\vec{B}-\vec{A} & \vec{C}-\vec{B}} = 1$ then points are collinear

	Forming the collinearity matrix
	\begin{align}
		\myvec{6&-3\\-4&2} \xleftrightarrow{R_{2}\rightarrow R_{2}+\frac{2}{3}R_{1}}= \myvec{6&-3\\0&0}
	\end{align}
	\end{enumerate}
	Hence, rank = 1

	Since none of the opposite sides are parallel to each other and three points are collinear so these does not form a quadilateral.

	As shown in Figure \ref{fig:10/7/1/6/Fig2} we can see that $ABCD$ does not form a quadilateral and three points are collinear hence, our theoritical result is verified.
	
\begin{figure}[!h]
	\begin{center} 
	    \includegraphics[width=\columnwidth]{chapters/10/7/1/6/figs/quad2}
	\end{center}
\caption{}
\label{fig:10/7/1/6/Fig2}
\end{figure}
	
\item The coordinates are given as
	\begin{align}
	\vec{A} = \myvec{
		4\\
		5\\
		},
	\vec{B} = \myvec{
		7\\
		6\\
		},
	\vec{C} = \myvec{
		4\\
		3\\
		} \text{ and }
	\vec{D} = \myvec{
		1\\
		2\\
		}
	\end{align}
	\begin{align}
		\vec{B} - \vec{A} &= \myvec{7\\6} - \myvec{4\\5} = \myvec{3\\1}\\
		\vec{C} - \vec{B} &= \myvec{4\\3} - \myvec{7\\6} = \myvec{-3\\-3}\\
		\vec{C} - \vec{D} &= \myvec{4\\3} - \myvec{1\\2} = \myvec{3\\1}\\
		\vec{D} - \vec{A} &= \myvec{1\\2} - \myvec{4\\5} = \myvec{-3\\-3}
	\end{align}
	\begin{align}
		\vec{C} - \vec{A} &= \myvec{4\\3} - \myvec{4\\5} = \myvec{0\\-2}\\
		\vec{D} - \vec{B} &= \myvec{1\\2} - \myvec{7\\6} = \myvec{-6\\-4}
	\end{align}
	\begin{align}
		\vec{B}-\vec{A} = \vec{C}-\vec{D} \text{ and } \vec{C}-\vec{B} = \vec{D}-\vec{A},
	\end{align}
	Hence, $ABCD$ is a parallelogram.
	\begin{enumerate}
		\item Now checking if the adjacent sides are orthogonal to each other
	\begin{align}
		(\vec{B}-\vec{A})^\top (\vec{C}-\vec{B}) = \myvec{3&1} \myvec{-3\\-3} = -9-3 = -12
	\end{align}
	Since inner product is not zero so adjacent sides are not orthogonal.

	Hence, we can say that $ABCD$ is neither a rectangle nor a square.

		\item Now checking if the diagonals are orthogonal then it is a Rhombus.
	\begin{align}
		(\vec{C}- \vec{A})^\top (\vec{D}-\vec{B}) = \myvec{0&-2} \myvec{-6\\-4} = 0+8 = 8
	\end{align}
	\end{enumerate}		
	Hence the diagonals are also not orthogonal so we conclude that $ABCD$ is a parallelogram.

	As shown in Figure \ref{fig:10/7/1/6/Fig3} we can see that $ABCD$ forms a parallelogram hence, our theoritical result is verified.

\begin{figure}[!h]
	\begin{center} 
	    \includegraphics[width=\columnwidth]{chapters/10/7/1/6/figs/quad3}
	\end{center}
\caption{}
\label{fig:10/7/1/6/Fig3}
\end{figure}
\end{enumerate}



\item If $\vec{a},\vec{b},\vec{c}$ are mutually perpendicular vectors of equal magnitudes, show that the vector $\vec{c}.\vec{d}$=15 is equally inclined to $\vec{a},\vec{b}$ and $\vec{c}$.\\
    \item If $ \vec{A},\vec{B},\vec{C} $ are mutually perpendicular vectors of equal magnitudes,show that the  $ \vec{A}+\vec{B}+\vec{C} $ is equally inclined to $ \vec{A},\vec{B}  \text{ and }  \vec{C} $.\\
    \textbf{Solution:}
    
    Suppose we have the following vectors:
    \begin{align*}
        \vec{v}_1 = \myvec{1 \\1 \\1}  \\
        \vec{v}_2 = \myvec{6\\ 4\\ 5}  \\
        \vec{v}_3 = \myvec{3 \\6\\ 9}
    \end{align*}
        

\textbf{Step 1: Initialize}

Set $\vec{u}_1 = \vec{v}_1$:\\

 $\vec{u}_1 = \myvec{1\\1\\1}$
 

\textbf{Step 2: Orthogonalization}

For  $ \vec{v}_2$ :
 \begin{align}
     \vec{u}_2 = \vec{v}_2 - \frac{\langle \vec{v}_2, \vec{u}_1 \rangle}{\langle \vec{u}_1, \vec{u}_1 \rangle} \vec{u}_1 \\
     \vec{u}_2=\vec{v}_2- \brak{\vec{u}_1 ^\top \vec{v}_2} \vec{u}_1\\ 
     \vec{u}_2\implies \myvec{1\\-1\\0}
 \end{align}

For $\vec{v}_3 $:
\begin{align}
    \vec{u}_3 = \vec{v}_3 - \frac{\langle \vec{v}_3, \vec{u}_1 \rangle}{\langle \vec{u}_1, \vec{u}_1 \rangle} \vec{u}_1 - \frac{\langle \vec{v}_3, \vec{u}_2 \rangle}{\langle \vec{u}_2, \vec{u}_2 \rangle} \vec{u}_2 \\
    \vec{u}_3=\vec{v}_3- \brak{\vec{u}_2 ^\top \vec{v}_3} \vec{u}_2- \brak{\vec{u}_1 ^\top \vec{v}_3} \vec{u}_1\\ 
\vec{u}_3\implies \myvec{-1\\-1\\2}
\end{align}

\textbf{Step 3: Normalization}

Normalize each vector:
\begin{align}
\vec{u}_1 = \frac{\vec{u}_1}{\norm{\vec{u}_1}} \\
\vec{u}_2 = \frac{\vec{u}_2}{\norm{\vec{u}_2}} \\
\vec{u}_3 = \frac{\vec{u}_3}{\norm{\vec{u}_3}} 
\end{align}

The final orthonormal basis is:
\begin{align*}
\vec{u}_1 = \myvec{\frac{1}{\sqrt{3}}\\ \frac{1}{\sqrt{3}}\\ \frac{1}{\sqrt{3}}} 
\implies\myvec{0.577\\0.577\\0.577}\\
\vec{u}_2 = \myvec{\frac{1}{\sqrt{2}}\\\frac{-1}{\sqrt{2}}\\0 }
\implies \myvec{0.707 \\ -0.707 \\ 0}\\
\vec{u}_3 = \myvec{\frac{-1}{\sqrt{6}}\\ \frac{-1}{\sqrt{6}}\\ \frac{2}{\sqrt{6}}} 
\implies \myvec{-0.408\\-0.408\\0.816}
\end{align*}
\textbf{Step 4: QR Decoposition}

we calculate Q by means of Gram–Schmidt process\\
$Q$ is an orthogonal matrix 
\begin{align*}
    Q=\myvec{ 0.577&0.707&-0.408\\0.577&-0.707&-0.408\\0.577&0&0.816}
\end{align*}
To verify it as a orthonormal matrix we have to check this property i.e,  $Q^{\top}.Q =I$
\begin{align*}
    \implies Q^\top Q &= \myvec{1&0&0\\0&1&0\\0&0&1}
\end{align*}
\textbf{Step 5: Findings angles between $\vec{u}_1,\vec{u}_2,\vec{u}_3  \text{ and } \vec{u}_1+\vec{u}_2+\vec{u}_3 $}
\begin{align}
    \vec{u}_1=\myvec{0.577\\0.577\\0.577}\\
    \vec{u}_2=\myvec{0.707 \\ -0.707 \\ 0} \\
    \vec{u}_3=\myvec{-0.408\\-0.408\\0.816}\\
    \vec{u}_1+\vec{u}_2+\vec{u}_3\implies\vec{y}=\myvec{0.876\\-0.538\\1.393}
\end{align}
Normalize each vector:\\
   \begin{align}
    \norm{\vec{u}_1}=1\\
    \norm{\vec{u}_2}=0.9998\\
     \norm{\vec{u}_3}=0.9987\\
     \norm{\vec{y}}=1.732
   \end{align}
Finding angles:
\begin{align}
    \cos{\theta_1}&=\frac{\myvec{0.577 \\0.577 \\0.577}\myvec{0.876&-0.538&1.393}}{\brak{1}\brak{1.732}}\\
    \cos{\theta_1}&=\frac{1}{1.732}\\
    \theta_1=\cos^{-1}\brak{\frac{1}{1.732}}
    \implies 54.73\degree
\end{align}
\begin{align}
   \cos{\theta_2}&=\frac{\myvec{0.707 \\-0.707 \\0}\myvec{0.876&-0.538&1.393}}{\brak{0.9998}\brak{1.732}}\\
    \cos{\theta_2}&=\frac{1}{1.732}\\
    \theta_2=\cos^{-1}\brak{\frac{1}{1.732}}
    \implies 54.73\degree
\end{align}
\begin{align}
     \cos{\theta_3}&=\frac{\myvec{-0.408 \\-0.408\\0.816}\myvec{0.876&-0.538&1.393}}{\brak{0.9987}\brak{1.732}}\\
    \cos{\theta_3}&=\frac{1}{1.732}\\
    \theta_3=\cos^{-1}\brak{\frac{1}{1.732}}
    \implies 54.73\degree
\end{align}
\begin{align*}
    \therefore  \theta_1=\theta_2=\theta_3
\end{align*}
Hence  we can say that $\vec{u}_1+\vec{u}_2+\vec{u}_3 $ is equally inclined to $\vec{u}_1,\vec{u}_2 \text{and} \vec{u}_3  $ 
\end{enumerate}

\subsection{Exercises}
\begin{enumerate}[label=\thesection.\arabic*,ref=\thesection.\theenumi]
\numberwithin{equation}{enumi}
\numberwithin{figure}{enumi}
\numberwithin{table}{enumi}
\item Check whether$(5,-2),(6,4)$ and $(7,-2)$ are the vertices of an isosceles triangle.
\item Name the type of quadrilateral formed, if any, by the following points,and give reasons for your answer
\begin{enumerate}
\item $(-1,-2),(1,0),(-1,2),(-3,0)$
\item $(-3,5),(-3,1),(0,3),(-1,-4)$
\item $(4,5),(7,6),(4,3),(1,2)$
\end{enumerate}
\solution
		\iffalse
\documentclass[12pt]{article}
\usepackage{graphicx}
\usepackage{amsmath}
\usepackage{mathtools}
\usepackage{gensymb}

\newcommand{\mydet}[1]{\ensuremath{\begin{vmatrix}#1\end{vmatrix}}}
\providecommand{\brak}[1]{\ensuremath{\left(#1\right)}}
\providecommand{\norm}[1]{\left\lVert#1\right\rVert}
\newcommand{\solution}{\noindent \textbf{Solution: }}
\newcommand{\myvec}[1]{\ensuremath{\begin{pmatrix}#1\end{pmatrix}}}
\let\vec\mathbf

\begin{document}
\begin{center}
\textbf\large{CHAPTER-7 \\ COORDINATE GEOMETRY}

\end{center}
\section*{Excercise 7.1}

Q6.Name the type of quadilateral formed,if any, by the following points, and give reasons for your answer:
\begin{enumerate}
	\item $\brak{-1,-2}, \brak{1,0}, \brak{-1,2}, \brak{-3,0}$ 
	\item $\brak{-3,5}, \brak{3,1}, \brak{0,3}, \brak{-1,-4}$
	\item $\brak{4,5}, \brak{7,6}, \brak{4,3}, \brak{1,2}$
\end{enumerate}
\solution
\fi
\begin{enumerate}
\item The coordinates are given as
	\begin{align}
	\vec{A} = \myvec{
		-1\\
		-2\\
		},
	\vec{B} = \myvec{
		1\\
		0\\
		},
	\vec{C} = \myvec{
		-1\\
		2\\
		} \text{ and }
	\vec{D} = \myvec{
		-3\\
		0\\
		}
	\end{align}
	\begin{align}
		\vec{B} - \vec{A} &= \myvec{1\\0} - \myvec{-1\\-2} = \myvec{2\\2}\\
		\vec{C} - \vec{B} &= \myvec{-1\\2} - \myvec{1\\0} = \myvec{-2\\2}\\
		\vec{C} - \vec{D} &= \myvec{-1\\2} - \myvec{-3\\0} = \myvec{2\\2}\\
		\vec{D} - \vec{A} &= \myvec{-3\\0} - \myvec{-1\\-2} = \myvec{-2\\2}
	\end{align}
	\begin{align}	
		\vec{C} - \vec{A} &= \myvec{-1\\2} - \myvec{-1\\-2} = \myvec{0\\4}\\
		\vec{D} - \vec{B} &= \myvec{-3\\0} - \myvec{1\\0} = \myvec{-4\\0}
	\end{align}
	\begin{align}	
		\vec{B}-\vec{A} = \vec{C}-\vec{D} \text{ and } \vec{C}-\vec{B} = \vec{D}-\vec{A}.
	\end{align}
	Hence, $ABCD$ is a parallelogram.
	\begin{enumerate}
		\item Now checking if the adjacent sides are orthogonal to each other
	\begin{align}
		(\vec{B}-\vec{A})^\top (\vec{C}-\vec{B}) = \myvec{2&2} \myvec{-2\\2} = -4+4 = 0
	\end{align}
		\item Now checking if the diagonals are also orthogonal then it is a square else a rectangle.
	\end{enumerate}	
	\begin{align}
		(\vec{C}-\vec{A})^\top (\vec{D}-\vec{B}) = \myvec{0&4} \myvec{-4\\0} = 0
	\end{align}
	Hence the diagonals are orthogonal to each other.

	So, we can conclude that $ABCD$ is a square.

	As shown in Figure \ref{fig:10/7/1/6/Fig1} we can see that $ABCD$ is a square hence we can conclude that our theoritical result is verified.
 
\begin{figure}[!h]
	\begin{center} 
	    \includegraphics[width=\columnwidth]{chapters/10/7/1/6/figs/quad1}
	\end{center}
\caption{}
\label{fig:10/7/1/6/Fig1}
\end{figure}

\item The coordinates are given as
	\begin{align}
	\vec{A} = \myvec{
		-3\\
		5\\
		},
	\vec{B} = \myvec{
		3\\
		1\\
		},
	\vec{C} = \myvec{
		0\\
		3\\
		} \text{ and }
	\vec{D} = \myvec{
		-1\\
		-4\\
		}
	\end{align}
	\begin{align}
		\vec{B} - \vec{A} &= \myvec{3\\1} - \myvec{-3\\5} = \myvec{6\\-4}\\
		\vec{C} - \vec{B} &= \myvec{0\\3} - \myvec{3\\1} = \myvec{-3\\2}\\
		\vec{C} - \vec{D} &= \myvec{0\\3} - \myvec{-1\\-4} = \myvec{1\\7}\\
		\vec{D} - \vec{A} &= \myvec{-1\\-4} - \myvec{-3\\5} = \myvec{2\\-9}
	\end{align}
	\begin{align}
		\vec{C} - \vec{A} &= \myvec{0\\3} - \myvec{-3\\5} = \myvec{3\\-2}\\
		\vec{D} - \vec{B} &= \myvec{-1\\-4} - \myvec{3\\1} = \myvec{-4\\-5}
	\end{align}
	\begin{align}
	\vec{B}-\vec{A} \neq \vec{C}-\vec{D} \text{ and } \vec{C}-\vec{B} \neq \vec{D}-\vec{A},
	\end{align}
	Hence, $ABCD$ is not a parallelogram, it can be a irregular quadilateral.
	\begin{enumerate}
		\item Now to check if any three points are collinear,

	if rank of $\myvec{\vec{B}-\vec{A} & \vec{C}-\vec{B}} = 1$ then points are collinear

	Forming the collinearity matrix
	\begin{align}
		\myvec{6&-3\\-4&2} \xleftrightarrow{R_{2}\rightarrow R_{2}+\frac{2}{3}R_{1}}= \myvec{6&-3\\0&0}
	\end{align}
	\end{enumerate}
	Hence, rank = 1

	Since none of the opposite sides are parallel to each other and three points are collinear so these does not form a quadilateral.

	As shown in Figure \ref{fig:10/7/1/6/Fig2} we can see that $ABCD$ does not form a quadilateral and three points are collinear hence, our theoritical result is verified.
	
\begin{figure}[!h]
	\begin{center} 
	    \includegraphics[width=\columnwidth]{chapters/10/7/1/6/figs/quad2}
	\end{center}
\caption{}
\label{fig:10/7/1/6/Fig2}
\end{figure}
	
\item The coordinates are given as
	\begin{align}
	\vec{A} = \myvec{
		4\\
		5\\
		},
	\vec{B} = \myvec{
		7\\
		6\\
		},
	\vec{C} = \myvec{
		4\\
		3\\
		} \text{ and }
	\vec{D} = \myvec{
		1\\
		2\\
		}
	\end{align}
	\begin{align}
		\vec{B} - \vec{A} &= \myvec{7\\6} - \myvec{4\\5} = \myvec{3\\1}\\
		\vec{C} - \vec{B} &= \myvec{4\\3} - \myvec{7\\6} = \myvec{-3\\-3}\\
		\vec{C} - \vec{D} &= \myvec{4\\3} - \myvec{1\\2} = \myvec{3\\1}\\
		\vec{D} - \vec{A} &= \myvec{1\\2} - \myvec{4\\5} = \myvec{-3\\-3}
	\end{align}
	\begin{align}
		\vec{C} - \vec{A} &= \myvec{4\\3} - \myvec{4\\5} = \myvec{0\\-2}\\
		\vec{D} - \vec{B} &= \myvec{1\\2} - \myvec{7\\6} = \myvec{-6\\-4}
	\end{align}
	\begin{align}
		\vec{B}-\vec{A} = \vec{C}-\vec{D} \text{ and } \vec{C}-\vec{B} = \vec{D}-\vec{A},
	\end{align}
	Hence, $ABCD$ is a parallelogram.
	\begin{enumerate}
		\item Now checking if the adjacent sides are orthogonal to each other
	\begin{align}
		(\vec{B}-\vec{A})^\top (\vec{C}-\vec{B}) = \myvec{3&1} \myvec{-3\\-3} = -9-3 = -12
	\end{align}
	Since inner product is not zero so adjacent sides are not orthogonal.

	Hence, we can say that $ABCD$ is neither a rectangle nor a square.

		\item Now checking if the diagonals are orthogonal then it is a Rhombus.
	\begin{align}
		(\vec{C}- \vec{A})^\top (\vec{D}-\vec{B}) = \myvec{0&-2} \myvec{-6\\-4} = 0+8 = 8
	\end{align}
	\end{enumerate}		
	Hence the diagonals are also not orthogonal so we conclude that $ABCD$ is a parallelogram.

	As shown in Figure \ref{fig:10/7/1/6/Fig3} we can see that $ABCD$ forms a parallelogram hence, our theoritical result is verified.

\begin{figure}[!h]
	\begin{center} 
	    \includegraphics[width=\columnwidth]{chapters/10/7/1/6/figs/quad3}
	\end{center}
\caption{}
\label{fig:10/7/1/6/Fig3}
\end{figure}
\end{enumerate}



\item Find the projection of the vector $\hat{i}-\hat{j}$ on the vector $\hat{i}+\hat{j}$.
	\\
		\iffalse
\documentclass[12pt]{chapters/10/7/4/3/figsarticle}
\usepackage{graphicx}
\usepackage[none]{chapters/10/7/4/3/figshyphenat}
\usepackage{graphicx}
\usepackage{listings}
\usepackage[english]{chapters/10/7/4/3/figsbabel}
\usepackage{graphicx}
\usepackage{caption} 
\usepackage{booktabs}
\usepackage{array}
\usepackage{amssymb} % for \because
\usepackage{amsmath}   % for having text in math mode
\usepackage{extarrows} % for Row operations arrows
\usepackage{listings}
\lstset{
  frame=single,
  breaklines=true
}
\usepackage{hyperref}
  
%Following 2 lines were added to remove the blank page at the beginning
\usepackage{atbegshi}% http://ctan.org/pkg/atbegshi
\AtBeginDocument{\AtBeginShipoutNext{\AtBeginShipoutDiscard}}


%New macro definitions
\newcommand{\mydet}[1]{chapters/10/7/4/3/figs\ensuremath{\begin{vmatrix}#1\end{vmatrix}}}
\providecommand{\brak}[1]{chapters/10/7/4/3/figs\ensuremath{\left(#1\right)}}
\newcommand{\solution}{\noindent \textbf{Solution: }}
\newcommand{\myvec}[1]{chapters/10/7/4/3/figs\ensuremath{\begin{pmatrix}#1\end{pmatrix}}}
\providecommand{\norm}[1]{chapters/10/7/4/3/figs\left\lVert#1\right\rVert}
\providecommand{\abs}[1]{chapters/10/7/4/3/figs\left\vert#1\right\vert}
\let\vec\mathbf


\begin{document}

\begin{center}
\title{\textbf{VECTORS}}
\date{\vspace{-5ex}} %Not to print date automatically
\maketitle
\end{center}

\setcounter{page}{1}

\section{10$^{th}$ Maths - Chapter 10}

This is Problem-3 from Exercise 10.3

\begin{enumerate}
\item Find the projection of the vector $\hat{i}-\hat{j}$ on the vector $\hat{i}+\hat{j}$  
\end{enumerate}
\section{SOLUTION}
\fi
\solution
The given points are
\begin{align}
 \vec{A}=\myvec{1\\ -1},
 \vec{B}=\myvec{1\\ 1}
\end{align}
Since
\begin{align}
	\vec{A}^\top \vec{B} &= \myvec{1 &-1} \myvec{1\\ 1}=\myvec{1 \times 1}+\myvec{-1 \times  1}=0
	\\
	\norm {\vec {B}}^2 &= (\vec{B}^\top  \vec{B})=\myvec{1 & 1} \myvec{1\\ 1}= (1 \times  1)+(1 \times  1)=2,
\end{align}
and the project vector is given by 
\begin{align}
	\vec{C} &= 
	\frac{\vec{A}^\top  \vec{B}}{\norm {\vec{B}}}^2 \vec{B}
	&=\frac{0}{2} \myvec{1\\ 1}
	=\myvec{0\\ 0}
\end{align}
This is verfied in Fig.
		\ref{fig:12/10/3/3Figure}.
\begin{figure}[h]
\includegraphics[width=\columnwidth]{chapters/12/10/3/3/figs/vector.png}
\caption{}
		\label{fig:12/10/3/3Figure}
\end{figure}

\item Find the projection of the vector $\hat{i}+3\hat{j}+7\hat{k}$ on the vector $7\hat{i}-\hat{j}+8\hat{k}$.
	\\
	\solution
		\iffalse
\documentclass[12pt]{article}
\usepackage{graphicx}
\usepackage{amsmath}
\usepackage{mathtools}
\usepackage{gensymb}

\newcommand{\mydet}[1]{\ensuremath{\begin{vmatrix}#1\end{vmatrix}}}
\providecommand{\brak}[1]{\ensuremath{\left(#1\right)}}
\providecommand{\norm}[1]{\left\lVert#1\right\rVert}
\newcommand{\solution}{\noindent \textbf{Solution: }}
\newcommand{\myvec}[1]{\ensuremath{\begin{pmatrix}#1\end{pmatrix}}}
\let\vec\mathbf

\begin{document}
\begin{center}
\textbf\large{CHAPTER-7 \\ COORDINATE GEOMETRY}

\end{center}
\section*{Excercise 7.1}

Q6.Name the type of quadilateral formed,if any, by the following points, and give reasons for your answer:
\begin{enumerate}
	\item $\brak{-1,-2}, \brak{1,0}, \brak{-1,2}, \brak{-3,0}$ 
	\item $\brak{-3,5}, \brak{3,1}, \brak{0,3}, \brak{-1,-4}$
	\item $\brak{4,5}, \brak{7,6}, \brak{4,3}, \brak{1,2}$
\end{enumerate}
\solution
\fi
\begin{enumerate}
\item The coordinates are given as
	\begin{align}
	\vec{A} = \myvec{
		-1\\
		-2\\
		},
	\vec{B} = \myvec{
		1\\
		0\\
		},
	\vec{C} = \myvec{
		-1\\
		2\\
		} \text{ and }
	\vec{D} = \myvec{
		-3\\
		0\\
		}
	\end{align}
	\begin{align}
		\vec{B} - \vec{A} &= \myvec{1\\0} - \myvec{-1\\-2} = \myvec{2\\2}\\
		\vec{C} - \vec{B} &= \myvec{-1\\2} - \myvec{1\\0} = \myvec{-2\\2}\\
		\vec{C} - \vec{D} &= \myvec{-1\\2} - \myvec{-3\\0} = \myvec{2\\2}\\
		\vec{D} - \vec{A} &= \myvec{-3\\0} - \myvec{-1\\-2} = \myvec{-2\\2}
	\end{align}
	\begin{align}	
		\vec{C} - \vec{A} &= \myvec{-1\\2} - \myvec{-1\\-2} = \myvec{0\\4}\\
		\vec{D} - \vec{B} &= \myvec{-3\\0} - \myvec{1\\0} = \myvec{-4\\0}
	\end{align}
	\begin{align}	
		\vec{B}-\vec{A} = \vec{C}-\vec{D} \text{ and } \vec{C}-\vec{B} = \vec{D}-\vec{A}.
	\end{align}
	Hence, $ABCD$ is a parallelogram.
	\begin{enumerate}
		\item Now checking if the adjacent sides are orthogonal to each other
	\begin{align}
		(\vec{B}-\vec{A})^\top (\vec{C}-\vec{B}) = \myvec{2&2} \myvec{-2\\2} = -4+4 = 0
	\end{align}
		\item Now checking if the diagonals are also orthogonal then it is a square else a rectangle.
	\end{enumerate}	
	\begin{align}
		(\vec{C}-\vec{A})^\top (\vec{D}-\vec{B}) = \myvec{0&4} \myvec{-4\\0} = 0
	\end{align}
	Hence the diagonals are orthogonal to each other.

	So, we can conclude that $ABCD$ is a square.

	As shown in Figure \ref{fig:10/7/1/6/Fig1} we can see that $ABCD$ is a square hence we can conclude that our theoritical result is verified.
 
\begin{figure}[!h]
	\begin{center} 
	    \includegraphics[width=\columnwidth]{chapters/10/7/1/6/figs/quad1}
	\end{center}
\caption{}
\label{fig:10/7/1/6/Fig1}
\end{figure}

\item The coordinates are given as
	\begin{align}
	\vec{A} = \myvec{
		-3\\
		5\\
		},
	\vec{B} = \myvec{
		3\\
		1\\
		},
	\vec{C} = \myvec{
		0\\
		3\\
		} \text{ and }
	\vec{D} = \myvec{
		-1\\
		-4\\
		}
	\end{align}
	\begin{align}
		\vec{B} - \vec{A} &= \myvec{3\\1} - \myvec{-3\\5} = \myvec{6\\-4}\\
		\vec{C} - \vec{B} &= \myvec{0\\3} - \myvec{3\\1} = \myvec{-3\\2}\\
		\vec{C} - \vec{D} &= \myvec{0\\3} - \myvec{-1\\-4} = \myvec{1\\7}\\
		\vec{D} - \vec{A} &= \myvec{-1\\-4} - \myvec{-3\\5} = \myvec{2\\-9}
	\end{align}
	\begin{align}
		\vec{C} - \vec{A} &= \myvec{0\\3} - \myvec{-3\\5} = \myvec{3\\-2}\\
		\vec{D} - \vec{B} &= \myvec{-1\\-4} - \myvec{3\\1} = \myvec{-4\\-5}
	\end{align}
	\begin{align}
	\vec{B}-\vec{A} \neq \vec{C}-\vec{D} \text{ and } \vec{C}-\vec{B} \neq \vec{D}-\vec{A},
	\end{align}
	Hence, $ABCD$ is not a parallelogram, it can be a irregular quadilateral.
	\begin{enumerate}
		\item Now to check if any three points are collinear,

	if rank of $\myvec{\vec{B}-\vec{A} & \vec{C}-\vec{B}} = 1$ then points are collinear

	Forming the collinearity matrix
	\begin{align}
		\myvec{6&-3\\-4&2} \xleftrightarrow{R_{2}\rightarrow R_{2}+\frac{2}{3}R_{1}}= \myvec{6&-3\\0&0}
	\end{align}
	\end{enumerate}
	Hence, rank = 1

	Since none of the opposite sides are parallel to each other and three points are collinear so these does not form a quadilateral.

	As shown in Figure \ref{fig:10/7/1/6/Fig2} we can see that $ABCD$ does not form a quadilateral and three points are collinear hence, our theoritical result is verified.
	
\begin{figure}[!h]
	\begin{center} 
	    \includegraphics[width=\columnwidth]{chapters/10/7/1/6/figs/quad2}
	\end{center}
\caption{}
\label{fig:10/7/1/6/Fig2}
\end{figure}
	
\item The coordinates are given as
	\begin{align}
	\vec{A} = \myvec{
		4\\
		5\\
		},
	\vec{B} = \myvec{
		7\\
		6\\
		},
	\vec{C} = \myvec{
		4\\
		3\\
		} \text{ and }
	\vec{D} = \myvec{
		1\\
		2\\
		}
	\end{align}
	\begin{align}
		\vec{B} - \vec{A} &= \myvec{7\\6} - \myvec{4\\5} = \myvec{3\\1}\\
		\vec{C} - \vec{B} &= \myvec{4\\3} - \myvec{7\\6} = \myvec{-3\\-3}\\
		\vec{C} - \vec{D} &= \myvec{4\\3} - \myvec{1\\2} = \myvec{3\\1}\\
		\vec{D} - \vec{A} &= \myvec{1\\2} - \myvec{4\\5} = \myvec{-3\\-3}
	\end{align}
	\begin{align}
		\vec{C} - \vec{A} &= \myvec{4\\3} - \myvec{4\\5} = \myvec{0\\-2}\\
		\vec{D} - \vec{B} &= \myvec{1\\2} - \myvec{7\\6} = \myvec{-6\\-4}
	\end{align}
	\begin{align}
		\vec{B}-\vec{A} = \vec{C}-\vec{D} \text{ and } \vec{C}-\vec{B} = \vec{D}-\vec{A},
	\end{align}
	Hence, $ABCD$ is a parallelogram.
	\begin{enumerate}
		\item Now checking if the adjacent sides are orthogonal to each other
	\begin{align}
		(\vec{B}-\vec{A})^\top (\vec{C}-\vec{B}) = \myvec{3&1} \myvec{-3\\-3} = -9-3 = -12
	\end{align}
	Since inner product is not zero so adjacent sides are not orthogonal.

	Hence, we can say that $ABCD$ is neither a rectangle nor a square.

		\item Now checking if the diagonals are orthogonal then it is a Rhombus.
	\begin{align}
		(\vec{C}- \vec{A})^\top (\vec{D}-\vec{B}) = \myvec{0&-2} \myvec{-6\\-4} = 0+8 = 8
	\end{align}
	\end{enumerate}		
	Hence the diagonals are also not orthogonal so we conclude that $ABCD$ is a parallelogram.

	As shown in Figure \ref{fig:10/7/1/6/Fig3} we can see that $ABCD$ forms a parallelogram hence, our theoritical result is verified.

\begin{figure}[!h]
	\begin{center} 
	    \includegraphics[width=\columnwidth]{chapters/10/7/1/6/figs/quad3}
	\end{center}
\caption{}
\label{fig:10/7/1/6/Fig3}
\end{figure}
\end{enumerate}



\item Show that each of the given three vectors is a unit vector: 
 $\frac{1}{7}(2\hat{i}+3\hat{j}+6\hat{k}),\frac{1}{7}(3\hat{i}-6\hat{j}+2\hat{k}),\frac{1}{7}(6\hat{i}+2\hat{j}-3\hat{k}$)
Also,show that they are mutually perpendicular to each other.
	\\
	\solution
		\iffalse
\documentclass[12pt]{article}
\usepackage{graphicx}
\usepackage{amsmath}
\usepackage{mathtools}
\usepackage{gensymb}

\newcommand{\mydet}[1]{\ensuremath{\begin{vmatrix}#1\end{vmatrix}}}
\providecommand{\brak}[1]{\ensuremath{\left(#1\right)}}
\providecommand{\norm}[1]{\left\lVert#1\right\rVert}
\newcommand{\solution}{\noindent \textbf{Solution: }}
\newcommand{\myvec}[1]{\ensuremath{\begin{pmatrix}#1\end{pmatrix}}}
\let\vec\mathbf

\begin{document}
\begin{center}
\textbf\large{CHAPTER-7 \\ COORDINATE GEOMETRY}

\end{center}
\section*{Excercise 7.1}

Q6.Name the type of quadilateral formed,if any, by the following points, and give reasons for your answer:
\begin{enumerate}
	\item $\brak{-1,-2}, \brak{1,0}, \brak{-1,2}, \brak{-3,0}$ 
	\item $\brak{-3,5}, \brak{3,1}, \brak{0,3}, \brak{-1,-4}$
	\item $\brak{4,5}, \brak{7,6}, \brak{4,3}, \brak{1,2}$
\end{enumerate}
\solution
\fi
\begin{enumerate}
\item The coordinates are given as
	\begin{align}
	\vec{A} = \myvec{
		-1\\
		-2\\
		},
	\vec{B} = \myvec{
		1\\
		0\\
		},
	\vec{C} = \myvec{
		-1\\
		2\\
		} \text{ and }
	\vec{D} = \myvec{
		-3\\
		0\\
		}
	\end{align}
	\begin{align}
		\vec{B} - \vec{A} &= \myvec{1\\0} - \myvec{-1\\-2} = \myvec{2\\2}\\
		\vec{C} - \vec{B} &= \myvec{-1\\2} - \myvec{1\\0} = \myvec{-2\\2}\\
		\vec{C} - \vec{D} &= \myvec{-1\\2} - \myvec{-3\\0} = \myvec{2\\2}\\
		\vec{D} - \vec{A} &= \myvec{-3\\0} - \myvec{-1\\-2} = \myvec{-2\\2}
	\end{align}
	\begin{align}	
		\vec{C} - \vec{A} &= \myvec{-1\\2} - \myvec{-1\\-2} = \myvec{0\\4}\\
		\vec{D} - \vec{B} &= \myvec{-3\\0} - \myvec{1\\0} = \myvec{-4\\0}
	\end{align}
	\begin{align}	
		\vec{B}-\vec{A} = \vec{C}-\vec{D} \text{ and } \vec{C}-\vec{B} = \vec{D}-\vec{A}.
	\end{align}
	Hence, $ABCD$ is a parallelogram.
	\begin{enumerate}
		\item Now checking if the adjacent sides are orthogonal to each other
	\begin{align}
		(\vec{B}-\vec{A})^\top (\vec{C}-\vec{B}) = \myvec{2&2} \myvec{-2\\2} = -4+4 = 0
	\end{align}
		\item Now checking if the diagonals are also orthogonal then it is a square else a rectangle.
	\end{enumerate}	
	\begin{align}
		(\vec{C}-\vec{A})^\top (\vec{D}-\vec{B}) = \myvec{0&4} \myvec{-4\\0} = 0
	\end{align}
	Hence the diagonals are orthogonal to each other.

	So, we can conclude that $ABCD$ is a square.

	As shown in Figure \ref{fig:10/7/1/6/Fig1} we can see that $ABCD$ is a square hence we can conclude that our theoritical result is verified.
 
\begin{figure}[!h]
	\begin{center} 
	    \includegraphics[width=\columnwidth]{chapters/10/7/1/6/figs/quad1}
	\end{center}
\caption{}
\label{fig:10/7/1/6/Fig1}
\end{figure}

\item The coordinates are given as
	\begin{align}
	\vec{A} = \myvec{
		-3\\
		5\\
		},
	\vec{B} = \myvec{
		3\\
		1\\
		},
	\vec{C} = \myvec{
		0\\
		3\\
		} \text{ and }
	\vec{D} = \myvec{
		-1\\
		-4\\
		}
	\end{align}
	\begin{align}
		\vec{B} - \vec{A} &= \myvec{3\\1} - \myvec{-3\\5} = \myvec{6\\-4}\\
		\vec{C} - \vec{B} &= \myvec{0\\3} - \myvec{3\\1} = \myvec{-3\\2}\\
		\vec{C} - \vec{D} &= \myvec{0\\3} - \myvec{-1\\-4} = \myvec{1\\7}\\
		\vec{D} - \vec{A} &= \myvec{-1\\-4} - \myvec{-3\\5} = \myvec{2\\-9}
	\end{align}
	\begin{align}
		\vec{C} - \vec{A} &= \myvec{0\\3} - \myvec{-3\\5} = \myvec{3\\-2}\\
		\vec{D} - \vec{B} &= \myvec{-1\\-4} - \myvec{3\\1} = \myvec{-4\\-5}
	\end{align}
	\begin{align}
	\vec{B}-\vec{A} \neq \vec{C}-\vec{D} \text{ and } \vec{C}-\vec{B} \neq \vec{D}-\vec{A},
	\end{align}
	Hence, $ABCD$ is not a parallelogram, it can be a irregular quadilateral.
	\begin{enumerate}
		\item Now to check if any three points are collinear,

	if rank of $\myvec{\vec{B}-\vec{A} & \vec{C}-\vec{B}} = 1$ then points are collinear

	Forming the collinearity matrix
	\begin{align}
		\myvec{6&-3\\-4&2} \xleftrightarrow{R_{2}\rightarrow R_{2}+\frac{2}{3}R_{1}}= \myvec{6&-3\\0&0}
	\end{align}
	\end{enumerate}
	Hence, rank = 1

	Since none of the opposite sides are parallel to each other and three points are collinear so these does not form a quadilateral.

	As shown in Figure \ref{fig:10/7/1/6/Fig2} we can see that $ABCD$ does not form a quadilateral and three points are collinear hence, our theoritical result is verified.
	
\begin{figure}[!h]
	\begin{center} 
	    \includegraphics[width=\columnwidth]{chapters/10/7/1/6/figs/quad2}
	\end{center}
\caption{}
\label{fig:10/7/1/6/Fig2}
\end{figure}
	
\item The coordinates are given as
	\begin{align}
	\vec{A} = \myvec{
		4\\
		5\\
		},
	\vec{B} = \myvec{
		7\\
		6\\
		},
	\vec{C} = \myvec{
		4\\
		3\\
		} \text{ and }
	\vec{D} = \myvec{
		1\\
		2\\
		}
	\end{align}
	\begin{align}
		\vec{B} - \vec{A} &= \myvec{7\\6} - \myvec{4\\5} = \myvec{3\\1}\\
		\vec{C} - \vec{B} &= \myvec{4\\3} - \myvec{7\\6} = \myvec{-3\\-3}\\
		\vec{C} - \vec{D} &= \myvec{4\\3} - \myvec{1\\2} = \myvec{3\\1}\\
		\vec{D} - \vec{A} &= \myvec{1\\2} - \myvec{4\\5} = \myvec{-3\\-3}
	\end{align}
	\begin{align}
		\vec{C} - \vec{A} &= \myvec{4\\3} - \myvec{4\\5} = \myvec{0\\-2}\\
		\vec{D} - \vec{B} &= \myvec{1\\2} - \myvec{7\\6} = \myvec{-6\\-4}
	\end{align}
	\begin{align}
		\vec{B}-\vec{A} = \vec{C}-\vec{D} \text{ and } \vec{C}-\vec{B} = \vec{D}-\vec{A},
	\end{align}
	Hence, $ABCD$ is a parallelogram.
	\begin{enumerate}
		\item Now checking if the adjacent sides are orthogonal to each other
	\begin{align}
		(\vec{B}-\vec{A})^\top (\vec{C}-\vec{B}) = \myvec{3&1} \myvec{-3\\-3} = -9-3 = -12
	\end{align}
	Since inner product is not zero so adjacent sides are not orthogonal.

	Hence, we can say that $ABCD$ is neither a rectangle nor a square.

		\item Now checking if the diagonals are orthogonal then it is a Rhombus.
	\begin{align}
		(\vec{C}- \vec{A})^\top (\vec{D}-\vec{B}) = \myvec{0&-2} \myvec{-6\\-4} = 0+8 = 8
	\end{align}
	\end{enumerate}		
	Hence the diagonals are also not orthogonal so we conclude that $ABCD$ is a parallelogram.

	As shown in Figure \ref{fig:10/7/1/6/Fig3} we can see that $ABCD$ forms a parallelogram hence, our theoritical result is verified.

\begin{figure}[!h]
	\begin{center} 
	    \includegraphics[width=\columnwidth]{chapters/10/7/1/6/figs/quad3}
	\end{center}
\caption{}
\label{fig:10/7/1/6/Fig3}
\end{figure}
\end{enumerate}



\item If $\overrightarrow {a}=2\hat{i}+2\hat{j}3\hat{k},\overrightarrow {b}=\hat{-i}+2\hat{j}+\hat{k}$ and $\overrightarrow {c}=3\hat{i}+\hat{j}$ are such that $\overrightarrow {a}+\lambda\overrightarrow {b}$ is perpendicular to $\overrightarrow {c}$,then find the value of $\lambda$.
	\\
		\iffalse
\documentclass[12pt]{article}
\usepackage{graphicx}
\usepackage{amsmath}
\usepackage{mathtools}
\usepackage{gensymb}

\newcommand{\mydet}[1]{\ensuremath{\begin{vmatrix}#1\end{vmatrix}}}
\providecommand{\brak}[1]{\ensuremath{\left(#1\right)}}
\providecommand{\norm}[1]{\left\lVert#1\right\rVert}
\newcommand{\solution}{\noindent \textbf{Solution: }}
\newcommand{\myvec}[1]{\ensuremath{\begin{pmatrix}#1\end{pmatrix}}}
\let\vec\mathbf

\begin{document}
\begin{center}
\textbf\large{CHAPTER-10 \\ VECTOR ALGEBRA}

\end{center}
\section*{Excercise 10.3}

Q10.If $\vec{a} = 2\hat{i}+2\hat{j}+3\hat{k}, \vec{b} = -\hat{i}+2\hat{j}+\hat{k} \text{ and } \vec{c} = 3\hat{i}+\hat{j}$ are such that $\vec{a}+\lambda \vec{b}$ is perpendicular to $\vec{c}$, then find the value of $\lambda$.
\fi
\solution
Given that
\begin{align}
	(\vec{a}+\lambda \vec{b})^{\top} \vec{c} &= 0\\
\implies \vec{a}^{\top}\vec{c}+\lambda \vec{b}^{\top}\vec{c}&=0\\
\implies 	\lambda \vec{b}^{\top}\vec{c}&=-\vec{a}^{\top}\vec{c}\\
\implies 	\lambda(\vec{b}^{\top}\vec{c})(\vec{b}^{\top}\vec{c})^{-1}&=-(\vec{a}^{\top}\vec{c})(\vec{b}^{\top}\vec{c})^{-1}\\
\implies 	\lambda&=-(\vec{a}^{\top}\vec{c})(\vec{b}^{\top}\vec{c})^{-1}
\end{align}
Now substituting the values
\begin{align}
	\vec{a}^{\top}\vec{c}&=\myvec{2&2&3} \myvec{3\\1\\0} = 8\\
	\vec{b}^{\top}\vec{c}&=\myvec{-1&2&1} \myvec{3\\1\\0} = -1,
\end{align}
\begin{align}
	\lambda&=-(\vec{a}^{\top}\vec{c})(\vec{b}^{\top}\vec{c})^{-1}\\
	&=-(8)(-1)^{-1}\\
	&=8
\end{align}



\item Show that $\abs {\overrightarrow {a}}\overrightarrow {b}+\abs{\overrightarrow {b}}\overrightarrow {a}$ is perpendicular to $\abs{\overrightarrow {a}} \overrightarrow {b}-\abs{\overrightarrow {b}} \overrightarrow {a}$, for any two nonzero vectors $\overrightarrow {a}$ and $\overrightarrow {b}$.
	\\
	\solution
		\iffalse
\documentclass[12pt]{article}
\usepackage{graphicx}
\usepackage{amsmath}
\usepackage{mathtools}
\usepackage{gensymb}

\newcommand{\mydet}[1]{\ensuremath{\begin{vmatrix}#1\end{vmatrix}}}
\providecommand{\brak}[1]{\ensuremath{\left(#1\right)}}
\providecommand{\norm}[1]{\left\lVert#1\right\rVert}
\newcommand{\solution}{\noindent \textbf{Solution: }}
\newcommand{\myvec}[1]{\ensuremath{\begin{pmatrix}#1\end{pmatrix}}}
\let\vec\mathbf

\begin{document}
\begin{center}
\textbf\large{CHAPTER-7 \\ COORDINATE GEOMETRY}

\end{center}
\section*{Excercise 7.1}

Q6.Name the type of quadilateral formed,if any, by the following points, and give reasons for your answer:
\begin{enumerate}
	\item $\brak{-1,-2}, \brak{1,0}, \brak{-1,2}, \brak{-3,0}$ 
	\item $\brak{-3,5}, \brak{3,1}, \brak{0,3}, \brak{-1,-4}$
	\item $\brak{4,5}, \brak{7,6}, \brak{4,3}, \brak{1,2}$
\end{enumerate}
\solution
\fi
\begin{enumerate}
\item The coordinates are given as
	\begin{align}
	\vec{A} = \myvec{
		-1\\
		-2\\
		},
	\vec{B} = \myvec{
		1\\
		0\\
		},
	\vec{C} = \myvec{
		-1\\
		2\\
		} \text{ and }
	\vec{D} = \myvec{
		-3\\
		0\\
		}
	\end{align}
	\begin{align}
		\vec{B} - \vec{A} &= \myvec{1\\0} - \myvec{-1\\-2} = \myvec{2\\2}\\
		\vec{C} - \vec{B} &= \myvec{-1\\2} - \myvec{1\\0} = \myvec{-2\\2}\\
		\vec{C} - \vec{D} &= \myvec{-1\\2} - \myvec{-3\\0} = \myvec{2\\2}\\
		\vec{D} - \vec{A} &= \myvec{-3\\0} - \myvec{-1\\-2} = \myvec{-2\\2}
	\end{align}
	\begin{align}	
		\vec{C} - \vec{A} &= \myvec{-1\\2} - \myvec{-1\\-2} = \myvec{0\\4}\\
		\vec{D} - \vec{B} &= \myvec{-3\\0} - \myvec{1\\0} = \myvec{-4\\0}
	\end{align}
	\begin{align}	
		\vec{B}-\vec{A} = \vec{C}-\vec{D} \text{ and } \vec{C}-\vec{B} = \vec{D}-\vec{A}.
	\end{align}
	Hence, $ABCD$ is a parallelogram.
	\begin{enumerate}
		\item Now checking if the adjacent sides are orthogonal to each other
	\begin{align}
		(\vec{B}-\vec{A})^\top (\vec{C}-\vec{B}) = \myvec{2&2} \myvec{-2\\2} = -4+4 = 0
	\end{align}
		\item Now checking if the diagonals are also orthogonal then it is a square else a rectangle.
	\end{enumerate}	
	\begin{align}
		(\vec{C}-\vec{A})^\top (\vec{D}-\vec{B}) = \myvec{0&4} \myvec{-4\\0} = 0
	\end{align}
	Hence the diagonals are orthogonal to each other.

	So, we can conclude that $ABCD$ is a square.

	As shown in Figure \ref{fig:10/7/1/6/Fig1} we can see that $ABCD$ is a square hence we can conclude that our theoritical result is verified.
 
\begin{figure}[!h]
	\begin{center} 
	    \includegraphics[width=\columnwidth]{chapters/10/7/1/6/figs/quad1}
	\end{center}
\caption{}
\label{fig:10/7/1/6/Fig1}
\end{figure}

\item The coordinates are given as
	\begin{align}
	\vec{A} = \myvec{
		-3\\
		5\\
		},
	\vec{B} = \myvec{
		3\\
		1\\
		},
	\vec{C} = \myvec{
		0\\
		3\\
		} \text{ and }
	\vec{D} = \myvec{
		-1\\
		-4\\
		}
	\end{align}
	\begin{align}
		\vec{B} - \vec{A} &= \myvec{3\\1} - \myvec{-3\\5} = \myvec{6\\-4}\\
		\vec{C} - \vec{B} &= \myvec{0\\3} - \myvec{3\\1} = \myvec{-3\\2}\\
		\vec{C} - \vec{D} &= \myvec{0\\3} - \myvec{-1\\-4} = \myvec{1\\7}\\
		\vec{D} - \vec{A} &= \myvec{-1\\-4} - \myvec{-3\\5} = \myvec{2\\-9}
	\end{align}
	\begin{align}
		\vec{C} - \vec{A} &= \myvec{0\\3} - \myvec{-3\\5} = \myvec{3\\-2}\\
		\vec{D} - \vec{B} &= \myvec{-1\\-4} - \myvec{3\\1} = \myvec{-4\\-5}
	\end{align}
	\begin{align}
	\vec{B}-\vec{A} \neq \vec{C}-\vec{D} \text{ and } \vec{C}-\vec{B} \neq \vec{D}-\vec{A},
	\end{align}
	Hence, $ABCD$ is not a parallelogram, it can be a irregular quadilateral.
	\begin{enumerate}
		\item Now to check if any three points are collinear,

	if rank of $\myvec{\vec{B}-\vec{A} & \vec{C}-\vec{B}} = 1$ then points are collinear

	Forming the collinearity matrix
	\begin{align}
		\myvec{6&-3\\-4&2} \xleftrightarrow{R_{2}\rightarrow R_{2}+\frac{2}{3}R_{1}}= \myvec{6&-3\\0&0}
	\end{align}
	\end{enumerate}
	Hence, rank = 1

	Since none of the opposite sides are parallel to each other and three points are collinear so these does not form a quadilateral.

	As shown in Figure \ref{fig:10/7/1/6/Fig2} we can see that $ABCD$ does not form a quadilateral and three points are collinear hence, our theoritical result is verified.
	
\begin{figure}[!h]
	\begin{center} 
	    \includegraphics[width=\columnwidth]{chapters/10/7/1/6/figs/quad2}
	\end{center}
\caption{}
\label{fig:10/7/1/6/Fig2}
\end{figure}
	
\item The coordinates are given as
	\begin{align}
	\vec{A} = \myvec{
		4\\
		5\\
		},
	\vec{B} = \myvec{
		7\\
		6\\
		},
	\vec{C} = \myvec{
		4\\
		3\\
		} \text{ and }
	\vec{D} = \myvec{
		1\\
		2\\
		}
	\end{align}
	\begin{align}
		\vec{B} - \vec{A} &= \myvec{7\\6} - \myvec{4\\5} = \myvec{3\\1}\\
		\vec{C} - \vec{B} &= \myvec{4\\3} - \myvec{7\\6} = \myvec{-3\\-3}\\
		\vec{C} - \vec{D} &= \myvec{4\\3} - \myvec{1\\2} = \myvec{3\\1}\\
		\vec{D} - \vec{A} &= \myvec{1\\2} - \myvec{4\\5} = \myvec{-3\\-3}
	\end{align}
	\begin{align}
		\vec{C} - \vec{A} &= \myvec{4\\3} - \myvec{4\\5} = \myvec{0\\-2}\\
		\vec{D} - \vec{B} &= \myvec{1\\2} - \myvec{7\\6} = \myvec{-6\\-4}
	\end{align}
	\begin{align}
		\vec{B}-\vec{A} = \vec{C}-\vec{D} \text{ and } \vec{C}-\vec{B} = \vec{D}-\vec{A},
	\end{align}
	Hence, $ABCD$ is a parallelogram.
	\begin{enumerate}
		\item Now checking if the adjacent sides are orthogonal to each other
	\begin{align}
		(\vec{B}-\vec{A})^\top (\vec{C}-\vec{B}) = \myvec{3&1} \myvec{-3\\-3} = -9-3 = -12
	\end{align}
	Since inner product is not zero so adjacent sides are not orthogonal.

	Hence, we can say that $ABCD$ is neither a rectangle nor a square.

		\item Now checking if the diagonals are orthogonal then it is a Rhombus.
	\begin{align}
		(\vec{C}- \vec{A})^\top (\vec{D}-\vec{B}) = \myvec{0&-2} \myvec{-6\\-4} = 0+8 = 8
	\end{align}
	\end{enumerate}		
	Hence the diagonals are also not orthogonal so we conclude that $ABCD$ is a parallelogram.

	As shown in Figure \ref{fig:10/7/1/6/Fig3} we can see that $ABCD$ forms a parallelogram hence, our theoritical result is verified.

\begin{figure}[!h]
	\begin{center} 
	    \includegraphics[width=\columnwidth]{chapters/10/7/1/6/figs/quad3}
	\end{center}
\caption{}
\label{fig:10/7/1/6/Fig3}
\end{figure}
\end{enumerate}



\item If $\overrightarrow {a}.\overrightarrow {a}$=0 and $\overrightarrow {a}.\overrightarrow {b}$=0, then what can be conculded about the vector $\overrightarrow {b}$?
\item If $\overrightarrow {a},\overrightarrow {b},\overrightarrow {c}$ are unit vectors such that $\overrightarrow {a}+\overrightarrow {b}+\overrightarrow {c}=\overrightarrow {0}$, find the value of $\overrightarrow {a}.\overrightarrow {b}+\overrightarrow {b}.\overrightarrow {c}+\overrightarrow {c}.\overrightarrow {a}$.
	\\
	\solution
		\iffalse
\documentclass[12pt]{article}
\usepackage{graphicx}
%\documentclass[journal,12pt,twocolumn]{IEEEtran}
\usepackage[none]{hyphenat}
\usepackage{graphicx}
\usepackage{listings}
\usepackage[english]{babel}
\usepackage{graphicx}
\usepackage{caption} 
\usepackage{hyperref}
\usepackage{booktabs}
\usepackage{commath}
\usepackage{gensymb}
\usepackage{array}
\usepackage{amsmath}   % for having text in math mode
\usepackage{listings}
\let\vec\mathbf
\lstset{
  frame=single,
  breaklines=true
}
  
%Following 2 lines were added to remove the blank page at the beginning
\usepackage{atbegshi}% http://ctan.org/pkg/atbegshi
\AtBeginDocument{\AtBeginShipoutNext{\AtBeginShipoutDiscard}}
%
%New macro definitions
\newcommand{\mydet}[1]{\ensuremath{\begin{vmatrix}#1\end{vmatrix}}}
\providecommand{\brak}[1]{\ensuremath{\left(#1\right)}}
\providecommand{\norm}[1]{\left\lVert#1\right\rVert}
\newcommand{\solution}{\noindent \textbf{Solution: }}
\newcommand{\myvec}[1]{\ensuremath{\begin{pmatrix}#1\end{pmatrix}}}
\let\vec\mathbf
\begin{document}
\begin{center}
\title{\textbf{Vector Algebra}}
\date{\vspace{-5ex}} %Not to print date automatically
\maketitle
\end{center}
\setcounter{page}{1}
\section*{CHAPTER 10 - VECTOR ALGEBRA}
\section*{Excercise 10.3}
\solution 
\begin{enumerate}
\item If $\overrightarrow{a},\overrightarrow{b},\overrightarrow{c}$ are unit vectors such that $\overrightarrow{a}+\overrightarrow{b}+\overrightarrow{c}=0$, find the value of $\overrightarrow{a}.\overrightarrow{b}+\overrightarrow{b}.\overrightarrow{c}+\overrightarrow{c}.\overrightarrow{a}$.  
\section{Solution}
The given vectors $\vec{a},\vec{b}$ and $\vec{c}$ are unit vectors. Since the given vectors $\vec{a},\vec{b},\vec{c}$ are unit vector hence $\vec{a}=\vec{b}=\vec{c}$ which is equal to 1.
        \begin{align}
\norm{\vec{a}} &=\sqrt{1^2}=1\\ \norm{\vec{b}}&=\sqrt{1^2}=1\\ \norm{\vec{c}}&=\sqrt{1^2}=1
        \end{align}
The Given equation is 
        \begin{align}
\vec{a}+\vec{b}+\vec{c}=0
\end{align}      
Squaring on both sides,
\fi
\begin{align}
	\norm{{\vec{a}}+{\vec{b}}+{\vec{c}}}^2&=0
	\\
	\implies{\norm{\vec{a}}}^2+{\norm{\vec{b}}}^2+{\norm{\vec{c}}}^2+2({{\vec{a}^\top}{\vec{b}}+{\vec{b}^\top}{\vec{c}}+{\vec{c}^\top}{\vec{a}}})&=0\\
	\implies3+2({{\vec{a}^\top}{\vec{b}}+{\vec{b}^\top}{\vec{c}}+{\vec{c}^\top}{\vec{a}}})&=0\\
	\implies{\vec{a}^\top}{\vec{b}}+{\vec{b}^\top}{\vec{c}}+{\vec{c}^\top}\vec{a}&=-\frac{3}{2}
\end{align}

\item If either vector $\overrightarrow {a}=0$ or $\overrightarrow {b}=0$, then $\overrightarrow {a}.\overrightarrow {b}$=0. But the converse need not be true. Justify your answer with an example.
	\\
	\solution
		\iffalse
\documentclass[12pt]{article}
\usepackage{graphicx}
\usepackage{amsmath}
\usepackage{mathtools}
\usepackage{gensymb}

\newcommand{\mydet}[1]{\ensuremath{\begin{vmatrix}#1\end{vmatrix}}}
\providecommand{\brak}[1]{\ensuremath{\left(#1\right)}}
\providecommand{\norm}[1]{\left\lVert#1\right\rVert}
\newcommand{\solution}{\noindent \textbf{Solution: }}
\newcommand{\myvec}[1]{\ensuremath{\begin{pmatrix}#1\end{pmatrix}}}
\let\vec\mathbf

\begin{document}
\begin{center}
\textbf\large{CHAPTER-7 \\ COORDINATE GEOMETRY}

\end{center}
\section*{Excercise 7.1}

Q6.Name the type of quadilateral formed,if any, by the following points, and give reasons for your answer:
\begin{enumerate}
	\item $\brak{-1,-2}, \brak{1,0}, \brak{-1,2}, \brak{-3,0}$ 
	\item $\brak{-3,5}, \brak{3,1}, \brak{0,3}, \brak{-1,-4}$
	\item $\brak{4,5}, \brak{7,6}, \brak{4,3}, \brak{1,2}$
\end{enumerate}
\solution
\fi
\begin{enumerate}
\item The coordinates are given as
	\begin{align}
	\vec{A} = \myvec{
		-1\\
		-2\\
		},
	\vec{B} = \myvec{
		1\\
		0\\
		},
	\vec{C} = \myvec{
		-1\\
		2\\
		} \text{ and }
	\vec{D} = \myvec{
		-3\\
		0\\
		}
	\end{align}
	\begin{align}
		\vec{B} - \vec{A} &= \myvec{1\\0} - \myvec{-1\\-2} = \myvec{2\\2}\\
		\vec{C} - \vec{B} &= \myvec{-1\\2} - \myvec{1\\0} = \myvec{-2\\2}\\
		\vec{C} - \vec{D} &= \myvec{-1\\2} - \myvec{-3\\0} = \myvec{2\\2}\\
		\vec{D} - \vec{A} &= \myvec{-3\\0} - \myvec{-1\\-2} = \myvec{-2\\2}
	\end{align}
	\begin{align}	
		\vec{C} - \vec{A} &= \myvec{-1\\2} - \myvec{-1\\-2} = \myvec{0\\4}\\
		\vec{D} - \vec{B} &= \myvec{-3\\0} - \myvec{1\\0} = \myvec{-4\\0}
	\end{align}
	\begin{align}	
		\vec{B}-\vec{A} = \vec{C}-\vec{D} \text{ and } \vec{C}-\vec{B} = \vec{D}-\vec{A}.
	\end{align}
	Hence, $ABCD$ is a parallelogram.
	\begin{enumerate}
		\item Now checking if the adjacent sides are orthogonal to each other
	\begin{align}
		(\vec{B}-\vec{A})^\top (\vec{C}-\vec{B}) = \myvec{2&2} \myvec{-2\\2} = -4+4 = 0
	\end{align}
		\item Now checking if the diagonals are also orthogonal then it is a square else a rectangle.
	\end{enumerate}	
	\begin{align}
		(\vec{C}-\vec{A})^\top (\vec{D}-\vec{B}) = \myvec{0&4} \myvec{-4\\0} = 0
	\end{align}
	Hence the diagonals are orthogonal to each other.

	So, we can conclude that $ABCD$ is a square.

	As shown in Figure \ref{fig:10/7/1/6/Fig1} we can see that $ABCD$ is a square hence we can conclude that our theoritical result is verified.
 
\begin{figure}[!h]
	\begin{center} 
	    \includegraphics[width=\columnwidth]{chapters/10/7/1/6/figs/quad1}
	\end{center}
\caption{}
\label{fig:10/7/1/6/Fig1}
\end{figure}

\item The coordinates are given as
	\begin{align}
	\vec{A} = \myvec{
		-3\\
		5\\
		},
	\vec{B} = \myvec{
		3\\
		1\\
		},
	\vec{C} = \myvec{
		0\\
		3\\
		} \text{ and }
	\vec{D} = \myvec{
		-1\\
		-4\\
		}
	\end{align}
	\begin{align}
		\vec{B} - \vec{A} &= \myvec{3\\1} - \myvec{-3\\5} = \myvec{6\\-4}\\
		\vec{C} - \vec{B} &= \myvec{0\\3} - \myvec{3\\1} = \myvec{-3\\2}\\
		\vec{C} - \vec{D} &= \myvec{0\\3} - \myvec{-1\\-4} = \myvec{1\\7}\\
		\vec{D} - \vec{A} &= \myvec{-1\\-4} - \myvec{-3\\5} = \myvec{2\\-9}
	\end{align}
	\begin{align}
		\vec{C} - \vec{A} &= \myvec{0\\3} - \myvec{-3\\5} = \myvec{3\\-2}\\
		\vec{D} - \vec{B} &= \myvec{-1\\-4} - \myvec{3\\1} = \myvec{-4\\-5}
	\end{align}
	\begin{align}
	\vec{B}-\vec{A} \neq \vec{C}-\vec{D} \text{ and } \vec{C}-\vec{B} \neq \vec{D}-\vec{A},
	\end{align}
	Hence, $ABCD$ is not a parallelogram, it can be a irregular quadilateral.
	\begin{enumerate}
		\item Now to check if any three points are collinear,

	if rank of $\myvec{\vec{B}-\vec{A} & \vec{C}-\vec{B}} = 1$ then points are collinear

	Forming the collinearity matrix
	\begin{align}
		\myvec{6&-3\\-4&2} \xleftrightarrow{R_{2}\rightarrow R_{2}+\frac{2}{3}R_{1}}= \myvec{6&-3\\0&0}
	\end{align}
	\end{enumerate}
	Hence, rank = 1

	Since none of the opposite sides are parallel to each other and three points are collinear so these does not form a quadilateral.

	As shown in Figure \ref{fig:10/7/1/6/Fig2} we can see that $ABCD$ does not form a quadilateral and three points are collinear hence, our theoritical result is verified.
	
\begin{figure}[!h]
	\begin{center} 
	    \includegraphics[width=\columnwidth]{chapters/10/7/1/6/figs/quad2}
	\end{center}
\caption{}
\label{fig:10/7/1/6/Fig2}
\end{figure}
	
\item The coordinates are given as
	\begin{align}
	\vec{A} = \myvec{
		4\\
		5\\
		},
	\vec{B} = \myvec{
		7\\
		6\\
		},
	\vec{C} = \myvec{
		4\\
		3\\
		} \text{ and }
	\vec{D} = \myvec{
		1\\
		2\\
		}
	\end{align}
	\begin{align}
		\vec{B} - \vec{A} &= \myvec{7\\6} - \myvec{4\\5} = \myvec{3\\1}\\
		\vec{C} - \vec{B} &= \myvec{4\\3} - \myvec{7\\6} = \myvec{-3\\-3}\\
		\vec{C} - \vec{D} &= \myvec{4\\3} - \myvec{1\\2} = \myvec{3\\1}\\
		\vec{D} - \vec{A} &= \myvec{1\\2} - \myvec{4\\5} = \myvec{-3\\-3}
	\end{align}
	\begin{align}
		\vec{C} - \vec{A} &= \myvec{4\\3} - \myvec{4\\5} = \myvec{0\\-2}\\
		\vec{D} - \vec{B} &= \myvec{1\\2} - \myvec{7\\6} = \myvec{-6\\-4}
	\end{align}
	\begin{align}
		\vec{B}-\vec{A} = \vec{C}-\vec{D} \text{ and } \vec{C}-\vec{B} = \vec{D}-\vec{A},
	\end{align}
	Hence, $ABCD$ is a parallelogram.
	\begin{enumerate}
		\item Now checking if the adjacent sides are orthogonal to each other
	\begin{align}
		(\vec{B}-\vec{A})^\top (\vec{C}-\vec{B}) = \myvec{3&1} \myvec{-3\\-3} = -9-3 = -12
	\end{align}
	Since inner product is not zero so adjacent sides are not orthogonal.

	Hence, we can say that $ABCD$ is neither a rectangle nor a square.

		\item Now checking if the diagonals are orthogonal then it is a Rhombus.
	\begin{align}
		(\vec{C}- \vec{A})^\top (\vec{D}-\vec{B}) = \myvec{0&-2} \myvec{-6\\-4} = 0+8 = 8
	\end{align}
	\end{enumerate}		
	Hence the diagonals are also not orthogonal so we conclude that $ABCD$ is a parallelogram.

	As shown in Figure \ref{fig:10/7/1/6/Fig3} we can see that $ABCD$ forms a parallelogram hence, our theoritical result is verified.

\begin{figure}[!h]
	\begin{center} 
	    \includegraphics[width=\columnwidth]{chapters/10/7/1/6/figs/quad3}
	\end{center}
\caption{}
\label{fig:10/7/1/6/Fig3}
\end{figure}
\end{enumerate}



\item Show that the vectors $2\hat{i}-\hat{j}+\hat{k},\hat{i}-3\hat{j}-5\hat{k}$ and  $3\hat{i}-4\hat{j}-4\hat{k}$ from the vertices of a right angled triangle.
	\\
	\solution
		\iffalse
\documentclass[journal,12pt,twocolumn]{IEEEtran}
\usepackage{setspace}
\usepackage{gensymb}
\singlespacing
\usepackage[cmex10]{amsmath}
\usepackage{amsthm}
\usepackage{mathrsfs}
\usepackage{txfonts}
\usepackage{stfloats}
\usepackage{bm}
\usepackage{cite}
\usepackage{cases}
\usepackage{subfig}
\usepackage{longtable}
\usepackage{multirow}
\usepackage{enumitem}
\usepackage{mathtools}
\usepackage{steinmetz}
\usepackage{tikz}
\usepackage{circuitikz}
\usepackage{verbatim}
\usepackage{tfrupee}
\usepackage[breaklinks=true]{hyperref}
\usepackage{tkz-euclide}
\usetikzlibrary{calc,math}
\usepackage{listings}
    \usepackage{color}                                            %%
    \usepackage{array}                                            %%
    \usepackage{longtable}                                        %%
    \usepackage{calc}                                             %%
    \usepackage{multirow}                                         %%
    \usepackage{hhline}                                           %%
    \usepackage{ifthen}                                           %%
  %optionally (for landscape tables embedded in another document): %%
    \usepackage{lscape}     
\usepackage{multicol}
\usepackage{chngcntr}
\DeclareMathOperator*{\Res}{Res}
\renewcommand\thesection{\arabic{section}}
\renewcommand\thesubsection{\thesection.\arabic{subsection}}
\renewcommand\thesubsubsection{\thesubsection.\arabic{subsubsection}}

\renewcommand\thesectiondis{\arabic{section}}
\renewcommand\thesubsectiondis{\thesectiondis.\arabic{subsection}}
\renewcommand\thesubsubsectiondis{\thesubsectiondis.\arabic{subsubsection}}

% correct bad hyphenation here
\hyphenation{op-tical net-works semi-conduc-tor}
\def\inputGnumericTable{}                                 %%

\lstset{
frame=single, 
breaklines=true,
columns=fullflexible
}

\begin{document}


\newtheorem{theorem}{Theorem}[section]
\newtheorem{problem}{Problem}
\newtheorem{proposition}{Proposition}[section]
\newtheorem{lemma}{Lemma}[section]
\newtheorem{corollary}[theorem]{Corollary}
\newtheorem{example}{Example}[section]
\newtheorem{definition}[problem]{Definition}
\newcommand{\BEQA}{\begin{eqnarray}}
\newcommand{\EEQA}{\end{eqnarray}}
\newcommand{\define}{\stackrel{\triangle}{=}}

\bibliographystyle{IEEEtran}
\providecommand{\mbf}{\mathbf}
\providecommand{\pr}[1]{\ensuremath{\Pr\left(#1\right)}}
\providecommand{\qfunc}[1]{\ensuremath{Q\left(#1\right)}}
\providecommand{\sbrak}[1]{\ensuremath{{}\left[#1\right]}}
\providecommand{\lsbrak}[1]{\ensuremath{{}\left[#1\right.}}
\providecommand{\rsbrak}[1]{\ensuremath{{}\left.#1\right]}}
\providecommand{\brak}[1]{\ensuremath{\left(#1\right)}}
\providecommand{\lbrak}[1]{\ensuremath{\left(#1\right.}}
\providecommand{\rbrak}[1]{\ensuremath{\left.#1\right)}}
\providecommand{\cbrak}[1]{\ensuremath{\left\{#1\right\}}}
\providecommand{\lcbrak}[1]{\ensuremath{\left\{#1\right.}}
\providecommand{\rcbrak}[1]{\ensuremath{\left.#1\right\}}}
\theoremstyle{remark}
\newtheorem{rem}{Remark}
\newcommand{\sgn}{\mathop{\mathrm{sgn}}}
\providecommand{\abs}[1]{\left\vert#1\right\vert}
\providecommand{\res}[1]{\Res\displaylimits_{#1}} 
\providecommand{\norm}[1]{\left\lVert#1\right\rVert}
\providecommand{\mtx}[1]{\mathbf{#1}}
\providecommand{\mean}[1]{E\left[ #1 \right]}
\providecommand{\fourier}{\overset{\mathcal{F}}{ \rightleftharpoons}}
\providecommand{\system}{\overset{\mathcal{H}}{ \longleftrightarrow}}
\newcommand{\solution}{\noindent \textbf{Solution: }}
\newcommand{\cosec}{\,\text{cosec}\,}
\providecommand{\dec}[2]{\ensuremath{\overset{#1}{\underset{#2}{\gtrless}}}}
\newcommand{\myvec}[1]{\ensuremath{\begin{pmatrix}#1\end{pmatrix}}}
\newcommand{\mydet}[1]{\ensuremath{\begin{vmatrix}#1\end{vmatrix}}}
\numberwithin{equation}{subsection}
\makeatletter
\@addtoreset{figure}{problem}
\makeatother

\let\StandardTheFigure\thefigure
\let\vec\mathbf
\renewcommand{\thefigure}{\theproblem}



\def\putbox#1#2#3{\makebox[0in][l]{\makebox[#1][l]{}\raisebox{\baselineskip}[0in][0in]{\raisebox{#2}[0in][0in]{#3}}}}
     \def\rightbox#1{\makebox[0in][r]{#1}}
     \def\centbox#1{\makebox[0in]{#1}}
     \def\topbox#1{\raisebox{-\baselineskip}[0in][0in]{#1}}
     \def\midbox#1{\raisebox{-0.5\baselineskip}[0in][0in]{#1}}

\vspace{3cm}


\title{Assignment 1}
\author{Jaswanth Chowdary Madala}





% make the title area
\maketitle

\newpage

%\tableofcontents

\bigskip

\renewcommand{\thefigure}{\theenumi}
\renewcommand{\thetable}{\theenumi}


\begin{enumerate}

\item Show that the vectors $2\hat{i}-\hat{j}+\hat{k}$, $\hat{i}-3\hat{j}-5\hat{k}$ and $3\hat{i}-4\hat{j}-4\hat{k}$ form the vertices of a right angled triangle.
\fi
Let
\begin{align}
\vec{A} = \myvec{2\\-1\\1}, \, \vec{B} = \myvec{1\\-3\\-5}, \, \vec{C}=\myvec{3\\-4\\-4} 
\end{align}
 Form the matrix 
\begin{align}
\myvec{\vec{A}&\vec{B}&\vec{C}} = \myvec{2&1&3\\-1&-3&-4\\1&-5&-4}\\
\xleftrightarrow [R_2 \leftarrow R_2+\frac{1}{2}R_1]{R_3 \leftarrow R_3-\frac{1}{2}R_1} \\
\myvec{2&1&3\\ \\0&-\dfrac{5}{2}&-\dfrac{5}{2}\\\\0&-\dfrac{11}{2}&-\dfrac{11}{2}}\\
\xleftrightarrow[]{R_3 \leftarrow R_3-\frac{11}{5}R_2}\\
\myvec{2&1&3\\ \\0&-\dfrac{5}{2}&-\dfrac{5}{2}\\\\0&0&0},
\end{align}
the rank of the matrix is 2 and the points are in 3-Dimensional space, so the points $\vec{A},\vec{B},\vec{C}$ form a triangle.
\begin{enumerate}
\item checking whether the triangle is right angled at $\vec{A}$
\begin{align}
\vec{B}-\vec{A} &= \myvec{-1\\-2\\-6} \\
\vec{C}-\vec{A} &= \myvec{1\\-3\\-5} \\
\brak{\vec{B}-\vec{A}}^{\top}\brak{\vec{C}-\vec{A}} &= \myvec{-1&-2&-6}\myvec{1\\-3\\-5} = 35
\neq 0
\end{align}
The triangle is not right angled at $\vec{A}$.
%
\item checking whether the triangle is right angled at $\vec{B}$
\begin{align}
\vec{A}-\vec{B} &= \myvec{1\\2\\6} \\
\vec{C}-\vec{B} &= \myvec{2\\-1\\1} 
\end{align}
\begin{align}
\brak{\vec{A}-\vec{B}}^{\top}\brak{\vec{C}-\vec{B}} &= \myvec{1&2&6}\myvec{2\\-1\\1} = 6
\neq 0
\end{align}
The triangle is not right angled at $\vec{B}$.
%
\item checking whether the triangle is right angled at $\vec{C}$
\begin{align}
\vec{A}-\vec{C} &= \myvec{-1\\3\\5} \\
\vec{B}-\vec{C} &= \myvec{-2\\1\\-1} \\
\brak{\vec{A}-\vec{C}}^{\top}\brak{\vec{B}-\vec{C}} &= \myvec{-1&3&5}\myvec{-2\\1\\-1} = 0\\
\end{align}
Hence the triangle is right angled at $\vec{C}$.
\end{enumerate}





\item Show that the points A, B and C with position vectors,$\vec{a}=3\hat{i}-4\hat{j}-4\hat{k}$,$\vec{b}=2\hat{i}-\hat{j}+\hat{k}$ and $\vec{c}=\hat{i}-3\hat{j}-5\hat{k}$, respectively form the vertices of a right angled
triangle.
\\
\solution
		\iffalse
\documentclass[12pt]{article}
\usepackage{graphicx}
\usepackage{amsmath}
\usepackage{mathtools}
\usepackage{gensymb}

\newcommand{\mydet}[1]{\ensuremath{\begin{vmatrix}#1\end{vmatrix}}}
\providecommand{\brak}[1]{\ensuremath{\left(#1\right)}}
\providecommand{\norm}[1]{\left\lVert#1\right\rVert}
\newcommand{\solution}{\noindent \textbf{Solution: }}
\newcommand{\myvec}[1]{\ensuremath{\begin{pmatrix}#1\end{pmatrix}}}
\let\vec\mathbf

\begin{document}
\begin{center}
\textbf\large{CHAPTER-7 \\ COORDINATE GEOMETRY}

\end{center}
\section*{Excercise 7.1}

Q6.Name the type of quadilateral formed,if any, by the following points, and give reasons for your answer:
\begin{enumerate}
	\item $\brak{-1,-2}, \brak{1,0}, \brak{-1,2}, \brak{-3,0}$ 
	\item $\brak{-3,5}, \brak{3,1}, \brak{0,3}, \brak{-1,-4}$
	\item $\brak{4,5}, \brak{7,6}, \brak{4,3}, \brak{1,2}$
\end{enumerate}
\solution
\fi
\begin{enumerate}
\item The coordinates are given as
	\begin{align}
	\vec{A} = \myvec{
		-1\\
		-2\\
		},
	\vec{B} = \myvec{
		1\\
		0\\
		},
	\vec{C} = \myvec{
		-1\\
		2\\
		} \text{ and }
	\vec{D} = \myvec{
		-3\\
		0\\
		}
	\end{align}
	\begin{align}
		\vec{B} - \vec{A} &= \myvec{1\\0} - \myvec{-1\\-2} = \myvec{2\\2}\\
		\vec{C} - \vec{B} &= \myvec{-1\\2} - \myvec{1\\0} = \myvec{-2\\2}\\
		\vec{C} - \vec{D} &= \myvec{-1\\2} - \myvec{-3\\0} = \myvec{2\\2}\\
		\vec{D} - \vec{A} &= \myvec{-3\\0} - \myvec{-1\\-2} = \myvec{-2\\2}
	\end{align}
	\begin{align}	
		\vec{C} - \vec{A} &= \myvec{-1\\2} - \myvec{-1\\-2} = \myvec{0\\4}\\
		\vec{D} - \vec{B} &= \myvec{-3\\0} - \myvec{1\\0} = \myvec{-4\\0}
	\end{align}
	\begin{align}	
		\vec{B}-\vec{A} = \vec{C}-\vec{D} \text{ and } \vec{C}-\vec{B} = \vec{D}-\vec{A}.
	\end{align}
	Hence, $ABCD$ is a parallelogram.
	\begin{enumerate}
		\item Now checking if the adjacent sides are orthogonal to each other
	\begin{align}
		(\vec{B}-\vec{A})^\top (\vec{C}-\vec{B}) = \myvec{2&2} \myvec{-2\\2} = -4+4 = 0
	\end{align}
		\item Now checking if the diagonals are also orthogonal then it is a square else a rectangle.
	\end{enumerate}	
	\begin{align}
		(\vec{C}-\vec{A})^\top (\vec{D}-\vec{B}) = \myvec{0&4} \myvec{-4\\0} = 0
	\end{align}
	Hence the diagonals are orthogonal to each other.

	So, we can conclude that $ABCD$ is a square.

	As shown in Figure \ref{fig:10/7/1/6/Fig1} we can see that $ABCD$ is a square hence we can conclude that our theoritical result is verified.
 
\begin{figure}[!h]
	\begin{center} 
	    \includegraphics[width=\columnwidth]{chapters/10/7/1/6/figs/quad1}
	\end{center}
\caption{}
\label{fig:10/7/1/6/Fig1}
\end{figure}

\item The coordinates are given as
	\begin{align}
	\vec{A} = \myvec{
		-3\\
		5\\
		},
	\vec{B} = \myvec{
		3\\
		1\\
		},
	\vec{C} = \myvec{
		0\\
		3\\
		} \text{ and }
	\vec{D} = \myvec{
		-1\\
		-4\\
		}
	\end{align}
	\begin{align}
		\vec{B} - \vec{A} &= \myvec{3\\1} - \myvec{-3\\5} = \myvec{6\\-4}\\
		\vec{C} - \vec{B} &= \myvec{0\\3} - \myvec{3\\1} = \myvec{-3\\2}\\
		\vec{C} - \vec{D} &= \myvec{0\\3} - \myvec{-1\\-4} = \myvec{1\\7}\\
		\vec{D} - \vec{A} &= \myvec{-1\\-4} - \myvec{-3\\5} = \myvec{2\\-9}
	\end{align}
	\begin{align}
		\vec{C} - \vec{A} &= \myvec{0\\3} - \myvec{-3\\5} = \myvec{3\\-2}\\
		\vec{D} - \vec{B} &= \myvec{-1\\-4} - \myvec{3\\1} = \myvec{-4\\-5}
	\end{align}
	\begin{align}
	\vec{B}-\vec{A} \neq \vec{C}-\vec{D} \text{ and } \vec{C}-\vec{B} \neq \vec{D}-\vec{A},
	\end{align}
	Hence, $ABCD$ is not a parallelogram, it can be a irregular quadilateral.
	\begin{enumerate}
		\item Now to check if any three points are collinear,

	if rank of $\myvec{\vec{B}-\vec{A} & \vec{C}-\vec{B}} = 1$ then points are collinear

	Forming the collinearity matrix
	\begin{align}
		\myvec{6&-3\\-4&2} \xleftrightarrow{R_{2}\rightarrow R_{2}+\frac{2}{3}R_{1}}= \myvec{6&-3\\0&0}
	\end{align}
	\end{enumerate}
	Hence, rank = 1

	Since none of the opposite sides are parallel to each other and three points are collinear so these does not form a quadilateral.

	As shown in Figure \ref{fig:10/7/1/6/Fig2} we can see that $ABCD$ does not form a quadilateral and three points are collinear hence, our theoritical result is verified.
	
\begin{figure}[!h]
	\begin{center} 
	    \includegraphics[width=\columnwidth]{chapters/10/7/1/6/figs/quad2}
	\end{center}
\caption{}
\label{fig:10/7/1/6/Fig2}
\end{figure}
	
\item The coordinates are given as
	\begin{align}
	\vec{A} = \myvec{
		4\\
		5\\
		},
	\vec{B} = \myvec{
		7\\
		6\\
		},
	\vec{C} = \myvec{
		4\\
		3\\
		} \text{ and }
	\vec{D} = \myvec{
		1\\
		2\\
		}
	\end{align}
	\begin{align}
		\vec{B} - \vec{A} &= \myvec{7\\6} - \myvec{4\\5} = \myvec{3\\1}\\
		\vec{C} - \vec{B} &= \myvec{4\\3} - \myvec{7\\6} = \myvec{-3\\-3}\\
		\vec{C} - \vec{D} &= \myvec{4\\3} - \myvec{1\\2} = \myvec{3\\1}\\
		\vec{D} - \vec{A} &= \myvec{1\\2} - \myvec{4\\5} = \myvec{-3\\-3}
	\end{align}
	\begin{align}
		\vec{C} - \vec{A} &= \myvec{4\\3} - \myvec{4\\5} = \myvec{0\\-2}\\
		\vec{D} - \vec{B} &= \myvec{1\\2} - \myvec{7\\6} = \myvec{-6\\-4}
	\end{align}
	\begin{align}
		\vec{B}-\vec{A} = \vec{C}-\vec{D} \text{ and } \vec{C}-\vec{B} = \vec{D}-\vec{A},
	\end{align}
	Hence, $ABCD$ is a parallelogram.
	\begin{enumerate}
		\item Now checking if the adjacent sides are orthogonal to each other
	\begin{align}
		(\vec{B}-\vec{A})^\top (\vec{C}-\vec{B}) = \myvec{3&1} \myvec{-3\\-3} = -9-3 = -12
	\end{align}
	Since inner product is not zero so adjacent sides are not orthogonal.

	Hence, we can say that $ABCD$ is neither a rectangle nor a square.

		\item Now checking if the diagonals are orthogonal then it is a Rhombus.
	\begin{align}
		(\vec{C}- \vec{A})^\top (\vec{D}-\vec{B}) = \myvec{0&-2} \myvec{-6\\-4} = 0+8 = 8
	\end{align}
	\end{enumerate}		
	Hence the diagonals are also not orthogonal so we conclude that $ABCD$ is a parallelogram.

	As shown in Figure \ref{fig:10/7/1/6/Fig3} we can see that $ABCD$ forms a parallelogram hence, our theoritical result is verified.

\begin{figure}[!h]
	\begin{center} 
	    \includegraphics[width=\columnwidth]{chapters/10/7/1/6/figs/quad3}
	\end{center}
\caption{}
\label{fig:10/7/1/6/Fig3}
\end{figure}
\end{enumerate}



\item Let $\vec{a}=\hat{i}+4\hat{j}+2\hat{k}$,$\vec{b}=3\hat{i}-2\hat{j}+7\hat{k}$ and $\vec{c}=2\hat{i}-\hat{j}+4\hat{k}$.Find a vector $\vec{d}$ which is perpendicular to both $\vec{a}$ and $\vec{b}$,and $\vec{c}.\vec{d}$=15.\\
	\solution
		\begin{enumerate}[label=\thesection.\arabic*,ref=\thesection.\theenumi]
\numberwithin{equation}{enumi}
\numberwithin{figure}{enumi}
\numberwithin{table}{enumi}
\item  Find the vector equation of the line which is parallel to the vector $3\hat{i}-2\hat{j}+6\hat{k}$ and which passes through the point $(1,-2,3)$.
\item Find the equations of the two lines through the origin which intersect the line $ \dfrac{x-3}{2}=\dfrac{y-3}{1}=\dfrac{z}{1}$ at angles of  $\dfrac{\pi}{3}$each.
\item Find the equations of the line passing through the point $(3,0,1)$ and parallel to the planes $x+2y=0$ and $3y-z=0.$
\item The vector equation of the line $\dfrac{x-5}{3}=\dfrac{y+4}{7}=\dfrac{z-6}{2}$ is \noindent\rule{2cm}{0.4pt}. 
\item The vector equation of the line through the points $(3,4,-7)$ and $(1,-1,6)$ is \noindent\rule{2cm}{0.4pt}.
\item the unit vector normal to the plane $x+2y+3z-6=0$ is $\dfrac{1}{\sqrt{14}}\hat{i} + \dfrac{2}{\sqrt{14}}\hat{j} + \dfrac{3}{\sqrt{14}}\hat{k}$.
\item The vector equation of the line $\dfrac{x-5}{3}=\dfrac{y+4}{7}=\dfrac{z-6}{2}$ is
$$\overrightarrow{r}=5\hat{i}-4\hat{j}+6\hat{k}+\lambda(3\hat{i}+7\hat{j}+2\hat{k}).$$
\item The equation of a line, which is parallel to $2\hat{i}+\hat{j}+3\hat{k}$ and which passes through the point $(5,-2,4)$ is $\dfrac{x-5}{2}=\dfrac{y+2}{-1}=\dfrac{z-4}{3}$.
\end{enumerate}

\item Prove that $(\vec{a}+\vec{b}).(\vec{a}+\vec{b})$=$|{\vec{a}}|^2+|{\vec{b}}|^2$,if and only if $\vec{a},\vec{b}$ are perpendicular, given $\vec{a}\neq\vec{0}$,$\vec{b}\neq\vec{0}$.\\
	\solution
		\iffalse
\documentclass[12pt]{article}
\usepackage{graphicx}
\usepackage{amsmath}
\usepackage{mathtools}
\usepackage{gensymb}

\newcommand{\mydet}[1]{\ensuremath{\begin{vmatrix}#1\end{vmatrix}}}
\providecommand{\brak}[1]{\ensuremath{\left(#1\right)}}
\providecommand{\norm}[1]{\left\lVert#1\right\rVert}
\newcommand{\solution}{\noindent \textbf{Solution: }}
\newcommand{\myvec}[1]{\ensuremath{\begin{pmatrix}#1\end{pmatrix}}}
\let\vec\mathbf

\begin{document}
\begin{center}
\textbf\large{CHAPTER-7 \\ COORDINATE GEOMETRY}

\end{center}
\section*{Excercise 7.1}

Q6.Name the type of quadilateral formed,if any, by the following points, and give reasons for your answer:
\begin{enumerate}
	\item $\brak{-1,-2}, \brak{1,0}, \brak{-1,2}, \brak{-3,0}$ 
	\item $\brak{-3,5}, \brak{3,1}, \brak{0,3}, \brak{-1,-4}$
	\item $\brak{4,5}, \brak{7,6}, \brak{4,3}, \brak{1,2}$
\end{enumerate}
\solution
\fi
\begin{enumerate}
\item The coordinates are given as
	\begin{align}
	\vec{A} = \myvec{
		-1\\
		-2\\
		},
	\vec{B} = \myvec{
		1\\
		0\\
		},
	\vec{C} = \myvec{
		-1\\
		2\\
		} \text{ and }
	\vec{D} = \myvec{
		-3\\
		0\\
		}
	\end{align}
	\begin{align}
		\vec{B} - \vec{A} &= \myvec{1\\0} - \myvec{-1\\-2} = \myvec{2\\2}\\
		\vec{C} - \vec{B} &= \myvec{-1\\2} - \myvec{1\\0} = \myvec{-2\\2}\\
		\vec{C} - \vec{D} &= \myvec{-1\\2} - \myvec{-3\\0} = \myvec{2\\2}\\
		\vec{D} - \vec{A} &= \myvec{-3\\0} - \myvec{-1\\-2} = \myvec{-2\\2}
	\end{align}
	\begin{align}	
		\vec{C} - \vec{A} &= \myvec{-1\\2} - \myvec{-1\\-2} = \myvec{0\\4}\\
		\vec{D} - \vec{B} &= \myvec{-3\\0} - \myvec{1\\0} = \myvec{-4\\0}
	\end{align}
	\begin{align}	
		\vec{B}-\vec{A} = \vec{C}-\vec{D} \text{ and } \vec{C}-\vec{B} = \vec{D}-\vec{A}.
	\end{align}
	Hence, $ABCD$ is a parallelogram.
	\begin{enumerate}
		\item Now checking if the adjacent sides are orthogonal to each other
	\begin{align}
		(\vec{B}-\vec{A})^\top (\vec{C}-\vec{B}) = \myvec{2&2} \myvec{-2\\2} = -4+4 = 0
	\end{align}
		\item Now checking if the diagonals are also orthogonal then it is a square else a rectangle.
	\end{enumerate}	
	\begin{align}
		(\vec{C}-\vec{A})^\top (\vec{D}-\vec{B}) = \myvec{0&4} \myvec{-4\\0} = 0
	\end{align}
	Hence the diagonals are orthogonal to each other.

	So, we can conclude that $ABCD$ is a square.

	As shown in Figure \ref{fig:10/7/1/6/Fig1} we can see that $ABCD$ is a square hence we can conclude that our theoritical result is verified.
 
\begin{figure}[!h]
	\begin{center} 
	    \includegraphics[width=\columnwidth]{chapters/10/7/1/6/figs/quad1}
	\end{center}
\caption{}
\label{fig:10/7/1/6/Fig1}
\end{figure}

\item The coordinates are given as
	\begin{align}
	\vec{A} = \myvec{
		-3\\
		5\\
		},
	\vec{B} = \myvec{
		3\\
		1\\
		},
	\vec{C} = \myvec{
		0\\
		3\\
		} \text{ and }
	\vec{D} = \myvec{
		-1\\
		-4\\
		}
	\end{align}
	\begin{align}
		\vec{B} - \vec{A} &= \myvec{3\\1} - \myvec{-3\\5} = \myvec{6\\-4}\\
		\vec{C} - \vec{B} &= \myvec{0\\3} - \myvec{3\\1} = \myvec{-3\\2}\\
		\vec{C} - \vec{D} &= \myvec{0\\3} - \myvec{-1\\-4} = \myvec{1\\7}\\
		\vec{D} - \vec{A} &= \myvec{-1\\-4} - \myvec{-3\\5} = \myvec{2\\-9}
	\end{align}
	\begin{align}
		\vec{C} - \vec{A} &= \myvec{0\\3} - \myvec{-3\\5} = \myvec{3\\-2}\\
		\vec{D} - \vec{B} &= \myvec{-1\\-4} - \myvec{3\\1} = \myvec{-4\\-5}
	\end{align}
	\begin{align}
	\vec{B}-\vec{A} \neq \vec{C}-\vec{D} \text{ and } \vec{C}-\vec{B} \neq \vec{D}-\vec{A},
	\end{align}
	Hence, $ABCD$ is not a parallelogram, it can be a irregular quadilateral.
	\begin{enumerate}
		\item Now to check if any three points are collinear,

	if rank of $\myvec{\vec{B}-\vec{A} & \vec{C}-\vec{B}} = 1$ then points are collinear

	Forming the collinearity matrix
	\begin{align}
		\myvec{6&-3\\-4&2} \xleftrightarrow{R_{2}\rightarrow R_{2}+\frac{2}{3}R_{1}}= \myvec{6&-3\\0&0}
	\end{align}
	\end{enumerate}
	Hence, rank = 1

	Since none of the opposite sides are parallel to each other and three points are collinear so these does not form a quadilateral.

	As shown in Figure \ref{fig:10/7/1/6/Fig2} we can see that $ABCD$ does not form a quadilateral and three points are collinear hence, our theoritical result is verified.
	
\begin{figure}[!h]
	\begin{center} 
	    \includegraphics[width=\columnwidth]{chapters/10/7/1/6/figs/quad2}
	\end{center}
\caption{}
\label{fig:10/7/1/6/Fig2}
\end{figure}
	
\item The coordinates are given as
	\begin{align}
	\vec{A} = \myvec{
		4\\
		5\\
		},
	\vec{B} = \myvec{
		7\\
		6\\
		},
	\vec{C} = \myvec{
		4\\
		3\\
		} \text{ and }
	\vec{D} = \myvec{
		1\\
		2\\
		}
	\end{align}
	\begin{align}
		\vec{B} - \vec{A} &= \myvec{7\\6} - \myvec{4\\5} = \myvec{3\\1}\\
		\vec{C} - \vec{B} &= \myvec{4\\3} - \myvec{7\\6} = \myvec{-3\\-3}\\
		\vec{C} - \vec{D} &= \myvec{4\\3} - \myvec{1\\2} = \myvec{3\\1}\\
		\vec{D} - \vec{A} &= \myvec{1\\2} - \myvec{4\\5} = \myvec{-3\\-3}
	\end{align}
	\begin{align}
		\vec{C} - \vec{A} &= \myvec{4\\3} - \myvec{4\\5} = \myvec{0\\-2}\\
		\vec{D} - \vec{B} &= \myvec{1\\2} - \myvec{7\\6} = \myvec{-6\\-4}
	\end{align}
	\begin{align}
		\vec{B}-\vec{A} = \vec{C}-\vec{D} \text{ and } \vec{C}-\vec{B} = \vec{D}-\vec{A},
	\end{align}
	Hence, $ABCD$ is a parallelogram.
	\begin{enumerate}
		\item Now checking if the adjacent sides are orthogonal to each other
	\begin{align}
		(\vec{B}-\vec{A})^\top (\vec{C}-\vec{B}) = \myvec{3&1} \myvec{-3\\-3} = -9-3 = -12
	\end{align}
	Since inner product is not zero so adjacent sides are not orthogonal.

	Hence, we can say that $ABCD$ is neither a rectangle nor a square.

		\item Now checking if the diagonals are orthogonal then it is a Rhombus.
	\begin{align}
		(\vec{C}- \vec{A})^\top (\vec{D}-\vec{B}) = \myvec{0&-2} \myvec{-6\\-4} = 0+8 = 8
	\end{align}
	\end{enumerate}		
	Hence the diagonals are also not orthogonal so we conclude that $ABCD$ is a parallelogram.

	As shown in Figure \ref{fig:10/7/1/6/Fig3} we can see that $ABCD$ forms a parallelogram hence, our theoritical result is verified.

\begin{figure}[!h]
	\begin{center} 
	    \includegraphics[width=\columnwidth]{chapters/10/7/1/6/figs/quad3}
	\end{center}
\caption{}
\label{fig:10/7/1/6/Fig3}
\end{figure}
\end{enumerate}



\item $ABCD$ is a rectangle formed by the points $\vec{A}(–1, –1), \vec{B}(– 1, 4), \vec{C}(5, 4)$  and  $\vec{D}(5, – 1)$. $\vec{P}, \vec{Q}, \vec{R}$ and $\vec{S}$ are the mid-points of $AB, BC, CD$ and $DA$ respectively. Is the quadrilateral $PQRS$ a square? a rectangle? or a rhombus? Justify your answer.
	\\
	\iffalse
\documentclass[12pt]{article}
\usepackage{graphicx}
\usepackage{amsmath}
\usepackage{mathtools}
\usepackage{gensymb}

\newcommand{\mydet}[1]{\ensuremath{\begin{vmatrix}#1\end{vmatrix}}}
\providecommand{\brak}[1]{\ensuremath{\left(#1\right)}}
\providecommand{\norm}[1]{\left\lVert#1\right\rVert}
\newcommand{\solution}{\noindent \textbf{Solution: }}
\newcommand{\myvec}[1]{\ensuremath{\begin{pmatrix}#1\end{pmatrix}}}
\let\vec\mathbf

\begin{document}
\begin{center}
\textbf\large{CHAPTER-7 \\ COORDINATE GEOMETRY}

\end{center}
\section*{Excercise 7.1}

Q6.Name the type of quadilateral formed,if any, by the following points, and give reasons for your answer:
\begin{enumerate}
	\item $\brak{-1,-2}, \brak{1,0}, \brak{-1,2}, \brak{-3,0}$ 
	\item $\brak{-3,5}, \brak{3,1}, \brak{0,3}, \brak{-1,-4}$
	\item $\brak{4,5}, \brak{7,6}, \brak{4,3}, \brak{1,2}$
\end{enumerate}
\solution
\fi
\begin{enumerate}
\item The coordinates are given as
	\begin{align}
	\vec{A} = \myvec{
		-1\\
		-2\\
		},
	\vec{B} = \myvec{
		1\\
		0\\
		},
	\vec{C} = \myvec{
		-1\\
		2\\
		} \text{ and }
	\vec{D} = \myvec{
		-3\\
		0\\
		}
	\end{align}
	\begin{align}
		\vec{B} - \vec{A} &= \myvec{1\\0} - \myvec{-1\\-2} = \myvec{2\\2}\\
		\vec{C} - \vec{B} &= \myvec{-1\\2} - \myvec{1\\0} = \myvec{-2\\2}\\
		\vec{C} - \vec{D} &= \myvec{-1\\2} - \myvec{-3\\0} = \myvec{2\\2}\\
		\vec{D} - \vec{A} &= \myvec{-3\\0} - \myvec{-1\\-2} = \myvec{-2\\2}
	\end{align}
	\begin{align}	
		\vec{C} - \vec{A} &= \myvec{-1\\2} - \myvec{-1\\-2} = \myvec{0\\4}\\
		\vec{D} - \vec{B} &= \myvec{-3\\0} - \myvec{1\\0} = \myvec{-4\\0}
	\end{align}
	\begin{align}	
		\vec{B}-\vec{A} = \vec{C}-\vec{D} \text{ and } \vec{C}-\vec{B} = \vec{D}-\vec{A}.
	\end{align}
	Hence, $ABCD$ is a parallelogram.
	\begin{enumerate}
		\item Now checking if the adjacent sides are orthogonal to each other
	\begin{align}
		(\vec{B}-\vec{A})^\top (\vec{C}-\vec{B}) = \myvec{2&2} \myvec{-2\\2} = -4+4 = 0
	\end{align}
		\item Now checking if the diagonals are also orthogonal then it is a square else a rectangle.
	\end{enumerate}	
	\begin{align}
		(\vec{C}-\vec{A})^\top (\vec{D}-\vec{B}) = \myvec{0&4} \myvec{-4\\0} = 0
	\end{align}
	Hence the diagonals are orthogonal to each other.

	So, we can conclude that $ABCD$ is a square.

	As shown in Figure \ref{fig:10/7/1/6/Fig1} we can see that $ABCD$ is a square hence we can conclude that our theoritical result is verified.
 
\begin{figure}[!h]
	\begin{center} 
	    \includegraphics[width=\columnwidth]{chapters/10/7/1/6/figs/quad1}
	\end{center}
\caption{}
\label{fig:10/7/1/6/Fig1}
\end{figure}

\item The coordinates are given as
	\begin{align}
	\vec{A} = \myvec{
		-3\\
		5\\
		},
	\vec{B} = \myvec{
		3\\
		1\\
		},
	\vec{C} = \myvec{
		0\\
		3\\
		} \text{ and }
	\vec{D} = \myvec{
		-1\\
		-4\\
		}
	\end{align}
	\begin{align}
		\vec{B} - \vec{A} &= \myvec{3\\1} - \myvec{-3\\5} = \myvec{6\\-4}\\
		\vec{C} - \vec{B} &= \myvec{0\\3} - \myvec{3\\1} = \myvec{-3\\2}\\
		\vec{C} - \vec{D} &= \myvec{0\\3} - \myvec{-1\\-4} = \myvec{1\\7}\\
		\vec{D} - \vec{A} &= \myvec{-1\\-4} - \myvec{-3\\5} = \myvec{2\\-9}
	\end{align}
	\begin{align}
		\vec{C} - \vec{A} &= \myvec{0\\3} - \myvec{-3\\5} = \myvec{3\\-2}\\
		\vec{D} - \vec{B} &= \myvec{-1\\-4} - \myvec{3\\1} = \myvec{-4\\-5}
	\end{align}
	\begin{align}
	\vec{B}-\vec{A} \neq \vec{C}-\vec{D} \text{ and } \vec{C}-\vec{B} \neq \vec{D}-\vec{A},
	\end{align}
	Hence, $ABCD$ is not a parallelogram, it can be a irregular quadilateral.
	\begin{enumerate}
		\item Now to check if any three points are collinear,

	if rank of $\myvec{\vec{B}-\vec{A} & \vec{C}-\vec{B}} = 1$ then points are collinear

	Forming the collinearity matrix
	\begin{align}
		\myvec{6&-3\\-4&2} \xleftrightarrow{R_{2}\rightarrow R_{2}+\frac{2}{3}R_{1}}= \myvec{6&-3\\0&0}
	\end{align}
	\end{enumerate}
	Hence, rank = 1

	Since none of the opposite sides are parallel to each other and three points are collinear so these does not form a quadilateral.

	As shown in Figure \ref{fig:10/7/1/6/Fig2} we can see that $ABCD$ does not form a quadilateral and three points are collinear hence, our theoritical result is verified.
	
\begin{figure}[!h]
	\begin{center} 
	    \includegraphics[width=\columnwidth]{chapters/10/7/1/6/figs/quad2}
	\end{center}
\caption{}
\label{fig:10/7/1/6/Fig2}
\end{figure}
	
\item The coordinates are given as
	\begin{align}
	\vec{A} = \myvec{
		4\\
		5\\
		},
	\vec{B} = \myvec{
		7\\
		6\\
		},
	\vec{C} = \myvec{
		4\\
		3\\
		} \text{ and }
	\vec{D} = \myvec{
		1\\
		2\\
		}
	\end{align}
	\begin{align}
		\vec{B} - \vec{A} &= \myvec{7\\6} - \myvec{4\\5} = \myvec{3\\1}\\
		\vec{C} - \vec{B} &= \myvec{4\\3} - \myvec{7\\6} = \myvec{-3\\-3}\\
		\vec{C} - \vec{D} &= \myvec{4\\3} - \myvec{1\\2} = \myvec{3\\1}\\
		\vec{D} - \vec{A} &= \myvec{1\\2} - \myvec{4\\5} = \myvec{-3\\-3}
	\end{align}
	\begin{align}
		\vec{C} - \vec{A} &= \myvec{4\\3} - \myvec{4\\5} = \myvec{0\\-2}\\
		\vec{D} - \vec{B} &= \myvec{1\\2} - \myvec{7\\6} = \myvec{-6\\-4}
	\end{align}
	\begin{align}
		\vec{B}-\vec{A} = \vec{C}-\vec{D} \text{ and } \vec{C}-\vec{B} = \vec{D}-\vec{A},
	\end{align}
	Hence, $ABCD$ is a parallelogram.
	\begin{enumerate}
		\item Now checking if the adjacent sides are orthogonal to each other
	\begin{align}
		(\vec{B}-\vec{A})^\top (\vec{C}-\vec{B}) = \myvec{3&1} \myvec{-3\\-3} = -9-3 = -12
	\end{align}
	Since inner product is not zero so adjacent sides are not orthogonal.

	Hence, we can say that $ABCD$ is neither a rectangle nor a square.

		\item Now checking if the diagonals are orthogonal then it is a Rhombus.
	\begin{align}
		(\vec{C}- \vec{A})^\top (\vec{D}-\vec{B}) = \myvec{0&-2} \myvec{-6\\-4} = 0+8 = 8
	\end{align}
	\end{enumerate}		
	Hence the diagonals are also not orthogonal so we conclude that $ABCD$ is a parallelogram.

	As shown in Figure \ref{fig:10/7/1/6/Fig3} we can see that $ABCD$ forms a parallelogram hence, our theoritical result is verified.

\begin{figure}[!h]
	\begin{center} 
	    \includegraphics[width=\columnwidth]{chapters/10/7/1/6/figs/quad3}
	\end{center}
\caption{}
\label{fig:10/7/1/6/Fig3}
\end{figure}
\end{enumerate}



\item Without using the Baudhayana theorem, show that the points $(4,4),(3,5)$ and $(-1,-1)$ are the vertices of a right angled triangle.
\label{chapters/11/10/1/6}
\iffalse
\documentclass[journal,12pt,twocolumn]{IEEEtran}
\usepackage[none]{hyphenat}
\usepackage{graphicx}
\usepackage{listings}
\usepackage[english]{babel}
\usepackage{graphicx}
\usepackage{caption} 
\usepackage{amsmath}
\usepackage{hyperref}
\usepackage{booktabs}
\usepackage{array}


\title{\textbf{\\Assignment on line}}
\author{Sireesha Abbavaram - FWC22060}
\begin{document}
\maketitle


\section{Question}
\textbf{\textit{Class 11, Exercise 10.1, Q(6):}
\fi
	\begin{figure}[!ht]
		\centering
 \includegraphics[width=\columnwidth]{chapters/11/10/1/6/figs/triangle.png}
		\caption{}
		\label{fig:11/10/1/6}
  	\end{figure}
\iffalse
}

\begin{figure}[h!]
\centering
\includegraphics[scale=0.35]{triangle.png}
\centering
\caption{Traingle ABC}
\end{figure}


\section{Solution}
\raggedright 
\vspace{0.25cm}
Let A,B and C be the vertices of a given traingle with coordinates $\myvec{
4 \\
4
}
, \myvec{
3 \\
5
}
 and \myvec{
-1 \\
-1
} $
\raggedright
. we have verify whther the given vertices are of right angled triangle or not.\\
\begin{center}
\raggedright
Let The directional vector of two vectors A and B is given as AB (m1)=A-B.
\end{center}
\vspace{0.25cm}
\begin{center}
The directional vector of the vectors B and C is given as BC (m2)=B-C.
\end{center}
\vspace{0.25cm}
\begin{center}
The directinal vector of the vectors C and A is given as CA (m3)=C-A.
\end{center}
\vspace{0.25cm}
The angle between any two vectors is given by
\boldmath
\\ $ cos b =\frac{(m1)^T(m2)}{||m1|| ||m2||}  equation-1$
\unboldmath
\vspace{0.5cm}\raggedright\\
Where b is the angle between the two vectors .
when the angle b=90 ,we get cos 90=0.
\vspace{0.5cm}\raggedright\\
It implies that the numerator of the equation 1 should be zero.

\vspace{0.25cm}
 In order to prove that the triangle is right angled we have to show any two vectors should be orthogonal to each other.
 
\vspace{0.25cm}\raggedright
So we need to show $(A-B)^T(B-C) or (B-C)^T(C-A) or (C-A)^T(A-B) $ is equal to zero.
\fi

\vspace{0.5cm}\raggedright
\begin{align}
	\vec{C}-\vec{A}&=\myvec{
-5 \\
	-5},
\\
	\vec{A}-\vec{B}&=\myvec{
1 \\
-1 
}
\\
	\implies \brak{\vec{C}-\vec{A}}^{\top}
	\brak{\vec{A}-\vec{B}}&=0
\end{align}
Thus, $AB \perp AC$.
\iffalse
\vspace{0.25cm}\raggedright
Thus we have right angle at the vertex A.


\vspace{0.2cm}
\section*{Construction}
\centering
\vspace{0.2cm}
{
\setlength\extrarowheight{2pt}
\begin{tabular}{|c|c|c|}
	\hline
	\textbf{Symbol}&\textbf{Value}&\textbf{Description}\\
	\hline
	A & (4,4) & Vertex A\\
	\hline
	B & (3,5) & Vertex B\\
	\hline
	C & (-1,-1) & Vertex C\\
	\hline
	
\end{tabular}
}

\vspace{0.6cm}
Get the python code of the figures from
\begin{table}[h]
\large
\centering
\begin{tabular}{|l|}
\hline
https://github.com/Sireesha1602/sireesha/
\\blob/main/line assignment \\
\hline
\end{tabular}

\end{table}




\end{document}
\fi

\item The line through the points $(h, 3)$ and $(4, 1)$ intersects the line $7x- 9y- 19= 0$ at right angle. Find the value of $h$.
\label{chapters/11/10/3/10}
\\
\solution
\iffalse
\documentclass[12pt]{article}
\usepackage{graphicx}
\usepackage[none]{hyphenat}
\usepackage{graphicx}
\usepackage{listings}
\usepackage[english]{babel}
\usepackage{graphicx}
\usepackage{caption} 
\usepackage{booktabs}
\usepackage{array}
\usepackage{amssymb} % for \because
\usepackage{amsmath}   % for having text in math mode
\usepackage{extarrows} % for Row operations arrows
\usepackage{listings}
\usepackage[utf8]{inputenc}
\lstset{
  frame=single,
  breaklines=true
}
\usepackage{hyperref}
  
%Following 2 lines were added to remove the blank page at the beginning
\usepackage{atbegshi}% http://ctan.org/pkg/atbegshi
\AtBeginDocument{\AtBeginShipoutNext{\AtBeginShipoutDiscard}}


%New macro definitions
\newcommand{\mydet}[1]{\ensuremath{\begin{vmatrix}#1\end{vmatrix}}}
\providecommand{\brak}[1]{\ensuremath{\left(#1\right)}}
\newcommand{\solution}{\noindent \textbf{Solution: }}
\newcommand{\myvec}[1]{\ensuremath{\begin{pmatrix}#1\end{pmatrix}}}
\providecommand{\norm}[1]{\left\lVert#1\right\rVert}
\providecommand{\abs}[1]{\left\vert#1\right\vert}
\let\vec\mathbf

\begin{document}

\begin{center}
\title{\textbf{LINE}}
\date{\vspace{-5ex}} %Not to print date automatically
\maketitle
\end{center}

\section{11$^{th}$ Maths - EXERCISE-10.3}
\begin{enumerate}
\end{enumerate}
\section{SOLUTION}
\fi
Let
Given points are 
\begin{align}
\vec{A}=\myvec{h\\ 3},\vec{B}=\myvec{4\\ 1} 
\implies \vec{B}-\vec{A}=
\myvec{4-h\\ -2}
\end{align}
The given line equation  can be expressed as
\begin{align}
\myvec{7& -9}\vec{x}&=19
\end{align}
yielding 
\begin{align}
\vec{n}=\myvec{7\\ -9},
\vec{m}=\myvec{9\\ 7}
\end{align}
Thus, 
\begin{align}
	\vec{m}^\top\brak{\vec{B}- \vec{A}}&=0\\
\implies\myvec{9& 7}\myvec{4-h\\ -2}&=0\\
\implies h&=\frac{22}{9}
\end{align}
See Fig. 
		\ref{fig:chapters/11/10/3/10/Figure}.
\begin{figure}[h]
\centering
\includegraphics[width=\columnwidth]{chapters/11/10/3/10/figs/fig.pdf}
\caption{}
		\label{fig:chapters/11/10/3/10/Figure}
\end{figure}

\item In the following cases, determine whether the given planes are parallel or perpendicular, and in case they are neither, find the angles between them.
\begin{enumerate}
\item $7x + 5y + 6z + 30 = 0$ and $3x – y – 10z + 4 = 0$
\item $2x + y + 3z – 2 = 0$ and $x – 2y + 5 = 0$
\item $2x – 2y + 4z + 5 = 0$ and $3x – 3y + 6z – 1 = 0$
\item $2x – y + 3z – 1 = 0$ and $2x – y + 3z + 3 = 0$
\item $4x + 8y + z – 8 = 0$ and $y + z – 4 = 0$
\end{enumerate}
    \solution
		\iffalse
\documentclass[12pt]{article}
\usepackage{graphicx}
\usepackage{amsmath}
\usepackage{mathtools}
\usepackage{gensymb}

\newcommand{\mydet}[1]{\ensuremath{\begin{vmatrix}#1\end{vmatrix}}}
\providecommand{\brak}[1]{\ensuremath{\left(#1\right)}}
\providecommand{\norm}[1]{\left\lVert#1\right\rVert}
\newcommand{\solution}{\noindent \textbf{Solution: }}
\newcommand{\myvec}[1]{\ensuremath{\begin{pmatrix}#1\end{pmatrix}}}
\let\vec\mathbf

\begin{document}
\begin{center}
\textbf\large{CHAPTER-7 \\ COORDINATE GEOMETRY}

\end{center}
\section*{Excercise 7.1}

Q6.Name the type of quadilateral formed,if any, by the following points, and give reasons for your answer:
\begin{enumerate}
	\item $\brak{-1,-2}, \brak{1,0}, \brak{-1,2}, \brak{-3,0}$ 
	\item $\brak{-3,5}, \brak{3,1}, \brak{0,3}, \brak{-1,-4}$
	\item $\brak{4,5}, \brak{7,6}, \brak{4,3}, \brak{1,2}$
\end{enumerate}
\solution
\fi
\begin{enumerate}
\item The coordinates are given as
	\begin{align}
	\vec{A} = \myvec{
		-1\\
		-2\\
		},
	\vec{B} = \myvec{
		1\\
		0\\
		},
	\vec{C} = \myvec{
		-1\\
		2\\
		} \text{ and }
	\vec{D} = \myvec{
		-3\\
		0\\
		}
	\end{align}
	\begin{align}
		\vec{B} - \vec{A} &= \myvec{1\\0} - \myvec{-1\\-2} = \myvec{2\\2}\\
		\vec{C} - \vec{B} &= \myvec{-1\\2} - \myvec{1\\0} = \myvec{-2\\2}\\
		\vec{C} - \vec{D} &= \myvec{-1\\2} - \myvec{-3\\0} = \myvec{2\\2}\\
		\vec{D} - \vec{A} &= \myvec{-3\\0} - \myvec{-1\\-2} = \myvec{-2\\2}
	\end{align}
	\begin{align}	
		\vec{C} - \vec{A} &= \myvec{-1\\2} - \myvec{-1\\-2} = \myvec{0\\4}\\
		\vec{D} - \vec{B} &= \myvec{-3\\0} - \myvec{1\\0} = \myvec{-4\\0}
	\end{align}
	\begin{align}	
		\vec{B}-\vec{A} = \vec{C}-\vec{D} \text{ and } \vec{C}-\vec{B} = \vec{D}-\vec{A}.
	\end{align}
	Hence, $ABCD$ is a parallelogram.
	\begin{enumerate}
		\item Now checking if the adjacent sides are orthogonal to each other
	\begin{align}
		(\vec{B}-\vec{A})^\top (\vec{C}-\vec{B}) = \myvec{2&2} \myvec{-2\\2} = -4+4 = 0
	\end{align}
		\item Now checking if the diagonals are also orthogonal then it is a square else a rectangle.
	\end{enumerate}	
	\begin{align}
		(\vec{C}-\vec{A})^\top (\vec{D}-\vec{B}) = \myvec{0&4} \myvec{-4\\0} = 0
	\end{align}
	Hence the diagonals are orthogonal to each other.

	So, we can conclude that $ABCD$ is a square.

	As shown in Figure \ref{fig:10/7/1/6/Fig1} we can see that $ABCD$ is a square hence we can conclude that our theoritical result is verified.
 
\begin{figure}[!h]
	\begin{center} 
	    \includegraphics[width=\columnwidth]{chapters/10/7/1/6/figs/quad1}
	\end{center}
\caption{}
\label{fig:10/7/1/6/Fig1}
\end{figure}

\item The coordinates are given as
	\begin{align}
	\vec{A} = \myvec{
		-3\\
		5\\
		},
	\vec{B} = \myvec{
		3\\
		1\\
		},
	\vec{C} = \myvec{
		0\\
		3\\
		} \text{ and }
	\vec{D} = \myvec{
		-1\\
		-4\\
		}
	\end{align}
	\begin{align}
		\vec{B} - \vec{A} &= \myvec{3\\1} - \myvec{-3\\5} = \myvec{6\\-4}\\
		\vec{C} - \vec{B} &= \myvec{0\\3} - \myvec{3\\1} = \myvec{-3\\2}\\
		\vec{C} - \vec{D} &= \myvec{0\\3} - \myvec{-1\\-4} = \myvec{1\\7}\\
		\vec{D} - \vec{A} &= \myvec{-1\\-4} - \myvec{-3\\5} = \myvec{2\\-9}
	\end{align}
	\begin{align}
		\vec{C} - \vec{A} &= \myvec{0\\3} - \myvec{-3\\5} = \myvec{3\\-2}\\
		\vec{D} - \vec{B} &= \myvec{-1\\-4} - \myvec{3\\1} = \myvec{-4\\-5}
	\end{align}
	\begin{align}
	\vec{B}-\vec{A} \neq \vec{C}-\vec{D} \text{ and } \vec{C}-\vec{B} \neq \vec{D}-\vec{A},
	\end{align}
	Hence, $ABCD$ is not a parallelogram, it can be a irregular quadilateral.
	\begin{enumerate}
		\item Now to check if any three points are collinear,

	if rank of $\myvec{\vec{B}-\vec{A} & \vec{C}-\vec{B}} = 1$ then points are collinear

	Forming the collinearity matrix
	\begin{align}
		\myvec{6&-3\\-4&2} \xleftrightarrow{R_{2}\rightarrow R_{2}+\frac{2}{3}R_{1}}= \myvec{6&-3\\0&0}
	\end{align}
	\end{enumerate}
	Hence, rank = 1

	Since none of the opposite sides are parallel to each other and three points are collinear so these does not form a quadilateral.

	As shown in Figure \ref{fig:10/7/1/6/Fig2} we can see that $ABCD$ does not form a quadilateral and three points are collinear hence, our theoritical result is verified.
	
\begin{figure}[!h]
	\begin{center} 
	    \includegraphics[width=\columnwidth]{chapters/10/7/1/6/figs/quad2}
	\end{center}
\caption{}
\label{fig:10/7/1/6/Fig2}
\end{figure}
	
\item The coordinates are given as
	\begin{align}
	\vec{A} = \myvec{
		4\\
		5\\
		},
	\vec{B} = \myvec{
		7\\
		6\\
		},
	\vec{C} = \myvec{
		4\\
		3\\
		} \text{ and }
	\vec{D} = \myvec{
		1\\
		2\\
		}
	\end{align}
	\begin{align}
		\vec{B} - \vec{A} &= \myvec{7\\6} - \myvec{4\\5} = \myvec{3\\1}\\
		\vec{C} - \vec{B} &= \myvec{4\\3} - \myvec{7\\6} = \myvec{-3\\-3}\\
		\vec{C} - \vec{D} &= \myvec{4\\3} - \myvec{1\\2} = \myvec{3\\1}\\
		\vec{D} - \vec{A} &= \myvec{1\\2} - \myvec{4\\5} = \myvec{-3\\-3}
	\end{align}
	\begin{align}
		\vec{C} - \vec{A} &= \myvec{4\\3} - \myvec{4\\5} = \myvec{0\\-2}\\
		\vec{D} - \vec{B} &= \myvec{1\\2} - \myvec{7\\6} = \myvec{-6\\-4}
	\end{align}
	\begin{align}
		\vec{B}-\vec{A} = \vec{C}-\vec{D} \text{ and } \vec{C}-\vec{B} = \vec{D}-\vec{A},
	\end{align}
	Hence, $ABCD$ is a parallelogram.
	\begin{enumerate}
		\item Now checking if the adjacent sides are orthogonal to each other
	\begin{align}
		(\vec{B}-\vec{A})^\top (\vec{C}-\vec{B}) = \myvec{3&1} \myvec{-3\\-3} = -9-3 = -12
	\end{align}
	Since inner product is not zero so adjacent sides are not orthogonal.

	Hence, we can say that $ABCD$ is neither a rectangle nor a square.

		\item Now checking if the diagonals are orthogonal then it is a Rhombus.
	\begin{align}
		(\vec{C}- \vec{A})^\top (\vec{D}-\vec{B}) = \myvec{0&-2} \myvec{-6\\-4} = 0+8 = 8
	\end{align}
	\end{enumerate}		
	Hence the diagonals are also not orthogonal so we conclude that $ABCD$ is a parallelogram.

	As shown in Figure \ref{fig:10/7/1/6/Fig3} we can see that $ABCD$ forms a parallelogram hence, our theoritical result is verified.

\begin{figure}[!h]
	\begin{center} 
	    \includegraphics[width=\columnwidth]{chapters/10/7/1/6/figs/quad3}
	\end{center}
\caption{}
\label{fig:10/7/1/6/Fig3}
\end{figure}
\end{enumerate}



		\item 
 Show that the line joining the origin to the point $(2, 1, 1)$ is perpendicular to the
line determined by the points $(3, 5, – 1), (4, 3, – 1)$.
\\
    \solution
		\iffalse
\documentclass[12pt]{article}
\usepackage{graphicx}
\usepackage{amsmath}
\usepackage{mathtools}
\usepackage{gensymb}
\usepackage[utf8]{inputenc}
\usepackage{float}
\newcommand{\mydet}[1]{\ensuremath{\begin{vmatrix}#1\end{vmatrix}}}
\providecommand{\brak}[1]{\ensuremath{\left(#1\right)}}
\providecommand{\norm}[1]{\left\lVert#1\right\rVert}
\newcommand{\solution}{\noindent \textbf{Solution: }}
\newcommand{\myvec}[1]{\ensuremath{\begin{pmatrix}#1\end{pmatrix}}}
\let\vec\mathbf

\begin{document}
\begin{center}
\textbf\large{CLASS-12 \\ CHAPTER-11 \\ THREE DIMENSIONAL GEOMETRY}
\end{center}
\section*{Excercise 11.4}

\\
\solution
\\
\fi
Let
\begin{align}
  \vec{P}=\myvec{2\\1\\1},\vec{A}=\myvec{3\\5\\-1},\vec{B}=\myvec{4\\3\\-1}
\end{align}
Then
		\begin{align}
	\vec{m}=\vec{A}-\vec{B}=\myvec{-1\\2\\0}
		\end{align}
		and
		\begin{align}
			\vec{m}^\top\vec{P}=
			\myvec{-1&2&0}\myvec{2\\1\\1}=0
		\end{align}
		This proves the result.


	\item  If $l_1, m_1,n_1 \text{ and } l_2,m_2,n_2$ are the direction cosines of two mutually perpendicular lines, show that the direction cosines of the line perpendicular to both these are  $m_1n_2-m_2n_1,n_1l_2-n_2l_1,l_1m_2-l_2m_1$.
\\
    \solution
		\iffalse
\documentclass[12pt]{article}
\usepackage{graphicx}
\usepackage{amsmath}
\usepackage{mathtools}
\usepackage{gensymb}

\newcommand{\mydet}[1]{\ensuremath{\begin{vmatrix}#1\end{vmatrix}}}
\providecommand{\brak}[1]{\ensuremath{\left(#1\right)}}
\providecommand{\norm}[1]{\left\lVert#1\right\rVert}
\newcommand{\solution}{\noindent \textbf{Solution: }}
\newcommand{\myvec}[1]{\ensuremath{\begin{pmatrix}#1\end{pmatrix}}}
\let\vec\mathbf

\begin{document}
\begin{center}
\textbf\large{CHAPTER-7 \\ COORDINATE GEOMETRY}

\end{center}
\section*{Excercise 7.1}

Q6.Name the type of quadilateral formed,if any, by the following points, and give reasons for your answer:
\begin{enumerate}
	\item $\brak{-1,-2}, \brak{1,0}, \brak{-1,2}, \brak{-3,0}$ 
	\item $\brak{-3,5}, \brak{3,1}, \brak{0,3}, \brak{-1,-4}$
	\item $\brak{4,5}, \brak{7,6}, \brak{4,3}, \brak{1,2}$
\end{enumerate}
\solution
\fi
\begin{enumerate}
\item The coordinates are given as
	\begin{align}
	\vec{A} = \myvec{
		-1\\
		-2\\
		},
	\vec{B} = \myvec{
		1\\
		0\\
		},
	\vec{C} = \myvec{
		-1\\
		2\\
		} \text{ and }
	\vec{D} = \myvec{
		-3\\
		0\\
		}
	\end{align}
	\begin{align}
		\vec{B} - \vec{A} &= \myvec{1\\0} - \myvec{-1\\-2} = \myvec{2\\2}\\
		\vec{C} - \vec{B} &= \myvec{-1\\2} - \myvec{1\\0} = \myvec{-2\\2}\\
		\vec{C} - \vec{D} &= \myvec{-1\\2} - \myvec{-3\\0} = \myvec{2\\2}\\
		\vec{D} - \vec{A} &= \myvec{-3\\0} - \myvec{-1\\-2} = \myvec{-2\\2}
	\end{align}
	\begin{align}	
		\vec{C} - \vec{A} &= \myvec{-1\\2} - \myvec{-1\\-2} = \myvec{0\\4}\\
		\vec{D} - \vec{B} &= \myvec{-3\\0} - \myvec{1\\0} = \myvec{-4\\0}
	\end{align}
	\begin{align}	
		\vec{B}-\vec{A} = \vec{C}-\vec{D} \text{ and } \vec{C}-\vec{B} = \vec{D}-\vec{A}.
	\end{align}
	Hence, $ABCD$ is a parallelogram.
	\begin{enumerate}
		\item Now checking if the adjacent sides are orthogonal to each other
	\begin{align}
		(\vec{B}-\vec{A})^\top (\vec{C}-\vec{B}) = \myvec{2&2} \myvec{-2\\2} = -4+4 = 0
	\end{align}
		\item Now checking if the diagonals are also orthogonal then it is a square else a rectangle.
	\end{enumerate}	
	\begin{align}
		(\vec{C}-\vec{A})^\top (\vec{D}-\vec{B}) = \myvec{0&4} \myvec{-4\\0} = 0
	\end{align}
	Hence the diagonals are orthogonal to each other.

	So, we can conclude that $ABCD$ is a square.

	As shown in Figure \ref{fig:10/7/1/6/Fig1} we can see that $ABCD$ is a square hence we can conclude that our theoritical result is verified.
 
\begin{figure}[!h]
	\begin{center} 
	    \includegraphics[width=\columnwidth]{chapters/10/7/1/6/figs/quad1}
	\end{center}
\caption{}
\label{fig:10/7/1/6/Fig1}
\end{figure}

\item The coordinates are given as
	\begin{align}
	\vec{A} = \myvec{
		-3\\
		5\\
		},
	\vec{B} = \myvec{
		3\\
		1\\
		},
	\vec{C} = \myvec{
		0\\
		3\\
		} \text{ and }
	\vec{D} = \myvec{
		-1\\
		-4\\
		}
	\end{align}
	\begin{align}
		\vec{B} - \vec{A} &= \myvec{3\\1} - \myvec{-3\\5} = \myvec{6\\-4}\\
		\vec{C} - \vec{B} &= \myvec{0\\3} - \myvec{3\\1} = \myvec{-3\\2}\\
		\vec{C} - \vec{D} &= \myvec{0\\3} - \myvec{-1\\-4} = \myvec{1\\7}\\
		\vec{D} - \vec{A} &= \myvec{-1\\-4} - \myvec{-3\\5} = \myvec{2\\-9}
	\end{align}
	\begin{align}
		\vec{C} - \vec{A} &= \myvec{0\\3} - \myvec{-3\\5} = \myvec{3\\-2}\\
		\vec{D} - \vec{B} &= \myvec{-1\\-4} - \myvec{3\\1} = \myvec{-4\\-5}
	\end{align}
	\begin{align}
	\vec{B}-\vec{A} \neq \vec{C}-\vec{D} \text{ and } \vec{C}-\vec{B} \neq \vec{D}-\vec{A},
	\end{align}
	Hence, $ABCD$ is not a parallelogram, it can be a irregular quadilateral.
	\begin{enumerate}
		\item Now to check if any three points are collinear,

	if rank of $\myvec{\vec{B}-\vec{A} & \vec{C}-\vec{B}} = 1$ then points are collinear

	Forming the collinearity matrix
	\begin{align}
		\myvec{6&-3\\-4&2} \xleftrightarrow{R_{2}\rightarrow R_{2}+\frac{2}{3}R_{1}}= \myvec{6&-3\\0&0}
	\end{align}
	\end{enumerate}
	Hence, rank = 1

	Since none of the opposite sides are parallel to each other and three points are collinear so these does not form a quadilateral.

	As shown in Figure \ref{fig:10/7/1/6/Fig2} we can see that $ABCD$ does not form a quadilateral and three points are collinear hence, our theoritical result is verified.
	
\begin{figure}[!h]
	\begin{center} 
	    \includegraphics[width=\columnwidth]{chapters/10/7/1/6/figs/quad2}
	\end{center}
\caption{}
\label{fig:10/7/1/6/Fig2}
\end{figure}
	
\item The coordinates are given as
	\begin{align}
	\vec{A} = \myvec{
		4\\
		5\\
		},
	\vec{B} = \myvec{
		7\\
		6\\
		},
	\vec{C} = \myvec{
		4\\
		3\\
		} \text{ and }
	\vec{D} = \myvec{
		1\\
		2\\
		}
	\end{align}
	\begin{align}
		\vec{B} - \vec{A} &= \myvec{7\\6} - \myvec{4\\5} = \myvec{3\\1}\\
		\vec{C} - \vec{B} &= \myvec{4\\3} - \myvec{7\\6} = \myvec{-3\\-3}\\
		\vec{C} - \vec{D} &= \myvec{4\\3} - \myvec{1\\2} = \myvec{3\\1}\\
		\vec{D} - \vec{A} &= \myvec{1\\2} - \myvec{4\\5} = \myvec{-3\\-3}
	\end{align}
	\begin{align}
		\vec{C} - \vec{A} &= \myvec{4\\3} - \myvec{4\\5} = \myvec{0\\-2}\\
		\vec{D} - \vec{B} &= \myvec{1\\2} - \myvec{7\\6} = \myvec{-6\\-4}
	\end{align}
	\begin{align}
		\vec{B}-\vec{A} = \vec{C}-\vec{D} \text{ and } \vec{C}-\vec{B} = \vec{D}-\vec{A},
	\end{align}
	Hence, $ABCD$ is a parallelogram.
	\begin{enumerate}
		\item Now checking if the adjacent sides are orthogonal to each other
	\begin{align}
		(\vec{B}-\vec{A})^\top (\vec{C}-\vec{B}) = \myvec{3&1} \myvec{-3\\-3} = -9-3 = -12
	\end{align}
	Since inner product is not zero so adjacent sides are not orthogonal.

	Hence, we can say that $ABCD$ is neither a rectangle nor a square.

		\item Now checking if the diagonals are orthogonal then it is a Rhombus.
	\begin{align}
		(\vec{C}- \vec{A})^\top (\vec{D}-\vec{B}) = \myvec{0&-2} \myvec{-6\\-4} = 0+8 = 8
	\end{align}
	\end{enumerate}		
	Hence the diagonals are also not orthogonal so we conclude that $ABCD$ is a parallelogram.

	As shown in Figure \ref{fig:10/7/1/6/Fig3} we can see that $ABCD$ forms a parallelogram hence, our theoritical result is verified.

\begin{figure}[!h]
	\begin{center} 
	    \includegraphics[width=\columnwidth]{chapters/10/7/1/6/figs/quad3}
	\end{center}
\caption{}
\label{fig:10/7/1/6/Fig3}
\end{figure}
\end{enumerate}



	\item If the lines $\frac{x-1}{-3} = \frac{y-2}{2k} = \frac{z-3}{2}$ and  $\frac{x-1}{3k} = \frac{y-1}{1} = \frac{z-6}{-5}$ are perpendicular, find the value of k.\\
    \solution
		\iffalse
\documentclass[12pt]{article}
\usepackage{graphicx}
\usepackage{amsmath}
\usepackage{mathtools}
\usepackage{gensymb}

\newcommand{\mydet}[1]{\ensuremath{\begin{vmatrix}#1\end{vmatrix}}}
\providecommand{\brak}[1]{\ensuremath{\left(#1\right)}}
\providecommand{\norm}[1]{\left\lVert#1\right\rVert}
\newcommand{\solution}{\noindent \textbf{Solution: }}
\newcommand{\myvec}[1]{\ensuremath{\begin{pmatrix}#1\end{pmatrix}}}
\let\vec\mathbf

\begin{document}
\begin{center}
\textbf\large{CHAPTER-7 \\ COORDINATE GEOMETRY}

\end{center}
\section*{Excercise 7.1}

Q6.Name the type of quadilateral formed,if any, by the following points, and give reasons for your answer:
\begin{enumerate}
	\item $\brak{-1,-2}, \brak{1,0}, \brak{-1,2}, \brak{-3,0}$ 
	\item $\brak{-3,5}, \brak{3,1}, \brak{0,3}, \brak{-1,-4}$
	\item $\brak{4,5}, \brak{7,6}, \brak{4,3}, \brak{1,2}$
\end{enumerate}
\solution
\fi
\begin{enumerate}
\item The coordinates are given as
	\begin{align}
	\vec{A} = \myvec{
		-1\\
		-2\\
		},
	\vec{B} = \myvec{
		1\\
		0\\
		},
	\vec{C} = \myvec{
		-1\\
		2\\
		} \text{ and }
	\vec{D} = \myvec{
		-3\\
		0\\
		}
	\end{align}
	\begin{align}
		\vec{B} - \vec{A} &= \myvec{1\\0} - \myvec{-1\\-2} = \myvec{2\\2}\\
		\vec{C} - \vec{B} &= \myvec{-1\\2} - \myvec{1\\0} = \myvec{-2\\2}\\
		\vec{C} - \vec{D} &= \myvec{-1\\2} - \myvec{-3\\0} = \myvec{2\\2}\\
		\vec{D} - \vec{A} &= \myvec{-3\\0} - \myvec{-1\\-2} = \myvec{-2\\2}
	\end{align}
	\begin{align}	
		\vec{C} - \vec{A} &= \myvec{-1\\2} - \myvec{-1\\-2} = \myvec{0\\4}\\
		\vec{D} - \vec{B} &= \myvec{-3\\0} - \myvec{1\\0} = \myvec{-4\\0}
	\end{align}
	\begin{align}	
		\vec{B}-\vec{A} = \vec{C}-\vec{D} \text{ and } \vec{C}-\vec{B} = \vec{D}-\vec{A}.
	\end{align}
	Hence, $ABCD$ is a parallelogram.
	\begin{enumerate}
		\item Now checking if the adjacent sides are orthogonal to each other
	\begin{align}
		(\vec{B}-\vec{A})^\top (\vec{C}-\vec{B}) = \myvec{2&2} \myvec{-2\\2} = -4+4 = 0
	\end{align}
		\item Now checking if the diagonals are also orthogonal then it is a square else a rectangle.
	\end{enumerate}	
	\begin{align}
		(\vec{C}-\vec{A})^\top (\vec{D}-\vec{B}) = \myvec{0&4} \myvec{-4\\0} = 0
	\end{align}
	Hence the diagonals are orthogonal to each other.

	So, we can conclude that $ABCD$ is a square.

	As shown in Figure \ref{fig:10/7/1/6/Fig1} we can see that $ABCD$ is a square hence we can conclude that our theoritical result is verified.
 
\begin{figure}[!h]
	\begin{center} 
	    \includegraphics[width=\columnwidth]{chapters/10/7/1/6/figs/quad1}
	\end{center}
\caption{}
\label{fig:10/7/1/6/Fig1}
\end{figure}

\item The coordinates are given as
	\begin{align}
	\vec{A} = \myvec{
		-3\\
		5\\
		},
	\vec{B} = \myvec{
		3\\
		1\\
		},
	\vec{C} = \myvec{
		0\\
		3\\
		} \text{ and }
	\vec{D} = \myvec{
		-1\\
		-4\\
		}
	\end{align}
	\begin{align}
		\vec{B} - \vec{A} &= \myvec{3\\1} - \myvec{-3\\5} = \myvec{6\\-4}\\
		\vec{C} - \vec{B} &= \myvec{0\\3} - \myvec{3\\1} = \myvec{-3\\2}\\
		\vec{C} - \vec{D} &= \myvec{0\\3} - \myvec{-1\\-4} = \myvec{1\\7}\\
		\vec{D} - \vec{A} &= \myvec{-1\\-4} - \myvec{-3\\5} = \myvec{2\\-9}
	\end{align}
	\begin{align}
		\vec{C} - \vec{A} &= \myvec{0\\3} - \myvec{-3\\5} = \myvec{3\\-2}\\
		\vec{D} - \vec{B} &= \myvec{-1\\-4} - \myvec{3\\1} = \myvec{-4\\-5}
	\end{align}
	\begin{align}
	\vec{B}-\vec{A} \neq \vec{C}-\vec{D} \text{ and } \vec{C}-\vec{B} \neq \vec{D}-\vec{A},
	\end{align}
	Hence, $ABCD$ is not a parallelogram, it can be a irregular quadilateral.
	\begin{enumerate}
		\item Now to check if any three points are collinear,

	if rank of $\myvec{\vec{B}-\vec{A} & \vec{C}-\vec{B}} = 1$ then points are collinear

	Forming the collinearity matrix
	\begin{align}
		\myvec{6&-3\\-4&2} \xleftrightarrow{R_{2}\rightarrow R_{2}+\frac{2}{3}R_{1}}= \myvec{6&-3\\0&0}
	\end{align}
	\end{enumerate}
	Hence, rank = 1

	Since none of the opposite sides are parallel to each other and three points are collinear so these does not form a quadilateral.

	As shown in Figure \ref{fig:10/7/1/6/Fig2} we can see that $ABCD$ does not form a quadilateral and three points are collinear hence, our theoritical result is verified.
	
\begin{figure}[!h]
	\begin{center} 
	    \includegraphics[width=\columnwidth]{chapters/10/7/1/6/figs/quad2}
	\end{center}
\caption{}
\label{fig:10/7/1/6/Fig2}
\end{figure}
	
\item The coordinates are given as
	\begin{align}
	\vec{A} = \myvec{
		4\\
		5\\
		},
	\vec{B} = \myvec{
		7\\
		6\\
		},
	\vec{C} = \myvec{
		4\\
		3\\
		} \text{ and }
	\vec{D} = \myvec{
		1\\
		2\\
		}
	\end{align}
	\begin{align}
		\vec{B} - \vec{A} &= \myvec{7\\6} - \myvec{4\\5} = \myvec{3\\1}\\
		\vec{C} - \vec{B} &= \myvec{4\\3} - \myvec{7\\6} = \myvec{-3\\-3}\\
		\vec{C} - \vec{D} &= \myvec{4\\3} - \myvec{1\\2} = \myvec{3\\1}\\
		\vec{D} - \vec{A} &= \myvec{1\\2} - \myvec{4\\5} = \myvec{-3\\-3}
	\end{align}
	\begin{align}
		\vec{C} - \vec{A} &= \myvec{4\\3} - \myvec{4\\5} = \myvec{0\\-2}\\
		\vec{D} - \vec{B} &= \myvec{1\\2} - \myvec{7\\6} = \myvec{-6\\-4}
	\end{align}
	\begin{align}
		\vec{B}-\vec{A} = \vec{C}-\vec{D} \text{ and } \vec{C}-\vec{B} = \vec{D}-\vec{A},
	\end{align}
	Hence, $ABCD$ is a parallelogram.
	\begin{enumerate}
		\item Now checking if the adjacent sides are orthogonal to each other
	\begin{align}
		(\vec{B}-\vec{A})^\top (\vec{C}-\vec{B}) = \myvec{3&1} \myvec{-3\\-3} = -9-3 = -12
	\end{align}
	Since inner product is not zero so adjacent sides are not orthogonal.

	Hence, we can say that $ABCD$ is neither a rectangle nor a square.

		\item Now checking if the diagonals are orthogonal then it is a Rhombus.
	\begin{align}
		(\vec{C}- \vec{A})^\top (\vec{D}-\vec{B}) = \myvec{0&-2} \myvec{-6\\-4} = 0+8 = 8
	\end{align}
	\end{enumerate}		
	Hence the diagonals are also not orthogonal so we conclude that $ABCD$ is a parallelogram.

	As shown in Figure \ref{fig:10/7/1/6/Fig3} we can see that $ABCD$ forms a parallelogram hence, our theoritical result is verified.

\begin{figure}[!h]
	\begin{center} 
	    \includegraphics[width=\columnwidth]{chapters/10/7/1/6/figs/quad3}
	\end{center}
\caption{}
\label{fig:10/7/1/6/Fig3}
\end{figure}
\end{enumerate}



\item If $\vec{a},\vec{b},\vec{c}$ are mutually perpendicular vectors of equal magnitudes, show that the vector $\vec{c}.\vec{d}$=15 is equally inclined to $\vec{a},\vec{b}$ and $\vec{c}$.\\
    \item If $ \vec{A},\vec{B},\vec{C} $ are mutually perpendicular vectors of equal magnitudes,show that the  $ \vec{A}+\vec{B}+\vec{C} $ is equally inclined to $ \vec{A},\vec{B}  \text{ and }  \vec{C} $.\\
    \textbf{Solution:}
    
    Suppose we have the following vectors:
    \begin{align*}
        \vec{v}_1 = \myvec{1 \\1 \\1}  \\
        \vec{v}_2 = \myvec{6\\ 4\\ 5}  \\
        \vec{v}_3 = \myvec{3 \\6\\ 9}
    \end{align*}
        

\textbf{Step 1: Initialize}

Set $\vec{u}_1 = \vec{v}_1$:\\

 $\vec{u}_1 = \myvec{1\\1\\1}$
 

\textbf{Step 2: Orthogonalization}

For  $ \vec{v}_2$ :
 \begin{align}
     \vec{u}_2 = \vec{v}_2 - \frac{\langle \vec{v}_2, \vec{u}_1 \rangle}{\langle \vec{u}_1, \vec{u}_1 \rangle} \vec{u}_1 \\
     \vec{u}_2=\vec{v}_2- \brak{\vec{u}_1 ^\top \vec{v}_2} \vec{u}_1\\ 
     \vec{u}_2\implies \myvec{1\\-1\\0}
 \end{align}

For $\vec{v}_3 $:
\begin{align}
    \vec{u}_3 = \vec{v}_3 - \frac{\langle \vec{v}_3, \vec{u}_1 \rangle}{\langle \vec{u}_1, \vec{u}_1 \rangle} \vec{u}_1 - \frac{\langle \vec{v}_3, \vec{u}_2 \rangle}{\langle \vec{u}_2, \vec{u}_2 \rangle} \vec{u}_2 \\
    \vec{u}_3=\vec{v}_3- \brak{\vec{u}_2 ^\top \vec{v}_3} \vec{u}_2- \brak{\vec{u}_1 ^\top \vec{v}_3} \vec{u}_1\\ 
\vec{u}_3\implies \myvec{-1\\-1\\2}
\end{align}

\textbf{Step 3: Normalization}

Normalize each vector:
\begin{align}
\vec{u}_1 = \frac{\vec{u}_1}{\norm{\vec{u}_1}} \\
\vec{u}_2 = \frac{\vec{u}_2}{\norm{\vec{u}_2}} \\
\vec{u}_3 = \frac{\vec{u}_3}{\norm{\vec{u}_3}} 
\end{align}

The final orthonormal basis is:
\begin{align*}
\vec{u}_1 = \myvec{\frac{1}{\sqrt{3}}\\ \frac{1}{\sqrt{3}}\\ \frac{1}{\sqrt{3}}} 
\implies\myvec{0.577\\0.577\\0.577}\\
\vec{u}_2 = \myvec{\frac{1}{\sqrt{2}}\\\frac{-1}{\sqrt{2}}\\0 }
\implies \myvec{0.707 \\ -0.707 \\ 0}\\
\vec{u}_3 = \myvec{\frac{-1}{\sqrt{6}}\\ \frac{-1}{\sqrt{6}}\\ \frac{2}{\sqrt{6}}} 
\implies \myvec{-0.408\\-0.408\\0.816}
\end{align*}
\textbf{Step 4: QR Decoposition}

we calculate Q by means of Gram–Schmidt process\\
$Q$ is an orthogonal matrix 
\begin{align*}
    Q=\myvec{ 0.577&0.707&-0.408\\0.577&-0.707&-0.408\\0.577&0&0.816}
\end{align*}
To verify it as a orthonormal matrix we have to check this property i.e,  $Q^{\top}.Q =I$
\begin{align*}
    \implies Q^\top Q &= \myvec{1&0&0\\0&1&0\\0&0&1}
\end{align*}
\textbf{Step 5: Findings angles between $\vec{u}_1,\vec{u}_2,\vec{u}_3  \text{ and } \vec{u}_1+\vec{u}_2+\vec{u}_3 $}
\begin{align}
    \vec{u}_1=\myvec{0.577\\0.577\\0.577}\\
    \vec{u}_2=\myvec{0.707 \\ -0.707 \\ 0} \\
    \vec{u}_3=\myvec{-0.408\\-0.408\\0.816}\\
    \vec{u}_1+\vec{u}_2+\vec{u}_3\implies\vec{y}=\myvec{0.876\\-0.538\\1.393}
\end{align}
Normalize each vector:\\
   \begin{align}
    \norm{\vec{u}_1}=1\\
    \norm{\vec{u}_2}=0.9998\\
     \norm{\vec{u}_3}=0.9987\\
     \norm{\vec{y}}=1.732
   \end{align}
Finding angles:
\begin{align}
    \cos{\theta_1}&=\frac{\myvec{0.577 \\0.577 \\0.577}\myvec{0.876&-0.538&1.393}}{\brak{1}\brak{1.732}}\\
    \cos{\theta_1}&=\frac{1}{1.732}\\
    \theta_1=\cos^{-1}\brak{\frac{1}{1.732}}
    \implies 54.73\degree
\end{align}
\begin{align}
   \cos{\theta_2}&=\frac{\myvec{0.707 \\-0.707 \\0}\myvec{0.876&-0.538&1.393}}{\brak{0.9998}\brak{1.732}}\\
    \cos{\theta_2}&=\frac{1}{1.732}\\
    \theta_2=\cos^{-1}\brak{\frac{1}{1.732}}
    \implies 54.73\degree
\end{align}
\begin{align}
     \cos{\theta_3}&=\frac{\myvec{-0.408 \\-0.408\\0.816}\myvec{0.876&-0.538&1.393}}{\brak{0.9987}\brak{1.732}}\\
    \cos{\theta_3}&=\frac{1}{1.732}\\
    \theta_3=\cos^{-1}\brak{\frac{1}{1.732}}
    \implies 54.73\degree
\end{align}
\begin{align*}
    \therefore  \theta_1=\theta_2=\theta_3
\end{align*}
Hence  we can say that $\vec{u}_1+\vec{u}_2+\vec{u}_3 $ is equally inclined to $\vec{u}_1,\vec{u}_2 \text{and} \vec{u}_3  $ 
\end{enumerate}

\subsection{Vector Product}
\begin{enumerate}[label=\thesection.\arabic*,ref=\thesection.\theenumi]
		\item Find $\abs{\overrightarrow{a}\times\overrightarrow{b}},\text{ if }\overrightarrow{a}=\hat{i}-7\hat{j}+7\hat{k}\text{ and } \overrightarrow{b}=3\hat{i}-2\hat{j}+2\hat{k}$.
	\\
		\solution
		\iffalse
\documentclass[12pt]{article}
\usepackage{graphicx}
%\documentclass[journal,12pt,twocolumn]{IEEEtran}
\usepackage[none]{hyphenat}
\usepackage{graphicx}
\usepackage{listings}
\usepackage[english]{babel}
\usepackage{graphicx}
\usepackage{caption} 
\usepackage{hyperref}
\usepackage{booktabs}
\def\inputGnumericTable{}
\usepackage{color}                                            %%
    \usepackage{array}                                            %%
    \usepackage{longtable}                                        %%
    \usepackage{calc}                                             %%
    \usepackage{multirow}                                         %%
    \usepackage{hhline}                                           %%
    \usepackage{ifthen}
\usepackage{array}
\usepackage{amsmath}   % for having text in math mode
\usepackage{listings}
\lstset{
language=tex,
frame=single, 
breaklines=true
}
  
%Following 2 lines were added to remove the blank page at the beginning
\usepackage{atbegshi}% http://ctan.org/pkg/atbegshi
\AtBeginDocument{\AtBeginShipoutNext{\AtBeginShipoutDiscard}}
%


%New macro definitions
\newcommand{\mydet}[1]{\ensuremath{\begin{vmatrix}#1\end{vmatrix}}}
\providecommand{\brak}[1]{\ensuremath{\left(#1\right)}}
\providecommand{\norm}[1]{\left\lVert#1\right\rVert}
\newcommand{\solution}{\noindent \textbf{Solution: }}
\newcommand{\myvec}[1]{\ensuremath{\begin{pmatrix}#1\end{pmatrix}}}
\let\vec\mathbf

\begin{document}

\begin{center}
\title{\textbf{Coordinate Geometry}}
\date{\vspace{-5ex}} %Not to print date automatically
\maketitle
\end{center}

\setcounter{page}{1}



\begin{enumerate}

\item\textbf{Problem statement :} Find the area of a rhombus of its vertices are $\myvec{3 ,0}$, $\myvec{4 ,5}$, $\myvec{-1 ,4}$ and $\myvec{-2 ,-1}$taken in order

\solution \\
\fi
The input vertices for this problem are given as
	\begin{align}
	\vec{A} = \myvec{
		3\\
		0
		},
	\vec{B} = \myvec{
		4\\
		5
		},
        \vec{C} = \myvec{
		-1\\
		4
		},
        \vec{D} = \myvec{
		-2\\
		-1
		}
	\end{align}
Since		
\begin{align}
 \vec{A-D}= \myvec{3 \\ 0} - \myvec{-2 \\-1}= \myvec{5\\1}
 \\
  \vec{B-A}= \myvec{4 \\ 5} - \myvec{3 \\0}= \myvec{1\\5}
\end{align}
the area of the rhombus is
\begin{align}
                \norm{\myvec{\vec{A-D}}\times \myvec{\vec{B-A}}}=\mydet{5 & 1\\1 & 5} = 24
\end{align}
See Fig. 
\ref{fig:chapters/10/7/2/10/gFig1}.
\begin{figure}[!h]
 \begin{center}
  \includegraphics[width=\columnwidth]{chapters/10/7/2/10/figs/fig.pdf}
 \end{center}
\caption{}
\label{fig:chapters/10/7/2/10/gFig1}
\end{figure}

\item Show that $$(\overrightarrow{a}-\overrightarrow{b})\times (\overrightarrow{a}+\overrightarrow{b})=2(\overrightarrow{a}\times \overrightarrow{b})$$
	\\
		\solution
		\iffalse
\documentclass[12pt]{article}
\usepackage{graphicx}
%\documentclass[journal,12pt,twocolumn]{IEEEtran}
\usepackage[none]{hyphenat}
\usepackage{graphicx}
\usepackage{listings}
\usepackage[english]{babel}
\usepackage{graphicx}
\usepackage{caption} 
\usepackage{hyperref}
\usepackage{booktabs}
\def\inputGnumericTable{}
\usepackage{color}                                            %%
    \usepackage{array}                                            %%
    \usepackage{longtable}                                        %%
    \usepackage{calc}                                             %%
    \usepackage{multirow}                                         %%
    \usepackage{hhline}                                           %%
    \usepackage{ifthen}
\usepackage{array}
\usepackage{amsmath}   % for having text in math mode
\usepackage{listings}
\lstset{
language=tex,
frame=single, 
breaklines=true
}
  
%Following 2 lines were added to remove the blank page at the beginning
\usepackage{atbegshi}% http://ctan.org/pkg/atbegshi
\AtBeginDocument{\AtBeginShipoutNext{\AtBeginShipoutDiscard}}
%


%New macro definitions
\newcommand{\mydet}[1]{\ensuremath{\begin{vmatrix}#1\end{vmatrix}}}
\providecommand{\brak}[1]{\ensuremath{\left(#1\right)}}
\providecommand{\norm}[1]{\left\lVert#1\right\rVert}
\newcommand{\solution}{\noindent \textbf{Solution: }}
\newcommand{\myvec}[1]{\ensuremath{\begin{pmatrix}#1\end{pmatrix}}}
\let\vec\mathbf

\begin{document}

\begin{center}
\title{\textbf{Coordinate Geometry}}
\date{\vspace{-5ex}} %Not to print date automatically
\maketitle
\end{center}

\setcounter{page}{1}



\begin{enumerate}

\item\textbf{Problem statement :} Find the area of a rhombus of its vertices are $\myvec{3 ,0}$, $\myvec{4 ,5}$, $\myvec{-1 ,4}$ and $\myvec{-2 ,-1}$taken in order

\solution \\
\fi
The input vertices for this problem are given as
	\begin{align}
	\vec{A} = \myvec{
		3\\
		0
		},
	\vec{B} = \myvec{
		4\\
		5
		},
        \vec{C} = \myvec{
		-1\\
		4
		},
        \vec{D} = \myvec{
		-2\\
		-1
		}
	\end{align}
Since		
\begin{align}
 \vec{A-D}= \myvec{3 \\ 0} - \myvec{-2 \\-1}= \myvec{5\\1}
 \\
  \vec{B-A}= \myvec{4 \\ 5} - \myvec{3 \\0}= \myvec{1\\5}
\end{align}
the area of the rhombus is
\begin{align}
                \norm{\myvec{\vec{A-D}}\times \myvec{\vec{B-A}}}=\mydet{5 & 1\\1 & 5} = 24
\end{align}
See Fig. 
\ref{fig:chapters/10/7/2/10/gFig1}.
\begin{figure}[!h]
 \begin{center}
  \includegraphics[width=\columnwidth]{chapters/10/7/2/10/figs/fig.pdf}
 \end{center}
\caption{}
\label{fig:chapters/10/7/2/10/gFig1}
\end{figure}

\item Find $\lambda$ and $\mu$ if $(2\hat{i}+6\hat{j}+27\hat{k})\times(\hat{i}+\lambda \hat{j} + \mu \hat{k})=\overrightarrow{0}$.
	\\
		\solution
		\iffalse
\documentclass[12pt]{article}
\usepackage{graphicx}
%\documentclass[journal,12pt,twocolumn]{IEEEtran}
\usepackage[none]{hyphenat}
\usepackage{graphicx}
\usepackage{listings}
\usepackage[english]{babel}
\usepackage{graphicx}
\usepackage{caption} 
\usepackage{hyperref}
\usepackage{booktabs}
\def\inputGnumericTable{}
\usepackage{color}                                            %%
    \usepackage{array}                                            %%
    \usepackage{longtable}                                        %%
    \usepackage{calc}                                             %%
    \usepackage{multirow}                                         %%
    \usepackage{hhline}                                           %%
    \usepackage{ifthen}
\usepackage{array}
\usepackage{amsmath}   % for having text in math mode
\usepackage{listings}
\lstset{
language=tex,
frame=single, 
breaklines=true
}
  
%Following 2 lines were added to remove the blank page at the beginning
\usepackage{atbegshi}% http://ctan.org/pkg/atbegshi
\AtBeginDocument{\AtBeginShipoutNext{\AtBeginShipoutDiscard}}
%


%New macro definitions
\newcommand{\mydet}[1]{\ensuremath{\begin{vmatrix}#1\end{vmatrix}}}
\providecommand{\brak}[1]{\ensuremath{\left(#1\right)}}
\providecommand{\norm}[1]{\left\lVert#1\right\rVert}
\newcommand{\solution}{\noindent \textbf{Solution: }}
\newcommand{\myvec}[1]{\ensuremath{\begin{pmatrix}#1\end{pmatrix}}}
\let\vec\mathbf

\begin{document}

\begin{center}
\title{\textbf{Coordinate Geometry}}
\date{\vspace{-5ex}} %Not to print date automatically
\maketitle
\end{center}

\setcounter{page}{1}



\begin{enumerate}

\item\textbf{Problem statement :} Find the area of a rhombus of its vertices are $\myvec{3 ,0}$, $\myvec{4 ,5}$, $\myvec{-1 ,4}$ and $\myvec{-2 ,-1}$taken in order

\solution \\
\fi
The input vertices for this problem are given as
	\begin{align}
	\vec{A} = \myvec{
		3\\
		0
		},
	\vec{B} = \myvec{
		4\\
		5
		},
        \vec{C} = \myvec{
		-1\\
		4
		},
        \vec{D} = \myvec{
		-2\\
		-1
		}
	\end{align}
Since		
\begin{align}
 \vec{A-D}= \myvec{3 \\ 0} - \myvec{-2 \\-1}= \myvec{5\\1}
 \\
  \vec{B-A}= \myvec{4 \\ 5} - \myvec{3 \\0}= \myvec{1\\5}
\end{align}
the area of the rhombus is
\begin{align}
                \norm{\myvec{\vec{A-D}}\times \myvec{\vec{B-A}}}=\mydet{5 & 1\\1 & 5} = 24
\end{align}
See Fig. 
\ref{fig:chapters/10/7/2/10/gFig1}.
\begin{figure}[!h]
 \begin{center}
  \includegraphics[width=\columnwidth]{chapters/10/7/2/10/figs/fig.pdf}
 \end{center}
\caption{}
\label{fig:chapters/10/7/2/10/gFig1}
\end{figure}

\item Given that $\overrightarrow{a} \cdot \overrightarrow{b} = 0$ and $\overrightarrow{a} \times \overrightarrow{b} = \overrightarrow{0}$. What can you conclude about the vectors $\overrightarrow{a} \text{ and }\overrightarrow{b}$?
\item Let the vectors be given as $\overrightarrow{a},\overrightarrow{b},\overrightarrow{c}\text{ be given as }\ a_1 \hat{i}+\ a_2 \hat{j}+\ a_3 \hat{k},\ b_1 \hat{i}+\ b_2 \hat{j}+\ b_3 \hat{k},\ c_1 \hat{i}+\ c_2 \hat{j}+\ c_3 \hat{k}$. Then show that $\overrightarrow{a} \times (\overrightarrow{b} + \overrightarrow{c}) = \overrightarrow{a} \times \overrightarrow{b}+\overrightarrow{a} \times \overrightarrow{c}$.
	\\
		\solution
		\iffalse
\documentclass[12pt]{article}
\usepackage{graphicx}
%\documentclass[journal,12pt,twocolumn]{IEEEtran}
\usepackage[none]{hyphenat}
\usepackage{graphicx}
\usepackage{listings}
\usepackage[english]{babel}
\usepackage{graphicx}
\usepackage{caption} 
\usepackage{hyperref}
\usepackage{booktabs}
\def\inputGnumericTable{}
\usepackage{color}                                            %%
    \usepackage{array}                                            %%
    \usepackage{longtable}                                        %%
    \usepackage{calc}                                             %%
    \usepackage{multirow}                                         %%
    \usepackage{hhline}                                           %%
    \usepackage{ifthen}
\usepackage{array}
\usepackage{amsmath}   % for having text in math mode
\usepackage{listings}
\lstset{
language=tex,
frame=single, 
breaklines=true
}
  
%Following 2 lines were added to remove the blank page at the beginning
\usepackage{atbegshi}% http://ctan.org/pkg/atbegshi
\AtBeginDocument{\AtBeginShipoutNext{\AtBeginShipoutDiscard}}
%


%New macro definitions
\newcommand{\mydet}[1]{\ensuremath{\begin{vmatrix}#1\end{vmatrix}}}
\providecommand{\brak}[1]{\ensuremath{\left(#1\right)}}
\providecommand{\norm}[1]{\left\lVert#1\right\rVert}
\newcommand{\solution}{\noindent \textbf{Solution: }}
\newcommand{\myvec}[1]{\ensuremath{\begin{pmatrix}#1\end{pmatrix}}}
\let\vec\mathbf

\begin{document}

\begin{center}
\title{\textbf{Coordinate Geometry}}
\date{\vspace{-5ex}} %Not to print date automatically
\maketitle
\end{center}

\setcounter{page}{1}



\begin{enumerate}

\item\textbf{Problem statement :} Find the area of a rhombus of its vertices are $\myvec{3 ,0}$, $\myvec{4 ,5}$, $\myvec{-1 ,4}$ and $\myvec{-2 ,-1}$taken in order

\solution \\
\fi
The input vertices for this problem are given as
	\begin{align}
	\vec{A} = \myvec{
		3\\
		0
		},
	\vec{B} = \myvec{
		4\\
		5
		},
        \vec{C} = \myvec{
		-1\\
		4
		},
        \vec{D} = \myvec{
		-2\\
		-1
		}
	\end{align}
Since		
\begin{align}
 \vec{A-D}= \myvec{3 \\ 0} - \myvec{-2 \\-1}= \myvec{5\\1}
 \\
  \vec{B-A}= \myvec{4 \\ 5} - \myvec{3 \\0}= \myvec{1\\5}
\end{align}
the area of the rhombus is
\begin{align}
                \norm{\myvec{\vec{A-D}}\times \myvec{\vec{B-A}}}=\mydet{5 & 1\\1 & 5} = 24
\end{align}
See Fig. 
\ref{fig:chapters/10/7/2/10/gFig1}.
\begin{figure}[!h]
 \begin{center}
  \includegraphics[width=\columnwidth]{chapters/10/7/2/10/figs/fig.pdf}
 \end{center}
\caption{}
\label{fig:chapters/10/7/2/10/gFig1}
\end{figure}

\item If either $\overrightarrow{a} = \overrightarrow{0}$ or $\overrightarrow{b} = \overrightarrow{0}$, then $\overrightarrow{a} \times \overrightarrow{b} = \overrightarrow{0}$. Is the converse true? Justify your answer with an example.
	\\
		\solution
		\iffalse
\documentclass[12pt]{article}
\usepackage{graphicx}
%\documentclass[journal,12pt,twocolumn]{IEEEtran}
\usepackage[none]{hyphenat}
\usepackage{graphicx}
\usepackage{listings}
\usepackage[english]{babel}
\usepackage{graphicx}
\usepackage{caption} 
\usepackage{hyperref}
\usepackage{booktabs}
\def\inputGnumericTable{}
\usepackage{color}                                            %%
    \usepackage{array}                                            %%
    \usepackage{longtable}                                        %%
    \usepackage{calc}                                             %%
    \usepackage{multirow}                                         %%
    \usepackage{hhline}                                           %%
    \usepackage{ifthen}
\usepackage{array}
\usepackage{amsmath}   % for having text in math mode
\usepackage{listings}
\lstset{
language=tex,
frame=single, 
breaklines=true
}
  
%Following 2 lines were added to remove the blank page at the beginning
\usepackage{atbegshi}% http://ctan.org/pkg/atbegshi
\AtBeginDocument{\AtBeginShipoutNext{\AtBeginShipoutDiscard}}
%


%New macro definitions
\newcommand{\mydet}[1]{\ensuremath{\begin{vmatrix}#1\end{vmatrix}}}
\providecommand{\brak}[1]{\ensuremath{\left(#1\right)}}
\providecommand{\norm}[1]{\left\lVert#1\right\rVert}
\newcommand{\solution}{\noindent \textbf{Solution: }}
\newcommand{\myvec}[1]{\ensuremath{\begin{pmatrix}#1\end{pmatrix}}}
\let\vec\mathbf

\begin{document}

\begin{center}
\title{\textbf{Coordinate Geometry}}
\date{\vspace{-5ex}} %Not to print date automatically
\maketitle
\end{center}

\setcounter{page}{1}



\begin{enumerate}

\item\textbf{Problem statement :} Find the area of a rhombus of its vertices are $\myvec{3 ,0}$, $\myvec{4 ,5}$, $\myvec{-1 ,4}$ and $\myvec{-2 ,-1}$taken in order

\solution \\
\fi
The input vertices for this problem are given as
	\begin{align}
	\vec{A} = \myvec{
		3\\
		0
		},
	\vec{B} = \myvec{
		4\\
		5
		},
        \vec{C} = \myvec{
		-1\\
		4
		},
        \vec{D} = \myvec{
		-2\\
		-1
		}
	\end{align}
Since		
\begin{align}
 \vec{A-D}= \myvec{3 \\ 0} - \myvec{-2 \\-1}= \myvec{5\\1}
 \\
  \vec{B-A}= \myvec{4 \\ 5} - \myvec{3 \\0}= \myvec{1\\5}
\end{align}
the area of the rhombus is
\begin{align}
                \norm{\myvec{\vec{A-D}}\times \myvec{\vec{B-A}}}=\mydet{5 & 1\\1 & 5} = 24
\end{align}
See Fig. 
\ref{fig:chapters/10/7/2/10/gFig1}.
\begin{figure}[!h]
 \begin{center}
  \includegraphics[width=\columnwidth]{chapters/10/7/2/10/figs/fig.pdf}
 \end{center}
\caption{}
\label{fig:chapters/10/7/2/10/gFig1}
\end{figure}

\item Find the area of the triangle with vertices $A(1, 1, 2)$, $B(2, 3, 5)$, and $C(1, 5, 5)$
	\\
		\solution
		\iffalse
\documentclass[12pt]{article}
\usepackage{graphicx}
%\documentclass[journal,12pt,twocolumn]{IEEEtran}
\usepackage[none]{hyphenat}
\usepackage{graphicx}
\usepackage{listings}
\usepackage[english]{babel}
\usepackage{graphicx}
\usepackage{caption} 
\usepackage{hyperref}
\usepackage{booktabs}
\def\inputGnumericTable{}
\usepackage{color}                                            %%
    \usepackage{array}                                            %%
    \usepackage{longtable}                                        %%
    \usepackage{calc}                                             %%
    \usepackage{multirow}                                         %%
    \usepackage{hhline}                                           %%
    \usepackage{ifthen}
\usepackage{array}
\usepackage{amsmath}   % for having text in math mode
\usepackage{listings}
\lstset{
language=tex,
frame=single, 
breaklines=true
}
  
%Following 2 lines were added to remove the blank page at the beginning
\usepackage{atbegshi}% http://ctan.org/pkg/atbegshi
\AtBeginDocument{\AtBeginShipoutNext{\AtBeginShipoutDiscard}}
%


%New macro definitions
\newcommand{\mydet}[1]{\ensuremath{\begin{vmatrix}#1\end{vmatrix}}}
\providecommand{\brak}[1]{\ensuremath{\left(#1\right)}}
\providecommand{\norm}[1]{\left\lVert#1\right\rVert}
\newcommand{\solution}{\noindent \textbf{Solution: }}
\newcommand{\myvec}[1]{\ensuremath{\begin{pmatrix}#1\end{pmatrix}}}
\let\vec\mathbf

\begin{document}

\begin{center}
\title{\textbf{Coordinate Geometry}}
\date{\vspace{-5ex}} %Not to print date automatically
\maketitle
\end{center}

\setcounter{page}{1}



\begin{enumerate}

\item\textbf{Problem statement :} Find the area of a rhombus of its vertices are $\myvec{3 ,0}$, $\myvec{4 ,5}$, $\myvec{-1 ,4}$ and $\myvec{-2 ,-1}$taken in order

\solution \\
\fi
The input vertices for this problem are given as
	\begin{align}
	\vec{A} = \myvec{
		3\\
		0
		},
	\vec{B} = \myvec{
		4\\
		5
		},
        \vec{C} = \myvec{
		-1\\
		4
		},
        \vec{D} = \myvec{
		-2\\
		-1
		}
	\end{align}
Since		
\begin{align}
 \vec{A-D}= \myvec{3 \\ 0} - \myvec{-2 \\-1}= \myvec{5\\1}
 \\
  \vec{B-A}= \myvec{4 \\ 5} - \myvec{3 \\0}= \myvec{1\\5}
\end{align}
the area of the rhombus is
\begin{align}
                \norm{\myvec{\vec{A-D}}\times \myvec{\vec{B-A}}}=\mydet{5 & 1\\1 & 5} = 24
\end{align}
See Fig. 
\ref{fig:chapters/10/7/2/10/gFig1}.
\begin{figure}[!h]
 \begin{center}
  \includegraphics[width=\columnwidth]{chapters/10/7/2/10/figs/fig.pdf}
 \end{center}
\caption{}
\label{fig:chapters/10/7/2/10/gFig1}
\end{figure}

\item Find the area of the parallelogram whose adjacent sides are determined by the vectors $\overrightarrow{a}=\hat{i}-\hat{j}+3\hat{k}$ and $\overrightarrow{b}=2\hat{i}-7\hat{j}+\hat{k}$.
	\\
		\solution
		\iffalse
\documentclass[12pt]{article}
\usepackage{graphicx}
%\documentclass[journal,12pt,twocolumn]{IEEEtran}
\usepackage[none]{hyphenat}
\usepackage{graphicx}
\usepackage{listings}
\usepackage[english]{babel}
\usepackage{graphicx}
\usepackage{caption} 
\usepackage{hyperref}
\usepackage{booktabs}
\def\inputGnumericTable{}
\usepackage{color}                                            %%
    \usepackage{array}                                            %%
    \usepackage{longtable}                                        %%
    \usepackage{calc}                                             %%
    \usepackage{multirow}                                         %%
    \usepackage{hhline}                                           %%
    \usepackage{ifthen}
\usepackage{array}
\usepackage{amsmath}   % for having text in math mode
\usepackage{listings}
\lstset{
language=tex,
frame=single, 
breaklines=true
}
  
%Following 2 lines were added to remove the blank page at the beginning
\usepackage{atbegshi}% http://ctan.org/pkg/atbegshi
\AtBeginDocument{\AtBeginShipoutNext{\AtBeginShipoutDiscard}}
%


%New macro definitions
\newcommand{\mydet}[1]{\ensuremath{\begin{vmatrix}#1\end{vmatrix}}}
\providecommand{\brak}[1]{\ensuremath{\left(#1\right)}}
\providecommand{\norm}[1]{\left\lVert#1\right\rVert}
\newcommand{\solution}{\noindent \textbf{Solution: }}
\newcommand{\myvec}[1]{\ensuremath{\begin{pmatrix}#1\end{pmatrix}}}
\let\vec\mathbf

\begin{document}

\begin{center}
\title{\textbf{Coordinate Geometry}}
\date{\vspace{-5ex}} %Not to print date automatically
\maketitle
\end{center}

\setcounter{page}{1}



\begin{enumerate}

\item\textbf{Problem statement :} Find the area of a rhombus of its vertices are $\myvec{3 ,0}$, $\myvec{4 ,5}$, $\myvec{-1 ,4}$ and $\myvec{-2 ,-1}$taken in order

\solution \\
\fi
The input vertices for this problem are given as
	\begin{align}
	\vec{A} = \myvec{
		3\\
		0
		},
	\vec{B} = \myvec{
		4\\
		5
		},
        \vec{C} = \myvec{
		-1\\
		4
		},
        \vec{D} = \myvec{
		-2\\
		-1
		}
	\end{align}
Since		
\begin{align}
 \vec{A-D}= \myvec{3 \\ 0} - \myvec{-2 \\-1}= \myvec{5\\1}
 \\
  \vec{B-A}= \myvec{4 \\ 5} - \myvec{3 \\0}= \myvec{1\\5}
\end{align}
the area of the rhombus is
\begin{align}
                \norm{\myvec{\vec{A-D}}\times \myvec{\vec{B-A}}}=\mydet{5 & 1\\1 & 5} = 24
\end{align}
See Fig. 
\ref{fig:chapters/10/7/2/10/gFig1}.
\begin{figure}[!h]
 \begin{center}
  \includegraphics[width=\columnwidth]{chapters/10/7/2/10/figs/fig.pdf}
 \end{center}
\caption{}
\label{fig:chapters/10/7/2/10/gFig1}
\end{figure}

\item Let the vectors $\overrightarrow{a}$ and $\overrightarrow{b}$ be such that $|\overrightarrow{a}| = 3$ and $|\overrightarrow{b}| = \dfrac{\sqrt{2}}{3}$, then $\overrightarrow{a} \times \overrightarrow{b}$ is a unit vector, if the angle between $\overrightarrow{a}$ and $\overrightarrow{b}$ is
\begin{enumerate}
\item $\dfrac{\pi}{6}$
\item $\dfrac{\pi}{4}$
\item $\dfrac{\pi}{3}$
\item $\dfrac{\pi}{2}$
\end{enumerate}
		\solution
		\iffalse
\documentclass[12pt]{article}
\usepackage{graphicx}
%\documentclass[journal,12pt,twocolumn]{IEEEtran}
\usepackage[none]{hyphenat}
\usepackage{graphicx}
\usepackage{listings}
\usepackage[english]{babel}
\usepackage{graphicx}
\usepackage{caption} 
\usepackage{hyperref}
\usepackage{booktabs}
\def\inputGnumericTable{}
\usepackage{color}                                            %%
    \usepackage{array}                                            %%
    \usepackage{longtable}                                        %%
    \usepackage{calc}                                             %%
    \usepackage{multirow}                                         %%
    \usepackage{hhline}                                           %%
    \usepackage{ifthen}
\usepackage{array}
\usepackage{amsmath}   % for having text in math mode
\usepackage{listings}
\lstset{
language=tex,
frame=single, 
breaklines=true
}
  
%Following 2 lines were added to remove the blank page at the beginning
\usepackage{atbegshi}% http://ctan.org/pkg/atbegshi
\AtBeginDocument{\AtBeginShipoutNext{\AtBeginShipoutDiscard}}
%


%New macro definitions
\newcommand{\mydet}[1]{\ensuremath{\begin{vmatrix}#1\end{vmatrix}}}
\providecommand{\brak}[1]{\ensuremath{\left(#1\right)}}
\providecommand{\norm}[1]{\left\lVert#1\right\rVert}
\newcommand{\solution}{\noindent \textbf{Solution: }}
\newcommand{\myvec}[1]{\ensuremath{\begin{pmatrix}#1\end{pmatrix}}}
\let\vec\mathbf

\begin{document}

\begin{center}
\title{\textbf{Coordinate Geometry}}
\date{\vspace{-5ex}} %Not to print date automatically
\maketitle
\end{center}

\setcounter{page}{1}



\begin{enumerate}

\item\textbf{Problem statement :} Find the area of a rhombus of its vertices are $\myvec{3 ,0}$, $\myvec{4 ,5}$, $\myvec{-1 ,4}$ and $\myvec{-2 ,-1}$taken in order

\solution \\
\fi
The input vertices for this problem are given as
	\begin{align}
	\vec{A} = \myvec{
		3\\
		0
		},
	\vec{B} = \myvec{
		4\\
		5
		},
        \vec{C} = \myvec{
		-1\\
		4
		},
        \vec{D} = \myvec{
		-2\\
		-1
		}
	\end{align}
Since		
\begin{align}
 \vec{A-D}= \myvec{3 \\ 0} - \myvec{-2 \\-1}= \myvec{5\\1}
 \\
  \vec{B-A}= \myvec{4 \\ 5} - \myvec{3 \\0}= \myvec{1\\5}
\end{align}
the area of the rhombus is
\begin{align}
                \norm{\myvec{\vec{A-D}}\times \myvec{\vec{B-A}}}=\mydet{5 & 1\\1 & 5} = 24
\end{align}
See Fig. 
\ref{fig:chapters/10/7/2/10/gFig1}.
\begin{figure}[!h]
 \begin{center}
  \includegraphics[width=\columnwidth]{chapters/10/7/2/10/figs/fig.pdf}
 \end{center}
\caption{}
\label{fig:chapters/10/7/2/10/gFig1}
\end{figure}

\item Area of a rectangle having vertices A, B, C and D with position vectors $ -\hat{i}+ \dfrac{1}{2} \hat{j}+4\hat{k},\hat{i}+ \dfrac{1}{2} \hat{j}+4\hat{k},\hat{i}-\dfrac{1}{2} \hat{j}+4\hat{k}\text{ and }-\hat{i}- \dfrac{1}{2} \hat{j}+4\hat{k}$, respectively is
\begin{enumerate}
\item $\dfrac{1}{2}$
\item 1
\item 2
\item 4
\end{enumerate}
		\solution
		\iffalse
\documentclass[12pt]{article}
\usepackage{graphicx}
%\documentclass[journal,12pt,twocolumn]{IEEEtran}
\usepackage[none]{hyphenat}
\usepackage{graphicx}
\usepackage{listings}
\usepackage[english]{babel}
\usepackage{graphicx}
\usepackage{caption} 
\usepackage{hyperref}
\usepackage{booktabs}
\def\inputGnumericTable{}
\usepackage{color}                                            %%
    \usepackage{array}                                            %%
    \usepackage{longtable}                                        %%
    \usepackage{calc}                                             %%
    \usepackage{multirow}                                         %%
    \usepackage{hhline}                                           %%
    \usepackage{ifthen}
\usepackage{array}
\usepackage{amsmath}   % for having text in math mode
\usepackage{listings}
\lstset{
language=tex,
frame=single, 
breaklines=true
}
  
%Following 2 lines were added to remove the blank page at the beginning
\usepackage{atbegshi}% http://ctan.org/pkg/atbegshi
\AtBeginDocument{\AtBeginShipoutNext{\AtBeginShipoutDiscard}}
%


%New macro definitions
\newcommand{\mydet}[1]{\ensuremath{\begin{vmatrix}#1\end{vmatrix}}}
\providecommand{\brak}[1]{\ensuremath{\left(#1\right)}}
\providecommand{\norm}[1]{\left\lVert#1\right\rVert}
\newcommand{\solution}{\noindent \textbf{Solution: }}
\newcommand{\myvec}[1]{\ensuremath{\begin{pmatrix}#1\end{pmatrix}}}
\let\vec\mathbf

\begin{document}

\begin{center}
\title{\textbf{Coordinate Geometry}}
\date{\vspace{-5ex}} %Not to print date automatically
\maketitle
\end{center}

\setcounter{page}{1}



\begin{enumerate}

\item\textbf{Problem statement :} Find the area of a rhombus of its vertices are $\myvec{3 ,0}$, $\myvec{4 ,5}$, $\myvec{-1 ,4}$ and $\myvec{-2 ,-1}$taken in order

\solution \\
\fi
The input vertices for this problem are given as
	\begin{align}
	\vec{A} = \myvec{
		3\\
		0
		},
	\vec{B} = \myvec{
		4\\
		5
		},
        \vec{C} = \myvec{
		-1\\
		4
		},
        \vec{D} = \myvec{
		-2\\
		-1
		}
	\end{align}
Since		
\begin{align}
 \vec{A-D}= \myvec{3 \\ 0} - \myvec{-2 \\-1}= \myvec{5\\1}
 \\
  \vec{B-A}= \myvec{4 \\ 5} - \myvec{3 \\0}= \myvec{1\\5}
\end{align}
the area of the rhombus is
\begin{align}
                \norm{\myvec{\vec{A-D}}\times \myvec{\vec{B-A}}}=\mydet{5 & 1\\1 & 5} = 24
\end{align}
See Fig. 
\ref{fig:chapters/10/7/2/10/gFig1}.
\begin{figure}[!h]
 \begin{center}
  \includegraphics[width=\columnwidth]{chapters/10/7/2/10/figs/fig.pdf}
 \end{center}
\caption{}
\label{fig:chapters/10/7/2/10/gFig1}
\end{figure}

\item Find the area of the triangle whose vertices are 
\begin{enumerate}
\item $(2, 3), (–1, 0), (2, – 4)$
\item $(–5, –1), (3, –5), (5, 2)$ 
\end{enumerate}
		\label{10/7/3/1}
\solution
		\iffalse
\documentclass[12pt]{article}
\usepackage{graphicx}
%\documentclass[journal,12pt,twocolumn]{IEEEtran}
\usepackage[none]{hyphenat}
\usepackage{graphicx}
\usepackage{listings}
\usepackage[english]{babel}
\usepackage{graphicx}
\usepackage{caption} 
\usepackage{hyperref}
\usepackage{booktabs}
\usepackage{array}
\usepackage{amsmath}   % for having text in math mode

%Following 2 lines were added to remove the blank page at the beginning
\usepackage{atbegshi}% http://ctan.org/pkg/atbegshi
\AtBeginDocument{\AtBeginShipoutNext{\AtBeginShipoutDiscard}}
%


%New macro definitions
\newcommand{\mydet}[1]{\ensuremath{\begin{vmatrix}#1\end{vmatrix}}}
\providecommand{\brak}[1]{\ensuremath{\left(#1\right)}}
\providecommand{\norm}[1]{\left\lVert#1\right\rVert}
\newcommand{\solution}{\noindent \textbf{Solution: }}
\newcommand{\myvec}[1]{\ensuremath{\begin{pmatrix}#1\end{pmatrix}}}
\let\vec\mathbf

\begin{document}

\begin{center}
\title{\textbf{Area of a Traingle}}
\date{\vspace{-5ex}} %Not to print date automatically
\maketitle
\end{center}

\setcounter{page}{1}



\section{10$^{th}$ Maths - Chapter 7}

This is Problem-1 from Exercise 7.3

\begin{enumerate}
\item Find the area of the triangle whose vertices are :
	\fi
\begin{enumerate}
\item 
In this case, the area  is given by  
  \label{prop:10/7/3/1area2d}
  \begin{align}
    \label{eq:10/7/3/1area2d}
	\frac{1}{2}\norm{\brak{\vec{A}-\vec{B}} \times \brak{\vec{A}-\vec{C}}} \\
  \end{align}
  Since
  \begin{align}
	 \vec{A}-\vec{B} =  \myvec{
  2 \\
  3 \\
 } - \myvec{
  -1 \\
  0 \\
 } = \myvec{
 3 \\
 3 \\
 }
 \\
 \vec{A}-\vec{C} =  \myvec{
  2 \\
  3 \\
 } - \myvec{
  2 \\
  -4 \\
 } = \myvec{
 0 \\
 7 \\
 }
 \end{align}
 the desired area is given by 
 \iffalse
The value of the cross product of two vectors is given by
\begin{align}
  \label{eq:10/7/3/1det2d}
  \mydet{\vec{M}} &= \mydet{\vec{A} & \vec{B}} 
  \\
  &= \mydet{a_1 & b_1\\a_2 & b_2} = a_1b_2 - a_2 b_1
\end{align}

		Therefore, \eqref{eq:10/7/3/1area2d} equals \\
		\fi
\begin{align}
	\frac{1}{2}\mydet{3 & 0\\3 & 7}  
	&=	\frac{21}{2}
\end{align}
\iffalse
\begin{figure}[!h]
	\begin{center}
		\includegraphics[width=\columnwidth]{./figs/problem1a.pdf}
	\end{center}
\caption{}
\label{fig:Fig1}
\end{figure}
\fi

\item In this case, 
	\iffalse
\solution The area of the triangle with vertices $\vec{A}, \vec{B}, \vec{C}$ is given by  
  \label{prop:10/7/3/1area2e}
  \begin{align}
    \label{eq:10/7/3/1area2e}
	\frac{1}{2}\norm{\brak{\vec{A}-\vec{B}} \times \brak{\vec{A}-\vec{C}}} \\
	\fi
  \begin{align}
	 \vec{A}-\vec{B} =  \myvec{
  -5 \\
  -1 \\
 } - \myvec{
  3 \\
  -5 \\
 } = \myvec{
 -8 \\
 4 \\
 }
 \\
 \vec{A}-\vec{C} =  \myvec{
  -5 \\
  -1 \\
 } - \myvec{
  5 \\
  2 \\
 } = \myvec{
 -10 \\
 -3 \\
 }
 \\
	  \implies
\text{Area} =	\frac{1}{2}\mydet{-8 & -10\\4 & -3}  
	=  32 
\end{align}
\iffalse
\begin{figure}[!h]
	\begin{center}
		\includegraphics[width=\columnwidth]{./figs/problem1b.pdf}
	\end{center}
\caption{}
\label{fig:Fig2}
\end{figure}
\fi
\end{enumerate}



\item Find the area of the triangle formed by joining the mid-points of the sides of the triangle whose vertices are $(0, –1), (2, 1) \text{ and } (0, 3)$. Find the ratio of this area to the area of the given triangle.
	\\
\solution
		\iffalse
\documentclass[12pt]{article}
\usepackage{graphicx}
\usepackage{amsmath}
\usepackage{mathtools}
\usepackage{gensymb}

\newcommand{\mydet}[1]{\ensuremath{\begin{vmatrix}#1\end{vmatrix}}}
\providecommand{\brak}[1]{\ensuremath{\left(#1\right)}}
\providecommand{\norm}[1]{\left\lVert#1\right\rVert}
\newcommand{\solution}{\noindent \textbf{Solution: }}
\newcommand{\myvec}[1]{\ensuremath{\begin{pmatrix}#1\end{pmatrix}}}
\let\vec\mathbf

\begin{document}
\begin{center}
\textbf\large{CHAPTER-7 \\ COORDINATE GEOMETRY}
\end{center}
\section*{Excercise 7.2}

Q3. Find the area of the triangle formed by joining the mid-points of the sides of the triangle
whose vertices are $\vec(0, –1), \vec(2, 1) \text{ and } \vec(0, 3)$. Find the ratio of this area to the area of the
given triangle
\\
\solution
\\
\fi
The coordinates are given as
	\begin{align}
	\vec{A} = \myvec{
		0\\
		-1\\
		},
	\vec{B} = \myvec{
		2\\
		1\\
		},
	\vec{C} = \myvec{
		0\\
		3\\
		}
	\end{align}
Calculating midpoints,
	\begin{align}
		\vec{P} = \frac{1}{2}\vec(\vec{A}+\vec{B}) = \frac{1}{2}\myvec{2\\0\\} = \myvec{1\\0\\}\\
		\vec{Q} = \frac{1}{2}\vec(\vec{B}+\vec{C}) = \frac{1}{2}\myvec{2\\4\\} = \myvec{1\\2\\}\\
		\vec{R} = \frac{1}{2}\vec(\vec{A}+\vec{C}) = \frac{1}{2}\myvec{0\\2\\} = \myvec{0\\1\\}
	\end{align}
	Since
	\begin{align}
		\vec{P}-\vec{Q} &=  \myvec{
  1 \\
  0 
 } - \myvec{
  1 \\
  2 
 } = \myvec{
 0 \\
 -2 
 }
		\\
		\vec{Q}-\vec{R} &=  \myvec{
  1 \\
  2 \\
 } - \myvec{
  0 \\
  1 \\
 } = \myvec{
 1 \\
 1 \\
 }
	\end{align}
	the area is obtained as
	\begin{align}
		ar(PQR)&=\frac{1}{2}{\norm{\vec(\vec{P}-\vec{Q})\times\vec(\vec{Q}-\vec{R})}}
		\\
		&=\frac{1}{2}\mydet{0 & 1\\-2 & 1}
		=1
	\end{align}
	Similarly, 
	\begin{align}
		\vec{A}-\vec{B} &=  \myvec{
  0 \\
  -1 \\
 } - \myvec{
  2 \\
  1 \\
 } = \myvec{
 -2 \\
 -2 \\
 }
 \\
		\vec{A}-\vec{C} &=  \myvec{
  0 \\
  -1 \\
 } - \myvec{
  0 \\
  3 \\
 } = \myvec{
 0 \\
 -4 \\
 }
	\end{align}
 the area is obained as
	\begin{align}
		ar(ABC)&=\frac{1}{2}{\norm{\vec(\vec{A}-\vec{B})\times\vec(\vec{A}-\vec{C})}}\\
		&=\frac{1}{2}\mydet{-2 & 0\\-2 & -4}
=4
	\end{align}
	Thus, the resultant ratio of two areas is 1:4.
	See Fig.
\ref{fig:10/7/3/3Fig}
\begin{figure}[!h]
	\begin{center} 
	    \includegraphics[width=\columnwidth]{chapters/10/7/3/3/figs/trigraph.png}
	\end{center}
\caption{}
\label{fig:10/7/3/3Fig}
\end{figure}


\item Find the area of the quadrilateral whose vertices, taken in order, are $(– 4, – 2), (– 3, – 5), (3, – 2)$  and $ (2, 3)$.
	\\
\solution
		\iffalse
\documentclass[12pt]{article}
\usepackage{graphicx}
%\documentclass[journal,12pt,twocolumn]{IEEEtran}
\usepackage[none]{hyphenat}
\usepackage{graphicx}
\usepackage{listings}
\usepackage[english]{babel}
\usepackage{graphicx}
\usepackage{caption} 
\usepackage{hyperref}
\usepackage{booktabs}
\def\inputGnumericTable{}
\usepackage{color}                                            %%
    \usepackage{array}                                            %%
    \usepackage{longtable}                                        %%
    \usepackage{calc}                                             %%
    \usepackage{multirow}                                         %%
    \usepackage{hhline}                                           %%
    \usepackage{ifthen}
\usepackage{array}
\usepackage{amsmath}   % for having text in math mode
\usepackage{listings}
\lstset{
language=tex,
frame=single, 
breaklines=true
}
  
%Following 2 lines were added to remove the blank page at the beginning
\usepackage{atbegshi}% http://ctan.org/pkg/atbegshi
\AtBeginDocument{\AtBeginShipoutNext{\AtBeginShipoutDiscard}}
%


%New macro definitions
\newcommand{\mydet}[1]{\ensuremath{\begin{vmatrix}#1\end{vmatrix}}}
\providecommand{\brak}[1]{\ensuremath{\left(#1\right)}}
\providecommand{\norm}[1]{\left\lVert#1\right\rVert}
\newcommand{\solution}{\noindent \textbf{Solution: }}
\newcommand{\myvec}[1]{\ensuremath{\begin{pmatrix}#1\end{pmatrix}}}
\let\vec\mathbf

\begin{document}

\begin{center}
\title{\textbf{Coordinate Geometry}}
\date{\vspace{-5ex}} %Not to print date automatically
\maketitle
\end{center}

\setcounter{page}{1}



\begin{enumerate}

\item\textbf{Problem statement :} Find the area of a rhombus of its vertices are $\myvec{3 ,0}$, $\myvec{4 ,5}$, $\myvec{-1 ,4}$ and $\myvec{-2 ,-1}$taken in order

\solution \\
\fi
The input vertices for this problem are given as
	\begin{align}
	\vec{A} = \myvec{
		3\\
		0
		},
	\vec{B} = \myvec{
		4\\
		5
		},
        \vec{C} = \myvec{
		-1\\
		4
		},
        \vec{D} = \myvec{
		-2\\
		-1
		}
	\end{align}
Since		
\begin{align}
 \vec{A-D}= \myvec{3 \\ 0} - \myvec{-2 \\-1}= \myvec{5\\1}
 \\
  \vec{B-A}= \myvec{4 \\ 5} - \myvec{3 \\0}= \myvec{1\\5}
\end{align}
the area of the rhombus is
\begin{align}
                \norm{\myvec{\vec{A-D}}\times \myvec{\vec{B-A}}}=\mydet{5 & 1\\1 & 5} = 24
\end{align}
See Fig. 
\ref{fig:chapters/10/7/2/10/gFig1}.
\begin{figure}[!h]
 \begin{center}
  \includegraphics[width=\columnwidth]{chapters/10/7/2/10/figs/fig.pdf}
 \end{center}
\caption{}
\label{fig:chapters/10/7/2/10/gFig1}
\end{figure}


\item Verify that a median of a triangle divides it into two triangles of equal areas for $\triangle ABC$ whose vertices are $\vec{A}(4, -6), \vec{B}(3, 2), \text{ and } \vec{C}(5, 2)$. 
		\label{10/7/3/5}
		\\
\solution
		\iffalse
\documentclass[12pt]{article}
\usepackage{graphicx}
\usepackage[none]{hyphenat}
\usepackage{graphicx}
\usepackage{listings}
\usepackage[english]{babel}
\usepackage{graphicx}
\usepackage{caption} 
\usepackage{booktabs}
\usepackage{array}
\usepackage{amssymb} % for \because
\usepackage{amsmath}   % for having text in math mode
\usepackage{extarrows} % for Row operations arrows
\usepackage{listings}
\usepackage[utf8]{inputenc}
\lstset{
  frame=single,
  breaklines=true
}
\usepackage{hyperref}
  
%Following 2 lines were added to remove the blank page at the beginning
\usepackage{atbegshi}% http://ctan.org/pkg/atbegshi
\AtBeginDocument{\AtBeginShipoutNext{\AtBeginShipoutDiscard}}


%New macro definitions
\newcommand{\mydet}[1]{\ensuremath{\begin{vmatrix}#1\end{vmatrix}}}
\providecommand{\brak}[1]{\ensuremath{\left(#1\right)}}
\newcommand{\solution}{\noindent \textbf{Solution: }}
\newcommand{\myvec}[1]{\ensuremath{\begin{pmatrix}#1\end{pmatrix}}}
\providecommand{\norm}[1]{\left\lVert#1\right\rVert}
\providecommand{\abs}[1]{\left\vert#1\right\vert}
\let\vec\mathbf

\begin{document}

\begin{center}
\title{\textbf{VECTORS}}
\date{\vspace{-5ex}} %Not to print date automatically
\maketitle
\end{center}

\section{10$^{th}$ Maths - EXERCISE-7.3}

\begin{enumerate}
\item That a median of a triangle divides it into two triangles  of equal areas. verify this result for $\triangle ABC$ whose vertices are $\vec{A}(4,-6),\vec{B}(3,-2)\text{ and }\vec{C}(5,2)$.
\end{enumerate}

\section{SOLUTION}
Given points are
\begin{align}
\vec{A}=\myvec{4\\ -6} ,
\vec{B}=\myvec{3\\ -2} ,
\vec{C}=\myvec{5\\ 2}
\end{align}
\fi
The median of the triangle 
\begin{align}
\vec{D}&=\frac{\vec{B}+\vec{C}}{2}\\
&=\myvec{4\\ 0}
\end{align}
Since 
\begin{align}
	\vec{A}- \vec{B} &= \myvec{4\\ -6}-\myvec{3\\ -2}=\myvec{1\\ -4}\label{eq:10/7/3/5/7}\\
	  \vec{A}- \vec{D} &= \myvec{4\\ -6}-\myvec{4\\ 0}=\myvec{0\\ -6}\label{eq:10/7/3/5/8}
  \end{align}
 \begin{align}
  ar(ABD)&=\frac{1}{2} \norm{\brak{\vec{A}-\vec{B}}  \times 
   \brak{\vec{A}- \vec{D}}} \label{eq:10/7/3/5/6} 
   \\
&=\frac{1}{2}\mydet{1 & 0\\-4 & -6}
	       =3	
\end{align}
upon
Substituting from \eqref{eq:10/7/3/5/7} and \eqref{eq:10/7/3/5/8} in \eqref{eq:10/7/3/5/6}.
		Similarly, 
\begin{align}
	\vec{A}- \vec{C} &= \myvec{4\\ -6}-\myvec{5\\ 2}=\myvec{-1\\ -8}\label{eq:10/7/3/5/13} \\
	  \vec{A}- \vec{D} &= \myvec{4\\ -6}-\myvec{4\\ 0}=\myvec{0\\ -6}\label{eq:10/7/3/5/14} 
  \end{align}
  yielding
  \begin{align}
  ar(ACD)&=\frac{1}{2} \norm{\brak{\vec{A}-\vec{C}}  \times 
   \brak{\vec{A}- \vec{D}}} \label{eq:10/7/3/5/12}
   \\
	&=\frac{1}{2}\mydet{-1 & 0\\-8 & -6}= 3
\end{align}
upon substituting from \eqref{eq:10/7/3/5/13} and \eqref{eq:10/7/3/5/14} in \eqref{eq:10/7/3/5/12}.
Thus,
\begin{align}
ar(ABD)=ar(ACD)
\end{align}
See Fig. 
\ref{fig:10/7/3/5/}.
\begin{figure}[h!]
\centering
\includegraphics[width=\columnwidth]{chapters/10/7/3/5/figs/fig.pdf}
\caption{}
\label{fig:10/7/3/5/}
\end{figure} 


\item The two adjacent sides of a parallelogram are 
$2\hat{i}-4\hat{j}+5\hat{k}$  and  $\hat{i}-2\hat{j}-3\hat{k}$.
Find the unit vector parallel to its diagonal. Also, find its area.\\
	\solution
		\iffalse
\documentclass[12pt]{article}
\usepackage{graphicx}
%\documentclass[journal,12pt,twocolumn]{IEEEtran}
\usepackage[none]{hyphenat}
\usepackage{graphicx}
\usepackage{listings}
\usepackage[english]{babel}
\usepackage{graphicx}
\usepackage{caption} 
\usepackage{hyperref}
\usepackage{booktabs}
\def\inputGnumericTable{}
\usepackage{color}                                            %%
    \usepackage{array}                                            %%
    \usepackage{longtable}                                        %%
    \usepackage{calc}                                             %%
    \usepackage{multirow}                                         %%
    \usepackage{hhline}                                           %%
    \usepackage{ifthen}
\usepackage{array}
\usepackage{amsmath}   % for having text in math mode
\usepackage{listings}
\lstset{
language=tex,
frame=single, 
breaklines=true
}
  
%Following 2 lines were added to remove the blank page at the beginning
\usepackage{atbegshi}% http://ctan.org/pkg/atbegshi
\AtBeginDocument{\AtBeginShipoutNext{\AtBeginShipoutDiscard}}
%


%New macro definitions
\newcommand{\mydet}[1]{\ensuremath{\begin{vmatrix}#1\end{vmatrix}}}
\providecommand{\brak}[1]{\ensuremath{\left(#1\right)}}
\providecommand{\norm}[1]{\left\lVert#1\right\rVert}
\newcommand{\solution}{\noindent \textbf{Solution: }}
\newcommand{\myvec}[1]{\ensuremath{\begin{pmatrix}#1\end{pmatrix}}}
\let\vec\mathbf

\begin{document}

\begin{center}
\title{\textbf{Coordinate Geometry}}
\date{\vspace{-5ex}} %Not to print date automatically
\maketitle
\end{center}

\setcounter{page}{1}



\begin{enumerate}

\item\textbf{Problem statement :} Find the area of a rhombus of its vertices are $\myvec{3 ,0}$, $\myvec{4 ,5}$, $\myvec{-1 ,4}$ and $\myvec{-2 ,-1}$taken in order

\solution \\
\fi
The input vertices for this problem are given as
	\begin{align}
	\vec{A} = \myvec{
		3\\
		0
		},
	\vec{B} = \myvec{
		4\\
		5
		},
        \vec{C} = \myvec{
		-1\\
		4
		},
        \vec{D} = \myvec{
		-2\\
		-1
		}
	\end{align}
Since		
\begin{align}
 \vec{A-D}= \myvec{3 \\ 0} - \myvec{-2 \\-1}= \myvec{5\\1}
 \\
  \vec{B-A}= \myvec{4 \\ 5} - \myvec{3 \\0}= \myvec{1\\5}
\end{align}
the area of the rhombus is
\begin{align}
                \norm{\myvec{\vec{A-D}}\times \myvec{\vec{B-A}}}=\mydet{5 & 1\\1 & 5} = 24
\end{align}
See Fig. 
\ref{fig:chapters/10/7/2/10/gFig1}.
\begin{figure}[!h]
 \begin{center}
  \includegraphics[width=\columnwidth]{chapters/10/7/2/10/figs/fig.pdf}
 \end{center}
\caption{}
\label{fig:chapters/10/7/2/10/gFig1}
\end{figure}

\item The vertices of a $\triangle ABC$ are $\vec{A}(4,6), \vec{B}(1,5)$ and  $\vec{C}(7,2)$. A line is drawn to intersect sides $AB$ and $AC$ at $\vec{D}$ and $\vec{E}$ respectively, such that $\frac{AD}{AB} = \frac{AE}{AC} = \frac{1}{4}$. Calculate the area of $\triangle ADE$ and compare it with the area of the $\triangle ABC$.
\\
\solution
	\begin{enumerate}[label=\thesection.\arabic*,ref=\thesection.\theenumi]
\numberwithin{equation}{enumi}
\numberwithin{figure}{enumi}
\numberwithin{table}{enumi}

\item Find the coordinates of the point which divides the join of $(-1,7) \text{ and } (4,-3)$ in the ratio 2:3.
	\\
		\solution
	\iffalse
\documentclass[12pt]{article}
\usepackage{graphicx}
\usepackage{amsmath}
\usepackage{mathtools}
\usepackage{gensymb}

\newcommand{\mydet}[1]{\ensuremath{\begin{vmatrix}#1\end{vmatrix}}}
\providecommand{\brak}[1]{\ensuremath{\left(#1\right)}}
\providecommand{\norm}[1]{\left\lVert#1\right\rVert}
\newcommand{\solution}{\noindent \textbf{Solution: }}
\newcommand{\myvec}[1]{\ensuremath{\begin{pmatrix}#1\end{pmatrix}}}
\let\vec\mathbf

\begin{document}
\begin{center}
\textbf\large{CHAPTER-7 \\ COORDINATE GEOMETRY}
\end{center}
\section*{Excercise 7.2}

1. Find the coordinates of the point which divides the join $\vec(-1,7) \text{ and } \vec(4,-3)$ in the ratio 2:3 :
\\
\\
\solution\\		
\fi
The coordinates and ratio are given as
\begin{align}
\vec{P}=\myvec{-1\\7\\},
\vec{Q}=\myvec{4\\-3\\},
n=\frac{3}{2}
\end{align}
Using section formula
\begin{align}
\vec{R}&=\frac{\vec{Q}+n\vec{P}}{1+n}\\
&=\frac{1}{1+\frac{3}{2}}  \myvec{\myvec{
4\\
-3\\
}
  +
   \frac{3}{2}\myvec{
-1\\
7\\
}}\\
&=\myvec{
1\\
3
}
\end{align}
See Fig. 
\ref{fig:chapters/10/7/2/1/Fig}
\begin{figure}[!h]
\begin{center}
   \includegraphics[width=\columnwidth]{chapters/10/7/2/1/figs/linefig.png}
\end{center}
\caption{}
\label{fig:chapters/10/7/2/1/Fig}
\end{figure}


\item Find the coordinates of the points of trisection of the line segment joining $(4,-1) \text{ and } (-2,3)$.
	\\
		\solution
	\begin{enumerate}[label=\thesection.\arabic*,ref=\thesection.\theenumi]
\numberwithin{equation}{enumi}
\numberwithin{figure}{enumi}
\numberwithin{table}{enumi}

\item Find the coordinates of the point which divides the join of $(-1,7) \text{ and } (4,-3)$ in the ratio 2:3.
	\\
		\solution
	\input{chapters/10/7/2/1/section.tex}
\item Find the coordinates of the points of trisection of the line segment joining $(4,-1) \text{ and } (-2,3)$.
	\\
		\solution
	\input{chapters/10/7/2/2/section.tex}
\item
	\iffalse
\item To conduct Sports Day activities, in your rectangular shaped school                   
ground ABCD, lines have 
drawn with chalk powder at a                 
distance of 1m each. 100 flower pots have been placed at a distance of 1m 
from each other along AD, as shown 
in Fig. 7.12. Niharika runs $ \frac {1}{4} $th the 
distance AD on the 2nd line and 
posts a green flag. Preet runs $ \frac {1}{5} $th 
the distance AD on the eighth line 
and posts a red flag. What is the 
distance between both the flags? If 
Rashmi has to post a blue flag exactly 
halfway between the line segment 
joining the two flags, where should 
she post her flag?
\begin{figure}[h!]
  \centering
  \includegraphics[width=\columnwidth]{sc.png}
  \caption{}
\label{fig:10/7/12Fig1}
\end{figure}               
\fi
      
\item Find the ratio in which the line segment joining the points $(-3,10) \text{ and } (6,-8)$ $\text{ is divided by } (-1,6)$.
	\\
		\solution
	\input{chapters/10/7/2/4/section.tex}
\item Find the ratio in which the line segment joining $A(1,-5) \text{ and } B(-4,5)$ $\text{is divided by the x-axis}$. Also find the coordinates of the point of division.
\item If $(1,2), (4,y), (x,6), (3,5)$ are the vertices of a parallelogram taken in order, find x and y.
	\\
		\solution
	\input{chapters/10/7/2/6/para1.tex}
\item Find the coordinates of a point A, where AB is the diameter of a circle whose centre is $(2,-3) \text{ and }$ B is $(1,4)$.
	\\
		\solution
	\input{chapters/10/7/2/7/section.tex}
\item If A \text{ and } B are $(-2,-2) \text{ and } (2,-4)$, respectively, find the coordinates of P such that AP= $\frac {3}{7}$AB $\text{ and }$ P lies on the line segment AB.
	\\
		\solution
	\input{chapters/10/7/2/8/section.tex}
\item Find the coordinates of the points which divide the line segment joining $A(-2,2) \text{ and } B(2,8)$ into four equal parts.
	\\
		\solution
	\input{chapters/10/7/2/9/section.tex}
\item Find the area of a rhombus if its vertices are $(3,0), (4,5), (-1,4) \text{ and } (-2,-1)$ taken in order. [$\vec{Hint}$ : Area of rhombus =$\frac {1}{2}$(product of its diagonals)]
	\\
		\solution
	\input{chapters/10/7/2/10/cross.tex}
\item Find the position vector of a point R which divides the line joining two points $\vec{P}$
and $\vec{Q}$ whose position vectors are $\hat{i}+2\hat{j}-\hat{k}$ and $-\hat{i}+\hat{j}+\hat{k}$ respectively, in the
ratio 2 : 1
\begin{enumerate}
    \item  internally
    \item  externally
\end{enumerate}
\solution
		\input{chapters/12/10/2/15/section.tex}
\item Find the position vector of the mid point of the vector joining the points $\vec{P}$(2, 3, 4)
and $\vec{Q}$(4, 1, –2).
\\
\solution
		\input{chapters/12/10/2/16/section.tex}
\item Determine the ratio in which the line $2x+y  - 4=0$ divides the line segment joining the points $\vec{A}(2, - 2)$  and  $\vec{B}(3, 7)$.
\\
\solution
	\input{chapters/10/7/4/1/section.tex}
\item Let $\vec{A}(4, 2), \vec{B}(6, 5)$  and $ \vec{C}(1, 4)$ be the vertices of $\triangle ABC$.
\begin{enumerate}
\item The median from $\vec{A}$ meets $BC$ at $\vec{D}$. Find the coordinates of the point $\vec{D}$.
\item Find the coordinates of the point $\vec{P}$ on $AD$ such that $AP : PD = 2 : 1$.
\item Find the coordinates of points $\vec{Q}$ and $\vec{R}$ on medians $BE$ and $CF$ respectively such that $BQ : QE = 2 : 1$  and  $CR : RF = 2 : 1$.
\item What do you observe?
\item If $\vec{A}, \vec{B}$ and $\vec{C}$  are the vertices of $\triangle ABC$, find the coordinates of the centroid of the triangle.
\end{enumerate}
\solution
	\input{chapters/10/7/4/7/section.tex}
\item Find the slope of a line, which passes through the origin and the mid point of the line segment joining the points $\vec{P}$(0,-4) and $\vec{B}$(8,0).
\label{chapters/11/10/1/5}
\input{chapters/11/10/1/5/matrix.tex}
\item Find the position vector of a point R which divides the line joining two points P and Q whose position vectors are $(2\vec{a}+\vec{b})$ and $(\vec{a}-3\vec{b})$
externally in the ratio 1 : 2. Also, show that P is the mid point of the line segment RQ.\\
	\solution
%		\input{chapters/12/10/5/9/section.tex}

\end{enumerate}


\item
	\iffalse
\item To conduct Sports Day activities, in your rectangular shaped school                   
ground ABCD, lines have 
drawn with chalk powder at a                 
distance of 1m each. 100 flower pots have been placed at a distance of 1m 
from each other along AD, as shown 
in Fig. 7.12. Niharika runs $ \frac {1}{4} $th the 
distance AD on the 2nd line and 
posts a green flag. Preet runs $ \frac {1}{5} $th 
the distance AD on the eighth line 
and posts a red flag. What is the 
distance between both the flags? If 
Rashmi has to post a blue flag exactly 
halfway between the line segment 
joining the two flags, where should 
she post her flag?
\begin{figure}[h!]
  \centering
  \includegraphics[width=\columnwidth]{sc.png}
  \caption{}
\label{fig:10/7/12Fig1}
\end{figure}               
\fi
      
\item Find the ratio in which the line segment joining the points $(-3,10) \text{ and } (6,-8)$ $\text{ is divided by } (-1,6)$.
	\\
		\solution
	\iffalse
\documentclass[12pt]{article}
\usepackage{graphicx}
%\documentclass[journal,12pt,twocolumn]{IEEEtran}
\usepackage[none]{hyphenat}
\usepackage{graphicx}
\usepackage{listings}
\usepackage[english]{babel}
\usepackage{graphicx}
\usepackage{caption} 
\usepackage{hyperref}
\usepackage{booktabs}
\def\inputGnumericTable{}
\usepackage{color}                                            %%
    \usepackage{array}                                            %%
    \usepackage{longtable}                                        %%
    \usepackage{calc}                                             %%
    \usepackage{multirow}                                         %%
    \usepackage{hhline}                                           %%
    \usepackage{ifthen}
\usepackage{array}
\usepackage{amsmath}   % for having text in math mode
\usepackage{listings}
\lstset{
language=tex,
frame=single, 
breaklines=true
}
  
%Following 2 lines were added to remove the blank page at the beginning
\usepackage{atbegshi}% http://ctan.org/pkg/atbegshi
\AtBeginDocument{\AtBeginShipoutNext{\AtBeginShipoutDiscard}}
%
%New macro definitions
\newcommand{\mydet}[1]{\ensuremath{\begin{vmatrix}#1\end{vmatrix}}}
\providecommand{\brak}[1]{\ensuremath{\left(#1\right)}}
\providecommand{\norm}[1]{\left\lVert#1\right\rVert}
\newcommand{\solution}{\noindent \textbf{Solution: }}
\newcommand{\myvec}[1]{\ensuremath{\begin{pmatrix}#1\end{pmatrix}}}
\let\vec\mathbf
\begin{document}
\begin{center}
\title{\textbf{Coordinate Geometry}}
\date{\vspace{-5ex}} %Not to print date automatically
\maketitle
\end{center}
\setcounter{page}{1}
\section*{10$^{th}$ Maths - Chapter 7}
This is Problem-4 from Exercise 7.2
\begin{enumerate}
\item Find the ratio in which the line segement joining the points $\myvec{-3 \\ 10}$ and $\myvec{6\\-8}$ is divided by $\myvec{-1\\6}$.\\
\solution \\
\fi
		The input parameters for this problem are available in Table \eqref{tab:10/7/2/4-1}.
\begin{table}[ht!]
\input{chapters/10/7/2/4/tables/table.tex}
\caption{}
\label{tab:10/7/2/4-1} 
\end{table}
Using section formula,
\begin{align}
         \vec{R} &=\frac{\vec{Q}+n\vec{P}}{1+n}\label{eq:chapters/10/7/2/4/1}
\end{align}
Substituting the values of $\vec{P},\vec{Q}$ and $\vec{R}$ in \eqref{eq:chapters/10/7/2/4/1}
\begin{align}
         \myvec{-1\\6} &=\frac{{\myvec{-3\\10}+n\myvec{6\\-8}}}{1+n}\\
 &=\frac{1}{1+n}\brak{{\myvec{-3\\10}+n\myvec{6\\-8}}} \\
 &=\frac{1}{1+n}\myvec{-3+6n\\10-8n} \label{eq:chapters/10/7/2/4/4}
\end{align}
Simplifying \eqref{eq:chapters/10/7/2/4/4} yeilds,
\begin{align}
          -1 &=\frac{-3+6n}{1+n}\\
\implies          n &=\frac{2}{7}
\end{align}
Also,
\begin{align}
          6 &=\frac{10-8n}{1+n}\\
    \implies      n &=\frac{2}{7}
\end{align}
Hence the desired ratio is $\dfrac{2}{7}$.  
\begin{figure}[!h]
 \begin{center}
  \includegraphics[width=\columnwidth]{chapters/10/7/2/4/figs/fig.png}
 \end{center}
\caption{}
\label{fig:10/7/2/4Fig1}
\end{figure}

\item Find the ratio in which the line segment joining $A(1,-5) \text{ and } B(-4,5)$ $\text{is divided by the x-axis}$. Also find the coordinates of the point of division.
\item If $(1,2), (4,y), (x,6), (3,5)$ are the vertices of a parallelogram taken in order, find x and y.
	\\
		\solution
	\iffalse
\documentclass[12pt]{article}
\usepackage{graphicx}
%\documentclass[journal,12pt,twocolumn]{IEEEtran}
\def\inputGnumericTable{}
\usepackage{color}                                            %%
    \usepackage{array}                                            %%
    \usepackage{longtable}                                        %%
    \usepackage{calc}                                             %%
    \usepackage{multirow}                                         %%
    \usepackage{hhline}                                           %%
    \usepackage{ifthen}
\usepackage[none]{hyphenat}
\usepackage{graphicx}
\usepackage{listings}
\usepackage[english]{babel}
\usepackage{graphicx}
\usepackage{caption} 
\usepackage{hyperref}
\usepackage{booktabs}
\usepackage{array}
\usepackage{amsmath}   % for having text in math mode
\usepackage{listings}
\lstset{
  frame=single,
  breaklines=true
}
  
%Following 2 lines were added to remove the blank page at the beginning
\usepackage{atbegshi}% http://ctan.org/pkg/atbegshi
\AtBeginDocument{\AtBeginShipoutNext{\AtBeginShipoutDiscard}}
%


%New macro definitions
\newcommand{\mydet}[1]{\ensuremath{\begin{vmatrix}#1\end{vmatrix}}}
\providecommand{\brak}[1]{\ensuremath{\left(#1\right)}}
\providecommand{\norm}[1]{\left\lVert#1\right\rVert}
\newcommand{\solution}{\noindent \textbf{Solution: }}
\newcommand{\myvec}[1]{\ensuremath{\begin{pmatrix}#1\end{pmatrix}}}
\let\vec\mathbf

\begin{document}

\begin{center}
\title{\textbf{Properties of Parallelegram}}
\date{\vspace{-5ex}} %Not to print date automatically
\maketitle
\end{center}

\setcounter{page}{1}

\section{10$^{th}$ Maths - Chapter 7}

This is Problem-6 from Exercise 7.2

\begin{enumerate}
\item If $\vec{A}(1, 2),\vec{B}(4, x),\vec{C}(y, 6) \text{and } \vec{D}(3, 5)$ are the vertices of a parallelogram taken in order,find x and y.
\end{enumerate}
\fi

The input parameters for this problem are available in
\ref{table:chapters/10/7/2/6/tables/}.	
\begin{table}[!ht]
	\centering
	\input{chapters/10/7/2/6/tables/table.tex}
\caption{}
\label{table:chapters/10/7/2/6/tables/}	
\end{table}
From the given information,
\begin{align}
  \label{eq:chapters/10/7/2/6/tables/det2f}
	\vec{B}-\vec{A} &= \myvec{4 \\y } - \myvec{1 \\2 }  = \myvec{3 \\y-2 }\\
	\vec{C}-\vec{D} &= \myvec{x \\6 } - \myvec{3 \\5 }  = \myvec{x-3 \\1}
\end{align}
Since $ABCD$ is a parallellogram,
\begin{align}
	\myvec{3\\y-2}&=\myvec{x-3\\1}\\
	\implies x&=6 ,y=3
\end{align}
Fig. \ref{fig:chapters/10/7/2/6/Fig3}
provides a verification.
\begin{figure}[h!]
	\begin{center}
  \includegraphics[width=\columnwidth]{chapters/10/7/2/6/figs/para.pdf}
	\end{center}
\caption{}
\label{fig:chapters/10/7/2/6/Fig3}
\end{figure}


\item Find the coordinates of a point A, where AB is the diameter of a circle whose centre is $(2,-3) \text{ and }$ B is $(1,4)$.
	\\
		\solution
	\iffalse
\documentclass[12pt]{article}
\usepackage{graphicx}
\usepackage{amsmath}
\usepackage{mathtools}
\usepackage{gensymb}

\newcommand{\mydet}[1]{\ensuremath{\begin{vmatrix}#1\end{vmatrix}}}
\providecommand{\brak}[1]{\ensuremath{\left(#1\right)}}
\providecommand{\norm}[1]{\left\lVert#1\right\rVert}
\newcommand{\solution}{\noindent \textbf{Solution: }}
\newcommand{\myvec}[1]{\ensuremath{\begin{pmatrix}#1\end{pmatrix}}}
\let\vec\mathbf

\begin{document}
\begin{center}
\section*{CHAPTER 7 - COORDINATE GEOMETRY}

\end{center}
\section*{Excercise 7.2}

Q7.Find the coordinates of point $\vec{A}$, where AB is the diameter of a circle where the center is (2,-3) and $\vec{B}$ is the point (1,4):

\solution
\begin{enumerate}
\item The coordinates $\vec{B}$ and center $\vec{C}$ are given, where:
	\fi
	Let
	\begin{align}
	\vec{B} = \myvec{
		1\\
	    4\\
		},
	\vec{C} = \myvec{
	    2\\
	   -3\\
		}
	\end{align}
	\iffalse
Let us assume the coordinates of $\vec{A}$. Now, $\vec{C}$ is the center which is midpoint of line AB and $\vec{B}$ is one of the coordinate of diameter AB of a circle.
	\fi	
Hence,	
	\begin{align}
	\vec{C} &= \frac{\vec{A+B}}{2} \\
\implies	2\vec{C} &= \vec{A}+\vec{B} \\
		\text{or, }	\vec{A} &= 2\vec{C}-\vec{B} \\
	 &= \myvec{3\\-10\\}	
	\end{align}       
	See Fig. 
\ref{fig:chapters/10/7/2/7Fig}.
\begin{figure}[!h]
\begin{center}	
	\includegraphics[width=\columnwidth]{chapters/10/7/2/7/figs/Vector1.png}
\end{center}
\caption{}
\label{fig:chapters/10/7/2/7Fig}
\end{figure}
	

\item If A \text{ and } B are $(-2,-2) \text{ and } (2,-4)$, respectively, find the coordinates of P such that AP= $\frac {3}{7}$AB $\text{ and }$ P lies on the line segment AB.
	\\
		\solution
	\iffalse
\documentclass[journal,10pt,twocolumn]{article}
\usepackage{graphicx}
\usepackage[none]{hyphenat}
\usepackage{graphicx}
\usepackage{listings}
\usepackage[english]{babel}
\usepackage{graphicx}
\usepackage{caption} 
\usepackage{booktabs}
\usepackage{array}
\usepackage{amssymb} % for \because
\usepackage{amsmath}   % for having text in math mode
\usepackage{extarrows} % for Row operations arrows
\usepackage{listings}
\usepackage[utf8]{inputenc}
\lstset{
  frame=single,
  breaklines=true
}
\usepackage{hyperref}
  
%Following 2 lines were added to remove the blank page at the beginning
\usepackage{atbegshi}% http://ctan.org/pkg/atbegshi
\AtBeginDocument{\AtBeginShipoutNext{\AtBeginShipoutDiscard}}


%New macro definitions
\newcommand{\mydet}[1]{\ensuremath{\begin{vmatrix}#1\end{vmatrix}}}
\providecommand{\brak}[1]{\ensuremath{\left(#1\right)}}
\newcommand{\solution}{\noindent \textbf{Solution: }}
\newcommand{\myvec}[1]{\ensuremath{\begin{pmatrix}#1\end{pmatrix}}}
\providecommand{\norm}[1]{\left\lVert#1\right\rVert}
\providecommand{\abs}[1]{\left\vert#1\right\vert}
\let\vec\mathbf

\begin{document}

\begin{center}
\title{\textbf{VECTORS}}
\date{\vspace{-5ex}} %Not to print date automatically
\maketitle
\end{center}

\section{10$^{th}$ Maths - EXERCISE-7.2}

\begin{enumerate}
\item If A and B are $(– 2, – 2)\text{ and }(2, – 4)$, respectively, find the coordinates of P such that $AP =\frac{3}{7}AB$ and P lies on the line segment AB. 

\section{SOLUTION}
Given points are
\begin{align}
\vec{A}=\myvec{-2\\ -2} ,
\vec{B}=\myvec{2\\ -4}
\end{align}
The equation of the formula is
\fi
Using section formula, 
\begin{align}
\vec{P}&=\frac{\vec{A}+n\vec{B}}{1+n}
\end{align}
where
\begin{align}
	n =\frac{3}{4}
\end{align}
Thus,
\begin{align}
\vec{P}&=\frac{1}{1+\frac{3}{4}}\brak{\myvec{-2\\-2}+\frac{3}{4}\myvec{2\\-4}}\\
&=\myvec{\frac{-2}{7}\\[1pt] \frac{-20}{7}}
\end{align}
See Fig. 
   \ref{fig:chapters/10/7/2/8/vec.png}
\begin{figure}
   \centering 
 \includegraphics[width=\columnwidth]{chapters/10/7/2/8/figs/vec.png}
   \caption{}
   \label{fig:chapters/10/7/2/8/vec.png}
   \end{figure}

\item Find the coordinates of the points which divide the line segment joining $A(-2,2) \text{ and } B(2,8)$ into four equal parts.
	\\
		\solution
	\begin{enumerate}[label=\thesection.\arabic*,ref=\thesection.\theenumi]
\numberwithin{equation}{enumi}
\numberwithin{figure}{enumi}
\numberwithin{table}{enumi}

\item Find the coordinates of the point which divides the join of $(-1,7) \text{ and } (4,-3)$ in the ratio 2:3.
	\\
		\solution
	\input{chapters/10/7/2/1/section.tex}
\item Find the coordinates of the points of trisection of the line segment joining $(4,-1) \text{ and } (-2,3)$.
	\\
		\solution
	\input{chapters/10/7/2/2/section.tex}
\item
	\iffalse
\item To conduct Sports Day activities, in your rectangular shaped school                   
ground ABCD, lines have 
drawn with chalk powder at a                 
distance of 1m each. 100 flower pots have been placed at a distance of 1m 
from each other along AD, as shown 
in Fig. 7.12. Niharika runs $ \frac {1}{4} $th the 
distance AD on the 2nd line and 
posts a green flag. Preet runs $ \frac {1}{5} $th 
the distance AD on the eighth line 
and posts a red flag. What is the 
distance between both the flags? If 
Rashmi has to post a blue flag exactly 
halfway between the line segment 
joining the two flags, where should 
she post her flag?
\begin{figure}[h!]
  \centering
  \includegraphics[width=\columnwidth]{sc.png}
  \caption{}
\label{fig:10/7/12Fig1}
\end{figure}               
\fi
      
\item Find the ratio in which the line segment joining the points $(-3,10) \text{ and } (6,-8)$ $\text{ is divided by } (-1,6)$.
	\\
		\solution
	\input{chapters/10/7/2/4/section.tex}
\item Find the ratio in which the line segment joining $A(1,-5) \text{ and } B(-4,5)$ $\text{is divided by the x-axis}$. Also find the coordinates of the point of division.
\item If $(1,2), (4,y), (x,6), (3,5)$ are the vertices of a parallelogram taken in order, find x and y.
	\\
		\solution
	\input{chapters/10/7/2/6/para1.tex}
\item Find the coordinates of a point A, where AB is the diameter of a circle whose centre is $(2,-3) \text{ and }$ B is $(1,4)$.
	\\
		\solution
	\input{chapters/10/7/2/7/section.tex}
\item If A \text{ and } B are $(-2,-2) \text{ and } (2,-4)$, respectively, find the coordinates of P such that AP= $\frac {3}{7}$AB $\text{ and }$ P lies on the line segment AB.
	\\
		\solution
	\input{chapters/10/7/2/8/section.tex}
\item Find the coordinates of the points which divide the line segment joining $A(-2,2) \text{ and } B(2,8)$ into four equal parts.
	\\
		\solution
	\input{chapters/10/7/2/9/section.tex}
\item Find the area of a rhombus if its vertices are $(3,0), (4,5), (-1,4) \text{ and } (-2,-1)$ taken in order. [$\vec{Hint}$ : Area of rhombus =$\frac {1}{2}$(product of its diagonals)]
	\\
		\solution
	\input{chapters/10/7/2/10/cross.tex}
\item Find the position vector of a point R which divides the line joining two points $\vec{P}$
and $\vec{Q}$ whose position vectors are $\hat{i}+2\hat{j}-\hat{k}$ and $-\hat{i}+\hat{j}+\hat{k}$ respectively, in the
ratio 2 : 1
\begin{enumerate}
    \item  internally
    \item  externally
\end{enumerate}
\solution
		\input{chapters/12/10/2/15/section.tex}
\item Find the position vector of the mid point of the vector joining the points $\vec{P}$(2, 3, 4)
and $\vec{Q}$(4, 1, –2).
\\
\solution
		\input{chapters/12/10/2/16/section.tex}
\item Determine the ratio in which the line $2x+y  - 4=0$ divides the line segment joining the points $\vec{A}(2, - 2)$  and  $\vec{B}(3, 7)$.
\\
\solution
	\input{chapters/10/7/4/1/section.tex}
\item Let $\vec{A}(4, 2), \vec{B}(6, 5)$  and $ \vec{C}(1, 4)$ be the vertices of $\triangle ABC$.
\begin{enumerate}
\item The median from $\vec{A}$ meets $BC$ at $\vec{D}$. Find the coordinates of the point $\vec{D}$.
\item Find the coordinates of the point $\vec{P}$ on $AD$ such that $AP : PD = 2 : 1$.
\item Find the coordinates of points $\vec{Q}$ and $\vec{R}$ on medians $BE$ and $CF$ respectively such that $BQ : QE = 2 : 1$  and  $CR : RF = 2 : 1$.
\item What do you observe?
\item If $\vec{A}, \vec{B}$ and $\vec{C}$  are the vertices of $\triangle ABC$, find the coordinates of the centroid of the triangle.
\end{enumerate}
\solution
	\input{chapters/10/7/4/7/section.tex}
\item Find the slope of a line, which passes through the origin and the mid point of the line segment joining the points $\vec{P}$(0,-4) and $\vec{B}$(8,0).
\label{chapters/11/10/1/5}
\input{chapters/11/10/1/5/matrix.tex}
\item Find the position vector of a point R which divides the line joining two points P and Q whose position vectors are $(2\vec{a}+\vec{b})$ and $(\vec{a}-3\vec{b})$
externally in the ratio 1 : 2. Also, show that P is the mid point of the line segment RQ.\\
	\solution
%		\input{chapters/12/10/5/9/section.tex}

\end{enumerate}


\item Find the area of a rhombus if its vertices are $(3,0), (4,5), (-1,4) \text{ and } (-2,-1)$ taken in order. [$\vec{Hint}$ : Area of rhombus =$\frac {1}{2}$(product of its diagonals)]
	\\
		\solution
	\iffalse
\documentclass[12pt]{article}
\usepackage{graphicx}
%\documentclass[journal,12pt,twocolumn]{IEEEtran}
\usepackage[none]{hyphenat}
\usepackage{graphicx}
\usepackage{listings}
\usepackage[english]{babel}
\usepackage{graphicx}
\usepackage{caption} 
\usepackage{hyperref}
\usepackage{booktabs}
\def\inputGnumericTable{}
\usepackage{color}                                            %%
    \usepackage{array}                                            %%
    \usepackage{longtable}                                        %%
    \usepackage{calc}                                             %%
    \usepackage{multirow}                                         %%
    \usepackage{hhline}                                           %%
    \usepackage{ifthen}
\usepackage{array}
\usepackage{amsmath}   % for having text in math mode
\usepackage{listings}
\lstset{
language=tex,
frame=single, 
breaklines=true
}
  
%Following 2 lines were added to remove the blank page at the beginning
\usepackage{atbegshi}% http://ctan.org/pkg/atbegshi
\AtBeginDocument{\AtBeginShipoutNext{\AtBeginShipoutDiscard}}
%


%New macro definitions
\newcommand{\mydet}[1]{\ensuremath{\begin{vmatrix}#1\end{vmatrix}}}
\providecommand{\brak}[1]{\ensuremath{\left(#1\right)}}
\providecommand{\norm}[1]{\left\lVert#1\right\rVert}
\newcommand{\solution}{\noindent \textbf{Solution: }}
\newcommand{\myvec}[1]{\ensuremath{\begin{pmatrix}#1\end{pmatrix}}}
\let\vec\mathbf

\begin{document}

\begin{center}
\title{\textbf{Coordinate Geometry}}
\date{\vspace{-5ex}} %Not to print date automatically
\maketitle
\end{center}

\setcounter{page}{1}



\begin{enumerate}

\item\textbf{Problem statement :} Find the area of a rhombus of its vertices are $\myvec{3 ,0}$, $\myvec{4 ,5}$, $\myvec{-1 ,4}$ and $\myvec{-2 ,-1}$taken in order

\solution \\
\fi
The input vertices for this problem are given as
	\begin{align}
	\vec{A} = \myvec{
		3\\
		0
		},
	\vec{B} = \myvec{
		4\\
		5
		},
        \vec{C} = \myvec{
		-1\\
		4
		},
        \vec{D} = \myvec{
		-2\\
		-1
		}
	\end{align}
Since		
\begin{align}
 \vec{A-D}= \myvec{3 \\ 0} - \myvec{-2 \\-1}= \myvec{5\\1}
 \\
  \vec{B-A}= \myvec{4 \\ 5} - \myvec{3 \\0}= \myvec{1\\5}
\end{align}
the area of the rhombus is
\begin{align}
                \norm{\myvec{\vec{A-D}}\times \myvec{\vec{B-A}}}=\mydet{5 & 1\\1 & 5} = 24
\end{align}
See Fig. 
\ref{fig:chapters/10/7/2/10/gFig1}.
\begin{figure}[!h]
 \begin{center}
  \includegraphics[width=\columnwidth]{chapters/10/7/2/10/figs/fig.pdf}
 \end{center}
\caption{}
\label{fig:chapters/10/7/2/10/gFig1}
\end{figure}

\item Find the position vector of a point R which divides the line joining two points $\vec{P}$
and $\vec{Q}$ whose position vectors are $\hat{i}+2\hat{j}-\hat{k}$ and $-\hat{i}+\hat{j}+\hat{k}$ respectively, in the
ratio 2 : 1
\begin{enumerate}
    \item  internally
    \item  externally
\end{enumerate}
\solution
		\begin{enumerate}[label=\thesection.\arabic*,ref=\thesection.\theenumi]
\numberwithin{equation}{enumi}
\numberwithin{figure}{enumi}
\numberwithin{table}{enumi}

\item Find the coordinates of the point which divides the join of $(-1,7) \text{ and } (4,-3)$ in the ratio 2:3.
	\\
		\solution
	\input{chapters/10/7/2/1/section.tex}
\item Find the coordinates of the points of trisection of the line segment joining $(4,-1) \text{ and } (-2,3)$.
	\\
		\solution
	\input{chapters/10/7/2/2/section.tex}
\item
	\iffalse
\item To conduct Sports Day activities, in your rectangular shaped school                   
ground ABCD, lines have 
drawn with chalk powder at a                 
distance of 1m each. 100 flower pots have been placed at a distance of 1m 
from each other along AD, as shown 
in Fig. 7.12. Niharika runs $ \frac {1}{4} $th the 
distance AD on the 2nd line and 
posts a green flag. Preet runs $ \frac {1}{5} $th 
the distance AD on the eighth line 
and posts a red flag. What is the 
distance between both the flags? If 
Rashmi has to post a blue flag exactly 
halfway between the line segment 
joining the two flags, where should 
she post her flag?
\begin{figure}[h!]
  \centering
  \includegraphics[width=\columnwidth]{sc.png}
  \caption{}
\label{fig:10/7/12Fig1}
\end{figure}               
\fi
      
\item Find the ratio in which the line segment joining the points $(-3,10) \text{ and } (6,-8)$ $\text{ is divided by } (-1,6)$.
	\\
		\solution
	\input{chapters/10/7/2/4/section.tex}
\item Find the ratio in which the line segment joining $A(1,-5) \text{ and } B(-4,5)$ $\text{is divided by the x-axis}$. Also find the coordinates of the point of division.
\item If $(1,2), (4,y), (x,6), (3,5)$ are the vertices of a parallelogram taken in order, find x and y.
	\\
		\solution
	\input{chapters/10/7/2/6/para1.tex}
\item Find the coordinates of a point A, where AB is the diameter of a circle whose centre is $(2,-3) \text{ and }$ B is $(1,4)$.
	\\
		\solution
	\input{chapters/10/7/2/7/section.tex}
\item If A \text{ and } B are $(-2,-2) \text{ and } (2,-4)$, respectively, find the coordinates of P such that AP= $\frac {3}{7}$AB $\text{ and }$ P lies on the line segment AB.
	\\
		\solution
	\input{chapters/10/7/2/8/section.tex}
\item Find the coordinates of the points which divide the line segment joining $A(-2,2) \text{ and } B(2,8)$ into four equal parts.
	\\
		\solution
	\input{chapters/10/7/2/9/section.tex}
\item Find the area of a rhombus if its vertices are $(3,0), (4,5), (-1,4) \text{ and } (-2,-1)$ taken in order. [$\vec{Hint}$ : Area of rhombus =$\frac {1}{2}$(product of its diagonals)]
	\\
		\solution
	\input{chapters/10/7/2/10/cross.tex}
\item Find the position vector of a point R which divides the line joining two points $\vec{P}$
and $\vec{Q}$ whose position vectors are $\hat{i}+2\hat{j}-\hat{k}$ and $-\hat{i}+\hat{j}+\hat{k}$ respectively, in the
ratio 2 : 1
\begin{enumerate}
    \item  internally
    \item  externally
\end{enumerate}
\solution
		\input{chapters/12/10/2/15/section.tex}
\item Find the position vector of the mid point of the vector joining the points $\vec{P}$(2, 3, 4)
and $\vec{Q}$(4, 1, –2).
\\
\solution
		\input{chapters/12/10/2/16/section.tex}
\item Determine the ratio in which the line $2x+y  - 4=0$ divides the line segment joining the points $\vec{A}(2, - 2)$  and  $\vec{B}(3, 7)$.
\\
\solution
	\input{chapters/10/7/4/1/section.tex}
\item Let $\vec{A}(4, 2), \vec{B}(6, 5)$  and $ \vec{C}(1, 4)$ be the vertices of $\triangle ABC$.
\begin{enumerate}
\item The median from $\vec{A}$ meets $BC$ at $\vec{D}$. Find the coordinates of the point $\vec{D}$.
\item Find the coordinates of the point $\vec{P}$ on $AD$ such that $AP : PD = 2 : 1$.
\item Find the coordinates of points $\vec{Q}$ and $\vec{R}$ on medians $BE$ and $CF$ respectively such that $BQ : QE = 2 : 1$  and  $CR : RF = 2 : 1$.
\item What do you observe?
\item If $\vec{A}, \vec{B}$ and $\vec{C}$  are the vertices of $\triangle ABC$, find the coordinates of the centroid of the triangle.
\end{enumerate}
\solution
	\input{chapters/10/7/4/7/section.tex}
\item Find the slope of a line, which passes through the origin and the mid point of the line segment joining the points $\vec{P}$(0,-4) and $\vec{B}$(8,0).
\label{chapters/11/10/1/5}
\input{chapters/11/10/1/5/matrix.tex}
\item Find the position vector of a point R which divides the line joining two points P and Q whose position vectors are $(2\vec{a}+\vec{b})$ and $(\vec{a}-3\vec{b})$
externally in the ratio 1 : 2. Also, show that P is the mid point of the line segment RQ.\\
	\solution
%		\input{chapters/12/10/5/9/section.tex}

\end{enumerate}


\item Find the position vector of the mid point of the vector joining the points $\vec{P}$(2, 3, 4)
and $\vec{Q}$(4, 1, –2).
\\
\solution
		\begin{enumerate}[label=\thesection.\arabic*,ref=\thesection.\theenumi]
\numberwithin{equation}{enumi}
\numberwithin{figure}{enumi}
\numberwithin{table}{enumi}

\item Find the coordinates of the point which divides the join of $(-1,7) \text{ and } (4,-3)$ in the ratio 2:3.
	\\
		\solution
	\input{chapters/10/7/2/1/section.tex}
\item Find the coordinates of the points of trisection of the line segment joining $(4,-1) \text{ and } (-2,3)$.
	\\
		\solution
	\input{chapters/10/7/2/2/section.tex}
\item
	\iffalse
\item To conduct Sports Day activities, in your rectangular shaped school                   
ground ABCD, lines have 
drawn with chalk powder at a                 
distance of 1m each. 100 flower pots have been placed at a distance of 1m 
from each other along AD, as shown 
in Fig. 7.12. Niharika runs $ \frac {1}{4} $th the 
distance AD on the 2nd line and 
posts a green flag. Preet runs $ \frac {1}{5} $th 
the distance AD on the eighth line 
and posts a red flag. What is the 
distance between both the flags? If 
Rashmi has to post a blue flag exactly 
halfway between the line segment 
joining the two flags, where should 
she post her flag?
\begin{figure}[h!]
  \centering
  \includegraphics[width=\columnwidth]{sc.png}
  \caption{}
\label{fig:10/7/12Fig1}
\end{figure}               
\fi
      
\item Find the ratio in which the line segment joining the points $(-3,10) \text{ and } (6,-8)$ $\text{ is divided by } (-1,6)$.
	\\
		\solution
	\input{chapters/10/7/2/4/section.tex}
\item Find the ratio in which the line segment joining $A(1,-5) \text{ and } B(-4,5)$ $\text{is divided by the x-axis}$. Also find the coordinates of the point of division.
\item If $(1,2), (4,y), (x,6), (3,5)$ are the vertices of a parallelogram taken in order, find x and y.
	\\
		\solution
	\input{chapters/10/7/2/6/para1.tex}
\item Find the coordinates of a point A, where AB is the diameter of a circle whose centre is $(2,-3) \text{ and }$ B is $(1,4)$.
	\\
		\solution
	\input{chapters/10/7/2/7/section.tex}
\item If A \text{ and } B are $(-2,-2) \text{ and } (2,-4)$, respectively, find the coordinates of P such that AP= $\frac {3}{7}$AB $\text{ and }$ P lies on the line segment AB.
	\\
		\solution
	\input{chapters/10/7/2/8/section.tex}
\item Find the coordinates of the points which divide the line segment joining $A(-2,2) \text{ and } B(2,8)$ into four equal parts.
	\\
		\solution
	\input{chapters/10/7/2/9/section.tex}
\item Find the area of a rhombus if its vertices are $(3,0), (4,5), (-1,4) \text{ and } (-2,-1)$ taken in order. [$\vec{Hint}$ : Area of rhombus =$\frac {1}{2}$(product of its diagonals)]
	\\
		\solution
	\input{chapters/10/7/2/10/cross.tex}
\item Find the position vector of a point R which divides the line joining two points $\vec{P}$
and $\vec{Q}$ whose position vectors are $\hat{i}+2\hat{j}-\hat{k}$ and $-\hat{i}+\hat{j}+\hat{k}$ respectively, in the
ratio 2 : 1
\begin{enumerate}
    \item  internally
    \item  externally
\end{enumerate}
\solution
		\input{chapters/12/10/2/15/section.tex}
\item Find the position vector of the mid point of the vector joining the points $\vec{P}$(2, 3, 4)
and $\vec{Q}$(4, 1, –2).
\\
\solution
		\input{chapters/12/10/2/16/section.tex}
\item Determine the ratio in which the line $2x+y  - 4=0$ divides the line segment joining the points $\vec{A}(2, - 2)$  and  $\vec{B}(3, 7)$.
\\
\solution
	\input{chapters/10/7/4/1/section.tex}
\item Let $\vec{A}(4, 2), \vec{B}(6, 5)$  and $ \vec{C}(1, 4)$ be the vertices of $\triangle ABC$.
\begin{enumerate}
\item The median from $\vec{A}$ meets $BC$ at $\vec{D}$. Find the coordinates of the point $\vec{D}$.
\item Find the coordinates of the point $\vec{P}$ on $AD$ such that $AP : PD = 2 : 1$.
\item Find the coordinates of points $\vec{Q}$ and $\vec{R}$ on medians $BE$ and $CF$ respectively such that $BQ : QE = 2 : 1$  and  $CR : RF = 2 : 1$.
\item What do you observe?
\item If $\vec{A}, \vec{B}$ and $\vec{C}$  are the vertices of $\triangle ABC$, find the coordinates of the centroid of the triangle.
\end{enumerate}
\solution
	\input{chapters/10/7/4/7/section.tex}
\item Find the slope of a line, which passes through the origin and the mid point of the line segment joining the points $\vec{P}$(0,-4) and $\vec{B}$(8,0).
\label{chapters/11/10/1/5}
\input{chapters/11/10/1/5/matrix.tex}
\item Find the position vector of a point R which divides the line joining two points P and Q whose position vectors are $(2\vec{a}+\vec{b})$ and $(\vec{a}-3\vec{b})$
externally in the ratio 1 : 2. Also, show that P is the mid point of the line segment RQ.\\
	\solution
%		\input{chapters/12/10/5/9/section.tex}

\end{enumerate}


\item Determine the ratio in which the line $2x+y  - 4=0$ divides the line segment joining the points $\vec{A}(2, - 2)$  and  $\vec{B}(3, 7)$.
\\
\solution
	\iffalse
\documentclass[journal,12pt,twocolumn]{IEEEtran}
\usepackage{graphicx}
\graphicspath{{./chapters/10/7/4/1/figs/}}{}
\usepackage{amsmath,amssymb,amsfonts,amsthm}
\newcommand{\myvec}[1]{\ensuremath{\begin{pmatrix}#1\end{pmatrix}}}
\providecommand{\norm}[1]{\lVert#1\rVert}
\usepackage{listings}
\usepackage{watermark}
\usepackage{titlesec}
\usepackage{caption}
\let\vec\mathbf
\lstset{
frame=single, 
breaklines=true,
columns=fullflexible
}
\thiswatermark{\centering \put(0,-105.0){\includegraphics[scale=0.15]{/sdcard/IITH/vector/vectpr-4/chapters/10/7/4/1/figs/logo.png}} }
\title{\mytitle}
\title{
Assignment - Vector-4
}
\author{Surajit Sarkar}
\begin{document}
\maketitle
%\tableofcontents
\bigskip
\section{\textbf{Problem}}
Determine the ratio in which the line 2x+y–4=0 divides the line segment joining the points A(2,–2) and B(3,7).
\section{\textbf{Solution}}
\begin{table}[h]
    \centering
    \begin{tabular}{|c|c|}
       \hline
       \textbf{Symbol}&\textbf{Value}  \\
       \hline
	    $\vec{A}$ & $\myvec{2\\-2}$\\
        \hline
	    $\vec{B}$ & $\myvec{3\\7}$\\
        \hline
	    c&$4$\\
        \hline
       $\vec{n}$ & $\myvec{2\\1}$\\
       \hline
    \end{tabular}
    \caption{Parameters}
    \label{tab:my_label}
\end{table}
Given equation
\fi
The given equation can be expressed as
\begin{align}
    \myvec{2&1}\vec{x}&=4\\
\end{align}
Using section formula, the point of division 
\begin{align}
    \vec{P} = \frac{k\vec{B+A}}{k+1}
\end{align}
which upon substitution in the equation of a line yields
\begin{align}
    \implies\vec{n}^{\top}\myvec{\frac{k\vec{B+A}}{k+1}}&=c\\
    \implies k&=\frac{c-\vec{n}^{\top}\vec{A}}{\vec{n}^{\top}\vec{B}-c}\\
\end{align}
upon simplification.  Substituting numerical values, 
\begin{align}
    k=\frac{2}{9}
\end{align}
See Fig. 
\ref{fig:chapters/10/7/4/1vec}.
\begin{figure}[!h]
\centering
\includegraphics[width=\columnwidth]{chapters/10/7/4/1/figs/vec.pdf}
\caption{}
\label{fig:chapters/10/7/4/1vec}
\end{figure}


\item Let $\vec{A}(4, 2), \vec{B}(6, 5)$  and $ \vec{C}(1, 4)$ be the vertices of $\triangle ABC$.
\begin{enumerate}
\item The median from $\vec{A}$ meets $BC$ at $\vec{D}$. Find the coordinates of the point $\vec{D}$.
\item Find the coordinates of the point $\vec{P}$ on $AD$ such that $AP : PD = 2 : 1$.
\item Find the coordinates of points $\vec{Q}$ and $\vec{R}$ on medians $BE$ and $CF$ respectively such that $BQ : QE = 2 : 1$  and  $CR : RF = 2 : 1$.
\item What do you observe?
\item If $\vec{A}, \vec{B}$ and $\vec{C}$  are the vertices of $\triangle ABC$, find the coordinates of the centroid of the triangle.
\end{enumerate}
\solution
	\iffalse
\documentclass[12pt]{article}
\usepackage{graphicx}
\usepackage[none]{hyphenat}
\usepackage{graphicx}
\usepackage{listings}
\usepackage[english]{babel}
\usepackage{graphicx}
\usepackage{caption} 
\usepackage{booktabs}
\usepackage{array}
\usepackage{amssymb} % for \because
\usepackage{amsmath}   % for having text in math mode
\usepackage{extarrows} % for Row operations arrows
\usepackage{listings}
\usepackage[utf8]{inputenc}
\lstset{
  frame=single,
  breaklines=true
}
\usepackage{hyperref}
  
%Following 2 lines were added to remove the blank page at the beginning
\usepackage{atbegshi}% http://ctan.org/pkg/atbegshi
\AtBeginDocument{\AtBeginShipoutNext{\AtBeginShipoutDiscard}}


%New macro definitions
\newcommand{\mydet}[1]{\ensuremath{\begin{vmatrix}#1\end{vmatrix}}}
\providecommand{\brak}[1]{\ensuremath{\left(#1\right)}}
\newcommand{\solution}{\noindent \textbf{Solution: }}
\newcommand{\myvec}[1]{\ensuremath{\begin{pmatrix}#1\end{pmatrix}}}
\providecommand{\norm}[1]{\left\lVert#1\right\rVert}
\providecommand{\abs}[1]{\left\vert#1\right\vert}
\let\vec\mathbf

\begin{document}

\begin{center}
\title{\textbf{VECTORS}}
\date{\vspace{-5ex}} %Not to print date automatically
\maketitle
\end{center}

\section{10$^{th}$ Maths - EXERCISE-7.4}

Let A(4, 2), B(6, 5) and C(1, 4) be the vertices of $\triangle ABC$
\begin{enumerate}
\item The median from A meets BC at D. Find the coordinates of the point D.
\item Find the coordinates of the point P on AD such that $AP : PD = 2 : 1$
\item Find the coordinates of points Q and R on medians BE and CF respectively such
that $BQ : QE = 2 : 1 \text{and} CR : RF = 2 : 1.$
\item What do yo observe?
\item If $A(x_1, y_1), B(x_2, y_2) \text{and} C(x_3, y_3)$ are the vertices of $\triangle ABC$, find the coordinates of the centroid of the triangle.
\end{enumerate}

Given points are
\begin{align}
\vec{A}=\myvec{4\\ 2} ,
\vec{B}=\myvec{6\\ 5} ,
\vec{C}=\myvec{1\\ 4}
\end{align}
\fi

\begin{enumerate}
\item 
\begin{align}
\vec{D}&=\frac{\vec{B}+\vec{C}}{2}\\
&=\myvec{\frac{7}{2}\\[2pt] \frac{9}{2}}\\
\vec{E}&=\frac{\vec{A}+\vec{C}}{2}\\
&=\myvec{\frac{5}{2}\\ 3}\\
\vec{F}&=\frac{\vec{A}+\vec{B}}{2}\\
&=\myvec{5\\ \frac{7}{2}}
\end{align}

\item 
	For
$n=2$,
\begin{align}
\vec{P}&=\frac{1}{1+n}\brak{\myvec{\vec{A}+n\vec{D}}}\\
&=\frac{1}{3}\myvec{11\\11}
\end{align}

\item 
\begin{align}
\vec{Q}&=\frac{1}{1+n}\brak{\myvec{\vec{B}+n\vec{E}}}\\
&=\frac{1}{3}\myvec{11\\11}\\
\vec{R}&=\frac{1}{1+n}\brak{\myvec{\vec{C}+n\vec{F}}}\\
&=\frac{1}{3}\myvec{11\\11}\\
\end{align}

\item 
 $\vec{P},\vec{Q},\vec{R}$ are the same point.
   
\item 
\begin{align}
\vec{G}&=\frac{\vec{D}+\vec{E}+\vec{F}}{3}\\
&=\frac{1}{3}\myvec{11\\11}\\
\end{align} 
\end{enumerate}
See Fig.  
  \ref{fig:chapters/10/7/4/7/Figure}.
\begin{figure}[h!]
\centering
\includegraphics[width=\columnwidth]{chapters/10/7/4/7/figs/dj.pdf}
\caption{}
  \label{fig:chapters/10/7/4/7/Figure}
\end{figure}

\item Find the slope of a line, which passes through the origin and the mid point of the line segment joining the points $\vec{P}$(0,-4) and $\vec{B}$(8,0).
\label{chapters/11/10/1/5}
\iffalse
\documentclass[journal,12pt,twocolumn]{IEEEtran}
\usepackage{graphicx}
\graphicspath{{./figs/}}{}
\usepackage{amsmath,amssymb,amsfonts,amsthm}
\newcommand{\myvec}[1]{\ensuremath{\begin{pmatrix}#1\end{pmatrix}}}

\let\vec\mathbf

\title{
Matrix-Lines
}
\author{Jyothsna Paluchuri-FWC22059\\}
\begin{document}
\maketitle
\tableofcontents
\bigskip
\section{Problem Statement}
\fi
	\begin{figure}[!ht]
		\centering
 \includegraphics[width=\columnwidth]{chapters/11/10/1/5/figs/line.png}
		\caption{}
		\label{fig:11/10/1/5}
  	\end{figure}
	\\
	\solution
\iffalse
\section{Construction}
\begin{figure}[h]
    \centering
\includegraphics[width=\columnwidth]{line.png}
    \caption{Equation of the slope}
    \label{fig:my_label}
\end{figure}
\vspace{2cm}
\begin{table}[h]
    \centering
    \begin{tabular}{|c|c|c|c|}
       \hline
       \textbf{Symbol}&\textbf{Value}&\textbf{Description}  \\
       \hline
	    $\vec{P}$ & $\myvec{
		    0\\
		    -4}$
	    & Point on Y-axis\\
        \hline
	    $\vec{B}$ & $\myvec{8\\0}$
 & Point on X-axis\\
        \hline
	    $\vec{0}$ & $\myvec{0\\0}$
 & Origin\\
        \hline
    \end{tabular}
    \caption{Parameters}
    \label{tab:my_label}
\end{table}


\section{Solution}
Given that resultant line passes through origin and mid point of the line segment joining point P(0,-4) and B(8,0) \\
\\
\\
given ${\vec{P}}$=$\myvec{
  0\\
  -4}$
 , ${\vec{B}}$=$\myvec{
  8\\
  0}$
  
 \fi 
The mid point of $PB$ is
\begin{align}
\vec{M} &=\frac{1}{2}(\vec{P}+\vec{B})
	= \myvec{4 \\ -2}  
\end{align}
The direction vector of line joining $\vec{O}, \vec{M}$ is 
\begin{align}
\vec{m}&=\vec{O}-\vec{M}
 = -\vec{M}
\end{align}
which can be expressed as
\begin{align}
	\myvec{1 \\ -\frac{1}{2}}
\end{align}
Thus the slope is
\begin{align}
	m = -\frac{1}{2}
\end{align}
\iffalse
\textbf{The direction vector of a line expressed as}
\begin{align}
\implies\vec{m} &= \begin{pmatrix}1 \\ m \\ \end{pmatrix}
\end{align}

\textbf{By solving equation (5) and (6),we get the slope of $\vec{O}$ $\vec{M}$ line}
\begin{align}
        \boxed{m=-0.5}
 \end{align}

\section{Software}
Download the following code using,
\begin{table}[h]
    \centering
    \begin{tabular}{|c|}
    \hline \\
   https://github.com/jyothsna777/jyothsna-fwc.git  \\
         \\
\hline
    \end{tabular}
\end{table}
\\
and execute the code by using command
\begin{center}
\textbf{Python3 lines.py}\\
\end{center}

\section{Conclusion}
Hence the slope of line $\vec{O}$ $\vec{M}$ lineis $\vec{m}$=-0.5

\end{document}
\fi

\item Find the position vector of a point R which divides the line joining two points P and Q whose position vectors are $(2\vec{a}+\vec{b})$ and $(\vec{a}-3\vec{b})$
externally in the ratio 1 : 2. Also, show that P is the mid point of the line segment RQ.\\
	\solution
%		\begin{enumerate}[label=\thesection.\arabic*,ref=\thesection.\theenumi]
\numberwithin{equation}{enumi}
\numberwithin{figure}{enumi}
\numberwithin{table}{enumi}

\item Find the coordinates of the point which divides the join of $(-1,7) \text{ and } (4,-3)$ in the ratio 2:3.
	\\
		\solution
	\input{chapters/10/7/2/1/section.tex}
\item Find the coordinates of the points of trisection of the line segment joining $(4,-1) \text{ and } (-2,3)$.
	\\
		\solution
	\input{chapters/10/7/2/2/section.tex}
\item
	\iffalse
\item To conduct Sports Day activities, in your rectangular shaped school                   
ground ABCD, lines have 
drawn with chalk powder at a                 
distance of 1m each. 100 flower pots have been placed at a distance of 1m 
from each other along AD, as shown 
in Fig. 7.12. Niharika runs $ \frac {1}{4} $th the 
distance AD on the 2nd line and 
posts a green flag. Preet runs $ \frac {1}{5} $th 
the distance AD on the eighth line 
and posts a red flag. What is the 
distance between both the flags? If 
Rashmi has to post a blue flag exactly 
halfway between the line segment 
joining the two flags, where should 
she post her flag?
\begin{figure}[h!]
  \centering
  \includegraphics[width=\columnwidth]{sc.png}
  \caption{}
\label{fig:10/7/12Fig1}
\end{figure}               
\fi
      
\item Find the ratio in which the line segment joining the points $(-3,10) \text{ and } (6,-8)$ $\text{ is divided by } (-1,6)$.
	\\
		\solution
	\input{chapters/10/7/2/4/section.tex}
\item Find the ratio in which the line segment joining $A(1,-5) \text{ and } B(-4,5)$ $\text{is divided by the x-axis}$. Also find the coordinates of the point of division.
\item If $(1,2), (4,y), (x,6), (3,5)$ are the vertices of a parallelogram taken in order, find x and y.
	\\
		\solution
	\input{chapters/10/7/2/6/para1.tex}
\item Find the coordinates of a point A, where AB is the diameter of a circle whose centre is $(2,-3) \text{ and }$ B is $(1,4)$.
	\\
		\solution
	\input{chapters/10/7/2/7/section.tex}
\item If A \text{ and } B are $(-2,-2) \text{ and } (2,-4)$, respectively, find the coordinates of P such that AP= $\frac {3}{7}$AB $\text{ and }$ P lies on the line segment AB.
	\\
		\solution
	\input{chapters/10/7/2/8/section.tex}
\item Find the coordinates of the points which divide the line segment joining $A(-2,2) \text{ and } B(2,8)$ into four equal parts.
	\\
		\solution
	\input{chapters/10/7/2/9/section.tex}
\item Find the area of a rhombus if its vertices are $(3,0), (4,5), (-1,4) \text{ and } (-2,-1)$ taken in order. [$\vec{Hint}$ : Area of rhombus =$\frac {1}{2}$(product of its diagonals)]
	\\
		\solution
	\input{chapters/10/7/2/10/cross.tex}
\item Find the position vector of a point R which divides the line joining two points $\vec{P}$
and $\vec{Q}$ whose position vectors are $\hat{i}+2\hat{j}-\hat{k}$ and $-\hat{i}+\hat{j}+\hat{k}$ respectively, in the
ratio 2 : 1
\begin{enumerate}
    \item  internally
    \item  externally
\end{enumerate}
\solution
		\input{chapters/12/10/2/15/section.tex}
\item Find the position vector of the mid point of the vector joining the points $\vec{P}$(2, 3, 4)
and $\vec{Q}$(4, 1, –2).
\\
\solution
		\input{chapters/12/10/2/16/section.tex}
\item Determine the ratio in which the line $2x+y  - 4=0$ divides the line segment joining the points $\vec{A}(2, - 2)$  and  $\vec{B}(3, 7)$.
\\
\solution
	\input{chapters/10/7/4/1/section.tex}
\item Let $\vec{A}(4, 2), \vec{B}(6, 5)$  and $ \vec{C}(1, 4)$ be the vertices of $\triangle ABC$.
\begin{enumerate}
\item The median from $\vec{A}$ meets $BC$ at $\vec{D}$. Find the coordinates of the point $\vec{D}$.
\item Find the coordinates of the point $\vec{P}$ on $AD$ such that $AP : PD = 2 : 1$.
\item Find the coordinates of points $\vec{Q}$ and $\vec{R}$ on medians $BE$ and $CF$ respectively such that $BQ : QE = 2 : 1$  and  $CR : RF = 2 : 1$.
\item What do you observe?
\item If $\vec{A}, \vec{B}$ and $\vec{C}$  are the vertices of $\triangle ABC$, find the coordinates of the centroid of the triangle.
\end{enumerate}
\solution
	\input{chapters/10/7/4/7/section.tex}
\item Find the slope of a line, which passes through the origin and the mid point of the line segment joining the points $\vec{P}$(0,-4) and $\vec{B}$(8,0).
\label{chapters/11/10/1/5}
\input{chapters/11/10/1/5/matrix.tex}
\item Find the position vector of a point R which divides the line joining two points P and Q whose position vectors are $(2\vec{a}+\vec{b})$ and $(\vec{a}-3\vec{b})$
externally in the ratio 1 : 2. Also, show that P is the mid point of the line segment RQ.\\
	\solution
%		\input{chapters/12/10/5/9/section.tex}

\end{enumerate}



\end{enumerate}


    \item Draw a quadrilateral in the Cartesian plane, whose vertices are 
    \begin{align}
        \vec{A} = \myvec{-4\\5} \quad \vec{B} = \myvec{0\\7} \\
        \vec{C} = \myvec{5\\-5} \quad \vec{D} = \myvec{-4\\-2}
    \end{align}
    Also, find its area.
\label{chapters/11/10/1/1}
   \\ 
    \solution 
\iffalse
\documentclass[journal,12pt,twocolumn]{IEEEtran}
\usepackage{setspace}
\usepackage{gensymb}
\usepackage{xcolor}
\usepackage{caption}
\singlespacing
\usepackage{siunitx}
\usepackage[cmex10]{amsmath}
\usepackage{mathtools}
\usepackage{hyperref}
\usepackage{amsthm}
\usepackage{mathrsfs}
\usepackage{txfonts}
\usepackage{stfloats}
\usepackage{cite}
\usepackage{cases}
\usepackage{subfig}
\usepackage{longtable}
\usepackage{multirow}
\usepackage{enumitem}
\usepackage{mathtools}
\usepackage{listings}
\usepackage{tikz}
\usetikzlibrary{shapes,arrows,positioning}
\usepackage{circuitikz}
\let\vec\mathbf
\DeclareMathOperator*{\Res}{Res}
\renewcommand\thesection{\arabic{section}}
\renewcommand\thesubsection{\thesection.\arabic{subsection}}
\renewcommand\thesubsubsection{\thesubsection.\arabic{subsubsection}}

\renewcommand\thesectiondis{\arabic{section}}
\renewcommand\thesubsectiondis{\thesectiondis.\arabic{subsection}}
\renewcommand\thesubsubsectiondis{\thesubsectiondis.\arabic{subsubsection}}
\hyphenation{op-tical net-works semi-conduc-tor}

\lstset{
language=Python,
frame=single, 
breaklines=true,
columns=fullflexible
}
\begin{document}
\theoremstyle{definition}
\newtheorem{theorem}{Theorem}[section]
\newtheorem{problem}{Problem}
\newtheorem{proposition}{Proposition}[section]
\newtheorem{lemma}{Lemma}[section]
\newtheorem{corollary}[theorem]{Corollary}
\newtheorem{example}{Example}[section]
\newtheorem{definition}{Definition}[section]
\newcommand{\BEQA}{\begin{eqnarray}}
\newcommand{\EEQA}{\end{eqnarray}}
\newcommand{\define}{\stackrel{\triangle}{=}}
\newcommand{\myvec}[1]{\ensuremath{\begin{pmatrix}#1\end{pmatrix}}}
\newcommand{\mydet}[1]{\ensuremath{\begin{vmatrix}#1\end{vmatrix}}}

\bibliographystyle{IEEEtran}
\providecommand{\nCr}[2]{\,^{#1}C_{#2}} % nCr
\providecommand{\nPr}[2]{\,^{#1}P_{#2}} % nPr
\providecommand{\mbf}{\mathbf}
\providecommand{\pr}[1]{\ensuremath{\Pr\left(#1\right)}}
\providecommand{\qfunc}[1]{\ensuremath{Q\left(#1\right)}}
\providecommand{\sbrak}[1]{\ensuremath{{}\left[#1\right]}}
\providecommand{\lsbrak}[1]{\ensuremath{{}\left[#1\right.}}
\providecommand{\rsbrak}[1]{\ensuremath{{}\left.#1\right]}}
\providecommand{\brak}[1]{\ensuremath{\left(#1\right)}}
\providecommand{\lbrak}[1]{\ensuremath{\left(#1\right.}}
\providecommand{\rbrak}[1]{\ensuremath{\left.#1\right)}}
\providecommand{\cbrak}[1]{\ensuremath{\left\{#1\right\}}}
\providecommand{\lcbrak}[1]{\ensuremath{\left\{#1\right.}}
\providecommand{\rcbrak}[1]{\ensuremath{\left.#1\right\}}}
\theoremstyle{remark}
\newtheorem{rem}{Remark}
\newcommand{\sgn}{\mathop{\mathrm{sgn}}}
\newcommand{\rect}{\mathop{\mathrm{rect}}}
\newcommand{\sinc}{\mathop{\mathrm{sinc}}}
\providecommand{\abs}[1]{\left\vert#1\right\vert}
\providecommand{\res}[1]{\Res\displaylimits_{#1}} 
\providecommand{\norm}[1]{\left\Vert#1\right\Vert}
\providecommand{\mtx}[1]{\mathbf{#1}}
\providecommand{\mean}[1]{E\left[ #1 \right]}
\providecommand{\fourier}{\overset{\mathcal{F}}{ \rightleftharpoons}}
\providecommand{\ztrans}{\overset{\mathcal{Z}}{ \rightleftharpoons}}
\providecommand{\system}[1]{\overset{\mathcal{#1}}{ \longleftrightarrow}}
\newcommand{\solution}{\noindent \textbf{Solution: }}
\providecommand{\dec}[2]{\ensuremath{\overset{#1}{\underset{#2}{\gtrless}}}}
\let\StandardTheFigure\thefigure
\def\putbox#1#2#3{\makebox[0in][l]{\makebox[#1][l]{}\raisebox{\baselineskip}[0in][0in]{\raisebox{#2}[0in][0in]{#3}}}}
     \def\rightbox#1{\makebox[0in][r]{#1}}
     \def\centbox#1{\makebox[0in]{#1}}
     \def\topbox#1{\raisebox{-\baselineskip}[0in][0in]{#1}}
     \def\midbox#1{\raisebox{-0.5\baselineskip}[0in][0in]{#1}}

\vspace{3cm}
\title{Straight Lines Assignment}
\author{Gautam Singh}
\maketitle
\bigskip

\begin{abstract}
    This document contains the solution to Question 1 of Exercise 1 in Chapter
    10 of the class 11 NCERT textbook.
\end{abstract}

\begin{enumerate}
\fi
		The points are plotted in Fig. \ref{fig:11/10/1/1quad}. The plot is 
    generated using the Python code \texttt{codes/quad.py}.

    The area vector (denoted by $\vec{R_X}$ for region $X$) of the quadrilateral 
    is perpendicular to the plane of the quadrilateral and its orientation is 
    assumed to be in the positive $z$-direction here.
    \begin{align}
        &\vec{R_{ABCD}} = \vec{R_{ABC}} + \vec{R_{ACD}} \\
        &= \frac{1}{2}\brak{\brak{\vec{B}-\vec{A}}\times\brak{\vec{C}-\vec{A}} + 
        \brak{\vec{C}-\vec{A}}\times\brak{\vec{D}-\vec{A}}} \\
        &= \frac{1}{2}\brak{\brak{\vec{C}-\vec{A}}\times
        \brak{\vec{D}-\vec{A}+\vec{A}-\vec{B}}} \\
        &= \frac{1}{2}\brak{\brak{\vec{C}-\vec{A}}\times\brak{\vec{D}-\vec{B}}} \\
        \label{eq:11/10/1/1area-diag} 
    \end{align}
    Thus the area of quadrilateral ABCD is
    \begin{align}
        \textrm{ar}\brak{ABCD} &= \norm{\vec{R_{ABCD}}} \\
                               &= \frac{1}{2}\norm{\brak{\vec{C}-\vec{A}}\times\brak{\vec{D}-\vec{B}}} \\ 
                               &= \frac{1}{2}\mydet{9&-4\\-10&-9} \\
                               &= 60.5\ \textrm{sq. units.}
        \label{eq:11/10/1/1ans}
    \end{align}
    \begin{figure}[!htb]
        \centering
        \includegraphics[width=\columnwidth]{chapters/11/10/1/1/figs/quad.png}
        \caption{Plot of quadrilateral $ABCD$}
        \label{fig:11/10/1/1quad}
    \end{figure}

\item Find the area of region bounded by the triangle whose
	vertices are $(1, 0), (2, 2) \text{ and } (3, 1)$. 
\item Find the area of region bounded by the triangle whose vertices
	are $(– 1, 0), (1, 3) \text{ and } (3, 2)$. 
\item Find the area of the $\triangle ABC$, coordinates of whose vertices are $\vec{A}(2, 0), \vec{B}(4, 5), \text{ and } \vec{C}(6, 3)$.


\item 
\documentclass[journal,12pt,twocolumn]{IEEEtran}
\usepackage{setspace}
\usepackage{gensymb}
\usepackage{xcolor}
\usepackage{caption}
\singlespacing
\usepackage{siunitx}
\usepackage[cmex10]{amsmath}
\usepackage{mathtools}
\usepackage{hyperref}
\usepackage{amsthm}
\usepackage{mathrsfs}
\usepackage{txfonts}
\usepackage{stfloats}
\usepackage{cite}
\usepackage{cases}
\usepackage{subfig}
\usepackage{longtable}
\usepackage{multirow}
\usepackage{enumitem}
\usepackage{bm}
\usepackage{mathtools}
\usepackage{listings}
\usepackage{tikz}
\usetikzlibrary{shapes,arrows,positioning}
\usepackage{circuitikz}
\renewcommand{\vec}[1]{\boldsymbol{\mathbf{#1}}}
\DeclareMathOperator*{\Res}{Res}
\renewcommand\thesection{\arabic{section}}
\renewcommand\thesubsection{\thesection.\arabic{subsection}}
\renewcommand\thesubsubsection{\thesubsection.\arabic{subsubsection}}

\renewcommand\thesectiondis{\arabic{section}}
\renewcommand\thesubsectiondis{\thesectiondis.\arabic{subsection}}
\renewcommand\thesubsubsectiondis{\thesubsectiondis.\arabic{subsubsection}}
\hyphenation{op-tical net-works semi-conduc-tor}

\lstset{
language=Python,
frame=single, 
breaklines=true,
columns=fullflexible
}
\begin{document}
\theoremstyle{definition}
\newtheorem{theorem}{Theorem}[section]
\newtheorem{problem}{Problem}
\newtheorem{proposition}{Proposition}[section]
\newtheorem{lemma}{Lemma}[section]
\newtheorem{corollary}[theorem]{Corollary}
\newtheorem{example}{Example}[section]
\newtheorem{definition}{Definition}[section]
\newcommand{\BEQA}{\begin{eqnarray}}
        \newcommand{\EEQA}{\end{eqnarray}}
\newcommand{\define}{\stackrel{\triangle}{=}}
\newcommand{\myvec}[1]{\ensuremath{\begin{pmatrix}#1\end{pmatrix}}}
\newcommand{\mydet}[1]{\ensuremath{\begin{vmatrix}#1\end{vmatrix}}}
\bibliographystyle{IEEEtran}
\providecommand{\nCr}[2]{\,^{#1}C_{#2}} % nCr
\providecommand{\nPr}[2]{\,^{#1}P_{#2}} % nPr
\providecommand{\mbf}{\mathbf}
\providecommand{\pr}[1]{\ensuremath{\Pr\left(#1\right)}}
\providecommand{\qfunc}[1]{\ensuremath{Q\left(#1\right)}}
\providecommand{\sbrak}[1]{\ensuremath{{}\left[#1\right]}}
\providecommand{\lsbrak}[1]{\ensuremath{{}\left[#1\right.}}
\providecommand{\rsbrak}[1]{\ensuremath{{}\left.#1\right]}}
\providecommand{\brak}[1]{\ensuremath{\left(#1\right)}}
\providecommand{\lbrak}[1]{\ensuremath{\left(#1\right.}}
\providecommand{\rbrak}[1]{\ensuremath{\left.#1\right)}}
\providecommand{\cbrak}[1]{\ensuremath{\left\{#1\right\}}}
\providecommand{\lcbrak}[1]{\ensuremath{\left\{#1\right.}}
\providecommand{\rcbrak}[1]{\ensuremath{\left.#1\right\}}}
\theoremstyle{remark}
\newtheorem{rem}{Remark}
\newcommand{\sgn}{\mathop{\mathrm{sgn}}}
\newcommand{\rect}{\mathop{\mathrm{rect}}}
\newcommand{\sinc}{\mathop{\mathrm{sinc}}}
\providecommand{\abs}[1]{\left\vert#1\right\vert}
\providecommand{\res}[1]{\Res\displaylimits_{#1}}
\providecommand{\norm}[1]{\lVert#1\rVert}
\providecommand{\mtx}[1]{\mathbf{#1}}
\providecommand{\mean}[1]{E\left[ #1 \right]}
\providecommand{\fourier}{\overset{\mathcal{F}}{ \rightleftharpoons}}
\providecommand{\ztrans}{\overset{\mathcal{Z}}{ \rightleftharpoons}}
\providecommand{\system}[1]{\overset{\mathcal{#1}}{ \longleftrightarrow}}
\newcommand{\solution}{\noindent \textbf{Solution: }}
\providecommand{\dec}[2]{\ensuremath{\overset{#1}{\underset{#2}{\gtrless}}}}
\let\StandardTheFigure\thefigure
\def\putbox#1#2#3{\makebox[0in][l]{\makebox[#1][l]{}\raisebox{\baselineskip}[0in][0in]{\raisebox{#2}[0in][0in]{#3}}}}
\def\rightbox#1{\makebox[0in][r]{#1}}
\def\centbox#1{\makebox[0in]{#1}}
\def\topbox#1{\raisebox{-\baselineskip}[0in][0in]{#1}}
\def\midbox#1{\raisebox{-0.5\baselineskip}[0in][0in]{#1}}

\vspace{3cm}
\title{12.11.3.9}
\author{Lokesh Surana}
\maketitle
\section*{Class 12, Chapter 11, Exercise 4.19}

Q. Find the vector equation of the line passing through $\myvec{1\\2\\3}$ and parallel to the planes $\myvec{1\\-1\\2}^{\top}\vec{r} = 5$ and $\myvec{3\\1\\1}^{\top}\vec{r} = 6$.  

\solution
The line equations are given as
\begin{align}
    \label{eq:1} \vec{r} = \vec{A} + \lambda\vec{m}
\end{align}
where $\vec{m}$ is the direction vector of the line and $\vec{A}$ is any point on the line. 

The planes are given as
\begin{align}
    \label{eq:2} {P}_1: \myvec{1&-1&2}\vec{r} = 5 \\
    \label{eq:3} \implies \vec{n}_1 = \myvec{1\\-1\\2}\\
    \label{eq:4} {P}_2: \myvec{3&1&1}\vec{r} = 6 \\
    \label{eq:5} \implies \vec{n}_2 = \myvec{3\\1\\1}
\end{align}
The expected line is parallel to both the planes, then the direction vector of the line must be perpendicular to both the normal vectors. This means that

\begin{align}
    \label{eq:6} \vec{n}_1^{\top}\vec{m} = 0 \\
    \label{eq:7} \vec{n}_2^{\top}\vec{m} = 0 \\
    \label{eq:8} \implies \myvec{1&-1&2 \\ 3&1&1}\vec{m} = 0
\end{align}

Let's reduce the matrix from equation \eqref{eq:8} to row-echelon form:
\begin{align}
    \label{eq:9} \myvec{1&-1&2 \\ 3&1&1} &\xleftrightarrow[]{R_2\rightarrow -\frac{3}{4}{R_1} + \frac{1}{4}{R_2}} \myvec{1&-1&2 \\ 0&1&-\frac{5}{4}}\\
    \label{eq:10} \myvec{1&-1&2 \\ 0&1&-\frac{5}{4}} &\xleftrightarrow[]{R_1\rightarrow {R_1} + {R_2}} \myvec{1&0&\frac{3}{4} \\ 0&1&-\frac{5}{4}}
\end{align}

Using \eqref{eq:8}, \eqref{eq:9} and \eqref{eq:10}, we get:
\begin{align}
    \implies \myvec{1&0&\frac{3}{4} \\ 0&1&-\frac{5}{4}}\vec{m} &= 0 \\
    \implies \myvec{{m}_1\\{m}_2\\{m}_3} &= \myvec{-\frac{3}{4}{m}_3\\\frac{5}{4}{m}_3\\{m}_3} \\
    \implies \myvec{{m}_1\\{m}_2\\{m}_3} &= {m}_3\myvec{-\frac{3}{4}\\\frac{5}{4}\\1} \\
    \implies \vec{m} = \myvec{-3\\5\\4}
\end{align}

It is given that line passes through point $\myvec{1\\2\\3}$, so the final equation of line implies
\begin{align}
    \vec{r} = \myvec{1\\2\\3} + \lambda\myvec{-3\\5\\4} \\
\end{align}

\end{document}
\end{enumerate}


\subsection{Exercises}
\begin{enumerate}[label=\thesection.\arabic*,ref=\thesection.\theenumi]
\item The area of a triangle with vertices $\vec{A}(3, 0), \vec{B}(7, 0) \text{ and } \vec{C}(8, 4)$ is
\begin{enumerate}
\item 14
\item 28
\item 8
\item 6
\end{enumerate}
\item The area of a triangle with vertices $(a,b+c), (b,c+a)\text{ and }(c,a+b)$ is
\begin{enumerate}
\item $(a+b+c)^2$
\item 0
\item a+b+c
\item abc 
\end{enumerate}
\item Find the area of the triangle whose vertices are $(-8,4),(-6,6)\text{ and }(-3,9)$.
\item If $\vec{D}\brak{\frac{-1}{2},\frac{5}{2}},\vec{E}(7,3)\text{ and }\vec{F}\brak{\frac{7}{2},\frac{7}{2}}$ are the midpoints of sides of $\triangle \vec{ABC}$, find the area of the $\triangle \vec{ABC}$.
\item If $\vec{a}+\vec{b}+\vec{c}$=0, show that $\vec{a}\times\vec{b}$=$\vec{b}\times\vec{c}$=$\vec{c}\times\vec{a}$. Interpret the result geometrically?
\item Find the sine of the angle between the vectors $\vec{a}=3\hat{i}+\hat{j}+2\hat{k}$ $\text{ and }$ $\vec{b}=2\hat{i}-2\hat{j}+4\hat{k}$.
\item Using vectors, find the area of $\triangle{ABC}$ with vertices A(1,2,3), B(2,-1,4) and C(4,5,-1).
\item Using vectors, prove that the parallelogram on the same base and between the same parallels are equal in area.
\item If $\vec{a}$, $\vec{b}$, $\vec{c}$ ,determine the vertices of a triangle, show that $\frac{1}{2}$ $\left[\vec{b} \times\vec{c}+\vec{c} \times\vec{a}+\vec{a}\times\vec{b} \right]$ gives the vector area of the trianlge. Hence deduce the condition that the three points $\vec{a},\vec{b},\vec{c},$ are collinear. Also find the unit vector normal to the plane of the triangle.
\item Show that area of the parallelogram whose diagonals are given by $\vec{a}\times\vec{b}$ is $\frac{\abs{\vec{a}\times\vec{b}}}{2}$. Also find the area of the parallelogram whose diagonals are $2\hat{i}-\hat{j}+\hat{k}$ $\text{and}$ $\hat{i}+3\hat{j}-\hat{k}$.

\item The vector from origin to the points A and B are $\vec{a}$ = $2\hat{i}-3\hat{j}+2\hat{k}$ $\text{and}$  $\vec{b}$ = $2\hat{i}+3\hat{j}+\hat{k}$, respectively, then the area of $\triangle {OAB}$ is
	\begin{enumerate}
\item 340 
\item $\sqrt{25}$
\item $\sqrt{229}$
\item $\frac{1}{2}\sqrt{229}$
\end{enumerate}


\item For any vector $\vec{a}$, the value of $(\vec{a}\times\hat{i})^2+(\vec{a}\times\hat{j})^2 + (\vec{a}\times\hat{k})^2$is equal to 
	\begin{enumerate}
\item a 
\item 3a
\item 4a
\item 2a
\end{enumerate}


\item If $\abs{\vec{a}}$=10, $\abs{\vec{b}}=2$ $\text{ and }$  $\vec{a}$, $\vec{b}$=12, then value of $\abs{\vec{a}\times\vec{b}}$ is
	\begin{enumerate}
\item 5 
\item 10 
\item 14 
\item 16
\end{enumerate}

\item If $\vec{a}$ = $\hat{i}+\hat{j}+\hat{k}$ $\text{and}$ $\vec{b}$ = $\hat{j}-\hat{k}$, find a the vector $\vec{c}$ such that $\vec{a}\times\vec{c}$ = $\vec{b}$ $\text{ and}$ $\vec{a}$.$\vec{c}$ = 3.

\item The formula $(\vec{a}+\vec{b})$ = $\vec{a}$+$\vec{b}$ + 2$\vec{a}\times\vec{b}$ is valid for non-zero vectors $\vec{a}$ $\text{and}$ $\vec{b}$.
\item The area of the quadrilateral ABCD, where A$(0,4,1)$, B$(2,3,-1)$, C$(4,5,0)$ and D$(2,6,2)$, is equal to 
\begin{enumerate}
	\item 9 sq. units
	\item 18 sq. units 
	\item 27 sq. units 
	\item 81 sq. units
\end{enumerate}
\item Find the area of region bounded by the triangle whose vertices are (-1, 1), (0, 5) and (3, 2).
\end{enumerate}


\subsection{Miscellaneous}
\begin{enumerate}[label=\thesection.\arabic*,ref=\thesection.\theenumi]
\item Find the sum of the vectors $\vec{a}=\hat{i}-2\hat{j}+\hat{k}$, $\vec{b}=-2\hat{i}+4\hat{j}+5\hat{k}$ and $\vec{c}=\hat{i}-6\hat{j}-7\hat{k}$.
\item 

	In triangle ABC (Fig 10.18), which of the following is not true:
 \begin{enumerate}
         \item $\overrightarrow{AB}+\overrightarrow{BC}+\overrightarrow{CA}$=$\vec{0}$
         \item $\overrightarrow{AB}+\overrightarrow{BC}-\overrightarrow{CA}$=$\vec{0}$
         \item $\overrightarrow{AB}+\overrightarrow{BC}-\overrightarrow{CA}$=$\vec{0}$
         \item $\overrightarrow{AB}-\overrightarrow{BC}+\overrightarrow{CA}$=$\vec{0}$
\end{enumerate}
\begin{figure}[h]
\centering
\includegraphics[width = \columnwidth]{./chapters/12/10/2/18/figs/triangle.png}
\caption{}
	\label{fig:chapters/12/10/2/18/}
\end{figure}
\solution
		\iffalse
\documentclass[journal,12pt,twocolumn]{IEEEtran}
\usepackage{romannum}
\usepackage{float}
\usepackage{setspace}
\usepackage{gensymb}
\singlespacing
\usepackage[cmex10]{amsmath}
\usepackage{amsthm}
\usepackage{mathrsfs}
\usepackage{txfonts}
\usepackage{stfloats}
\usepackage{bm}
\usepackage{cite}
\usepackage{cases}
\usepackage{subfig}
\usepackage{longtable}
\usepackage{multirow}
\usepackage{enumitem}
\usepackage{mathtools}
\usepackage{steinmetz}
\usepackage{tikz}
\usepackage{circuitikz}
\usepackage{verbatim}
\usepackage{tfrupee}
\usepackage[breaklinks=true]{hyperref}
\usepackage{tkz-euclide}
\usetikzlibrary{calc,math}
\usepackage{listings}
    \usepackage{color}                                            %%
    \usepackage{array}                                            %%
    \usepackage{longtable}                                        %%
    \usepackage{calc}                                             %%
    \usepackage{multirow}                                         %%
    \usepackage{hhline}                                           %%
    \usepackage{ifthen}                                           %%
  %optionally (for landscape tables embedded in another document): %%
    \usepackage{lscape}     
\usepackage{multicol}
\usepackage{chngcntr}
\DeclareMathOperator*{\Res}{Res}
\renewcommand\thesection{\arabic{section}}
\renewcommand\thesubsection{\thesection.\arabic{subsection}}
\renewcommand\thesubsubsection{\thesubsection.\arabic{subsubsection}}

\renewcommand\thesectiondis{\arabic{section}}
\renewcommand\thesubsectiondis{\thesectiondis.\arabic{subsection}}
\renewcommand\thesubsubsectiondis{\thesubsectiondis.\arabic{subsubsection}}

% correct bad hyphenation here
\hyphenation{op-tical net-works semi-conduc-tor}
\def\inputGnumericTable{}                                 %%

\lstset{
frame=single, 
breaklines=true,
columns=fullflexible
}

\begin{document}


\newtheorem{theorem}{Theorem}[section]
\newtheorem{problem}{Problem}
\newtheorem{proposition}{Proposition}[section]
\newtheorem{lemma}{Lemma}[section]
\newtheorem{corollary}[theorem]{Corollary}
\newtheorem{example}{Example}[section]
\newtheorem{definition}[problem]{Definition}
\newcommand{\BEQA}{\begin{eqnarray}}
\newcommand{\EEQA}{\end{eqnarray}}
\newcommand{\define}{\stackrel{\triangle}{=}}

\bibliographystyle{IEEEtran}
\providecommand{\mbf}{\mathbf}
\providecommand{\pr}[1]{\ensuremath{\Pr\left(#1\right)}}
\providecommand{\qfunc}[1]{\ensuremath{Q\left(#1\right)}}
\providecommand{\sbrak}[1]{\ensuremath{{}\left[#1\right]}}
\providecommand{\lsbrak}[1]{\ensuremath{{}\left[#1\right.}}
\providecommand{\rsbrak}[1]{\ensuremath{{}\left.#1\right]}}
\providecommand{\brak}[1]{\ensuremath{\left(#1\right)}}
\providecommand{\lbrak}[1]{\ensuremath{\left(#1\right.}}
\providecommand{\rbrak}[1]{\ensuremath{\left.#1\right)}}
\providecommand{\cbrak}[1]{\ensuremath{\left\{#1\right\}}}
\providecommand{\lcbrak}[1]{\ensuremath{\left\{#1\right.}}
\providecommand{\rcbrak}[1]{\ensuremath{\left.#1\right\}}}
\theoremstyle{remark}
\newtheorem{rem}{Remark}
\newcommand{\sgn}{\mathop{\mathrm{sgn}}}
\providecommand{\abs}[1]{\left\vert#1\right\vert}
\providecommand{\res}[1]{\Res\displaylimits_{#1}} 
\providecommand{\norm}[1]{\left\lVert#1\right\rVert}
\providecommand{\mtx}[1]{\mathbf{#1}}
\providecommand{\mean}[1]{E\left[ #1 \right]}
\providecommand{\fourier}{\overset{\mathcal{F}}{ \rightleftharpoons}}
\providecommand{\system}{\overset{\mathcal{H}}{ \longleftrightarrow}}
\newcommand{\solution}{\noindent \textbf{Solution: }}
\newcommand{\cosec}{\,\text{cosec}\,}
\providecommand{\dec}[2]{\ensuremath{\overset{#1}{\underset{#2}{\gtrless}}}}
\newcommand{\myvec}[1]{\ensuremath{\begin{pmatrix}#1\end{pmatrix}}}
\newcommand{\mydet}[1]{\ensuremath{\begin{vmatrix}#1\end{vmatrix}}}
\numberwithin{equation}{subsection}
\makeatletter
\@addtoreset{figure}{problem}
\makeatother

\let\StandardTheFigure\thefigure
\let\vec\mathbf
\renewcommand{\thefigure}{\theproblem}



\def\putbox#1#2#3{\makebox[0in][l]{\makebox[#1][l]{}\raisebox{\baselineskip}[0in][0in]{\raisebox{#2}[0in][0in]{#3}}}}
     \def\rightbox#1{\makebox[0in][r]{#1}}
     \def\centbox#1{\makebox[0in]{#1}}
     \def\topbox#1{\raisebox{-\baselineskip}[0in][0in]{#1}}
     \def\midbox#1{\raisebox{-0.5\baselineskip}[0in][0in]{#1}}

\vspace{3cm}


\title{Assignment 1}
\author{Jaswanth Chowdary Madala}





% make the title area
\maketitle

\newpage

%\tableofcontents

\bigskip

\renewcommand{\thefigure}{\theenumi}
\renewcommand{\thetable}{\theenumi}

\begin{enumerate}
\item In the following figure for the triangle ABC, which of the following is not true:

(A) $\overrightarrow{AB}+\overrightarrow{BC}+\overrightarrow{CA} = \overrightarrow{0}$

(B) $\overrightarrow{AB}+\overrightarrow{BC}-\overrightarrow{AC} = \overrightarrow{0}$

(C) $\overrightarrow{AB}+\overrightarrow{BC}+\overrightarrow{AC} = \overrightarrow{0}$

(D) $\overrightarrow{AB}-\overrightarrow{CB}+\overrightarrow{CA} = \overrightarrow{0}$

\textbf{Solution:}
\fi
		We know that,
\begin{align}
\overrightarrow{AB} = \vec{B} - \vec{A}\\
\overrightarrow{BC} = \vec{C} - \vec{B}\\
\overrightarrow{CA} = \vec{A} - \vec{C}
\end{align}
By usinig this we verify each of the given option

\begin{enumerate}
\item 
\begin{align}
\overrightarrow{AB}+\overrightarrow{BC}+\overrightarrow{CA} &=
\vec{B}-\vec{A} + \vec{C} - \vec{B} + \vec{A} - \vec{C}\\
 &= 0
\end{align}
Option A is correct.

\item
\begin{align}
\overrightarrow{AB}+\overrightarrow{BC}-\overrightarrow{AC} &=
\vec{B}-\vec{A} + \vec{C} - \vec{B} - (\vec{C} - \vec{A})\\
 &= 0
\end{align}
Option B is correct.

\item 
\begin{align}
\overrightarrow{AB}+\overrightarrow{BC}+\overrightarrow{AC} &=
\vec{B}-\vec{A} + \vec{C} - \vec{B} + \vec{C} - \vec{A}\\
 &= 2(\vec{C}-\vec{A})
\end{align}
Option C is incorrect.

\item
\begin{align}
\overrightarrow{AB}-\overrightarrow{CB}+\overrightarrow{CA} &=
\vec{B}-\vec{A} - (\vec{B} - \vec{C}) + \vec{A} - \vec{C}\\
 &= 0
\end{align}
Option D is correct.
\end{enumerate}
Verification: Let us take an example to verify 
\begin{align}
\vec{A} = \myvec{1\\1}, 
\vec{B} = \myvec{3\\1},
\vec{C} = \myvec{6\\6} \\
\overrightarrow{AB} = \vec{B} - \vec{A} = \myvec{2\\0}, 
\overrightarrow{BC} = \vec{C} - \vec{B} = \myvec{3\\5},
\overrightarrow{CA} = \vec{A} - \vec{C} = \myvec{-5\\-5} 
\end{align}
Thus,
\begin{align}
\overrightarrow{AB}+\overrightarrow{BC}+\overrightarrow{CA} = \myvec{2+3+(-5) \\ 0+5+(-5)} = \myvec{0 \\0}
\end{align}
Similarly other options can be verified.



\item If $\vec{a}$ and $\vec{b}$ are two collinear vectors, then which of the following are incorrect:
\begin{enumerate}
    \item $\vec{b}=\lambda\vec{a},$
 for some scalar $\lambda$
    \item $\vec{a}=\pm\vec{b}$
    \item the respective components of $\vec{a}$ and $\vec{b}$ are not proportiona
    \item both the vectors $\vec{a}$ and $\vec{b}$ have same direction, but different magnitudes.
\end{enumerate}
	\item If a line makes angles $90\degree,135\degree,45\degree$ with x,y and z-axis respectivly. Find its direction cosines.
		\\
		\solution
		\iffalse
\documentclass[12pt]{article}
\usepackage{graphicx}
%\documentclass[journal,12pt,twocolumn]{IEEEtran}
\usepackage[none]{hyphenat}
\usepackage{graphicx}
\usepackage{listings}
\usepackage[english]{babel}
\usepackage{graphicx}
\usepackage{caption} 
\usepackage{hyperref}
\usepackage{booktabs}
\def\inputGnumericTable{}
\usepackage{color}                                            %%
    \usepackage{array}                                            %%
    \usepackage{longtable}                                        %%
    \usepackage{calc}                                             %%
    \usepackage{multirow}                                         %%
    \usepackage{hhline}                                           %%
    \usepackage{ifthen}
\usepackage{array}
\usepackage{amsmath}   % for having text in math mode
\usepackage{listings}
\lstset{
language=tex,
frame=single, 
breaklines=true
}
  
%Following 2 lines were added to remove the blank page at the beginning
\usepackage{atbegshi}% http://ctan.org/pkg/atbegshi
\AtBeginDocument{\AtBeginShipoutNext{\AtBeginShipoutDiscard}}
%


%New macro definitions
\newcommand{\mydet}[1]{\ensuremath{\begin{vmatrix}#1\end{vmatrix}}}
\providecommand{\brak}[1]{\ensuremath{\left(#1\right)}}
\providecommand{\norm}[1]{\left\lVert#1\right\rVert}
\newcommand{\solution}{\noindent \textbf{Solution: }}
\newcommand{\myvec}[1]{\ensuremath{\begin{pmatrix}#1\end{pmatrix}}}
\let\vec\mathbf

\begin{document}

\begin{center}
\title{\textbf{Coordinate Geometry}}
\date{\vspace{-5ex}} %Not to print date automatically
\maketitle
\end{center}

\setcounter{page}{1}



\begin{enumerate}

\item\textbf{Problem statement :} Find a relation between x and y such that the point $\myvec{x ,y}$ is equidistant from the point $\myvec{3 ,6}$ and $\myvec{-3 ,4}$

\solution \\
\textbf{\centering{Method I}}
\fi
The input parameters for this problem are given as
	\begin{align}
	\vec{P} = \myvec{
		x\\
		y\\
		},
	\vec{A} = \myvec{
		3\\
		6\\
		},
        \vec{B} = \myvec{
		3\\
		-4\\
		}
	\end{align}
\iffalse

  If $\vec{P}$ equidistant from the points $\vec{A}$ and $\vec{B}$, 
\begin{align}
 \norm{\vec{P}-\vec{A}} &=
\norm{\vec{P}-\vec{B}} 
\\
 \implies \norm{\vec{P}-\vec{A}}^2 &=
\norm{\vec{P}-\vec{B}}^2 
\end{align}
which can be expressed as 
\begin{align}
%  \label{eq:chapters/10/7/1/10/norm2d_dist}
 \brak{\vec{P}-\vec{A}}^{\top} \brak{\vec{P}-\vec{A}}=
 \brak{\vec{P}-\vec{B}}^{\top} 
\brak{\vec{P}-\vec{B}}
\\
 \implies \norm{\vec{P}}^2-2{\vec{P}}^{\top}\vec{A} + \norm{\vec{A}}^2
 \\= \norm{\vec{P}}^2-2{\vec{P}}^{\top}\vec{B} + \norm{\vec{B}}^2
\end{align}
which can be simplified to obtain
%  \eqref{eq:chapters/10/7/1/10/norm2d_equidist}.
  \fi
  \begin{align}
   \vec{P} =
    y\vec{e}_1
  \end{align}
  where 
  \begin{align}
   y &=\frac{\norm{\vec{A}}^2 -\norm{\vec{B}}^2 }{2\brak{\vec{A}-\vec{B}}^{\top }\vec{e}_1
}\label{eq:chapters/10/7/1/10/5}  
  \end{align}
  Substituting the $\vec{A}, \vec{B}$ values in \eqref{eq:chapters/10/7/1/10/5},
\begin{align}
 \brak{\vec{A}-\vec{B}}=
 \myvec{6 \\ 2},
   \norm{\vec{A}}^2 = 45,
   \norm{\vec{B}}^2 = 25
    \end{align}
    yielding
$y =  5$.
Hence, 
\begin{align}
\vec{P} = \myvec{ 0 \\ 5}
\end{align}
%
See Fig. 
\ref{fig:chapters/10/7/1/10/Fig1}.
\begin{figure}[!h]
 \begin{center}
  \includegraphics[width=\columnwidth]{./chapters/10/7/1/10/figs/fig.png}
 \end{center}
\caption{}
\label{fig:chapters/10/7/1/10/Fig1}
\end{figure}


\item A girl walks 4 km towards west, then she walks 3 km in a direction 30$^{\circ}$ east of north and stops. Determine the girl's displacement from her initial point of departure.\\
	\solution
		\iffalse
\documentclass[12pt]{article}
\usepackage{graphicx}
%\documentclass[journal,12pt,twocolumn]{IEEEtran}
\usepackage[none]{hyphenat}
\usepackage{graphicx}
\usepackage{listings}
\usepackage[english]{babel}
\usepackage{graphicx}
\usepackage{caption} 
\usepackage{hyperref}
\usepackage{booktabs}
\def\inputGnumericTable{}
\usepackage{color}                                            %%
    \usepackage{array}                                            %%
    \usepackage{longtable}                                        %%
    \usepackage{calc}                                             %%
    \usepackage{multirow}                                         %%
    \usepackage{hhline}                                           %%
    \usepackage{ifthen}
\usepackage{array}
\usepackage{amsmath}   % for having text in math mode
\usepackage{listings}
\lstset{
language=tex,
frame=single, 
breaklines=true
}
  
%Following 2 lines were added to remove the blank page at the beginning
\usepackage{atbegshi}% http://ctan.org/pkg/atbegshi
\AtBeginDocument{\AtBeginShipoutNext{\AtBeginShipoutDiscard}}
%


%New macro definitions
\newcommand{\mydet}[1]{\ensuremath{\begin{vmatrix}#1\end{vmatrix}}}
\providecommand{\brak}[1]{\ensuremath{\left(#1\right)}}
\providecommand{\norm}[1]{\left\lVert#1\right\rVert}
\newcommand{\solution}{\noindent \textbf{Solution: }}
\newcommand{\myvec}[1]{\ensuremath{\begin{pmatrix}#1\end{pmatrix}}}
\let\vec\mathbf

\begin{document}

\begin{center}
\title{\textbf{Coordinate Geometry}}
\date{\vspace{-5ex}} %Not to print date automatically
\maketitle
\end{center}

\setcounter{page}{1}



\begin{enumerate}

\item\textbf{Problem statement :} Find a relation between x and y such that the point $\myvec{x ,y}$ is equidistant from the point $\myvec{3 ,6}$ and $\myvec{-3 ,4}$

\solution \\
\textbf{\centering{Method I}}
\fi
The input parameters for this problem are given as
	\begin{align}
	\vec{P} = \myvec{
		x\\
		y\\
		},
	\vec{A} = \myvec{
		3\\
		6\\
		},
        \vec{B} = \myvec{
		3\\
		-4\\
		}
	\end{align}
\iffalse

  If $\vec{P}$ equidistant from the points $\vec{A}$ and $\vec{B}$, 
\begin{align}
 \norm{\vec{P}-\vec{A}} &=
\norm{\vec{P}-\vec{B}} 
\\
 \implies \norm{\vec{P}-\vec{A}}^2 &=
\norm{\vec{P}-\vec{B}}^2 
\end{align}
which can be expressed as 
\begin{align}
%  \label{eq:chapters/10/7/1/10/norm2d_dist}
 \brak{\vec{P}-\vec{A}}^{\top} \brak{\vec{P}-\vec{A}}=
 \brak{\vec{P}-\vec{B}}^{\top} 
\brak{\vec{P}-\vec{B}}
\\
 \implies \norm{\vec{P}}^2-2{\vec{P}}^{\top}\vec{A} + \norm{\vec{A}}^2
 \\= \norm{\vec{P}}^2-2{\vec{P}}^{\top}\vec{B} + \norm{\vec{B}}^2
\end{align}
which can be simplified to obtain
%  \eqref{eq:chapters/10/7/1/10/norm2d_equidist}.
  \fi
  \begin{align}
   \vec{P} =
    y\vec{e}_1
  \end{align}
  where 
  \begin{align}
   y &=\frac{\norm{\vec{A}}^2 -\norm{\vec{B}}^2 }{2\brak{\vec{A}-\vec{B}}^{\top }\vec{e}_1
}\label{eq:chapters/10/7/1/10/5}  
  \end{align}
  Substituting the $\vec{A}, \vec{B}$ values in \eqref{eq:chapters/10/7/1/10/5},
\begin{align}
 \brak{\vec{A}-\vec{B}}=
 \myvec{6 \\ 2},
   \norm{\vec{A}}^2 = 45,
   \norm{\vec{B}}^2 = 25
    \end{align}
    yielding
$y =  5$.
Hence, 
\begin{align}
\vec{P} = \myvec{ 0 \\ 5}
\end{align}
%
See Fig. 
\ref{fig:chapters/10/7/1/10/Fig1}.
\begin{figure}[!h]
 \begin{center}
  \includegraphics[width=\columnwidth]{./chapters/10/7/1/10/figs/fig.png}
 \end{center}
\caption{}
\label{fig:chapters/10/7/1/10/Fig1}
\end{figure}


\item If $\vec{a}=\hat{i}+\hat{j}+\hat{k}$,$\vec{b}=2\hat{i}-\hat{j}+3\hat{k}$ and $\vec{c}=\hat{i}-2\hat{j}+\hat{k}$, find a unit vector parallel to the vector $2\vec{a}-\vec{b}+3\vec{c}$.\\
	\solution
		\iffalse
\documentclass[12pt]{article}
\usepackage{graphicx}
%\documentclass[journal,12pt,twocolumn]{IEEEtran}
\usepackage[none]{hyphenat}
\usepackage{graphicx}
\usepackage{listings}
\usepackage[english]{babel}
\usepackage{graphicx}
\usepackage{caption} 
\usepackage{hyperref}
\usepackage{booktabs}
\usepackage{array}
\usepackage{amsmath}   % for having text in math mode
\usepackage{listings}
\lstset{
  frame=single,
  breaklines=true
}
  
%Following 2 lines were added to remove the blank page at the beginning
\usepackage{atbegshi}% http://ctan.org/pkg/atbegshi
\AtBeginDocument{\AtBeginShipoutNext{\AtBeginShipoutDiscard}}
%


%New macro definitions
\newcommand{\mydet}[1]{\ensuremath{\begin{vmatrix}#1\end{vmatrix}}}
\providecommand{\brak}[1]{\ensuremath{\left(#1\right)}}
\providecommand{\norm}[1]{\left\lVert#1\right\rVert}
\newcommand{\solution}{\noindent \textbf{Solution: }}
\newcommand{\myvec}[1]{\ensuremath{\begin{pmatrix}#1\end{pmatrix}}}
\let\vec\mathbf

\begin{document}

\begin{center}
\title{\textbf{Vector Dot Product}}
\date{\vspace{-5ex}} %Not to print date automatically
\maketitle
\end{center}
\setcounter{page}{1}

\section{12$^{th}$ Maths - Chapter 10}
This is Problem-9 from Exercise 10.3
\begin{enumerate}
\item Find $\norm{\vec{x}}$, if for a unit vector $\vec{a}$, $\brak{\vec{x}-\vec{a}}.\brak{\vec{x}+\vec{a}} = 12$.\\
	\fi
\solution 
From the given information,
\begin{align}
  \label{eq:12/10/3/9det2f}
  \brak{\vec{x}-\vec{a}}^\top\brak{\vec{x}+\vec{a}} &= 12 \\
  \implies \vec{x}^\top\vec{x} - \vec{a}^\top\vec{x} + \vec{x}^\top\vec{a} - \vec{a}^\top\vec{a} &= 12 \\
  \implies \norm{\vec{x}}^{2} - \norm{\vec{a}}^{2} &= 12 \\
\implies   \norm{\vec{x}}^{2} - 1 &= 12  \\
	\text{or, }  
	\norm{\vec{x}} &= \sqrt{13}
\end{align}

\item The two opposite vertices of a square are $(–1, 2)$  and $ (3, 2)$. Find the coordinates of the other two vertices.
\\
	\iffalse
\documentclass[12pt]{article}
\usepackage{graphicx}
\usepackage{amsmath}
\usepackage{mathtools}
\usepackage{gensymb}

\newcommand{\mydet}[1]{\ensuremath{\begin{vmatrix}#1\end{vmatrix}}}
\providecommand{\brak}[1]{\ensuremath{\left(#1\right)}}
\providecommand{\norm}[1]{\left\lVert#1\right\rVert}
\newcommand{\solution}{\noindent \textbf{Solution: }}
\newcommand{\myvec}[1]{\ensuremath{\begin{pmatrix}#1\end{pmatrix}}}
\let\vec\mathbf

\begin{document}
\begin{center}
\textbf\large{CHAPTER-7 \\ COORDINATE GEOMETRY}

\end{center}
\section*{Excercise 7.4}

Q4.The two opposite vertices of a square are $(–1, 2) \text{ and } (3, 2)$. Find the coordinates of the other two vertices.\\
\fi
\solution
Let
\begin{align}
\vec{A} = \myvec
{
-1 \\
 2\\
},
\vec{C} = 
\myvec
{
3\\
2\\
}
\end{align}

\begin{figure}[!h]
	\begin{center} 
	    \includegraphics[width=\columnwidth]{chapters/10/7/4/4/figs/square}
	\end{center}
\caption{}
\label{fig:7/4/4/4Fig1}
\end{figure}

Shifting $\vec{A}$ to origin with reference to Fig. \ref{fig:7/4/4/4Fig2},
\begin{align}
\vec{A^{\prime}} =
\myvec{
0 \\
0\\
},
\vec{C^{\prime}} = \vec{C}-\vec{A} = 
\myvec{
4 \\
0\\
}
\end{align}

\begin{figure}[!h]
	\begin{center} 
	    \includegraphics[width=\columnwidth]{chapters/10/7/4/4/figs/square1}
	\end{center}
\caption{}
\label{fig:7/4/4/4Fig2}
\end{figure}
\iffalse
In general,
the angle made by $AC$ with the x-axis is 
		\begin{align}
\beta = \theta + 45\degree
		\end{align}
\fi
Since
\begin{align}
\vec{C} - \vec{A} = \myvec{
4\\
0
} \equiv 
\myvec{
1\\
0
},
	\tan\theta&= \frac{0}{4} \implies 
\theta= 0\degree
\end{align}
		where
$\theta$ is the angle made by $AC$ with the x-axis.
Considering the rotation matrix 
\begin{align}
\vec{P} =
\myvec{
\cos\brak{\frac{\pi}{4}-\theta} & -\sin\brak{\frac{\pi}{4}-\theta} \\
\sin\brak{\frac{\pi}{4}-\theta} & \cos\brak{\frac{\pi}{4}-\theta} 
}
\end{align}
\iffalse
from Fig. \ref{fig:7/4/4/4Fig3},
\begin{align}
\vec{C^{\prime \prime}} = \vec{P}^\top (\vec{C}-\vec{A}) =
\myvec{
\frac{1}{\sqrt{2}} & -\frac{1}{\sqrt{2}} \\
\frac{1}{\sqrt{2}} & \frac{1}{\sqrt{2}}\\
}
\myvec{
4 \\
0\\
} = 
\myvec{
\frac{4}{\sqrt{2}} \\
\frac{4}{\sqrt{2}}\\
}
\end{align}
\begin{align}
\vec{B^{\prime \prime}} = \myvec{
 1&0\\
 0&0\\
}\vec{C^{\prime \prime}}=
\myvec{
 \frac{4}{\sqrt{2}}\\
 0\\
},
\vec{D^{\prime \prime}} = \myvec{
 0&0\\
 0&1\\
}\vec{C^{\prime \prime}}=
\myvec{
 0\\
 \frac{4}{\sqrt{2}}\\
} \text{ and }
\vec{A^{\prime \prime}} =
\myvec{
0 \\
0\\
}
\end{align}
\fi
\begin{figure}[!h]
	\begin{center} 
	    \includegraphics[width=\columnwidth]{chapters/10/7/4/4/figs/square2}
	\end{center}
\caption{}
\label{fig:7/4/4/4Fig3}
\end{figure}

\newpage
\iffalse
Again tranforming(rotating) the coordinates back to the original axis.

We know for anti-clockwise direction the rotation matrix is given as
\begin{align}
\vec{P} =
\myvec{
\cos\theta & -\sin\theta \\
\sin\theta & \cos\theta \\
}
\end{align}

Again we know that the angle is negative so the rotation will be in clockwise direction. So now the transformed(rotated) coordinates $\vec{B} \text{ and } \vec{D}$ are with refrence to 
\fi
from Figure 
%\ref{fig:7/4/4/4Fig4},
\ref{fig:7/4/4/4Fig3},
\begin{align}
	\vec{C^{\prime \prime}} &= \vec{P} (\vec{C}-\vec{A}) 
	\\
\label{eq:7/4/4/4bp}
	\vec{B^{\prime \prime}} &= \myvec{\vec{e}_1 & \vec{0}}\vec{C^{\prime \prime}}
	\\
\label{eq:7/4/4/4dp}
	\vec{D^{\prime \prime}} &= \myvec{ \vec{0} & \vec{e}_2}\vec{C^{\prime \prime}}
\end{align}
Now, 
\begin{align}
\label{eq:7/4/4/4b}
	\vec{B} = \vec{P}^{\top}\vec{B}^{\prime \prime}+\vec{A}
	\\
\label{eq:7/4/4/4d}
	\vec{D} = \vec{P}^{\top}\vec{D}^{\prime \prime}+\vec{A}
\end{align}
by reversing the process of translation and rotation.  Thus, 
from
\eqref{eq:7/4/4/4b}
\eqref{eq:7/4/4/4bp},
\eqref{eq:7/4/4/4d}
and
\eqref{eq:7/4/4/4dp}
\begin{align}
	\vec{B} = \vec{P}^{\top}\myvec{\vec{e}_1 & \vec{0}}\vec{P} (\vec{C}-\vec{A}) +\vec{A}
	\\
	\vec{D} = \vec{P}^{\top}\myvec{\vec{0} & \vec{e}_2  }\vec{P} (\vec{C}-\vec{A}) +\vec{A}
%	\vec{B} &= \brak{(\vec{C}-\vec{A})^{\top}\vec{P}^{\top} \vec{e}_1}\vec{P}^{\top}\vec{e}_1+\vec{A}
%	\\
%	\vec{D} &= \brak{(\vec{C}-\vec{A})^{\top}\vec{P}^{\top} \vec{e}_2}\vec{P}^{\top}\vec{e}_2+\vec{A}
\end{align}
yielding
		\begin{align}
\vec{B}=
\myvec{
1\\
0
},
\vec{D}
\myvec{
1\\
4
}.
		\end{align}
\iffalse
\begin{align}
\vec{B^{\prime}} &= \vec{P}\vec{B^{\prime \prime}} = \myvec{
\frac{1}{\sqrt{2}} & \frac{1}{\sqrt{2}} \\
-\frac{1}{\sqrt{2}} & \frac{1}{\sqrt{2}}\\
}
\myvec{
 \frac{4}{\sqrt{2}}\\
 0\\
} = 
\myvec{
2 \\
-2\\
}\\
\vec{D^{\prime}} &= \vec{P}\vec{D^{\prime \prime}} = \myvec{
\frac{1}{\sqrt{2}} & \frac{1}{\sqrt{2}} \\
-\frac{1}{\sqrt{2}} & \frac{1}{\sqrt{2}}\\
}
\myvec{
 0\\
 \frac{4}{\sqrt{2}}\\
} = 
\myvec{
2 \\
2 \\
}
\end{align}

\begin{figure}[!h]
	\begin{center} 
	    \includegraphics[width=\columnwidth]{chapters/10/7/4/4/figs/square3}
	\end{center}
\caption{}
\label{fig:7/4/4/4Fig4}
\end{figure}

Again transforming(shifting) the axis back to the original with refrence to Figure \ref{fig:7/4/4/4Fig5}
\begin{align}
\vec{B} &= \vec{B^{\prime}}+\vec{A} = \myvec{
2 \\
-2\\
}+\myvec{
-1 \\
2\\
} = 
\myvec{
1 \\
0\\
}\\
\vec{D} &= \vec{D^{\prime}}+\vec{A} = \myvec{
2 \\
2\\
}+\myvec{
-1 \\
2\\
} = 
\myvec{
1 \\
4 \\
}
\end{align}

Hence, the other two vertices are $\vec{B}(1,0) \text{ and } \vec{D}(1,4)$   

\begin{figure}[!h]
	\begin{center} 
	    \includegraphics[width=\columnwidth]{chapters/10/7/4/4/figs/square4}
	\end{center}
\caption{}
\label{fig:7/4/4/4Fig5}
\end{figure}
which can also be expressed as
\begin{align}
\vec{B} &= \vec{A} + \vec{P}\myvec{
\vec{e_{1}}&\vec{0}\\
}
\vec{P}^\top \brak{\vec{C}-\vec{A}}\\
\vec{D} &= \vec{A} + \vec{P}\myvec{
\vec{0}&\vec{e_{2}}\\
}
\vec{P}^\top \brak{\vec{C}-\vec{A}}\\
\end{align}
where $\vec{P}$ is the rotation matrix and $\vec{A} \text{ and } \vec{C}$ are the position vectors of opposite vertices.

Derivation of the above formulas:

We know that after shifting the axis and rotating by the required angle any arbitrary square will be aligned with the x and y axis so that we can directly get the vectors $\vec{B} \text{ and } \vec{D}$ as follows
\begin{align}
\vec{C^{\prime\prime}} &= \vec{P}^\top \brak{\vec{C} - \vec{A}}\\
\vec{B^{\prime\prime}} &= \myvec{
\vec{e_{1}} & \vec{0}
}\vec{C^{\prime\prime}} = \myvec{
\vec{e_{1}} & \vec{0}
}\vec{P}^\top \brak{\vec{C} - \vec{A}}\\
\vec{B^{\prime}} &= \vec{P} \vec{B^{\prime\prime}}  = \vec{P}\myvec{
\vec{e_{1}} & \vec{0}
}\vec{P}^\top \brak{\vec{C} - \vec{A}}\\
\vec{B} &= \vec{A}+\vec{B^{\prime}}\\
 &= \vec{A} + \vec{P}\myvec{
\vec{e_{1}}&\vec{0}\\
}
\vec{P}^\top\brak{\vec{C}-\vec{A}}
\end{align}

Similarly for D it can be derived as
\begin{align}
\vec{C^{\prime\prime}} &= \vec{P}^\top \brak{\vec{C} - \vec{A}}\\
\vec{D^{\prime\prime}} &= \myvec{
\vec{0} & \vec{e_{2}}
}\vec{C^{\prime\prime}} = \myvec{
\vec{0} & \vec{e_{2}}
}\vec{P}^\top \brak{\vec{C} - \vec{A}}\\
\vec{D^{\prime}} &= \vec{P} \vec{D^{\prime\prime}} = \vec{P} \myvec{
\vec{0} & \vec{e_{2}}
}\vec{P}^\top \brak{\vec{C} - \vec{A}}\\
\vec{D} &= \vec{A}+\vec{D^{\prime}}\\
 &= \vec{A} + \vec{P}\myvec{
\vec{0}&\vec{e_{2}}\\
}
\vec{P}^\top\brak{\vec{C}-\vec{A}}
\end{align}


Verification of the above formula for the given question

\begin{align}
\vec{B} &= \myvec{
-1\\
2\\
}+\myvec{
\frac{1}{\sqrt{2}} & \frac{1}{\sqrt{2}} \\
-\frac{1}{\sqrt{2}} & \frac{1}{\sqrt{2}}\\
}\myvec{
 1&0\\
 0&0\\
}\myvec{
\frac{1}{\sqrt{2}} & -\frac{1}{\sqrt{2}} \\
\frac{1}{\sqrt{2}} & \frac{1}{\sqrt{2}}\\
}\myvec{
4\\
0\\
}\\
 &= \myvec{
-1\\
2\\
}+\myvec{
\frac{1}{\sqrt{2}} & \frac{1}{\sqrt{2}} \\
-\frac{1}{\sqrt{2}} & \frac{1}{\sqrt{2}}\\
}\myvec{
 1&0\\
 0&0\\
}\myvec{
\frac{4}{\sqrt{2}}\\
\frac{4}{\sqrt{2}}\\
}\\
 &= \myvec{
-1\\
2\\
}+\myvec{
\frac{1}{\sqrt{2}} & \frac{1}{\sqrt{2}} \\
-\frac{1}{\sqrt{2}} & \frac{1}{\sqrt{2}}\\
}\myvec{
\frac{4}{\sqrt{2}}\\
0\\
}\\
 &= \myvec{
-1\\
2\\
}+\myvec{
2\\
-2\\
}\\
 &= \myvec{
1\\
0\\
}\\
\vec{D} &= \myvec{
-1\\
2\\
}+\myvec{
\frac{1}{\sqrt{2}} & \frac{1}{\sqrt{2}} \\
-\frac{1}{\sqrt{2}} & \frac{1}{\sqrt{2}}\\
}\myvec{
 0&0\\
 0&1\\
}\myvec{
\frac{1}{\sqrt{2}} & -\frac{1}{\sqrt{2}} \\
\frac{1}{\sqrt{2}} & \frac{1}{\sqrt{2}}\\
}\myvec{
4\\
0\\
}\\
 &= \myvec{
-1\\
2\\
}+\myvec{
\frac{1}{\sqrt{2}} & \frac{1}{\sqrt{2}} \\
-\frac{1}{\sqrt{2}} & \frac{1}{\sqrt{2}}\\
}\myvec{
 0&0\\
 0&1\\
}\myvec{
\frac{4}{\sqrt{2}}\\
\frac{4}{\sqrt{2}}\\
}\\
 &= \myvec{
-1\\
2\\
}+\myvec{
\frac{1}{\sqrt{2}} & \frac{1}{\sqrt{2}} \\
-\frac{1}{\sqrt{2}} & \frac{1}{\sqrt{2}}\\
}\myvec{
0\\
\frac{4}{\sqrt{2}}\\
}\\
 &= \myvec{
-1\\
2\\
}+\myvec{
2\\
2\\
}\\
 &= \myvec{
1\\
4\\
}
\end{align}
\fi








\item The base of an equilateral triangle with side $2a$ lies along the y-axis such that the mid-point of the base is at the origin. Find vertices of the triangle.
\label{chapters/11/10/1/2}
\iffalse
\documentclass[journal,12pt,twocolumn]{IEEEtran}
\usepackage{graphicx}
\graphicspath{{./figs/}}{}
\usepackage{amsmath,amssymb,amsfonts,amsthm}
\newcommand{\myvec}[1]{\ensuremath{\begin{pmatrix}#1\end{pmatrix}}}

\let\vec\mathbf

\title{
Matrix-Lines
}
\author{Jyothsna Paluchuri-FWC22059\\}
\begin{document}
\maketitle
\tableofcontents
\bigskip
\section{Problem Statement}
\fi
	\begin{figure}[!ht]
		\centering
 \includegraphics[width=\columnwidth]{chapters/11/10/1/5/figs/line.png}
		\caption{}
		\label{fig:11/10/1/5}
  	\end{figure}
	\\
	\solution
\iffalse
\section{Construction}
\begin{figure}[h]
    \centering
\includegraphics[width=\columnwidth]{line.png}
    \caption{Equation of the slope}
    \label{fig:my_label}
\end{figure}
\vspace{2cm}
\begin{table}[h]
    \centering
    \begin{tabular}{|c|c|c|c|}
       \hline
       \textbf{Symbol}&\textbf{Value}&\textbf{Description}  \\
       \hline
	    $\vec{P}$ & $\myvec{
		    0\\
		    -4}$
	    & Point on Y-axis\\
        \hline
	    $\vec{B}$ & $\myvec{8\\0}$
 & Point on X-axis\\
        \hline
	    $\vec{0}$ & $\myvec{0\\0}$
 & Origin\\
        \hline
    \end{tabular}
    \caption{Parameters}
    \label{tab:my_label}
\end{table}


\section{Solution}
Given that resultant line passes through origin and mid point of the line segment joining point P(0,-4) and B(8,0) \\
\\
\\
given ${\vec{P}}$=$\myvec{
  0\\
  -4}$
 , ${\vec{B}}$=$\myvec{
  8\\
  0}$
  
 \fi 
The mid point of $PB$ is
\begin{align}
\vec{M} &=\frac{1}{2}(\vec{P}+\vec{B})
	= \myvec{4 \\ -2}  
\end{align}
The direction vector of line joining $\vec{O}, \vec{M}$ is 
\begin{align}
\vec{m}&=\vec{O}-\vec{M}
 = -\vec{M}
\end{align}
which can be expressed as
\begin{align}
	\myvec{1 \\ -\frac{1}{2}}
\end{align}
Thus the slope is
\begin{align}
	m = -\frac{1}{2}
\end{align}
\iffalse
\textbf{The direction vector of a line expressed as}
\begin{align}
\implies\vec{m} &= \begin{pmatrix}1 \\ m \\ \end{pmatrix}
\end{align}

\textbf{By solving equation (5) and (6),we get the slope of $\vec{O}$ $\vec{M}$ line}
\begin{align}
        \boxed{m=-0.5}
 \end{align}

\section{Software}
Download the following code using,
\begin{table}[h]
    \centering
    \begin{tabular}{|c|}
    \hline \\
   https://github.com/jyothsna777/jyothsna-fwc.git  \\
         \\
\hline
    \end{tabular}
\end{table}
\\
and execute the code by using command
\begin{center}
\textbf{Python3 lines.py}\\
\end{center}

\section{Conclusion}
Hence the slope of line $\vec{O}$ $\vec{M}$ lineis $\vec{m}$=-0.5

\end{document}
\fi

\item Without using distance formula, show that points (– 2, – 1), (4, 0), (3, 3) and (–3, 2) are the vertices of a parallelogram.
\label{chapters/11/10/1/9}
\documentclass{article}
\usepackage{amsmath}
\usepackage{xcolor}
\usepackage{gensymb}
\usepackage{ragged2e}
\usepackage{graphicx}
\usepackage{gensymb}
\usepackage{mathtools}
\newcommand{\mydet}[1]{\ensuremath{\begin{vmatrix}#1\end{vmatrix}}}
\providecommand{\brak}[1]{\ensuremath{\left(#1\right)}}
\providecommand{\norm}[1]{\left\lVert#1\right\rVert}
\newcommand{\solution}{\noindent \textbf{Solution: }}
\newcommand{\myvec}[1]{\ensuremath{\begin{pmatrix}#1\end{pmatrix}}}
\let\vec\mathbf 


\begin{document}
\begin{center}
        \textbf\large{CHAPTER-7 \\ TRIANGLES}
\end{center}
\section{Exercise 7.1}
Q3. $AD$ and $BC$ are equal perpendiculars to a line segment $AB$. Show that $CD$ bisects $AB$.\\
\textbf{Construction}\\
\begin{figure}[h]
	\begin{center}
		\includegraphics[width=\columnwidth]{figs/Figure1.png}
	\end{center}
	\label{fig:Fig1}
\end{figure}
The input parameters for construction are shown in \ref{tab:Table1}:\\
\begin{table}[h]
	  \centering
	  \begin{tabular}{|c|c|c|}
  \hline
  \textbf{Symbol}&\textbf{Value}&\textbf{Description}\\
  \hline
  $a$ & 8 & $BC$\\
  \hline
	$\angle{B}$ & 45$\degree{}$ & $\angle{B}$ in $\triangle$$ABC$ \\
  \hline
	$k$ & 3.5 & $AB-AC$ i.e $c-b$ \\
  \hline 
	$\vec{e_2}$ & $\myvec{
			0\\
			1\\
			}$ & Basis vector\\
 \hline			
\end{tabular}

	  \caption{Parameters}
	  \label{tab:Table1}
\end{table}
\pagebreak
\begin{align}
	\vec{A} = a\vec{e_1},\vec{B} = \myvec{a\\b},\vec{C} = \myvec{2a\\b},\vec{D} = \myvec{0\\0}
\end{align}
\solution
Given
\begin{align}
	\vec{D}-\vec{A}&=\vec{B}-\vec{C}\\
	\angle DAB &= \angle CBA=90\degree
\end{align}
\textbf{To Prove:}\\
\begin{align}
	\vec{C}-\vec{O}&=\vec{O}-\vec{D}
\end{align}
\textbf{Proof:}\\
Consider linesegment $DC$\\
Let $\vec{O}$ represent the Midpoint of $DC$
\begin{align}
	\vec{O}&=\frac{1}{2}(\vec{C}+\vec{D})\\
	\implies &= \frac{1}{2}\myvec{6\\8}+\frac{1}{2}\myvec{0\\0}\\
	\implies &= \frac{1}{2}\myvec{6\\8}\\
		 &=\myvec{3\\4}
		 \label{eq:1}\\
\end{align}
\begin{align}
	\text { Since $AB\perp DA$, $AB$ is parallel to $x=0$ }\\
	\text { Equation of $AB$ is defined as $x=3$}\\
	\label{eq:2}\\
\end{align}
from $\eqref{eq:1}$ and $\eqref{eq:2}$
$\vec{O}$ lies on linesegment $CD$ and line $DC$ intersects $BA$ at its midpoint $O$.
\begin{align}
\vec{C}-\vec{O}=\vec{O}-\vec{D}
\end{align}
\end{document}

\item A line passes through $(x_1,y_1)$ and $(h,k)$. If slope of the line is m show that $(k-y_1)=m(h-x_1)$.
\label{chapters/11/10/1/12}
\iffalse
\documentclass[journal,12pt,twocolumn]{IEEEtran}
\usepackage{graphicx}
\graphicspath{{./figs/}}{}
\usepackage{amsmath,amssymb,amsfonts,amsthm}
\newcommand{\myvec}[1]{\ensuremath{\begin{pmatrix}#1\end{pmatrix}}}

\let\vec\mathbf

\title{
Matrix-Lines
}
\author{Jyothsna Paluchuri-FWC22059\\}
\begin{document}
\maketitle
\tableofcontents
\bigskip
\section{Problem Statement}
\fi
	\begin{figure}[!ht]
		\centering
 \includegraphics[width=\columnwidth]{chapters/11/10/1/5/figs/line.png}
		\caption{}
		\label{fig:11/10/1/5}
  	\end{figure}
	\\
	\solution
\iffalse
\section{Construction}
\begin{figure}[h]
    \centering
\includegraphics[width=\columnwidth]{line.png}
    \caption{Equation of the slope}
    \label{fig:my_label}
\end{figure}
\vspace{2cm}
\begin{table}[h]
    \centering
    \begin{tabular}{|c|c|c|c|}
       \hline
       \textbf{Symbol}&\textbf{Value}&\textbf{Description}  \\
       \hline
	    $\vec{P}$ & $\myvec{
		    0\\
		    -4}$
	    & Point on Y-axis\\
        \hline
	    $\vec{B}$ & $\myvec{8\\0}$
 & Point on X-axis\\
        \hline
	    $\vec{0}$ & $\myvec{0\\0}$
 & Origin\\
        \hline
    \end{tabular}
    \caption{Parameters}
    \label{tab:my_label}
\end{table}


\section{Solution}
Given that resultant line passes through origin and mid point of the line segment joining point P(0,-4) and B(8,0) \\
\\
\\
given ${\vec{P}}$=$\myvec{
  0\\
  -4}$
 , ${\vec{B}}$=$\myvec{
  8\\
  0}$
  
 \fi 
The mid point of $PB$ is
\begin{align}
\vec{M} &=\frac{1}{2}(\vec{P}+\vec{B})
	= \myvec{4 \\ -2}  
\end{align}
The direction vector of line joining $\vec{O}, \vec{M}$ is 
\begin{align}
\vec{m}&=\vec{O}-\vec{M}
 = -\vec{M}
\end{align}
which can be expressed as
\begin{align}
	\myvec{1 \\ -\frac{1}{2}}
\end{align}
Thus the slope is
\begin{align}
	m = -\frac{1}{2}
\end{align}
\iffalse
\textbf{The direction vector of a line expressed as}
\begin{align}
\implies\vec{m} &= \begin{pmatrix}1 \\ m \\ \end{pmatrix}
\end{align}

\textbf{By solving equation (5) and (6),we get the slope of $\vec{O}$ $\vec{M}$ line}
\begin{align}
        \boxed{m=-0.5}
 \end{align}

\section{Software}
Download the following code using,
\begin{table}[h]
    \centering
    \begin{tabular}{|c|}
    \hline \\
   https://github.com/jyothsna777/jyothsna-fwc.git  \\
         \\
\hline
    \end{tabular}
\end{table}
\\
and execute the code by using command
\begin{center}
\textbf{Python3 lines.py}\\
\end{center}

\section{Conclusion}
Hence the slope of line $\vec{O}$ $\vec{M}$ lineis $\vec{m}$=-0.5

\end{document}
\fi

\item Consider the following population and year graph, Find the slope of the line AB and using it, find what will be the population in the year 2010?
\\
\begin{figure}[ht]
\centering
\includegraphics[width = \columnwidth]{chapters/11/10/1/14/figs/fig.png}
\caption{}
\label{fig:chapters/11/10/1/14/1}
\end{figure}
\solution
\iffalse
\documentclass[journal,10pt,twocolumn]{article}
\usepackage{graphicx}
\usepackage[margin=0.5in]{geometry}
\usepackage[cmex10]{amsmath}
\usepackage{array}
\usepackage{booktabs}
\usepackage{listings}
\title{\textbf{Line Assignment}}
\author{Bhavani Kanike}
\date{October 2022}

\providecommand{\norm}[1]{\left\lVert#1\right\rVert}
\providecommand{\abs}[1]{\left\vert#1\right\vert}
\let\vec\mathbf
\newcommand{\myvec}[1]{\ensuremath{\begin{pmatrix}#1\end{pmatrix}}}
\newcommand{\mydet}[1]{\ensuremath{\begin{vmatrix}#1\end{vmatrix}}}
\providecommand{\brak}[1]{\ensuremath{\left(#1\right)}}

\begin{document}

\maketitle
\paragraph{\textit{Problem Statement} 
\fi
ABCD is a quadrilateral in which $\vec{P}, \vec{Q}, \vec{R}$ and $\vec{S}$ are mid-points of the sides AB, BC, CD and DA (see Fig \ref{fig:9/8/2/1}). AC is a diagonal. 
		
Show that 
\begin{enumerate}
	\item $SR \parallel AC$ and $SR =\frac{1}{2} AC$
\item $PQ = SR$
\item $PQRS$ is a parallelogram.
\end{enumerate}
 	\begin{figure}
		\centering
 \includegraphics[width=\columnwidth]{chapters/9/8/2/1/figs/line1.pdf}
		\caption{}
		\label{fig:9/8/2/1}
  	\end{figure}
	\solution 
	Using 
	  \eqref{eq:section_formula},
	\begin{align}
		\label{eq:9/8/2/1}
		\begin{split}
		\vec{P} &= \frac{\vec{A}+\vec{B}}{2}\\
 \vec{Q} &= \frac{\vec{C}+\vec{B}}{2}\\
 \vec{R} &= \frac{\vec{C}+\vec{D}}{2}\\
 \vec{S} &= \frac{\vec{D}+\vec{A}}{2}
		\end{split}
	\end{align}
\begin{enumerate}
	\item
	Consequently, 
	\begin{align}
\vec{R}
		-\vec{S} &= \frac{\vec{C}-\vec{A}}{2}
		\\
		\implies SR &\parallel AC
	\end{align}
	Also, 
	\begin{align}
		\norm{\vec{R}
		-\vec{S}} &= \frac{\norm{\vec{C}-\vec{A}}}{2}
		\\
		\implies SR &= \frac{1}{2}AC
	\end{align}
\item 	From 
		\eqref{eq:9/8/2/1},
	\begin{align}
\vec{R}
		-\vec{S} = \vec{Q}-\vec{P}
	\end{align}
	which means that $PQRS$ is a parallelogram and $PQ = SR$.
\end{enumerate}
%
\iffalse
\begin{figure}[h]
\centering
\includegraphics[width=1\columnwidth]
\caption{Figure}
\label{fig:triangle}
\end{figure}

\section*{Solution}

$\boldsymbol Given :$  ABCD is a Quadrilateral P,Q,R and S are the midpoints of line AB,BC,CD,DA.We can obtain the points P,Q,R and S from A,B,C and D and are given by\\\\
\boldmath
\unboldmath
(3) To prove that PQRS is a parallelogram we need to prove  PQ // SR
To prove SR $\parallel$ PQ\\
Direction vector of line SR  $\boldsymbol {(R-S) =  \frac{(C-A)}{2}}$\\\\
Direction vector of line PQ  $\boldsymbol {(Q-P)= \frac{(C-A)}{2}}$\\\\
\begin{equation}
	\boldsymbol {(R-S) = (Q-P) = \frac{(C-A)}{2}}\\
\end{equation}
Since the direction vectors of line SR and PQ are in same direction\\\\
$SR \parallel PQ$\\
Therefore,
$\boldsymbol{ PQRS }$ is a parallelogram\\\\

	
(1)  Directional vector of line SR  = $\boldsymbol {(R-S)}$ = $\frac{\boldsymbol{(C-A)}}{2} $\\
Directional vector of line AC  = $\boldsymbol {(C-A)}$\\

It is observed that the constant k is $\frac{1}{2}$

Therefore
\begin{equation}
	SR \parallel AC
\end{equation} 

and from equation 1 
\begin{equation}
	\boldsymbol {SR = \frac{1}{2}AC}    
\end{equation}\\


(2)   To prove PQ = SR\\ 
		From euqation 1\\\\
\begin{equation}
		\boldsymbol{ (Q-P) = (R-S) = \frac{(C-A)}{2}}
\end{equation}
	 



\section{Execution}
The below python code realizes the construction:
\begin{lstlisting}
https://github.com/bhavani360/FWC_assignments
\end{lstlisting}
	
\section*{Construction}
The dimensions of the Quadrilateral ABCD are taken as below\\
{
\setlength\extrarowheight{2pt}
\centering
	\begin{tabular}{|c|c|}
	\hline
	\textbf{symbol}&\textbf{value}\\
	\hline
	r&8\\
	\hline
	$\theta$&pi/2.5\\
	\hline
	d&7\\
	\hline
	A&(0,0)\\
	\hline
	B&(d,0)\\
	\hline
	D&(rcos$\theta$,rsin$\theta$)\\
	\hline
	C&(D/1.5)+B\\
	\hline
\end{tabular}
}
\end{document}
\fi

\item Find a vector of magnitude 5 units, and parallel to the resultant of the vectors $\vec{a}=2\hat{i}+3\hat{j}-\hat{k}$ and $\vec{b}=\hat{i}-2\hat{j}+\hat{k}$.\\

\item Let $\vec{a}$ and $\vec{b}$ be two unit vectors and $\theta$ is the angle between them. Then $\vec{a}+\vec{b}$ is a unit vector if
		\begin{tasks}(4)
			\task $\theta = \frac{\pi}{4}$
			\task $\theta = \frac{\pi}{3}$
			\task $\theta = \frac{\pi}{2}$
			\task $\theta = \frac{2\pi}{3}$
			\end{tasks}
\solution
Given,
\begin{align}
	\norm{\vec{a}}=\norm{\vec{b}}=1\label{eq:12/10/5/17/1}
	\\
	\norm{\vec{a}+\vec{b}}=1\label{eq:12/10/5/17/2}
\end{align}
Squaring both sides of \eqref{eq:12/10/5/17/2}  , we get
\begin{align}
	\norm{\vec{a}+\vec{b}}^2=1^2
\\	
	\implies \norm{\vec{a}}^2 + \norm{\vec{b}}^2 + 2\vec{a}^{\top}\vec{b} = 1\label{eq:12/10/5/17/3}	
\end{align}
Substituting \eqref{eq:12/10/5/17/1} in \eqref{eq:12/10/5/17/3}, we get
\\
\begin{align}
	\implies 1+1+2(\norm{\vec{a}}\norm{\vec{b}}\cos{\theta})=1
	\\
	\implies 2+2(\norm{\vec{a}}\norm{\vec{b}}\cos{\theta})=1
        \\
	\implies 2(\norm{\vec{a}}\norm{\vec{b}}\cos{\theta})=-1
	\\
	\implies (\norm{\vec{a}}\norm{\vec{b}}\cos{\theta})=\frac{-1}{2}\label{eq:12/10/5/17/4}
\end{align}
Subtituting \eqref{eq:12/10/5/17/1} in \eqref{eq:12/10/5/17/4}, we get
\begin{align}
	\implies \cos{\theta}=\frac{-1}{2}
	\\
	\implies \theta=\frac{2\pi}{3}
\end{align}

\end{enumerate}

\subsection{Exercises}
\begin{enumerate}[label=\thesection.\arabic*,ref=\thesection.\theenumi]
\item The fourth vertex $\vec{D}$ of a parallelogram $\vec{ABCD}$ whose three vertices are
	$\vec{A} (–2, 3), \vec{B} (6, 7)\text { and } \vec{C} (8, 3)$ is
\begin{enumerate}
	\item $(0, 1)$
	\item $(0, –1)$
	\item $ (–1,0)$
	\item$(1, 0)$
\end{enumerate}
\item Points $\vec{A}(4,3), \vec{B}(6,4),{c}(5,-6) \text{ and } \vec{D}(-3,5)$ are the vertices of a parallelogram  
\end{enumerate}

\subsection{Triangle}
\begin{enumerate}[label=\thesection.\arabic*,ref=\thesection.\theenumi]
\numberwithin{equation}{enumi}
\numberwithin{figure}{enumi}
\numberwithin{table}{enumi}
\item Construct a triangle $ABC$ in which $BC=7cm, \angle{B}=75\degree$ and $AB + AC = 13 cm$.
\label{chapters/9/11/2/1}
%\iffalse
\documentclass[journal,10pt,twocolumn]{article}
\usepackage{graphicx}
\usepackage[margin=0.5in]{geometry}
\usepackage[cmex10]{amsmath}
\usepackage{array}
\usepackage{booktabs}
\usepackage{listings}
\title{\textbf{Line Assignment}}
\author{Bhavani Kanike}
\date{October 2022}

\providecommand{\norm}[1]{\left\lVert#1\right\rVert}
\providecommand{\abs}[1]{\left\vert#1\right\vert}
\let\vec\mathbf
\newcommand{\myvec}[1]{\ensuremath{\begin{pmatrix}#1\end{pmatrix}}}
\newcommand{\mydet}[1]{\ensuremath{\begin{vmatrix}#1\end{vmatrix}}}
\providecommand{\brak}[1]{\ensuremath{\left(#1\right)}}

\begin{document}

\maketitle
\paragraph{\textit{Problem Statement} 
\fi
ABCD is a quadrilateral in which $\vec{P}, \vec{Q}, \vec{R}$ and $\vec{S}$ are mid-points of the sides AB, BC, CD and DA (see Fig \ref{fig:9/8/2/1}). AC is a diagonal. 
		
Show that 
\begin{enumerate}
	\item $SR \parallel AC$ and $SR =\frac{1}{2} AC$
\item $PQ = SR$
\item $PQRS$ is a parallelogram.
\end{enumerate}
 	\begin{figure}
		\centering
 \includegraphics[width=\columnwidth]{chapters/9/8/2/1/figs/line1.pdf}
		\caption{}
		\label{fig:9/8/2/1}
  	\end{figure}
	\solution 
	Using 
	  \eqref{eq:section_formula},
	\begin{align}
		\label{eq:9/8/2/1}
		\begin{split}
		\vec{P} &= \frac{\vec{A}+\vec{B}}{2}\\
 \vec{Q} &= \frac{\vec{C}+\vec{B}}{2}\\
 \vec{R} &= \frac{\vec{C}+\vec{D}}{2}\\
 \vec{S} &= \frac{\vec{D}+\vec{A}}{2}
		\end{split}
	\end{align}
\begin{enumerate}
	\item
	Consequently, 
	\begin{align}
\vec{R}
		-\vec{S} &= \frac{\vec{C}-\vec{A}}{2}
		\\
		\implies SR &\parallel AC
	\end{align}
	Also, 
	\begin{align}
		\norm{\vec{R}
		-\vec{S}} &= \frac{\norm{\vec{C}-\vec{A}}}{2}
		\\
		\implies SR &= \frac{1}{2}AC
	\end{align}
\item 	From 
		\eqref{eq:9/8/2/1},
	\begin{align}
\vec{R}
		-\vec{S} = \vec{Q}-\vec{P}
	\end{align}
	which means that $PQRS$ is a parallelogram and $PQ = SR$.
\end{enumerate}
%
\iffalse
\begin{figure}[h]
\centering
\includegraphics[width=1\columnwidth]
\caption{Figure}
\label{fig:triangle}
\end{figure}

\section*{Solution}

$\boldsymbol Given :$  ABCD is a Quadrilateral P,Q,R and S are the midpoints of line AB,BC,CD,DA.We can obtain the points P,Q,R and S from A,B,C and D and are given by\\\\
\boldmath
\unboldmath
(3) To prove that PQRS is a parallelogram we need to prove  PQ // SR
To prove SR $\parallel$ PQ\\
Direction vector of line SR  $\boldsymbol {(R-S) =  \frac{(C-A)}{2}}$\\\\
Direction vector of line PQ  $\boldsymbol {(Q-P)= \frac{(C-A)}{2}}$\\\\
\begin{equation}
	\boldsymbol {(R-S) = (Q-P) = \frac{(C-A)}{2}}\\
\end{equation}
Since the direction vectors of line SR and PQ are in same direction\\\\
$SR \parallel PQ$\\
Therefore,
$\boldsymbol{ PQRS }$ is a parallelogram\\\\

	
(1)  Directional vector of line SR  = $\boldsymbol {(R-S)}$ = $\frac{\boldsymbol{(C-A)}}{2} $\\
Directional vector of line AC  = $\boldsymbol {(C-A)}$\\

It is observed that the constant k is $\frac{1}{2}$

Therefore
\begin{equation}
	SR \parallel AC
\end{equation} 

and from equation 1 
\begin{equation}
	\boldsymbol {SR = \frac{1}{2}AC}    
\end{equation}\\


(2)   To prove PQ = SR\\ 
		From euqation 1\\\\
\begin{equation}
		\boldsymbol{ (Q-P) = (R-S) = \frac{(C-A)}{2}}
\end{equation}
	 



\section{Execution}
The below python code realizes the construction:
\begin{lstlisting}
https://github.com/bhavani360/FWC_assignments
\end{lstlisting}
	
\section*{Construction}
The dimensions of the Quadrilateral ABCD are taken as below\\
{
\setlength\extrarowheight{2pt}
\centering
	\begin{tabular}{|c|c|}
	\hline
	\textbf{symbol}&\textbf{value}\\
	\hline
	r&8\\
	\hline
	$\theta$&pi/2.5\\
	\hline
	d&7\\
	\hline
	A&(0,0)\\
	\hline
	B&(d,0)\\
	\hline
	D&(rcos$\theta$,rsin$\theta$)\\
	\hline
	C&(D/1.5)+B\\
	\hline
\end{tabular}
}
\end{document}
\fi

%
\item Construct a triangle $ABC$ in which $BC=8cm, \angle{B}=45\degree$ and $AB - AC = 3.5 cm$.
\label{chapters/9/11/2/2}
\\
\solution
\input{chapters/9/11/2/2-new/vector-2.tex}
%
\item Construct a triangle $PQR$ in which $QR=6cm, \angle{Q}=60\degree$ and $PR - PQ = 2cm$.
\label{chapters/9/11/2/3}
%\iffalse
\documentclass[journal,10pt,twocolumn]{article}
\usepackage{graphicx}
\usepackage[margin=0.5in]{geometry}
\usepackage[cmex10]{amsmath}
\usepackage{array}
\usepackage{booktabs}
\usepackage{listings}
\title{\textbf{Line Assignment}}
\author{Bhavani Kanike}
\date{October 2022}

\providecommand{\norm}[1]{\left\lVert#1\right\rVert}
\providecommand{\abs}[1]{\left\vert#1\right\vert}
\let\vec\mathbf
\newcommand{\myvec}[1]{\ensuremath{\begin{pmatrix}#1\end{pmatrix}}}
\newcommand{\mydet}[1]{\ensuremath{\begin{vmatrix}#1\end{vmatrix}}}
\providecommand{\brak}[1]{\ensuremath{\left(#1\right)}}

\begin{document}

\maketitle
\paragraph{\textit{Problem Statement} 
\fi
ABCD is a quadrilateral in which $\vec{P}, \vec{Q}, \vec{R}$ and $\vec{S}$ are mid-points of the sides AB, BC, CD and DA (see Fig \ref{fig:9/8/2/1}). AC is a diagonal. 
		
Show that 
\begin{enumerate}
	\item $SR \parallel AC$ and $SR =\frac{1}{2} AC$
\item $PQ = SR$
\item $PQRS$ is a parallelogram.
\end{enumerate}
 	\begin{figure}
		\centering
 \includegraphics[width=\columnwidth]{chapters/9/8/2/1/figs/line1.pdf}
		\caption{}
		\label{fig:9/8/2/1}
  	\end{figure}
	\solution 
	Using 
	  \eqref{eq:section_formula},
	\begin{align}
		\label{eq:9/8/2/1}
		\begin{split}
		\vec{P} &= \frac{\vec{A}+\vec{B}}{2}\\
 \vec{Q} &= \frac{\vec{C}+\vec{B}}{2}\\
 \vec{R} &= \frac{\vec{C}+\vec{D}}{2}\\
 \vec{S} &= \frac{\vec{D}+\vec{A}}{2}
		\end{split}
	\end{align}
\begin{enumerate}
	\item
	Consequently, 
	\begin{align}
\vec{R}
		-\vec{S} &= \frac{\vec{C}-\vec{A}}{2}
		\\
		\implies SR &\parallel AC
	\end{align}
	Also, 
	\begin{align}
		\norm{\vec{R}
		-\vec{S}} &= \frac{\norm{\vec{C}-\vec{A}}}{2}
		\\
		\implies SR &= \frac{1}{2}AC
	\end{align}
\item 	From 
		\eqref{eq:9/8/2/1},
	\begin{align}
\vec{R}
		-\vec{S} = \vec{Q}-\vec{P}
	\end{align}
	which means that $PQRS$ is a parallelogram and $PQ = SR$.
\end{enumerate}
%
\iffalse
\begin{figure}[h]
\centering
\includegraphics[width=1\columnwidth]
\caption{Figure}
\label{fig:triangle}
\end{figure}

\section*{Solution}

$\boldsymbol Given :$  ABCD is a Quadrilateral P,Q,R and S are the midpoints of line AB,BC,CD,DA.We can obtain the points P,Q,R and S from A,B,C and D and are given by\\\\
\boldmath
\unboldmath
(3) To prove that PQRS is a parallelogram we need to prove  PQ // SR
To prove SR $\parallel$ PQ\\
Direction vector of line SR  $\boldsymbol {(R-S) =  \frac{(C-A)}{2}}$\\\\
Direction vector of line PQ  $\boldsymbol {(Q-P)= \frac{(C-A)}{2}}$\\\\
\begin{equation}
	\boldsymbol {(R-S) = (Q-P) = \frac{(C-A)}{2}}\\
\end{equation}
Since the direction vectors of line SR and PQ are in same direction\\\\
$SR \parallel PQ$\\
Therefore,
$\boldsymbol{ PQRS }$ is a parallelogram\\\\

	
(1)  Directional vector of line SR  = $\boldsymbol {(R-S)}$ = $\frac{\boldsymbol{(C-A)}}{2} $\\
Directional vector of line AC  = $\boldsymbol {(C-A)}$\\

It is observed that the constant k is $\frac{1}{2}$

Therefore
\begin{equation}
	SR \parallel AC
\end{equation} 

and from equation 1 
\begin{equation}
	\boldsymbol {SR = \frac{1}{2}AC}    
\end{equation}\\


(2)   To prove PQ = SR\\ 
		From euqation 1\\\\
\begin{equation}
		\boldsymbol{ (Q-P) = (R-S) = \frac{(C-A)}{2}}
\end{equation}
	 



\section{Execution}
The below python code realizes the construction:
\begin{lstlisting}
https://github.com/bhavani360/FWC_assignments
\end{lstlisting}
	
\section*{Construction}
The dimensions of the Quadrilateral ABCD are taken as below\\
{
\setlength\extrarowheight{2pt}
\centering
	\begin{tabular}{|c|c|}
	\hline
	\textbf{symbol}&\textbf{value}\\
	\hline
	r&8\\
	\hline
	$\theta$&pi/2.5\\
	\hline
	d&7\\
	\hline
	A&(0,0)\\
	\hline
	B&(d,0)\\
	\hline
	D&(rcos$\theta$,rsin$\theta$)\\
	\hline
	C&(D/1.5)+B\\
	\hline
\end{tabular}
}
\end{document}
\fi

%
\item Construct a triangle $XYZ$ in which $\angle{Y}=30\degree, \angle{Z}=90\degree$ and  $XY+YZ+ZX=11cm$.
\label{chapters/9/11/2/4}
%\iffalse
\documentclass[journal,10pt,twocolumn]{article}
\usepackage{graphicx}
\usepackage[margin=0.5in]{geometry}
\usepackage[cmex10]{amsmath}
\usepackage{array}
\usepackage{booktabs}
\usepackage{listings}
\title{\textbf{Line Assignment}}
\author{Bhavani Kanike}
\date{October 2022}

\providecommand{\norm}[1]{\left\lVert#1\right\rVert}
\providecommand{\abs}[1]{\left\vert#1\right\vert}
\let\vec\mathbf
\newcommand{\myvec}[1]{\ensuremath{\begin{pmatrix}#1\end{pmatrix}}}
\newcommand{\mydet}[1]{\ensuremath{\begin{vmatrix}#1\end{vmatrix}}}
\providecommand{\brak}[1]{\ensuremath{\left(#1\right)}}

\begin{document}

\maketitle
\paragraph{\textit{Problem Statement} 
\fi
ABCD is a quadrilateral in which $\vec{P}, \vec{Q}, \vec{R}$ and $\vec{S}$ are mid-points of the sides AB, BC, CD and DA (see Fig \ref{fig:9/8/2/1}). AC is a diagonal. 
		
Show that 
\begin{enumerate}
	\item $SR \parallel AC$ and $SR =\frac{1}{2} AC$
\item $PQ = SR$
\item $PQRS$ is a parallelogram.
\end{enumerate}
 	\begin{figure}
		\centering
 \includegraphics[width=\columnwidth]{chapters/9/8/2/1/figs/line1.pdf}
		\caption{}
		\label{fig:9/8/2/1}
  	\end{figure}
	\solution 
	Using 
	  \eqref{eq:section_formula},
	\begin{align}
		\label{eq:9/8/2/1}
		\begin{split}
		\vec{P} &= \frac{\vec{A}+\vec{B}}{2}\\
 \vec{Q} &= \frac{\vec{C}+\vec{B}}{2}\\
 \vec{R} &= \frac{\vec{C}+\vec{D}}{2}\\
 \vec{S} &= \frac{\vec{D}+\vec{A}}{2}
		\end{split}
	\end{align}
\begin{enumerate}
	\item
	Consequently, 
	\begin{align}
\vec{R}
		-\vec{S} &= \frac{\vec{C}-\vec{A}}{2}
		\\
		\implies SR &\parallel AC
	\end{align}
	Also, 
	\begin{align}
		\norm{\vec{R}
		-\vec{S}} &= \frac{\norm{\vec{C}-\vec{A}}}{2}
		\\
		\implies SR &= \frac{1}{2}AC
	\end{align}
\item 	From 
		\eqref{eq:9/8/2/1},
	\begin{align}
\vec{R}
		-\vec{S} = \vec{Q}-\vec{P}
	\end{align}
	which means that $PQRS$ is a parallelogram and $PQ = SR$.
\end{enumerate}
%
\iffalse
\begin{figure}[h]
\centering
\includegraphics[width=1\columnwidth]
\caption{Figure}
\label{fig:triangle}
\end{figure}

\section*{Solution}

$\boldsymbol Given :$  ABCD is a Quadrilateral P,Q,R and S are the midpoints of line AB,BC,CD,DA.We can obtain the points P,Q,R and S from A,B,C and D and are given by\\\\
\boldmath
\unboldmath
(3) To prove that PQRS is a parallelogram we need to prove  PQ // SR
To prove SR $\parallel$ PQ\\
Direction vector of line SR  $\boldsymbol {(R-S) =  \frac{(C-A)}{2}}$\\\\
Direction vector of line PQ  $\boldsymbol {(Q-P)= \frac{(C-A)}{2}}$\\\\
\begin{equation}
	\boldsymbol {(R-S) = (Q-P) = \frac{(C-A)}{2}}\\
\end{equation}
Since the direction vectors of line SR and PQ are in same direction\\\\
$SR \parallel PQ$\\
Therefore,
$\boldsymbol{ PQRS }$ is a parallelogram\\\\

	
(1)  Directional vector of line SR  = $\boldsymbol {(R-S)}$ = $\frac{\boldsymbol{(C-A)}}{2} $\\
Directional vector of line AC  = $\boldsymbol {(C-A)}$\\

It is observed that the constant k is $\frac{1}{2}$

Therefore
\begin{equation}
	SR \parallel AC
\end{equation} 

and from equation 1 
\begin{equation}
	\boldsymbol {SR = \frac{1}{2}AC}    
\end{equation}\\


(2)   To prove PQ = SR\\ 
		From euqation 1\\\\
\begin{equation}
		\boldsymbol{ (Q-P) = (R-S) = \frac{(C-A)}{2}}
\end{equation}
	 



\section{Execution}
The below python code realizes the construction:
\begin{lstlisting}
https://github.com/bhavani360/FWC_assignments
\end{lstlisting}
	
\section*{Construction}
The dimensions of the Quadrilateral ABCD are taken as below\\
{
\setlength\extrarowheight{2pt}
\centering
	\begin{tabular}{|c|c|}
	\hline
	\textbf{symbol}&\textbf{value}\\
	\hline
	r&8\\
	\hline
	$\theta$&pi/2.5\\
	\hline
	d&7\\
	\hline
	A&(0,0)\\
	\hline
	B&(d,0)\\
	\hline
	D&(rcos$\theta$,rsin$\theta$)\\
	\hline
	C&(D/1.5)+B\\
	\hline
\end{tabular}
}
\end{document}
\fi

%
\item Construct a right triangle whose base is 12cm and sum of its hypotenuse and other side is 18cm.
\label{chapters/9/11/2/5}
%\iffalse
\documentclass[journal,10pt,twocolumn]{article}
\usepackage{graphicx}
\usepackage[margin=0.5in]{geometry}
\usepackage[cmex10]{amsmath}
\usepackage{array}
\usepackage{booktabs}
\usepackage{listings}
\title{\textbf{Line Assignment}}
\author{Bhavani Kanike}
\date{October 2022}

\providecommand{\norm}[1]{\left\lVert#1\right\rVert}
\providecommand{\abs}[1]{\left\vert#1\right\vert}
\let\vec\mathbf
\newcommand{\myvec}[1]{\ensuremath{\begin{pmatrix}#1\end{pmatrix}}}
\newcommand{\mydet}[1]{\ensuremath{\begin{vmatrix}#1\end{vmatrix}}}
\providecommand{\brak}[1]{\ensuremath{\left(#1\right)}}

\begin{document}

\maketitle
\paragraph{\textit{Problem Statement} 
\fi
ABCD is a quadrilateral in which $\vec{P}, \vec{Q}, \vec{R}$ and $\vec{S}$ are mid-points of the sides AB, BC, CD and DA (see Fig \ref{fig:9/8/2/1}). AC is a diagonal. 
		
Show that 
\begin{enumerate}
	\item $SR \parallel AC$ and $SR =\frac{1}{2} AC$
\item $PQ = SR$
\item $PQRS$ is a parallelogram.
\end{enumerate}
 	\begin{figure}
		\centering
 \includegraphics[width=\columnwidth]{chapters/9/8/2/1/figs/line1.pdf}
		\caption{}
		\label{fig:9/8/2/1}
  	\end{figure}
	\solution 
	Using 
	  \eqref{eq:section_formula},
	\begin{align}
		\label{eq:9/8/2/1}
		\begin{split}
		\vec{P} &= \frac{\vec{A}+\vec{B}}{2}\\
 \vec{Q} &= \frac{\vec{C}+\vec{B}}{2}\\
 \vec{R} &= \frac{\vec{C}+\vec{D}}{2}\\
 \vec{S} &= \frac{\vec{D}+\vec{A}}{2}
		\end{split}
	\end{align}
\begin{enumerate}
	\item
	Consequently, 
	\begin{align}
\vec{R}
		-\vec{S} &= \frac{\vec{C}-\vec{A}}{2}
		\\
		\implies SR &\parallel AC
	\end{align}
	Also, 
	\begin{align}
		\norm{\vec{R}
		-\vec{S}} &= \frac{\norm{\vec{C}-\vec{A}}}{2}
		\\
		\implies SR &= \frac{1}{2}AC
	\end{align}
\item 	From 
		\eqref{eq:9/8/2/1},
	\begin{align}
\vec{R}
		-\vec{S} = \vec{Q}-\vec{P}
	\end{align}
	which means that $PQRS$ is a parallelogram and $PQ = SR$.
\end{enumerate}
%
\iffalse
\begin{figure}[h]
\centering
\includegraphics[width=1\columnwidth]
\caption{Figure}
\label{fig:triangle}
\end{figure}

\section*{Solution}

$\boldsymbol Given :$  ABCD is a Quadrilateral P,Q,R and S are the midpoints of line AB,BC,CD,DA.We can obtain the points P,Q,R and S from A,B,C and D and are given by\\\\
\boldmath
\unboldmath
(3) To prove that PQRS is a parallelogram we need to prove  PQ // SR
To prove SR $\parallel$ PQ\\
Direction vector of line SR  $\boldsymbol {(R-S) =  \frac{(C-A)}{2}}$\\\\
Direction vector of line PQ  $\boldsymbol {(Q-P)= \frac{(C-A)}{2}}$\\\\
\begin{equation}
	\boldsymbol {(R-S) = (Q-P) = \frac{(C-A)}{2}}\\
\end{equation}
Since the direction vectors of line SR and PQ are in same direction\\\\
$SR \parallel PQ$\\
Therefore,
$\boldsymbol{ PQRS }$ is a parallelogram\\\\

	
(1)  Directional vector of line SR  = $\boldsymbol {(R-S)}$ = $\frac{\boldsymbol{(C-A)}}{2} $\\
Directional vector of line AC  = $\boldsymbol {(C-A)}$\\

It is observed that the constant k is $\frac{1}{2}$

Therefore
\begin{equation}
	SR \parallel AC
\end{equation} 

and from equation 1 
\begin{equation}
	\boldsymbol {SR = \frac{1}{2}AC}    
\end{equation}\\


(2)   To prove PQ = SR\\ 
		From euqation 1\\\\
\begin{equation}
		\boldsymbol{ (Q-P) = (R-S) = \frac{(C-A)}{2}}
\end{equation}
	 



\section{Execution}
The below python code realizes the construction:
\begin{lstlisting}
https://github.com/bhavani360/FWC_assignments
\end{lstlisting}
	
\section*{Construction}
The dimensions of the Quadrilateral ABCD are taken as below\\
{
\setlength\extrarowheight{2pt}
\centering
	\begin{tabular}{|c|c|}
	\hline
	\textbf{symbol}&\textbf{value}\\
	\hline
	r&8\\
	\hline
	$\theta$&pi/2.5\\
	\hline
	d&7\\
	\hline
	A&(0,0)\\
	\hline
	B&(d,0)\\
	\hline
	D&(rcos$\theta$,rsin$\theta$)\\
	\hline
	C&(D/1.5)+B\\
	\hline
\end{tabular}
}
\end{document}
\fi

%
\item In Fig. \ref{fig:chapters/9/7/1/6/1}, $AC=AE,AB=AD$ and $\angle BAD=\angle EAC$. Show that $BC=DE$.
\label{chapters/9/7/1/6}
\begin{figure}[!h]
	\begin{center}
	\includegraphics[width=\columnwidth]{chapters/9/7/1/6/figs/fig.pdf}
	\end{center}
\caption{}
\label{fig:chapters/9/7/1/6/1}
\end{figure}
\\
\solution
\iffalse
\documentclass[10pt]{article}
\usepackage{graphicx}
\def\inputGnumericTable{}
\usepackage[latin1]{inputenc}
\usepackage{fullpage}
\usepackage{color}
\usepackage{array}
\usepackage{longtable}
\usepackage{calc}
\usepackage{multirow}
\usepackage{hhline}
\usepackage{ifthen}
\usepackage{amsmath}
\usepackage[none]{hyphenat}
\usepackage{listings}
\usepackage[english]{babel}
\usepackage{siunitx}
\usepackage{caption}
\usepackage{booktabs}
\usepackage{array}
\usepackage{extarrows}
\usepackage{enumerate}
\usepackage{enumitem}
\usepackage{amsmath}
\usepackage{commath}
\usepackage{gensymb}
\usepackage{amssymb}
\usepackage{multicol}
%\usepackage[utf8]{inputenc}
\lstset{
 frame=single,
 breaklines=true
}
\usepackage{hyperref}
\usepackage[margin=0.5in]{geometry}	 
%\usepackage{exsheets}% also loads the `tasks' package
\usepackage{atbegshi}
\AtBeginDocument{\AtBeginShipoutNext{\AtBeginShipoutDiscard}}

%new macro definitions
\renewcommand{\labelenumi}{(\roman{enumi})}
\newcommand{\mydet}[1]{\ensuremath{\begin{vmatrix}#1\end{vmatrix}}}
\providecommand{\brak}[1]{\ensuremath{\left(#1\right)}}
\newcommand{\solution}{\noindent \textbf{Solution: }}
\newcommand{\myvec}[1]{\ensuremath{\begin{pmatrix}#1\end{pmatrix}}}
\newenvironment{amatrix}[1]{%
	\left(\begin{array}{@{}*{#1}{c}|c@{}}
}{%
	\end{array}\right)
}

\newcommand{\myaugvec}[2]{\ensuremath{\begin{amatrix}{#1}#2\end{amatrix}}}
\providecommand{\norm}[1]{\left\1Vert#1\right\rVert}
\let\vec\mathbf{}


%\SetEnumitemKey{twocol}{
% before=\raggedcolumns\begin{multicols}{2},
% after=\end{multicols}}
%\SetEnumitemKey{fourcol}{
% before=\raggedcolumns\begin{multicols}{4},
% after=\end{multicols}} 


\begin{document}
\begin{center}
\title{\textbf{TRIANGLES}}
\date{\vspace{-5ex}}
\maketitle
\end{center}
\section*{9$^{th}$Math - Chapter 7}
\section*{Problem}
\section*{Construction}
\solution
\textbf{Given:}
\fi
The input parameters are available in Table 
\ref{tab:chapters/9/7/1/6/1}.
\begin{table}[!h]
\centering
\begin{tabular}{|c|c|p{5cm}|}
\hline
\textbf{Symbol} & \textbf{Value} & \textbf{Description} \\
\hline
$\theta$ & $30\degree$ & $\angle{BAP} = \angle{BAQ}$ \\
\hline
$a$ & $9$ & $AB$ \\
\hline
$c$ & $8$ & $AQ$ \\
\hline
$\vec{e}_1$ & $\myvec{1\\0}$ & Basis vector \\
\hline
\end{tabular}

\label{tab:chapters/9/7/1/6/1}
\end{table}
The vertices of $\triangle ABC$ are given by 
\begin{align}
	\vec{A}=c\myvec{\cos\frac{\theta}{2}\\\sin\frac{\theta}{2}},\,
\vec{B}=\myvec{0\\0},\,
\vec{C}=a\vec{e}_1,
\end{align}
and 
\begin{align}
\vec{D}=\brak{2c\sin\frac{\theta}{2}}\vec{e_1}
\end{align}
Then, using the cosine formula and  the fact that $\triangle AEC$ is iscosceles,
\begin{align}
AC = b=\sqrt{a^2+c^2-2ac\cos\theta},\, EC = 2b\sin\frac{\theta}{2}
\end{align}
Also, 
\begin{align}
\angle BCA=\cos^{-1}\brak{\frac{a^2+b^2-c^2}{2ab}},\,
\angle ACE=90\degree-\frac{\theta}{2}
\end{align}
Let $\phi$ be the angle made by the vector 	$EC$ with  the $x$-axis.  Then,  
\begin{align}
\phi=180\degree-\brak{\angle BCA+\angle ACE} = =90\degree-\frac{\theta}{2}\cos^{-1}\brak{\frac{a^2+b^2-c^2}{2ab}}
\end{align}
Consequently, 
Using vector addition,
\begin{align}
	\vec{E}-\vec{C}&=\brak{2b\sin\frac{\theta}{2}}\myvec{\cos\phi\\\sin\phi}\\
	\implies \vec{E}&=\vec{C}+2b\sin\frac{\theta}{2}\myvec{\cos\phi\\\sin\phi}\\
\end{align}
Substituting numerical values, 
\begin{align}
	\norm{\vec{B}-\vec{C}}
	=\norm{\vec{D}-\vec{E}}
\label{eq:chapters/9/7/1/6/5}
\end{align}

\item 
	$AB$ is a line segment and $\vec{P}$ is its mid-point. $\vec{D}$ and $\vec{E}$ are points on the same side of
$AB$ such that $\angle BAD = \angle ABE$ and $\angle EPA = \angle DPB$. Show that
\label{chapters/9/7/1/7}
\begin{enumerate}
\item $\triangle DAP \cong \triangle EBP$
\item $AD = BE$.
\end{enumerate}
\solution
\iffalse
\documentclass[10pt]{article}
\usepackage{graphicx}
\def\inputGnumericTable{}
\usepackage[latin1]{inputenc}
\usepackage{fullpage}
\usepackage{color}
\usepackage{array}
\usepackage{longtable}
\usepackage{calc}
\usepackage{multirow}
\usepackage{hhline}
\usepackage{ifthen}
\usepackage{amsmath}
\usepackage[none]{hyphenat}
\usepackage{listings}
\usepackage[english]{babel}
\usepackage{siunitx}
\usepackage{caption}
\usepackage{booktabs}
\usepackage{array}
\usepackage{extarrows}
\usepackage{enumerate}
\usepackage{enumitem}
\usepackage{amsmath}
\usepackage{commath}
\usepackage{gensymb}
\usepackage{amssymb}
\usepackage{multicol}
%\usepackage[utf8]{inputenc}
\lstset{
 frame=single,
 breaklines=true
}
\usepackage{hyperref}
\usepackage[margin=0.5in]{geometry}	 
%\usepackage{exsheets}% also loads the `tasks' package
\usepackage{atbegshi}
\AtBeginDocument{\AtBeginShipoutNext{\AtBeginShipoutDiscard}}

%new macro definitions
\renewcommand{\labelenumi}{(\roman{enumi})}
\newcommand{\mydet}[1]{\ensuremath{\begin{vmatrix}#1\end{vmatrix}}}
\providecommand{\brak}[1]{\ensuremath{\left(#1\right)}}
\newcommand{\solution}{\noindent \textbf{Solution: }}
\newcommand{\myvec}[1]{\ensuremath{\begin{pmatrix}#1\end{pmatrix}}}
\newenvironment{amatrix}[1]{%
	\left(\begin{array}{@{}*{#1}{c}|c@{}}
}{%
	\end{array}\right)
}

\newcommand{\myaugvec}[2]{\ensuremath{\begin{amatrix}{#1}#2\end{amatrix}}}
\providecommand{\norm}[1]{\left\1Vert#1\right\rVert}
\let\vec\mathbf{}


%\SetEnumitemKey{twocol}{
% before=\raggedcolumns\begin{multicols}{2},
% after=\end{multicols}}
%\SetEnumitemKey{fourcol}{
% before=\raggedcolumns\begin{multicols}{4},
% after=\end{multicols}} 


\begin{document}
\begin{center}
\title{\textbf{TRIANGLES}}
\date{\vspace{-5ex}}
\maketitle
\end{center}
\section*{9$^{th}$Math - Chapter 7}
\section*{Problem}
\section*{Construction}
\solution
\textbf{Given:}
\fi
The input parameters are available in Table 
\ref{tab:chapters/9/7/1/6/1}.
\begin{table}[!h]
\centering
\begin{tabular}{|c|c|p{5cm}|}
\hline
\textbf{Symbol} & \textbf{Value} & \textbf{Description} \\
\hline
$\theta$ & $30\degree$ & $\angle{BAP} = \angle{BAQ}$ \\
\hline
$a$ & $9$ & $AB$ \\
\hline
$c$ & $8$ & $AQ$ \\
\hline
$\vec{e}_1$ & $\myvec{1\\0}$ & Basis vector \\
\hline
\end{tabular}

\label{tab:chapters/9/7/1/6/1}
\end{table}
The vertices of $\triangle ABC$ are given by 
\begin{align}
	\vec{A}=c\myvec{\cos\frac{\theta}{2}\\\sin\frac{\theta}{2}},\,
\vec{B}=\myvec{0\\0},\,
\vec{C}=a\vec{e}_1,
\end{align}
and 
\begin{align}
\vec{D}=\brak{2c\sin\frac{\theta}{2}}\vec{e_1}
\end{align}
Then, using the cosine formula and  the fact that $\triangle AEC$ is iscosceles,
\begin{align}
AC = b=\sqrt{a^2+c^2-2ac\cos\theta},\, EC = 2b\sin\frac{\theta}{2}
\end{align}
Also, 
\begin{align}
\angle BCA=\cos^{-1}\brak{\frac{a^2+b^2-c^2}{2ab}},\,
\angle ACE=90\degree-\frac{\theta}{2}
\end{align}
Let $\phi$ be the angle made by the vector 	$EC$ with  the $x$-axis.  Then,  
\begin{align}
\phi=180\degree-\brak{\angle BCA+\angle ACE} = =90\degree-\frac{\theta}{2}\cos^{-1}\brak{\frac{a^2+b^2-c^2}{2ab}}
\end{align}
Consequently, 
Using vector addition,
\begin{align}
	\vec{E}-\vec{C}&=\brak{2b\sin\frac{\theta}{2}}\myvec{\cos\phi\\\sin\phi}\\
	\implies \vec{E}&=\vec{C}+2b\sin\frac{\theta}{2}\myvec{\cos\phi\\\sin\phi}\\
\end{align}
Substituting numerical values, 
\begin{align}
	\norm{\vec{B}-\vec{C}}
	=\norm{\vec{D}-\vec{E}}
\label{eq:chapters/9/7/1/6/5}
\end{align}

\item In right triangle ABC, right angled at C, M is the mid-point of hypotenuse AB. C is joined to M and produced to a point D such that DM = CM. Point D is joined to point B (see Figure \ref{fig:chapters/9/7/1/8/1}). Show that:
\begin{enumerate}
\item $\triangle AMC \cong \triangle BMD$
\item $\angle DBC$ is a right angle.
\item $\triangle DBC \cong \triangle ACB$
\item $CM = \dfrac{1}{2}AB$
\end{enumerate}
\label{chapters/9/7/1/8}
\iffalse
\documentclass[10pt]{article}
\usepackage{graphicx}
\def\inputGnumericTable{}
\usepackage[latin1]{inputenc}
\usepackage{fullpage}
\usepackage{color}
\usepackage{array}
\usepackage{longtable}
\usepackage{calc}
\usepackage{multirow}
\usepackage{hhline}
\usepackage{ifthen}
\usepackage{amsmath}
\usepackage[none]{hyphenat}
\usepackage{listings}
\usepackage[english]{babel}
\usepackage{siunitx}
\usepackage{caption}
\usepackage{booktabs}
\usepackage{array}
\usepackage{extarrows}
\usepackage{enumerate}
\usepackage{enumitem}
\usepackage{amsmath}
\usepackage{commath}
\usepackage{gensymb}
\usepackage{amssymb}
\usepackage{multicol}
%\usepackage[utf8]{inputenc}
\lstset{
 frame=single,
 breaklines=true
}
\usepackage{hyperref}
\usepackage[margin=0.65in]{geometry}	 
%\usepackage{exsheets}% also loads the `tasks' package
\usepackage{atbegshi}
\AtBeginDocument{\AtBeginShipoutNext{\AtBeginShipoutDiscard}}

%new macro definitions
\renewcommand{\labelenumi}{(\roman{enumi})}
\newcommand{\mydet}[1]{\ensuremath{\begin{vmatrix}#1\end{vmatrix}}}
\providecommand{\brak}[1]{\ensuremath{\left(#1\right)}}
\newcommand{\solution}{\noindent \textbf{Solution: }}
\newcommand{\myvec}[1]{\ensuremath{\begin{pmatrix}#1\end{pmatrix}}}
\newenvironment{amatrix}[1]{%
	\left(\begin{array}{@{}*{#1}{c}|c@{}}
}{%
	\end{array}\right)
}

\newcommand{\myaugvec}[2]{\ensuremath{\begin{amatrix}{#1}#2\end{amatrix}}}
\providecommand{\norm}[1]{\left\1Vert#1\right\rVert}
\let\vec\mathbf{}


%\SetEnumitemKey{twocol}{
% before=\raggedcolumns\begin{multicols}{2},
% after=\end{multicols}}
%\SetEnumitemKey{fourcol}{
% before=\raggedcolumns\begin{multicols}{4},
% after=\end{multicols}} 


\begin{document}
\begin{center}
\title{\textbf{TRIANGLES}}
\date{\vspace{-5ex}}
\maketitle
\end{center}
\section*{9$^{th}$Math - Chapter 7}
This is Problem-8 from Exercise 7.1\\\\


\section*{\large Construction:}
\fi
The input parameters for construction
	are available in Table \ref{tab:chapters/9/7/1/8/table}.
\begin{table}[h!]
	\centering
	%\subimport{../chapters/9/7/1/8/tables/}{table.tex}
     \begin{tabular}{|c|c|p{5cm}|}
\hline
\textbf{Symbol} & \textbf{Value} & \textbf{Description} \\
\hline
$\theta$ & $30\degree$ & $\angle{BAP} = \angle{BAQ}$ \\
\hline
$a$ & $9$ & $AB$ \\
\hline
$c$ & $8$ & $AQ$ \\
\hline
$\vec{e}_1$ & $\myvec{1\\0}$ & Basis vector \\
\hline
\end{tabular}

	\caption{}
	\label{tab:chapters/9/7/1/8/table}
\end{table}
Thus, 
\begin{align}
	\vec{A}=\myvec{0\\b},\,
	\vec{B}=\myvec{a\\0},\,
	\vec{C}=\myvec{0\\0}
\end{align}
yielding
\begin{align}
	\vec{M}&=\frac{\vec{A}+\vec{B}}{2}=\frac{1}{2}\myvec{a\\b}
\end{align}
Also, 
\begin{align}
	\vec{M}&=\frac{\vec{C}+\vec{D}}{2}\\
	\implies \vec{D}&=2\vec{M}-\vec{C}=\myvec{a\\b}
\end{align}
\iffalse
\solution
Given
\begin{align}
	\vec{M}&=\frac{\vec{A}+\vec{B}}{2}
	\label{eq:chapters/9/7/1/8/1}\\
	\vec{D}-\vec{M}&=\vec{C}-\vec{M}
	\label{eq:chapters/9/7/1/8/2}\\
	\angle ACB&=90\degree
\end{align}
\textbf{Proof:} From Figure \ref{fig:chapters/9/7/1/8/1}
\fi
Thus,
\begin{align}
	\brak{\vec{D}-\vec{B}}^{\top}\brak{\vec{B}-\vec{C}} &= \myvec{0 & b}\myvec{a\\0}=0\\
	\implies BD & \perp BC\\
\end{align}
Also, 
\begin{align}
	\norm{\vec{A}-\vec{B}}&=\norm{\myvec{-a\\b}}\\
	\norm{\vec{C}-\vec{D}}&=\norm{\myvec{-a\\-b}}\\
	\implies \norm{\vec{A}-\vec{B}} &= \norm{\vec{C}-\vec{D}}\\
	\text{ or, } AB &= CD
	\label{eq:chapters/9/7/1/8/3}	
\end{align}
From \eqref{eq:chapters/9/7/1/8/3}
\begin{align}
	\implies CM = \frac{1}{2}CD = \frac{1}{2}AB 
\end{align}
See Fig. 
\ref{fig:chapters/9/7/1/8/1}.
\begin{figure}[H]
	\begin{center}
		\includegraphics[width=\columnwidth]{./chapters/9/7/1/8/figs/fig.pdf}
	\end{center}
\caption{}
\label{fig:chapters/9/7/1/8/1}
\end{figure}


\item
Construct a triangle $APB$ in which $BC = 7cm$,$\angle B = 75\degree$ and $AB+AC = 13cm$.

\textbf{Figure :}
\begin{figure}[H]
    \centering
          \includegraphics[width=\columnwidth]{chapters/const/examples/figs/2.png}
    \caption{}
    \label{fig:fig:1}
\end{figure}

\textbf{Solution :}
\begin{table}[H]
    \centering
      \begin{tabular}{|c|c|c|}
    \hline
    \textbf{Input parameters}&\textbf{Description}&\textbf{Value}\\
    \hline
    $\Vec{B}$&Vertex(at origin)&$\Vec{0}$ \\
    \hline
    $a$&Side of $\triangle ABC,BC$&$7$ \\
    \hline
    $b$&Side of $\triangle ABC,AB$&$b$ \\
    \hline
    $c$&Side of $\triangle ABC,AC$&$c$ \\
    \hline
    $\theta$&Angle of $\triangle ABC,\angle B$&$75\degree$ \\
   \hline
    \end{tabular}

    \caption{Table of input parameters}
    \label{tab:tab:1}
\end{table} 
\begin{table}[H]
    \centering
      \begin{tabular}{|c|c|c|}
    \hline
    \textbf{Output parameters}&\textbf{Description}&\textbf{Value}\\
    \hline
    $\Vec{C}$&Vertex&$ae_1$\\
    \hline
    $\Vec{A}$&Vertex&$c\begin{pmatrix}
        \cos{\theta}\\\sin{\theta}
    \end{pmatrix}$\\
    \hline
    $b+c$&$AB + AC$&13\\
    \hline
    \end{tabular}

  \caption{Table of output parameters}
    \label{tab:tab:2}
\end{table}


From appendix,
\begin{align}
    c&=\frac{k^2-a^2}{2\brak{k-a\cos{\theta}}}\\
    &=\frac{240}{52-7\sqrt{6}+7\sqrt{2}}
    \end{align}
Therefore,
\begin{align}
    \Vec{A}&=c\begin{pmatrix}
        \cos{\theta}\\\sin{\theta}
    \end{pmatrix}\\
    &=\frac{240}{52-7\sqrt{6}+7\sqrt{2}}\begin{pmatrix}
        \cos{75\degree}\\\sin{75\degree}
    \end{pmatrix}\\
    &=\begin{pmatrix}
        1.388\\5.18
    \end{pmatrix}\\
\end{align}

\end{enumerate}

\subsection{Exercises}
\begin{enumerate}[label=\thesection.\arabic*,ref=\thesection.\theenumi]
\numberwithin{equation}{enumi}
\numberwithin{figure}{enumi}
\numberwithin{table}{enumi}
\item Draw a right triangle ${ABC}$ in which $BC=12$ cm, $AB=5$ cm and $\angle{B}=90\degree$.
\item Draw an isosceles triangle ${ABC}$ in which $AB$=$AC$=6 cm and $BC$ =6 cm.
\item Draw a triangle ${ABC}$ in which $AB$=5 cm. $BC=6 cm\text{ and }\angle {ABC}=60\degree$.
\item Draw a triangle ${ABC}$ in which $AB$=4 cm, $BC=6 cm\text{ and }AC=9$
\item Draw a triangle ${ABC}$ in which $BC=6$ cm, $CA=5$ cm and $AB=4$ cm. 
\item Draw a parallelogram ${ABCD}$ in which $BC=5$ cm, $AB=3$ cm and $\angle{ABC}=60\degree$, divide it into triangles ${ACB}\text{ and }{ABD}$ by the diagonal $BD$. 
\item In triangles ABC and PQR, $ \angle{A} = \angle{Q} $ and $ \angle{B} = \angle{R} $.Which side of $ \triangle{PQR} $ should be equal to side AB of $ \triangle{ABC} $ so that the two triangles are congruent?Give reason for your answer.

\item In triangles ABC and PQR, $ \angle{A} = \angle{Q} $ and $ \angle{B} = \angle{R} $.Which side of $ \triangle{PQR} $should be equal to side BC of $ \triangle{ABC} $ so that the two triangles are congruent?Give reason for your answer.

\item "If two sides and an angle of one triangle are equal to two sides and an angle ofanother triangle,then two triangles must be congruent."Is the statement true?Why?

\item "If two angles and a side one triangle are equal to two angles and a side of another triangle,then the two triangles must be congruent."Is the statement true?Why?

\item Is it possible to construct a triangle with lengths of its sides as 4 cm, 3 cm and 7 cm?Give reason for your answer.

\item It is given that $ \triangle{ABC} \cong \triangle{RPQ} $.Is it true to say that BC = QR?Why?

\item If $ \triangle{PQR} \cong \triangle{EDF} $,then is it true to say that PR = EF?Give reason for your answer.

\item In $ \triangle{PQR} $,$ \angle{P} = 70\degree $ and $ \angle{R} = 30\degree $.Which side of the triangle is the longest?Give reason for your answer.

\item AD is a median of the triangle ABC.Is it true that AB + BC + CA $ > $ 2AD?Give reason for your answer.

\item M is a point on side BC of a triangle ABC such that AM is the bisector of $ \angle{BAC} $.Is it true to say that perimeter of the triangle is greater than 2AM?Give reason for your answer.

\item Is it possible to construct a triangle with lengths of its sides as 9 cm, 7 cm and 17 cm?Give reason for your answer.

\item Is it possible to cosntruct a triangle with lengths of its sides as 8 cm, 7 cm and 4 cm?Give reason for your answer.

	\item $\vec{ABC}$ is an isosceles triangle with $\vec{AB=AC}$ and $\vec{BD}$ and $\vec{CE}$ are its two medians. Show that $\vec{BD=CE}$.
	\item In Fig. \ref{fig:exemplar/9.7.37.3.2}, $\vec{D}$ and $\vec{E}$ are the points on side $\vec{BC}$ of a $\triangle \vec{ABC}$ such that $\vec{BD=CE}$ and $\vec{AD=AE}$. Show that $\triangle \vec{ABD} \cong \triangle \vec{ACE}$.
\begin{figure}[h]
	\centering
	\includegraphics[width=\columnwidth]{exemplar/9.7.3/figs/Figure1.png}
	\caption{}
	\label{fig:exemplar/9.7.37.3.2}
\end{figure}
\item $\vec{CDE}$ is an equilateral triangle formed on a side $\vec{CD}$ of a square $\vec{ABCD}$ (Fig. \ref{fig:exemplar/9.7.37.3.3}). Show that $\triangle \vec{ADE} \cong \triangle \vec{BCE}$.
\begin{figure}[h]
	\centering
	\includegraphics[width=\columnwidth]{exemplar/9.7.3/figs/Figure2.png}
	\caption{}
	\label{fig:exemplar/9.7.37.3.3}
\end{figure}
\item In Fig. \ref{fig:exemplar/9.7.37.3.4}, $\vec{BA} \perp \vec{AC}$, $\vec{DE} \perp \vec{DF}$ such that $\vec{BA=DE}$ and $\vec{BF=EC}$. Show that $\triangle \vec{ABC} \cong \triangle \vec{DEF}$.
\begin{figure}[h]
	\centering
	\includegraphics[width=\columnwidth]{exemplar/9.7.3/figs/Figure3.png}
	\caption{}
	\label{fig:exemplar/9.7.37.3.4}
\end{figure}
\item $\vec{Q}$ is a point on the side $\vec{SR}$ of $\triangle \vec{PSR}$ such that $\vec{PQ=PR}$. Prove that $\vec{PS>PQ}$.
\item $\vec{S}$ is any point on side $\vec{QR}$ of a $\triangle \vec{PQR}$. Show that $\vec{PQ+QR+RP>2PS}$.
\item $\vec{D}$ is any point on side $\vec{AC}$ of a $\triangle \vec{ABC}$ with $\vec{AB=AC}$. Show that $\vec{CD<BD}$.
\item In Fig. \ref{fig:exemplar/9.7.37.3.8}, $\vec{l} \| \vec{m}$ an $\vec{M}$ is the mid-point of a line segment $\vec{AB}$. Show that $\vec{M}$ is also the mid-point of any line segment $\vec{CD}$, having its end points on $\vec{l}$ and $\vec{m}$, respectively.
\begin{figure}[h]
	\centering
	\includegraphics[width=\columnwidth]{exemplar/9.7.3/figs/Figure4.png}
	\caption{}
	\label{fig:exemplar/9.7.37.3.8}
\end{figure}
\item Bisectors of the $\angle \vec{B}$ and $\angle \vec{C}$ of an isosceles triangle with $\vec{AB=AC}$ intersect each other at $\vec{O}$. $\vec{BO}$ is produced to a point $\vec{M}$. Prove that $\angle \vec{MOC}= \angle \vec{ABC}$.
\item Bisectors of the $\angle \vec{B}$ and $\angle \vec{C}$ of an isosceles triangle $\vec{ABC}$ with $\vec{AB=AC}$ intersect each other at $\vec{O}$. Show that the external angle adjacent to $\angle \vec{ABC}$ is equal to $\angle \vec{BOC}$.
\item In Fig. \ref{fig:exemplar/9.7.37.3.11}, $\vec{AD}$ is the bisector of $\angle \vec{BAC}$. Prove that $\vec{AB>BD}$.
\begin{figure}[h]
	\centering
	\includegraphics[width=\columnwidth]{exemplar/9.7.3/figs/Figure5.png}
	\caption{}
	\label{fig:exemplar/9.7.37.3.11}
\end{figure}
\item Find all the angles of an equilateral triangle.
%\item The image of an object placed at a point $A$ before a plane mirror $LM$ is seen at the point $B$ by an observer at $D$ as shown in \eqref{fig:exemplar/9.7.4Figure 1} Prove that the image is as far behind the mirror as the object is in front  of the mirror.
%
%[\textcolor{cyan}{Hint:} $CN$ is normal to the mirror. Also, angle of incidence = angle of reflection].
%\begin{figure}[!h]
%\centering
% \includegraphics[width=\columnwidth]{./exemplar/9.7.4/figs/7A.png}
%\caption{}
%\label{fig:exemplar/9.7.4Figure 1}
%\end{figure}
\item $P$ is a point on the bisector of $\angle ABC$. If the line through $P$, parallel to $BA$ meet $BC$ at $Q$, prove that $BPQ$ is an isosceles triangle.
\item $ABC$ is a right triangle with $AB = AC$. Bisector of $\angle A$ meets $BC$ at $D$. Prove that $BC = 2 AD$
\item $ABC$ and $DBC$ are two triangles on the same base $BC$ such that $A$ and $D$ lie on the opposite sides of $BC$, $AB = AC$ and $DB = DC$. Show that $AD$ is the perpendicular bisector of BC.
\item ABC is an isosceles triangle in which $AC = BC$. $AD$ and $BE$ are respectively two altitudes to sides $BC$ and $AC$. Prove that $AE = BD$.
\item Prove that sum of any two sides of a triangle is greater than twice the median with respect to the third side.
\item In a triangle $ABC$, $D$ is the mid-point of side $AC$ such that $ BD = \frac{1}{2} AC $. Show that $\angle ABC$ is a right angle.
\item In a right triangle, prove that the line-segment joining the mid-point of the hypotenuse to the opposite vertex is half the hypotenuse.
\item Two lines $l$ and $m$ intersect at the point $O$ and $P$ is a point on a line $n$ passing through the point $O$ such that $P$ is equidistant from $l$ and $m$. Prove that $n$ is the bisector of the angle formed by $l$ and $m$.
\item $ABC$ is a right triangle such that $AB = AC$ and bisector of angle $C$ intersects the side $AB$ at $D$. Prove that $AC + AD = BC$.
\item Prove that in a triangle, other than an equilateral triangle, angle opposite the longest side is greater than $\frac{2}{3}$ of a right angle.
\item In $\triangle$ABC,$AB=AC$ and $\angle$$B=50$$\degree$.Then $\angle$C is equal to
\begin{enumerate}
\item 40$\degree$
\item 50$\degree$
\item 80$\degree$
\item 130$\degree$
\end{enumerate}
\item In $\triangle$ABC, $BC=AB$ and $\angle$$B=80$$\degree$.Then $\angle$A is equal to
\begin{enumerate}
\item 80$\degree$
\item 40$\degree$
\item 50$\degree$
\item 100$\degree$
\end{enumerate}
\item In $\triangle$PQR, $\angle$$R$=$\angle$$P$ and $QR=4cm$ and $PR=5cm$. Then the length of PQ is
\begin{enumerate}
\item $4cm$
\item $5cm$
\item $2cm$
\item $2.5cm$
\end{enumerate}
\item D is a Point on the side BC of a $\triangle$ABC such that AD bisects $\angle$BAC. Then
\begin{enumerate}
\item $BD=CD$
\item $BA>BD$
\item $BD>BA$
\item $CD>CA$
\end{enumerate}
\item Two sides of a triangle are of lengths 5cm and 1.5cm.The length of the third side of the triangle cannot be
\begin{enumerate}
\item $3.6$cm
\item $4.1$cm
\item $3.8$cm
\item $3.4$cm
\end{enumerate}
\item In$\triangle$PQR, if $\angle R$ $>$ $\angle Q$,then
\begin{enumerate}
\item QR $>$ PR
\item PQ $>$ PR
\item PQ $<$ PR
\item QR $<$ PR
\end{enumerate}
\item The construction of a triangle $ABC$, given that $BC = 6 cm, \angle B = 45\degree$ is not possible when difference of $AB$ and $AC$ is equal to
		\begin{enumerate}
			\item 6.9 cm
			\item 5.2 cm
			\item 5.0 cm
			\item 4.0 cm
		\end{enumerate}
	\item The construction of a triangle $ABC$, given that $BC = 6 cm, \angle C = 60\degree$ is possible when difference of $AB$ and $AC$ is equal to
		\begin{enumerate}
			\item 3.2 cm
			\item 3.1 cm
			\item 3 cm
			\item 2.8 cm
		\end{enumerate}
\item Construct a triangle whose sides are $3.6 cm$, $3.0 cm$ and $4.8 cm$. Bisect the smallest angle and measure each part.
\item Construct a triangle $ABC$ in which $BC = 5 cm$, $\angle B = 60\degree$ and $AC+AB = 7.5cm$.
\end{enumerate}
Construct each of the following and give justification :
\begin{enumerate}[label=\thesection.\arabic*,ref=\thesection.\theenumi,resume*]
\item A triangle if its perimeter is 10.4 cm and two angles are 45\degree and 120\degree.
\item A triangle $PQR$ given that $QR$ = 3cm, $\angle PQR = 45\degree$ and $QP - PR = 2 cm$.
\item A right triangle when one side is 3.5 cm and sum of other sides and the hypotenuse
is 5.5 cm.
\item An equilateral triangle if its altitude is 3.2 cm.
\end{enumerate}                               
Write true or false in each of the following. Give reasons for your answer:
\begin{enumerate}[label=\thesection.\arabic*,ref=\thesection.\theenumi,resume*]
\item A triangle $ABC$ can be constructed in which $AB$ = 5 cm, $\angle$A =45$\degree$ and $BC$ + $AC$ = 5 cm.
\item A triangle $ABC$ can be constructed in which $BC$ = 6 cm, $\angle$30$\degree$ and $AC$ - $AB$=4 cm.
\item A triangle $ABC$ can be constructed in which $\angle$B =105$\degree$,$\angle$C =90$\degree$ and $AB$ + $BC$ + $AC$ = 10 cm.        
\item A triangle $ABC$ can be constructed in which $\angle$B =60$\degree$,$\angle$C =45$\degree$ and $AB$ + $BC$ + $AC$ = 12 cm           
\end{enumerate}                               

\subsection{ Quadrilateral}
\begin{enumerate}[label=\thesection.\arabic*,ref=\thesection.\theenumi]
\numberwithin{equation}{enumi}
\numberwithin{figure}{enumi}
\numberwithin{table}{enumi}
\item In the Figure \ref{fig:9/9/2/1}, $ABCD$ is a parallelogram, $AE \perp DC$ and $CF \perp AD$. If $AB = 16 cm$, $AE = 8 cm$, and $CF = 10cm$, find $AD$.
	\begin{figure}[!h]
		\centering
 \includegraphics[width=\columnwidth]{chapters/9/9/2/1/figs/fig1.pdf}
		\caption{}
		\label{fig:9/9/2/1}
  	\end{figure}
\iffalse
\documentclass[journal,10pt,twocolumn]{article}
\usepackage{graphicx}
\usepackage[margin=0.5in]{geometry}
\usepackage[cmex10]{amsmath}
\usepackage{array}
\usepackage{booktabs}
\usepackage{listings}
\title{\textbf{Line Assignment}}
\author{Bhavani Kanike}
\date{October 2022}

\providecommand{\norm}[1]{\left\lVert#1\right\rVert}
\providecommand{\abs}[1]{\left\vert#1\right\vert}
\let\vec\mathbf
\newcommand{\myvec}[1]{\ensuremath{\begin{pmatrix}#1\end{pmatrix}}}
\newcommand{\mydet}[1]{\ensuremath{\begin{vmatrix}#1\end{vmatrix}}}
\providecommand{\brak}[1]{\ensuremath{\left(#1\right)}}

\begin{document}

\maketitle
\paragraph{\textit{Problem Statement} 
\fi
ABCD is a quadrilateral in which $\vec{P}, \vec{Q}, \vec{R}$ and $\vec{S}$ are mid-points of the sides AB, BC, CD and DA (see Fig \ref{fig:9/8/2/1}). AC is a diagonal. 
		
Show that 
\begin{enumerate}
	\item $SR \parallel AC$ and $SR =\frac{1}{2} AC$
\item $PQ = SR$
\item $PQRS$ is a parallelogram.
\end{enumerate}
 	\begin{figure}
		\centering
 \includegraphics[width=\columnwidth]{chapters/9/8/2/1/figs/line1.pdf}
		\caption{}
		\label{fig:9/8/2/1}
  	\end{figure}
	\solution 
	Using 
	  \eqref{eq:section_formula},
	\begin{align}
		\label{eq:9/8/2/1}
		\begin{split}
		\vec{P} &= \frac{\vec{A}+\vec{B}}{2}\\
 \vec{Q} &= \frac{\vec{C}+\vec{B}}{2}\\
 \vec{R} &= \frac{\vec{C}+\vec{D}}{2}\\
 \vec{S} &= \frac{\vec{D}+\vec{A}}{2}
		\end{split}
	\end{align}
\begin{enumerate}
	\item
	Consequently, 
	\begin{align}
\vec{R}
		-\vec{S} &= \frac{\vec{C}-\vec{A}}{2}
		\\
		\implies SR &\parallel AC
	\end{align}
	Also, 
	\begin{align}
		\norm{\vec{R}
		-\vec{S}} &= \frac{\norm{\vec{C}-\vec{A}}}{2}
		\\
		\implies SR &= \frac{1}{2}AC
	\end{align}
\item 	From 
		\eqref{eq:9/8/2/1},
	\begin{align}
\vec{R}
		-\vec{S} = \vec{Q}-\vec{P}
	\end{align}
	which means that $PQRS$ is a parallelogram and $PQ = SR$.
\end{enumerate}
%
\iffalse
\begin{figure}[h]
\centering
\includegraphics[width=1\columnwidth]
\caption{Figure}
\label{fig:triangle}
\end{figure}

\section*{Solution}

$\boldsymbol Given :$  ABCD is a Quadrilateral P,Q,R and S are the midpoints of line AB,BC,CD,DA.We can obtain the points P,Q,R and S from A,B,C and D and are given by\\\\
\boldmath
\unboldmath
(3) To prove that PQRS is a parallelogram we need to prove  PQ // SR
To prove SR $\parallel$ PQ\\
Direction vector of line SR  $\boldsymbol {(R-S) =  \frac{(C-A)}{2}}$\\\\
Direction vector of line PQ  $\boldsymbol {(Q-P)= \frac{(C-A)}{2}}$\\\\
\begin{equation}
	\boldsymbol {(R-S) = (Q-P) = \frac{(C-A)}{2}}\\
\end{equation}
Since the direction vectors of line SR and PQ are in same direction\\\\
$SR \parallel PQ$\\
Therefore,
$\boldsymbol{ PQRS }$ is a parallelogram\\\\

	
(1)  Directional vector of line SR  = $\boldsymbol {(R-S)}$ = $\frac{\boldsymbol{(C-A)}}{2} $\\
Directional vector of line AC  = $\boldsymbol {(C-A)}$\\

It is observed that the constant k is $\frac{1}{2}$

Therefore
\begin{equation}
	SR \parallel AC
\end{equation} 

and from equation 1 
\begin{equation}
	\boldsymbol {SR = \frac{1}{2}AC}    
\end{equation}\\


(2)   To prove PQ = SR\\ 
		From euqation 1\\\\
\begin{equation}
		\boldsymbol{ (Q-P) = (R-S) = \frac{(C-A)}{2}}
\end{equation}
	 



\section{Execution}
The below python code realizes the construction:
\begin{lstlisting}
https://github.com/bhavani360/FWC_assignments
\end{lstlisting}
	
\section*{Construction}
The dimensions of the Quadrilateral ABCD are taken as below\\
{
\setlength\extrarowheight{2pt}
\centering
	\begin{tabular}{|c|c|}
	\hline
	\textbf{symbol}&\textbf{value}\\
	\hline
	r&8\\
	\hline
	$\theta$&pi/2.5\\
	\hline
	d&7\\
	\hline
	A&(0,0)\\
	\hline
	B&(d,0)\\
	\hline
	D&(rcos$\theta$,rsin$\theta$)\\
	\hline
	C&(D/1.5)+B\\
	\hline
\end{tabular}
}
\end{document}
\fi

\item For a given Parallelogram $ABCD$, show that for any
point $\vec{P}$ inside the parallelogram,
\begin{enumerate}
	\item $Ar(APD)+Ar(PBC) = \frac{1}{2}Ar(ABCD)$
	\item $Ar(APD)+Ar(PBC) = Ar(APB)+Ar(PCD)$
\end{enumerate}
\documentclass[journal,10pt,twocolumn]{article}
\usepackage{graphicx}
\graphicspath{{./Figures/}}
\usepackage[margin=0.5in]{geometry}
\usepackage[cmex10]{amsmath}
\usepackage{amssymb}
\usepackage{array}
\usepackage{booktabs}

\title{\textbf{Line Assignment}}
\author{Bole Manideep}
\date{September 2022}

\providecommand{\norm}[1]{\left\lVert#1\right\rVert}
\providecommand{\abs}[1]{\left\vert#1\right\vert}
\let\vec\mathbf
\newcommand{\myvec}[1]{\ensuremath{\begin{pmatrix}#1\end{pmatrix}}}
\newcommand{\mydet}[1]{\ensuremath{\begin{vmatrix}#1\end{vmatrix}}}
\providecommand{\brak}[1]{\ensuremath{\left(#1\right)}}

\begin{document}
\maketitle
\paragraph{\textit{Problem Statement} - For a given Parallelogram ABCD, show that for any
point ’P’ inside the parallelogram,}
\begin{enumerate}
	\item $\boldsymbol{Ar(APD)+Ar(PBC) = \frac{1}{2}Ar(ABCD)}$
	\item $\boldsymbol{Ar(APD)+Ar(PBC) = Ar(APB)+Ar(PCD)}$
\end{enumerate}

\begin{figure}[h]
\centering
\includegraphics[width=1\columnwidth]{Question.png}
\caption{Parallelogram ABCD with interior point P}
\label{fig:Parallelogram}
\end{figure}

\section*{Solution}

\subsection*{Part 1}
WKT, area of a parallelogram with adjacent sides a \& b is,
\begin{equation}
\text{Area of parallelogram = } \norm{\vec{a} \times \vec{b}}
\label{eq-1-}
\end{equation}
And, area of a triangle with adjacent sides p \& q is,
\begin{equation}
\text{Area of triangle = } \frac{1}{2} \norm{\vec{p} \times \vec{q}}
\label{eq-2-}
\end{equation}
From Figure 1,\\ $\vec{(A-D)}\hspace{2mm}\&\hspace{2mm}\vec{(B-C)}$ are equal,\\
\begin{equation}
\norm{\vec{A-D}} = \norm{\vec{B-C}}
\label{eq-3-}
\end{equation}
Consider $\triangle$APD
\begin{equation}
\text{Area of } \triangle APD = \frac{1}{2}\norm{\vec{(A-D)} \times \vec{(A-P)}}
\label{eq-4-}
\end{equation}
Consider $\triangle$PBC
\begin{equation}
\text{Area of } \triangle BPC =  \frac{1}{2}\norm{\vec{(B-C)} \times \vec{(P-B)}}
\label{eq-5-}
\end{equation}
On adding \eqref{eq-4-} \& \eqref{eq-5-},
\begin{multline}
Ar(APD) + Ar(PBC) =\\
 \frac{1}{2}\norm{\vec{(A-D)} \times \vec{(A-P)}} + \frac{1}{2}\norm{\vec{(B-C)} \times \vec{(P-B)}}
\label{eq-6-}
\end{multline}
From equation \eqref{eq-3-},
\begin{multline}
Ar(APD) + Ar(PBC) =\\
 \frac{1}{2}\norm{\vec{(A-D)} \times \vec{(A-P)}} + \frac{1}{2}\norm{\vec{(A-D)} \times \vec{(P-B)}}
\label{eq-7-}
\end{multline}
\begin{multline}
\implies Ar(APD) + Ar(PBC) =\\ \frac{1}{2}\norm{\vec{(A-D)}\times[\vec{(A-P)} + \vec{(P-B)}]}
\label{eq-8-}
\end{multline}
Here, AP \& PB are adjacent sides of $\triangle$ APB
\\From Triangle law of vector addition, \\ $\vec{(A-P)} + \vec{(P-B)} = \vec{(A-B)}$
\begin{equation}
\implies Ar(APD) + Ar(PBC) = \frac{1}{2}\norm{\vec{(A-D)}\times\vec{(A-B)}}
\label{eq-9-}
\end{equation}
Since, $\vec{(A-D)} \hspace{2mm} \& \hspace{2mm} 1\vec{(A-B)}$ are adjacent sides of paralleogram ABCD
\\With reference to \eqref{eq-2-},
\begin{equation}
Ar(ABCD) = \norm{\vec{(A-D)}\times\vec{(A-B)}}
\label{eq-10-}
\end{equation}
From \eqref{eq-9-} \& \eqref{eq-10-}
\begin{equation}
\therefore \hspace{3mm} Ar(APD)+Ar(PBC) = \frac{1}{2}Ar(ABCD)
\label{eq-11-}
\end{equation}

\subsection*{Part 2}
Similarly, we can prove taht,
\begin{equation}
Ar(APB)+Ar(PBD) = \frac{1}{2}Ar(ABCD)
\label{eq-12-}
\end{equation}
On Comparing \eqref{eq-11-} and \eqref{eq-12-},
\begin{equation}
Ar(APD)+Ar(PBC) = Ar(APB)+Ar(PCD)
\label{eq-13-}
\end{equation}
\begin{center}
Hence Proved
\end{center}
\section*{Construction}
\raggedright A parallelogram ABCD is constructed unsing python,with the parameters that are mentioned in the table below.
\vspace{5mm}
\begin{center}
    \setlength{\arrayrulewidth}{0.1mm}
	\setlength{\tabcolsep}{12pt}
	\renewcommand{\arraystretch}{1.5}
\begin{tabular}{|c|c|c|}
	\hline 
    \textbf{Symbol} & \textbf{Value} & \textbf{Description}\\ 		\hline
    a & 4 & AB \\ \hline
    b & 2 & AD \\ \hline
    $\theta$ & 60$^{\circ}$ & $\angle$A \\ \hline
    $\vec{A}$ & $\myvec{0 \\ 0}$ & Vertex A \\ \hline
    $\vec{B}$ & $\myvec{a \\ 0}$ & Vertex B \\ \hline
    $\vec{D}$ & $b\myvec{\cos\theta \\ \sin\theta}$ & Vertex B \\ \hline
    $\vec{C}$ & $\vec{B+D}$ & Vertex C \\ \hline
    
\end{tabular}\\ \vspace{2mm}
Table 1: Parameter's Table
\end{center}

\section*{Proofs}
Triangle law of vector addition \\
Consider a triangle APB with vertices, \\ \vspace{2mm}
$\vec{A} = \myvec{0 \\ 0}$ \hspace{2mm}
$\vec{P} = \myvec{2 \\ 1}$ \hspace{2mm}
$\vec{B} = \myvec{0 \\ 4}$ \\ \vspace{2mm}
 
Vectors $\vec{(A-P)}, \vec{(P-B)} \hspace{1mm} \& \hspace{1mm} \vec{(A-B)} \hspace{1mm} are \hspace{1mm} sides \hspace{1mm} of \hspace{1mm} \triangle APB$\\
Let's consider,
\begin{equation}
\vec{(A-P)} + \vec{(P-B)} = \myvec{0 \\ 0} - \myvec{2 \\ 1} \hspace{2mm} + \hspace{2mm} \myvec{2 \\ 1} - \myvec{0 \\ 4}
\end{equation}
\begin{equation}
\implies \vec{(A-P)} + \vec{(P-B)} = \myvec{-2 \\ -1} \hspace{2mm} + \hspace{2mm} \myvec{2 \\ -3}
\end{equation}
\begin{equation}
\implies \vec{(A-P)} + \vec{(P-B)} = \myvec{0 \\ -4}
\end{equation}
\begin{equation}
\implies \vec{(A-P)} + \vec{(P-B)} = \myvec{0 \\ 0} - \myvec{0 \\ 4}
\end{equation} \vspace{1mm}
\begin{equation}
\therefore \vec{(A-P)} + \vec{(P-B)} = \vec{(A-B)}
\end{equation}  \\
\vspace{2mm}Thus, Triangle law of vector addition says taht sum of two adjacent side vectors of a triangle is equal to third side vector but in opposite direction.

\end{document}

\item In Fig.
		\ref{fig:9/9/2/5},
$PQRS$ and $ABRS$ are parallelograms
and $\vec{X}$ is any point on side $BR$. Show that  
\begin{enumerate}
    \item $ar (PQRS) = ar(ABRS)$
	    \label{prop:9/9/2/5}
    \item $ar(AXS) = \frac{1}{2} ar(PQRS)$
\end{enumerate}
	\begin{figure}[!h]
		\centering
 \includegraphics[width=\columnwidth]{chapters/9/9/2/5/figs/parallelogram1.pdf}
		\caption{}
		\label{fig:9/9/2/5}
  	\end{figure}
\iffalse
\documentclass[journal,10pt,twocolumn]{article}
\usepackage{graphicx}
\usepackage[margin=0.5in]{geometry}
\usepackage[cmex10]{amsmath}
\usepackage{array}
\usepackage{booktabs}
\usepackage{listings}
\title{\textbf{Line Assignment}}
\author{Bhavani Kanike}
\date{October 2022}

\providecommand{\norm}[1]{\left\lVert#1\right\rVert}
\providecommand{\abs}[1]{\left\vert#1\right\vert}
\let\vec\mathbf
\newcommand{\myvec}[1]{\ensuremath{\begin{pmatrix}#1\end{pmatrix}}}
\newcommand{\mydet}[1]{\ensuremath{\begin{vmatrix}#1\end{vmatrix}}}
\providecommand{\brak}[1]{\ensuremath{\left(#1\right)}}

\begin{document}

\maketitle
\paragraph{\textit{Problem Statement} 
\fi
ABCD is a quadrilateral in which $\vec{P}, \vec{Q}, \vec{R}$ and $\vec{S}$ are mid-points of the sides AB, BC, CD and DA (see Fig \ref{fig:9/8/2/1}). AC is a diagonal. 
		
Show that 
\begin{enumerate}
	\item $SR \parallel AC$ and $SR =\frac{1}{2} AC$
\item $PQ = SR$
\item $PQRS$ is a parallelogram.
\end{enumerate}
 	\begin{figure}
		\centering
 \includegraphics[width=\columnwidth]{chapters/9/8/2/1/figs/line1.pdf}
		\caption{}
		\label{fig:9/8/2/1}
  	\end{figure}
	\solution 
	Using 
	  \eqref{eq:section_formula},
	\begin{align}
		\label{eq:9/8/2/1}
		\begin{split}
		\vec{P} &= \frac{\vec{A}+\vec{B}}{2}\\
 \vec{Q} &= \frac{\vec{C}+\vec{B}}{2}\\
 \vec{R} &= \frac{\vec{C}+\vec{D}}{2}\\
 \vec{S} &= \frac{\vec{D}+\vec{A}}{2}
		\end{split}
	\end{align}
\begin{enumerate}
	\item
	Consequently, 
	\begin{align}
\vec{R}
		-\vec{S} &= \frac{\vec{C}-\vec{A}}{2}
		\\
		\implies SR &\parallel AC
	\end{align}
	Also, 
	\begin{align}
		\norm{\vec{R}
		-\vec{S}} &= \frac{\norm{\vec{C}-\vec{A}}}{2}
		\\
		\implies SR &= \frac{1}{2}AC
	\end{align}
\item 	From 
		\eqref{eq:9/8/2/1},
	\begin{align}
\vec{R}
		-\vec{S} = \vec{Q}-\vec{P}
	\end{align}
	which means that $PQRS$ is a parallelogram and $PQ = SR$.
\end{enumerate}
%
\iffalse
\begin{figure}[h]
\centering
\includegraphics[width=1\columnwidth]
\caption{Figure}
\label{fig:triangle}
\end{figure}

\section*{Solution}

$\boldsymbol Given :$  ABCD is a Quadrilateral P,Q,R and S are the midpoints of line AB,BC,CD,DA.We can obtain the points P,Q,R and S from A,B,C and D and are given by\\\\
\boldmath
\unboldmath
(3) To prove that PQRS is a parallelogram we need to prove  PQ // SR
To prove SR $\parallel$ PQ\\
Direction vector of line SR  $\boldsymbol {(R-S) =  \frac{(C-A)}{2}}$\\\\
Direction vector of line PQ  $\boldsymbol {(Q-P)= \frac{(C-A)}{2}}$\\\\
\begin{equation}
	\boldsymbol {(R-S) = (Q-P) = \frac{(C-A)}{2}}\\
\end{equation}
Since the direction vectors of line SR and PQ are in same direction\\\\
$SR \parallel PQ$\\
Therefore,
$\boldsymbol{ PQRS }$ is a parallelogram\\\\

	
(1)  Directional vector of line SR  = $\boldsymbol {(R-S)}$ = $\frac{\boldsymbol{(C-A)}}{2} $\\
Directional vector of line AC  = $\boldsymbol {(C-A)}$\\

It is observed that the constant k is $\frac{1}{2}$

Therefore
\begin{equation}
	SR \parallel AC
\end{equation} 

and from equation 1 
\begin{equation}
	\boldsymbol {SR = \frac{1}{2}AC}    
\end{equation}\\


(2)   To prove PQ = SR\\ 
		From euqation 1\\\\
\begin{equation}
		\boldsymbol{ (Q-P) = (R-S) = \frac{(C-A)}{2}}
\end{equation}
	 



\section{Execution}
The below python code realizes the construction:
\begin{lstlisting}
https://github.com/bhavani360/FWC_assignments
\end{lstlisting}
	
\section*{Construction}
The dimensions of the Quadrilateral ABCD are taken as below\\
{
\setlength\extrarowheight{2pt}
\centering
	\begin{tabular}{|c|c|}
	\hline
	\textbf{symbol}&\textbf{value}\\
	\hline
	r&8\\
	\hline
	$\theta$&pi/2.5\\
	\hline
	d&7\\
	\hline
	A&(0,0)\\
	\hline
	B&(d,0)\\
	\hline
	D&(rcos$\theta$,rsin$\theta$)\\
	\hline
	C&(D/1.5)+B\\
	\hline
\end{tabular}
}
\end{document}
\fi



\end{enumerate}

\subsection{Exercises}
\begin{enumerate}[label=\thesection.\arabic*,ref=\thesection.\theenumi]
\numberwithin{equation}{enumi}
\numberwithin{figure}{enumi}
\numberwithin{table}{enumi}

\item $ABCD$ is a quadrilateral in which $AB = BC$ and $AD = CD$. Show that $BD$ bisects both the angles $ABC$ and $ADC$.
\item $O$ is a point in the interior of a square $ABCD$ suchthat $OAB$ is an equilateral triangle. Show that $\triangle  OCD$ is an isosceles triangle.
\item Show that in a quadrilateral $ABCD$, 
\begin{align}
     AB + BC + CD + DA  <  (BD + AC)
\end{align} 
\item Show that in a quadrilateral $ABCD$,
\begin{align}
 AB + BC + CD + DA  >   AC + BD
\end{align}
\item Line segment joining the mid-points $M$ and $N$ of parallel sides $AB$ and $DC$, respectively of a trapezium $ABCD$ is perpendicular to both the sides $AB$ and $DC$. Prove that $AD = BC$.
\item $ABCD$ is a quadrilateral such that diagonal $AC$ bisects the angles $A$ and $C$. Prove that $AB = AD$ and $CB = CD$.
\item $AB$ and $CD$ are the smallest and largest sides of a quadrilateral $ABCD$. Out of $\angle B$ and $\angle D$ decide which is greater.
\item $ABCD$ is quadrilateral such that $AB = AD$ and $CB = CD$. Prove that $AC$ is the perpendicular bisector of $BD$.
\item A point $\vec{E} $ is taken on the side $BC$ of a parallelogram ABCD.$AE$ and  $DC$ are produced to meet at $\vec{F}$.Prove that  $ar (ADF) = ar (ABFC)$.
\item The diagonals of a parallelogram ABCD intersect at a point $\vec{O}$.Through $\vec{O}$,a line is drawn to intersect $AD$ at $\vec{P}$ and $BC$ at $\vec{Q}$.Show that $PQ$ divides the parallelogram into two parts of equal area.
\item The medians $BE$ and $CF$ of a triangle ABC intersect at $\vec{G}$.Prove that the area of $ \triangle${GBC}= area of the quadrilateral AFGE.	
\item In Fig.\ref{fig:exemplar/9.9.4/9.24},$CD \parallel AE$  and $CY \parallel BA$.Prove that  $ar (CBX) =  ar (AXY)$
\begin{figure}[h]
	\centering
	\includegraphics[width=\columnwidth]{exemplar/9.9.4/figs/Fig9.24.png}
	\caption{}
	\label{fig:exemplar/9.9.4/9.24}
\end{figure}
\item ABCD is a trapezium in which $AB \parallel DC$,$DC = 30cm$  and $AB = 50cm$.If $\vec{X}$ and $\vec{Y}$ are,respectively the mid-points of $AD$ and $BC$,prove that  $ar (DCYX) = \frac{7}{9} ar (XYBA)$.
\item  In $ \triangle${ABC},if $\vec{L}$ and $\vec{M}$ are the points on $AB$ and $AC$,respectively such that $LM \parallel BC$.Prove that $ar (LOB) = ar (MOC)$.
\item In Fig.\ref{fig:exemplar/9.9.4/9.25},ABCDE is any pentagon.$BP$ drawn parallel to $AC$ meets $DC$ produced at $\vec{P}$ and $EQ$ drawn parallel to $AD$ meets $CD$ produced at $\vec{Q}$.Prove that  $ ar (ABCDE) = ar (APQ) $.
\begin{figure}[h]
	\centering
	\includegraphics[width=\columnwidth]{exemplar/9.9.4/figs/Fig9.25.png}
	\caption{}
	\label{fig:exemplar/9.9.4/9.25}
\end{figure}
\item If the medians of a $ \triangle$ ABC  intersect at $\vec{G}$,show that
	\begin{align} 
		{ar (AGB)} &={ar (AGC)}= {ar (BGC)} = \frac{1}{3} {ar (ABC)}
	\end{align}
\item In Fig.\ref{fig:exemplar/9.9.4/9.26},$\vec{X}$ and $\vec{Y}$ are the mid-points of $AC$ and $AB$ respectively,$QP \parallel BC$ and $CYQ$ and $BXP$ are straight lines.Prove that $ ar (ABP) = ar (ACQ) $.
\begin{figure}[h]
	\centering
	\includegraphics[width=\columnwidth]{exemplar/9.9.4/figs/Fig9.26.png}
	\caption{}
	\label{fig:exemplar/9.9.4/9.26}
\end{figure}
\item In Fig.\ref{fig:exemplar/9.9.4/9.27},ABCD and AEFD are two parallelograms.Prove that $ ar (PEA) = ar (QFD) $ [Hint:Join PD].
\begin{figure}[h]
	\centering
	\includegraphics[width=\columnwidth]{exemplar/9.9.4/figs/Fig9.27.png}
	\caption{}
	\label{fig:exemplar/9.9.4/9.27}
\end{figure}
\end{enumerate}

%
\section{Linear Forms}
\subsection{Equation of a Line}
Find the equation of line 
\begin{enumerate}[label=\thesection.\arabic*,ref=\thesection.\theenumi]
\numberwithin{equation}{enumi}
\numberwithin{figure}{enumi}
\numberwithin{table}{enumi}

\item 
\label{chapters/11/10/2/1}
%\iffalse
\documentclass[12pt]{article}
\usepackage{graphicx}
\usepackage{amsmath}
\usepackage{mathtools}
\usepackage{gensymb}

\newcommand{\mydet}[1]{\ensuremath{\begin{vmatrix}#1\end{vmatrix}}}
\providecommand{\brak}[1]{\ensuremath{\left(#1\right)}}
\providecommand{\norm}[1]{\left\lVert#1\right\rVert}
\newcommand{\solution}{\noindent \textbf{Solution: }}
\newcommand{\myvec}[1]{\ensuremath{\begin{pmatrix}#1\end{pmatrix}}}
\let\vec\mathbf

\begin{document}
\begin{center}
\textbf\large{CHAPTER-11 \\ CIRCLES}

\end{center}
\section*{Excercise 11.1}

Q4.Find the equation of the circle with centre $(1,1)$ and radius $\sqrt{2}$.

\solution
\fi
Given
\begin{align}
	\vec{c} &= \myvec{1\\1} \text{ and } r = \sqrt{2},
	\\
	\vec{u}&=\vec{-c}
	 = \myvec{-1\\-1}\\
	 \\
	f &= \norm{\vec{u}}^2 - r^2
	  =0	
\end{align}
Thus, the equation of circle is 
\begin{align}
	\norm{\vec{x}}^2 -2\myvec{1&1}\vec{x} = 0       		       
\end{align}	
See Fig. 
\ref{fig:chapters/11/11/1/4/Fig1}.
\begin{figure}[!h]
	\begin{center} 
	  \includegraphics[width=\columnwidth]{chapters/11/11/1/4/figs/circ.png}
	\end{center}
\caption{}
\label{fig:chapters/11/11/1/4/Fig1}
\end{figure}

\item passing through the point (– 4, 3) with slope $\frac{1}{2}$.
\label{chapters/11/10/2/2}
\iffalse
\documentclass[journal,10pt,twocolumn]{article}
\usepackage{graphicx}
\usepackage[margin=0.5in]{geometry}
\usepackage[cmex10]{amsmath}
\usepackage{array}
\usepackage{booktabs}
\usepackage{listings}
\title{\textbf{Line Assignment}}
\author{Bhavani Kanike}
\date{October 2022}

\providecommand{\norm}[1]{\left\lVert#1\right\rVert}
\providecommand{\abs}[1]{\left\vert#1\right\vert}
\let\vec\mathbf
\newcommand{\myvec}[1]{\ensuremath{\begin{pmatrix}#1\end{pmatrix}}}
\newcommand{\mydet}[1]{\ensuremath{\begin{vmatrix}#1\end{vmatrix}}}
\providecommand{\brak}[1]{\ensuremath{\left(#1\right)}}

\begin{document}

\maketitle
\paragraph{\textit{Problem Statement} 
\fi
ABCD is a quadrilateral in which $\vec{P}, \vec{Q}, \vec{R}$ and $\vec{S}$ are mid-points of the sides AB, BC, CD and DA (see Fig \ref{fig:9/8/2/1}). AC is a diagonal. 
		
Show that 
\begin{enumerate}
	\item $SR \parallel AC$ and $SR =\frac{1}{2} AC$
\item $PQ = SR$
\item $PQRS$ is a parallelogram.
\end{enumerate}
 	\begin{figure}
		\centering
 \includegraphics[width=\columnwidth]{chapters/9/8/2/1/figs/line1.pdf}
		\caption{}
		\label{fig:9/8/2/1}
  	\end{figure}
	\solution 
	Using 
	  \eqref{eq:section_formula},
	\begin{align}
		\label{eq:9/8/2/1}
		\begin{split}
		\vec{P} &= \frac{\vec{A}+\vec{B}}{2}\\
 \vec{Q} &= \frac{\vec{C}+\vec{B}}{2}\\
 \vec{R} &= \frac{\vec{C}+\vec{D}}{2}\\
 \vec{S} &= \frac{\vec{D}+\vec{A}}{2}
		\end{split}
	\end{align}
\begin{enumerate}
	\item
	Consequently, 
	\begin{align}
\vec{R}
		-\vec{S} &= \frac{\vec{C}-\vec{A}}{2}
		\\
		\implies SR &\parallel AC
	\end{align}
	Also, 
	\begin{align}
		\norm{\vec{R}
		-\vec{S}} &= \frac{\norm{\vec{C}-\vec{A}}}{2}
		\\
		\implies SR &= \frac{1}{2}AC
	\end{align}
\item 	From 
		\eqref{eq:9/8/2/1},
	\begin{align}
\vec{R}
		-\vec{S} = \vec{Q}-\vec{P}
	\end{align}
	which means that $PQRS$ is a parallelogram and $PQ = SR$.
\end{enumerate}
%
\iffalse
\begin{figure}[h]
\centering
\includegraphics[width=1\columnwidth]
\caption{Figure}
\label{fig:triangle}
\end{figure}

\section*{Solution}

$\boldsymbol Given :$  ABCD is a Quadrilateral P,Q,R and S are the midpoints of line AB,BC,CD,DA.We can obtain the points P,Q,R and S from A,B,C and D and are given by\\\\
\boldmath
\unboldmath
(3) To prove that PQRS is a parallelogram we need to prove  PQ // SR
To prove SR $\parallel$ PQ\\
Direction vector of line SR  $\boldsymbol {(R-S) =  \frac{(C-A)}{2}}$\\\\
Direction vector of line PQ  $\boldsymbol {(Q-P)= \frac{(C-A)}{2}}$\\\\
\begin{equation}
	\boldsymbol {(R-S) = (Q-P) = \frac{(C-A)}{2}}\\
\end{equation}
Since the direction vectors of line SR and PQ are in same direction\\\\
$SR \parallel PQ$\\
Therefore,
$\boldsymbol{ PQRS }$ is a parallelogram\\\\

	
(1)  Directional vector of line SR  = $\boldsymbol {(R-S)}$ = $\frac{\boldsymbol{(C-A)}}{2} $\\
Directional vector of line AC  = $\boldsymbol {(C-A)}$\\

It is observed that the constant k is $\frac{1}{2}$

Therefore
\begin{equation}
	SR \parallel AC
\end{equation} 

and from equation 1 
\begin{equation}
	\boldsymbol {SR = \frac{1}{2}AC}    
\end{equation}\\


(2)   To prove PQ = SR\\ 
		From euqation 1\\\\
\begin{equation}
		\boldsymbol{ (Q-P) = (R-S) = \frac{(C-A)}{2}}
\end{equation}
	 



\section{Execution}
The below python code realizes the construction:
\begin{lstlisting}
https://github.com/bhavani360/FWC_assignments
\end{lstlisting}
	
\section*{Construction}
The dimensions of the Quadrilateral ABCD are taken as below\\
{
\setlength\extrarowheight{2pt}
\centering
	\begin{tabular}{|c|c|}
	\hline
	\textbf{symbol}&\textbf{value}\\
	\hline
	r&8\\
	\hline
	$\theta$&pi/2.5\\
	\hline
	d&7\\
	\hline
	A&(0,0)\\
	\hline
	B&(d,0)\\
	\hline
	D&(rcos$\theta$,rsin$\theta$)\\
	\hline
	C&(D/1.5)+B\\
	\hline
\end{tabular}
}
\end{document}
\fi

	\item passing through $\myvec{0\\0}$ with slope $m$.\\
\label{chapters/11/10/2/3}
\solution
\iffalse
\documentclass[journal,10pt,twocolumn]{article}
\usepackage{graphicx}
\usepackage[margin=0.5in]{geometry}
\usepackage[cmex10]{amsmath}
\usepackage{array}
\usepackage{booktabs}
\usepackage{listings}
\title{\textbf{Line Assignment}}
\author{Bhavani Kanike}
\date{October 2022}

\providecommand{\norm}[1]{\left\lVert#1\right\rVert}
\providecommand{\abs}[1]{\left\vert#1\right\vert}
\let\vec\mathbf
\newcommand{\myvec}[1]{\ensuremath{\begin{pmatrix}#1\end{pmatrix}}}
\newcommand{\mydet}[1]{\ensuremath{\begin{vmatrix}#1\end{vmatrix}}}
\providecommand{\brak}[1]{\ensuremath{\left(#1\right)}}

\begin{document}

\maketitle
\paragraph{\textit{Problem Statement} 
\fi
ABCD is a quadrilateral in which $\vec{P}, \vec{Q}, \vec{R}$ and $\vec{S}$ are mid-points of the sides AB, BC, CD and DA (see Fig \ref{fig:9/8/2/1}). AC is a diagonal. 
		
Show that 
\begin{enumerate}
	\item $SR \parallel AC$ and $SR =\frac{1}{2} AC$
\item $PQ = SR$
\item $PQRS$ is a parallelogram.
\end{enumerate}
 	\begin{figure}
		\centering
 \includegraphics[width=\columnwidth]{chapters/9/8/2/1/figs/line1.pdf}
		\caption{}
		\label{fig:9/8/2/1}
  	\end{figure}
	\solution 
	Using 
	  \eqref{eq:section_formula},
	\begin{align}
		\label{eq:9/8/2/1}
		\begin{split}
		\vec{P} &= \frac{\vec{A}+\vec{B}}{2}\\
 \vec{Q} &= \frac{\vec{C}+\vec{B}}{2}\\
 \vec{R} &= \frac{\vec{C}+\vec{D}}{2}\\
 \vec{S} &= \frac{\vec{D}+\vec{A}}{2}
		\end{split}
	\end{align}
\begin{enumerate}
	\item
	Consequently, 
	\begin{align}
\vec{R}
		-\vec{S} &= \frac{\vec{C}-\vec{A}}{2}
		\\
		\implies SR &\parallel AC
	\end{align}
	Also, 
	\begin{align}
		\norm{\vec{R}
		-\vec{S}} &= \frac{\norm{\vec{C}-\vec{A}}}{2}
		\\
		\implies SR &= \frac{1}{2}AC
	\end{align}
\item 	From 
		\eqref{eq:9/8/2/1},
	\begin{align}
\vec{R}
		-\vec{S} = \vec{Q}-\vec{P}
	\end{align}
	which means that $PQRS$ is a parallelogram and $PQ = SR$.
\end{enumerate}
%
\iffalse
\begin{figure}[h]
\centering
\includegraphics[width=1\columnwidth]
\caption{Figure}
\label{fig:triangle}
\end{figure}

\section*{Solution}

$\boldsymbol Given :$  ABCD is a Quadrilateral P,Q,R and S are the midpoints of line AB,BC,CD,DA.We can obtain the points P,Q,R and S from A,B,C and D and are given by\\\\
\boldmath
\unboldmath
(3) To prove that PQRS is a parallelogram we need to prove  PQ // SR
To prove SR $\parallel$ PQ\\
Direction vector of line SR  $\boldsymbol {(R-S) =  \frac{(C-A)}{2}}$\\\\
Direction vector of line PQ  $\boldsymbol {(Q-P)= \frac{(C-A)}{2}}$\\\\
\begin{equation}
	\boldsymbol {(R-S) = (Q-P) = \frac{(C-A)}{2}}\\
\end{equation}
Since the direction vectors of line SR and PQ are in same direction\\\\
$SR \parallel PQ$\\
Therefore,
$\boldsymbol{ PQRS }$ is a parallelogram\\\\

	
(1)  Directional vector of line SR  = $\boldsymbol {(R-S)}$ = $\frac{\boldsymbol{(C-A)}}{2} $\\
Directional vector of line AC  = $\boldsymbol {(C-A)}$\\

It is observed that the constant k is $\frac{1}{2}$

Therefore
\begin{equation}
	SR \parallel AC
\end{equation} 

and from equation 1 
\begin{equation}
	\boldsymbol {SR = \frac{1}{2}AC}    
\end{equation}\\


(2)   To prove PQ = SR\\ 
		From euqation 1\\\\
\begin{equation}
		\boldsymbol{ (Q-P) = (R-S) = \frac{(C-A)}{2}}
\end{equation}
	 



\section{Execution}
The below python code realizes the construction:
\begin{lstlisting}
https://github.com/bhavani360/FWC_assignments
\end{lstlisting}
	
\section*{Construction}
The dimensions of the Quadrilateral ABCD are taken as below\\
{
\setlength\extrarowheight{2pt}
\centering
	\begin{tabular}{|c|c|}
	\hline
	\textbf{symbol}&\textbf{value}\\
	\hline
	r&8\\
	\hline
	$\theta$&pi/2.5\\
	\hline
	d&7\\
	\hline
	A&(0,0)\\
	\hline
	B&(d,0)\\
	\hline
	D&(rcos$\theta$,rsin$\theta$)\\
	\hline
	C&(D/1.5)+B\\
	\hline
\end{tabular}
}
\end{document}
\fi

    \item passing through 
    $\vec{A} = \myvec{2\\2\sqrt{3}}$ and inclined with the x-axis at an angle 
    of 75\textdegree.
\label{chapters/11/10/2/4}
\\
    \solution 
\iffalse
\documentclass[journal,10pt,twocolumn]{article}
\usepackage{graphicx}
\usepackage[margin=0.5in]{geometry}
\usepackage[cmex10]{amsmath}
\usepackage{array}
\usepackage{booktabs}
\usepackage{listings}
\title{\textbf{Line Assignment}}
\author{Bhavani Kanike}
\date{October 2022}

\providecommand{\norm}[1]{\left\lVert#1\right\rVert}
\providecommand{\abs}[1]{\left\vert#1\right\vert}
\let\vec\mathbf
\newcommand{\myvec}[1]{\ensuremath{\begin{pmatrix}#1\end{pmatrix}}}
\newcommand{\mydet}[1]{\ensuremath{\begin{vmatrix}#1\end{vmatrix}}}
\providecommand{\brak}[1]{\ensuremath{\left(#1\right)}}

\begin{document}

\maketitle
\paragraph{\textit{Problem Statement} 
\fi
ABCD is a quadrilateral in which $\vec{P}, \vec{Q}, \vec{R}$ and $\vec{S}$ are mid-points of the sides AB, BC, CD and DA (see Fig \ref{fig:9/8/2/1}). AC is a diagonal. 
		
Show that 
\begin{enumerate}
	\item $SR \parallel AC$ and $SR =\frac{1}{2} AC$
\item $PQ = SR$
\item $PQRS$ is a parallelogram.
\end{enumerate}
 	\begin{figure}
		\centering
 \includegraphics[width=\columnwidth]{chapters/9/8/2/1/figs/line1.pdf}
		\caption{}
		\label{fig:9/8/2/1}
  	\end{figure}
	\solution 
	Using 
	  \eqref{eq:section_formula},
	\begin{align}
		\label{eq:9/8/2/1}
		\begin{split}
		\vec{P} &= \frac{\vec{A}+\vec{B}}{2}\\
 \vec{Q} &= \frac{\vec{C}+\vec{B}}{2}\\
 \vec{R} &= \frac{\vec{C}+\vec{D}}{2}\\
 \vec{S} &= \frac{\vec{D}+\vec{A}}{2}
		\end{split}
	\end{align}
\begin{enumerate}
	\item
	Consequently, 
	\begin{align}
\vec{R}
		-\vec{S} &= \frac{\vec{C}-\vec{A}}{2}
		\\
		\implies SR &\parallel AC
	\end{align}
	Also, 
	\begin{align}
		\norm{\vec{R}
		-\vec{S}} &= \frac{\norm{\vec{C}-\vec{A}}}{2}
		\\
		\implies SR &= \frac{1}{2}AC
	\end{align}
\item 	From 
		\eqref{eq:9/8/2/1},
	\begin{align}
\vec{R}
		-\vec{S} = \vec{Q}-\vec{P}
	\end{align}
	which means that $PQRS$ is a parallelogram and $PQ = SR$.
\end{enumerate}
%
\iffalse
\begin{figure}[h]
\centering
\includegraphics[width=1\columnwidth]
\caption{Figure}
\label{fig:triangle}
\end{figure}

\section*{Solution}

$\boldsymbol Given :$  ABCD is a Quadrilateral P,Q,R and S are the midpoints of line AB,BC,CD,DA.We can obtain the points P,Q,R and S from A,B,C and D and are given by\\\\
\boldmath
\unboldmath
(3) To prove that PQRS is a parallelogram we need to prove  PQ // SR
To prove SR $\parallel$ PQ\\
Direction vector of line SR  $\boldsymbol {(R-S) =  \frac{(C-A)}{2}}$\\\\
Direction vector of line PQ  $\boldsymbol {(Q-P)= \frac{(C-A)}{2}}$\\\\
\begin{equation}
	\boldsymbol {(R-S) = (Q-P) = \frac{(C-A)}{2}}\\
\end{equation}
Since the direction vectors of line SR and PQ are in same direction\\\\
$SR \parallel PQ$\\
Therefore,
$\boldsymbol{ PQRS }$ is a parallelogram\\\\

	
(1)  Directional vector of line SR  = $\boldsymbol {(R-S)}$ = $\frac{\boldsymbol{(C-A)}}{2} $\\
Directional vector of line AC  = $\boldsymbol {(C-A)}$\\

It is observed that the constant k is $\frac{1}{2}$

Therefore
\begin{equation}
	SR \parallel AC
\end{equation} 

and from equation 1 
\begin{equation}
	\boldsymbol {SR = \frac{1}{2}AC}    
\end{equation}\\


(2)   To prove PQ = SR\\ 
		From euqation 1\\\\
\begin{equation}
		\boldsymbol{ (Q-P) = (R-S) = \frac{(C-A)}{2}}
\end{equation}
	 



\section{Execution}
The below python code realizes the construction:
\begin{lstlisting}
https://github.com/bhavani360/FWC_assignments
\end{lstlisting}
	
\section*{Construction}
The dimensions of the Quadrilateral ABCD are taken as below\\
{
\setlength\extrarowheight{2pt}
\centering
	\begin{tabular}{|c|c|}
	\hline
	\textbf{symbol}&\textbf{value}\\
	\hline
	r&8\\
	\hline
	$\theta$&pi/2.5\\
	\hline
	d&7\\
	\hline
	A&(0,0)\\
	\hline
	B&(d,0)\\
	\hline
	D&(rcos$\theta$,rsin$\theta$)\\
	\hline
	C&(D/1.5)+B\\
	\hline
\end{tabular}
}
\end{document}
\fi

\item intersecting the x-axis at a distance of 3 units to the left of origin with slope of -2.
\label{chapters/11/10/2/5}
\\
\solution 
\iffalse
\documentclass[journal,10pt,twocolumn]{article}
\usepackage{graphicx}
\usepackage[margin=0.5in]{geometry}
\usepackage[cmex10]{amsmath}
\usepackage{array}
\usepackage{booktabs}
\usepackage{listings}
\title{\textbf{Line Assignment}}
\author{Bhavani Kanike}
\date{October 2022}

\providecommand{\norm}[1]{\left\lVert#1\right\rVert}
\providecommand{\abs}[1]{\left\vert#1\right\vert}
\let\vec\mathbf
\newcommand{\myvec}[1]{\ensuremath{\begin{pmatrix}#1\end{pmatrix}}}
\newcommand{\mydet}[1]{\ensuremath{\begin{vmatrix}#1\end{vmatrix}}}
\providecommand{\brak}[1]{\ensuremath{\left(#1\right)}}

\begin{document}

\maketitle
\paragraph{\textit{Problem Statement} 
\fi
ABCD is a quadrilateral in which $\vec{P}, \vec{Q}, \vec{R}$ and $\vec{S}$ are mid-points of the sides AB, BC, CD and DA (see Fig \ref{fig:9/8/2/1}). AC is a diagonal. 
		
Show that 
\begin{enumerate}
	\item $SR \parallel AC$ and $SR =\frac{1}{2} AC$
\item $PQ = SR$
\item $PQRS$ is a parallelogram.
\end{enumerate}
 	\begin{figure}
		\centering
 \includegraphics[width=\columnwidth]{chapters/9/8/2/1/figs/line1.pdf}
		\caption{}
		\label{fig:9/8/2/1}
  	\end{figure}
	\solution 
	Using 
	  \eqref{eq:section_formula},
	\begin{align}
		\label{eq:9/8/2/1}
		\begin{split}
		\vec{P} &= \frac{\vec{A}+\vec{B}}{2}\\
 \vec{Q} &= \frac{\vec{C}+\vec{B}}{2}\\
 \vec{R} &= \frac{\vec{C}+\vec{D}}{2}\\
 \vec{S} &= \frac{\vec{D}+\vec{A}}{2}
		\end{split}
	\end{align}
\begin{enumerate}
	\item
	Consequently, 
	\begin{align}
\vec{R}
		-\vec{S} &= \frac{\vec{C}-\vec{A}}{2}
		\\
		\implies SR &\parallel AC
	\end{align}
	Also, 
	\begin{align}
		\norm{\vec{R}
		-\vec{S}} &= \frac{\norm{\vec{C}-\vec{A}}}{2}
		\\
		\implies SR &= \frac{1}{2}AC
	\end{align}
\item 	From 
		\eqref{eq:9/8/2/1},
	\begin{align}
\vec{R}
		-\vec{S} = \vec{Q}-\vec{P}
	\end{align}
	which means that $PQRS$ is a parallelogram and $PQ = SR$.
\end{enumerate}
%
\iffalse
\begin{figure}[h]
\centering
\includegraphics[width=1\columnwidth]
\caption{Figure}
\label{fig:triangle}
\end{figure}

\section*{Solution}

$\boldsymbol Given :$  ABCD is a Quadrilateral P,Q,R and S are the midpoints of line AB,BC,CD,DA.We can obtain the points P,Q,R and S from A,B,C and D and are given by\\\\
\boldmath
\unboldmath
(3) To prove that PQRS is a parallelogram we need to prove  PQ // SR
To prove SR $\parallel$ PQ\\
Direction vector of line SR  $\boldsymbol {(R-S) =  \frac{(C-A)}{2}}$\\\\
Direction vector of line PQ  $\boldsymbol {(Q-P)= \frac{(C-A)}{2}}$\\\\
\begin{equation}
	\boldsymbol {(R-S) = (Q-P) = \frac{(C-A)}{2}}\\
\end{equation}
Since the direction vectors of line SR and PQ are in same direction\\\\
$SR \parallel PQ$\\
Therefore,
$\boldsymbol{ PQRS }$ is a parallelogram\\\\

	
(1)  Directional vector of line SR  = $\boldsymbol {(R-S)}$ = $\frac{\boldsymbol{(C-A)}}{2} $\\
Directional vector of line AC  = $\boldsymbol {(C-A)}$\\

It is observed that the constant k is $\frac{1}{2}$

Therefore
\begin{equation}
	SR \parallel AC
\end{equation} 

and from equation 1 
\begin{equation}
	\boldsymbol {SR = \frac{1}{2}AC}    
\end{equation}\\


(2)   To prove PQ = SR\\ 
		From euqation 1\\\\
\begin{equation}
		\boldsymbol{ (Q-P) = (R-S) = \frac{(C-A)}{2}}
\end{equation}
	 



\section{Execution}
The below python code realizes the construction:
\begin{lstlisting}
https://github.com/bhavani360/FWC_assignments
\end{lstlisting}
	
\section*{Construction}
The dimensions of the Quadrilateral ABCD are taken as below\\
{
\setlength\extrarowheight{2pt}
\centering
	\begin{tabular}{|c|c|}
	\hline
	\textbf{symbol}&\textbf{value}\\
	\hline
	r&8\\
	\hline
	$\theta$&pi/2.5\\
	\hline
	d&7\\
	\hline
	A&(0,0)\\
	\hline
	B&(d,0)\\
	\hline
	D&(rcos$\theta$,rsin$\theta$)\\
	\hline
	C&(D/1.5)+B\\
	\hline
\end{tabular}
}
\end{document}
\fi

\item Find the equation of the line which satisfy the given conditions: Intersecting the y-axis at a distance of 2 units above the origin and making an
angle of $30\degree$ with positive direction of the x-axis.
\\
\solution 
\iffalse
\documentclass[journal,10pt,twocolumn]{article}
\usepackage{graphicx}
\usepackage[margin=0.5in]{geometry}
\usepackage[cmex10]{amsmath}
\usepackage{array}
\usepackage{booktabs}
\usepackage{listings}
\title{\textbf{Line Assignment}}
\author{Bhavani Kanike}
\date{October 2022}

\providecommand{\norm}[1]{\left\lVert#1\right\rVert}
\providecommand{\abs}[1]{\left\vert#1\right\vert}
\let\vec\mathbf
\newcommand{\myvec}[1]{\ensuremath{\begin{pmatrix}#1\end{pmatrix}}}
\newcommand{\mydet}[1]{\ensuremath{\begin{vmatrix}#1\end{vmatrix}}}
\providecommand{\brak}[1]{\ensuremath{\left(#1\right)}}

\begin{document}

\maketitle
\paragraph{\textit{Problem Statement} 
\fi
ABCD is a quadrilateral in which $\vec{P}, \vec{Q}, \vec{R}$ and $\vec{S}$ are mid-points of the sides AB, BC, CD and DA (see Fig \ref{fig:9/8/2/1}). AC is a diagonal. 
		
Show that 
\begin{enumerate}
	\item $SR \parallel AC$ and $SR =\frac{1}{2} AC$
\item $PQ = SR$
\item $PQRS$ is a parallelogram.
\end{enumerate}
 	\begin{figure}
		\centering
 \includegraphics[width=\columnwidth]{chapters/9/8/2/1/figs/line1.pdf}
		\caption{}
		\label{fig:9/8/2/1}
  	\end{figure}
	\solution 
	Using 
	  \eqref{eq:section_formula},
	\begin{align}
		\label{eq:9/8/2/1}
		\begin{split}
		\vec{P} &= \frac{\vec{A}+\vec{B}}{2}\\
 \vec{Q} &= \frac{\vec{C}+\vec{B}}{2}\\
 \vec{R} &= \frac{\vec{C}+\vec{D}}{2}\\
 \vec{S} &= \frac{\vec{D}+\vec{A}}{2}
		\end{split}
	\end{align}
\begin{enumerate}
	\item
	Consequently, 
	\begin{align}
\vec{R}
		-\vec{S} &= \frac{\vec{C}-\vec{A}}{2}
		\\
		\implies SR &\parallel AC
	\end{align}
	Also, 
	\begin{align}
		\norm{\vec{R}
		-\vec{S}} &= \frac{\norm{\vec{C}-\vec{A}}}{2}
		\\
		\implies SR &= \frac{1}{2}AC
	\end{align}
\item 	From 
		\eqref{eq:9/8/2/1},
	\begin{align}
\vec{R}
		-\vec{S} = \vec{Q}-\vec{P}
	\end{align}
	which means that $PQRS$ is a parallelogram and $PQ = SR$.
\end{enumerate}
%
\iffalse
\begin{figure}[h]
\centering
\includegraphics[width=1\columnwidth]
\caption{Figure}
\label{fig:triangle}
\end{figure}

\section*{Solution}

$\boldsymbol Given :$  ABCD is a Quadrilateral P,Q,R and S are the midpoints of line AB,BC,CD,DA.We can obtain the points P,Q,R and S from A,B,C and D and are given by\\\\
\boldmath
\unboldmath
(3) To prove that PQRS is a parallelogram we need to prove  PQ // SR
To prove SR $\parallel$ PQ\\
Direction vector of line SR  $\boldsymbol {(R-S) =  \frac{(C-A)}{2}}$\\\\
Direction vector of line PQ  $\boldsymbol {(Q-P)= \frac{(C-A)}{2}}$\\\\
\begin{equation}
	\boldsymbol {(R-S) = (Q-P) = \frac{(C-A)}{2}}\\
\end{equation}
Since the direction vectors of line SR and PQ are in same direction\\\\
$SR \parallel PQ$\\
Therefore,
$\boldsymbol{ PQRS }$ is a parallelogram\\\\

	
(1)  Directional vector of line SR  = $\boldsymbol {(R-S)}$ = $\frac{\boldsymbol{(C-A)}}{2} $\\
Directional vector of line AC  = $\boldsymbol {(C-A)}$\\

It is observed that the constant k is $\frac{1}{2}$

Therefore
\begin{equation}
	SR \parallel AC
\end{equation} 

and from equation 1 
\begin{equation}
	\boldsymbol {SR = \frac{1}{2}AC}    
\end{equation}\\


(2)   To prove PQ = SR\\ 
		From euqation 1\\\\
\begin{equation}
		\boldsymbol{ (Q-P) = (R-S) = \frac{(C-A)}{2}}
\end{equation}
	 



\section{Execution}
The below python code realizes the construction:
\begin{lstlisting}
https://github.com/bhavani360/FWC_assignments
\end{lstlisting}
	
\section*{Construction}
The dimensions of the Quadrilateral ABCD are taken as below\\
{
\setlength\extrarowheight{2pt}
\centering
	\begin{tabular}{|c|c|}
	\hline
	\textbf{symbol}&\textbf{value}\\
	\hline
	r&8\\
	\hline
	$\theta$&pi/2.5\\
	\hline
	d&7\\
	\hline
	A&(0,0)\\
	\hline
	B&(d,0)\\
	\hline
	D&(rcos$\theta$,rsin$\theta$)\\
	\hline
	C&(D/1.5)+B\\
	\hline
\end{tabular}
}
\end{document}
\fi

\item Find the equation of line passing through the points $\myvec{-1\\1}$ and $\myvec{2\\-4}$.
\\
\solution 
\iffalse
\documentclass[journal,10pt,twocolumn]{article}
\usepackage{graphicx}
\usepackage[margin=0.5in]{geometry}
\usepackage[cmex10]{amsmath}
\usepackage{array}
\usepackage{booktabs}
\usepackage{listings}
\title{\textbf{Line Assignment}}
\author{Bhavani Kanike}
\date{October 2022}

\providecommand{\norm}[1]{\left\lVert#1\right\rVert}
\providecommand{\abs}[1]{\left\vert#1\right\vert}
\let\vec\mathbf
\newcommand{\myvec}[1]{\ensuremath{\begin{pmatrix}#1\end{pmatrix}}}
\newcommand{\mydet}[1]{\ensuremath{\begin{vmatrix}#1\end{vmatrix}}}
\providecommand{\brak}[1]{\ensuremath{\left(#1\right)}}

\begin{document}

\maketitle
\paragraph{\textit{Problem Statement} 
\fi
ABCD is a quadrilateral in which $\vec{P}, \vec{Q}, \vec{R}$ and $\vec{S}$ are mid-points of the sides AB, BC, CD and DA (see Fig \ref{fig:9/8/2/1}). AC is a diagonal. 
		
Show that 
\begin{enumerate}
	\item $SR \parallel AC$ and $SR =\frac{1}{2} AC$
\item $PQ = SR$
\item $PQRS$ is a parallelogram.
\end{enumerate}
 	\begin{figure}
		\centering
 \includegraphics[width=\columnwidth]{chapters/9/8/2/1/figs/line1.pdf}
		\caption{}
		\label{fig:9/8/2/1}
  	\end{figure}
	\solution 
	Using 
	  \eqref{eq:section_formula},
	\begin{align}
		\label{eq:9/8/2/1}
		\begin{split}
		\vec{P} &= \frac{\vec{A}+\vec{B}}{2}\\
 \vec{Q} &= \frac{\vec{C}+\vec{B}}{2}\\
 \vec{R} &= \frac{\vec{C}+\vec{D}}{2}\\
 \vec{S} &= \frac{\vec{D}+\vec{A}}{2}
		\end{split}
	\end{align}
\begin{enumerate}
	\item
	Consequently, 
	\begin{align}
\vec{R}
		-\vec{S} &= \frac{\vec{C}-\vec{A}}{2}
		\\
		\implies SR &\parallel AC
	\end{align}
	Also, 
	\begin{align}
		\norm{\vec{R}
		-\vec{S}} &= \frac{\norm{\vec{C}-\vec{A}}}{2}
		\\
		\implies SR &= \frac{1}{2}AC
	\end{align}
\item 	From 
		\eqref{eq:9/8/2/1},
	\begin{align}
\vec{R}
		-\vec{S} = \vec{Q}-\vec{P}
	\end{align}
	which means that $PQRS$ is a parallelogram and $PQ = SR$.
\end{enumerate}
%
\iffalse
\begin{figure}[h]
\centering
\includegraphics[width=1\columnwidth]
\caption{Figure}
\label{fig:triangle}
\end{figure}

\section*{Solution}

$\boldsymbol Given :$  ABCD is a Quadrilateral P,Q,R and S are the midpoints of line AB,BC,CD,DA.We can obtain the points P,Q,R and S from A,B,C and D and are given by\\\\
\boldmath
\unboldmath
(3) To prove that PQRS is a parallelogram we need to prove  PQ // SR
To prove SR $\parallel$ PQ\\
Direction vector of line SR  $\boldsymbol {(R-S) =  \frac{(C-A)}{2}}$\\\\
Direction vector of line PQ  $\boldsymbol {(Q-P)= \frac{(C-A)}{2}}$\\\\
\begin{equation}
	\boldsymbol {(R-S) = (Q-P) = \frac{(C-A)}{2}}\\
\end{equation}
Since the direction vectors of line SR and PQ are in same direction\\\\
$SR \parallel PQ$\\
Therefore,
$\boldsymbol{ PQRS }$ is a parallelogram\\\\

	
(1)  Directional vector of line SR  = $\boldsymbol {(R-S)}$ = $\frac{\boldsymbol{(C-A)}}{2} $\\
Directional vector of line AC  = $\boldsymbol {(C-A)}$\\

It is observed that the constant k is $\frac{1}{2}$

Therefore
\begin{equation}
	SR \parallel AC
\end{equation} 

and from equation 1 
\begin{equation}
	\boldsymbol {SR = \frac{1}{2}AC}    
\end{equation}\\


(2)   To prove PQ = SR\\ 
		From euqation 1\\\\
\begin{equation}
		\boldsymbol{ (Q-P) = (R-S) = \frac{(C-A)}{2}}
\end{equation}
	 



\section{Execution}
The below python code realizes the construction:
\begin{lstlisting}
https://github.com/bhavani360/FWC_assignments
\end{lstlisting}
	
\section*{Construction}
The dimensions of the Quadrilateral ABCD are taken as below\\
{
\setlength\extrarowheight{2pt}
\centering
	\begin{tabular}{|c|c|}
	\hline
	\textbf{symbol}&\textbf{value}\\
	\hline
	r&8\\
	\hline
	$\theta$&pi/2.5\\
	\hline
	d&7\\
	\hline
	A&(0,0)\\
	\hline
	B&(d,0)\\
	\hline
	D&(rcos$\theta$,rsin$\theta$)\\
	\hline
	C&(D/1.5)+B\\
	\hline
\end{tabular}
}
\end{document}
\fi

\item Find the equation of line whose perpendicular distance from the origin is 5 units and the angle made by the perpendicular with the positive $x$-axis is $30\degree$.
\label{chapters/11/10/2/8}
\\
\solution
\iffalse
\documentclass[journal,10pt,twocolumn]{article}
\usepackage{graphicx}
\usepackage[margin=0.5in]{geometry}
\usepackage[cmex10]{amsmath}
\usepackage{array}
\usepackage{booktabs}
\usepackage{listings}
\title{\textbf{Line Assignment}}
\author{Bhavani Kanike}
\date{October 2022}

\providecommand{\norm}[1]{\left\lVert#1\right\rVert}
\providecommand{\abs}[1]{\left\vert#1\right\vert}
\let\vec\mathbf
\newcommand{\myvec}[1]{\ensuremath{\begin{pmatrix}#1\end{pmatrix}}}
\newcommand{\mydet}[1]{\ensuremath{\begin{vmatrix}#1\end{vmatrix}}}
\providecommand{\brak}[1]{\ensuremath{\left(#1\right)}}

\begin{document}

\maketitle
\paragraph{\textit{Problem Statement} 
\fi
ABCD is a quadrilateral in which $\vec{P}, \vec{Q}, \vec{R}$ and $\vec{S}$ are mid-points of the sides AB, BC, CD and DA (see Fig \ref{fig:9/8/2/1}). AC is a diagonal. 
		
Show that 
\begin{enumerate}
	\item $SR \parallel AC$ and $SR =\frac{1}{2} AC$
\item $PQ = SR$
\item $PQRS$ is a parallelogram.
\end{enumerate}
 	\begin{figure}
		\centering
 \includegraphics[width=\columnwidth]{chapters/9/8/2/1/figs/line1.pdf}
		\caption{}
		\label{fig:9/8/2/1}
  	\end{figure}
	\solution 
	Using 
	  \eqref{eq:section_formula},
	\begin{align}
		\label{eq:9/8/2/1}
		\begin{split}
		\vec{P} &= \frac{\vec{A}+\vec{B}}{2}\\
 \vec{Q} &= \frac{\vec{C}+\vec{B}}{2}\\
 \vec{R} &= \frac{\vec{C}+\vec{D}}{2}\\
 \vec{S} &= \frac{\vec{D}+\vec{A}}{2}
		\end{split}
	\end{align}
\begin{enumerate}
	\item
	Consequently, 
	\begin{align}
\vec{R}
		-\vec{S} &= \frac{\vec{C}-\vec{A}}{2}
		\\
		\implies SR &\parallel AC
	\end{align}
	Also, 
	\begin{align}
		\norm{\vec{R}
		-\vec{S}} &= \frac{\norm{\vec{C}-\vec{A}}}{2}
		\\
		\implies SR &= \frac{1}{2}AC
	\end{align}
\item 	From 
		\eqref{eq:9/8/2/1},
	\begin{align}
\vec{R}
		-\vec{S} = \vec{Q}-\vec{P}
	\end{align}
	which means that $PQRS$ is a parallelogram and $PQ = SR$.
\end{enumerate}
%
\iffalse
\begin{figure}[h]
\centering
\includegraphics[width=1\columnwidth]
\caption{Figure}
\label{fig:triangle}
\end{figure}

\section*{Solution}

$\boldsymbol Given :$  ABCD is a Quadrilateral P,Q,R and S are the midpoints of line AB,BC,CD,DA.We can obtain the points P,Q,R and S from A,B,C and D and are given by\\\\
\boldmath
\unboldmath
(3) To prove that PQRS is a parallelogram we need to prove  PQ // SR
To prove SR $\parallel$ PQ\\
Direction vector of line SR  $\boldsymbol {(R-S) =  \frac{(C-A)}{2}}$\\\\
Direction vector of line PQ  $\boldsymbol {(Q-P)= \frac{(C-A)}{2}}$\\\\
\begin{equation}
	\boldsymbol {(R-S) = (Q-P) = \frac{(C-A)}{2}}\\
\end{equation}
Since the direction vectors of line SR and PQ are in same direction\\\\
$SR \parallel PQ$\\
Therefore,
$\boldsymbol{ PQRS }$ is a parallelogram\\\\

	
(1)  Directional vector of line SR  = $\boldsymbol {(R-S)}$ = $\frac{\boldsymbol{(C-A)}}{2} $\\
Directional vector of line AC  = $\boldsymbol {(C-A)}$\\

It is observed that the constant k is $\frac{1}{2}$

Therefore
\begin{equation}
	SR \parallel AC
\end{equation} 

and from equation 1 
\begin{equation}
	\boldsymbol {SR = \frac{1}{2}AC}    
\end{equation}\\


(2)   To prove PQ = SR\\ 
		From euqation 1\\\\
\begin{equation}
		\boldsymbol{ (Q-P) = (R-S) = \frac{(C-A)}{2}}
\end{equation}
	 



\section{Execution}
The below python code realizes the construction:
\begin{lstlisting}
https://github.com/bhavani360/FWC_assignments
\end{lstlisting}
	
\section*{Construction}
The dimensions of the Quadrilateral ABCD are taken as below\\
{
\setlength\extrarowheight{2pt}
\centering
	\begin{tabular}{|c|c|}
	\hline
	\textbf{symbol}&\textbf{value}\\
	\hline
	r&8\\
	\hline
	$\theta$&pi/2.5\\
	\hline
	d&7\\
	\hline
	A&(0,0)\\
	\hline
	B&(d,0)\\
	\hline
	D&(rcos$\theta$,rsin$\theta$)\\
	\hline
	C&(D/1.5)+B\\
	\hline
\end{tabular}
}
\end{document}
\fi

\item 
\label{chapters/11/10/2/9}

\documentclass[journal,12pt,twocolumn]{IEEEtran}
\usepackage{graphicx}
\usepackage[margin=0.5in]{geometry}
\graphicspath{{./figs/}}{}
\usepackage{amsmath,amssymb,amsfonts,amsthm}
\newcommand{\myvec}[1]{\ensuremath{\begin{pmatrix}#1\end{pmatrix}}}
\usepackage{listings}
\usepackage{watermark}
\usepackage{titlesec}
\let\vec\mathbf
\lstset{
frame=single, 
breaklines=true,
columns=fullflexible
}
%\thiswatermark{\centering \put(0,-105.0){\includegraphics[scale=0.5]{iith.png}} }

\title{\mytitle}
\title{
Matrix Assignment - Lines
}
\author{Adarsh Kumar (FWC22068)}
\begin{document}
\maketitle
\tableofcontents
\bigskip


\section{\textbf{Problem}}
The Vertices of Triangle PQR is P(2,1), Q(-2,3), R(4,5) . Find the equation of the Median Through R.\\


\section{\textbf{Solution}}
Given the Vertices are :\\
\linebreak
$\vec{P} = \myvec{2 \\ 1}$ \hspace{5mm}
$\vec{Q} = \myvec{-2 \\ 3}$ \hspace{5mm}
$\vec{R} = \myvec{4 \\ 5}$ \hspace{5mm}
\linebreak


We know that the Median through R ,
will divide the side PQ in two equal parts.
\\
We know that the the median through R will
divide or intersect the side PQ into two equal parts.
\\
So , By using section formula , we can find the Coordinates of the point A(say) on PQ where the median intersect the side PQ.

\textbf{Section Formula :}

\begin{equation}
 Point A = \frac{P + K(Q)}{1+K}  \label{eq-1}
\end{equation}
\\
where PQ is a line and P and Q is the coordinates and K is the ratio in which the line is being divided.
\\

Now , we know that the Median from R will divide the side PQ in two equal parts 
\\( i.e in th ratio 1 : 1 )

So ,  by using Section Formula,

\begin{equation}
A = \frac{P + K(Q)}{1+K}
\end{equation}

  where , K=1 \\ 
  \linebreak
$\vec{P} = \myvec{2 \\ 1}$ \hspace{5mm}
$\vec{Q} = \myvec{-2 \\ 3}$ \hspace{5mm}

\begin{equation}
\vec{A} = \frac{{\myvec{2\\1}}  + 1{\myvec{-2\\3}}}{1+1}
\end{equation}                         

\begin{equation}
\vec{A} = {\myvec{0\\2}}
\end{equation} 
Now ,our Aim is to find the equation of Median(line AR)

So, Now we have two points \\
\linebreak
$\vec{A} = \myvec{0 \\ 2}$ \hspace{5mm}
$\vec{R} = \myvec{4 \\ 5}$ \\
\linebreak
We know that ,\\
The Parametric Equation of line is :\\
\begin{equation}
\vec{X} = \vec{A}+\lambda\vec{m}
\end{equation}
Where \textbf{m} is the direction vector of the line\\
\linebreak
So , the Direction Vector \textbf{m} of line AR is :
\begin{equation}
\vec{m} = \myvec{{\vec{R} - \vec{A}}}
\end{equation}
\begin{equation}
\vec{m} = \myvec{4 \\ 5} -\myvec{0 \\ 2}
\end{equation}
\begin{equation}
\vec{m} = \myvec{4 \\ 3}
\end{equation}
Therefore the equation of line AR will be:
\begin{equation}
\vec{X} = \myvec{0 \\ 2}+\lambda\myvec{ 4 \\ 3}
\end{equation}
Equation 9 ,represents the equation of line AR in Parametric Form . 

\newpage


\section{\textbf{Figure}}
\begin{figure}[h]
    \centering
\includegraphics[width=\columnwidth]{line.png}
    \label{fig:my_label}
\end{figure}


\section*{Construction}
The dimensions of the Triangle made by using Python  are taken as below\\
\linebreak
{
\centering
	\begin{tabular}{|c|c|}
	\hline
	\textbf{vertex}&\textbf{co-ordinates}\\
	\hline
	P&(2,1)\\
	\hline
	Q&(-2,3)\\
	\hline
	R&(4,5)\\
	\hline
	A&(0,2)\\
	\hline
\end{tabular}
}
\section{\textbf{Code Link}}

\begin{lstlisting}
https://github.com/aadrshptel/Fwc_module1/tree/main/Assignments/Matrix%20assignments/Lines/codes
\end{lstlisting}
Execute the code by using the command\\
\textbf{python3 line.py}



\end{document}

\item 
\label{chapters/11/10/2/10}
\def\mytitle{LINE USING PYTHON}
\def\myauthor{Mukesh Guptha.CH}
\def\contact{mukeshchinta1313@gmail.com}
\def\mymodule{Future Wireless Communication (FWC)}
\documentclass[10pt, a4paper]{article}
\usepackage[a4paper,outer=1.5cm,inner=1.5cm,top=1.75cm,bottom=1.5cm]{geometry}
\twocolumn
\usepackage{graphicx}
\graphicspath{{./images/}}
\usepackage[colorlinks,linkcolor={black},citecolor={blue!80!black},urlcolor={blue!80!black}]{hyperref}
\usepackage[parfill]{parskip}
\usepackage{lmodern}
\usepackage{tikz}
	\usepackage{physics}
%\documentclass[tikz, border=2mm]{standalone}
\usepackage{karnaugh-map}
%\documentclass{article}
\usepackage{tabularx}
\usepackage{circuitikz}
\usetikzlibrary{calc}
\usepackage{amsmath}
\usepackage{amssymb}
\renewcommand*\familydefault{\sfdefault}
\usepackage{watermark}
\usepackage{lipsum}
\usepackage{xcolor}
\usepackage{listings}
\usepackage{float}
\usepackage{titlesec}
\providecommand{\mtx}[1]{\mathbf{#1}}
\titlespacing{\subsection}{1pt}{\parskip}{3pt}
\titlespacing{\subsubsection}{0pt}{\parskip}{-\parskip}
\titlespacing{\paragraph}{0pt}{\parskip}{\parskip}
\newcommand{\figuremacro}[5]{
    \begin{figure}[#1]
        \centering
        \includegraphics[width=#5\columnwidth]{#2}
        \caption[#3]{\textbf{#3}#4}
        \label{fig:#2}
    \end{figure}
}
\newcommand{\myvec}[1]{\ensuremath{\begin{pmatrix}#1\end{pmatrix}}}
\let\vec\mathbf
\lstset{
frame=single, 
breaklines=true,
columns=fullflexible
}

\title{\mytitle}
\author{\myauthor\hspace{1em}\\\contact\\FWC22069\hspace{6.5em}IITH\hspace{0.5em}\mymodule\hspace{6em}ASSIGN-4}
\date{}
\begin{document}
	\maketitle
 \paragraph*{\large Problem Statement}
$-$ \textbf{ Find the equation of the line passing through  (-3,5) and perpendicular to the line through the points (2,5) and (-3,6).}
 
\begin{figure}[h]
\centering
\includegraphics[width=1\columnwidth]{Figure_1.png}
\caption{perpendicular intersection}
\end{figure}
	\section*{Construction}
\vspace{2mm}
 the input parameters are as follows
{
\setlength\extrarowheight{4pt}
 
 \begin{tabular}{|c|c|c|}
	\hline
	\textbf{Symbol}&\textbf{Value}&\textbf{Description}\\
	\hline
 c&$
	\begin{pmatrix}
		5\\
		-1\\
	\end{pmatrix}$
	&coefficients of line \\
	\hline
 d&$
	\begin{pmatrix}
		20\\
	\end{pmatrix}$
	&constants\\
	\hline
 \end{tabular}
}
\section*{\large solution}

\subsection*{\large part 1}
let us take A=(2,5),B=(-3,6) and P=(-3,5).Directional vector  of the points\vspace{4mm}m=B-A\\
\begin{equation}
m=\begin{pmatrix}
    2\\
    5\\
\end{pmatrix}-\begin{pmatrix}
    -3\\
    6\\
\end{pmatrix}\hspace{2em}
m=\begin{pmatrix}
    5\\
    -1\\
\end{pmatrix}
\label{eq-1}
\end{equation}


\begin{eqnarray}
\vec{m^t\vec{(X-P)}}=0
\end{eqnarray}
\begin{equation}
\begin{pmatrix}
    5 &-1\\
\end{pmatrix}\begin{pmatrix}
    X-P\\
\end{pmatrix}
    \label{eq-3}
\end{equation}

\begin{equation}
 \begin{pmatrix}
    5 & -1\\
\end{pmatrix}\begin{pmatrix}
    x+3\\
    y-5\\
\end{pmatrix}
\label{eq-4}
\end{equation}
The required line equation is 
\begin{equation}
 5\vec{x}-\vec{y}+20=0
\label{eq-5}
\end{equation}
\end{document}
\item 
\label{chapters/11/10/2/11}
\documentclass[journal,12pt,twocolumn]{article}
\usepackage{graphicx}
\graphicspath{{./figs/}}{}
\usepackage{amsmath,amssymb,amsfonts,amsthm}
\newcommand{\myvec}[1]{\ensuremath{\begin{pmatrix}#1\end{pmatrix}}}
\let\vec\mathbf
\title{
Matrix-Lines
}
\author{SHREYASH CHANDRA PUTTA}
\begin{document}
\maketitle
\tableofcontents

\section{Problem Statement}
A line perpendicular to the line segement joining the points (1,0)and(2,3)divides it in the ratio 1:n . find the equation of the line?\\
(note: we are taking n as user Input) .

% 

\begin{table}[h]
    \centering
    \begin{tabular}{|c|c|c|}
       \hline
       \textbf{Symbol}&\textbf{Value}&\textbf{Description}  \\
       \hline
	    $\vec{P}$ & $\myvec{
		    1\\
		    0}$
	    & given point\\
        \hline
	    $\vec{Q}$ & $\myvec{2\\3}$
 & given point\\
        \hline
	    $\vec{R}$ & $\myvec{
  \frac{2+n}{1+n}\\
  \frac{3}{1+n}}$
 & intersecting point  \\
       \hline
    \end{tabular}
    \caption{Parameters}
    \label{tab:my_label}
\end{table}

%\section{Construction}

\begin{figure}[h]
    \centering
\includegraphics[width=\columnwidth]{fig/linefig.pdf}
    \caption{Equation of the required Straight Line}
    \label{fig:my_label}
\end{figure}




\section{Solution}

Given that resultant will divide the equation of line in the ratio 1:n and the line is perpendicular to line segment joining the points (1,0)and(2,3)  ) \\
%so, b = 9 - a  \\
\\
Let ${\vec{P}}$=$\myvec{
  1\\
  0}$
 and ${\vec{Q}}$=$\myvec{
  2\\
  3}$
\\
\\
Equation of line is ${\vec{n^{\top}}\vec{X}} = c$.
\\
\\
We know if 2 points of the linesegment is given then,\\
 %The Equation of line through ${\vec{P}}$ is\\
%\begin{equation}
%	\vec{n^{\top}}
%	\myvec{
 % a\\
  %0}
  %= c \label{eq-1}
%\end{equation}
\\
Direction vector of line joining two points  ${\vec{P}}$ ${\vec{Q}}$ is given by\\

\begin{equation}
	\vec{M}=
     \vec{Q
 }-  \vec{P
 }
  \label{eq-2}
\end{equation}
\\

\begin{equation}
	\vec{M}=
     \myvec{
  2\\
  3
 }-  \myvec{
  1\\
  0
 }
  \label{eq-2}
\end{equation}
\\
\\
\begin{equation}
	\vec{M}=
     \myvec{
  1\\
  3
 }
   \label{eq-2}
\end{equation}
\\
We know, that position or  directional vector of points P and Q line segement used as the normal vector
\\
\\
 The general equation of the required perpendicular line is
 ${\vec{M^{\top}}\vec{X}} = c$.
 \\
 \\
 The perpendicular line cutting a line segment P and Q in ratio 1:n is passes through the point R.
 
 \begin{equation}
	 \vec{R}=\frac{\vec{Q}+n\vec{P}}{1+n}
	 \label{eq-4}
\end{equation}
 \\
Equation of line passing through ${\vec{R}}$ is\\
\begin{equation}
	\vec{M^{\top}}(\vec{X}-\vec{R})=0
	 \label{eq-4}
\end{equation}
\\
\begin{equation}
	 \vec{M^{\top}}
	 \vec{X} - \vec{M^{\top}}
	 \vec{R} = 0
	 \label{eq-5}
\end{equation}
 \\
 From eq4, eq6 and eq3 we can find the required Perpenducular line equation. 
 \begin{equation}
	   \myvec{
  1\  3}\vec{X}
	 = \myvec{
  1\ 3}\myvec{
  \frac{2+n}{1+n}\\
  \frac{3}{1+n}} 
  \label{eq-5}
\end{equation}
\\
Therefore the equation of a line perpendicular to the given line segement divides it in the ratio 1:n is:
 \begin{equation}
	   \myvec{
  1\  3}\vec{X}
	 = \frac{11+n}{1+n} 
  \label{eq-5}
\end{equation}



 
\section{Software}
Download the following code using,
\begin{table}[h]
    \centering
    \begin{tabular}{|c|}
    \hline \\
         svn co https://github.com/chanduputta/ \\FWC-Module1Assignments/blob/\\main/assignment4/line/lines3.py  \\
         \\
\hline
    \end{tabular}
\end{table}
\\
and execute the code by using command
\begin{center}
	\textbf{cmd:}
{Python3  lines3.py}\\
	\textbf{Then,}
{input your required n value}
\end{center}

\section{Conclusion}
\begin{center}
We found the equation of a line perpendicular to the line segement joining the points (1,0)and(2,3) divides it in the ratio 1:n .
\end{center}
\end{document}

\item 
\label{chapters/11/10/2/12}
\iffalse
\documentclass[12pt]{article}
\usepackage{graphicx}
\usepackage[none]{hyphenat}
\usepackage{graphicx}
\usepackage{listings}
\usepackage[english]{babel}
\usepackage{graphicx}
\usepackage{caption} 
\usepackage{hyperref}
\usepackage{booktabs}
\usepackage{array}
\usepackage{amsmath}   % for having text in math mode
\usepackage{extarrows} % for Row operations arrows
\usepackage{listings}
\lstset{
  frame=single,
  breaklines=true
}
  
%Following 2 lines were added to remove the blank page at the beginning
\usepackage{atbegshi}% http://ctan.org/pkg/atbegshi
\AtBeginDocument{\AtBeginShipoutNext{\AtBeginShipoutDiscard}}


%New macro definitions
\newcommand{\mydet}[1]{\ensuremath{\begin{vmatrix}#1\end{vmatrix}}}
\providecommand{\brak}[1]{\ensuremath{\left(#1\right)}}
\providecommand{\norm}[1]{\left\lVert#1\right\rVert}
\newcommand{\solution}{\noindent \textbf{Solution: }}
\newcommand{\myvec}[1]{\ensuremath{\begin{pmatrix}#1\end{pmatrix}}}
\let\vec\mathbf

\begin{document}

\begin{center}
\title{\textbf{Equation  of Line}}
\date{\vspace{-5ex}} %Not to print date automatically
\maketitle
\end{center}
\setcounter{page}{1}

\section{11$^{th}$ Maths - Chapter 10}
This is Problem-12 from Exercise 10.2
\begin{enumerate}
		\fi
\item Find the equation of a line that cuts off equal intercepts on the coordinate axes and passes through the point $(2,3)$.  
	\\
\solution 
Let $\vec{P}(a,0), \text{ and } \vec{Q}(0,a)$ be the 2 points on x and y-axes respectively having $a$ as the intercept on both the axes. We know that the the direction vector $\vec{m}$ of the line joining two points $\vec{P}, \vec{Q}$ is given by  
\begin{align}
\vec{m} &=   \vec{P} - \vec{Q} \\
        &=   \myvec{
		a \\
		0 
		} - \myvec{
		   0 \\
		   a
		}  = a\myvec{ 
                     1 \\
		   -1 
        		}  \equiv \myvec{
                           1 \\
			   -1 
		         } 
\end{align}
Thus, the normal vector $\vec{n}$ to the line is given as
\begin{align}
\vec{n} &=  \myvec{
		     1 \\
		     1
	     } 
\end{align}
The equation of a line with normal vector $\vec{n}$ and passing through a point $\vec{A}(2,3)$ is given by
\begin{align}
	\vec{n}^\top\brak{\vec{x}-\vec{A}} &= 0 \\
	\myvec { 1 & 1 } \brak{ \vec{ x  - \myvec{ 2 \\
                                   3
			     }
		}}  &= 0  \\
	\myvec{ 1 & 1} \vec{x} -5 &= 0 \\
        \label{eq:11/10/2/12/1}
	\myvec{ 1 & 1} \vec{x}  &= 5 
\end{align}
To find the intercepts, we know that, since $\vec{P} \text{ and } \vec{Q}$ lie on the straight line, they should satisfy \eqref{eq:11/10/2/12/1}.
\begin{align}
	\myvec{ 1 & 1} \vec{P}  &= 5 \\
	\myvec{ 1 & 1} \myvec{a \\
	                      0 }  &= 5 \\ 
	a + 0 &= 5 \\
	a &= 5 
\end{align}
Both $\vec{P} \text{ and } \vec{Q}$ have the same intercept value, hence the intercept on both x and y-axes is 5 units. The line segment is as shown in Fig. \ref{fig:11/10/2/12/Fig1}.
\begin{figure}[!h]
	\begin{center}
		\includegraphics[width=\columnwidth]{chapters/11/10/2/12/figs/problem12.pdf}
	\end{center}
\caption{}
\label{fig:11/10/2/12/Fig1}
\end{figure}


\item 
\label{chapters/11/10/2/13}
\iffalse
\documentclass[journal,10pt,twocolumn]{article}
\usepackage{graphicx}
\usepackage[margin=0.5in]{geometry}
\usepackage[cmex10]{amsmath}
\usepackage{array}
\usepackage{booktabs}
\usepackage{listings}
\title{\textbf{Line Assignment}}
\author{Bhavani Kanike}
\date{October 2022}

\providecommand{\norm}[1]{\left\lVert#1\right\rVert}
\providecommand{\abs}[1]{\left\vert#1\right\vert}
\let\vec\mathbf
\newcommand{\myvec}[1]{\ensuremath{\begin{pmatrix}#1\end{pmatrix}}}
\newcommand{\mydet}[1]{\ensuremath{\begin{vmatrix}#1\end{vmatrix}}}
\providecommand{\brak}[1]{\ensuremath{\left(#1\right)}}

\begin{document}

\maketitle
\paragraph{\textit{Problem Statement} 
\fi
ABCD is a quadrilateral in which $\vec{P}, \vec{Q}, \vec{R}$ and $\vec{S}$ are mid-points of the sides AB, BC, CD and DA (see Fig \ref{fig:9/8/2/1}). AC is a diagonal. 
		
Show that 
\begin{enumerate}
	\item $SR \parallel AC$ and $SR =\frac{1}{2} AC$
\item $PQ = SR$
\item $PQRS$ is a parallelogram.
\end{enumerate}
 	\begin{figure}
		\centering
 \includegraphics[width=\columnwidth]{chapters/9/8/2/1/figs/line1.pdf}
		\caption{}
		\label{fig:9/8/2/1}
  	\end{figure}
	\solution 
	Using 
	  \eqref{eq:section_formula},
	\begin{align}
		\label{eq:9/8/2/1}
		\begin{split}
		\vec{P} &= \frac{\vec{A}+\vec{B}}{2}\\
 \vec{Q} &= \frac{\vec{C}+\vec{B}}{2}\\
 \vec{R} &= \frac{\vec{C}+\vec{D}}{2}\\
 \vec{S} &= \frac{\vec{D}+\vec{A}}{2}
		\end{split}
	\end{align}
\begin{enumerate}
	\item
	Consequently, 
	\begin{align}
\vec{R}
		-\vec{S} &= \frac{\vec{C}-\vec{A}}{2}
		\\
		\implies SR &\parallel AC
	\end{align}
	Also, 
	\begin{align}
		\norm{\vec{R}
		-\vec{S}} &= \frac{\norm{\vec{C}-\vec{A}}}{2}
		\\
		\implies SR &= \frac{1}{2}AC
	\end{align}
\item 	From 
		\eqref{eq:9/8/2/1},
	\begin{align}
\vec{R}
		-\vec{S} = \vec{Q}-\vec{P}
	\end{align}
	which means that $PQRS$ is a parallelogram and $PQ = SR$.
\end{enumerate}
%
\iffalse
\begin{figure}[h]
\centering
\includegraphics[width=1\columnwidth]
\caption{Figure}
\label{fig:triangle}
\end{figure}

\section*{Solution}

$\boldsymbol Given :$  ABCD is a Quadrilateral P,Q,R and S are the midpoints of line AB,BC,CD,DA.We can obtain the points P,Q,R and S from A,B,C and D and are given by\\\\
\boldmath
\unboldmath
(3) To prove that PQRS is a parallelogram we need to prove  PQ // SR
To prove SR $\parallel$ PQ\\
Direction vector of line SR  $\boldsymbol {(R-S) =  \frac{(C-A)}{2}}$\\\\
Direction vector of line PQ  $\boldsymbol {(Q-P)= \frac{(C-A)}{2}}$\\\\
\begin{equation}
	\boldsymbol {(R-S) = (Q-P) = \frac{(C-A)}{2}}\\
\end{equation}
Since the direction vectors of line SR and PQ are in same direction\\\\
$SR \parallel PQ$\\
Therefore,
$\boldsymbol{ PQRS }$ is a parallelogram\\\\

	
(1)  Directional vector of line SR  = $\boldsymbol {(R-S)}$ = $\frac{\boldsymbol{(C-A)}}{2} $\\
Directional vector of line AC  = $\boldsymbol {(C-A)}$\\

It is observed that the constant k is $\frac{1}{2}$

Therefore
\begin{equation}
	SR \parallel AC
\end{equation} 

and from equation 1 
\begin{equation}
	\boldsymbol {SR = \frac{1}{2}AC}    
\end{equation}\\


(2)   To prove PQ = SR\\ 
		From euqation 1\\\\
\begin{equation}
		\boldsymbol{ (Q-P) = (R-S) = \frac{(C-A)}{2}}
\end{equation}
	 



\section{Execution}
The below python code realizes the construction:
\begin{lstlisting}
https://github.com/bhavani360/FWC_assignments
\end{lstlisting}
	
\section*{Construction}
The dimensions of the Quadrilateral ABCD are taken as below\\
{
\setlength\extrarowheight{2pt}
\centering
	\begin{tabular}{|c|c|}
	\hline
	\textbf{symbol}&\textbf{value}\\
	\hline
	r&8\\
	\hline
	$\theta$&pi/2.5\\
	\hline
	d&7\\
	\hline
	A&(0,0)\\
	\hline
	B&(d,0)\\
	\hline
	D&(rcos$\theta$,rsin$\theta$)\\
	\hline
	C&(D/1.5)+B\\
	\hline
\end{tabular}
}
\end{document}
\fi

\item 
\label{chapters/11/10/2/14}
\iffalse
\documentclass[journal,10pt,twocolumn]{article}
\usepackage{graphicx, float}
\usepackage[margin=0.5in]{geometry}
\usepackage{amsmath, bm}
\usepackage{array}
\usepackage{booktabs}
\usepackage{xfrac}

\providecommand{\norm}[1]{\left\lVert#1\right\rVert}
\let\vec\mathbf
\newcommand{\myvec}[1]{\ensuremath{\begin{pmatrix}#1\end{pmatrix}}}
\newcommand{\mydet}[1]{\ensuremath{\begin{vmatrix}#1\end{vmatrix}}}

\title{\textbf{Line Assignment}}
\author{Harsha sai sampath kumar}
\date{September 2022}

\begin{document}

\maketitle
\paragraph{\textit{\large Problem Statement} -
\fi
Find the equation of the line through the point (0,2) making an angle \begin{align}2\pi/3\end{align} with the positive X-axis. Also find the equation of the line parallel to it and crossing the Y-axis at a distance of 2 units below the origin
	\begin{figure}[!ht]
		\centering
 \includegraphics[width=\columnwidth]{chapters/11/10/2/14/figs/fig.pdf}
		\caption{}
		\label{fig:11/10/2/14}
  	\end{figure}
	\\
	\solution
\iffalse
	}

\section*{\large Solution}

\begin{figure}[H]
\centering
\includegraphics[width=1\columnwidth]{fig.pdf}
\caption{}
\end{figure}


\section{construction}

\begin{tabular}{|c|c|}
	\hline
	\textbf{Point}&\textbf{Value}\\
	\hline
	A&\myvec{0\\2}\\
	\hline
	$\theta$&2$\pi$/3\\
	\hline
	D&\myvec{0\\-2}\\
	\hline
	
	
\end{tabular}


\section*{Assumptions}
To find the line equation  through the point (0,2)
\vspace*{3mm}
\fi
From the  given information, the direction vector is
\begin{align}
	\vec{m}=\myvec{1\\-\sqrt{3}}
\end{align}
  
Thus, 
the normal vector is
\begin{align}
	\vec{n}=\myvec{\sqrt{3}\\1}
\end{align}

\iffalse
\begin{align}
	\vec{m}=\myvec{1\\-\sqrt{3}}
	\vec{n}=\myvec{\sqrt{3}\\1}
\end{align}


\begin{align}
	\vec{n^T}=(\sqrt{3}&1)	
\end{align}






Where line equation  is given by:
\begin{align}
\vec{n^T(x-p)}=0
\label{pf2-eq-1}
\end{align}

By substituting the values in the above equation:
\fi
and the 
equation of the line is 
\begin{align}
	\myvec{\sqrt{3} &1}\myvec{\vec{x}-\myvec{0\\2}}&=0
	\\
	\implies 
	\myvec{\sqrt{3}&1}
	\vec{x}&=2
\end{align}
The equation of the parallel crossing the Y-axis at a distance of 2 units below the origin is given by 
\begin{align}
	\myvec{\sqrt{3} &1}\myvec{\vec{x}-\myvec{0\\-2}}&=0
	\\
	\implies 
	\myvec{\sqrt{3}&1}
\vec{x}=-2
\end{align}

\item 
\label{chapters/11/10/2/15}
\iffalse
\documentclass[journal,10pt,twocolumn]{article}
\usepackage{graphicx}
\usepackage[margin=0.5in]{geometry}
\usepackage[cmex10]{amsmath}
\usepackage{array}
\usepackage{booktabs}
\usepackage{listings}
\title{\textbf{Line Assignment}}
\author{Bhavani Kanike}
\date{October 2022}

\providecommand{\norm}[1]{\left\lVert#1\right\rVert}
\providecommand{\abs}[1]{\left\vert#1\right\vert}
\let\vec\mathbf
\newcommand{\myvec}[1]{\ensuremath{\begin{pmatrix}#1\end{pmatrix}}}
\newcommand{\mydet}[1]{\ensuremath{\begin{vmatrix}#1\end{vmatrix}}}
\providecommand{\brak}[1]{\ensuremath{\left(#1\right)}}

\begin{document}

\maketitle
\paragraph{\textit{Problem Statement} 
\fi
ABCD is a quadrilateral in which $\vec{P}, \vec{Q}, \vec{R}$ and $\vec{S}$ are mid-points of the sides AB, BC, CD and DA (see Fig \ref{fig:9/8/2/1}). AC is a diagonal. 
		
Show that 
\begin{enumerate}
	\item $SR \parallel AC$ and $SR =\frac{1}{2} AC$
\item $PQ = SR$
\item $PQRS$ is a parallelogram.
\end{enumerate}
 	\begin{figure}
		\centering
 \includegraphics[width=\columnwidth]{chapters/9/8/2/1/figs/line1.pdf}
		\caption{}
		\label{fig:9/8/2/1}
  	\end{figure}
	\solution 
	Using 
	  \eqref{eq:section_formula},
	\begin{align}
		\label{eq:9/8/2/1}
		\begin{split}
		\vec{P} &= \frac{\vec{A}+\vec{B}}{2}\\
 \vec{Q} &= \frac{\vec{C}+\vec{B}}{2}\\
 \vec{R} &= \frac{\vec{C}+\vec{D}}{2}\\
 \vec{S} &= \frac{\vec{D}+\vec{A}}{2}
		\end{split}
	\end{align}
\begin{enumerate}
	\item
	Consequently, 
	\begin{align}
\vec{R}
		-\vec{S} &= \frac{\vec{C}-\vec{A}}{2}
		\\
		\implies SR &\parallel AC
	\end{align}
	Also, 
	\begin{align}
		\norm{\vec{R}
		-\vec{S}} &= \frac{\norm{\vec{C}-\vec{A}}}{2}
		\\
		\implies SR &= \frac{1}{2}AC
	\end{align}
\item 	From 
		\eqref{eq:9/8/2/1},
	\begin{align}
\vec{R}
		-\vec{S} = \vec{Q}-\vec{P}
	\end{align}
	which means that $PQRS$ is a parallelogram and $PQ = SR$.
\end{enumerate}
%
\iffalse
\begin{figure}[h]
\centering
\includegraphics[width=1\columnwidth]
\caption{Figure}
\label{fig:triangle}
\end{figure}

\section*{Solution}

$\boldsymbol Given :$  ABCD is a Quadrilateral P,Q,R and S are the midpoints of line AB,BC,CD,DA.We can obtain the points P,Q,R and S from A,B,C and D and are given by\\\\
\boldmath
\unboldmath
(3) To prove that PQRS is a parallelogram we need to prove  PQ // SR
To prove SR $\parallel$ PQ\\
Direction vector of line SR  $\boldsymbol {(R-S) =  \frac{(C-A)}{2}}$\\\\
Direction vector of line PQ  $\boldsymbol {(Q-P)= \frac{(C-A)}{2}}$\\\\
\begin{equation}
	\boldsymbol {(R-S) = (Q-P) = \frac{(C-A)}{2}}\\
\end{equation}
Since the direction vectors of line SR and PQ are in same direction\\\\
$SR \parallel PQ$\\
Therefore,
$\boldsymbol{ PQRS }$ is a parallelogram\\\\

	
(1)  Directional vector of line SR  = $\boldsymbol {(R-S)}$ = $\frac{\boldsymbol{(C-A)}}{2} $\\
Directional vector of line AC  = $\boldsymbol {(C-A)}$\\

It is observed that the constant k is $\frac{1}{2}$

Therefore
\begin{equation}
	SR \parallel AC
\end{equation} 

and from equation 1 
\begin{equation}
	\boldsymbol {SR = \frac{1}{2}AC}    
\end{equation}\\


(2)   To prove PQ = SR\\ 
		From euqation 1\\\\
\begin{equation}
		\boldsymbol{ (Q-P) = (R-S) = \frac{(C-A)}{2}}
\end{equation}
	 



\section{Execution}
The below python code realizes the construction:
\begin{lstlisting}
https://github.com/bhavani360/FWC_assignments
\end{lstlisting}
	
\section*{Construction}
The dimensions of the Quadrilateral ABCD are taken as below\\
{
\setlength\extrarowheight{2pt}
\centering
	\begin{tabular}{|c|c|}
	\hline
	\textbf{symbol}&\textbf{value}\\
	\hline
	r&8\\
	\hline
	$\theta$&pi/2.5\\
	\hline
	d&7\\
	\hline
	A&(0,0)\\
	\hline
	B&(d,0)\\
	\hline
	D&(rcos$\theta$,rsin$\theta$)\\
	\hline
	C&(D/1.5)+B\\
	\hline
\end{tabular}
}
\end{document}
\fi

\item 
$P(a,b)$ is the mid-point of the line segment between axes. Show that the equation of the line is $\frac{x}{a}+\frac{y}{b}=2$
\label{chapters/11/10/2/18}
\\
\solution
\iffalse

\documentclass[12pt]{article}
\usepackage{graphicx}
\usepackage{amsmath}
\usepackage{mathtools}
\usepackage{gensymb}
\usepackage{amssymb}

\newcommand{\mydet}[1]{\ensuremath{\begin{vmatrix}#1\end{vmatrix}}}
\providecommand{\brak}[1]{\ensuremath{\left(#1\right)}}
\providecommand{\norm}[1]{\left\lVert#1\right\rVert}
\newcommand{\solution}{\noindent \textbf{Solution: }}
\newcommand{\myvec}[1]{\ensuremath{\begin{pmatrix}#1\end{pmatrix}}}
\let\vec\mathbf

\begin{document}
\begin{center}
\textbf\large{CLASS 11 CHAPTER-11 \\ LINES}

\end{center}
\section*{Excercise 10.2}


\solution
\fi
Let
\begin{align}
	\vec{A}&=x\vec{e_{1}},
	\vec{B}&=y\vec{e_{2}},
	\vec{P}&=\myvec{a\\b}
\end{align}
where
\begin{align}
	\vec{e_{1}}=\myvec{1\\0} \text{ and } \vec{e_{2}}=\myvec{0\\1}
\end{align}
as shown in Fig. \ref{fig:11/10/2/18Fig1}
\begin{figure}[!h]
	\begin{center} 
	    \includegraphics[width=\columnwidth]{chapters/11/10/2/18/figs/line1}
	\end{center}
\caption{}
\label{fig:11/10/2/18Fig1}
\end{figure}
Given that
\begin{align}
	\vec{P}&=\frac{\vec{A}+\vec{B}}{2}=\frac{x\vec{e_{1}}+y\vec{e_{2}}}{2}\\
\implies 	2\vec{P}&=x\vec{e_{1}}+y\vec{e_{2}}\\
	\vec{e_{1}}^{\top}\brak{2\vec{P}}&=\vec{e_{1}^{\top}}\brak{x\vec{e_{1}}+y\vec{e_{2}}}
=x\\				 
	\text{ and }\vec{e_{2}}^{\top}\brak{2\vec{P}}&=\vec{e_{2}^{\top}}\brak{x\vec{e_{1}}+y\vec{e_{2}}}
=y				 
\end{align}
Thus,
\begin{align}
	x&=2\vec{e_{1}}^{\top}\vec{P}=2a\\
	y&=2\vec{e_{2}}^{\top}\vec{P}=2b
\end{align}
yielding
\begin{align}
	\vec{A} &= 2a\vec{e_{1}}
	\vec{B} &= 2b\vec{e_{1}}
\end{align}
Thus, the direction vector of the line is 
\begin{align}
	\vec{m} &= \vec{A}-\vec{B}\\
	&=\myvec{a \\ -b}
\end{align}
and the normal vector is
\begin{align}
	\vec{n} = \myvec{b \\ a}
\end{align}
The equation of line passing through $\vec{P}$ is then obtained as
\begin{align}
	\vec{n}^{\top} \brak{\vec{x}-\vec{P}} &= 0\\
	\myvec{b & a}\vec{x} &= 2ab.
\end{align}



\item Point $\vec{R}\brak{h, k}$ divides a line segment between the axes in the ratio 1: 2. Find the equation of the line.
\label{chapters/11/10/2/19}
\iffalse
\documentclass[journal,10pt,twocolumn]{article}
\usepackage{graphicx}
\usepackage[margin=0.5in]{geometry}
\usepackage[cmex10]{amsmath}
\usepackage{array}
\usepackage{booktabs}
\usepackage{listings}
\title{\textbf{Line Assignment}}
\author{Bhavani Kanike}
\date{October 2022}

\providecommand{\norm}[1]{\left\lVert#1\right\rVert}
\providecommand{\abs}[1]{\left\vert#1\right\vert}
\let\vec\mathbf
\newcommand{\myvec}[1]{\ensuremath{\begin{pmatrix}#1\end{pmatrix}}}
\newcommand{\mydet}[1]{\ensuremath{\begin{vmatrix}#1\end{vmatrix}}}
\providecommand{\brak}[1]{\ensuremath{\left(#1\right)}}

\begin{document}

\maketitle
\paragraph{\textit{Problem Statement} 
\fi
ABCD is a quadrilateral in which $\vec{P}, \vec{Q}, \vec{R}$ and $\vec{S}$ are mid-points of the sides AB, BC, CD and DA (see Fig \ref{fig:9/8/2/1}). AC is a diagonal. 
		
Show that 
\begin{enumerate}
	\item $SR \parallel AC$ and $SR =\frac{1}{2} AC$
\item $PQ = SR$
\item $PQRS$ is a parallelogram.
\end{enumerate}
 	\begin{figure}
		\centering
 \includegraphics[width=\columnwidth]{chapters/9/8/2/1/figs/line1.pdf}
		\caption{}
		\label{fig:9/8/2/1}
  	\end{figure}
	\solution 
	Using 
	  \eqref{eq:section_formula},
	\begin{align}
		\label{eq:9/8/2/1}
		\begin{split}
		\vec{P} &= \frac{\vec{A}+\vec{B}}{2}\\
 \vec{Q} &= \frac{\vec{C}+\vec{B}}{2}\\
 \vec{R} &= \frac{\vec{C}+\vec{D}}{2}\\
 \vec{S} &= \frac{\vec{D}+\vec{A}}{2}
		\end{split}
	\end{align}
\begin{enumerate}
	\item
	Consequently, 
	\begin{align}
\vec{R}
		-\vec{S} &= \frac{\vec{C}-\vec{A}}{2}
		\\
		\implies SR &\parallel AC
	\end{align}
	Also, 
	\begin{align}
		\norm{\vec{R}
		-\vec{S}} &= \frac{\norm{\vec{C}-\vec{A}}}{2}
		\\
		\implies SR &= \frac{1}{2}AC
	\end{align}
\item 	From 
		\eqref{eq:9/8/2/1},
	\begin{align}
\vec{R}
		-\vec{S} = \vec{Q}-\vec{P}
	\end{align}
	which means that $PQRS$ is a parallelogram and $PQ = SR$.
\end{enumerate}
%
\iffalse
\begin{figure}[h]
\centering
\includegraphics[width=1\columnwidth]
\caption{Figure}
\label{fig:triangle}
\end{figure}

\section*{Solution}

$\boldsymbol Given :$  ABCD is a Quadrilateral P,Q,R and S are the midpoints of line AB,BC,CD,DA.We can obtain the points P,Q,R and S from A,B,C and D and are given by\\\\
\boldmath
\unboldmath
(3) To prove that PQRS is a parallelogram we need to prove  PQ // SR
To prove SR $\parallel$ PQ\\
Direction vector of line SR  $\boldsymbol {(R-S) =  \frac{(C-A)}{2}}$\\\\
Direction vector of line PQ  $\boldsymbol {(Q-P)= \frac{(C-A)}{2}}$\\\\
\begin{equation}
	\boldsymbol {(R-S) = (Q-P) = \frac{(C-A)}{2}}\\
\end{equation}
Since the direction vectors of line SR and PQ are in same direction\\\\
$SR \parallel PQ$\\
Therefore,
$\boldsymbol{ PQRS }$ is a parallelogram\\\\

	
(1)  Directional vector of line SR  = $\boldsymbol {(R-S)}$ = $\frac{\boldsymbol{(C-A)}}{2} $\\
Directional vector of line AC  = $\boldsymbol {(C-A)}$\\

It is observed that the constant k is $\frac{1}{2}$

Therefore
\begin{equation}
	SR \parallel AC
\end{equation} 

and from equation 1 
\begin{equation}
	\boldsymbol {SR = \frac{1}{2}AC}    
\end{equation}\\


(2)   To prove PQ = SR\\ 
		From euqation 1\\\\
\begin{equation}
		\boldsymbol{ (Q-P) = (R-S) = \frac{(C-A)}{2}}
\end{equation}
	 



\section{Execution}
The below python code realizes the construction:
\begin{lstlisting}
https://github.com/bhavani360/FWC_assignments
\end{lstlisting}
	
\section*{Construction}
The dimensions of the Quadrilateral ABCD are taken as below\\
{
\setlength\extrarowheight{2pt}
\centering
	\begin{tabular}{|c|c|}
	\hline
	\textbf{symbol}&\textbf{value}\\
	\hline
	r&8\\
	\hline
	$\theta$&pi/2.5\\
	\hline
	d&7\\
	\hline
	A&(0,0)\\
	\hline
	B&(d,0)\\
	\hline
	D&(rcos$\theta$,rsin$\theta$)\\
	\hline
	C&(D/1.5)+B\\
	\hline
\end{tabular}
}
\end{document}
\fi

\item 
\label{chapters/11/10/2/20}
\iffalse
\documentclass[journal,10pt,twocolumn]{article}
\usepackage{graphicx}
\usepackage[margin=0.5in]{geometry}
\usepackage[cmex10]{amsmath}
\usepackage{array}
\usepackage{booktabs}
\usepackage{listings}
\title{\textbf{Line Assignment}}
\author{Bhavani Kanike}
\date{October 2022}

\providecommand{\norm}[1]{\left\lVert#1\right\rVert}
\providecommand{\abs}[1]{\left\vert#1\right\vert}
\let\vec\mathbf
\newcommand{\myvec}[1]{\ensuremath{\begin{pmatrix}#1\end{pmatrix}}}
\newcommand{\mydet}[1]{\ensuremath{\begin{vmatrix}#1\end{vmatrix}}}
\providecommand{\brak}[1]{\ensuremath{\left(#1\right)}}

\begin{document}

\maketitle
\paragraph{\textit{Problem Statement} 
\fi
ABCD is a quadrilateral in which $\vec{P}, \vec{Q}, \vec{R}$ and $\vec{S}$ are mid-points of the sides AB, BC, CD and DA (see Fig \ref{fig:9/8/2/1}). AC is a diagonal. 
		
Show that 
\begin{enumerate}
	\item $SR \parallel AC$ and $SR =\frac{1}{2} AC$
\item $PQ = SR$
\item $PQRS$ is a parallelogram.
\end{enumerate}
 	\begin{figure}
		\centering
 \includegraphics[width=\columnwidth]{chapters/9/8/2/1/figs/line1.pdf}
		\caption{}
		\label{fig:9/8/2/1}
  	\end{figure}
	\solution 
	Using 
	  \eqref{eq:section_formula},
	\begin{align}
		\label{eq:9/8/2/1}
		\begin{split}
		\vec{P} &= \frac{\vec{A}+\vec{B}}{2}\\
 \vec{Q} &= \frac{\vec{C}+\vec{B}}{2}\\
 \vec{R} &= \frac{\vec{C}+\vec{D}}{2}\\
 \vec{S} &= \frac{\vec{D}+\vec{A}}{2}
		\end{split}
	\end{align}
\begin{enumerate}
	\item
	Consequently, 
	\begin{align}
\vec{R}
		-\vec{S} &= \frac{\vec{C}-\vec{A}}{2}
		\\
		\implies SR &\parallel AC
	\end{align}
	Also, 
	\begin{align}
		\norm{\vec{R}
		-\vec{S}} &= \frac{\norm{\vec{C}-\vec{A}}}{2}
		\\
		\implies SR &= \frac{1}{2}AC
	\end{align}
\item 	From 
		\eqref{eq:9/8/2/1},
	\begin{align}
\vec{R}
		-\vec{S} = \vec{Q}-\vec{P}
	\end{align}
	which means that $PQRS$ is a parallelogram and $PQ = SR$.
\end{enumerate}
%
\iffalse
\begin{figure}[h]
\centering
\includegraphics[width=1\columnwidth]
\caption{Figure}
\label{fig:triangle}
\end{figure}

\section*{Solution}

$\boldsymbol Given :$  ABCD is a Quadrilateral P,Q,R and S are the midpoints of line AB,BC,CD,DA.We can obtain the points P,Q,R and S from A,B,C and D and are given by\\\\
\boldmath
\unboldmath
(3) To prove that PQRS is a parallelogram we need to prove  PQ // SR
To prove SR $\parallel$ PQ\\
Direction vector of line SR  $\boldsymbol {(R-S) =  \frac{(C-A)}{2}}$\\\\
Direction vector of line PQ  $\boldsymbol {(Q-P)= \frac{(C-A)}{2}}$\\\\
\begin{equation}
	\boldsymbol {(R-S) = (Q-P) = \frac{(C-A)}{2}}\\
\end{equation}
Since the direction vectors of line SR and PQ are in same direction\\\\
$SR \parallel PQ$\\
Therefore,
$\boldsymbol{ PQRS }$ is a parallelogram\\\\

	
(1)  Directional vector of line SR  = $\boldsymbol {(R-S)}$ = $\frac{\boldsymbol{(C-A)}}{2} $\\
Directional vector of line AC  = $\boldsymbol {(C-A)}$\\

It is observed that the constant k is $\frac{1}{2}$

Therefore
\begin{equation}
	SR \parallel AC
\end{equation} 

and from equation 1 
\begin{equation}
	\boldsymbol {SR = \frac{1}{2}AC}    
\end{equation}\\


(2)   To prove PQ = SR\\ 
		From euqation 1\\\\
\begin{equation}
		\boldsymbol{ (Q-P) = (R-S) = \frac{(C-A)}{2}}
\end{equation}
	 



\section{Execution}
The below python code realizes the construction:
\begin{lstlisting}
https://github.com/bhavani360/FWC_assignments
\end{lstlisting}
	
\section*{Construction}
The dimensions of the Quadrilateral ABCD are taken as below\\
{
\setlength\extrarowheight{2pt}
\centering
	\begin{tabular}{|c|c|}
	\hline
	\textbf{symbol}&\textbf{value}\\
	\hline
	r&8\\
	\hline
	$\theta$&pi/2.5\\
	\hline
	d&7\\
	\hline
	A&(0,0)\\
	\hline
	B&(d,0)\\
	\hline
	D&(rcos$\theta$,rsin$\theta$)\\
	\hline
	C&(D/1.5)+B\\
	\hline
\end{tabular}
}
\end{document}
\fi

\item Find the equation of the line  parallel to the line 3x-4y+2=0 and passing through the point (-2,3).
\label{chapters/11/10/3/7}
\iffalse
\documentclass[journal,10pt,twocolumn]{article}
\usepackage{graphicx}
\usepackage[margin=0.5in]{geometry}
\usepackage[cmex10]{amsmath}
\usepackage{array}
\usepackage{booktabs}
\usepackage{listings}
\title{\textbf{Line Assignment}}
\author{Bhavani Kanike}
\date{October 2022}

\providecommand{\norm}[1]{\left\lVert#1\right\rVert}
\providecommand{\abs}[1]{\left\vert#1\right\vert}
\let\vec\mathbf
\newcommand{\myvec}[1]{\ensuremath{\begin{pmatrix}#1\end{pmatrix}}}
\newcommand{\mydet}[1]{\ensuremath{\begin{vmatrix}#1\end{vmatrix}}}
\providecommand{\brak}[1]{\ensuremath{\left(#1\right)}}

\begin{document}

\maketitle
\paragraph{\textit{Problem Statement} 
\fi
ABCD is a quadrilateral in which $\vec{P}, \vec{Q}, \vec{R}$ and $\vec{S}$ are mid-points of the sides AB, BC, CD and DA (see Fig \ref{fig:9/8/2/1}). AC is a diagonal. 
		
Show that 
\begin{enumerate}
	\item $SR \parallel AC$ and $SR =\frac{1}{2} AC$
\item $PQ = SR$
\item $PQRS$ is a parallelogram.
\end{enumerate}
 	\begin{figure}
		\centering
 \includegraphics[width=\columnwidth]{chapters/9/8/2/1/figs/line1.pdf}
		\caption{}
		\label{fig:9/8/2/1}
  	\end{figure}
	\solution 
	Using 
	  \eqref{eq:section_formula},
	\begin{align}
		\label{eq:9/8/2/1}
		\begin{split}
		\vec{P} &= \frac{\vec{A}+\vec{B}}{2}\\
 \vec{Q} &= \frac{\vec{C}+\vec{B}}{2}\\
 \vec{R} &= \frac{\vec{C}+\vec{D}}{2}\\
 \vec{S} &= \frac{\vec{D}+\vec{A}}{2}
		\end{split}
	\end{align}
\begin{enumerate}
	\item
	Consequently, 
	\begin{align}
\vec{R}
		-\vec{S} &= \frac{\vec{C}-\vec{A}}{2}
		\\
		\implies SR &\parallel AC
	\end{align}
	Also, 
	\begin{align}
		\norm{\vec{R}
		-\vec{S}} &= \frac{\norm{\vec{C}-\vec{A}}}{2}
		\\
		\implies SR &= \frac{1}{2}AC
	\end{align}
\item 	From 
		\eqref{eq:9/8/2/1},
	\begin{align}
\vec{R}
		-\vec{S} = \vec{Q}-\vec{P}
	\end{align}
	which means that $PQRS$ is a parallelogram and $PQ = SR$.
\end{enumerate}
%
\iffalse
\begin{figure}[h]
\centering
\includegraphics[width=1\columnwidth]
\caption{Figure}
\label{fig:triangle}
\end{figure}

\section*{Solution}

$\boldsymbol Given :$  ABCD is a Quadrilateral P,Q,R and S are the midpoints of line AB,BC,CD,DA.We can obtain the points P,Q,R and S from A,B,C and D and are given by\\\\
\boldmath
\unboldmath
(3) To prove that PQRS is a parallelogram we need to prove  PQ // SR
To prove SR $\parallel$ PQ\\
Direction vector of line SR  $\boldsymbol {(R-S) =  \frac{(C-A)}{2}}$\\\\
Direction vector of line PQ  $\boldsymbol {(Q-P)= \frac{(C-A)}{2}}$\\\\
\begin{equation}
	\boldsymbol {(R-S) = (Q-P) = \frac{(C-A)}{2}}\\
\end{equation}
Since the direction vectors of line SR and PQ are in same direction\\\\
$SR \parallel PQ$\\
Therefore,
$\boldsymbol{ PQRS }$ is a parallelogram\\\\

	
(1)  Directional vector of line SR  = $\boldsymbol {(R-S)}$ = $\frac{\boldsymbol{(C-A)}}{2} $\\
Directional vector of line AC  = $\boldsymbol {(C-A)}$\\

It is observed that the constant k is $\frac{1}{2}$

Therefore
\begin{equation}
	SR \parallel AC
\end{equation} 

and from equation 1 
\begin{equation}
	\boldsymbol {SR = \frac{1}{2}AC}    
\end{equation}\\


(2)   To prove PQ = SR\\ 
		From euqation 1\\\\
\begin{equation}
		\boldsymbol{ (Q-P) = (R-S) = \frac{(C-A)}{2}}
\end{equation}
	 



\section{Execution}
The below python code realizes the construction:
\begin{lstlisting}
https://github.com/bhavani360/FWC_assignments
\end{lstlisting}
	
\section*{Construction}
The dimensions of the Quadrilateral ABCD are taken as below\\
{
\setlength\extrarowheight{2pt}
\centering
	\begin{tabular}{|c|c|}
	\hline
	\textbf{symbol}&\textbf{value}\\
	\hline
	r&8\\
	\hline
	$\theta$&pi/2.5\\
	\hline
	d&7\\
	\hline
	A&(0,0)\\
	\hline
	B&(d,0)\\
	\hline
	D&(rcos$\theta$,rsin$\theta$)\\
	\hline
	C&(D/1.5)+B\\
	\hline
\end{tabular}
}
\end{document}
\fi

\item Find the equation of line perpendicular to the line $x-7y+5=0$ and having $x$ intercept $3$\\
\label{chapters/11/10/3/8}
\solution
\iffalse
\documentclass[journal,10pt,twocolumn]{article}
\usepackage{graphicx}
\usepackage[margin=0.5in]{geometry}
\usepackage[cmex10]{amsmath}
\usepackage{array}
\usepackage{booktabs}
\usepackage{listings}
\title{\textbf{Line Assignment}}
\author{Bhavani Kanike}
\date{October 2022}

\providecommand{\norm}[1]{\left\lVert#1\right\rVert}
\providecommand{\abs}[1]{\left\vert#1\right\vert}
\let\vec\mathbf
\newcommand{\myvec}[1]{\ensuremath{\begin{pmatrix}#1\end{pmatrix}}}
\newcommand{\mydet}[1]{\ensuremath{\begin{vmatrix}#1\end{vmatrix}}}
\providecommand{\brak}[1]{\ensuremath{\left(#1\right)}}

\begin{document}

\maketitle
\paragraph{\textit{Problem Statement} 
\fi
ABCD is a quadrilateral in which $\vec{P}, \vec{Q}, \vec{R}$ and $\vec{S}$ are mid-points of the sides AB, BC, CD and DA (see Fig \ref{fig:9/8/2/1}). AC is a diagonal. 
		
Show that 
\begin{enumerate}
	\item $SR \parallel AC$ and $SR =\frac{1}{2} AC$
\item $PQ = SR$
\item $PQRS$ is a parallelogram.
\end{enumerate}
 	\begin{figure}
		\centering
 \includegraphics[width=\columnwidth]{chapters/9/8/2/1/figs/line1.pdf}
		\caption{}
		\label{fig:9/8/2/1}
  	\end{figure}
	\solution 
	Using 
	  \eqref{eq:section_formula},
	\begin{align}
		\label{eq:9/8/2/1}
		\begin{split}
		\vec{P} &= \frac{\vec{A}+\vec{B}}{2}\\
 \vec{Q} &= \frac{\vec{C}+\vec{B}}{2}\\
 \vec{R} &= \frac{\vec{C}+\vec{D}}{2}\\
 \vec{S} &= \frac{\vec{D}+\vec{A}}{2}
		\end{split}
	\end{align}
\begin{enumerate}
	\item
	Consequently, 
	\begin{align}
\vec{R}
		-\vec{S} &= \frac{\vec{C}-\vec{A}}{2}
		\\
		\implies SR &\parallel AC
	\end{align}
	Also, 
	\begin{align}
		\norm{\vec{R}
		-\vec{S}} &= \frac{\norm{\vec{C}-\vec{A}}}{2}
		\\
		\implies SR &= \frac{1}{2}AC
	\end{align}
\item 	From 
		\eqref{eq:9/8/2/1},
	\begin{align}
\vec{R}
		-\vec{S} = \vec{Q}-\vec{P}
	\end{align}
	which means that $PQRS$ is a parallelogram and $PQ = SR$.
\end{enumerate}
%
\iffalse
\begin{figure}[h]
\centering
\includegraphics[width=1\columnwidth]
\caption{Figure}
\label{fig:triangle}
\end{figure}

\section*{Solution}

$\boldsymbol Given :$  ABCD is a Quadrilateral P,Q,R and S are the midpoints of line AB,BC,CD,DA.We can obtain the points P,Q,R and S from A,B,C and D and are given by\\\\
\boldmath
\unboldmath
(3) To prove that PQRS is a parallelogram we need to prove  PQ // SR
To prove SR $\parallel$ PQ\\
Direction vector of line SR  $\boldsymbol {(R-S) =  \frac{(C-A)}{2}}$\\\\
Direction vector of line PQ  $\boldsymbol {(Q-P)= \frac{(C-A)}{2}}$\\\\
\begin{equation}
	\boldsymbol {(R-S) = (Q-P) = \frac{(C-A)}{2}}\\
\end{equation}
Since the direction vectors of line SR and PQ are in same direction\\\\
$SR \parallel PQ$\\
Therefore,
$\boldsymbol{ PQRS }$ is a parallelogram\\\\

	
(1)  Directional vector of line SR  = $\boldsymbol {(R-S)}$ = $\frac{\boldsymbol{(C-A)}}{2} $\\
Directional vector of line AC  = $\boldsymbol {(C-A)}$\\

It is observed that the constant k is $\frac{1}{2}$

Therefore
\begin{equation}
	SR \parallel AC
\end{equation} 

and from equation 1 
\begin{equation}
	\boldsymbol {SR = \frac{1}{2}AC}    
\end{equation}\\


(2)   To prove PQ = SR\\ 
		From euqation 1\\\\
\begin{equation}
		\boldsymbol{ (Q-P) = (R-S) = \frac{(C-A)}{2}}
\end{equation}
	 



\section{Execution}
The below python code realizes the construction:
\begin{lstlisting}
https://github.com/bhavani360/FWC_assignments
\end{lstlisting}
	
\section*{Construction}
The dimensions of the Quadrilateral ABCD are taken as below\\
{
\setlength\extrarowheight{2pt}
\centering
	\begin{tabular}{|c|c|}
	\hline
	\textbf{symbol}&\textbf{value}\\
	\hline
	r&8\\
	\hline
	$\theta$&pi/2.5\\
	\hline
	d&7\\
	\hline
	A&(0,0)\\
	\hline
	B&(d,0)\\
	\hline
	D&(rcos$\theta$,rsin$\theta$)\\
	\hline
	C&(D/1.5)+B\\
	\hline
\end{tabular}
}
\end{document}
\fi

\item Prove that the line through the point$(x_1,y_1)$ and parallel to the line A$x$+B$y$+C=0 is A$(x-x_1)$+B$(y-y_1)$=0.
\label{chapters/11/10/3/11}
\\
\solution
\iffalse
\documentclass[10pt]{article}
\usepackage{graphicx}
\usepackage[none]{hyphenat}
\usepackage{graphicx}
\usepackage{listings}
\usepackage[english]{babel}
\usepackage{siunitx}
\usepackage{graphicx}
\usepackage{caption} 
\usepackage{booktabs}
\usepackage{array}
\usepackage{amssymb} % for \because
\usepackage{amsmath}   % for having text in math mode
\usepackage{extarrows} % for Row operations arrows
\usepackage{listings}
\usepackage[utf8]{inputenc}
\lstset{
  frame=single,
  breaklines=true
}
\usepackage{hyperref}
  
%Following 2 lines were added to remove the blank page at the beginning
\usepackage{atbegshi}% http://ctan.org/pkg/atbegshi
\AtBeginDocument{\AtBeginShipoutNext{\AtBeginShipoutDiscard}}


%New macro definitions
\newcommand{\mydet}[1]{\ensuremath{\begin{vmatrix}#1\end{vmatrix}}}
\providecommand{\brak}[1]{\ensuremath{\left(#1\right)}}
\newcommand{\solution}{\noindent \textbf{Solution: }}
\newcommand{\myvec}[1]{\ensuremath{\begin{pmatrix}#1\end{pmatrix}}}
\providecommand{\norm}[1]{\left\lVert#1\right\rVert}
\providecommand{\abs}[1]{\left\vert#1\right\vert}
\let\vec\mathbf{}
\begin{document}

\begin{center}
\title{\textbf{STRAIGHT LINES}}
\date{\vspace{-5ex}} %Not to print date automatically
\maketitle
\end{center}

\section{11$^{th}$ Maths - Chapter 10}
This is Problem 11 from Exercise-10.3
\begin{enumerate}
\item Prove that the line through the point$(x_1,y_1)$ and parallel to the line A$x$+B$y$+C=0 is A$(x-x_1)$+B$(y-y_1)$=0.

\solution
Given 
\fi
The given line parameters are
\begin{align}
\vec{n}=\myvec{A\\B}, \, c = C
\end{align}
Let 
\begin{align}
\vec{P}=\myvec{x_1\\y_1}\\
\end{align}
Then the equation of the desired line is
\begin{align}
	\vec{n}^\top\brak{\vec{x}-\vec{P}}&=0\\
	\implies A(x-x_1)+B(y-y_1)&=0
\end{align}

	\item Find the equation of the line passing through the point $\brak{1,2,-4}$ and perpendicular to the two lines
\begin{align}
	\frac{x-8}{3}=\frac{y+19}{-16}=\frac{z-10}{7} \text{ and }\\ \frac{x-15}{3}=\frac{y-29}{8}=\frac{z-5}{-5} 
\end{align}
    \solution
		\iffalse
\documentclass[journal,12pt,twocolumn]{IEEEtran}
%
\usepackage{setspace}
\usepackage{gensymb}
%\doublespacing
\singlespacing

%\usepackage{graphicx}
%\usepackage{amssymb}
%\usepackage{relsize}
\usepackage[cmex10]{amsmath}
%\usepackage{amsthm}
%\interdisplaylinepenalty=2500
%\savesymbol{iint}
%\usepackage{txfonts}
%\restoresymbol{TXF}{iint}
%\usepackage{wasysym}
\usepackage{amsthm}
%\usepackage{iithtlc}
\usepackage{mathrsfs}
\usepackage{txfonts}
\usepackage{stfloats}
\usepackage{bm}
\usepackage{cite}
\usepackage{cases}
\usepackage{subfig}
%\usepackage{xtab}
\usepackage{longtable}
\usepackage{multirow}
%\usepackage{algorithm}
%\usepackage{algpseudocode}
\usepackage{enumitem}
\usepackage{mathtools}
\usepackage{steinmetz}
\usepackage{tikz}
\usepackage{circuitikz}
\usepackage{verbatim}
\usepackage{tfrupee}
\usepackage[breaklinks=true]{hyperref}
%\usepackage{stmaryrd}
\usepackage{tkz-euclide} % loads  TikZ and tkz-base
%\usetkzobj{all}
\usetikzlibrary{calc,math}
\usepackage{listings}
    \usepackage{color}                                            %%
    \usepackage{array}                                            %%
    \usepackage{longtable}                                        %%
    \usepackage{calc}                                             %%
    \usepackage{multirow}                                         %%
    \usepackage{hhline}                                           %%
    \usepackage{ifthen}                                           %%
  %optionally (for landscape tables embedded in another document): %%
    \usepackage{lscape}     
\usepackage{multicol}
\usepackage{chngcntr}
%\usepackage{enumerate}

%\usepackage{wasysym}
%\newcounter{MYtempeqncnt}
\DeclareMathOperator*{\Res}{Res}
%\renewcommand{\baselinestretch}{2}
\renewcommand\thesection{\arabic{section}}
\renewcommand\thesubsection{\thesection.\arabic{subsection}}
\renewcommand\thesubsubsection{\thesubsection.\arabic{subsubsection}}

\renewcommand\thesectiondis{\arabic{section}}
\renewcommand\thesubsectiondis{\thesectiondis.\arabic{subsection}}
\renewcommand\thesubsubsectiondis{\thesubsectiondis.\arabic{subsubsection}}

% correct bad hyphenation here
\hyphenation{op-tical net-works semi-conduc-tor}
\def\inputGnumericTable{}                                 %%

\lstset{
%language=C,
frame=single, 
breaklines=true,
columns=fullflexible
}
%\lstset{
%language=tex,
%frame=single, 
%breaklines=true
%}


\begin{document}
%


\newtheorem{theorem}{Theorem}[section]
\newtheorem{problem}{Problem}
\newtheorem{proposition}{Proposition}[section]
\newtheorem{lemma}{Lemma}[section]
\newtheorem{corollary}[theorem]{Corollary}
\newtheorem{example}{Example}[section]
\newtheorem{definition}[problem]{Definition}
%\newtheorem{thm}{Theorem}[section] 
%\newtheorem{defn}[thm]{Definition}
%\newtheorem{algorithm}{Algorithm}[section]
%\newtheorem{cor}{Corollary}
\newcommand{\BEQA}{\begin{eqnarray}}
\newcommand{\EEQA}{\end{eqnarray}}
\newcommand{\define}{\stackrel{\triangle}{=}}

\bibliographystyle{IEEEtran}
%\bibliographystyle{ieeetr}


\providecommand{\mbf}{\mathbf}
\providecommand{\pr}[1]{\ensuremath{\Pr\left(#1\right)}}
\providecommand{\qfunc}[1]{\ensuremath{Q\left(#1\right)}}
\providecommand{\sbrak}[1]{\ensuremath{{}\left[#1\right]}}
\providecommand{\lsbrak}[1]{\ensuremath{{}\left[#1\right.}}
\providecommand{\rsbrak}[1]{\ensuremath{{}\left.#1\right]}}
\providecommand{\brak}[1]{\ensuremath{\left(#1\right)}}
\providecommand{\lbrak}[1]{\ensuremath{\left(#1\right.}}
\providecommand{\rbrak}[1]{\ensuremath{\left.#1\right)}}
\providecommand{\cbrak}[1]{\ensuremath{\left\{#1\right\}}}
\providecommand{\lcbrak}[1]{\ensuremath{\left\{#1\right.}}
\providecommand{\rcbrak}[1]{\ensuremath{\left.#1\right\}}}
\theoremstyle{remark}
\newtheorem{rem}{Remark}
\newcommand{\sgn}{\mathop{\mathrm{sgn}}}
\providecommand{\abs}[1]{\left\vert#1\right\vert}
\providecommand{\res}[1]{\Res\displaylimits_{#1}} 
\providecommand{\norm}[1]{\left\lVert#1\right\rVert}
%\providecommand{\norm}[1]{\lVert#1\rVert}
\providecommand{\mtx}[1]{\mathbf{#1}}
\providecommand{\mean}[1]{E\left[ #1 \right]}
\providecommand{\fourier}{\overset{\mathcal{F}}{ \rightleftharpoons}}
%\providecommand{\hilbert}{\overset{\mathcal{H}}{ \rightleftharpoons}}
\providecommand{\system}{\overset{\mathcal{H}}{ \longleftrightarrow}}
	%\newcommand{\solution}[2]{\textbf{Solution:}{#1}}
\newcommand{\solution}{\noindent \textbf{Solution: }}
\newcommand{\cosec}{\,\text{cosec}\,}
\providecommand{\dec}[2]{\ensuremath{\overset{#1}{\underset{#2}{\gtrless}}}}
\newcommand{\myvec}[1]{\ensuremath{\begin{pmatrix}#1\end{pmatrix}}}
\newcommand{\mydet}[1]{\ensuremath{\begin{vmatrix}#1\end{vmatrix}}}
%\numberwithin{equation}{section}
\numberwithin{equation}{subsection}
%\numberwithin{problem}{section}
%\numberwithin{definition}{section}
\makeatletter
\@addtoreset{figure}{problem}
\makeatother

\let\StandardTheFigure\thefigure
\let\vec\mathbf
%\renewcommand{\thefigure}{\theproblem.\arabic{figure}}
\renewcommand{\thefigure}{\theproblem}
%\setlist[enumerate,1]{before=\renewcommand\theequation{\theenumi.\arabic{equation}}
%\counterwithin{equation}{enumi}


%\renewcommand{\theequation}{\arabic{subsection}.\arabic{equation}}

\def\putbox#1#2#3{\makebox[0in][l]{\makebox[#1][l]{}\raisebox{\baselineskip}[0in][0in]{\raisebox{#2}[0in][0in]{#3}}}}
     \def\rightbox#1{\makebox[0in][r]{#1}}
     \def\centbox#1{\makebox[0in]{#1}}
     \def\topbox#1{\raisebox{-\baselineskip}[0in][0in]{#1}}
     \def\midbox#1{\raisebox{-0.5\baselineskip}[0in][0in]{#1}}

\vspace{3cm}


\title{Quiz 8}
\author{S Nithish}
% make the title area
\maketitle

\newpage

%\tableofcontents

\bigskip

\renewcommand{\thefigure}{\theenumi}
\renewcommand{\thetable}{\theenumi}
%\renewcommand{\theequation}{\theenumi}


\begin{abstract}
This document contains the solution of the question from NCERT 12th standard chapter 11 exercise 11.4 problem 20
\end{abstract}

%Download all python codes 
%
%\begin{lstlisting}
%svn co https://github.com/JayatiD93/trunk/My_solution_design/codes
%\end{lstlisting}

%Download all and latex-tikz codes from 
%
%\begin{lstlisting}
%svn co https://github.com/gadepall/school/trunk/ncert/geometry/figs
%\end{lstlisting}
%
\section{Exercise 11.4}
\begin{enumerate}

		\fi
The direction vectors of the lines 
are
\begin{align}
	\vec{m_1} = \myvec{3\\-16\\7},
	\vec{m_2} = \myvec{3\\8\\-5}
\end{align}
Let $\vec{m}$ denote the direction vector of the line perpendicular to the given two lines. Then,
\begin{align}
	\vec{m_1}^{\top}\vec{m} &= 0 
	\\
	\vec{m_2}^{\top}\vec{m} &= 0 
	\\
	\implies 
	\myvec{3 & -16 & 7\\3 & 8 & -5}\vec{m} = 0
\end{align}
Row reducing the augmented matrix, 
\begin{align}
	\xleftrightarrow[]{R_2\leftarrow R_2-R_1}
 	\myvec{3 & -16 & 7\\0 & 24 & -12}
	\xleftrightarrow[]{R_1\leftarrow R_1+\frac{2}{3}R_2}
	\myvec{3 & 0 & -1\\0 & 24 & -12}\\
	\xleftrightarrow[]{R_2\leftarrow R_2/12}
	\myvec{3 & 0 & -1\\0 & 2 & -1}
\end{align}
yielding
\begin{align}
	\vec{m} = \myvec{2\\3\\6}
\end{align}
Hence the vector equation of the line passing through $\brak{1,2,-4}$ is,
\begin{align}
	\vec{x} = \myvec{1\\2\\-4} + \lambda \myvec{2\\3\\6}
\end{align}



	\item  Find the vector equation of the line passing through $\myvec{1\\2\\3}$ and parallel to the planes $\myvec{1\\-1\\2}^{\top}\vec{x} = 5$ and $\myvec{3\\1\\1}^{\top}\vec{x} = 6$.  
    \solution
		\iffalse
\documentclass[journal,10pt,twocolumn]{article}
\usepackage{graphicx}
\usepackage[margin=0.5in]{geometry}
\usepackage[cmex10]{amsmath}
\usepackage{array}
\usepackage{booktabs}
\usepackage{listings}
\title{\textbf{Line Assignment}}
\author{Bhavani Kanike}
\date{October 2022}

\providecommand{\norm}[1]{\left\lVert#1\right\rVert}
\providecommand{\abs}[1]{\left\vert#1\right\vert}
\let\vec\mathbf
\newcommand{\myvec}[1]{\ensuremath{\begin{pmatrix}#1\end{pmatrix}}}
\newcommand{\mydet}[1]{\ensuremath{\begin{vmatrix}#1\end{vmatrix}}}
\providecommand{\brak}[1]{\ensuremath{\left(#1\right)}}

\begin{document}

\maketitle
\paragraph{\textit{Problem Statement} 
\fi
ABCD is a quadrilateral in which $\vec{P}, \vec{Q}, \vec{R}$ and $\vec{S}$ are mid-points of the sides AB, BC, CD and DA (see Fig \ref{fig:9/8/2/1}). AC is a diagonal. 
		
Show that 
\begin{enumerate}
	\item $SR \parallel AC$ and $SR =\frac{1}{2} AC$
\item $PQ = SR$
\item $PQRS$ is a parallelogram.
\end{enumerate}
 	\begin{figure}
		\centering
 \includegraphics[width=\columnwidth]{chapters/9/8/2/1/figs/line1.pdf}
		\caption{}
		\label{fig:9/8/2/1}
  	\end{figure}
	\solution 
	Using 
	  \eqref{eq:section_formula},
	\begin{align}
		\label{eq:9/8/2/1}
		\begin{split}
		\vec{P} &= \frac{\vec{A}+\vec{B}}{2}\\
 \vec{Q} &= \frac{\vec{C}+\vec{B}}{2}\\
 \vec{R} &= \frac{\vec{C}+\vec{D}}{2}\\
 \vec{S} &= \frac{\vec{D}+\vec{A}}{2}
		\end{split}
	\end{align}
\begin{enumerate}
	\item
	Consequently, 
	\begin{align}
\vec{R}
		-\vec{S} &= \frac{\vec{C}-\vec{A}}{2}
		\\
		\implies SR &\parallel AC
	\end{align}
	Also, 
	\begin{align}
		\norm{\vec{R}
		-\vec{S}} &= \frac{\norm{\vec{C}-\vec{A}}}{2}
		\\
		\implies SR &= \frac{1}{2}AC
	\end{align}
\item 	From 
		\eqref{eq:9/8/2/1},
	\begin{align}
\vec{R}
		-\vec{S} = \vec{Q}-\vec{P}
	\end{align}
	which means that $PQRS$ is a parallelogram and $PQ = SR$.
\end{enumerate}
%
\iffalse
\begin{figure}[h]
\centering
\includegraphics[width=1\columnwidth]
\caption{Figure}
\label{fig:triangle}
\end{figure}

\section*{Solution}

$\boldsymbol Given :$  ABCD is a Quadrilateral P,Q,R and S are the midpoints of line AB,BC,CD,DA.We can obtain the points P,Q,R and S from A,B,C and D and are given by\\\\
\boldmath
\unboldmath
(3) To prove that PQRS is a parallelogram we need to prove  PQ // SR
To prove SR $\parallel$ PQ\\
Direction vector of line SR  $\boldsymbol {(R-S) =  \frac{(C-A)}{2}}$\\\\
Direction vector of line PQ  $\boldsymbol {(Q-P)= \frac{(C-A)}{2}}$\\\\
\begin{equation}
	\boldsymbol {(R-S) = (Q-P) = \frac{(C-A)}{2}}\\
\end{equation}
Since the direction vectors of line SR and PQ are in same direction\\\\
$SR \parallel PQ$\\
Therefore,
$\boldsymbol{ PQRS }$ is a parallelogram\\\\

	
(1)  Directional vector of line SR  = $\boldsymbol {(R-S)}$ = $\frac{\boldsymbol{(C-A)}}{2} $\\
Directional vector of line AC  = $\boldsymbol {(C-A)}$\\

It is observed that the constant k is $\frac{1}{2}$

Therefore
\begin{equation}
	SR \parallel AC
\end{equation} 

and from equation 1 
\begin{equation}
	\boldsymbol {SR = \frac{1}{2}AC}    
\end{equation}\\


(2)   To prove PQ = SR\\ 
		From euqation 1\\\\
\begin{equation}
		\boldsymbol{ (Q-P) = (R-S) = \frac{(C-A)}{2}}
\end{equation}
	 



\section{Execution}
The below python code realizes the construction:
\begin{lstlisting}
https://github.com/bhavani360/FWC_assignments
\end{lstlisting}
	
\section*{Construction}
The dimensions of the Quadrilateral ABCD are taken as below\\
{
\setlength\extrarowheight{2pt}
\centering
	\begin{tabular}{|c|c|}
	\hline
	\textbf{symbol}&\textbf{value}\\
	\hline
	r&8\\
	\hline
	$\theta$&pi/2.5\\
	\hline
	d&7\\
	\hline
	A&(0,0)\\
	\hline
	B&(d,0)\\
	\hline
	D&(rcos$\theta$,rsin$\theta$)\\
	\hline
	C&(D/1.5)+B\\
	\hline
\end{tabular}
}
\end{document}
\fi

	\item
		 Two lines passing through the point(2,3) intersect each other at an angle of $60\degree$.If slope of one line is 2,find equation of the other line.

\textbf{Solution :}

\begin{table}[H]
    \centering
    \begin{tabular}{|c|c|c|}
    \hline
        \textbf{Symbol} &\textbf{Description}&\textbf{Value}  \\
        \hline
         $m_1$&Slope of one line &2\\
         \hline
         $m_2$&Slope of another line&$m$\\
         \hline
         $\vec{P}$&Intersecting point&$\myvec{
             2\\3
         }$\\
         \hline
         $\theta$&Angle between two lines&$60\degree$\\
         \hline
    \end{tabular}


    \caption{Table of input parameters}
    \label{tab:11.10.3.12.1}
\end{table}



\begin{table}[H]
    \centering
    \begin{tabular}{|c|c|c|}
    \hline
        \textbf{Symbol} &\textbf{Description}&\textbf{Value}  \\
        \hline
         $\vec{m_1}$&Direction vector of one line&$\myvec{
             1\\2
         }$\\
         \hline
         $\vec{n_1}$&Normal vector of one line&$\myvec{
             2\\-1
         }$\\
         \hline
        $\vec{m_2}$&Direction vector of another line&$\myvec{
             1\\m
         }$\\
         \hline
          $\vec{n_2}$&Normal vector of another line&$\myvec{
             m\\-1
         }$\\
         \hline
    \end{tabular}


    \caption{Table of output parameters}
    \label{tab:11.10.3.12}
\end{table}


So,
\begin{align}
  \cos{60\degree}&=\frac{\vec{\brak{m_1^T}m_2}}{\vec{\norm{m_1}\norm{m_2}}}\\
    or,\frac{1}{2}&=\frac{\myvec{
        1&2
    }\myvec{
        1\\m
    }}{\sqrt{5}\sqrt{m^2+1}}\\
   or, \frac{1}{2}&=\frac{1+2m}{\sqrt{5m^2+5}}\\
   or,11m^2+16m-1&=0\\
   or, m&=\frac{-8\pm5\sqrt{3}}{11}    
\end{align}
Therefore,the direction vector is,$\vec{m_2}=\myvec{
    1\\\frac{-8+5\sqrt{3}}{11}
}$ or,$\myvec{
    1\\\frac{-8-5\sqrt{3}}{11}}$

The normal vector is,$\vec{n}=\myvec{
    \frac{-8+5\sqrt{3}}{11}\\-1
}$or,$\myvec{
    \frac{-8-5\sqrt{3}}{11}\\-1
}$

So, the equation of the line is
\begin{align}
\vec{n^Tx}=c\\
\myvec{
    \frac{-8\pm5\sqrt{3}}{11}&-1
}\vec{x}&=c
\end{align}
Passes through the point $\vec{P}=\myvec{
    2\\3}$
\begin{align}
\myvec{
    \frac{-8\pm5\sqrt{3}}{11}&-1
}\vec{P}=c\\
 or,c=\frac{-49\pm16\sqrt{3}}{11}
\end{align}


\textbf{Figure :}
\begin{figure}[H]
    \centering
    \includegraphics[width=\columnwidth]{chapters/11/10/3/12/fig/asgnt1.png}
    \caption{}
    \label{fig:11.10.3.12}
\end{figure}


\label{chapters/11/10/3/12}
\item
Find the value of $p$ so that the three lines $3x+y-2=0,px+2y-3=0$ and $2x-y-3=0$ may intersect at one point.


\textbf{Solution :}
\begin{align}  
&\myvec{
    3 &1&-2 \\
     $p$&2&-3\\
     2&-1&-3
}\\
\xrightarrow[R_2'=R_2+2R_3]{R_1'=R_1+R_3}&\myvec{
    5&0&-5\\
     $4+p$&0&-9\\
     2&-1&-3
}
\end{align}
For intersecting at one point the above expression should be zero.
So,
\begin{align}
 \begin{tabular}{|c c c|}
    5&0&-5\\
     $4+p$&0&-9\\
     2&-1&-3
\end{tabular}&=0\\
or,0+0+1\begin{tabular}{|c c|}
    5&-5  \\
    4+p &-9
\end{tabular}&=0\\
or,p&=5
\end{align}


\begin{figure}[H]
    \centering
	\includegraphics[width=\columnwidth]{chapters/11/10/4/9/fig/11.10.4.9.png}
    \caption{}
    \label{11.10.4.9}
\end{figure}

%\label{11.10.4.9}
\item
	Show that the line through the points \brak{4,7,8},\brak{2,3,4} is parallel to the line through the points\brak{-1,-2,1},\brak{1,2,5}.

\textbf{Solution :}
For line passing through \brak{4,7,8},\brak{2,3,4},the direction vector,\begin{align}
    \vec{m_1}&=\myvec{-2\\-4\\-4}
\end{align}
For line passing through \brak{-1,-2,1},\brak{1,2,5},the direction vector,\begin{align}
    \vec{m_2}&=\myvec{2\\4\\4}
\end{align}
Therefore,the lines are parallel to each other.

	\label{12.11.2.3}
\end{enumerate}

\subsection{Perpendicular}
\begin{enumerate}[label=\thesection.\arabic*,ref=\thesection.\theenumi]
\numberwithin{equation}{enumi}
\numberwithin{figure}{enumi}
\numberwithin{table}{enumi}

\item 
\item 
\item  Reduce the following equations into normal form. Find their perpendicular distances from the origin and angle between perpendicular and the positive $x$-axis.
\label{chapters/11/10/3/3}
\begin{enumerate}
	\item $x-\sqrt{3}y+8=0$ 
	\item $y-2=0$
	\item $x-y=4$
\end{enumerate}
\solution
\begin{enumerate}[label=\thesection.\arabic*,ref=\thesection.\theenumi]
\numberwithin{equation}{enumi}
\numberwithin{figure}{enumi}
\numberwithin{table}{enumi}

	\item The distance of the point $\vec{P}(2, 3)$ from the x-axis is

\begin{enumerate}
\item 2
\item 3
\item 1
\item 5 
\end{enumerate}

\item Find the foot of perpendicular from the point $(2,3,-8)$ to the line  $\dfrac{4-x}{2}=\dfrac{y}{6}=\dfrac{1-z}{3}$.Also, find the perpendicular distance from the given point to the line.
\item Find the distance of a point $(2,4,-1)$ from the line $$\frac{x+5}{1}=\frac{y+3}{4}=\frac{z-6}{-9}$$
\item Find the length and the foot of perpendicular from the point $ \brak{1,\dfrac{3}{2} ,2 }$ to the plane $2x-2y+4z+5=0.$
\item Show that the points $(\hat{i}-\hat{j}+3\hat{k})$ and $3(\hat{i}+\hat{j}+\hat{k})$ are equidistant from the plane $\overrightarrow{r} \cdot (5\hat{i}+2\hat{j}-7\hat{k})+9=0$ and lies on opposite side of it.
\item The distance of the plane $\overrightarrow{r} \cdot \brak{ \dfrac{2}{7}\hat{i}+\dfrac{3}{7}\hat{j}-\dfrac{6}{7}\hat{k}}=1$ from the origin is 
\begin{enumerate}
	\item 1
	\item 7
	\item $\dfrac{1}{7}$
	\item None of these	
\end{enumerate}
\item If the foot of perpendicular drawn from the origin to a plane is $(5,-3,-2)$, then the equation of plane is $\overrightarrow{r} \cdot (5\hat{i}-3\hat{j}-2\hat{k})=38.$
\end{enumerate}

\item Find the distance of the point $(-1,1)$ from the line $12\brak{x+6} = 5\brak{y-2}$. 
\label{chapters/11/10/3/4}
\iffalse
\documentclass[12pt]{article}
\usepackage{graphicx}
\usepackage[none]{hyphenat}
\usepackage{graphicx}
\usepackage{listings}
\usepackage[english]{babel}
\usepackage{graphicx}
\usepackage{caption} 
\usepackage{booktabs}
\usepackage{array}
\usepackage{amssymb} % for \because
\usepackage{amsmath}   % for having text in math mode
\usepackage{extarrows} % for Row operations arrows
\usepackage{listings}
\lstset{
  frame=single,
  breaklines=true
}
\usepackage{hyperref}
  
%Following 2 lines were added to remove the blank page at the beginning
\usepackage{atbegshi}% http://ctan.org/pkg/atbegshi
\AtBeginDocument{\AtBeginShipoutNext{\AtBeginShipoutDiscard}}


%New macro definitions
\newcommand{\mydet}[1]{\ensuremath{\begin{vmatrix}#1\end{vmatrix}}}
\providecommand{\brak}[1]{\ensuremath{\left(#1\right)}}
\providecommand{\norm}[1]{\left\lVert#1\right\rVert}
\newcommand{\solution}{\noindent \textbf{Solution: }}
\newcommand{\myvec}[1]{\ensuremath{\begin{pmatrix}#1\end{pmatrix}}}
\providecommand{\abs}[1]{\left\vert#1\right\vert}
\let\vec\mathbf

\begin{document}

\begin{center}
\title{\textbf{Equation  of Line}}
\date{\vspace{-5ex}} %Not to print date automatically
\maketitle
\end{center}
\setcounter{page}{1}

\section{11$^{th}$ Maths - Chapter 10}
This is Problem-4 from Exercise 10.3
\begin{enumerate}
		\fi
\item Find the distance of the point $(-1,1)$ from the line $12\brak{x+6} = 5\brak{y-2}$. 
	\\
\solution 
\begin{enumerate}
\item The equation of the line is $12\brak{x+6} = 5\brak{y-2}$. Rearranging the equation, 
\begin{align}
12x-5y = -10-72 \\
12x-5y = -82
\end{align}

This can be equated to

\begin{align}
	\label{eq:11/10/3/4/2Dline}
	\vec{n}^\top\vec{x} = c 
\end{align}
\begin{align}
	\text{ where }
		\vec{n} = \myvec{
	  12 \\
	  -5 
	  } ,   c = -82 
\end{align}
		We need to compute the distance from a point $\vec{P}\myvec{-1 \\ 1}$ to the line. 
Without loss of generality, let $\vec{A}$ be the foot of the perpendicular from $\vec{P}$ to the line in Equation \eqref{eq:11/10/3/4/2Dline}. 
The equation of the normal to Equation \eqref{eq:11/10/3/4/2Dline} can then be expressed as 

\begin{align}
	\label{eq:11/10/3/4/dir_line_normal_dist}
	\vec{x} &= \vec{A} + \lambda \vec{n}
	\\
	\implies 
	\label{eq:11/10/3/4/dir_line_normal_dist_pa}
	\vec{P}- \vec{A} &=  \lambda \vec{n}
\end{align}

$\because \vec{P}$ lies on 
		\eqref{eq:11/10/3/4/dir_line_normal_dist}.
From the above, the desired distance can be expressed as 

\begin{align}
	\label{eq:11/10/3/4/dir_line_normal_dist_pa_d}
d = 	\norm{\vec{P}- \vec{A}}= \abs{\lambda} \norm{\vec{n}}
\end{align}

From 
	\eqref{eq:11/10/3/4/dir_line_normal_dist_pa},

\begin{align}
	\vec{n}^{\top}
	\brak{\vec{P}- \vec{A}} &=  \lambda \vec{n}^{\top}\vec{n} = \lambda\norm{\vec{n}}^2
	\\
	\implies \abs{\lambda}&= \frac{\abs{\vec{n}^{\top}
	\brak{\vec{P}- \vec{A}}}}{\norm{\vec{n}}^2} 
\end{align}

Substituting the above in \eqref{eq:11/10/3/4/dir_line_normal_dist_pa_d} and using the fact that

\begin{align}
   \vec{n}^{\top}\vec{A} = c
\end{align}

from 	\eqref{eq:11/10/3/4/2Dline}, yields 

\begin{align}
	\label{eq:11/10/3/4/line_dist_2d}
	d = \frac{\abs{   \vec{n}^{\top}\vec{P}-c }}{\norm{\vec{n}}}	
\end{align}

\begin{align}
	= \frac{\abs{  \myvec{12 & -5 }\myvec{-1 \\ 1}-\brak{-82} }}{\sqrt{12^2+\brak{-5}^2}} \\	
	= \frac{\abs{  -17 + 82 }}{\sqrt{169}}	
	= \frac{\abs{65 }}{13}
	= 5 \text{ units }
\end{align}
\item The foot of the perpendicular from $\vec{P}\myvec{-1 \\ 1}$ to line in \eqref{eq:11/10/3/4/2Dline} is expressed as
\begin{align}
	\label{eq:11/10/3/4/foot_of_perpendicular}
	\myvec{\vec{m} & \vec{n}}^\top\vec{A} &= 
	   \myvec{
              \vec{m}^\top\vec{P}\\
	      c
	      }
\end{align}
where $\vec{m}$ is the direction vector of the given line
\begin{align}
    \because \vec{n} = \myvec{ 12 \\ -5},   
    \vec{m} = \myvec{ 5 \\ 12} \\ 
	\eqref{eq:11/10/3/4/foot_of_perpendicular} \implies \myvec{5&12 \\ 12 & -5}\vec{A} &= \myvec{\myvec{5 & 12}\myvec{-1 \\ 1}\\ -82} \\
	\label{eq:11/10/3/4/sysEq1}
	\myvec{5&12 \\ 12 & -5}\vec{A} &= \myvec{7 \\ -82} 
\end{align}	
The augmented matrix for the system equations in \eqref{eq:11/10/3/4/sysEq1} is expressed as
\begin{align}
	\myvec{5&12 & \vrule & 7 \\ 12 & -5 & \vrule & -82} 
\end{align}
Performing sequence of row operations to transform into RREF form
\begin{align}
        \xleftrightarrow[]{{R_2\rightarrow R_2-\frac{12}{5}R_1}}  
	\myvec{5&12 & \vrule & 7 \\ 0 & -\frac{169}{5} & \vrule & -\frac{494}{5}} \\
	\xleftrightarrow[{R_1\rightarrow \frac{1}{5}}R_1]{{R_2\rightarrow \frac{-5}{169}R_2}}  
	\myvec{1 & \frac{12}{5} & \vrule & \frac{7}{5} \\ 0 & 1 & \vrule & \frac{38}{13}} \\
	\xleftrightarrow[]{{R_1\rightarrow R_1-\frac{12}{5}R_2}}  
	\myvec{1 & 0 & \vrule & -\frac{73}{13} \\ 0 & 1 & \vrule & \frac{38}{13}} \\
	\vec{A} = \myvec{ -\frac{73}{13} \\ \frac{38}{13} }
\end{align}
\end{enumerate}
The desired line and the perpendicular line from $\vec{P}$ is shown as in Fig. \ref{fig:11/10/3/4/Fig1}
\begin{figure}[!h]
	\begin{center}
		\includegraphics[width=\columnwidth]{chapters/11/10/3/4/figs/problem4.pdf}
	\end{center}
\caption{}
\label{fig:11/10/3/4/Fig1}
\end{figure}

\item Find the points on the x-axis, whose distances from the line $\frac{x}{3}+\frac{y}{4}=1$ are 4 units.
\label{chapters/11/10/3/5}
	\\
	\solution
\iffalse
\documentclass[12pt]{article}
\usepackage{graphicx}
\usepackage{amsmath}
\usepackage{mathtools}
\usepackage{gensymb}
\usepackage{amssymb}

\newcommand{\mydet}[1]{\ensuremath{\begin{vmatrix}#1\end{vmatrix}}}
\providecommand{\brak}[1]{\ensuremath{\left(#1\right)}}
\providecommand{\norm}[1]{\left\lVert#1\right\rVert}
\newcommand{\solution}{\noindent \textbf{Solution: }}
\newcommand{\myvec}[1]{\ensuremath{\begin{pmatrix}#1\end{pmatrix}}}
\providecommand{\abs}[1]{\left\vert#1\right\vert}	
\let\vec\mathbf

\begin{document}
\begin{center}
\textbf\large{CLASS 11 CHAPTER-11 \\ LINES}

\end{center}
\section*{Exercise 10.3}


\solution
\fi
The given line can be expressed as 
\begin{align}
	\vec{n}^{\top}\vec{x}&=c,
	\text{ where }
		\vec{n} &= \myvec{4\\3} , c = 12
\end{align}
The distance formula is given by
\begin{align}
	d = \frac{\abs{\vec{n}^\top\vec{P}-c}}{\norm{\vec{n}}}
\end{align}
Let the desired point be
\begin{align}
	\vec{P} = x\vec{e}_{1} = \myvec{x\\0}
\end{align}
Substituting the values in the distance formula, 
\begin{align}
	d &= \frac{\abs{\vec{n}^\top\vec{P}-c}}{\norm{\vec{n}}}\\
	  &= \frac{\abs{x\vec{n}^\top\vec{e}_{1}-c}}{\norm{\vec{n}}}
	  \\
	  \implies 
	\abs{x\vec{n}^\top\vec{e}_{1}-c} &= d\norm{\vec{n}}
	\\
	\text{or, }	x = \frac{\pm d\norm{\vec{n}}+c}{\vec{n}^\top\vec{e}_{1}}
\end{align}
Since 
\begin{align}
	d &= 4,
\end{align}
substituting numerical values, 
\begin{align}
	x = 8,
	 -2
\end{align}
This is verified in Fig. 
\ref{fig:11/10/3/5/Fig1}.	
\begin{figure}[!h]
	\begin{center} 
	    \includegraphics[width=\columnwidth]{chapters/11/10/3/5/figs/line2}
	\end{center}
\caption{}
\label{fig:11/10/3/5/Fig1}
\end{figure}



\item Find the distance between parallel lines
\label{chapters/11/10/3/6}
\begin{enumerate}
	\item $15x+8y-34=0$ and  $15x+8y+31=0$ \\
	\item  $l(x+y)+p=0$ and  $l(x+y)-r=0$
\end{enumerate}
	\solution
\iffalse
\documentclass[10pt]{article}
       \usepackage[latin1]{inputenc}
       \usepackage{fullpage}
       \usepackage{color}
       \usepackage{array}
       \usepackage{longtable}
       \usepackage{calc}
       \usepackage{multirow}
       \usepackage{hhline}
       \usepackage{ifthen}
\usepackage{graphicx}
\def\inputGnumericTable{}
\usepackage[none]{hyphenat}
\usepackage{graphicx}
\usepackage{listings}
\usepackage[english]{babel}
\usepackage{graphicx}
\usepackage{caption} 
\usepackage{booktabs}
\usepackage{gensymb}
\usepackage{array}
\usepackage{amssymb} % for \because
\usepackage{amsmath}   % for having text in math mode
\usepackage{extarrows} % for Row operations arrows
\usepackage{listings}
\lstset{
  frame=single,
  breaklines=true
}
\usepackage{hyperref}
%Following 2 lines were added to remove the blank page at the beginning
\usepackage{atbegshi}% http://ctan.org/pkg/atbegshi
\AtBeginDocument{\AtBeginShipoutNext{\AtBeginShipoutDiscard}}
%New macro definitions
\newcommand{\mydet}[1]{\ensuremath{\begin{vmatrix}#1\end{vmatrix}}}
\providecommand{\brak}[1]{\ensuremath{\left(#1\right)}}
\providecommand{\norm}[1]{\left\lVert#1\right\rVert}
\newcommand{\solution}{\noindent \textbf{Solution: }}
\newcommand{\myvec}[1]{\ensuremath{\begin{pmatrix}#1\end{pmatrix}}}
\providecommand{\abs}[1]{\left\vert#1\right\vert}
\let\vec\mathbf
\begin{document}

\begin{center}
\title{\textbf{STRAIGHT LINES}}
\date{\vspace{-5ex}} %Not to print date automatically
\maketitle
\end{center}

\section{11$^{th}$ Maths - Chapter 10}
This is Problem 5 from Exercise-10.4
\begin{enumerate}

\solution
\fi
Let
\begin{align}
	\vec{A}=\myvec{\cos\theta\\\sin\theta},\vec{B}&=\myvec{\cos\phi\\\sin\phi}\\
\implies	\vec{m}=\vec{B}-\vec{A}&=\myvec{\cos\phi-\cos\theta\\\sin\phi-\sin\theta}
\end{align}
The normal vector is then given by,
\begin{align}
\vec{n}=\myvec{\sin\phi-\sin\theta\\\cos\theta-\cos\phi} \implies
\norm{\vec{n}}=2\sin\brak{\frac{\phi-\theta}{2}}
\end{align}
The equation of the line is
\begin{align}
\vec{n}^\top\brak{\vec{x}-\vec{A}}&=0\\
\implies\myvec{\sin\phi-\sin\theta&\cos\theta-\cos\phi}\vec{x}&=\sin\brak{\phi-\theta}
\label{eq:chapters/11/10/4/5/1}
\end{align}
Thus, 
\begin{align}
c=\sin\brak{\phi-\theta}
\end{align}
The perpendicular distance from the origin to the line is
\begin{align}
d&=\frac{\abs{c}}{\norm{\vec{n}}}\\
\implies d&=\frac{\sin\brak{\phi-\theta}}{2\sin\brak{\frac{\phi-\theta}{2}}} = \cos\brak{\frac{\phi-\theta}{2}}
\label{eq:chapters/11/10/4/5/2}
\end{align}

\item Find the coordinates of the foot of the perpendicular from $(-1, 3)$ to the line $3x-4y-16=0$.  
\label{chapters/11/10/3/14}
\\
\solution
\iffalse
\documentclass[12pt]{article}
\usepackage{graphicx}
%\documentclass[journal,12pt,twocolumn]{IEEEtran}
\usepackage[none]{hyphenat}
\usepackage{graphicx}
\usepackage{listings}
\usepackage[english]{babel}
\usepackage{graphicx}
\usepackage{caption} 
\usepackage{hyperref}
\usepackage{booktabs}
\usepackage{commath}
\usepackage{gensymb}
\usepackage{array}
\usepackage{amsmath}   % for having text in math mode
\usepackage{mathtools}
\usepackage{listings}
\let\vec\mathbf
\lstset{
  frame=single,
  breaklines=true
}
  
%Following 2 lines were added to remove the blank page at the beginning
\usepackage{atbegshi}% http://ctan.org/pkg/atbegshi
\AtBeginDocument{\AtBeginShipoutNext{\AtBeginShipoutDiscard}}
%
%New macro definitions
\newcommand{\mydet}[1]{\ensuremath{\begin{vmatrix}#1\end{vmatrix}}}
\providecommand{\brak}[1]{\ensuremath{\left(#1\right)}}
\providecommand{\norm}[1]{\left\lVert#1\right\rVert}
\newcommand{\solution}{\noindent \textbf{Solution: }}
\newcommand{\myvec}[1]{\ensuremath{\begin{pmatrix}#1\end{pmatrix}}}
\let\vec\mathbf
\begin{document}
\begin{center}
\title{\textbf{LINES}}
\date{\vspace{-5ex}} %Not to print date automatically
\maketitle
\end{center}
\setcounter{page}{1}
\section*{CHAPTER 11 - STRAIGHT LINES}
\section*{Excercise 10.3}
\solution 
\begin{enumerate}
\section{Solution}
		\fi
		Let
\begin{align}
 \vec{P}=\myvec{
-1\\
3
}
\end{align}
The line parameters are
\begin{align}
\vec{n}=\myvec{
3\\
-4
}, c=16
\end{align}
The desired foot of the perpendicular is then given by 
\begin{align}
\myvec{4&3\\3&-4}\vec{A}&=\myvec{\myvec{4&3}\myvec{-1\\3}\\16}\\
&=\myvec{5\\16}  
\end{align}
The augmented matrix for the above system is
\begin{align}
  \myvec{
   4 &  3  & 5\\
   3 & -4  & 16} 
  \xleftrightarrow[]{R_2=R_2-\frac{3}{4}R_1}
  \myvec{
  4 & 3 & 5\\
  0 & \frac{-25}{4} & \frac{49}{4}} 
\\
  \xleftrightarrow{R_2=\frac{-4}{25}}
  \myvec{
  4 & 3 & 5\\
  0 & 1 & \frac{-49}{25}}
  \xleftrightarrow{R_1=\frac{1}{4}R_1}
  \myvec{
  1 & \frac{3}{4} & \frac{5}{4}\\
  0 & 1 & \frac{-49}{25}}
\\
  \xleftrightarrow{R_1=R_1-\frac{3}{4}R_2}
  \myvec{
  1 & 0 & \frac{68}{25}\\
  0 & 1 & \frac{-49}{25}}          
\end{align}
yielding
\begin{align}
\vec{A}=\myvec{
\frac{68}{25}\\[1pt]
\frac{-49}{25}
}
\end{align}
See Fig.
\ref{fig:chapters/11/10/3/14/Fig}.
\begin{figure}[!h]
	\begin{center} 
	    \includegraphics[width=\columnwidth]{chapters/11/10/3/14/figs/lines.png}
	\end{center}
\caption{}
\label{fig:chapters/11/10/3/14/Fig}
\end{figure}

\item  If ${p}$ and ${q}$ are the lengths of perpendiculars from the origin to the lines ${x}\cos\theta - {y}\sin\theta =  {k}\cos2\theta$ and ${x}\sec\theta + {y}\cosec\theta = {k}$, respectively, prove that ${p}^2 + 4{q}^2 = {k}^2$
\label{chapters/11/10/3/16}
\\
\solution
\iffalse
\documentclass[journal,10pt,twocolumn]{article}
\usepackage{graphicx}
\usepackage[margin=0.5in]{geometry}
\usepackage[cmex10]{amsmath}
\usepackage{array}
\usepackage{booktabs}
\usepackage{listings}
\title{\textbf{Line Assignment}}
\author{Bhavani Kanike}
\date{October 2022}

\providecommand{\norm}[1]{\left\lVert#1\right\rVert}
\providecommand{\abs}[1]{\left\vert#1\right\vert}
\let\vec\mathbf
\newcommand{\myvec}[1]{\ensuremath{\begin{pmatrix}#1\end{pmatrix}}}
\newcommand{\mydet}[1]{\ensuremath{\begin{vmatrix}#1\end{vmatrix}}}
\providecommand{\brak}[1]{\ensuremath{\left(#1\right)}}

\begin{document}

\maketitle
\paragraph{\textit{Problem Statement} 
\fi
ABCD is a quadrilateral in which $\vec{P}, \vec{Q}, \vec{R}$ and $\vec{S}$ are mid-points of the sides AB, BC, CD and DA (see Fig \ref{fig:9/8/2/1}). AC is a diagonal. 
		
Show that 
\begin{enumerate}
	\item $SR \parallel AC$ and $SR =\frac{1}{2} AC$
\item $PQ = SR$
\item $PQRS$ is a parallelogram.
\end{enumerate}
 	\begin{figure}
		\centering
 \includegraphics[width=\columnwidth]{chapters/9/8/2/1/figs/line1.pdf}
		\caption{}
		\label{fig:9/8/2/1}
  	\end{figure}
	\solution 
	Using 
	  \eqref{eq:section_formula},
	\begin{align}
		\label{eq:9/8/2/1}
		\begin{split}
		\vec{P} &= \frac{\vec{A}+\vec{B}}{2}\\
 \vec{Q} &= \frac{\vec{C}+\vec{B}}{2}\\
 \vec{R} &= \frac{\vec{C}+\vec{D}}{2}\\
 \vec{S} &= \frac{\vec{D}+\vec{A}}{2}
		\end{split}
	\end{align}
\begin{enumerate}
	\item
	Consequently, 
	\begin{align}
\vec{R}
		-\vec{S} &= \frac{\vec{C}-\vec{A}}{2}
		\\
		\implies SR &\parallel AC
	\end{align}
	Also, 
	\begin{align}
		\norm{\vec{R}
		-\vec{S}} &= \frac{\norm{\vec{C}-\vec{A}}}{2}
		\\
		\implies SR &= \frac{1}{2}AC
	\end{align}
\item 	From 
		\eqref{eq:9/8/2/1},
	\begin{align}
\vec{R}
		-\vec{S} = \vec{Q}-\vec{P}
	\end{align}
	which means that $PQRS$ is a parallelogram and $PQ = SR$.
\end{enumerate}
%
\iffalse
\begin{figure}[h]
\centering
\includegraphics[width=1\columnwidth]
\caption{Figure}
\label{fig:triangle}
\end{figure}

\section*{Solution}

$\boldsymbol Given :$  ABCD is a Quadrilateral P,Q,R and S are the midpoints of line AB,BC,CD,DA.We can obtain the points P,Q,R and S from A,B,C and D and are given by\\\\
\boldmath
\unboldmath
(3) To prove that PQRS is a parallelogram we need to prove  PQ // SR
To prove SR $\parallel$ PQ\\
Direction vector of line SR  $\boldsymbol {(R-S) =  \frac{(C-A)}{2}}$\\\\
Direction vector of line PQ  $\boldsymbol {(Q-P)= \frac{(C-A)}{2}}$\\\\
\begin{equation}
	\boldsymbol {(R-S) = (Q-P) = \frac{(C-A)}{2}}\\
\end{equation}
Since the direction vectors of line SR and PQ are in same direction\\\\
$SR \parallel PQ$\\
Therefore,
$\boldsymbol{ PQRS }$ is a parallelogram\\\\

	
(1)  Directional vector of line SR  = $\boldsymbol {(R-S)}$ = $\frac{\boldsymbol{(C-A)}}{2} $\\
Directional vector of line AC  = $\boldsymbol {(C-A)}$\\

It is observed that the constant k is $\frac{1}{2}$

Therefore
\begin{equation}
	SR \parallel AC
\end{equation} 

and from equation 1 
\begin{equation}
	\boldsymbol {SR = \frac{1}{2}AC}    
\end{equation}\\


(2)   To prove PQ = SR\\ 
		From euqation 1\\\\
\begin{equation}
		\boldsymbol{ (Q-P) = (R-S) = \frac{(C-A)}{2}}
\end{equation}
	 



\section{Execution}
The below python code realizes the construction:
\begin{lstlisting}
https://github.com/bhavani360/FWC_assignments
\end{lstlisting}
	
\section*{Construction}
The dimensions of the Quadrilateral ABCD are taken as below\\
{
\setlength\extrarowheight{2pt}
\centering
	\begin{tabular}{|c|c|}
	\hline
	\textbf{symbol}&\textbf{value}\\
	\hline
	r&8\\
	\hline
	$\theta$&pi/2.5\\
	\hline
	d&7\\
	\hline
	A&(0,0)\\
	\hline
	B&(d,0)\\
	\hline
	D&(rcos$\theta$,rsin$\theta$)\\
	\hline
	C&(D/1.5)+B\\
	\hline
\end{tabular}
}
\end{document}
\fi

\item In the triangle $ABC$ with vertices $\vec{A} \brak{2, 3}$, $\vec{B} \brak{4, –1}$ and $\vec{C} \brak{1, 2}$, find the equation and length of altitude from the vertex $\vec{A}$.
\label{chapters/11/10/3/17}
\\
\solution
\iffalse
\documentclass[journal,10pt,twocolumn]{article}
\usepackage{graphicx}
\usepackage[margin=0.5in]{geometry}
\usepackage[cmex10]{amsmath}
\usepackage{array}
\usepackage{booktabs}
\usepackage{listings}
\title{\textbf{Line Assignment}}
\author{Bhavani Kanike}
\date{October 2022}

\providecommand{\norm}[1]{\left\lVert#1\right\rVert}
\providecommand{\abs}[1]{\left\vert#1\right\vert}
\let\vec\mathbf
\newcommand{\myvec}[1]{\ensuremath{\begin{pmatrix}#1\end{pmatrix}}}
\newcommand{\mydet}[1]{\ensuremath{\begin{vmatrix}#1\end{vmatrix}}}
\providecommand{\brak}[1]{\ensuremath{\left(#1\right)}}

\begin{document}

\maketitle
\paragraph{\textit{Problem Statement} 
\fi
ABCD is a quadrilateral in which $\vec{P}, \vec{Q}, \vec{R}$ and $\vec{S}$ are mid-points of the sides AB, BC, CD and DA (see Fig \ref{fig:9/8/2/1}). AC is a diagonal. 
		
Show that 
\begin{enumerate}
	\item $SR \parallel AC$ and $SR =\frac{1}{2} AC$
\item $PQ = SR$
\item $PQRS$ is a parallelogram.
\end{enumerate}
 	\begin{figure}
		\centering
 \includegraphics[width=\columnwidth]{chapters/9/8/2/1/figs/line1.pdf}
		\caption{}
		\label{fig:9/8/2/1}
  	\end{figure}
	\solution 
	Using 
	  \eqref{eq:section_formula},
	\begin{align}
		\label{eq:9/8/2/1}
		\begin{split}
		\vec{P} &= \frac{\vec{A}+\vec{B}}{2}\\
 \vec{Q} &= \frac{\vec{C}+\vec{B}}{2}\\
 \vec{R} &= \frac{\vec{C}+\vec{D}}{2}\\
 \vec{S} &= \frac{\vec{D}+\vec{A}}{2}
		\end{split}
	\end{align}
\begin{enumerate}
	\item
	Consequently, 
	\begin{align}
\vec{R}
		-\vec{S} &= \frac{\vec{C}-\vec{A}}{2}
		\\
		\implies SR &\parallel AC
	\end{align}
	Also, 
	\begin{align}
		\norm{\vec{R}
		-\vec{S}} &= \frac{\norm{\vec{C}-\vec{A}}}{2}
		\\
		\implies SR &= \frac{1}{2}AC
	\end{align}
\item 	From 
		\eqref{eq:9/8/2/1},
	\begin{align}
\vec{R}
		-\vec{S} = \vec{Q}-\vec{P}
	\end{align}
	which means that $PQRS$ is a parallelogram and $PQ = SR$.
\end{enumerate}
%
\iffalse
\begin{figure}[h]
\centering
\includegraphics[width=1\columnwidth]
\caption{Figure}
\label{fig:triangle}
\end{figure}

\section*{Solution}

$\boldsymbol Given :$  ABCD is a Quadrilateral P,Q,R and S are the midpoints of line AB,BC,CD,DA.We can obtain the points P,Q,R and S from A,B,C and D and are given by\\\\
\boldmath
\unboldmath
(3) To prove that PQRS is a parallelogram we need to prove  PQ // SR
To prove SR $\parallel$ PQ\\
Direction vector of line SR  $\boldsymbol {(R-S) =  \frac{(C-A)}{2}}$\\\\
Direction vector of line PQ  $\boldsymbol {(Q-P)= \frac{(C-A)}{2}}$\\\\
\begin{equation}
	\boldsymbol {(R-S) = (Q-P) = \frac{(C-A)}{2}}\\
\end{equation}
Since the direction vectors of line SR and PQ are in same direction\\\\
$SR \parallel PQ$\\
Therefore,
$\boldsymbol{ PQRS }$ is a parallelogram\\\\

	
(1)  Directional vector of line SR  = $\boldsymbol {(R-S)}$ = $\frac{\boldsymbol{(C-A)}}{2} $\\
Directional vector of line AC  = $\boldsymbol {(C-A)}$\\

It is observed that the constant k is $\frac{1}{2}$

Therefore
\begin{equation}
	SR \parallel AC
\end{equation} 

and from equation 1 
\begin{equation}
	\boldsymbol {SR = \frac{1}{2}AC}    
\end{equation}\\


(2)   To prove PQ = SR\\ 
		From euqation 1\\\\
\begin{equation}
		\boldsymbol{ (Q-P) = (R-S) = \frac{(C-A)}{2}}
\end{equation}
	 



\section{Execution}
The below python code realizes the construction:
\begin{lstlisting}
https://github.com/bhavani360/FWC_assignments
\end{lstlisting}
	
\section*{Construction}
The dimensions of the Quadrilateral ABCD are taken as below\\
{
\setlength\extrarowheight{2pt}
\centering
	\begin{tabular}{|c|c|}
	\hline
	\textbf{symbol}&\textbf{value}\\
	\hline
	r&8\\
	\hline
	$\theta$&pi/2.5\\
	\hline
	d&7\\
	\hline
	A&(0,0)\\
	\hline
	B&(d,0)\\
	\hline
	D&(rcos$\theta$,rsin$\theta$)\\
	\hline
	C&(D/1.5)+B\\
	\hline
\end{tabular}
}
\end{document}
\fi

\item If $p$ is the length of perpendicular from origin to the line whose intercepts on the axes are $a$ and $b$, then show that 
\begin{align}
	\frac{1}{p^2} = \frac{1}{a^2}+ \frac{1}{b^2}
\end{align}
\label{chapters/11/10/3/18}
\iffalse
\documentclass[10pt]{article}
       \usepackage[latin1]{inputenc}
       \usepackage{fullpage}
       \usepackage{color}
       \usepackage{array}
       \usepackage{longtable}
       \usepackage{calc}
       \usepackage{multirow}
       \usepackage{hhline}
       \usepackage{ifthen}
\usepackage{graphicx}
\def\inputGnumericTable{}
\usepackage[none]{hyphenat}
\usepackage{graphicx}
\usepackage{listings}
\usepackage[english]{babel}
\usepackage{graphicx}
\usepackage{caption} 
\usepackage{booktabs}
\usepackage{gensymb}
\usepackage{array}
\usepackage{amssymb} % for \because
\usepackage{amsmath}   % for having text in math mode
\usepackage{extarrows} % for Row operations arrows
\usepackage{listings}
\lstset{
  frame=single,
  breaklines=true
}
\usepackage{hyperref}
%Following 2 lines were added to remove the blank page at the beginning
\usepackage{atbegshi}% http://ctan.org/pkg/atbegshi
\AtBeginDocument{\AtBeginShipoutNext{\AtBeginShipoutDiscard}}
%New macro definitions
\newcommand{\mydet}[1]{\ensuremath{\begin{vmatrix}#1\end{vmatrix}}}
\providecommand{\brak}[1]{\ensuremath{\left(#1\right)}}
\providecommand{\norm}[1]{\left\lVert#1\right\rVert}
\newcommand{\solution}{\noindent \textbf{Solution: }}
\newcommand{\myvec}[1]{\ensuremath{\begin{pmatrix}#1\end{pmatrix}}}
\providecommand{\abs}[1]{\left\vert#1\right\vert}
\let\vec\mathbf
\begin{document}

\begin{center}
\title{\textbf{STRAIGHT LINES}}
\date{\vspace{-5ex}} %Not to print date automatically
\maketitle
\end{center}

\section{11$^{th}$ Maths - Chapter 10}
This is Problem 5 from Exercise-10.4
\begin{enumerate}

\solution
\fi
Let
\begin{align}
	\vec{A}=\myvec{\cos\theta\\\sin\theta},\vec{B}&=\myvec{\cos\phi\\\sin\phi}\\
\implies	\vec{m}=\vec{B}-\vec{A}&=\myvec{\cos\phi-\cos\theta\\\sin\phi-\sin\theta}
\end{align}
The normal vector is then given by,
\begin{align}
\vec{n}=\myvec{\sin\phi-\sin\theta\\\cos\theta-\cos\phi} \implies
\norm{\vec{n}}=2\sin\brak{\frac{\phi-\theta}{2}}
\end{align}
The equation of the line is
\begin{align}
\vec{n}^\top\brak{\vec{x}-\vec{A}}&=0\\
\implies\myvec{\sin\phi-\sin\theta&\cos\theta-\cos\phi}\vec{x}&=\sin\brak{\phi-\theta}
\label{eq:chapters/11/10/4/5/1}
\end{align}
Thus, 
\begin{align}
c=\sin\brak{\phi-\theta}
\end{align}
The perpendicular distance from the origin to the line is
\begin{align}
d&=\frac{\abs{c}}{\norm{\vec{n}}}\\
\implies d&=\frac{\sin\brak{\phi-\theta}}{2\sin\brak{\frac{\phi-\theta}{2}}} = \cos\brak{\frac{\phi-\theta}{2}}
\label{eq:chapters/11/10/4/5/2}
\end{align}

\item What are the points on the y-axis whose distance from the line $\frac{x}{3}+\frac{y}{4}=1$ is 4 units.
\\
\solution
		\iffalse
\documentclass[12pt]{article}
\usepackage{graphicx}
\usepackage[none]{hyphenat}
\usepackage{graphicx}
\usepackage{listings}
\usepackage[english]{babel}
\usepackage{graphicx}
\usepackage{caption} 
\usepackage{booktabs}
\usepackage{array}
\usepackage{amssymb} % for \because
\usepackage{amsmath}   % for having text in math mode
\usepackage{extarrows} % for Row operations arrows
\usepackage{listings}
\usepackage[utf8]{inputenc}
\lstset{
  frame=single,
  breaklines=true
}
\usepackage{hyperref}
  
%Following 2 lines were added to remove the blank page at the beginning
\usepackage{atbegshi}% http://ctan.org/pkg/atbegshi
\AtBeginDocument{\AtBeginShipoutNext{\AtBeginShipoutDiscard}}


%New macro definitions
\newcommand{\mydet}[1]{\ensuremath{\begin{vmatrix}#1\end{vmatrix}}}
\providecommand{\brak}[1]{\ensuremath{\left(#1\right)}}
\newcommand{\solution}{\noindent \textbf{Solution: }}
\newcommand{\myvec}[1]{\ensuremath{\begin{pmatrix}#1\end{pmatrix}}}
\providecommand{\norm}[1]{\left\lVert#1\right\rVert}
\providecommand{\abs}[1]{\left\vert#1\right\vert}
\let\vec\mathbf

\begin{document}

\begin{center}
\title{\textbf{LINE}}
\date{\vspace{-5ex}} %Not to print date automatically
\maketitle
\end{center}

\section{11$^{th}$ Maths - EXERCISE-10.4}
\begin{enumerate}
\end{enumerate}
\section{SOLUTION}
\fi
Given line parameters are
\begin{align}
\vec{n}=\myvec{4\\3},\,
c=12.
\end{align}
The distance of the line from y-axis
\begin{align}
d&=\frac{\vec{n}^\top\vec{P}-c}{\abs{n}}\\
\implies\pm4&=\frac{\myvec{0\\ 3y}-12}{5}\\
	\implies y&= \frac{32}{3}\text{ or }y=\frac{-8}{3}
\end{align}
See Fig. 
		\ref{fig:chapters/11/10/4/4/Figure}.
\begin{figure}[h]
\centering
\includegraphics[width=\columnwidth]{chapters/11/10/4/4/figs/fig.png}
\caption{}
		\label{fig:chapters/11/10/4/4/Figure}
\end{figure}

\item Find perpendicular distance from the origin to the line joining the points$(\cos\theta,\sin\theta)$ and $(\cos\phi,\sin\phi)$.
\\
\solution
		\iffalse
\documentclass[10pt]{article}
       \usepackage[latin1]{inputenc}
       \usepackage{fullpage}
       \usepackage{color}
       \usepackage{array}
       \usepackage{longtable}
       \usepackage{calc}
       \usepackage{multirow}
       \usepackage{hhline}
       \usepackage{ifthen}
\usepackage{graphicx}
\def\inputGnumericTable{}
\usepackage[none]{hyphenat}
\usepackage{graphicx}
\usepackage{listings}
\usepackage[english]{babel}
\usepackage{graphicx}
\usepackage{caption} 
\usepackage{booktabs}
\usepackage{gensymb}
\usepackage{array}
\usepackage{amssymb} % for \because
\usepackage{amsmath}   % for having text in math mode
\usepackage{extarrows} % for Row operations arrows
\usepackage{listings}
\lstset{
  frame=single,
  breaklines=true
}
\usepackage{hyperref}
%Following 2 lines were added to remove the blank page at the beginning
\usepackage{atbegshi}% http://ctan.org/pkg/atbegshi
\AtBeginDocument{\AtBeginShipoutNext{\AtBeginShipoutDiscard}}
%New macro definitions
\newcommand{\mydet}[1]{\ensuremath{\begin{vmatrix}#1\end{vmatrix}}}
\providecommand{\brak}[1]{\ensuremath{\left(#1\right)}}
\providecommand{\norm}[1]{\left\lVert#1\right\rVert}
\newcommand{\solution}{\noindent \textbf{Solution: }}
\newcommand{\myvec}[1]{\ensuremath{\begin{pmatrix}#1\end{pmatrix}}}
\providecommand{\abs}[1]{\left\vert#1\right\vert}
\let\vec\mathbf
\begin{document}

\begin{center}
\title{\textbf{STRAIGHT LINES}}
\date{\vspace{-5ex}} %Not to print date automatically
\maketitle
\end{center}

\section{11$^{th}$ Maths - Chapter 10}
This is Problem 5 from Exercise-10.4
\begin{enumerate}

\solution
\fi
Let
\begin{align}
	\vec{A}=\myvec{\cos\theta\\\sin\theta},\vec{B}&=\myvec{\cos\phi\\\sin\phi}\\
\implies	\vec{m}=\vec{B}-\vec{A}&=\myvec{\cos\phi-\cos\theta\\\sin\phi-\sin\theta}
\end{align}
The normal vector is then given by,
\begin{align}
\vec{n}=\myvec{\sin\phi-\sin\theta\\\cos\theta-\cos\phi} \implies
\norm{\vec{n}}=2\sin\brak{\frac{\phi-\theta}{2}}
\end{align}
The equation of the line is
\begin{align}
\vec{n}^\top\brak{\vec{x}-\vec{A}}&=0\\
\implies\myvec{\sin\phi-\sin\theta&\cos\theta-\cos\phi}\vec{x}&=\sin\brak{\phi-\theta}
\label{eq:chapters/11/10/4/5/1}
\end{align}
Thus, 
\begin{align}
c=\sin\brak{\phi-\theta}
\end{align}
The perpendicular distance from the origin to the line is
\begin{align}
d&=\frac{\abs{c}}{\norm{\vec{n}}}\\
\implies d&=\frac{\sin\brak{\phi-\theta}}{2\sin\brak{\frac{\phi-\theta}{2}}} = \cos\brak{\frac{\phi-\theta}{2}}
\label{eq:chapters/11/10/4/5/2}
\end{align}

\item Find the equation of line which is equidistant from parallel lines $9x+6y-7=0$ and $3x+2y+6=0$.
\\
\solution
		\iffalse
\documentclass[journal,10pt,twocolumn]{article}
\usepackage{graphicx}
\usepackage[margin=0.5in]{geometry}
\usepackage[cmex10]{amsmath}
\usepackage{array}
\usepackage{booktabs}
\usepackage{listings}
\title{\textbf{Line Assignment}}
\author{Bhavani Kanike}
\date{October 2022}

\providecommand{\norm}[1]{\left\lVert#1\right\rVert}
\providecommand{\abs}[1]{\left\vert#1\right\vert}
\let\vec\mathbf
\newcommand{\myvec}[1]{\ensuremath{\begin{pmatrix}#1\end{pmatrix}}}
\newcommand{\mydet}[1]{\ensuremath{\begin{vmatrix}#1\end{vmatrix}}}
\providecommand{\brak}[1]{\ensuremath{\left(#1\right)}}

\begin{document}

\maketitle
\paragraph{\textit{Problem Statement} 
\fi
ABCD is a quadrilateral in which $\vec{P}, \vec{Q}, \vec{R}$ and $\vec{S}$ are mid-points of the sides AB, BC, CD and DA (see Fig \ref{fig:9/8/2/1}). AC is a diagonal. 
		
Show that 
\begin{enumerate}
	\item $SR \parallel AC$ and $SR =\frac{1}{2} AC$
\item $PQ = SR$
\item $PQRS$ is a parallelogram.
\end{enumerate}
 	\begin{figure}
		\centering
 \includegraphics[width=\columnwidth]{chapters/9/8/2/1/figs/line1.pdf}
		\caption{}
		\label{fig:9/8/2/1}
  	\end{figure}
	\solution 
	Using 
	  \eqref{eq:section_formula},
	\begin{align}
		\label{eq:9/8/2/1}
		\begin{split}
		\vec{P} &= \frac{\vec{A}+\vec{B}}{2}\\
 \vec{Q} &= \frac{\vec{C}+\vec{B}}{2}\\
 \vec{R} &= \frac{\vec{C}+\vec{D}}{2}\\
 \vec{S} &= \frac{\vec{D}+\vec{A}}{2}
		\end{split}
	\end{align}
\begin{enumerate}
	\item
	Consequently, 
	\begin{align}
\vec{R}
		-\vec{S} &= \frac{\vec{C}-\vec{A}}{2}
		\\
		\implies SR &\parallel AC
	\end{align}
	Also, 
	\begin{align}
		\norm{\vec{R}
		-\vec{S}} &= \frac{\norm{\vec{C}-\vec{A}}}{2}
		\\
		\implies SR &= \frac{1}{2}AC
	\end{align}
\item 	From 
		\eqref{eq:9/8/2/1},
	\begin{align}
\vec{R}
		-\vec{S} = \vec{Q}-\vec{P}
	\end{align}
	which means that $PQRS$ is a parallelogram and $PQ = SR$.
\end{enumerate}
%
\iffalse
\begin{figure}[h]
\centering
\includegraphics[width=1\columnwidth]
\caption{Figure}
\label{fig:triangle}
\end{figure}

\section*{Solution}

$\boldsymbol Given :$  ABCD is a Quadrilateral P,Q,R and S are the midpoints of line AB,BC,CD,DA.We can obtain the points P,Q,R and S from A,B,C and D and are given by\\\\
\boldmath
\unboldmath
(3) To prove that PQRS is a parallelogram we need to prove  PQ // SR
To prove SR $\parallel$ PQ\\
Direction vector of line SR  $\boldsymbol {(R-S) =  \frac{(C-A)}{2}}$\\\\
Direction vector of line PQ  $\boldsymbol {(Q-P)= \frac{(C-A)}{2}}$\\\\
\begin{equation}
	\boldsymbol {(R-S) = (Q-P) = \frac{(C-A)}{2}}\\
\end{equation}
Since the direction vectors of line SR and PQ are in same direction\\\\
$SR \parallel PQ$\\
Therefore,
$\boldsymbol{ PQRS }$ is a parallelogram\\\\

	
(1)  Directional vector of line SR  = $\boldsymbol {(R-S)}$ = $\frac{\boldsymbol{(C-A)}}{2} $\\
Directional vector of line AC  = $\boldsymbol {(C-A)}$\\

It is observed that the constant k is $\frac{1}{2}$

Therefore
\begin{equation}
	SR \parallel AC
\end{equation} 

and from equation 1 
\begin{equation}
	\boldsymbol {SR = \frac{1}{2}AC}    
\end{equation}\\


(2)   To prove PQ = SR\\ 
		From euqation 1\\\\
\begin{equation}
		\boldsymbol{ (Q-P) = (R-S) = \frac{(C-A)}{2}}
\end{equation}
	 



\section{Execution}
The below python code realizes the construction:
\begin{lstlisting}
https://github.com/bhavani360/FWC_assignments
\end{lstlisting}
	
\section*{Construction}
The dimensions of the Quadrilateral ABCD are taken as below\\
{
\setlength\extrarowheight{2pt}
\centering
	\begin{tabular}{|c|c|}
	\hline
	\textbf{symbol}&\textbf{value}\\
	\hline
	r&8\\
	\hline
	$\theta$&pi/2.5\\
	\hline
	d&7\\
	\hline
	A&(0,0)\\
	\hline
	B&(d,0)\\
	\hline
	D&(rcos$\theta$,rsin$\theta$)\\
	\hline
	C&(D/1.5)+B\\
	\hline
\end{tabular}
}
\end{document}
\fi

	\item Prove that the products of the lengths of the perpendiculars drawn from the points $\myvec{\sqrt{a^2-b^2}\\0}$ and $\myvec{-\sqrt{a^2-b^2} \\0} $ to the line $\frac{x}{a} \cos{\theta} + \frac{y}{b}\sin{\theta} =1 $ is $ b^2 $.
\\
    \solution 
		\iffalse
\documentclass[10pt]{article}
       \usepackage[latin1]{inputenc}
       \usepackage{fullpage}
       \usepackage{color}
       \usepackage{array}
       \usepackage{longtable}
       \usepackage{calc}
       \usepackage{multirow}
       \usepackage{hhline}
       \usepackage{ifthen}
\usepackage{graphicx}
\def\inputGnumericTable{}
\usepackage[none]{hyphenat}
\usepackage{graphicx}
\usepackage{listings}
\usepackage[english]{babel}
\usepackage{graphicx}
\usepackage{caption} 
\usepackage{booktabs}
\usepackage{gensymb}
\usepackage{array}
\usepackage{amssymb} % for \because
\usepackage{amsmath}   % for having text in math mode
\usepackage{extarrows} % for Row operations arrows
\usepackage{listings}
\lstset{
  frame=single,
  breaklines=true
}
\usepackage{hyperref}
%Following 2 lines were added to remove the blank page at the beginning
\usepackage{atbegshi}% http://ctan.org/pkg/atbegshi
\AtBeginDocument{\AtBeginShipoutNext{\AtBeginShipoutDiscard}}
%New macro definitions
\newcommand{\mydet}[1]{\ensuremath{\begin{vmatrix}#1\end{vmatrix}}}
\providecommand{\brak}[1]{\ensuremath{\left(#1\right)}}
\providecommand{\norm}[1]{\left\lVert#1\right\rVert}
\newcommand{\solution}{\noindent \textbf{Solution: }}
\newcommand{\myvec}[1]{\ensuremath{\begin{pmatrix}#1\end{pmatrix}}}
\providecommand{\abs}[1]{\left\vert#1\right\vert}
\let\vec\mathbf
\begin{document}

\begin{center}
\title{\textbf{STRAIGHT LINES}}
\date{\vspace{-5ex}} %Not to print date automatically
\maketitle
\end{center}

\section{11$^{th}$ Maths - Chapter 10}
This is Problem 5 from Exercise-10.4
\begin{enumerate}

\solution
\fi
Let
\begin{align}
	\vec{A}=\myvec{\cos\theta\\\sin\theta},\vec{B}&=\myvec{\cos\phi\\\sin\phi}\\
\implies	\vec{m}=\vec{B}-\vec{A}&=\myvec{\cos\phi-\cos\theta\\\sin\phi-\sin\theta}
\end{align}
The normal vector is then given by,
\begin{align}
\vec{n}=\myvec{\sin\phi-\sin\theta\\\cos\theta-\cos\phi} \implies
\norm{\vec{n}}=2\sin\brak{\frac{\phi-\theta}{2}}
\end{align}
The equation of the line is
\begin{align}
\vec{n}^\top\brak{\vec{x}-\vec{A}}&=0\\
\implies\myvec{\sin\phi-\sin\theta&\cos\theta-\cos\phi}\vec{x}&=\sin\brak{\phi-\theta}
\label{eq:chapters/11/10/4/5/1}
\end{align}
Thus, 
\begin{align}
c=\sin\brak{\phi-\theta}
\end{align}
The perpendicular distance from the origin to the line is
\begin{align}
d&=\frac{\abs{c}}{\norm{\vec{n}}}\\
\implies d&=\frac{\sin\brak{\phi-\theta}}{2\sin\brak{\frac{\phi-\theta}{2}}} = \cos\brak{\frac{\phi-\theta}{2}}
\label{eq:chapters/11/10/4/5/2}
\end{align}

\item Find the equation of line  drawn perpendicular to the line $\frac{x}{4}+\frac{y}{6}=1$ through the point where it meets the y-axis \\
\solution
		\iffalse
\documentclass[journal,10pt,twocolumn]{article}
\usepackage{graphicx}
\usepackage[margin=0.5in]{geometry}
\usepackage[cmex10]{amsmath}
\usepackage{array}
\usepackage{booktabs}
\usepackage{listings}
\title{\textbf{Line Assignment}}
\author{Bhavani Kanike}
\date{October 2022}

\providecommand{\norm}[1]{\left\lVert#1\right\rVert}
\providecommand{\abs}[1]{\left\vert#1\right\vert}
\let\vec\mathbf
\newcommand{\myvec}[1]{\ensuremath{\begin{pmatrix}#1\end{pmatrix}}}
\newcommand{\mydet}[1]{\ensuremath{\begin{vmatrix}#1\end{vmatrix}}}
\providecommand{\brak}[1]{\ensuremath{\left(#1\right)}}

\begin{document}

\maketitle
\paragraph{\textit{Problem Statement} 
\fi
ABCD is a quadrilateral in which $\vec{P}, \vec{Q}, \vec{R}$ and $\vec{S}$ are mid-points of the sides AB, BC, CD and DA (see Fig \ref{fig:9/8/2/1}). AC is a diagonal. 
		
Show that 
\begin{enumerate}
	\item $SR \parallel AC$ and $SR =\frac{1}{2} AC$
\item $PQ = SR$
\item $PQRS$ is a parallelogram.
\end{enumerate}
 	\begin{figure}
		\centering
 \includegraphics[width=\columnwidth]{chapters/9/8/2/1/figs/line1.pdf}
		\caption{}
		\label{fig:9/8/2/1}
  	\end{figure}
	\solution 
	Using 
	  \eqref{eq:section_formula},
	\begin{align}
		\label{eq:9/8/2/1}
		\begin{split}
		\vec{P} &= \frac{\vec{A}+\vec{B}}{2}\\
 \vec{Q} &= \frac{\vec{C}+\vec{B}}{2}\\
 \vec{R} &= \frac{\vec{C}+\vec{D}}{2}\\
 \vec{S} &= \frac{\vec{D}+\vec{A}}{2}
		\end{split}
	\end{align}
\begin{enumerate}
	\item
	Consequently, 
	\begin{align}
\vec{R}
		-\vec{S} &= \frac{\vec{C}-\vec{A}}{2}
		\\
		\implies SR &\parallel AC
	\end{align}
	Also, 
	\begin{align}
		\norm{\vec{R}
		-\vec{S}} &= \frac{\norm{\vec{C}-\vec{A}}}{2}
		\\
		\implies SR &= \frac{1}{2}AC
	\end{align}
\item 	From 
		\eqref{eq:9/8/2/1},
	\begin{align}
\vec{R}
		-\vec{S} = \vec{Q}-\vec{P}
	\end{align}
	which means that $PQRS$ is a parallelogram and $PQ = SR$.
\end{enumerate}
%
\iffalse
\begin{figure}[h]
\centering
\includegraphics[width=1\columnwidth]
\caption{Figure}
\label{fig:triangle}
\end{figure}

\section*{Solution}

$\boldsymbol Given :$  ABCD is a Quadrilateral P,Q,R and S are the midpoints of line AB,BC,CD,DA.We can obtain the points P,Q,R and S from A,B,C and D and are given by\\\\
\boldmath
\unboldmath
(3) To prove that PQRS is a parallelogram we need to prove  PQ // SR
To prove SR $\parallel$ PQ\\
Direction vector of line SR  $\boldsymbol {(R-S) =  \frac{(C-A)}{2}}$\\\\
Direction vector of line PQ  $\boldsymbol {(Q-P)= \frac{(C-A)}{2}}$\\\\
\begin{equation}
	\boldsymbol {(R-S) = (Q-P) = \frac{(C-A)}{2}}\\
\end{equation}
Since the direction vectors of line SR and PQ are in same direction\\\\
$SR \parallel PQ$\\
Therefore,
$\boldsymbol{ PQRS }$ is a parallelogram\\\\

	
(1)  Directional vector of line SR  = $\boldsymbol {(R-S)}$ = $\frac{\boldsymbol{(C-A)}}{2} $\\
Directional vector of line AC  = $\boldsymbol {(C-A)}$\\

It is observed that the constant k is $\frac{1}{2}$

Therefore
\begin{equation}
	SR \parallel AC
\end{equation} 

and from equation 1 
\begin{equation}
	\boldsymbol {SR = \frac{1}{2}AC}    
\end{equation}\\


(2)   To prove PQ = SR\\ 
		From euqation 1\\\\
\begin{equation}
		\boldsymbol{ (Q-P) = (R-S) = \frac{(C-A)}{2}}
\end{equation}
	 



\section{Execution}
The below python code realizes the construction:
\begin{lstlisting}
https://github.com/bhavani360/FWC_assignments
\end{lstlisting}
	
\section*{Construction}
The dimensions of the Quadrilateral ABCD are taken as below\\
{
\setlength\extrarowheight{2pt}
\centering
	\begin{tabular}{|c|c|}
	\hline
	\textbf{symbol}&\textbf{value}\\
	\hline
	r&8\\
	\hline
	$\theta$&pi/2.5\\
	\hline
	d&7\\
	\hline
	A&(0,0)\\
	\hline
	B&(d,0)\\
	\hline
	D&(rcos$\theta$,rsin$\theta$)\\
	\hline
	C&(D/1.5)+B\\
	\hline
\end{tabular}
}
\end{document}
\fi

 \item  In each of the following cases, determine the direction cosines of the normal to
the plane and the distance from the origin.
\begin{enumerate}
	\item $z=2$ 
	\item $x + y + z = 1$
	\item $2x + 3y – z = 5$
	\item $5y + 8 = 0$
\end{enumerate}
    \solution
		\iffalse
\documentclass[12pt]{article}
\usepackage{graphicx}
\usepackage{amsmath}
\usepackage{mathtools}
\usepackage{gensymb}
\usepackage[utf8]{inputenc}
\usepackage{float}
\newcommand{\mydet}[1]{\ensuremath{\begin{vmatrix}#1\end{vmatrix}}}
\providecommand{\brak}[1]{\ensuremath{\left(#1\right)}}
\providecommand{\norm}[1]{\left\lVert#1\right\rVert}
\newcommand{\solution}{\noindent \textbf{Solution: }}
\newcommand{\myvec}[1]{\ensuremath{\begin{pmatrix}#1\end{pmatrix}}}
\let\vec\mathbf

\begin{document}
\begin{center}
\textbf\large{CLASS-12 \\ CHAPTER-11 \\ THREE DIMENSIONAL GEOMETRY}
\end{center}
\section*{Excercise 11.3}

\solution
\fi
\begin{enumerate}
\item From the given equation
	\begin{align}
		\vec{n}=\myvec{0\\0\\1},c=2
	\end{align}
	The distance from the origin is given by:
		\begin{align}
			d=\frac{|c|}{\norm{\vec{n}}}=\frac{2}{1}=2
		\end{align}

\item From the given equation
         \begin{align}
		\vec{n}=\myvec{1\\1\\1},c=1
			\end{align}
	 
	The distance from the origin is given by
		\begin{align}
			d=\frac{|c|}{\norm{\vec{n}}}=\frac{1}{\sqrt{3}}
		\end{align}

\item From the given equation
         \begin{align}
		\vec{n}=\myvec{2\\3\\-1},c=5
			\end{align}
	
	The distance from the origin is given by
		\begin{align}
			d=\frac{|c|}{\norm{\vec{n}}}=\frac{5}{\sqrt{14}}
		\end{align}
		
\item From the given equation
         \begin{align}
		\vec{n}=\myvec{0\\-5\\0},c=8
			\end{align}
	
	The distance from the origin is given by
		\begin{align}
			d=\frac{|c|}{\norm{\vec{n}}}=\frac{8}{5}
		\end{align}

\end{enumerate}
 


\item
Find the angle between the lines whose direction ratios are $a,b,c$ and $b-c,c-a,a-b$.

\textbf{Solution :}
    \begin{align}
    \vec{m _1} &= \myvec{a\\b\\c}\\
    \vec{m_2} &= \myvec{b-c\\c-a\\a-b}\\
    \cos{\theta}&= \frac{\vec{m_1}^{\top}\vec{m_2}}{\vec{\norm{m_1}\norm{m_2}}
   } \\
   &=\frac{\myvec{a&b&c}\myvec{b-c\\c-a\\a-b}}{\sqrt{a^2+b^2+c^2}\sqrt{\brak{b-c}^2+\brak{c-a}^2+\brak{a-b}^2}}\\
   &=0\\
   or,\theta&=\frac{\pi}{2}
    \end{align}

\end{enumerate}

\subsection{Plane}
\begin{enumerate}[label=\thesection.\arabic*,ref=\thesection.\theenumi]
\item Find the vector equation of a plane which is at a distance of 7 units from the origin and normal to the vector $3\hat{i}+5\hat{j}-6\hat{k}$.
	\\
    \solution
		\iffalse
\documentclass[12pt]{article}
\usepackage{graphicx}
\usepackage[none]{hyphenat}
\usepackage{graphicx}
\usepackage{listings}
\usepackage[english]{babel}
\usepackage{graphicx}
\usepackage{caption} 
\usepackage{booktabs}
\usepackage{array}
\usepackage{amssymb} % for \because
\usepackage{amsmath}   % for having text in math mode
\usepackage{extarrows} % for Row operations arrows
\usepackage{listings}
\lstset{
  frame=single,
  breaklines=true
}
\usepackage{hyperref}
  
%Following 2 lines were added to remove the blank page at the beginning
\usepackage{atbegshi}% http://ctan.org/pkg/atbegshi
\AtBeginDocument{\AtBeginShipoutNext{\AtBeginShipoutDiscard}}


%New macro definitions
\newcommand{\mydet}[1]{\ensuremath{\begin{vmatrix}#1\end{vmatrix}}}
\providecommand{\brak}[1]{\ensuremath{\left(#1\right)}}
\providecommand{\norm}[1]{\left\lVert#1\right\rVert}
\newcommand{\solution}{\noindent \textbf{Solution: }}
\newcommand{\myvec}[1]{\ensuremath{\begin{pmatrix}#1\end{pmatrix}}}
\providecommand{\abs}[1]{\left\vert#1\right\vert}
\let\vec\mathbf

\begin{document}

\begin{center}
\title{\textbf{Equation  of Unit Vector}}
\date{\vspace{-5ex}} %Not to print date automatically
\maketitle
\end{center}
\setcounter{page}{1}
\section{12$^{th}$ Maths - Chapter 11}
\textbf{This is Problem-2 from Exercise 3.2}
\begin{enumerate}
\section{Solution}
\fi
From the given information, 
\begin{align} 
\vec{n}=\myvec{3\\5\\-6},\,
	d=\frac{\abs{c}}{\norm{\vec{n}}} = 7
\end{align}	  
yielding
\begin{align}
c =\pm7\sqrt{70}
\end{align}	  

	\item Find the equations of the planes that pass through the points
\begin{enumerate}
\item $\vec{A}= \myvec{1\\1\\– 1}, \vec{B}=\myvec{6\\4\\– 5},\vec{C}= \myvec{– 4\\– 2\\3}$
\item $\vec{A}= \myvec{1\\1\\0}, \vec{B}= \myvec{1\\2\\1}, \vec{C}= \myvec{– 2\\2\\-1}$
\end{enumerate}
    \solution
		\renewcommand{\theequation}{\theenumi}
\begin{enumerate}[label=\thesubsection.\arabic*.,ref=\thesubsection.\theenumi]
\numberwithin{equation}{enumi}

\item  Find the equation of a plane passing through the points $\vec{a}=\myvec{2\\5\\-3}, \vec{b}=\myvec{-2\\-3\\5}$ and $\vec{c}=\myvec{5\\3\\-3}$ 
\label{eq:plane}
\\
\solution
The equation of  plane is also  given by \eqref{eq:line_norm_eq_unit} in 3D.  Following the approach in the previous example results in the matrix equation, 
\begin{align}
\myvec{2&5&-3 \\ -2&-3&5 \\ 5&3&-3} \vec{n} &= \myvec{1\\1\\1}
\end{align}
Row reducing the augmented matrix, 
\begin{align}
\myvec{2&5&-3 & 1\\ -2&-3&5 & 1\\ 5&3&-3 & 1} 
\\
\xleftrightarrow[R_3\leftarrow 2R_3-5R_1]{R_2\leftarrow \frac{R_2+R_1}{2} }\myvec{2&5&-3 & 1\\ 0&1&1 & 1\\ 0&-19&9 & -3} 
\\
\xleftrightarrow[R_3\leftarrow \frac{R_3+19R_2}{4}]{R_1\leftarrow R_1-5R_2 }\myvec{2&0&-8 & -4\\ 0&1&1 & 1\\ 0&0&7 & 4}\\ 
\xleftrightarrow[R_3\leftarrow 7R2-R_3]{R_1\leftarrow \frac{7R_1+8R_3}{2} }\myvec{7&0&0 & 2\\ 0&7&0 & 3\\ 0&0&7 & 4} 
\\
\implies \vec{n} = \frac{1}{7}\myvec{2\\3\\4}
\end{align}
Thus, the equation of the plane passing through the given points is
%
\begin{align}
\myvec{2 & 3 & 4}\vec{x} = 7
\end{align} 
%\cite{twelve_two}.
\item  Find the angle between the two planes 
%\cite{twelve_two}
\begin{align}
\myvec{ 2 & 1 & -2}\vec{x} &=5
\\
\myvec{ 3 & -6 & -2}\vec{x} &=7
\end{align}
\\
\solution
The angle between two planes is the same as the angle between their normal vectors.  For 
\begin{align}
\vec{n}_1 = \myvec{ 2 \\ 1 \\ -2}
\vec{n}_2 = \myvec{ 3 \\ -6 \\ -2}
\end{align}
using \eqref{eq:vec_angle}, 
\begin{align}
\cos \theta = \frac{6-6+4}{\sqrt{9}\sqrt{49}} = \frac{4}{21}
\end{align}
\end{enumerate}



	\item Find the equation of the plane with an intercept 3 on the Y-axis and parallel to ZOX-Plane.\\
    \solution
		\renewcommand{\theequation}{\theenumi}
\begin{enumerate}[label=\thesubsection.\arabic*.,ref=\thesubsection.\theenumi]
\numberwithin{equation}{enumi}

\item  Find the equation of a plane passing through the points $\vec{a}=\myvec{2\\5\\-3}, \vec{b}=\myvec{-2\\-3\\5}$ and $\vec{c}=\myvec{5\\3\\-3}$ 
\label{eq:plane}
\\
\solution
The equation of  plane is also  given by \eqref{eq:line_norm_eq_unit} in 3D.  Following the approach in the previous example results in the matrix equation, 
\begin{align}
\myvec{2&5&-3 \\ -2&-3&5 \\ 5&3&-3} \vec{n} &= \myvec{1\\1\\1}
\end{align}
Row reducing the augmented matrix, 
\begin{align}
\myvec{2&5&-3 & 1\\ -2&-3&5 & 1\\ 5&3&-3 & 1} 
\\
\xleftrightarrow[R_3\leftarrow 2R_3-5R_1]{R_2\leftarrow \frac{R_2+R_1}{2} }\myvec{2&5&-3 & 1\\ 0&1&1 & 1\\ 0&-19&9 & -3} 
\\
\xleftrightarrow[R_3\leftarrow \frac{R_3+19R_2}{4}]{R_1\leftarrow R_1-5R_2 }\myvec{2&0&-8 & -4\\ 0&1&1 & 1\\ 0&0&7 & 4}\\ 
\xleftrightarrow[R_3\leftarrow 7R2-R_3]{R_1\leftarrow \frac{7R_1+8R_3}{2} }\myvec{7&0&0 & 2\\ 0&7&0 & 3\\ 0&0&7 & 4} 
\\
\implies \vec{n} = \frac{1}{7}\myvec{2\\3\\4}
\end{align}
Thus, the equation of the plane passing through the given points is
%
\begin{align}
\myvec{2 & 3 & 4}\vec{x} = 7
\end{align} 
%\cite{twelve_two}.
\item  Find the angle between the two planes 
%\cite{twelve_two}
\begin{align}
\myvec{ 2 & 1 & -2}\vec{x} &=5
\\
\myvec{ 3 & -6 & -2}\vec{x} &=7
\end{align}
\\
\solution
The angle between two planes is the same as the angle between their normal vectors.  For 
\begin{align}
\vec{n}_1 = \myvec{ 2 \\ 1 \\ -2}
\vec{n}_2 = \myvec{ 3 \\ -6 \\ -2}
\end{align}
using \eqref{eq:vec_angle}, 
\begin{align}
\cos \theta = \frac{6-6+4}{\sqrt{9}\sqrt{49}} = \frac{4}{21}
\end{align}
\end{enumerate}



	\item  Find the equation of the plane through the intersection of the planes $3{x} – {y} + 2{z} – 4 = 0 \text{ and } {x} + {y} + {z} – 2 = 0$ and the point $\myvec{2\\2\\1}$.
    \solution
		\renewcommand{\theequation}{\theenumi}
\begin{enumerate}[label=\thesubsection.\arabic*.,ref=\thesubsection.\theenumi]
\numberwithin{equation}{enumi}

\item  Find the equation of a plane passing through the points $\vec{a}=\myvec{2\\5\\-3}, \vec{b}=\myvec{-2\\-3\\5}$ and $\vec{c}=\myvec{5\\3\\-3}$ 
\label{eq:plane}
\\
\solution
The equation of  plane is also  given by \eqref{eq:line_norm_eq_unit} in 3D.  Following the approach in the previous example results in the matrix equation, 
\begin{align}
\myvec{2&5&-3 \\ -2&-3&5 \\ 5&3&-3} \vec{n} &= \myvec{1\\1\\1}
\end{align}
Row reducing the augmented matrix, 
\begin{align}
\myvec{2&5&-3 & 1\\ -2&-3&5 & 1\\ 5&3&-3 & 1} 
\\
\xleftrightarrow[R_3\leftarrow 2R_3-5R_1]{R_2\leftarrow \frac{R_2+R_1}{2} }\myvec{2&5&-3 & 1\\ 0&1&1 & 1\\ 0&-19&9 & -3} 
\\
\xleftrightarrow[R_3\leftarrow \frac{R_3+19R_2}{4}]{R_1\leftarrow R_1-5R_2 }\myvec{2&0&-8 & -4\\ 0&1&1 & 1\\ 0&0&7 & 4}\\ 
\xleftrightarrow[R_3\leftarrow 7R2-R_3]{R_1\leftarrow \frac{7R_1+8R_3}{2} }\myvec{7&0&0 & 2\\ 0&7&0 & 3\\ 0&0&7 & 4} 
\\
\implies \vec{n} = \frac{1}{7}\myvec{2\\3\\4}
\end{align}
Thus, the equation of the plane passing through the given points is
%
\begin{align}
\myvec{2 & 3 & 4}\vec{x} = 7
\end{align} 
%\cite{twelve_two}.
\item  Find the angle between the two planes 
%\cite{twelve_two}
\begin{align}
\myvec{ 2 & 1 & -2}\vec{x} &=5
\\
\myvec{ 3 & -6 & -2}\vec{x} &=7
\end{align}
\\
\solution
The angle between two planes is the same as the angle between their normal vectors.  For 
\begin{align}
\vec{n}_1 = \myvec{ 2 \\ 1 \\ -2}
\vec{n}_2 = \myvec{ 3 \\ -6 \\ -2}
\end{align}
using \eqref{eq:vec_angle}, 
\begin{align}
\cos \theta = \frac{6-6+4}{\sqrt{9}\sqrt{49}} = \frac{4}{21}
\end{align}
\end{enumerate}



\item 
	Find the Cartesian equation of the following planes:


\begin{enumerate}
    \item $\vec{r}.\brak{\hat{i}+\hat{j}-\hat{k}}=2$
    \item $\vec{r}.\brak{2\hat{i}+3\hat{j}-4\hat{k}}=1$
    \item $\vec{r}.[\brak{s-2t}\hat{i}+\brak{3-t}\hat{j}+\brak{2s+t}\hat{k}]=15$
\end{enumerate}


\textbf{Solution :}
\begin{enumerate}
    \item $\myvec{1&1&-1} \vec{x} = 2$
    \item $\myvec{2&3&-4} \vec{x} = 1$
    \item $\myvec{2&-5&-1} \vec{x} = 15$

\end{enumerate}




\end{enumerate}


\subsection{Miscellaneous }
\begin{enumerate}[label=\thesection.\arabic*,ref=\thesection.\theenumi]
\numberwithin{equation}{enumi}
\numberwithin{figure}{enumi}
\numberwithin{table}{enumi}


\item Find the values of $k$ for which the line 
\begin{align}
(k-3)x-(4-k^2)y+k^2-7k+6=0 \label{eq:chapters/11/10/4/1/1}
\end{align}
is
\begin{enumerate}
\item Parallel to the $x$-axis
\item Parallel to the $y$-axis
\item Passing through the origin
\end{enumerate}
    \solution 
		\iffalse
\documentclass[journal,10pt,twocolumn]{article}
\usepackage{graphicx}
\usepackage[margin=0.5in]{geometry}
\usepackage[cmex10]{amsmath}
\usepackage{array}
\usepackage{booktabs}
\usepackage{listings}
\title{\textbf{Line Assignment}}
\author{Bhavani Kanike}
\date{October 2022}

\providecommand{\norm}[1]{\left\lVert#1\right\rVert}
\providecommand{\abs}[1]{\left\vert#1\right\vert}
\let\vec\mathbf
\newcommand{\myvec}[1]{\ensuremath{\begin{pmatrix}#1\end{pmatrix}}}
\newcommand{\mydet}[1]{\ensuremath{\begin{vmatrix}#1\end{vmatrix}}}
\providecommand{\brak}[1]{\ensuremath{\left(#1\right)}}

\begin{document}

\maketitle
\paragraph{\textit{Problem Statement} 
\fi
ABCD is a quadrilateral in which $\vec{P}, \vec{Q}, \vec{R}$ and $\vec{S}$ are mid-points of the sides AB, BC, CD and DA (see Fig \ref{fig:9/8/2/1}). AC is a diagonal. 
		
Show that 
\begin{enumerate}
	\item $SR \parallel AC$ and $SR =\frac{1}{2} AC$
\item $PQ = SR$
\item $PQRS$ is a parallelogram.
\end{enumerate}
 	\begin{figure}
		\centering
 \includegraphics[width=\columnwidth]{chapters/9/8/2/1/figs/line1.pdf}
		\caption{}
		\label{fig:9/8/2/1}
  	\end{figure}
	\solution 
	Using 
	  \eqref{eq:section_formula},
	\begin{align}
		\label{eq:9/8/2/1}
		\begin{split}
		\vec{P} &= \frac{\vec{A}+\vec{B}}{2}\\
 \vec{Q} &= \frac{\vec{C}+\vec{B}}{2}\\
 \vec{R} &= \frac{\vec{C}+\vec{D}}{2}\\
 \vec{S} &= \frac{\vec{D}+\vec{A}}{2}
		\end{split}
	\end{align}
\begin{enumerate}
	\item
	Consequently, 
	\begin{align}
\vec{R}
		-\vec{S} &= \frac{\vec{C}-\vec{A}}{2}
		\\
		\implies SR &\parallel AC
	\end{align}
	Also, 
	\begin{align}
		\norm{\vec{R}
		-\vec{S}} &= \frac{\norm{\vec{C}-\vec{A}}}{2}
		\\
		\implies SR &= \frac{1}{2}AC
	\end{align}
\item 	From 
		\eqref{eq:9/8/2/1},
	\begin{align}
\vec{R}
		-\vec{S} = \vec{Q}-\vec{P}
	\end{align}
	which means that $PQRS$ is a parallelogram and $PQ = SR$.
\end{enumerate}
%
\iffalse
\begin{figure}[h]
\centering
\includegraphics[width=1\columnwidth]
\caption{Figure}
\label{fig:triangle}
\end{figure}

\section*{Solution}

$\boldsymbol Given :$  ABCD is a Quadrilateral P,Q,R and S are the midpoints of line AB,BC,CD,DA.We can obtain the points P,Q,R and S from A,B,C and D and are given by\\\\
\boldmath
\unboldmath
(3) To prove that PQRS is a parallelogram we need to prove  PQ // SR
To prove SR $\parallel$ PQ\\
Direction vector of line SR  $\boldsymbol {(R-S) =  \frac{(C-A)}{2}}$\\\\
Direction vector of line PQ  $\boldsymbol {(Q-P)= \frac{(C-A)}{2}}$\\\\
\begin{equation}
	\boldsymbol {(R-S) = (Q-P) = \frac{(C-A)}{2}}\\
\end{equation}
Since the direction vectors of line SR and PQ are in same direction\\\\
$SR \parallel PQ$\\
Therefore,
$\boldsymbol{ PQRS }$ is a parallelogram\\\\

	
(1)  Directional vector of line SR  = $\boldsymbol {(R-S)}$ = $\frac{\boldsymbol{(C-A)}}{2} $\\
Directional vector of line AC  = $\boldsymbol {(C-A)}$\\

It is observed that the constant k is $\frac{1}{2}$

Therefore
\begin{equation}
	SR \parallel AC
\end{equation} 

and from equation 1 
\begin{equation}
	\boldsymbol {SR = \frac{1}{2}AC}    
\end{equation}\\


(2)   To prove PQ = SR\\ 
		From euqation 1\\\\
\begin{equation}
		\boldsymbol{ (Q-P) = (R-S) = \frac{(C-A)}{2}}
\end{equation}
	 



\section{Execution}
The below python code realizes the construction:
\begin{lstlisting}
https://github.com/bhavani360/FWC_assignments
\end{lstlisting}
	
\section*{Construction}
The dimensions of the Quadrilateral ABCD are taken as below\\
{
\setlength\extrarowheight{2pt}
\centering
	\begin{tabular}{|c|c|}
	\hline
	\textbf{symbol}&\textbf{value}\\
	\hline
	r&8\\
	\hline
	$\theta$&pi/2.5\\
	\hline
	d&7\\
	\hline
	A&(0,0)\\
	\hline
	B&(d,0)\\
	\hline
	D&(rcos$\theta$,rsin$\theta$)\\
	\hline
	C&(D/1.5)+B\\
	\hline
\end{tabular}
}
\end{document}
\fi

	\item Find the values of $\theta \text{ and } p$, if the equation $x\cos\theta+y\sin\theta=p$ is the normal form
of the line $\sqrt{3}x+y+2=0$.
\\
\solution
		\iffalse
\documentclass[12pt]{article}
\usepackage{graphicx}
\usepackage{amsmath}
\usepackage{mathtools}
\usepackage{gensymb}
\usepackage[utf8]{inputenc}
\usepackage{float}
\newcommand{\mydet}[1]{\ensuremath{\begin{vmatrix}#1\end{vmatrix}}}
\providecommand{\brak}[1]{\ensuremath{\left(#1\right)}}
\providecommand{\norm}[1]{\left\lVert#1\right\rVert}
\newcommand{\solution}{\noindent \textbf{Solution: }}
\newcommand{\myvec}[1]{\ensuremath{\begin{pmatrix}#1\end{pmatrix}}}
\let\vec\mathbf

\begin{document}
\begin{center}
\textbf\large{CLASS-11 \\ CHAPTER-10 \\ STRAIGHT LINES}
\end{center}
\section*{Excercise 10.4}

\\
\fi
The parameters of the given line are
		\begin{align}
	\vec{n}=\myvec{\sqrt{3}\\1},
			c=-2
		\end{align}
		From the above, 
		\begin{align}
			\tan\theta&=-\sqrt{3}\\
		\implies \theta&=-60\degree
		\end{align}
		and 
		\begin{align}
			p=\frac{|c|}{\norm{\vec{n}}}=\frac{2}{2}=1
		\end{align}
\begin{figure}[H]
	\begin{center} 
	    \includegraphics[width=\columnwidth]{chapters/11/10/4/2/figs/line.png}
	\end{center}
\caption{}
\label{fig:chapters/11/10/4/2/Fig1}
\end{figure}


	\item Find the  equations of the lines, which cutoff intercepts on the axes  whose sum and product are 1 and -6 respectively.
\\
\solution
		\iffalse
\documentclass[12pt]{article}
\usepackage{graphicx}
%\documentclass[journal,12pt,twocolumn]{IEEEtran}
\usepackage[none]{hyphenat}
\usepackage{graphicx}
\usepackage{listings}
\usepackage[english]{babel}
\usepackage{graphicx}
\usepackage{caption}
\usepackage[parfill]{parskip}
\usepackage{hyperref}
\usepackage{booktabs}
%\usepackage{setspace}\doublespacing\pagestyle{plain}
\def\inputGnumericTable{}
\usepackage{color}                                            %%
    \usepackage{array}                                            %%
    \usepackage{longtable}                                        %%
    \usepackage{calc}                                             %%
    \usepackage{multirow}                                         %%
    \usepackage{hhline}                                           %%
    \usepackage{ifthen}
\usepackage{array}
\usepackage{amsmath}   % for having text in math mode
\usepackage{parallel,enumitem}
\usepackage{listings}
\lstset{
language=tex,
frame=single,
breaklines=true
}
%Following 2 lines were added to remove the blank page at the beginning
\usepackage{atbegshi}% http://ctan.org/pkg/atbegshi
\AtBeginDocument{\AtBeginShipoutNext{\AtBeginShipoutDiscard}}
%
%New macro definitions
\newcommand{\mydet}[1]{\ensuremath{\begin{vmatrix}#1\end{vmatrix}}}
\providecommand{\brak}[1]{\ensuremath{\left(#1\right)}}
\providecommand{\abs}[1]{\left\vert#1\right\vert}
\providecommand{\norm}[1]{\left\lVert#1\right\rVert}
\newcommand{\solution}{\noindent \textbf{Solution: }}
\newcommand{\myvec}[1]{\ensuremath{\begin{pmatrix}#1\end{pmatrix}}}
\let\vec\mathbf
\begin{document}
\begin{center}
\title{\textbf{ Intercepts Lines}}
\date{\vspace{-5ex}} %Not to print date automatically
\maketitle
\end{center}
\setcounter{page}{eq:11/10/4/31}
\section*{11$^{th}$ Maths - Chapter 10}
This is Problem-3 from Exercise 10.4
\section{Solution}
\fi
Let the $x$ intercept be $a$ and  the $y$ intercept be $b$ ,Then
\begin{align}
a+b&=1\label{eq:11/10/4/31}\\
ab&=-6 \label{eq:11/10/4/32}
\\
\implies  a = 3, b = -2
\end{align}
Thus, the possible 
intercepts are
\begin{align}
\myvec{3\\0}, \myvec{0\\-2},
\myvec{-2\\0}, \myvec{0\\3}
\end{align}
yielding
\begin{align}		
\vec{m}=\myvec{3\\2} \text{or,} \myvec{-2\\3}
\end{align}
\begin{enumerate}
\item For 
\begin{align}
\vec{n}
=\myvec{-2 \\3} 
\end{align}
the equation of the line is 
\begin{align}
	\vec{n}^\top\brak{\vec{x}-\vec{A}} &= 0 \\
	\myvec { -2 & 3 } \vec{x}  &= 6  
\end{align}
\item  For 
\begin{align}
\vec{n}
=\myvec{-3 \\-2} 
\end{align}
the equation of the line is 
\begin{align}
    \vec{n}^\top\brak{\vec{x}-\vec{B}} &= 0 \\  
	\myvec { -3 & -2 }  \vec{x}  &= 6        
\end{align}
\end{enumerate}
See Fig. 
\ref{fig:11/10/4/3line segment}.
\begin{figure}[h!]
\centering
\includegraphics[width=\columnwidth]{chapters/11/10/4/3/figs/inter.png}
\caption{}
\label{fig:11/10/4/3line segment}
\end{figure}

	\item  Find the equation of the line parallel to y-axis and drawn through the point of
intersection of the lines x – 7y + 5 = 0 and 3x + y = 0.
\\
\solution
		\iffalse
\documentclass[journal,10pt,twocolumn]{article}
\usepackage{graphicx}
\usepackage[margin=0.5in]{geometry}
\usepackage[cmex10]{amsmath}
\usepackage{array}
\usepackage{booktabs}
\usepackage{listings}
\title{\textbf{Line Assignment}}
\author{Bhavani Kanike}
\date{October 2022}

\providecommand{\norm}[1]{\left\lVert#1\right\rVert}
\providecommand{\abs}[1]{\left\vert#1\right\vert}
\let\vec\mathbf
\newcommand{\myvec}[1]{\ensuremath{\begin{pmatrix}#1\end{pmatrix}}}
\newcommand{\mydet}[1]{\ensuremath{\begin{vmatrix}#1\end{vmatrix}}}
\providecommand{\brak}[1]{\ensuremath{\left(#1\right)}}

\begin{document}

\maketitle
\paragraph{\textit{Problem Statement} 
\fi
ABCD is a quadrilateral in which $\vec{P}, \vec{Q}, \vec{R}$ and $\vec{S}$ are mid-points of the sides AB, BC, CD and DA (see Fig \ref{fig:9/8/2/1}). AC is a diagonal. 
		
Show that 
\begin{enumerate}
	\item $SR \parallel AC$ and $SR =\frac{1}{2} AC$
\item $PQ = SR$
\item $PQRS$ is a parallelogram.
\end{enumerate}
 	\begin{figure}
		\centering
 \includegraphics[width=\columnwidth]{chapters/9/8/2/1/figs/line1.pdf}
		\caption{}
		\label{fig:9/8/2/1}
  	\end{figure}
	\solution 
	Using 
	  \eqref{eq:section_formula},
	\begin{align}
		\label{eq:9/8/2/1}
		\begin{split}
		\vec{P} &= \frac{\vec{A}+\vec{B}}{2}\\
 \vec{Q} &= \frac{\vec{C}+\vec{B}}{2}\\
 \vec{R} &= \frac{\vec{C}+\vec{D}}{2}\\
 \vec{S} &= \frac{\vec{D}+\vec{A}}{2}
		\end{split}
	\end{align}
\begin{enumerate}
	\item
	Consequently, 
	\begin{align}
\vec{R}
		-\vec{S} &= \frac{\vec{C}-\vec{A}}{2}
		\\
		\implies SR &\parallel AC
	\end{align}
	Also, 
	\begin{align}
		\norm{\vec{R}
		-\vec{S}} &= \frac{\norm{\vec{C}-\vec{A}}}{2}
		\\
		\implies SR &= \frac{1}{2}AC
	\end{align}
\item 	From 
		\eqref{eq:9/8/2/1},
	\begin{align}
\vec{R}
		-\vec{S} = \vec{Q}-\vec{P}
	\end{align}
	which means that $PQRS$ is a parallelogram and $PQ = SR$.
\end{enumerate}
%
\iffalse
\begin{figure}[h]
\centering
\includegraphics[width=1\columnwidth]
\caption{Figure}
\label{fig:triangle}
\end{figure}

\section*{Solution}

$\boldsymbol Given :$  ABCD is a Quadrilateral P,Q,R and S are the midpoints of line AB,BC,CD,DA.We can obtain the points P,Q,R and S from A,B,C and D and are given by\\\\
\boldmath
\unboldmath
(3) To prove that PQRS is a parallelogram we need to prove  PQ // SR
To prove SR $\parallel$ PQ\\
Direction vector of line SR  $\boldsymbol {(R-S) =  \frac{(C-A)}{2}}$\\\\
Direction vector of line PQ  $\boldsymbol {(Q-P)= \frac{(C-A)}{2}}$\\\\
\begin{equation}
	\boldsymbol {(R-S) = (Q-P) = \frac{(C-A)}{2}}\\
\end{equation}
Since the direction vectors of line SR and PQ are in same direction\\\\
$SR \parallel PQ$\\
Therefore,
$\boldsymbol{ PQRS }$ is a parallelogram\\\\

	
(1)  Directional vector of line SR  = $\boldsymbol {(R-S)}$ = $\frac{\boldsymbol{(C-A)}}{2} $\\
Directional vector of line AC  = $\boldsymbol {(C-A)}$\\

It is observed that the constant k is $\frac{1}{2}$

Therefore
\begin{equation}
	SR \parallel AC
\end{equation} 

and from equation 1 
\begin{equation}
	\boldsymbol {SR = \frac{1}{2}AC}    
\end{equation}\\


(2)   To prove PQ = SR\\ 
		From euqation 1\\\\
\begin{equation}
		\boldsymbol{ (Q-P) = (R-S) = \frac{(C-A)}{2}}
\end{equation}
	 



\section{Execution}
The below python code realizes the construction:
\begin{lstlisting}
https://github.com/bhavani360/FWC_assignments
\end{lstlisting}
	
\section*{Construction}
The dimensions of the Quadrilateral ABCD are taken as below\\
{
\setlength\extrarowheight{2pt}
\centering
	\begin{tabular}{|c|c|}
	\hline
	\textbf{symbol}&\textbf{value}\\
	\hline
	r&8\\
	\hline
	$\theta$&pi/2.5\\
	\hline
	d&7\\
	\hline
	A&(0,0)\\
	\hline
	B&(d,0)\\
	\hline
	D&(rcos$\theta$,rsin$\theta$)\\
	\hline
	C&(D/1.5)+B\\
	\hline
\end{tabular}
}
\end{document}
\fi

	\item Find the area of triangle formed by the lines $y-x=0, x+y=0, \text{ and } x-k=0$.
\solution
		\iffalse
\documentclass[journal,10pt,twocolumn]{article}
\usepackage{graphicx}
\usepackage[margin=0.5in]{geometry}
\usepackage[cmex10]{amsmath}
\usepackage{array}
\usepackage{booktabs}
\usepackage{listings}
\title{\textbf{Line Assignment}}
\author{Bhavani Kanike}
\date{October 2022}

\providecommand{\norm}[1]{\left\lVert#1\right\rVert}
\providecommand{\abs}[1]{\left\vert#1\right\vert}
\let\vec\mathbf
\newcommand{\myvec}[1]{\ensuremath{\begin{pmatrix}#1\end{pmatrix}}}
\newcommand{\mydet}[1]{\ensuremath{\begin{vmatrix}#1\end{vmatrix}}}
\providecommand{\brak}[1]{\ensuremath{\left(#1\right)}}

\begin{document}

\maketitle
\paragraph{\textit{Problem Statement} 
\fi
ABCD is a quadrilateral in which $\vec{P}, \vec{Q}, \vec{R}$ and $\vec{S}$ are mid-points of the sides AB, BC, CD and DA (see Fig \ref{fig:9/8/2/1}). AC is a diagonal. 
		
Show that 
\begin{enumerate}
	\item $SR \parallel AC$ and $SR =\frac{1}{2} AC$
\item $PQ = SR$
\item $PQRS$ is a parallelogram.
\end{enumerate}
 	\begin{figure}
		\centering
 \includegraphics[width=\columnwidth]{chapters/9/8/2/1/figs/line1.pdf}
		\caption{}
		\label{fig:9/8/2/1}
  	\end{figure}
	\solution 
	Using 
	  \eqref{eq:section_formula},
	\begin{align}
		\label{eq:9/8/2/1}
		\begin{split}
		\vec{P} &= \frac{\vec{A}+\vec{B}}{2}\\
 \vec{Q} &= \frac{\vec{C}+\vec{B}}{2}\\
 \vec{R} &= \frac{\vec{C}+\vec{D}}{2}\\
 \vec{S} &= \frac{\vec{D}+\vec{A}}{2}
		\end{split}
	\end{align}
\begin{enumerate}
	\item
	Consequently, 
	\begin{align}
\vec{R}
		-\vec{S} &= \frac{\vec{C}-\vec{A}}{2}
		\\
		\implies SR &\parallel AC
	\end{align}
	Also, 
	\begin{align}
		\norm{\vec{R}
		-\vec{S}} &= \frac{\norm{\vec{C}-\vec{A}}}{2}
		\\
		\implies SR &= \frac{1}{2}AC
	\end{align}
\item 	From 
		\eqref{eq:9/8/2/1},
	\begin{align}
\vec{R}
		-\vec{S} = \vec{Q}-\vec{P}
	\end{align}
	which means that $PQRS$ is a parallelogram and $PQ = SR$.
\end{enumerate}
%
\iffalse
\begin{figure}[h]
\centering
\includegraphics[width=1\columnwidth]
\caption{Figure}
\label{fig:triangle}
\end{figure}

\section*{Solution}

$\boldsymbol Given :$  ABCD is a Quadrilateral P,Q,R and S are the midpoints of line AB,BC,CD,DA.We can obtain the points P,Q,R and S from A,B,C and D and are given by\\\\
\boldmath
\unboldmath
(3) To prove that PQRS is a parallelogram we need to prove  PQ // SR
To prove SR $\parallel$ PQ\\
Direction vector of line SR  $\boldsymbol {(R-S) =  \frac{(C-A)}{2}}$\\\\
Direction vector of line PQ  $\boldsymbol {(Q-P)= \frac{(C-A)}{2}}$\\\\
\begin{equation}
	\boldsymbol {(R-S) = (Q-P) = \frac{(C-A)}{2}}\\
\end{equation}
Since the direction vectors of line SR and PQ are in same direction\\\\
$SR \parallel PQ$\\
Therefore,
$\boldsymbol{ PQRS }$ is a parallelogram\\\\

	
(1)  Directional vector of line SR  = $\boldsymbol {(R-S)}$ = $\frac{\boldsymbol{(C-A)}}{2} $\\
Directional vector of line AC  = $\boldsymbol {(C-A)}$\\

It is observed that the constant k is $\frac{1}{2}$

Therefore
\begin{equation}
	SR \parallel AC
\end{equation} 

and from equation 1 
\begin{equation}
	\boldsymbol {SR = \frac{1}{2}AC}    
\end{equation}\\


(2)   To prove PQ = SR\\ 
		From euqation 1\\\\
\begin{equation}
		\boldsymbol{ (Q-P) = (R-S) = \frac{(C-A)}{2}}
\end{equation}
	 



\section{Execution}
The below python code realizes the construction:
\begin{lstlisting}
https://github.com/bhavani360/FWC_assignments
\end{lstlisting}
	
\section*{Construction}
The dimensions of the Quadrilateral ABCD are taken as below\\
{
\setlength\extrarowheight{2pt}
\centering
	\begin{tabular}{|c|c|}
	\hline
	\textbf{symbol}&\textbf{value}\\
	\hline
	r&8\\
	\hline
	$\theta$&pi/2.5\\
	\hline
	d&7\\
	\hline
	A&(0,0)\\
	\hline
	B&(d,0)\\
	\hline
	D&(rcos$\theta$,rsin$\theta$)\\
	\hline
	C&(D/1.5)+B\\
	\hline
\end{tabular}
}
\end{document}
\fi

\item A ray of light passing through the point $\brak{1, 2}$ reflects on the x-axis at point $\vec{A}$ and the reflected ray passes through the point $\brak{5, 3}$. Find the coordinates of $\vec{A}$.
\\
    \solution 
		\iffalse
\documentclass[journal,10pt,twocolumn]{article}
\usepackage{graphicx}
\usepackage[margin=0.5in]{geometry}
\usepackage[cmex10]{amsmath}
\usepackage{array}
\usepackage{booktabs}
\usepackage{listings}
\title{\textbf{Line Assignment}}
\author{Bhavani Kanike}
\date{October 2022}

\providecommand{\norm}[1]{\left\lVert#1\right\rVert}
\providecommand{\abs}[1]{\left\vert#1\right\vert}
\let\vec\mathbf
\newcommand{\myvec}[1]{\ensuremath{\begin{pmatrix}#1\end{pmatrix}}}
\newcommand{\mydet}[1]{\ensuremath{\begin{vmatrix}#1\end{vmatrix}}}
\providecommand{\brak}[1]{\ensuremath{\left(#1\right)}}

\begin{document}

\maketitle
\paragraph{\textit{Problem Statement} 
\fi
ABCD is a quadrilateral in which $\vec{P}, \vec{Q}, \vec{R}$ and $\vec{S}$ are mid-points of the sides AB, BC, CD and DA (see Fig \ref{fig:9/8/2/1}). AC is a diagonal. 
		
Show that 
\begin{enumerate}
	\item $SR \parallel AC$ and $SR =\frac{1}{2} AC$
\item $PQ = SR$
\item $PQRS$ is a parallelogram.
\end{enumerate}
 	\begin{figure}
		\centering
 \includegraphics[width=\columnwidth]{chapters/9/8/2/1/figs/line1.pdf}
		\caption{}
		\label{fig:9/8/2/1}
  	\end{figure}
	\solution 
	Using 
	  \eqref{eq:section_formula},
	\begin{align}
		\label{eq:9/8/2/1}
		\begin{split}
		\vec{P} &= \frac{\vec{A}+\vec{B}}{2}\\
 \vec{Q} &= \frac{\vec{C}+\vec{B}}{2}\\
 \vec{R} &= \frac{\vec{C}+\vec{D}}{2}\\
 \vec{S} &= \frac{\vec{D}+\vec{A}}{2}
		\end{split}
	\end{align}
\begin{enumerate}
	\item
	Consequently, 
	\begin{align}
\vec{R}
		-\vec{S} &= \frac{\vec{C}-\vec{A}}{2}
		\\
		\implies SR &\parallel AC
	\end{align}
	Also, 
	\begin{align}
		\norm{\vec{R}
		-\vec{S}} &= \frac{\norm{\vec{C}-\vec{A}}}{2}
		\\
		\implies SR &= \frac{1}{2}AC
	\end{align}
\item 	From 
		\eqref{eq:9/8/2/1},
	\begin{align}
\vec{R}
		-\vec{S} = \vec{Q}-\vec{P}
	\end{align}
	which means that $PQRS$ is a parallelogram and $PQ = SR$.
\end{enumerate}
%
\iffalse
\begin{figure}[h]
\centering
\includegraphics[width=1\columnwidth]
\caption{Figure}
\label{fig:triangle}
\end{figure}

\section*{Solution}

$\boldsymbol Given :$  ABCD is a Quadrilateral P,Q,R and S are the midpoints of line AB,BC,CD,DA.We can obtain the points P,Q,R and S from A,B,C and D and are given by\\\\
\boldmath
\unboldmath
(3) To prove that PQRS is a parallelogram we need to prove  PQ // SR
To prove SR $\parallel$ PQ\\
Direction vector of line SR  $\boldsymbol {(R-S) =  \frac{(C-A)}{2}}$\\\\
Direction vector of line PQ  $\boldsymbol {(Q-P)= \frac{(C-A)}{2}}$\\\\
\begin{equation}
	\boldsymbol {(R-S) = (Q-P) = \frac{(C-A)}{2}}\\
\end{equation}
Since the direction vectors of line SR and PQ are in same direction\\\\
$SR \parallel PQ$\\
Therefore,
$\boldsymbol{ PQRS }$ is a parallelogram\\\\

	
(1)  Directional vector of line SR  = $\boldsymbol {(R-S)}$ = $\frac{\boldsymbol{(C-A)}}{2} $\\
Directional vector of line AC  = $\boldsymbol {(C-A)}$\\

It is observed that the constant k is $\frac{1}{2}$

Therefore
\begin{equation}
	SR \parallel AC
\end{equation} 

and from equation 1 
\begin{equation}
	\boldsymbol {SR = \frac{1}{2}AC}    
\end{equation}\\


(2)   To prove PQ = SR\\ 
		From euqation 1\\\\
\begin{equation}
		\boldsymbol{ (Q-P) = (R-S) = \frac{(C-A)}{2}}
\end{equation}
	 



\section{Execution}
The below python code realizes the construction:
\begin{lstlisting}
https://github.com/bhavani360/FWC_assignments
\end{lstlisting}
	
\section*{Construction}
The dimensions of the Quadrilateral ABCD are taken as below\\
{
\setlength\extrarowheight{2pt}
\centering
	\begin{tabular}{|c|c|}
	\hline
	\textbf{symbol}&\textbf{value}\\
	\hline
	r&8\\
	\hline
	$\theta$&pi/2.5\\
	\hline
	d&7\\
	\hline
	A&(0,0)\\
	\hline
	B&(d,0)\\
	\hline
	D&(rcos$\theta$,rsin$\theta$)\\
	\hline
	C&(D/1.5)+B\\
	\hline
\end{tabular}
}
\end{document}
\fi

%\item Find the equations of the lines, which cut-off intercepts on the axes whose sum and product are 1 and -6, resspectively.
%	\\
%    \solution 
%		\iffalse
\documentclass[12pt]{article}
\usepackage{graphicx}
%\documentclass[journal,12pt,twocolumn]{IEEEtran}
\usepackage[none]{hyphenat}
\usepackage{graphicx}
\usepackage{listings}
\usepackage[english]{babel}
\usepackage{graphicx}
\usepackage{caption}
\usepackage[parfill]{parskip}
\usepackage{hyperref}
\usepackage{booktabs}
%\usepackage{setspace}\doublespacing\pagestyle{plain}
\def\inputGnumericTable{}
\usepackage{color}                                            %%
    \usepackage{array}                                            %%
    \usepackage{longtable}                                        %%
    \usepackage{calc}                                             %%
    \usepackage{multirow}                                         %%
    \usepackage{hhline}                                           %%
    \usepackage{ifthen}
\usepackage{array}
\usepackage{amsmath}   % for having text in math mode
\usepackage{parallel,enumitem}
\usepackage{listings}
\lstset{
language=tex,
frame=single,
breaklines=true
}
%Following 2 lines were added to remove the blank page at the beginning
\usepackage{atbegshi}% http://ctan.org/pkg/atbegshi
\AtBeginDocument{\AtBeginShipoutNext{\AtBeginShipoutDiscard}}
%
%New macro definitions
\newcommand{\mydet}[1]{\ensuremath{\begin{vmatrix}#1\end{vmatrix}}}
\providecommand{\brak}[1]{\ensuremath{\left(#1\right)}}
\providecommand{\abs}[1]{\left\vert#1\right\vert}
\providecommand{\norm}[1]{\left\lVert#1\right\rVert}
\newcommand{\solution}{\noindent \textbf{Solution: }}
\newcommand{\myvec}[1]{\ensuremath{\begin{pmatrix}#1\end{pmatrix}}}
\let\vec\mathbf
\begin{document}
\begin{center}
\title{\textbf{ Intercepts Lines}}
\date{\vspace{-5ex}} %Not to print date automatically
\maketitle
\end{center}
\setcounter{page}{eq:11/10/4/31}
\section*{11$^{th}$ Maths - Chapter 10}
This is Problem-3 from Exercise 10.4
\section{Solution}
\fi
Let the $x$ intercept be $a$ and  the $y$ intercept be $b$ ,Then
\begin{align}
a+b&=1\label{eq:11/10/4/31}\\
ab&=-6 \label{eq:11/10/4/32}
\\
\implies  a = 3, b = -2
\end{align}
Thus, the possible 
intercepts are
\begin{align}
\myvec{3\\0}, \myvec{0\\-2},
\myvec{-2\\0}, \myvec{0\\3}
\end{align}
yielding
\begin{align}		
\vec{m}=\myvec{3\\2} \text{or,} \myvec{-2\\3}
\end{align}
\begin{enumerate}
\item For 
\begin{align}
\vec{n}
=\myvec{-2 \\3} 
\end{align}
the equation of the line is 
\begin{align}
	\vec{n}^\top\brak{\vec{x}-\vec{A}} &= 0 \\
	\myvec { -2 & 3 } \vec{x}  &= 6  
\end{align}
\item  For 
\begin{align}
\vec{n}
=\myvec{-3 \\-2} 
\end{align}
the equation of the line is 
\begin{align}
    \vec{n}^\top\brak{\vec{x}-\vec{B}} &= 0 \\  
	\myvec { -3 & -2 }  \vec{x}  &= 6        
\end{align}
\end{enumerate}
See Fig. 
\ref{fig:11/10/4/3line segment}.
\begin{figure}[h!]
\centering
\includegraphics[width=\columnwidth]{chapters/11/10/4/3/figs/inter.png}
\caption{}
\label{fig:11/10/4/3line segment}
\end{figure}

    \item A person standing at the junction (crossing) of two straight paths 
    represented by the equations 
    \begin{align}
        \myvec{2&-3}\vec{x} = -4 
        \label{eq:chapters/11/10/4/24/L1}
    \end{align}
    and
    \begin{align}
        \myvec{3&4}\vec{x} = 5
        \label{eq:chapters/11/10/4/24/L2}
    \end{align} 
    wants to reach the path whose equation is 
    \begin{align}
        \myvec{6&-7}\vec{x} = -8
        \label{eq:chapters/11/10/4/24/L3}
    \end{align}
    Find equation of the path that he should follow.
\\
    \solution 
		\iffalse
\documentclass[journal,10pt,twocolumn]{article}
\usepackage{graphicx}
\usepackage[margin=0.5in]{geometry}
\usepackage[cmex10]{amsmath}
\usepackage{array}
\usepackage{booktabs}
\usepackage{listings}
\title{\textbf{Line Assignment}}
\author{Bhavani Kanike}
\date{October 2022}

\providecommand{\norm}[1]{\left\lVert#1\right\rVert}
\providecommand{\abs}[1]{\left\vert#1\right\vert}
\let\vec\mathbf
\newcommand{\myvec}[1]{\ensuremath{\begin{pmatrix}#1\end{pmatrix}}}
\newcommand{\mydet}[1]{\ensuremath{\begin{vmatrix}#1\end{vmatrix}}}
\providecommand{\brak}[1]{\ensuremath{\left(#1\right)}}

\begin{document}

\maketitle
\paragraph{\textit{Problem Statement} 
\fi
ABCD is a quadrilateral in which $\vec{P}, \vec{Q}, \vec{R}$ and $\vec{S}$ are mid-points of the sides AB, BC, CD and DA (see Fig \ref{fig:9/8/2/1}). AC is a diagonal. 
		
Show that 
\begin{enumerate}
	\item $SR \parallel AC$ and $SR =\frac{1}{2} AC$
\item $PQ = SR$
\item $PQRS$ is a parallelogram.
\end{enumerate}
 	\begin{figure}
		\centering
 \includegraphics[width=\columnwidth]{chapters/9/8/2/1/figs/line1.pdf}
		\caption{}
		\label{fig:9/8/2/1}
  	\end{figure}
	\solution 
	Using 
	  \eqref{eq:section_formula},
	\begin{align}
		\label{eq:9/8/2/1}
		\begin{split}
		\vec{P} &= \frac{\vec{A}+\vec{B}}{2}\\
 \vec{Q} &= \frac{\vec{C}+\vec{B}}{2}\\
 \vec{R} &= \frac{\vec{C}+\vec{D}}{2}\\
 \vec{S} &= \frac{\vec{D}+\vec{A}}{2}
		\end{split}
	\end{align}
\begin{enumerate}
	\item
	Consequently, 
	\begin{align}
\vec{R}
		-\vec{S} &= \frac{\vec{C}-\vec{A}}{2}
		\\
		\implies SR &\parallel AC
	\end{align}
	Also, 
	\begin{align}
		\norm{\vec{R}
		-\vec{S}} &= \frac{\norm{\vec{C}-\vec{A}}}{2}
		\\
		\implies SR &= \frac{1}{2}AC
	\end{align}
\item 	From 
		\eqref{eq:9/8/2/1},
	\begin{align}
\vec{R}
		-\vec{S} = \vec{Q}-\vec{P}
	\end{align}
	which means that $PQRS$ is a parallelogram and $PQ = SR$.
\end{enumerate}
%
\iffalse
\begin{figure}[h]
\centering
\includegraphics[width=1\columnwidth]
\caption{Figure}
\label{fig:triangle}
\end{figure}

\section*{Solution}

$\boldsymbol Given :$  ABCD is a Quadrilateral P,Q,R and S are the midpoints of line AB,BC,CD,DA.We can obtain the points P,Q,R and S from A,B,C and D and are given by\\\\
\boldmath
\unboldmath
(3) To prove that PQRS is a parallelogram we need to prove  PQ // SR
To prove SR $\parallel$ PQ\\
Direction vector of line SR  $\boldsymbol {(R-S) =  \frac{(C-A)}{2}}$\\\\
Direction vector of line PQ  $\boldsymbol {(Q-P)= \frac{(C-A)}{2}}$\\\\
\begin{equation}
	\boldsymbol {(R-S) = (Q-P) = \frac{(C-A)}{2}}\\
\end{equation}
Since the direction vectors of line SR and PQ are in same direction\\\\
$SR \parallel PQ$\\
Therefore,
$\boldsymbol{ PQRS }$ is a parallelogram\\\\

	
(1)  Directional vector of line SR  = $\boldsymbol {(R-S)}$ = $\frac{\boldsymbol{(C-A)}}{2} $\\
Directional vector of line AC  = $\boldsymbol {(C-A)}$\\

It is observed that the constant k is $\frac{1}{2}$

Therefore
\begin{equation}
	SR \parallel AC
\end{equation} 

and from equation 1 
\begin{equation}
	\boldsymbol {SR = \frac{1}{2}AC}    
\end{equation}\\


(2)   To prove PQ = SR\\ 
		From euqation 1\\\\
\begin{equation}
		\boldsymbol{ (Q-P) = (R-S) = \frac{(C-A)}{2}}
\end{equation}
	 



\section{Execution}
The below python code realizes the construction:
\begin{lstlisting}
https://github.com/bhavani360/FWC_assignments
\end{lstlisting}
	
\section*{Construction}
The dimensions of the Quadrilateral ABCD are taken as below\\
{
\setlength\extrarowheight{2pt}
\centering
	\begin{tabular}{|c|c|}
	\hline
	\textbf{symbol}&\textbf{value}\\
	\hline
	r&8\\
	\hline
	$\theta$&pi/2.5\\
	\hline
	d&7\\
	\hline
	A&(0,0)\\
	\hline
	B&(d,0)\\
	\hline
	D&(rcos$\theta$,rsin$\theta$)\\
	\hline
	C&(D/1.5)+B\\
	\hline
\end{tabular}
}
\end{document}
\fi

		\begin{figure}[H]
        \centering
        \includegraphics[width=\columnwidth]{chapters/11/10/4/24/figs/crossing.png}
        \caption{AF is the required line.}
        \label{fig:chapters/11/10/4/24/crossing}
    \end{figure}
	\item Find the equation of the line passing through the point of intersection of the lines $4x + 7y - 3 = 0$ and $2x - 3y + 1 = 0$ that has equal intercepts on the axes.\\
	\solution 
	  Given lines can be written in the form of \begin{align}
        \Vec{n}^{\top}\Vec{x} = c
    \end{align}
   Therefore,
		\begin{align}
       \myvec{4&7}\vec{x}=3
       \label{eq:11/10.4/12/2}
   \end{align} 
   \begin{align}
       \myvec{2&-3}\vec{x}=-1
       \label{eq:11/10.4/12/3}
   \end{align}
   Now, line equation that has equal intercepts on the axes is
   \begin{align}
       \myvec{1 & 1}\Vec{x}=c
       \label{eq:11/10.4/12/4}
   \end{align}
   Solving equations \eqref{eq:11/10.4/12/2} and \eqref{eq:11/10.4/12/3}
		augumented matrix is
 \begin{align}
    \myvec{4&7&3\\2&-3&-1}\\
    \xleftrightarrow{R_1 \leftarrow 4 R_1}
    \myvec{1&\frac{7}{4}&\frac{3}{4}\\2&-3&-1}
    \xleftrightarrow{R_2 \leftarrow R_2 - 2R_1}
    \myvec{1&\frac{7}{4}&\frac{3}{4}\\0&\frac{-13}{2}&\frac{-5}{2}}\\
    \xleftrightarrow{R_2 \leftarrow \frac{-2}{13}R_2}
    \myvec{1&\frac{7}{4}&\frac{3}{4}\\0&1&\frac{5}{13}}
    \xleftrightarrow{R_1 \leftarrow R_1-\frac{7}{4}R_2}
    \myvec{1&0&\frac{1}{13}\\0&1&\frac{5}{13}}
\end{align}
Therfore, \begin{align}    
\vec{x} = \myvec{\frac{1}{13}\\\frac{5}{13}}
\end{align}
Also this point lies on the equation \eqref{eq:11/10.4/12/4}
\begin{center}
\begin{align}
    \myvec{1 & 1}\myvec{\frac{1}{13}\\\frac{5}{13}} = c\\
    \frac{1}{13}+\frac{5}{13} = c
    \end{align}
    \end{center}
    Therefore, the equation is \begin{align}
        \myvec{1&1}\Vec{x}=\frac{6}{13}
    \end{align}

\begin{figure}[H]
    \centering
    \includegraphics[width=\columnwidth]{chapters/11/10/4/12/figs/straightline.png}
    \caption{Straight Lines}
    \label{fig:enter-label}
\end{figure}

    \item If three lines whose equations are $y=m_1x+c_1$, $y=m_2x+c_2$ and $y=m_3x+c_3$ are concurrent, then show that $m_1(c_2-c_3)+m_2(c_3-c_1)+m_3(c_1-c_2) = 0.$\\
    \solution
      \iffalse
\documentclass[journal,10pt,twocolumn]{article}
\usepackage{graphicx}
\usepackage[margin=0.5in]{geometry}
\usepackage[cmex10]{amsmath}
\usepackage{array}
\usepackage{booktabs}
\usepackage{listings}
\title{\textbf{Line Assignment}}
\author{Bhavani Kanike}
\date{October 2022}

\providecommand{\norm}[1]{\left\lVert#1\right\rVert}
\providecommand{\abs}[1]{\left\vert#1\right\vert}
\let\vec\mathbf
\newcommand{\myvec}[1]{\ensuremath{\begin{pmatrix}#1\end{pmatrix}}}
\newcommand{\mydet}[1]{\ensuremath{\begin{vmatrix}#1\end{vmatrix}}}
\providecommand{\brak}[1]{\ensuremath{\left(#1\right)}}

\begin{document}

\maketitle
\paragraph{\textit{Problem Statement} 
\fi
ABCD is a quadrilateral in which $\vec{P}, \vec{Q}, \vec{R}$ and $\vec{S}$ are mid-points of the sides AB, BC, CD and DA (see Fig \ref{fig:9/8/2/1}). AC is a diagonal. 
		
Show that 
\begin{enumerate}
	\item $SR \parallel AC$ and $SR =\frac{1}{2} AC$
\item $PQ = SR$
\item $PQRS$ is a parallelogram.
\end{enumerate}
 	\begin{figure}
		\centering
 \includegraphics[width=\columnwidth]{chapters/9/8/2/1/figs/line1.pdf}
		\caption{}
		\label{fig:9/8/2/1}
  	\end{figure}
	\solution 
	Using 
	  \eqref{eq:section_formula},
	\begin{align}
		\label{eq:9/8/2/1}
		\begin{split}
		\vec{P} &= \frac{\vec{A}+\vec{B}}{2}\\
 \vec{Q} &= \frac{\vec{C}+\vec{B}}{2}\\
 \vec{R} &= \frac{\vec{C}+\vec{D}}{2}\\
 \vec{S} &= \frac{\vec{D}+\vec{A}}{2}
		\end{split}
	\end{align}
\begin{enumerate}
	\item
	Consequently, 
	\begin{align}
\vec{R}
		-\vec{S} &= \frac{\vec{C}-\vec{A}}{2}
		\\
		\implies SR &\parallel AC
	\end{align}
	Also, 
	\begin{align}
		\norm{\vec{R}
		-\vec{S}} &= \frac{\norm{\vec{C}-\vec{A}}}{2}
		\\
		\implies SR &= \frac{1}{2}AC
	\end{align}
\item 	From 
		\eqref{eq:9/8/2/1},
	\begin{align}
\vec{R}
		-\vec{S} = \vec{Q}-\vec{P}
	\end{align}
	which means that $PQRS$ is a parallelogram and $PQ = SR$.
\end{enumerate}
%
\iffalse
\begin{figure}[h]
\centering
\includegraphics[width=1\columnwidth]
\caption{Figure}
\label{fig:triangle}
\end{figure}

\section*{Solution}

$\boldsymbol Given :$  ABCD is a Quadrilateral P,Q,R and S are the midpoints of line AB,BC,CD,DA.We can obtain the points P,Q,R and S from A,B,C and D and are given by\\\\
\boldmath
\unboldmath
(3) To prove that PQRS is a parallelogram we need to prove  PQ // SR
To prove SR $\parallel$ PQ\\
Direction vector of line SR  $\boldsymbol {(R-S) =  \frac{(C-A)}{2}}$\\\\
Direction vector of line PQ  $\boldsymbol {(Q-P)= \frac{(C-A)}{2}}$\\\\
\begin{equation}
	\boldsymbol {(R-S) = (Q-P) = \frac{(C-A)}{2}}\\
\end{equation}
Since the direction vectors of line SR and PQ are in same direction\\\\
$SR \parallel PQ$\\
Therefore,
$\boldsymbol{ PQRS }$ is a parallelogram\\\\

	
(1)  Directional vector of line SR  = $\boldsymbol {(R-S)}$ = $\frac{\boldsymbol{(C-A)}}{2} $\\
Directional vector of line AC  = $\boldsymbol {(C-A)}$\\

It is observed that the constant k is $\frac{1}{2}$

Therefore
\begin{equation}
	SR \parallel AC
\end{equation} 

and from equation 1 
\begin{equation}
	\boldsymbol {SR = \frac{1}{2}AC}    
\end{equation}\\


(2)   To prove PQ = SR\\ 
		From euqation 1\\\\
\begin{equation}
		\boldsymbol{ (Q-P) = (R-S) = \frac{(C-A)}{2}}
\end{equation}
	 



\section{Execution}
The below python code realizes the construction:
\begin{lstlisting}
https://github.com/bhavani360/FWC_assignments
\end{lstlisting}
	
\section*{Construction}
The dimensions of the Quadrilateral ABCD are taken as below\\
{
\setlength\extrarowheight{2pt}
\centering
	\begin{tabular}{|c|c|}
	\hline
	\textbf{symbol}&\textbf{value}\\
	\hline
	r&8\\
	\hline
	$\theta$&pi/2.5\\
	\hline
	d&7\\
	\hline
	A&(0,0)\\
	\hline
	B&(d,0)\\
	\hline
	D&(rcos$\theta$,rsin$\theta$)\\
	\hline
	C&(D/1.5)+B\\
	\hline
\end{tabular}
}
\end{document}
\fi

\begin{figure}[H]
    \centering
    \includegraphics[width=\columnwidth]{chapters/11/10/4/10/figs/concurrent.png}
    \caption{Concurrent Lines}
    \label{fig:chapters/11/10/4/10/figs/concurrent.png}
\end{figure}
\end{enumerate}

\subsection{Exemplar}

\begin{enumerate}[label=\thesection.\arabic*,ref=\thesection.\theenumi]
\item Find the equation of the straight line which passes through the point (1, -2) and cuts off equal intercepts from axes.
\item Find the equation of the line passing through the point (5,2) and perpendicular to the line joining the points (2,3) and (3, -1).
\item Find the angle between the lines $y(2-\sqrt{3})(x+5)\text{ and }y=(2+\sqrt{3})(x-7)$.
\item Find the equation of the lines which passes the point (3,4) and cuts off intercepts from the coordinate axes such that their sum is 14.
\item Find the points on the line $x+y=4$ which lie at a unit distance from the line $4x+3y=10$.
\item Show that the tangent of an angle between the lines $\frac{x}{a}+\frac{y}{b}=1 \text{ and }\frac{x}{a}-\frac{y}{b}=1$ is $\frac{2ab}{a^2-b^2}$.
\item Find the equation of lines passing through (1,2) and making angle $30\degree$ with $y$-axis.
\item Find the equation of the line passing through the point of intersection of $2x+y=5\text{ and }x+3y+8=0$ and parallel the line $3x+4y=7$.
\item For what values of $a$ and $b$ the intercepts cut off on the coordinate axes by the line $ax+by+8=0$are equal in length but opposite in signs to those cut off by the line $2x-3y+=0$ on the axes.
\item If the intercept of a line between the coordinate axes is divided by the point (-5,4) in the ratio 1:2 then find the equation of the line.
\item Find the equation of a straight line on which length of perpendicular from the origin is four units and the line makes on angle of 120$\degree$ with the positive direction of $x$-axis. [\textbf{Hint} : Use normal form, here $\omega =30\degree$.]
\item Find the equation of one of the sides of an isosceles right angled triangle whose hypotenuse is given by $3x+4y=4$ and the opposite vertex of the hypotenuse is (2,2).
\end{enumerate}
 \textbf{Long Answer Type}
 \begin{enumerate}[resume]
\item If the equation of the base of an equilateral triangle is $x+y=2$ and the vertex is (2,-1), then find the length of the side of the triangle. 
[\textbf{Hint} : Find length of perpendicular ($p$) from (2,-1) to the line and use $p=l \sin 60degree$,where $l$ is the length of the triangle].
\item A variable line passes through a fixed point $\vec{P}$.The algebraic sum of the perpendiculars drawn from the points (2,0),(0,2) and (1,1) on the line is zero. Find the coordinates of the point $\vec{P}$.  
[\textbf{Hint} : let the slope of the line be $m$. Then the equation of the line passing through the fixed point $\vec{P}(x_1,y_1) y-y_1=m(x-x_1)$. Taking the algebraic sum of perpendicular distances equal to zero, we get $y-l=m(x-1)$. Thus $(x_1,y_1)$ is (1,1).]
\item In what direction should a line be drawn through the point (1,2) so that its point of intersection with line $x+y=4$ is at a distance $\sqrt{6}{3}$ from the given equilateral    
\item A straight line moves so that the sum of the reciprocals of its intercepts made on axes is constant. Show that the line passes through a fixed point. [\textbf{Hint} : $\frac{x}{a}+\frac{y}{b}=1\text{ where} \frac{1}{a}+\frac{1}{b}=\text{ constant }=\frac{1}{k}$(say). This implies that $\frac{k}{a}+\frac{k}{b}=1$ line passes through the fixed point $(k,k)$.]
\item Find the equation of the line which passes through the point (-4,3) and the portion of the line intercepted between the axes is divided internally in ratio 5:3 by this point.
\item Find the equations of the lines through the point of intersection of the line $x-y+1=0 \text{ and }2x-3y+5=0$ and whose distance from the point (3,2) is $\frac{7}{5}$.
\item If the sum of the distances of a moving point in a plane from the axes is $l$, then finds the locus of the point. [\textbf{Hint} :Given that $\abs{x}+\abs{y}=1$, which  gives four  sides of a square.] 
\item $\vec{P}_1,\vec{P}_2$ are points on either of the two lines $y-\sqrt{3}\abs{x}=2$ at a distance of 5 units from their point of intersection. Find the coordinates of the root of perpendiculars drawn from $P_1, P_2$ on the bisector of the angle between the given lines.
[\textbf{Hint} : Lines are $y=\sqrt{3}x+2 \text{ and }y=-\sqrt{3}x+2$ according as $x\geq0$ or $x0. y$-xis is the bisector of the angles between the lines. $P_1, P_2$ are the points on these lines at a distance of 5 units from the point of intersection of these lines which have a point on $y$-axis as a common foot of perpendiculars from these points. The $y$-coordinate of the foot of the perpendicular is given by 2=5 $\cos{30\degree}$.]
\item If $p$ is the length of perpendicular from the origin on the lien $\frac{x}{a}+\frac{y}{b}=1$ and $a^2,p^2,b^2$ are in A.P, then show that $a^4+b^4=0$.
\end{enumerate}
\textbf{Objective Type Questions}\\
choose the correct answer from the given four options in Exercises 22 to 41
\begin{enumerate}[resume]
\item A line cutting off intercept -3 from the tangent at angle to the $x$-axis is $\sqrt{3}{5}$, its equation is
\begin{enumerate}
\item $5y-3x+15=0$
\item $3y-5x+15=0$
\item $5y-3x-15=0$
\item none of these
\end{enumerate}
\item Slope of a line which cuts off intercepts of equal length on the axes is 
\begin{enumerate}
\item -1
\item -0
\item 2
\item $\sqrt{3}$
\end{enumerate}
\item The equation of the straight line passing through the point (3,2) and perpendicular to the line $y=x$ is
\begin{enumerate}
\item $x-y=5$
\item $x+y=5$
\item $x+y=1$
\item $x-y=1$
\end{enumerate}
\item The equation of the line passing through the point (1,2) and perpendicular to the line $x+y+1=0$ is
\begin {enumerate}
\item $y-x+1=0$
\item $y-x-1=0$
\item $y-x+2=0$
\item $y-x-1=0$
\end{enumerate}
\item The tangent of angle between the lines whose intercepts on the axes are $a,-b$ and $b,-a$, respectively, is
\begin{enumerate}
\item $\frac{a^2-b^2}{ab}$
\item $\frac{b^2-a^2}{2}$
\item $\frac{b^2-a^2}{2ab}$
\item none of these 
\end{enumerate}
\item If the line $\frac{x}{a}+\frac{y}{b}=1$ passes the points (2,-3) and (4,-5), then $(a,b)$ is 
\begin{enumerate}
\item (1,1)
\item (-1,1)
\item (1,-1)
\item (-1,-1)
\end{enumerate}
\item The distance of the point of intersection of the lines $2x-3y+5=0 \text{ and }3x+4y=0$ from the line $5x-2y=0$ is
\begin{enumerate}
\item $\frac{130}{17\sqrt{29}}$
\item $\frac{13}{7\sqrt{29}}$
\item $\frac{130}{7}$
\item none of these
\end{enumerate}
\item The equations of the lines which pass through the point (3, -2) and are inclined at $60\degree$ to the line $\sqrt{3} x+y=1$ is
\begin{enumerate}
\item $y+2=0$, $\sqrt{3}x-y-2-3\sqrt{3}=0$
\item $x-2=0$, $\sqrt{3}x-y+2+3\sqrt{3}=0$
\item $\sqrt{3}x-y-2-3\sqrt{3}=0$
\item None of these
\end{enumerate}
\item The equations of the lines passing through the point (1,0) and at a distance $\frac{\sqrt{3}}{2}$ from the origin, are 
\begin{enumerate}
\item $\sqrt{3}x+y-\sqrt{3}=0$, $\sqrt{3}x-y-\sqrt{3}=0$
\item $\sqrt{3}x+y+\sqrt{3}=0$, $\sqrt{3}x-y+\sqrt{3}=0$
\item $x+\sqrt{3}y-\sqrt{3}=0$, $\sqrt{3}y-\sqrt{3}=0$
\item None of these.
\end{enumerate}
\item The distance between the lines $y=mx+c$,\text{ and }$y=mx+c^2$ is
\begin{enumerate}
\item $\frac{c_1-c_2}{\sqrt{m+1}}$
\item $\frac{\abs{c_1-c_2}}{\sqrt{1+m^2}}$
\item $\frac{c^2-c^1}{\sqrt{1+m^2}}$
\item 0
\end{enumerate}
\item The coordinates of the foot of perpendiculars from the point (2,3) on the line $y=3x+4$ is given by 
\begin{enumerate} 
\item $\frac{37}{10}$, $\frac{-1}{10}$
\item $\frac{-1}{10}$, $\frac{37}{10}$
\item $\frac{10}{37}$, -10
\item $\frac{2}{3}$, $\frac{-1}{3}$
\end{enumerate}
\item If the coordinates of middle point of the portion of a line intercepted between the coordinate axes is (3,2),then the equation of the line will be
\begin{enumerate}
\item $2x+3y=12$
\item $3x+2y=12$
\item $4x-3y=6$
\item $5x-2y=10$
\end{enumerate}
\item Equation of the line passing through (1,2) and parallel to the line $y=3x-1$ is
\begin{enumerate}
\item $y+2=x+1$
\item $y+2=3(x+1)$
\item $y-2=3(x-1)$
\item $y-2=x-1$
\end{enumerate}
\item Equations of diagonals of the square formed by the lines $x=0$, $y=0$, $x=1$ and $y=1$ are
\begin{enumerate}
\item $y=x$, $y+x=1$
\item $y=x$,$x+y=2$
\item $2y=x$, $y+x=\frac{1}{3}$
\item $y=2x$, $y+2x=1$
\end{enumerate}
\item For specifying a straight line, how many geometrical parameters should be known?
\begin{enumerate}
\item 1
\item 2
\item 4
\item 3
\end{enumerate}
\item The point (4,1)undergoes the following two successive transformations :
\begin{enumerate}
\item Reflection about the line $y=x$
\item Translation through a distance 2 units along the positive $x$-axis 
\end{enumerate}
Then the final coordinates of the point are
\begin{enumerate}
\item (4,3)
\item (3,4)
\item (1,4)
\item $\frac{7}{2}$,$\frac{7}{2}$
\end{enumerate}
\item A point equidistant from the lines $4x+3y+10=0$, $5x-12y+26=0$ and $7x+24y-50=0$ is
\begin{enumerate}
\item (1,-1)
\item (1,1)
\item (0,0)
\item (0,1)
\end{enumerate}
\item A line passes through (2,2) and is perpendicular to the line $3x+y=3$. Its $y$-intercept is 
\begin{enumerate}
\item $\frac{1}{3}$
\item $\frac{2}{3}$
\item 1
\item $\frac{4}{3}$
\end{enumerate}
\item The ratio in which the line $3x+4y+2=0$ divides the distance between the lines $3x+4y+5=0$ and $3x+4y-5=0$ is
\begin{enumerate}
\item 1:2
\item 3:7
\item 2:3
\item 2:5
\end{enumerate}
\item One vertex of the equilateral with centroid at the origin and one side as $x+y-2=0$ is
\begin{enumerate}
\item (-1,-1)
\item (2,2)
\item (-2-2)
\item (2,-2)
\end{enumerate}
[\textbf{Hint} : Let $ABC$ be the equilateral triangle with vertex $\vec{A}(h,k)\text{ and let }\vec{D}(\alpha,\beta)$ be the point on $BC$. Then $\frac{2\alpha+h}{3}=0=\frac{2\beta+k}{3}$. Also ${\alpha+\beta-2=0}\text{ and }\frac{k-0}{h-o}x(-1)=-1$] 
\end{enumerate}

Fill in the blank in Exercises 42 to 47.
\begin{enumerate}[resume]
\item If $a,b,c$ are is A.P.,then the straight lines $ax+by+c=0$ will always pass through \rule{1cm}{0.15mm}.
\item The line which cuts off equal intercept from the axes and pass through the equilateral2) is \rule{1cm}{0.15mm}.
\item Equations of the lines through the point (3,2) and making an angle of $40\degree$ with the line $x-2y=3$ are \rule{1cm}{0.15mm}.   
\item The points (3,4) and (2,-6)are situated on the \rule{1cm}{0.15mm} of the line $3x-4y-8=0$.
\item A point moves so that square of its distance from the point (3,-2) is numerically equal to its distance from the line $5x-12y=3$. The equation of its locus is %\rule{1cm}{0.15mm}.
\item Locus of the mid-points of the portion of the line $x\sin\theta+y\cos\theta=p$ intercepted between the axes is \rule{1cm}{0.15mm}.
State whether the statements in Exercises 48 to 56 are true or false. Justify.
\item If the vertices of a triangle have integral coordinates, then the triangle can not be equilateral.
\item The points $\vec{A}(2,1)$, $\vec{B}(0,5)$, $\vec{C}(-1,2)$ are collinear.
\item Equation of the line passing through the point $(a\cos^3\theta, a\sin^3\theta)$ and perpendicular to the line $x\sec\theta+y\csc\theta=a$ is $x\cos\theta-y\sin\theta=\alpha\sin2\theta$.
\item The straight line $5x+4y=0$ passes through the point of intersection of the straight lines $x+2y-10=0$ and $2x+y+5=0$.
\item The vertex of on equilateral triangle is (intercepted equation of the opposite side is $x+y=2$.then the other two sides are $y-3=(2\pm\sqrt{3})(x-2)$.
\item The equation of the line joining the point (3,5) to the point of intersection of the lines $4x+y-=0$ and $7x-3y-5=0$ is equidistant from the points (0,0) and (8,34).
\item The line $\frac{x}{a}+\frac{y}{b}=1$ moves in such a way that $\frac{1}{a^2}+\frac{1}{b^2}=\frac{1}{c^2}$, where $c$ is a constant.The locus of the foot of the perpendicular from the origin on the given line is $x^2+y^2=c^2$.
\item The lines $ax+2y+1=0$, $bx=3y+1=0\text{ and }cx+4y+1=0$ are concurrent if $a$, $b$, $c$ are in G.P.
\item Line joining the points (3,-4) and (-2,6) is perpendicular to the line joining the points (-3,6) and (9,-18).
\end{enumerate}
Match the questions given under Column $C_1$ with their appropriate answers given under the Column $C_2$ is Exercises 57 to 59.
\begin{enumerate}[resume]
\item 
\begin{center}
\begin{tabular}{cccccc}
\textbf{Column $C_1$} & & & & &  \textbf{Column $C_2$}\\
\end{tabular}   
\end{center}
\begin{matchtabular}
  The coordinates of the points P and Q on the line x + 5y = 13 which are at a distance of 2 units from the line 12x – 5y + 26 = 0 are & (3,1),(-7,11)\\
  The coordinates of the point on the line x + y = 4, which are at a unit distance from the line 4x + 3y – 10 = 0 are & $-\frac{1}{11},\frac{11}{3}$ , $\frac{4}{3},\frac{7}{3}$\\
  The coordinates of the point on the line joining A (–2, 5) and B (3, 1) such that AP = PQ = QB are & 1,$\frac{12}{5}$ , $-3,\frac{16}{5}$\\
\end{matchtabular}
\\
\item The value of the $\lambda$, if the lines\\$(2x+3y+4)+\lambda(6x-y+12)=0$ are
\begin{center}
\begin{tabular}{cccccc}
\textbf{Column $C_1$} & & & & &  \textbf{Column $C_2$}\\
\end{tabular}   
\end{center}
\begin{matchtabular}
parallel to $y$-axis is & $\lambda =-\frac{3}{4}$\\
perpendicular to $7x+y-4=0$ is & $\lambda=-\frac{1}{3}$\\
passes through (1,2) is & $\lambda=-\frac{17}{41}$\\
parallel to $x$ axis is & $\lambda=3$\\
\end{matchtabular}
\\
\item The equation of the line through the intersection of the lines $2x-3y=0$ and $4x-5y=2$ and
\begin{center}
\begin{tabular}{cccccc}
\textbf{Column $C_1$} & & & & &  \textbf{Column $C_2$}\\
\end{tabular}   
\end{center}

\begin{matchtabular}
through the point (2,1) is & $2x-y=4$\\
perpendicular to the line & $x+y-5=0$\\
parallel to the line $3x-4y+5=0$ is & $x-y-1=0$\\
equally inclined to the axes is & $3x-4y-1=0$\\
\end{matchtabular}
\end{enumerate}

\subsection{Singular Value Decomposition}
\begin{enumerate}[label=\thesection.\arabic*,ref=\thesection.\theenumi]
\item Find the shortest distance between the lines\\  $\overrightarrow{r}=(\hat{i}+2\hat{j}+\hat{k})+\lambda(\hat{i}-\hat{j}+\hat{k})$ and \\$\overrightarrow{r}=2\hat{i}-\hat{j}-\hat{k}+\mu(2\hat{i}+\hat{j}+2\hat{k})$
\item Find the shortest distance between the lines\\
$ \frac{x+1}{7}=\frac{y+1}{-6}=\frac{z+1}{1}$ and $ \frac{x-3}{1}=\frac{y-5}{-2}=\frac{z-7}{1}$ 
    \solution
%		\iffalse
\documentclass[journal,12pt,twocolumn]{IEEEtran}
\usepackage{romannum}
\usepackage{float}
\usepackage{setspace}
\usepackage{gensymb}
\singlespacing
\usepackage[cmex10]{amsmath}
\usepackage{amsthm}
\usepackage{mathrsfs}
\usepackage{txfonts}
\usepackage{stfloats}
\usepackage{bm}
\usepackage{cite}
\usepackage{cases}
\usepackage{subfig}
\usepackage{longtable}
\usepackage{multirow}
\usepackage{enumitem}
\usepackage{mathtools}
\usepackage{steinmetz}
\usepackage{tikz}
\usepackage{circuitikz}
\usepackage{verbatim}
\usepackage{tfrupee}
\usepackage[breaklinks=true]{hyperref}
\usepackage{tkz-euclide}
\usetikzlibrary{calc,math}
\usepackage{listings}
    \usepackage{color}                                            %%
    \usepackage{array}                                            %%
    \usepackage{longtable}                                        %%
    \usepackage{calc}                                             %%
    \usepackage{multirow}                                         %%
    \usepackage{hhline}                                           %%
    \usepackage{ifthen}                                           %%
  %optionally (for landscape tables embedded in another document): %%
    \usepackage{lscape}     
\usepackage{multicol}
\usepackage{chngcntr}
\DeclareMathOperator*{\Res}{Res}
\renewcommand\thesection{\arabic{section}}
\renewcommand\thesubsection{\thesection.\arabic{subsection}}
\renewcommand\thesubsubsection{\thesubsection.\arabic{subsubsection}}

\renewcommand\thesectiondis{\arabic{section}}
\renewcommand\thesubsectiondis{\thesectiondis.\arabic{subsection}}
\renewcommand\thesubsubsectiondis{\thesubsectiondis.\arabic{subsubsection}}

% correct bad hyphenation here
\hyphenation{op-tical net-works semi-conduc-tor}
\def\inputGnumericTable{}                                 %%

\lstset{
frame=single, 
breaklines=true,
columns=fullflexible
}

\begin{document}


\newtheorem{theorem}{Theorem}[section]
\newtheorem{problem}{Problem}
\newtheorem{proposition}{Proposition}[section]
\newtheorem{lemma}{Lemma}[section]
\newtheorem{corollary}[theorem]{Corollary}
\newtheorem{example}{Example}[section]
\newtheorem{definition}[problem]{Definition}
\newcommand{\BEQA}{\begin{eqnarray}}
\newcommand{\EEQA}{\end{eqnarray}}
\newcommand{\define}{\stackrel{\triangle}{=}}

\bibliographystyle{IEEEtran}
\providecommand{\mbf}{\mathbf}
\providecommand{\pr}[1]{\ensuremath{\Pr\left(#1\right)}}
\providecommand{\qfunc}[1]{\ensuremath{Q\left(#1\right)}}
\providecommand{\sbrak}[1]{\ensuremath{{}\left[#1\right]}}
\providecommand{\lsbrak}[1]{\ensuremath{{}\left[#1\right.}}
\providecommand{\rsbrak}[1]{\ensuremath{{}\left.#1\right]}}
\providecommand{\brak}[1]{\ensuremath{\left(#1\right)}}
\providecommand{\lbrak}[1]{\ensuremath{\left(#1\right.}}
\providecommand{\rbrak}[1]{\ensuremath{\left.#1\right)}}
\providecommand{\cbrak}[1]{\ensuremath{\left\{#1\right\}}}
\providecommand{\lcbrak}[1]{\ensuremath{\left\{#1\right.}}
\providecommand{\rcbrak}[1]{\ensuremath{\left.#1\right\}}}
\theoremstyle{remark}
\newtheorem{rem}{Remark}
\newcommand{\sgn}{\mathop{\mathrm{sgn}}}
\providecommand{\abs}[1]{\left\vert#1\right\vert}
\providecommand{\res}[1]{\Res\displaylimits_{#1}} 
\providecommand{\norm}[1]{\left\lVert#1\right\rVert}
\providecommand{\mtx}[1]{\mathbf{#1}}
\providecommand{\mean}[1]{E\left[ #1 \right]}
\providecommand{\fourier}{\overset{\mathcal{F}}{ \rightleftharpoons}}
\providecommand{\system}{\overset{\mathcal{H}}{ \longleftrightarrow}}
\newcommand{\solution}{\noindent \textbf{Solution: }}
\newcommand{\cosec}{\,\text{cosec}\,}
\providecommand{\dec}[2]{\ensuremath{\overset{#1}{\underset{#2}{\gtrless}}}}
\newcommand{\myvec}[1]{\ensuremath{\begin{pmatrix}#1\end{pmatrix}}}
\newcommand{\mydet}[1]{\ensuremath{\begin{vmatrix}#1\end{vmatrix}}}
\numberwithin{equation}{subsection}
\makeatletter
\@addtoreset{figure}{problem}
\makeatother

\let\StandardTheFigure\thefigure
\let\vec\mathbf
\renewcommand{\thefigure}{\theproblem}



\def\putbox#1#2#3{\makebox[0in][l]{\makebox[#1][l]{}\raisebox{\baselineskip}[0in][0in]{\raisebox{#2}[0in][0in]{#3}}}}
     \def\rightbox#1{\makebox[0in][r]{#1}}
     \def\centbox#1{\makebox[0in]{#1}}
     \def\topbox#1{\raisebox{-\baselineskip}[0in][0in]{#1}}
     \def\midbox#1{\raisebox{-0.5\baselineskip}[0in][0in]{#1}}

\vspace{3cm}


\title{Assignment 1}
\author{Jaswanth Chowdary Madala}





% make the title area
\maketitle

\newpage

%\tableofcontents

\bigskip

\renewcommand{\thefigure}{\theenumi}
\renewcommand{\thetable}{\theenumi}

\begin{enumerate}

\textbf{Solution:}
\fi
		The givne lines can be written  in vector form  as
\begin{align}
	\vec{x} &= \myvec{1\\1\\0} + \lambda_1\myvec{2\\-1\\1},
\vec{x} = \myvec{2\\1\\-1} + \lambda_2\myvec{3\\-5\\2}\\
\implies \vec{x_1} = \myvec{1\\1\\0},\, \vec{x_2} &= \myvec{2\\1\\-1}, \,\vec{m_1} = \myvec{2\\-1\\1}, \, \vec{m_2} = \myvec{3\\-5\\2}
\end{align}
%
We first check whether the given lines are skew. The lines 
\begin{align}
\vec{x} = \vec{x_1} + \lambda_1\vec{m_1},\, \vec{x} = \vec{x_2} + \lambda_2\vec{m_2} 
\label{eq:chapters/12/11/2/e11/1}
\end{align}
intersect if
\begin{align}
\vec{M}\bm{\lambda} &= \vec{x_2} - \vec{x_1}\\
\vec{M} &\triangleq \myvec{\vec{m_1} & \vec{m_2}} \\
\bm{\lambda} &\triangleq \myvec{\lambda_1\\-\lambda_2}\\
\end{align}
Here we have,
\begin{align}
\vec{M} &= \myvec{2&3\\-1&-5\\1&2},
\vec{x_2} - \vec{x_1} &= \myvec{1\\0\\-1}
\end{align}
We check whether the equation \eqref{eq:chapters/12/11/2/e11/2} has a solution
\begin{align}
\myvec{2&3\\-1&-5\\1&2}\bm{\lambda} = \myvec{1\\0\\-1}
\label{eq:chapters/12/11/2/e11/2}
\end{align}
The augmented matrix is given by,
\begin{align}
\myvec{2&3&\vrule&1\\-1&-5&\vrule&0\\1&2&\vrule&-1}
\xleftrightarrow[R_3 \leftarrow R_3 - \frac{1}{2}R_1]{R_2 \leftarrow R_2 + \frac{1}{2}R_1}
\myvec{2&3&\vrule&1\\&&\vrule\\0&-\frac{7}{2}&\vrule&\frac{1}{2}\\&&\vrule\\0&\frac{1}{2}&\vrule&-\frac{3}{2}}\\
\xleftrightarrow{R_3 \leftarrow R_3 + 7R_2}
\myvec{2&3&\vrule&1\\&&\vrule\\0&-\frac{7}{2}&\vrule&\frac{1}{2}\\&&\vrule\\0&0&\vrule&-10}
\end{align}
The rank of the matrix is 3. So the given lines are skew.
The closest points on two skew lines defined by \eqref{eq:chapters/12/11/2/e11/1} are given by 
\begin{align}
\vec{M}^\top \vec{M}\bm{\lambda} &= \vec{M}^\top\brak{\vec{x_2}-\vec{x_1}}\\
\implies \myvec{2&-1&1\\3&-5&2} \myvec{2&3\\-1&-5\\1&2}\bm{\lambda} &= \myvec{2&-1&1\\3&-5&2} \myvec{1\\0\\-1}\\
\implies \myvec{6&13\\13&38}\bm{\lambda} &= \myvec{1\\1}
\label{eq:chapters/12/11/2/e11/3}
\end{align}
The augmented matrix of the above equation \eqref{eq:chapters/12/11/2/e11/3} is given by,
\begin{align}
\myvec{6&13&\vrule&1\\13&38&\vrule&1}
\xleftrightarrow{R_2 \leftarrow R_2 - \frac{13}{6}R_1}
\myvec{6&13&\vrule&1 \\&&\vrule\\ 0&\frac{59}{6}&\vrule&-\frac{7}{6}}\\
\xleftrightarrow{R_1 \leftarrow R_1 - \frac{78}{59}R_2}
\myvec{6&0&\vrule&\frac{150}{59} \\&&\vrule\\ 0&\frac{59}{6}&\vrule&-\frac{7}{6}}
\end{align}
So, we get
\begin{align}
\myvec{\lambda_1\\-\lambda_2} &= \myvec{\frac{25}{59}\\\\-\frac{7}{59}}
\end{align}
The closest points $\vec{A}$ on line $l_1$ and $\vec{B}$ on line $l_2$ are given by,
\begin{align}
\vec{A} &= \vec{x_1} + \lambda_1\vec{m_1}
= \frac{1}{59}\myvec{109\\34\\25}\\
\vec{B} &= \vec{x_2} + \lambda_2\vec{m_2}
= \frac{1}{59}\myvec{139\\24\\-45}
\end{align}
The minimum distance between the lines is given by,
\begin{align}
\norm{\vec{B}-\vec{A}} &= \norm{\frac{1}{59}\myvec{30\\-10\\-70}}
= \frac{10}{\sqrt{59}}
\end{align}
See Fig. 
	\ref{fig:chapters/12/11/2/e11/}.
\begin{figure}[!ht]
\centering
\includegraphics[width=\columnwidth]{chapters/12/11/2/e11/figs/skew.png}
\caption{}
	\label{fig:chapters/12/11/2/e11/}
\end{figure}

    \item Find the shortest distance between the lines whose vector equations are
    \begin{align}
        \vec{x} = \myvec{1\\2\\3} + \lambda_1\myvec{1\\-3\\2}
        \label{eq:chapters/12/11/2/16/L1/svd}
    \end{align}
    and
    \begin{align}
        \vec{x} = \myvec{4\\5\\6} + \lambda_2\myvec{2\\3\\1}
        \label{eq:chapters/12/11/2/16/L2/svd}
    \end{align}
    \solution
		\iffalse
\documentclass[journal,12pt,twocolumn]{IEEEtran}
\usepackage{setspace}
\usepackage{gensymb}
\usepackage{xcolor}
\usepackage{caption}
\singlespacing
\usepackage{siunitx}
\usepackage[cmex10]{amsmath}
\usepackage{mathtools}
\usepackage{hyperref}
\usepackage{amsthm}
\usepackage{mathrsfs}
\usepackage{txfonts}
\usepackage{stfloats}
\usepackage{cite}
\usepackage{cases}
\usepackage{subfig}
\usepackage{longtable}
\usepackage{multirow}
\usepackage{enumitem}
\usepackage{bm}
\usepackage{mathtools}
\usepackage{listings}
\usepackage{tikz}
\usetikzlibrary{shapes,arrows,positioning}
\usepackage{circuitikz}
\renewcommand{\vec}[1]{\boldsymbol{\mathbf{#1}}}
\DeclareMathOperator*{\Res}{Res}
\renewcommand\thesection{\arabic{section}}
\renewcommand\thesubsection{\thesection.\arabic{subsection}}
\renewcommand\thesubsubsection{\thesubsection.\arabic{subsubsection}}

\renewcommand\thesectiondis{\arabic{section}}
\renewcommand\thesubsectiondis{\thesectiondis.\arabic{subsection}}
\renewcommand\thesubsubsectiondis{\thesubsectiondis.\arabic{subsubsection}}
\hyphenation{op-tical net-works semi-conduc-tor}

\lstset{
language=Python,
frame=single, 
breaklines=true,
columns=fullflexible
}
\begin{document}
\theoremstyle{definition}
\newtheorem{theorem}{Theorem}[section]
\newtheorem{problem}{Problem}
\newtheorem{proposition}{Proposition}[section]
\newtheorem{lemma}{Lemma}
\newtheorem{corollary}[theorem]{Corollary}
\newtheorem{example}{Example}[section]
\newtheorem{definition}{Definition}[section]
\newcommand{\BEQA}{\begin{eqnarray}}
\newcommand{\EEQA}{\end{eqnarray}}
\newcommand{\define}{\stackrel{\triangle}{=}}
\newcommand{\myvec}[1]{\ensuremath{\begin{pmatrix}#1\end{pmatrix}}}
\newcommand{\mydet}[1]{\ensuremath{\begin{vmatrix}#1\end{vmatrix}}}
\bibliographystyle{IEEEtran}
\providecommand{\nCr}[2]{\,^{#1}C_{#2}} % nCr
\providecommand{\nPr}[2]{\,^{#1}P_{#2}} % nPr
\providecommand{\mbf}{\mathbf}
\providecommand{\pr}[1]{\ensuremath{\Pr\left(#1\right)}}
\providecommand{\chr}[1]{\ensuremath{\textrm{char}\left(#1\right)}}
\providecommand{\qfunc}[1]{\ensuremath{Q\left(#1\right)}}
\providecommand{\sbrak}[1]{\ensuremath{{}\left[#1\right]}}
\providecommand{\lsbrak}[1]{\ensuremath{{}\left[#1\right.}}
\providecommand{\rsbrak}[1]{\ensuremath{{}\left.#1\right]}}
\providecommand{\brak}[1]{\ensuremath{\left(#1\right)}}
\providecommand{\lbrak}[1]{\ensuremath{\left(#1\right.}}
\providecommand{\rbrak}[1]{\ensuremath{\left.#1\right)}}
\providecommand{\cbrak}[1]{\ensuremath{\left\{#1\right\}}}
\providecommand{\lcbrak}[1]{\ensuremath{\left\{#1\right.}}
\providecommand{\rcbrak}[1]{\ensuremath{\left.#1\right\}}}
\theoremstyle{remark}
\newtheorem{rem}{Remark}
\newcommand{\sgn}{\mathop{\mathrm{sgn}}}
\newcommand{\rect}{\mathop{\mathrm{rect}}}
\newcommand{\sinc}{\mathop{\mathrm{sinc}}}
\providecommand{\abs}[1]{\left\vert#1\right\vert}
\providecommand{\res}[1]{\Res\displaylimits_{#1}} 
\providecommand{\norm}[1]{\left\Vert#1\right\Vert}
\providecommand{\mtx}[1]{\mathbf{#1}}
\providecommand{\mean}[1]{E\left[ #1 \right]}
\providecommand{\fourier}{\overset{\mathcal{F}}{ \rightleftharpoons}}
\providecommand{\ztrans}{\overset{\mathcal{Z}}{ \rightleftharpoons}}
\providecommand{\system}[1]{\overset{\mathcal{#1}}{ \longleftrightarrow}}
\newcommand{\solution}{\noindent \textbf{Solution: }}
\providecommand{\dec}[2]{\ensuremath{\overset{#1}{\underset{#2}{\gtrless}}}}
\let\StandardTheFigure\thefigure
\def\putbox#1#2#3{\makebox[0in][l]{\makebox[#1][l]{}\raisebox{\baselineskip}[0in][0in]{\raisebox{#2}[0in][0in]{#3}}}}
     \def\rightbox#1{\makebox[0in][r]{#1}}
     \def\centbox#1{\makebox[0in]{#1}}
     \def\topbox#1{\raisebox{-\baselineskip}[0in][0in]{#1}}
     \def\midbox#1{\raisebox{-0.5\baselineskip}[0in][0in]{#1}}

\vspace{3cm}
\title{Line Assignment}
\author{Gautam Singh}
\maketitle
\bigskip

\begin{abstract}
    This document contains a general solution to Question 16 of 
    Exercise 2 in Chapter 11 of the class 12 NCERT textbook.
\end{abstract}

\begin{enumerate}
    \item Find the shortest distance between the lines whose vector equations are
    \begin{align}
        L_1: \vec{x} = \vec{x_1} + \lambda_1\vec{m_1} \label{eq:chapters/12/11/2/16/svd/L1} \\
        L_2: \vec{x} = \vec{x_2} + \lambda_2\vec{m_2} \label{eq:chapters/12/11/2/16/svd/L2}
    \end{align}

    \solution 
\fi
		Let $\vec{A}$ and $\vec{B}$ be points on lines $L_1$ and $L_2$
    respectively such that $AB$ is normal to both lines. Define
    \begin{align}
        \vec{M} &\triangleq \myvec{\vec{m_1} & \vec{m_2}} \label{eq:chapters/12/11/2/16/svd/M-def} \\
        \vec{\lambda} &\triangleq \myvec{\lambda_1\\-\lambda_2} \label{eq:chapters/12/11/2/16/svd/lambda-def} \\
        \vec{x} &\triangleq \vec{x_2} - \vec{x_1} \label{eq:chapters/12/11/2/16/svd/x-def}
    \end{align}
    Then, we have the following equations:
    \begin{align}
        \vec{A} = \vec{x_1} + \lambda_1\vec{m_1} \label{eq:chapters/12/11/2/16/svd/A-def} \\
        \vec{B} = \vec{x_2} + \lambda_2\vec{m_2} \label{eq:chapters/12/11/2/16/svd/B-def}
    \end{align}
    From \eqref{eq:chapters/12/11/2/16/svd/A-def} and \eqref{eq:chapters/12/11/2/16/svd/B-def}, define the real-valued function
    $f$ as
    \begin{align}
        f\brak{\vec{\lambda}} &\triangleq \norm{\vec{A}-\vec{B}}^2 \\
                              &= \norm{\vec{M}\vec{\lambda}-\vec{x}}^2 \\
                              &= \brak{\vec{M\lambda}-\vec{x}}^\top\brak{\vec{M\lambda}-\vec{x}} \\
                              &= \vec{\lambda}^\top\brak{\vec{M}^\top\vec{M}}\vec{\lambda} - 2\vec{x}^\top\vec{M\lambda} + \norm{\vec{x}}^2
        \label{eq:chapters/12/11/2/16/svd/f-def}
    \end{align}
    From \eqref{eq:chapters/12/11/2/16/svd/f-def}, we see that $f$ is quadratic in $\vec{\lambda}$.

    We now prove a useful lemma here.
    \begin{lemma}
        The quadratic form
        \begin{align}
            q\brak{\vec{x}} \triangleq \vec{x}^\top\vec{Ax} + \vec{b}^\top\vec{x} + c
            \label{eq:chapters/12/11/2/16/svd/quad-x}
        \end{align}
        is convex iff $\vec{A}$ is positive semi-definite.
    \end{lemma}
    \begin{proof}
        Consider two points $\vec{x_1}$ and $\vec{x_2}$, and a real constant
        $0 \le \mu \le 1$. Then,
        \begin{align}
            &\mu f\brak{\vec{x_1}} + \brak{1-\mu}f\brak{\vec{x_2}} - f\brak{\mu\vec{x_1}+\brak{1-\mu}\vec{x_2}} \nonumber \\
            &= \brak{\mu-\mu^2}\vec{x_1}^\top\vec{Ax_1} + \brak{1-\mu-\brak{1-\mu}^2}\vec{x_2}^\top\vec{Ax_2} \nonumber \\
            &- 2\mu\brak{1-\mu}\vec{x_1}^\top\vec{Ax_2} \\
            &= \mu\brak{1-\mu}\brak{\vec{x_1}^\top\vec{Ax_1}-2\vec{x_1}^\top\vec{Ax_2}+\vec{x_2}^\top\vec{Ax_2}} \\
            &= \mu\brak{1-\mu}\brak{\vec{x_1}-\vec{x_2}}^\top\vec{A}\brak{\vec{x_1}-\vec{x_2}}
            \label{eq:chapters/12/11/2/16/svd/psd-iff}
        \end{align}
        Since $\vec{x_1}$ and $\vec{x_2}$ are arbitrary, it follows from 
        \eqref{eq:chapters/12/11/2/16/svd/psd-iff} that
        \begin{align}
            \mu f\brak{\vec{x_1}} + \brak{1-\mu}f\brak{\vec{x_2}} \ge f\brak{\mu\vec{x_1}+\brak{1-\mu}\vec{x_2}}
        \end{align}
        iff $\vec{A}$ is positive semi-definite, as required.
    \end{proof}
    Using the above lemma, we show that $f$ is convex by showing that 
    $\vec{M}^\top\vec{M}$ is positive semi-definite. Indeed, for any 
    $\vec{p} \triangleq \myvec{x\\y}$,
    \begin{align}
        \vec{p}^\top\vec{M}^\top\vec{Mp} = \norm{\vec{Mp}}^2 \ge 0
        \label{eq:chapters/12/11/2/16/svd/psd}
    \end{align}
    and thus, $f$ is convex.

    We need to minimize $f$ as a function of $\vec{\lambda}$.
    Differentiating \eqref{eq:chapters/12/11/2/16/svd/f-def} using the chain rule,
    \begin{align}
        \frac{df\brak{\vec{\lambda}}}{d\vec{\lambda}} &= \vec{M}^\top\brak{\vec{M\lambda}-\vec{x}}+\vec{M}\brak{\vec{M\lambda}-\vec{x}}^\top \\
                                                      &= 2\vec{M}^\top\brak{\vec{M\lambda}-\vec{x}}
        \label{eq:chapters/12/11/2/16/svd/vec-min}
    \end{align}
    Setting \eqref{eq:chapters/12/11/2/16/svd/vec-min} to zero gives the equation
    \begin{align}
        \vec{M}^\top\vec{M\lambda} = \vec{M}^\top\vec{x}
        \label{eq:chapters/12/11/2/16/svd/vec-eqn}
    \end{align}
    We use singular value decomposition here. Let
    \begin{align}
        \vec{M} = \vec{U\Sigma V}^\top
        \label{eq:chapters/12/11/2/16/svd/M-svd}
    \end{align}
    where $\vec{U}, \vec{V}$ are orthogonal and $\vec{\Sigma}$ is diagonal with
    nonnegative diagonal entries. Substituting in \eqref{eq:chapters/12/11/2/16/svd/vec-eqn},
    \begin{align}
        \vec{V\Sigma U}^\top\vec{U\Sigma V}^\top\vec{\lambda} = \vec{V\Sigma U}^\top\vec{x} \\
        \implies \vec{V\Sigma}^2\vec{V}^\top\vec{\lambda} = \vec{V\Sigma U}^\top\vec{x} \\
        \implies \vec{\lambda} = \brak{\vec{V\Sigma}^2\vec{V}^\top}^{-1}\vec{V\Sigma U}^\top\vec{x} \\
        \implies \vec{\lambda} = \vec{V\Sigma}^{-2}\vec{V}^\top\vec{V\Sigma U}^\top\vec{x} \\
        \implies \vec{\lambda} = \vec{V\Sigma}^{-1}\vec{U}^\top\vec{x} \label{eq:chapters/12/11/2/16/svd/lambda-sol}
    \end{align}
    where $\vec{\Sigma}^{-1}$ is obtained by inverting the nonzero elements of
    $\vec{\Sigma}$. Thus, the shortest distance is given by using \eqref{eq:chapters/12/11/2/16/svd/M-svd}
    and \eqref{eq:chapters/12/11/2/16/svd/lambda-sol} in \eqref{eq:chapters/12/11/2/16/svd/f-def}, and is given by
    \begin{align}
        d = \norm{\brak{\vec{U}\brak{\vec{\Sigma\Sigma}^{-1}}\vec{U}^\top-\vec{I}}\vec{x}}
        \label{eq:chapters/12/11/2/16/svd/min-sol}
    \end{align}
    For this problem,
    \begin{align}
        \vec{x} = \vec{x_2} - \vec{x_1} = \myvec{3\\3\\3} \\
        \vec{M} = \myvec{\vec{m_1} & \vec{m_2}} = \myvec{1&2\\-3&3\\2&1} 
    \end{align}
    Thus,
    \begin{align}
        \vec{M}^\top\vec{M} = \myvec{1&-3&2\\2&3&1}\myvec{1&2\\-3&3\\2&1} = \myvec{14&-5\\-5&14} \\
        \vec{MM}^\top = \myvec{1&2\\-3&3\\2&1}\myvec{1&-3&2\\2&3&1} = \myvec{5&3&4\\3&18&-3\\4&-3&5}
    \end{align}
    We perform the eigendecompositions for each matrix and bring them into the form
    \begin{align}
        \vec{MM}^\top &= \vec{P_1D_1P_1}^\top \label{eq:chapters/12/11/2/16/svd/decomp-1} \\
        \vec{M}^\top\vec{M} &= \vec{P_2D_2P_2}^\top \label{eq:chapters/12/11/2/16/svd/decomp-2}
    \end{align}
    \begin{enumerate}
        \item For $\vec{MM}^\top$, the characteristic polynomial is
        \begin{align}
		\text{char}{\vec{MM}^\top} &= \mydet{x-5&-3&-4\\-3&x-18&3\\-4&3&x-5} \\
                                      &= x\brak{x-9}\brak{x-19}
                                      \label{eq:chapters/12/11/2/16/svd/char-2}
        \end{align}
        Thus, the eigenvalues are given by
        \begin{align}
            \lambda_1 = 19,\ \lambda_2 = 9,\ \lambda_3 = 0
        \end{align}
        For $\lambda_1$, the augmented matrix formed from the 
        eigenvalue-eigenvector equation is
        \begin{align}
            &\myvec{-14&3&4&0\\3&-1&-3&0\\4&-3&-14&0} \nonumber \\
            &\xleftrightarrow[]{R_1 \leftarrow \frac{R_1+R_3}{-10}} \myvec{1&0&1&0\\3&-1&-3&0\\4&-3&-14&0} \\
            &\xleftrightarrow[]{R_2 \leftarrow -R_2+3R_1} \myvec{1&0&1&0\\0&1&6&0\\4&-3&-14&0} \\
            &\xleftrightarrow[]{R_3 \leftarrow R_3-4R_1} \myvec{1&0&1&0\\0&-1&-6&0\\0&-3&-18&0} \\
            &\xleftrightarrow[]{R_3 \leftarrow R_3-3R_2} \myvec{1&0&1&0\\0&-1&-6&0\\0&0&0&0}
        \end{align}
        Hence, the normalized eigenvector is
        \begin{align}
            \vec{p_1} = \frac{1}{\sqrt{38}}\myvec{-1\\-6\\1}
        \end{align}
        For $\lambda_2$, the augmented matrix formed from the 
        eigenvalue-eigenvector equation is
        \begin{align}
            &\myvec{-4&3&4&0\\3&9&-3&0\\4&3&-4&0} \nonumber \\
            &\xleftrightarrow[]{R_3 \leftarrow R_1+R_3} \myvec{-4&3&4&0\\3&9&-3&0\\0&0&0&0} \\
            &\xleftrightarrow[]{R_2 \leftarrow \frac{4R_2+3R_1}{45}} \myvec{-4&3&4&0\\0&1&0&0\\0&0&0&0} \\
            &\xleftrightarrow[]{R_1 \leftarrow \frac{R_1-3R_2}{-4}} \myvec{1&0&-1&0\\0&1&0&0\\0&0&0&0}
        \end{align}
        Hence, the normalized eigenvector is
        \begin{align}
            \vec{p_2} = \frac{1}{\sqrt{2}}\myvec{1\\0\\1}
        \end{align}
        For $\lambda_3$, the augmented matrix formed from the 
        eigenvalue-eigenvector equation is
        \begin{align}
            &\myvec{5&3&4&0\\3&18&-3&0\\4&-3&5&0} \nonumber \\ 
            &\xleftrightarrow[]{R_1 \leftarrow \frac{R_1+R_3}{9}} \myvec{1&0&1&0\\3&18&-3&0\\4&-3&5&0} \\
            &\xleftrightarrow[]{R_2 \leftarrow R_2-3R_1} \myvec{1&0&1&0\\0&18&-6&0\\4&-3&5&0} \\
            &\xleftrightarrow[]{R_3 \leftarrow R_3-4R_1} \myvec{1&0&1&0\\0&18&-6&0\\0&-3&1&0} \\
            &\xleftrightarrow[]{R_2 \leftarrow \frac{R_2}{6}} \myvec{1&0&1&0\\0&3&-1&0\\0&-3&1&0} \\
            &\xleftrightarrow[]{R_3 \leftarrow R_3+R_2} \myvec{1&0&1&0\\0&3&-1&0\\0&0&0&0}
        \end{align}
        Hence, the normalized eigenvector is
        \begin{align}
            \vec{p_3} = \frac{1}{\sqrt{19}}\myvec{-3\\1\\3}
        \end{align}
        Using \eqref{eq:chapters/12/11/2/16/svd/decomp-1}, we see that
        \begin{align}
            \vec{P_1} = \myvec{-\frac{1}{\sqrt{38}}&\frac{1}{\sqrt{2}}&-\frac{3}{\sqrt{19}}\\-\frac{6}{\sqrt{38}}&0&\frac{1}{\sqrt{19}}\\\frac{1}{\sqrt{38}}&-\frac{1}{\sqrt{2}}&\frac{3}{\sqrt{19}}} \\
            \vec{D_1} = \myvec{19&0&0\\0&9&0\\0&0&0}
            \label{eq:chapters/12/11/2/16/svd/eig-params-1}
        \end{align}
        \item For $\vec{M}^\top\vec{M}$, the characteristic polynomial is
        \begin{align}
		\text{char}{\vec{M}^\top\vec{M}} &= \mydet{x-14&5\\5&x-14} \\
                                      &= \brak{x-9}\brak{x-19}
%                                      \label{eq:chapters/12/11/2/16/svd/char-1}
        \end{align}
        Thus, the eigenvalues are given by
        \begin{align}
            \lambda_1 = 19,\ \lambda_2 = 9
        \end{align}
        For $\lambda_1$, the augmented matrix formed from the 
        eigenvalue-eigenvector equation is
        \begin{align}
            \myvec{-5&-5&0\\-5&-5&0} &\xleftrightarrow[]{R_1 \leftarrow R_1-R_2} \myvec{0&0&0\\-5&-5&0}
        \end{align}
        Hence, the normalized eigenvector is
        \begin{align}
            \vec{p_1} = \frac{1}{\sqrt{2}}\myvec{1\\-1}
        \end{align}
        For $\lambda_2$, the augmented matrix formed from the 
        eigenvalue-eigenvector equation is
        \begin{align}
            \myvec{5&-5&0\\-5&5&0} &\xleftrightarrow[]{R_1 \leftarrow R_1+R_2} \myvec{0&0&0\\5&-5&0}
        \end{align}
        Hence, the normalized eigenvector is
        \begin{align}
            \vec{p_2} = \frac{1}{\sqrt{2}}\myvec{1\\1}
        \end{align}
        Thus, from \eqref{eq:chapters/12/11/2/16/svd/decomp-2},
        \begin{align}
            \vec{P_2} = \myvec{\frac{1}{\sqrt{2}}&-\frac{1}{\sqrt{2}}\\\frac{1}{\sqrt{2}}&\frac{1}{\sqrt{2}}},\ \vec{D_2} = \myvec{9&0\\0&19}
            \label{eq:chapters/12/11/2/16/svd/eig-params-2}
        \end{align}
    \end{enumerate}
    Therefore, from \eqref{eq:chapters/12/11/2/16/svd/M-svd},
    \begin{align}
        \vec{U} &= \vec{P_1} \\ 
        \vec{V} &= \vec{P_2} \\
        \vec{\Sigma} &= \myvec{\sqrt{19}&0\\0&3\\0&0}
        \label{eq:chapters/12/11/2/16/svd/svd-params}
    \end{align}
    and substituting into \eqref{eq:chapters/12/11/2/16/svd/lambda-sol}, we get
    \begin{align}
        \vec{\lambda} = \frac{1}{19}\myvec{10\\28}
    \end{align}
    which agrees with earlier solutions as well.
    \iffalse
    The Python code
    \texttt{codes/svd.py} plots 
    \fi
    See Fig. \ref{fig:chapters/12/11/2/16/svd/svd} depicting the situation.
    \begin{figure}[!ht]
        \centering
        \includegraphics[width=\columnwidth]{chapters/12/11/2/16/svd/figs/skew_svd.png}
        \caption{Finding the shortest distance between two lines using SVD.}
        \label{fig:chapters/12/11/2/16/svd/svd}
    \end{figure}

\item Find the shortest distance between the lines $l_1$ and $l_2$ whose vector equations are ${\overrightarrow{r} = \hat{i}+\hat{j}+\lambda(2\hat{i}-\hat{j}+\hat{k})}$ and ${\overrightarrow{r} = 2\hat{i}+\hat{j}-\hat{k}+\mu(3\hat{i}-5\hat{j}+2\hat{k})}$.
\\
    \solution
		\iffalse
\documentclass[journal,12pt,twocolumn]{IEEEtran}
\usepackage{romannum}
\usepackage{float}
\usepackage{setspace}
\usepackage{gensymb}
\singlespacing
\usepackage[cmex10]{amsmath}
\usepackage{amsthm}
\usepackage{mathrsfs}
\usepackage{txfonts}
\usepackage{stfloats}
\usepackage{bm}
\usepackage{cite}
\usepackage{cases}
\usepackage{subfig}
\usepackage{longtable}
\usepackage{multirow}
\usepackage{enumitem}
\usepackage{mathtools}
\usepackage{steinmetz}
\usepackage{tikz}
\usepackage{circuitikz}
\usepackage{verbatim}
\usepackage{tfrupee}
\usepackage[breaklinks=true]{hyperref}
\usepackage{tkz-euclide}
\usetikzlibrary{calc,math}
\usepackage{listings}
    \usepackage{color}                                            %%
    \usepackage{array}                                            %%
    \usepackage{longtable}                                        %%
    \usepackage{calc}                                             %%
    \usepackage{multirow}                                         %%
    \usepackage{hhline}                                           %%
    \usepackage{ifthen}                                           %%
  %optionally (for landscape tables embedded in another document): %%
    \usepackage{lscape}     
\usepackage{multicol}
\usepackage{chngcntr}
\DeclareMathOperator*{\Res}{Res}
\renewcommand\thesection{\arabic{section}}
\renewcommand\thesubsection{\thesection.\arabic{subsection}}
\renewcommand\thesubsubsection{\thesubsection.\arabic{subsubsection}}

\renewcommand\thesectiondis{\arabic{section}}
\renewcommand\thesubsectiondis{\thesectiondis.\arabic{subsection}}
\renewcommand\thesubsubsectiondis{\thesubsectiondis.\arabic{subsubsection}}

% correct bad hyphenation here
\hyphenation{op-tical net-works semi-conduc-tor}
\def\inputGnumericTable{}                                 %%

\lstset{
frame=single, 
breaklines=true,
columns=fullflexible
}

\begin{document}


\newtheorem{theorem}{Theorem}[section]
\newtheorem{problem}{Problem}
\newtheorem{proposition}{Proposition}[section]
\newtheorem{lemma}{Lemma}[section]
\newtheorem{corollary}[theorem]{Corollary}
\newtheorem{example}{Example}[section]
\newtheorem{definition}[problem]{Definition}
\newcommand{\BEQA}{\begin{eqnarray}}
\newcommand{\EEQA}{\end{eqnarray}}
\newcommand{\define}{\stackrel{\triangle}{=}}

\bibliographystyle{IEEEtran}
\providecommand{\mbf}{\mathbf}
\providecommand{\pr}[1]{\ensuremath{\Pr\left(#1\right)}}
\providecommand{\qfunc}[1]{\ensuremath{Q\left(#1\right)}}
\providecommand{\sbrak}[1]{\ensuremath{{}\left[#1\right]}}
\providecommand{\lsbrak}[1]{\ensuremath{{}\left[#1\right.}}
\providecommand{\rsbrak}[1]{\ensuremath{{}\left.#1\right]}}
\providecommand{\brak}[1]{\ensuremath{\left(#1\right)}}
\providecommand{\lbrak}[1]{\ensuremath{\left(#1\right.}}
\providecommand{\rbrak}[1]{\ensuremath{\left.#1\right)}}
\providecommand{\cbrak}[1]{\ensuremath{\left\{#1\right\}}}
\providecommand{\lcbrak}[1]{\ensuremath{\left\{#1\right.}}
\providecommand{\rcbrak}[1]{\ensuremath{\left.#1\right\}}}
\theoremstyle{remark}
\newtheorem{rem}{Remark}
\newcommand{\sgn}{\mathop{\mathrm{sgn}}}
\providecommand{\abs}[1]{\left\vert#1\right\vert}
\providecommand{\res}[1]{\Res\displaylimits_{#1}} 
\providecommand{\norm}[1]{\left\lVert#1\right\rVert}
\providecommand{\mtx}[1]{\mathbf{#1}}
\providecommand{\mean}[1]{E\left[ #1 \right]}
\providecommand{\fourier}{\overset{\mathcal{F}}{ \rightleftharpoons}}
\providecommand{\system}{\overset{\mathcal{H}}{ \longleftrightarrow}}
\newcommand{\solution}{\noindent \textbf{Solution: }}
\newcommand{\cosec}{\,\text{cosec}\,}
\providecommand{\dec}[2]{\ensuremath{\overset{#1}{\underset{#2}{\gtrless}}}}
\newcommand{\myvec}[1]{\ensuremath{\begin{pmatrix}#1\end{pmatrix}}}
\newcommand{\mydet}[1]{\ensuremath{\begin{vmatrix}#1\end{vmatrix}}}
\numberwithin{equation}{subsection}
\makeatletter
\@addtoreset{figure}{problem}
\makeatother

\let\StandardTheFigure\thefigure
\let\vec\mathbf
\renewcommand{\thefigure}{\theproblem}



\def\putbox#1#2#3{\makebox[0in][l]{\makebox[#1][l]{}\raisebox{\baselineskip}[0in][0in]{\raisebox{#2}[0in][0in]{#3}}}}
     \def\rightbox#1{\makebox[0in][r]{#1}}
     \def\centbox#1{\makebox[0in]{#1}}
     \def\topbox#1{\raisebox{-\baselineskip}[0in][0in]{#1}}
     \def\midbox#1{\raisebox{-0.5\baselineskip}[0in][0in]{#1}}

\vspace{3cm}


\title{Assignment 1}
\author{Jaswanth Chowdary Madala}





% make the title area
\maketitle

\newpage

%\tableofcontents

\bigskip

\renewcommand{\thefigure}{\theenumi}
\renewcommand{\thetable}{\theenumi}

\begin{enumerate}
\item Find the shortest distance between the lines $l_1$ and $l_2$ whose vector equations are ${\overrightarrow{r} = \hat{i}+\hat{j}+\lambda(2\hat{i}-\hat{j}+\hat{k})}$ and ${\overrightarrow{r} = 2\hat{i}+\hat{j}-\hat{k}+\mu(3\hat{i}-5\hat{j}+2\hat{k})}$

\textbf{Solution:} 
\fi
The shortest distance between the lines whose vector equations are
\begin{align}
L_1: \vec{x} = \vec{x_1} + \lambda_1\vec{m_1} \label{eq:chapters/12/11/2/311/svd/1} \\
L_2: \vec{x} = \vec{x_2} + \lambda_2\vec{m_2} \label{eq:chapters/12/11/2/311/svd/2}
\end{align}
is given by,
\begin{align}
d = \norm{\brak{\vec{U}\brak{\vec{\Sigma\Sigma}^{-1}}\vec{U}^\top-\vec{I}}\vec{x}}
\end{align}
with the parameter $\lambda$ given by
\begin{align}
\bm{\lambda} = \vec{V\Sigma}^{-1}\vec{U}^\top\vec{x} \label{eq:chapters/12/11/2/311/svd/lambda-sol}
\end{align}
where
\begin{align}
\vec{M} &\triangleq \myvec{\vec{m_1} & \vec{m_2}} \label{eq:chapters/12/11/2/311/svd/3}\\
\bm{\lambda} &\triangleq \myvec{\lambda_1\\-\lambda_2}\label{eq:chapters/12/11/2/311/svd/4} \\
\vec{x} &\triangleq \vec{x_2} - \vec{x_1} \label{eq:chapters/12/11/2/311/svd/5}
\end{align}

We use singular value decomposition of the matrix $\vec{M}$
\begin{align}
\vec{M} = \vec{U\Sigma V}^\top \label{eq:chapters/12/11/2/311/svd/6}
\end{align}
where $\vec{U}, \vec{V}$ are orthogonal and $\vec{\Sigma}$ is diagonal with nonnegative diagonal entries.

\begin{enumerate}
\item In this problem we have the lines $l_1$ and $l_2$ as
\begin{align}
\vec{x} &= \myvec{1\\1\\0} + \lambda_1\myvec{2\\-1\\1}\\
\vec{x} &= \myvec{2\\1\\-1} + \lambda_2\myvec{3\\-5\\2}
\end{align}

We first need to check whether the given lines are skew.
The lines \eqref{eq:chapters/12/11/2/311/svd/1}, \eqref{eq:chapters/12/11/2/311/svd/2} intersect if
\begin{align}
\vec{M}\bm{\lambda} &= \vec{x_2} - \vec{x_1}
\end{align}
Here we have,
\begin{align}
\vec{M} &= \myvec{2&3\\-1&-5\\1&2} \label{eq:chapters/12/11/2/311/svd/M}\\
\vec{x} = \vec{x_2} - \vec{x_1} &= \myvec{1\\0\\-1} \label{eq:chapters/12/11/2/311/svd/x}
\end{align}
We check whether the equation \eqref{eq:chapters/12/11/2/311/svd/7} has a solution
\begin{align}
\myvec{2&3\\-1&-5\\1&2}\bm{\lambda} = \myvec{1\\0\\-1}
\label{eq:chapters/12/11/2/311/svd/7}
\end{align}
the augmented matrix is given by,
\begin{align}
&\myvec{2&3&\vrule&1\\-1&-5&\vrule&0\\1&2&\vrule&-1}\\
\xleftrightarrow[R_3 \leftarrow R_3 - \frac{1}{2}R_1]{R_2 \leftarrow R_2 + \frac{1}{2}R_1}
&\myvec{2&3&\vrule&1\\&&\vrule\\0&-\frac{7}{2}&\vrule&\frac{1}{2}\\&&\vrule\\0&\frac{1}{2}&\vrule&-\frac{3}{2}}\\
\xleftrightarrow{R_3 \leftarrow R_3 + 7R_2}
&\myvec{2&3&\vrule&1\\&&\vrule\\0&-\frac{7}{2}&\vrule&\frac{1}{2}\\&&\vrule\\0&0&\vrule&-10}
\end{align}
The rank of the matrix is 3. So the given lines are skew.

\item From \eqref{eq:chapters/12/11/2/311/svd/M} we have
\begin{align}
\vec{M}^\top\vec{M} &= \myvec{2&-1&1\\3&-5&2}\myvec{2&3\\-1&-5\\1&2} \\ 
&= \myvec{6&13\\13&38} \label{eq:chapters/12/11/2/311/svd/MtM}
\end{align}
\begin{align}
\vec{MM}^\top &= \myvec{2&3\\-1&-5\\1&2}\myvec{2&-1&1\\3&-5&2}\\
&= \myvec{13&-17&8\\-17&26&-11\\8&-11&5} \label{eq:chapters/12/11/2/311/svd/MMt}
\end{align}
We perform the eigen decompositions for the matrics \eqref{eq:chapters/12/11/2/311/svd/MMt}, \eqref{eq:chapters/12/11/2/311/svd/MtM} and write them in the form
\begin{align}
    \vec{MM}^\top &= \vec{P_1D_1P_1}^\top \label{eq:chapters/12/11/2/311/svd/decomp-1} \\
    \vec{M}^\top\vec{M} &= \vec{P_2D_2P_2}^\top \label{eq:chapters/12/11/2/311/svd/decomp-2}
\end{align}
The characteristic polynomial of the matrix $\vec{MM}^\top$ is given by,
\begin{align}
\text{char}\brak{\vec{MM}^\top} &= \mydet{13-x&-17&8\\-17&26-x&-11\\8&-11&5-x} \\
&= -x^3 + 44x^2-59x
%\label{eq:chapters/12/11/2/311/svd/char-1}
\end{align}
Thus, the eigenvalues are given by
\begin{align}
\lambda_1 = 22+5\sqrt{17},\ \lambda_2 = 22-5\sqrt{17},\ \lambda_3 = 0
\end{align}
From the augmented matrix formed from the eigen value - eigen vector equation we get, the normalized eigen vectors as
\begin{align}
    \vec{p_1} &= \frac{\sqrt{5}}{\sqrt{68-6\sqrt{17}}} \myvec{\frac{12-\sqrt{17}}{5}\\\frac{1-3\sqrt{17}}{5}\\1}\\
    \vec{p_2} &= \frac{\sqrt{5}}{\sqrt{68+6\sqrt{17}}} \myvec{\frac{12+\sqrt{17}}{5}\\\frac{1+3\sqrt{17}}{5}\\1}\\
    \vec{p_3} &= \frac{1}{\sqrt{59}}\myvec{-3\\1\\7}
\end{align}
where $\vec{p_1},\vec{p_2},\vec{p_3}$ corresponds to the  eigen values $\lambda_1, \lambda_2, \lambda_3$ respectively. Using \eqref{eq:chapters/12/11/2/311/svd/decomp-1}, we get
\begin{align}
    \vec{P_1} &= \myvec{\frac{12-\sqrt{17}}{\sqrt{5}\sqrt{68-6\sqrt{17}}} & \frac{12+\sqrt{17}}{\sqrt{5}\sqrt{68+6\sqrt{17}}} & -\frac{3}{\sqrt{59}}\\
    \frac{1-3\sqrt{17}}{\sqrt{5}\sqrt{68-6\sqrt{17}}}&\frac{1+3\sqrt{17}}{\sqrt{5}\sqrt{68+6\sqrt{17}}} & \frac{1}{\sqrt{59}}\\
\frac{\sqrt{5}}{\sqrt{68-6\sqrt{17}}}&\frac{\sqrt{5}}{\sqrt{68+6\sqrt{17}}} & \frac{7}{\sqrt{59}} }
    \label{eq:chapters/12/11/2/311/svd/eig-params-1(a)}
\end{align}
\begin{align}
    \vec{D_1} &= \myvec{22+5\sqrt{17}&0&0\\0&22-5\sqrt{17}&0\\0&0&0}
    \label{eq:chapters/12/11/2/311/svd/eig-params-1(b)}
\end{align}

For $\vec{M}^\top\vec{M}$, the characteristic polynomial is
\begin{align}
    \text{char}\brak{\vec{M}^\top\vec{M}} &= \mydet{6-x&13\\13&38-x} \\&= x^2-44x+59
    \label{eq:chapters/12/11/2/311/svd/char-1}
\end{align}
Thus, the eigenvalues are given by
\begin{align}
    \lambda_1 = 22+5\sqrt{17},\ \lambda_2 = 22-5\sqrt{17}
\end{align}
From the augmented matrix formed from the eigen value - eigen vector equation we get, the normalized eigen vectors as
\begin{align}
\vec{p_1} &= \frac{13}{\sqrt{850-160\sqrt{17}}}\myvec{\frac{-16+5\sqrt{17}}{13}\\1}\\
\vec{p_2} &= \frac{13}{\sqrt{850+160\sqrt{17}}} \myvec{\frac{-16-5\sqrt{17}}{13}\\1}
\end{align}
where $\vec{p_1},\vec{p_2}$ corresponds to the  eigen values $\lambda_1, \lambda_2$ respectively. Using \eqref{eq:chapters/12/11/2/311/svd/decomp-2}, we get
\begin{align}
    \vec{P_2} &= \myvec{\frac{-16-5\sqrt{17}}{\sqrt{850+160\sqrt{17}}}&\frac{13}{\sqrt{850-160\sqrt{17}}}\\\frac{13}{\sqrt{850+160\sqrt{17}}}&\frac{-16+5\sqrt{17}}{\sqrt{850-160\sqrt{17}}}}
     \label{eq:chapters/12/11/2/311/svd/eig-params-2(a)}\\ 
    \vec{D_2} &= \myvec{22-5\sqrt{17}&0\\0&22+5\sqrt{17}}
    \label{eq:chapters/12/11/2/311/svd/eig-params-2(b)}
\end{align}
Therefore, from \eqref{eq:chapters/12/11/2/311/svd/6} we have
\begin{align}
    \vec{U} &= \vec{P_1} \\ 
    \vec{V} &= \vec{P_2} \\
    \vec{\Sigma} &= \myvec{\sqrt{22+5\sqrt{17}}&0\\0&\sqrt{22-5\sqrt{17}}\\0&0}
    \label{eq:chapters/12/11/2/311/svd/svd-params}
\end{align}
and substituting into \eqref{eq:chapters/12/11/2/311/svd/lambda-sol}, we get
\begin{align}
    \bm{\lambda} =  \myvec{\frac{25}{59}\\\\-\frac{7}{59}}
\end{align}
The minimum distance between the lines is given by,
\begin{align}
\norm{\vec{B}-\vec{A}} &= \norm{\frac{1}{59}\myvec{30\\-10\\-70}}\\
&= \frac{\sqrt{30^2+10^2+70^2}}{59}\\
&= \frac{10}{\sqrt{59}}
\end{align}
The shortest distance between the given lines is $\frac{10}{\sqrt{59}}$ units.
See Fig. 
	\ref{fig:chapters/12/11/2/e11/svd/}.
\begin{figure}[!ht]
\centering
\includegraphics[width=\columnwidth]{./chapters/12/11/2/e11/svd/figs/skew.png}
\caption{$AB$ is the required shortest distance.}
	\label{fig:chapters/12/11/2/e11/svd/}
\end{figure}
\end{enumerate} 

\item Find the shortest distance between the lines given by $\overrightarrow{r}=(8+3\lambda\hat{i}-(9+16\lambda)\hat{j}+(10+7\lambda)\hat{k}$ and $\overrightarrow{r}=15\hat{i}+29\hat{j}+5\hat{k}+\mu(3\hat{i}+8\hat{j}-5\hat{k}).$
\end{enumerate}

%
%\subsection{JEE}
%\iffalse
\documentclass[12pt]{article}
\usepackage{graphicx}
\usepackage{enumerate}
\usepackage{amsmath}
\usepackage{ragged2e}
\newcommand{\myvec}[1]{\ensuremath{\begin{pmatrix}#1\end{pmatrix}}}
\newcommand{\mydet}[1]{\ensuremath{\begin{vmatrix}#1\end{vmatrix}}}
\providecommand{\brak}[1]{\ensuremath{\left(#1\right)}}
\providecommand{\lbrak}[1]{\ensuremath{\left(#1\right.}}
\providecommand{\rbrak}[1]{\ensuremath{\left(#1\right)}}
\providecommand{\sbrak}[1]{\ensuremath{{}\left[#1\right]}}
\let\vec\mathbf

\begin{document}
\begin{center}
\textbf\large{CHAPTER-7 \\ Straight Line and Pair of Straight Lines}

\end{center}


\section*{Section-A    [JEE Advanced/IIT-JEE]}
\section*{A    :  Fill in the Blanks}
\fi
\begin{enumerate}

\item  The area enclosed with in the curves $|x|+|y|$ is.........  (1981) \\
\item  $y=10^x$ is the reflection of $y=\log_{10}^x $ in the line whose equation is.........(1982)\\
\item The set of lines $ax+by+c=0$ where $3a+2b+4c=0$ is concurrent at the point..........(1982)\\
\item  Given the points $\vec{A}(0,4)$ and $\vec{B}(0,-4)$, the equation of the locus of the point $\vec{P}(x,y)$ such that $\abs{AP-BP}=6$ is.............(1983)\\
\item  If $a,b$ and $c$ are in A.P.. then the straight line $ax+by+c=0$ will always pass through a fixed point whose coordinates are........(1984)\\
\item  The orthocenter of the triangle formed by the lines $x+y=1, 2x+3y=6$  and  $4x-y+4=0$ lies in quadrant number............(1985)\\
\item Let the algebraic sum of the perpendicular distances from the points (2,0), (0,2) and (1,1) to a variable straight line be zero; then the line passes through a fixed point whose cordinates are.........(1991)\\
\item The vertices of a triangle are $\vec{A}(-1,-7)$, $\vec{B}(5,1)$ and $\vec{C}(1,4)$. The equation of the bisector of the angle $\angle{ABC}$ is........(1993)
\iffalse
\end{enumerate}

\section*{B    :    True/False}

\begin{enumerate}
		\fi
		\\
		True/False

\item  The sraight line $5x+4y=0$ passes through the point of intersection of the straight lines $x+2y-10=0$ and $2x+5+6=0$.(1983)\\
\item The lines $2x+3y=19=0$ $9x+6y-17=0$ cut the coordinate axes in concyclic points.(1988)\\
\iffalse
\end{enumerate}

\section*{C  :   MCQ'S with One Correct Answer}

\begin{enumerate}
		\fi
\item The points$(a,b),(0,0)(a,b)$ and $(a^2,ab)$ are: (1979)
\begin{enumerate}
\item Collinear
\item Vertices of a parallelogram
\item Vertices of a rectangle
\item None of the above
\end{enumerate}
\item The point $(4,1)$ undergoes the following three transformations successively.
\begin{enumerate}
\item Reflection about the line y=x\\
\item Translation through a distance 2 units along the positivedirection of x-axis.\\
\item Rotation through an angle $\pi/4$ about the origin the counter clockwise direction.\\
\end{enumerate}

then the final position of the point is given by the coordinates. (1980)
\begin{enumerate}
\item  $\sbrak{\frac{1}{\sqrt{2}},\frac{7}{\sqrt{2}}}$
\item  $\brak{-\sqrt{2}, \sqrt[7]{2}}$  
\item  $\sbrak{\frac{-1}{\sqrt{2}},\frac{7}{\sqrt{2}}}$
\item  $\brak{\sqrt{2}, \sqrt[7]{2}}$
\end{enumerate}

\item The straight lines $x+y=0$,$3x+y-4=0$,$x+3y-4=0$ form a triangle which is (1932)
\begin{enumerate}
\item isosceles
\item equilateral
\item right angled
\item none of these
\end{enumerate}
\item If $\vec{P}=(1,0),\vec{Q}=(-1,0) $and$ \vec{R}=(2,0)$ are three given points,then the locus of the point S satisfying the relation $SQ^2+SR^2=SP^2$, is (1988)
\begin{enumerate}
\item a straight line parallel to x-axis  
\item a circle passing hrough the origin 
\item a circle with the center at the origin 
\item a straight line parallel to y-axis
\end{enumerate}
\item Line L has intercepts a and b on the coordinate axes. When the axes are rotated through a given angle, keeping up the origin fixed, the same line L has intercepts  p and q then (1990)\\
\begin{enumerate}
\item $a^2+b^2=p^2+q^2$ 
\item $\frac{1}{a^2} +\frac{1}{b^2}=\frac{1}{p^2}+\frac{1}{q^2}$ 
\item $a^2+p^2=b^2+q^2$ 
\item $\frac{1}{a^2}+\frac{1}{p^2}=\frac{1}{b^2}+\frac{1}{q^2}$
\end{enumerate}
\item If the sum of the distances of point from two perpendicular lines in a plane is 1, then its locus is (1992)\\
\begin{enumerate}
\item square 
\item circle 
\item straight line  
\item two intersecting lines
\end{enumerate}
\item The locus of the variable point whose distance from $(-2,0)$ is $2/3$ times its distance from the line $x= \frac{-9}{2}$ is (1994)
\begin{enumerate}
\item ellipse 
\item parabola  
\item hyperbola  
\item none of the above
\end{enumerate}
\item The equation to a pair of  opposite sides of a parallelogram are $x^2-5x+6=0$ and $y^2-6y+5=0$, the equations to its diagonals are (1994)
\begin{enumerate}
\item $x+4y=13, y=4x-7$  
\item $4x+y=13, 4y=x-7$ 
\item $4x+y=13, y=4x-7$
\item $y-4x=13,y+4x=7$ 
\end{enumerate}
\item The orthocenter of the lines formed by $xy=0$ and $x+y=1$ is (1995'S)
\begin{enumerate}
\item $\brak{\frac{1}{2},\frac{1}{2}}$
\item $\brak{\frac{1}{3},\frac{1}{3}}$
\item (0,0)
\item $\brak{\frac{1}{4},\frac{1}{4}}$
\end{enumerate}
\item Let PQR be a right angled isosceles triangle, right angled at $\vec{P}(2,1)$. If the equation of the line QR is $2x+y=3$, then the equation representing the pair of lines PQ and PR is (1999)\\
\begin{enumerate}
\item $3x^2-3y^2+8xy+20x+10y+25=0$
\item $3x^2-3y^2+8xy-20x-10y+25=0$
\item $3x^2-3y^2+8xy+10x+15y+20=0$
\item $3x^2-3y^2-8xy-10x-15y-20=0$
\end{enumerate}
\item If $x_1,x_2,x_3$ as well as $y_1,y_2,y_3 $ are in GP with the same common ratio, then the points $(x_1,y_1),(x_2,y_2)$ and $(x_3,y_3)$ (1999 - 2 marks)\\
\begin{enumerate}
\item lie on a straight line 
\item lie on an ellipse 
\item lie on a circle  
\item are vertices of a triangle 
\end{enumerate}
\item Let PS be the median of the triangle with vertices $\vec{P}(2,2), \vec{Q}(6,1) and \vec{R}(7,3)$. The equation of the line passing through (1,-1) and parallel to PS is. (2000'S)
\begin{enumerate}
\item $2x-9y-7=0$ 
\item $2x-9y-11=0$ 
\item $2x+9y-11=0$
\item $2x+9y+7=0$
\end{enumerate}
\item The incenter of the triangle with vertices $(1,\sqrt{3})$,$(0,0)$ and $(2,0)$ is (2000'S)
\begin{enumerate}
\item $\sbrak{1,\frac{\sqrt{3}}{2}}$ 
\item $\sbrak{\frac{2}{3},\frac{\sqrt{3}}{2}}$ 
\item $\sbrak{\frac{2}{3},\frac{\sqrt{3}}{2}}$ 
\item $\sbrak{1,\frac{1}{{\sqrt{3}}}}$
\end{enumerate}
\item The number of integer values of $m$, for which the x-coordinate of the of intersection of line $3x+4y=9$ and $y=mx+1$ is also an integer, is (2001'S)
\begin{enumerate}
\item 2 
\item 0 
\item 4   
\item 1
\end{enumerate}
\item Area of parallelogram formed by the lines $y=mx$, $y=mx+1$, $y=nx$ and $y=nx+1$ equals (2001'S)
\begin{enumerate}
\item $\frac{\mid m+n\mid}{(m-n)^2}$
\item $\frac{2}{\mid m+n \mid}$
\item $\frac{1}{(\mid m+n \mid)}$
\item $\frac{1}{(\mid m-n\mid)}$
\end{enumerate}
\item Let $0<a<\frac{\pi}{2}$ be a fixed angle. If $\vec{P}=(\cos\theta,\sin\theta)$, $\vec{Q}=(\cos\alpha-\theta),(\sin\alpha-\theta)$, then $\vec{Q}$ is obtained from $\vec{P}$ by (2002S)
\begin{enumerate}
\item clockwise rotation around origin through an angle $\alpha$
\item anticlockwise rotation around origin through an angle $\alpha$
\item reflection in the line through origin with slope $\tan\alpha$
\item reflection in the line through origin with slope $\tan\alpha/2$
\end{enumerate}
\item Let $\vec{P}=(-1,0)$,$\vec{Q}=(0,0)$ and $\vec{R}=(3,\sqrt[3]{3})$ be three points.\\
Then the equation of the bisector of the angle PQR is (2002'S)
\begin{enumerate}
\item $\frac{\sqrt{3}}{2x}+y=0$ 
\item $x+\sqrt{3}y=0$
\item $\sqrt{3}x+y=0$ 
\item $x+\frac{\sqrt{3}}{2y}=0$
\end{enumerate}
\item A straight line through the origin $\vec{O}$ meets the parallel lines $4x+2y=9$ and $2x+y+6=0$ at points $\vec{P}$ and $\vec{Q}$ respectively. Then the point $\vec{O}$ divides the segment $PQ$ in the ratio (2002)
\begin{enumerate}
\item $1:2$   
\item $3:4$
\item $2:1$ 
\item $4:3$ 
\end{enumerate}
\item The number of integral points(integral points means both the coordinates should be integer) exactly in the interior of the triangle with vertices is $(0,0)(0,21)$ and $(21,0)$ is (2003)
\begin{enumerate}
\item 133  
\item 190  
\item 233 
\item 105
\end{enumerate}
\item Orthocenter of triangle with vertices $(0,0)(3,4)$ and $(4,0)$ is  (2003)
\begin{enumerate}
\item $\sbrak{3,\frac{5}{4}}$ 
\item $\sbrak{3,12}$   
\item $\sbrak{3,\frac{3}{4}}$ 
\item $\sbrak{3,9}$
\end{enumerate}
\item Area of the triangle formed by the line $x+y=3$ and angle bisectors of the pair of straight lines $x^2-y^2+2y=1$ is (2004)
\begin{enumerate}
\item 2 sq. units  
\item 4 sq. units 
\item 6 sq. units    
\item 8 sq. units
\end{enumerate}
\item Let $\vec{O}(0,0)$,$\vec{P}(3,4)$,$\vec{Q}(6,0)$ be the vertices of the triangle $OPQ$. The point $\vec{R}$ inside the triangle $OPQ$ is such that the triangles $OPR,PQR,OQR$ are of equal area. The coordinates of $\vec{R}$ are   (2007)
\begin{enumerate}
\item $\sbrak{\frac{4}{3}, 3}$   
\item $\sbrak{3,\frac{2}{3}}$  
\item $\sbrak{3,\frac{4}{3}}$  
\item $\sbrak{\frac{4}{3},\frac{2}{3}}$
\end{enumerate}
\item A straight line through the point $(3,2)$ is inclined at an angle $60^\circ$  to the line $\sqrt{3}x+y=1$. If L also intersects the x-axis, then the equation of L is   (2011)
\begin{enumerate}
\item $y+\sqrt{3}x+2+\sqrt[3]{3}=0$
\item $y-\sqrt{3}x+2+\sqrt[3]{3}=0$ 
\item $\sqrt{3}y-x+3+\sqrt[2]{3}=0$  
\item $\sqrt{3}y+x-3+\sqrt[2]{3}=0$
\end{enumerate}
\iffalse
\end{enumerate}
\section*{D  :  MCQ'S with One or More Than One Correct Answer}

\begin{enumerate}
		\fi
\item Three lines $px+qy+r=0$, $qx+ry+p=0$ and $rx+py+q=0$ are concurrent if  (1985)\\
\begin{enumerate}
\item $p+q+r=0$
\item $p^2+q^2+r^2=qr+rp+pq$
\item $p^3+q^3+r^3=3pqr$
\item none of these
\end{enumerate}
\item The points $\sbrak{0,\frac{8}{3}}$,$\sbrak{1,3}$ and $\sbrak{82,30}$ are vertices of (1986)\\
\begin{enumerate}
\item an obtuse angle triangle
\item an acute angle triangle 
\item a  right angled triangle
\item an isosceles triangle
\item none of these
\end{enumerate}
\item All points lying inside the triangle are formed by the points$\brak{1,3}$,$(5,0)$ and $(-1,2)$ satisfy (1986)\\
\begin{enumerate}
\item $3x+2y\ge=0$
\item $2x+3y-13\ge=0$
\item $2x-3y-12\le=0$
\item $-2x+y\ge=0$
\item none of these
\end{enumerate}
\item A vector $\bar{a}$ has components of 2p and 1 with respect to a rectangular cartesian system. This system is rotted through a certain angle about origin in the counter clockwise sense. If,with respect the new system,$\bar{a}$ has components $p+1$ and 1, then (1986)\\
\begin{enumerate}
\item $p=0$  
\item $p=1$ or  $p=-1/3$  
\item $p=-1$ or $p=1/3$ 
\item $p=1$ or  $p=-1$
\item none of these.
\end{enumerate}
\item  If $\vec{P}(1,2),\vec{Q}(4,6),vec{R}(5,7)$ and $\vec{S}(a,b)$ are the vertices of a parallelogram PQRS, /then (1998)\\
\begin{enumerate}
\item $a=2,b=4$
\item $a=3,b=4$ 
\item $a=2,b=3$
\item $a=3,b=5$
\item none of these
\end{enumerate}
\item The diagonals of a parallelogram PQRS are along the lines $x+3y=4$ and $6x-2y=7$ then PQRS must be a. (1998)\\
\begin{enumerate}
\item rectangle
\item square
\item cyclic quadrilateral
\item rhombus
\end{enumerate}
\item If the vertices P,Q,R of a triangle PQR are rational points, which of the following points of the triangle PQR is (are) always rational point(s)? (1998)\\
\begin{enumerate}
\item centroid 
\item incenter
\item circumcenter 
\item orthocenter
(A rational point is a point both of whose coordinates are rational numbers.)
\end{enumerate}
\item  Let $\vec{L_1}$ be a straight line passing through the origin and $L_2$ be the straight line $x+y=1$. If the intercepts made by the circle $x^2+y^2-x+3y=0$ on $\vec{L_1}$ and $\vec{L_2}$ are equal, then which of the equation can represents $\vec{L_1}$? (1999)\\
\begin{enumerate}
\item $x+y=0$   
\item $x-y=0$ 
\item $x+7y=0$  
\item $x-7y=0$
\end{enumerate}
\item For $a>b>c>0$,the distance between (1,1)and the point of intersection of the lines $ax+by+c=0$ and $ay+c=0$ is less than $\sqrt[2]{2}$. Then (JEE Adv. 2013)\\
\begin{enumerate}
\item $a+b-c>0$ 
\item $a-b+c<0$
\item $a+b-c>0$
\item $a+b-c<0$
\end{enumerate}

\iffalse
\end{enumerate}

\section*{E  :  Subjective Problems}

\begin{enumerate}
		\fi
\item  A straight line segment of length l moves with its ends on two mutually perpendicular lines. Find the locus of the point which divides the line segment in the ratio $1:2$. (1978)
\item The area of triangle is 5. Two of its vertices are $\vec{A}(2,1)$ and $\vec{B}(3-2)$. The third vertex C lies on $y=x+3$. Finf C.(1978)
\item One side of the rectangle lies along the line $4x+7y+5=0$. Two of its vertices are (-3,1) and (1,1). Find the equation of the other two sides.(1978)
\item(a) Two vertices of a triangle are (5,-1) and (-2,3) If the orthocenter of the triangle is the origin,find the coordinates of the third point. (1978)
(b) Find the equation of the line which bisects the obtuse angle between the lines $x-2y+4=0$ and $4x-3y+2=0$\\ (1979)
\item  A straight line L is perpendicular to the line $5x-y=1$. The area of the triangle formed by the line L and the coordinate axes is 5. Find the equation of the line.\\ (1980)
\item The end A,B of a straight line segment of constant length c slide upon he fixed rectangular axes $(X,Y)$ respectively. If a rectanle OAPB are completed, then show that the locus of the foot of the perpendicular drawn from P to AB is $x^\frac{2}{3}+y^\frac{2}{3}=c^\frac{2}{3}$.(1983)\\
\item The vertices of the triangle are $[at_1t_2,a(t_1+t_2)]$,$[at_1t_3,a(t_1+t_3)]$ and $[at_3t_4,a(t_3+t_4)]$. Find the orthocenter of the triangle.\\ (1983 - 2 marks)
\item The coordinates of A,B,C are  (6,3),(3,5),(4,2) respectively, and P is any point (x,y). Show that the ratio of the area of the triangle $\triangle PBC$ and $\triangle ABC $ is $\mid\frac{(x+y-2)}{7}\mid$ (1983)\\
\item Two equal sides of a isosceles triangle are given by the equations $7x-y+3=0$ and $x+y-3=0$ and its third side passes through the point $(1,10)$. Determaine the equation of the third side.(1984)\\
\item One of the diametes of the circle circumscribing the rectangle ABCD is $4y=x+7$. If A and B are the ponts $(-3,4)$ and $(5,4)$ respetively then find the area of the rectangle.(1985)\\
\item Two sides of a rhombus ABCD are parallel to the lines $y=x+2$ and $y=7x+3$. If the diagonals of the rhombus intersects at the point $(1,2)$ and the vertex A on the y axis, find the possible coordinates of A.(1985)\\
\item Lines $L_1=ax+by+c=0$ and $L_2=lx+my+n=0$ intersects at the point P and make an angle $\theta$ with each other. Find the equation of a line L different from $\vec{L_2}$ which passes through P and makes the same angle $\theta$ with $\vec{L_1}$.(1989)\\
\item Let ABC be a triangle with $\vec{AB}-\vec{AC}$. If D is the pont of BC, E is the foot of the perpendicular drawn from D to AC and F the mid point of DE, prove that AF is perpendicular to BE.(1989)\\
\item Sraight lines $3x+4y-5$ and $4x+3y-5$ intersects at the point A. Points B and C 
are choosen on these two lines such that $\vec{AB}=\vec{AC}$. Determine the possible equaton of the line BC passing through the point (1,2).(1990)\\
\item A line cuts the x-axis at $\vec{A}(7,0)$ and the y-axis at $\vec{B}(0,5)$. A variable line PQ drawn perpendicular to AB cutting the x-axis  in P and y-axis in Q. If $\vec{AQ}$ and $\vec{BP}$ intersets at R, find the locus of R.(1990)\\
\item Find the equation of the line passing through the point (2,3)and intersects of length 2 units between the lines $y+2x=3$ and $y+2x=5$.(1991)\\

\begin{figure}[!h]
\centering
  \includegraphics[width=\columnwidth]{jee/figs/7.png}
 \caption{}
 \label{}
 \end{figure}

\newpage
\item Show that all chords of the curve  $2x^2-y^2-2x+4y=0$. Which subtend a right angle at the origin. Passes through a fixed point. Find the coordinates of the point.(1991)\\
\item Determine all values of a for which the point $(a, a^2)$ lies inside the triangle formed by the lines\\
\begin{align}
 &2x+3y-1 =0 \\
 &x+2y-1 =0 \ \: (1992)\\
 &5x-6y-1 =0 
\end{align}

\item Tangent at a point $\vec{P_1}$ [other than (0,0)] on the curve $y-x^3$ meets the curve again at $\vec{P_2}$. The tangent at $\vec{P_1}$ meets the curve at $\vec{P_2}$ and so on. Show that the abscissae of $\vec{p_1+p_2+p_3+.....+p_n}$ form a G.P. Also find the ratio.(1993)\\
\item A line through $\vec{A}(5,4)$ meets the line $x+3y+2=0$ $2x+y+4=0$ and $x-y-5=0$ at points B,C and D respectively. If    $\frac{15}{AB}^2+\frac{10}{AC}^2-\frac{6}{AD}^2$, find the equation of the line.(1993)\\
\item A triangle PQRS has its side PQ parallel to the line $y-mx$ and vertices P,Q and S on the lines $y-a$,$ x-b$ and $x--b$, respectively find the locus of the vertex R. (1996)\\
\item Using co-ordinate geometry prove that the three altitudes of any triangle are concurrent (1998)\\
\item For points $P=(x_1,Y_1)$ and $Q=(x_2,y_2)$ of the coordinate palne, a new distance $d(P,Q)$ is defined by $d(P,Q)=\mid x_1-x_2\mid + \mid y_1-y_2\mid$. Let $\vec{O}=(0,0)$  and $\vec{A}=(3,2)$. Prove that the set of points in the first quadrant which are equidistance (with to line new distance) from O and A consists of the union of line segment of finite length and an infinite ray. Sketch this set in a labelled diagram. (2000)\\
\item Let ABC and PQR be any two triangles in the same plane. Assume that the perpendicular from the points A,B,C to the sides QR, RP, PQ respectively are concurrent. Using vector methods or otherwise, prove that the perpendiculars from P,Q,R to BC, CA, AB  respectively are also concurrent. (2000)\\
\item Let a,b,c are real numbers with $a^2+b^2+c^2=1$. Show that the equation\\ 
$\begin{vmatrix}
 ax-by-c & bx+ay   & cx+a\\
 bx+ay   & ax+by-c & cy+b\\
 cx+a    & cy+b    & ax-by+c
\end{vmatrix}$ \\
represents a straight line.    (2001)
\item A straight line L through the origin meets the lines $x+y+1$ and $x+y=3$ at P and Q respectively. Through  P and Q two straight lines $\vec{L_1} and \vec{L_2}$ intersects at R . Show that the locus of R, as L varies, is a staight line. (2002)\\
\item A straight line negative slope passes through the points $(8,2)$ cuts the positive 
coordnate axes at points P and Q. Find the absolute minimum value of $\vec{OP}+\vec{OQ}$, as L varies . Where O is the origin. (2002)\\
\item The area of the triangle formed by the intersection of a line parallel to x-axis and passing through $\vec{p}(h,k)$ with the lines $y-x$ and $x+y-2$ is $4h^2$. Find the locus of the point. (2002)\\
\iffalse
\end{enumerate}

\section*{H   : Assertion and Reason Type Questions}

\begin{enumerate}
		\fi
\item Lines $L_1: Y-X=0$ and $L_2 :2x+y=0$ intersects the line $L_3:y+2=0$ at P and Q, respectively. The bisector of the acute angle between $L_1 and L_2$ intersects $L_3$ at R.\\
STATEMENT-1 : The ratio $PR:RQ$equals $\sqrt[2]{2}:\sqrt{5}$. because \\
STATEMENT-2 :In any triangle, bisector of an angle divides the triangle into two triangles. (2007)
\begin{enumerate}
\item Statement-1 is True, Statement-2 is True;Satement-2 is not a correct explaination for Statement-1
\item Statement-1 is True, Statement-2 is True;Satement-2 is NOT a correct explaination for Statement-1
\item Statement-1 is True, Statement-2 is False
\item Statement-1 is False, Statement-2 is True
\end{enumerate}
\iffalse
\end{enumerate}

\section*{I    :     Integer Value Correct Type }
\begin{enumerate}
		\fi
\item For a point P in the plane, let $\vec{d_1}(p)$ and $\vec{d_2}(p)$ be the distance of a point P
from the lines $x-y=0$ and $x=y=0$ respectively. The area of the region R consistes of all points P lying in the first quadrant of the plane and satisfying $2\leq \vec{d_1}(p)+\vec{d_2}(p)\leq$, is (JEE Adv. 2014)
\iffalse
\end{enumerate}

\section*{Section-B   [JEE Main/AIEE]}
\begin{enumerate}
		\fi
\item A triangle with vertices $(4,0),(-1,-1 ),(3,5)$ is (2002)
\begin{enumerate}
\item isosceles and right angled
\item isosceles  but not right angled
\item right angled but not isosceles 
\item neither right angled nor isosceles 
\end{enumerate}
\item Locus of mid point of the portion between the axes of $x\cos\alpha+y\sin\alpha=p$. Where p is constant is. (2002)
\begin{enumerate}
\item $x^2+y^2=\frac{4}{p^2}$ 
\item $x^2+y^2=4p^2$
\item$\frac{1}{x^2}+\frac{1}{y^2}=\frac{2}{p^2}$ 
\item $\frac{1}{x^2}+\frac{1}{y^2}=\frac{4}{p^2}$ 
\end{enumerate}
\item If the pair of lines $ax^2+2hxy+by^2+2gx+2fy+c=0$ intersects on the y-axis then (2002)
\begin{enumerate}
\item $2fgh=bg^2+ch^2$ 
\item $bg^2\neq ch^2$
\item $abc=2fgh$
\item none of these
\end{enumerate}
\item The pair of lines represented by $3ax^2+5xy+(a^2-2)y^2=0$ are perpendicular to each other for (2002)
\begin{enumerate}
\item two values of a 
\item $\forall a$
\item for one value of a 
\item for no values of a
\end{enumerate}
\item A square of side a lies above the x-axis and has one vertex at the origin. The side passing through the origin makes an angle $\alpha \left[ 0<a<\frac{\Pi}{4}]\right]$ with the positive direction of x-axis. The equation of its diagonal passing through the origin is (2003)
\begin{enumerate}
\item $y\brak{\cos\alpha+\sin\alpha}+x\brak{\cos\alpha-\sin\alpha}$=a
\item $y\brak{\cos\alpha-\sin\alpha)-x(\sin\alpha-\cos\alpha}$=a
\item $y\brak{\cos\alpha+\sin\alpha)+x(\sin\alpha-\cos\alpha}$=a
\item $y\brak{\cos\alpha+\sin\alpha)+x(\sin\alpha+\cos\alpha}$=a
\end{enumerate}
\item If the pair of straight lines $x^2-2pxy-y^2=0$ and $x^2-2qxy-y^2=0$ be such that each pair bisects the angle between the other pair, then (2003)
\begin{enumerate}
\item $pq=-1$  
\item $p=q$ 
\item $p=-q$  
\item $pq=1$
\end{enumerate}
\item Locus of centroid of the triangle whose vertices are $(a\cos t, a\sin t)$, $(a\sin t,-b\cos t)$ and (1,0) where t is a parameter, is (2003)
\begin{enumerate}
\item $(3x+1)^2+(3y)^2=a^2-b^2$
\item $(3x-1)^2+(3y)^2=a^2-b^2$
\item $(3x-1)^2+(3y)^2=a^2+b^2$
\item $(3x+1)^2+(3y)^2=a^2+b^2$
\end{enumerate}
\item If $x_1,x_2,x_3$ and $y_1,y_2,y_3$ are both in G.P with the same common ratio then the points $(x_1,y_1),(x_2,y_2)$ and $(x_3,y_3)$ (2003)
\begin{enumerate}
\item are vertices of a triangle
\item lies on a straight line
\item lies on ellipse
\item lies on circle
\end{enumerate}
\item If the equation of the locus of a equidistance from the point $(a_1,b_1)$ and $(a_2,b_2)$ is $(a_1-b_2)x+(a_1-b_2)y+c=0$, then the value of 'c' is (2003)
\begin{enumerate}
\item $\sqrt{a_1^2+b_1^2-a_2^2-b_2^2}$
\item $\frac{1}{2}(a_2^2+b_2^2-a_1^2-b_1^2)$
\item $a+1^2-a_2^2+b_1^2-b_2^2$
\item $\frac{1}{2}(a_1^2+a_2^2+b_1^2+b_2^2)$
\end{enumerate}
\item Let $\vec{A}(2,-3)$ and $\vec{B}(-2,3)$ be vertices of a triangle ABC. If the centroid of this triangle moves on the line $2x+3y=1$, then the locus of the vertex C is in the line  (2004)
\begin{enumerate}
\item $3x-2y=0$
\item $2x-3y=7$ 
\item $3x+2y=5$ 
\item $2x+=3y=9$
\end{enumerate}
\item The equation of the straight line passing through the point $(4,3)$ and making intercepts on the coordinate axes whose sum is -1 is (2004)
t\begin{enumerate}
\item $\frac{x}{2}-\frac{y}{3}=1$ and $\frac{x}{-2}+\frac{y}{1}=1$
\item $\frac{x}{2}-\frac{y}{3}=-1$ and $\frac{x}{-2}+\frac{y}{1}=-1$
\item $\frac{x}{2}+\frac{y}{3}=1$ and $\frac{x}{2}+\frac{y}{1}=1$
\item $\frac{x}{2}+\frac{y}{3}=1$ and $\frac{x}{-2}+\frac{y}{1}=-1$
\end{enumerate}
\item If the sum of the slopes of the lines given by $x^2-2cxy-7y^2=0$ is four times the product c has the value (2004)
\begin{enumerate}
\item -2 
\item -1 
\item  2 
\item  1
\end{enumerate}
\item If one of the lines given by $6x^2-xy+4cy^2=0$ is $3x+4y=0$, then c equals (2004)
\begin{enumerate}
\item -3 
\item -1 
\item  3 
\item  1
\end{enumerate}
\item The line parallel to the x-axis and passing through the intersection of the lines $ax+2by+3b=0$ and $bx-2ay-3a=0$, where $(a,b) \neq (0,0)$ (2005)
\begin{enumerate}
\item below the x-axis at a distance of $\frac{3}{2}$ from it
\item below the x-axis at a distance of $\frac{2}{3}$ from it
\item above the x-axis at a distance of $\frac{3}{2}$ from it
\item above the x-axis at a distance of $\frac{2}{3}$ from it
\end{enumerate}
\item If a vertex of a triangle is (1,1) and the mid point of two sides of this vertex are (-1,2) and (3,2) then the centroid of the triangle is (2005)
\begin{enumerate}
\item $\sbrak{-1,\frac{7}{3}}$ 
\item $\sbrak{\frac{-1}{3},\frac{7}{3}}$ 
\item $\sbrak{1,\frac{7}{3}}$  
\item $\sbrak{\frac{1}{3},\frac{7}{3}}$ 
\end{enumerate}
\item A straigjt line through point $A(3,4)$ is such that its intercept between the axes is bisected at A. its equation is (2006)
\begin{enumerate}
\item $x+y=7$ 
\item $3x-4y+7=0$  
\item $4x+3y=24$ 
\item $3x+4y=25$
\end{enumerate}
\item If $(a,a^2)$ falls inside the angle made by the lines $y= \frac{x}{2}, x>0$ and $y=3x, x>0$, then a belong to (2006)
\begin{enumerate}
\item $\sbrak{0,\frac{1}{2}}$ 
\item $(3,\infty)$ 
\item $\sbrak{\frac{1}{2},3}$ 
\item $\sbrak{-3,\frac{1}{2}}$
\end{enumerate}
\item Let $\vec{A}(h,k)$ and $\vec{B}(1,1)$ and $\vec{C}(2,1)$ be the vertices of a right angle triangle with AC as its hypotenuse. If the area of the triangle is 1 square unit, then the set of values which 'k' can taken is given by (2007)
\begin{enumerate}
\item $(-1,3)$ 
\item $(-3,-2)$
\item $(1,3)$ 
\item $(0,2)$
\end{enumerate}
\item Let $\vec{P}=(-1,0),\vec{Q}=(0,0) and \vec{R}=(3,\sqrt[3]{3})$ be three points. The equation of the bisector of the angle PQR is (2007)
\begin{enumerate}
\item $\frac{\sqrt{3}}{2}x+y=0$ 
\item $x+\sqrt{3}y=0$ 
\item $\sqrt{3}x+y=0$ 
\item $x+\frac{\sqrt{3}}{2}y=0$.
\end{enumerate}
\item If one of the lines of $my^2+(1-m^2)xy-mx^2=0$ is a bisector of the angle between the lines $xy=0$, then m is (2007)
\begin{enumerate}
\item 1  
\item 2 
\item $\frac{-1}{2}$ 
\item -2
\end{enumerate}
\item The perpendicular bisector of the line segment joining $P(1,4)$ and $Q(k,3)$ has y-intercept -4. Then a possible value of k is (2008)
\begin{enumerate}
\item  1 
\item  2 
\item -2 
\item -4
\end{enumerate}
\item The shortest distance between the line $y-x=1$ and the curve $x=y^2$ is (2009)
\begin{enumerate}
\item $\frac{\sqrt{2}{3}}{8}$ 
\item $\frac{\sqrt{3}{2}}{5}$ 
\item $\frac{\sqrt{3}}{4}$ 
\item $\frac{\sqrt{3}{2}}{8}$.
\end{enumerate}
\item The lines $p(p^2+1)x-y+q=0$ and $(p^2+1)^2x+(p^2+1)y+2q=0$ are perpendicular to a common line for (2009)
\begin{enumerate}
\item exactly one value of p
\item exactly two values of p
\item more than two values of p
\item no value of p
\end{enumerate}
\item Three distinct points A,B and C are given i the 2-dimentional coordinates plane such that the ratio of the  distance of any one of them from the point (1,0) to the distance from the point (-1,0) is equal to $\frac{1}{3}$. Then the circumcenter of the triangle ABC is at the point; (2009)
\begin{enumerate}
\item $\sbrak{\frac{5}{4} ,0}$ 
\item $\sbrak{\frac{5}{2} ,0}$ 
\item $\sbrak{\frac{5}{3} ,0}$
\item (0,0)
\end{enumerate}
\item The line L given by $\frac{x}{5}+\frac{y}{b}=1$ passes through the point (13,32). The line K is parallel L and has the equation $\frac{x}{c}+\frac{y}{3}=1$. Then the distance between L and K is. (2010)
\begin{enumerate}
\item $\sqrt{17}$  
\item $\frac{17}{\sqrt{15}}$ 
\item $\frac{23}{\sqrt{17}}$ 
\item $\frac{23}{\sqrt{15}}$
\end{enumerate}
\item The line $L_1:y-x=0$ and $L_2: 2x+=y=0$ intersects the line $L_3: y+2=0$ at P and Q respectively. The bisector of the acute angle between $L_1 and L_2$ intersects $L_3$ at R 
STATEMENT-1: The ratio PR:RQ equals $\sqrt[2]{2}:\sqrt{5}$\\
STATEMENT-2: In any triangle,bisector of an angle divides the triangle into two similar triangles.(2011)
\begin{enumerate}
\item Statement-1 is True, Statement-2 is True,Statement-2 is not a correct explaination for the Statement-1.
\item Statement-1 is True, Statement-2 is False
\item Statement-1 is False, Statement-2 is True
\item Statement-1 is True, Statement-2 is True, tatement-2 is correct explaination for the Statement-1.
\end{enumerate}
\item If the line $2x +y=k$ passes through the point which divides the line segment joining the points $(1,1)$ and (2,4) in the ratio $3:2$,then k equals: (2012)
\begin{enumerate}
\item $\frac{29}{5}$ 
\item 5 
\item 6 
\item $\frac{11}{5}$
\end{enumerate}
\item A ray of light along $x+\sqrt{3}y=\sqrt{3}$ get reflected upon reaching x-axis, the equation of the reflected ray is (JEE M 2013)
\begin{enumerate}
\item $y=x+\sqrt{3}$ 
\item $\sqrt{3}y=x-\sqrt{3}$ 
\item $y=\sqrt{3}x-\sqrt{3}$ 
\item $\sqrt{3}y=x-1$ 
\end{enumerate}
\item The coordinate of the incenter of the triangle that has the coordinates of mid points of its sides as (0,1) (1,1) and (1,0) is; (JEE M 2013)
\begin{enumerate}
\item $2+\sqrt{2}$
\item $2-\sqrt{2}$ 
\item $1+\sqrt{2}$ 
\item $1-\sqrt{2}$
\end{enumerate}
\item Let PS e the median of the triangle with vertices $\vec{P}(2,2)$,$\vec{Q}(6,-1)$ and $\vec{R}(7,3)$. The equation of the line passing through (1,-1)and parallel to PS is: (JEE M 2014)
\begin{enumerate}
\item $4x+7y+3=0$ 
\item $2x-9y+11=0$ 
\item $4x-7y+11=0$ 
\item $2x+7y+9=0$ 
\end{enumerate}
\item Let a,b,c and d be non-zero numbers. If the point of intersection of the lines $4ax+2ay+c=0$ and $5bx+2by+d=0$ lies in the fourth quadrant and eqidistance from the two axes then (JEE M 2014)
\begin{enumerate}
\item $3bc_2ad=0$  
\item $3bc+2ad=0$
\item $2b-3ad=0$ 
\item $2bc+3ad=0$
\end{enumerate}
\item The number of points, having both co-ordinates as integers, that lie in the interior of the triangle ith vertices (0,0)(0,41) and(41,0)is. (JEE M 2015)
\begin{enumerate}
\item 820 
\item 780 
\item 901 
\item 861
\end{enumerate}
\item Two sides  of a rhombus are alone the lines, $x-y+1=0$ and $7x+y-5=0$. If its diagoals intersect at(-1,-2), then which one of the following is a vertex of this rhombus? (JEE M 2016)
\begin{enumerate}
\item $\brak{\frac{1}{3},\frac{8}{3}}$ 
\item $\brak{\frac{10}{3},\frac{7}{3}}$ 
\item $\brak{-3,-9}$ 
\item $\brak{-3,-8}$
\end{enumerate}
\item A straight the thrugh a fixed point (2,3)intersects the coordinate axes at distinct point P  and Q. If O is the origin and the rectangle OQPR is completed, then the locus of R is: (JEE M 2018)
\begin{enumerate}
\item $2x+3y=xy$ 
\item $3x+2y=xy$ 
\item $3x+2y=6xy$ 
\item $3x+2y=6$
\end{enumerate}
\item consider the set of all lines $px+qy+r=0$ such that $3p+2q+4r=0$. Which one of the following statements is true? [JEE M 2019-9 Jan (M)]
\begin{enumerate}
\item The lines are concurent at the point $\brak{\frac{3}{4},\frac{1}{2}} $.
\item Each the line passes through the origin.
\item The lines are parallel.
\item The lines are not concurrent.
\end{enumerate}
\item Slope of line passing through $P(2,3)$ and intersecting the line $x+y=7$ at a distance of 4 units from  P,  is : [JEE M 2019-9 April (M)]
\begin{enumerate}
\item $\frac{1-\sqrt{5}}{1+\sqrt{5}}$
\item $\frac{1-\sqrt{7}}{1+\sqrt{7}}$
\item $\frac{\sqrt{7}-1}{\sqrt{7}+1}$
\item $\frac{\sqrt{5}-1}{\sqrt{5}+1}$
\end{enumerate}
\end{enumerate}
\iffalse
\end{document}
\fi

%--------------------------------------------------------
%\subsection{Properties}
%\iffalse
\documentclass[12pt]{article}
\usepackage{graphicx}
\usepackage{commath}
\usepackage{gensymb}

\begin{document}
\begin{center}
\textbf\large{CHAPTER-10 \\ VECTOR ALGEBRA}
\end{center}

\section{EXERCISE - 10.3}
\begin{enumerate}
		\fi
\begin{enumerate}[label=\thesection.\arabic*,ref=\thesection.\theenumi]
\item Find the angle between two vectors $\overrightarrow{a}$ and $\overrightarrow {b} $ with magnitudes $\sqrt{3}$ and 2 respectively having $\overrightarrow {a}.\overrightarrow {b}=\sqrt{6}$.
\item Find the angle between the the vectors $\hat{i}-2\hat{j}+3\hat{k}$ and $3\hat{i}-2\hat{j}+\hat{k}$.
\item Find the projection of the vector $\hat{i}-\hat{j}$ on the vector $\hat{i}+\hat{j}$.
	\\
		\iffalse
\documentclass[12pt]{chapters/10/7/4/3/figsarticle}
\usepackage{graphicx}
\usepackage[none]{chapters/10/7/4/3/figshyphenat}
\usepackage{graphicx}
\usepackage{listings}
\usepackage[english]{chapters/10/7/4/3/figsbabel}
\usepackage{graphicx}
\usepackage{caption} 
\usepackage{booktabs}
\usepackage{array}
\usepackage{amssymb} % for \because
\usepackage{amsmath}   % for having text in math mode
\usepackage{extarrows} % for Row operations arrows
\usepackage{listings}
\lstset{
  frame=single,
  breaklines=true
}
\usepackage{hyperref}
  
%Following 2 lines were added to remove the blank page at the beginning
\usepackage{atbegshi}% http://ctan.org/pkg/atbegshi
\AtBeginDocument{\AtBeginShipoutNext{\AtBeginShipoutDiscard}}


%New macro definitions
\newcommand{\mydet}[1]{chapters/10/7/4/3/figs\ensuremath{\begin{vmatrix}#1\end{vmatrix}}}
\providecommand{\brak}[1]{chapters/10/7/4/3/figs\ensuremath{\left(#1\right)}}
\newcommand{\solution}{\noindent \textbf{Solution: }}
\newcommand{\myvec}[1]{chapters/10/7/4/3/figs\ensuremath{\begin{pmatrix}#1\end{pmatrix}}}
\providecommand{\norm}[1]{chapters/10/7/4/3/figs\left\lVert#1\right\rVert}
\providecommand{\abs}[1]{chapters/10/7/4/3/figs\left\vert#1\right\vert}
\let\vec\mathbf


\begin{document}

\begin{center}
\title{\textbf{VECTORS}}
\date{\vspace{-5ex}} %Not to print date automatically
\maketitle
\end{center}

\setcounter{page}{1}

\section{10$^{th}$ Maths - Chapter 10}

This is Problem-3 from Exercise 10.3

\begin{enumerate}
\item Find the projection of the vector $\hat{i}-\hat{j}$ on the vector $\hat{i}+\hat{j}$  
\end{enumerate}
\section{SOLUTION}
\fi
\solution
The given points are
\begin{align}
 \vec{A}=\myvec{1\\ -1},
 \vec{B}=\myvec{1\\ 1}
\end{align}
Since
\begin{align}
	\vec{A}^\top \vec{B} &= \myvec{1 &-1} \myvec{1\\ 1}=\myvec{1 \times 1}+\myvec{-1 \times  1}=0
	\\
	\norm {\vec {B}}^2 &= (\vec{B}^\top  \vec{B})=\myvec{1 & 1} \myvec{1\\ 1}= (1 \times  1)+(1 \times  1)=2,
\end{align}
and the project vector is given by 
\begin{align}
	\vec{C} &= 
	\frac{\vec{A}^\top  \vec{B}}{\norm {\vec{B}}}^2 \vec{B}
	&=\frac{0}{2} \myvec{1\\ 1}
	=\myvec{0\\ 0}
\end{align}
This is verfied in Fig.
		\ref{fig:12/10/3/3Figure}.
\begin{figure}[h]
\includegraphics[width=\columnwidth]{chapters/12/10/3/3/figs/vector.png}
\caption{}
		\label{fig:12/10/3/3Figure}
\end{figure}

\item Find the projection of the vector $\hat{i}+3\hat{j}+7\hat{k}$ on the vector $7\hat{i}-\hat{j}+8\hat{k}$.
\item Show that each of the given three vectors is a unit vector: 

 $\frac{1}{7}(2\hat{i}+3\hat{j}+6\hat{k}),\frac{1}{7}(3\hat{i}-6\hat{j}+2\hat{k}),\frac{1}{7}(6\hat{i}+2\hat{j}-3\hat{k}$)
 
Also,show that they are mutually perpendicular to each other.
\item Find $\abs{\overrightarrow {a}}$ and $\abs{\overrightarrow {b}}$,if ($\overrightarrow {a}+\overrightarrow {b}).(\overrightarrow {a}-\overrightarrow {b})=8$ and $\abs{\overrightarrow {a}}=8\abs{\overrightarrow {b}}$.
\item Evaluate the product(3$\overrightarrow {a}-5\overrightarrow {b}).(2\overrightarrow {a}+7\overrightarrow {b}$).
\item Find the magnitude of two vectors $\overrightarrow {a}$ and $\overrightarrow {b}$, having the same magnitude and such that the angle between them is $60\degree$ and their scalar product is $\frac{1}{2}$
\item Find $\abs{\overrightarrow {x}}$,if for a unit vector $\overrightarrow {a},(\overrightarrow {x}-\overrightarrow {a}).(\overrightarrow {x}+\overrightarrow {a}$)=12.
	\\
		\iffalse
\documentclass[12pt]{article}
\usepackage{graphicx}
%\documentclass[journal,12pt,twocolumn]{IEEEtran}
\usepackage[none]{hyphenat}
\usepackage{graphicx}
\usepackage{listings}
\usepackage[english]{babel}
\usepackage{graphicx}
\usepackage{caption} 
\usepackage{hyperref}
\usepackage{booktabs}
\usepackage{array}
\usepackage{amsmath}   % for having text in math mode
\usepackage{listings}
\lstset{
  frame=single,
  breaklines=true
}
  
%Following 2 lines were added to remove the blank page at the beginning
\usepackage{atbegshi}% http://ctan.org/pkg/atbegshi
\AtBeginDocument{\AtBeginShipoutNext{\AtBeginShipoutDiscard}}
%


%New macro definitions
\newcommand{\mydet}[1]{\ensuremath{\begin{vmatrix}#1\end{vmatrix}}}
\providecommand{\brak}[1]{\ensuremath{\left(#1\right)}}
\providecommand{\norm}[1]{\left\lVert#1\right\rVert}
\newcommand{\solution}{\noindent \textbf{Solution: }}
\newcommand{\myvec}[1]{\ensuremath{\begin{pmatrix}#1\end{pmatrix}}}
\let\vec\mathbf

\begin{document}

\begin{center}
\title{\textbf{Vector Dot Product}}
\date{\vspace{-5ex}} %Not to print date automatically
\maketitle
\end{center}
\setcounter{page}{1}

\section{12$^{th}$ Maths - Chapter 10}
This is Problem-9 from Exercise 10.3
\begin{enumerate}
\item Find $\norm{\vec{x}}$, if for a unit vector $\vec{a}$, $\brak{\vec{x}-\vec{a}}.\brak{\vec{x}+\vec{a}} = 12$.\\
	\fi
\solution 
From the given information,
\begin{align}
  \label{eq:12/10/3/9det2f}
  \brak{\vec{x}-\vec{a}}^\top\brak{\vec{x}+\vec{a}} &= 12 \\
  \implies \vec{x}^\top\vec{x} - \vec{a}^\top\vec{x} + \vec{x}^\top\vec{a} - \vec{a}^\top\vec{a} &= 12 \\
  \implies \norm{\vec{x}}^{2} - \norm{\vec{a}}^{2} &= 12 \\
\implies   \norm{\vec{x}}^{2} - 1 &= 12  \\
	\text{or, }  
	\norm{\vec{x}} &= \sqrt{13}
\end{align}

\item If $\overrightarrow {a}=2\hat{i}+2\hat{j}3\hat{k},\overrightarrow {b}=\hat{-i}+2\hat{j}+\hat{k}$ and $\overrightarrow {c}=3\hat{i}+\hat{j}$ are such that $\overrightarrow {a}+\lambda\overrightarrow {b}$ is perpendicular to $\overrightarrow {c}$,then find the value of $\lambda$.
	\\
		\iffalse
\documentclass[12pt]{article}
\usepackage{graphicx}
\usepackage{amsmath}
\usepackage{mathtools}
\usepackage{gensymb}

\newcommand{\mydet}[1]{\ensuremath{\begin{vmatrix}#1\end{vmatrix}}}
\providecommand{\brak}[1]{\ensuremath{\left(#1\right)}}
\providecommand{\norm}[1]{\left\lVert#1\right\rVert}
\newcommand{\solution}{\noindent \textbf{Solution: }}
\newcommand{\myvec}[1]{\ensuremath{\begin{pmatrix}#1\end{pmatrix}}}
\let\vec\mathbf

\begin{document}
\begin{center}
\textbf\large{CHAPTER-10 \\ VECTOR ALGEBRA}

\end{center}
\section*{Excercise 10.3}

Q10.If $\vec{a} = 2\hat{i}+2\hat{j}+3\hat{k}, \vec{b} = -\hat{i}+2\hat{j}+\hat{k} \text{ and } \vec{c} = 3\hat{i}+\hat{j}$ are such that $\vec{a}+\lambda \vec{b}$ is perpendicular to $\vec{c}$, then find the value of $\lambda$.
\fi
\solution
Given that
\begin{align}
	(\vec{a}+\lambda \vec{b})^{\top} \vec{c} &= 0\\
\implies \vec{a}^{\top}\vec{c}+\lambda \vec{b}^{\top}\vec{c}&=0\\
\implies 	\lambda \vec{b}^{\top}\vec{c}&=-\vec{a}^{\top}\vec{c}\\
\implies 	\lambda(\vec{b}^{\top}\vec{c})(\vec{b}^{\top}\vec{c})^{-1}&=-(\vec{a}^{\top}\vec{c})(\vec{b}^{\top}\vec{c})^{-1}\\
\implies 	\lambda&=-(\vec{a}^{\top}\vec{c})(\vec{b}^{\top}\vec{c})^{-1}
\end{align}
Now substituting the values
\begin{align}
	\vec{a}^{\top}\vec{c}&=\myvec{2&2&3} \myvec{3\\1\\0} = 8\\
	\vec{b}^{\top}\vec{c}&=\myvec{-1&2&1} \myvec{3\\1\\0} = -1,
\end{align}
\begin{align}
	\lambda&=-(\vec{a}^{\top}\vec{c})(\vec{b}^{\top}\vec{c})^{-1}\\
	&=-(8)(-1)^{-1}\\
	&=8
\end{align}



\item Show that $\abs {\overrightarrow {a}}\overrightarrow {b}+\abs{\overrightarrow {b}}\overrightarrow {a}$ is perpendicular to $\abs{\overrightarrow {a}} \overrightarrow {b}-\abs{\overrightarrow {b}} \overrightarrow {a}$, for any two nonzero vectors $\overrightarrow {a}$ and $\overrightarrow {b}$.
\item If $\overrightarrow {a}.\overrightarrow {a}$=0 and $\overrightarrow {a}.\overrightarrow {b}$=0, then what can be conculded about the vector $\overrightarrow {b}$?
\item If $\overrightarrow {a},\overrightarrow {b},\overrightarrow {c}$ are unit vectors such that $\overrightarrow {a}+\overrightarrow {b}+\overrightarrow {c}=\overrightarrow {0}$, find the value of $\overrightarrow {a}.\overrightarrow {b}+\overrightarrow {b}.\overrightarrow {c}+\overrightarrow {c}.\overrightarrow {a}$.
\item If either vector $\overrightarrow {a}=0$ or $\overrightarrow {b}=0$, then $\overrightarrow {a}.\overrightarrow {b}$=0. But the converse need not be true .Justify your answer with an example.
\item If the vertices A,B,C of a triangle ABC are (1,2,3),(-1,0,0)(0,1,2), respectively , then find  $\angle{ABC}. [\angle{ABC}$ is the angle between the vectors $\overrightarrow{BA}$ and $\overrightarrow{BC}$].
\item show that the points A(1,2,7),B(2,6,3)and C(3,10,-1) are collinear.
\item show that the vectors $2\hat{i}-\hat{j}+\hat{k},\hat{i}-3\hat{j}-5\hat{k}$ and  $3\hat{i}-4\hat{j}-4\hat{k}$ from the vertices of a right angled triangle.
\item If $\overrightarrow {a}$ is a nonzero vector of magnitude 'a' and $\lambda$ a nonzero scalar , then $\lambda\overrightarrow {a}$ is unit vector if

\begin{enumerate} 
\item $\lambda=1$ 
\item $\lambda=-1$
\item $a=\abs{\lambda}$
\item $a=1/\abs{\lambda}$  
\end{enumerate}
\iffalse
\item If $\overrightarrow {a}=2\hat{i}+2\hat{j}3\hat{k},\overrightarrow {b}=\hat{-i}+2\hat{j}+\hat{k}$ and $\overrightarrow {c}=3\hat{i}+\hat{j}$ are such that $\overrightarrow {a}+\lambda\overrightarrow {b}$ is perpendicular to $\overrightarrow {c}$,then find the value of $\lambda$.
	\\
		\iffalse
\documentclass[12pt]{article}
\usepackage{graphicx}
\usepackage{amsmath}
\usepackage{mathtools}
\usepackage{gensymb}

\newcommand{\mydet}[1]{\ensuremath{\begin{vmatrix}#1\end{vmatrix}}}
\providecommand{\brak}[1]{\ensuremath{\left(#1\right)}}
\providecommand{\norm}[1]{\left\lVert#1\right\rVert}
\newcommand{\solution}{\noindent \textbf{Solution: }}
\newcommand{\myvec}[1]{\ensuremath{\begin{pmatrix}#1\end{pmatrix}}}
\let\vec\mathbf

\begin{document}
\begin{center}
\textbf\large{CHAPTER-10 \\ VECTOR ALGEBRA}

\end{center}
\section*{Excercise 10.3}

Q10.If $\vec{a} = 2\hat{i}+2\hat{j}+3\hat{k}, \vec{b} = -\hat{i}+2\hat{j}+\hat{k} \text{ and } \vec{c} = 3\hat{i}+\hat{j}$ are such that $\vec{a}+\lambda \vec{b}$ is perpendicular to $\vec{c}$, then find the value of $\lambda$.
\fi
\solution
Given that
\begin{align}
	(\vec{a}+\lambda \vec{b})^{\top} \vec{c} &= 0\\
\implies \vec{a}^{\top}\vec{c}+\lambda \vec{b}^{\top}\vec{c}&=0\\
\implies 	\lambda \vec{b}^{\top}\vec{c}&=-\vec{a}^{\top}\vec{c}\\
\implies 	\lambda(\vec{b}^{\top}\vec{c})(\vec{b}^{\top}\vec{c})^{-1}&=-(\vec{a}^{\top}\vec{c})(\vec{b}^{\top}\vec{c})^{-1}\\
\implies 	\lambda&=-(\vec{a}^{\top}\vec{c})(\vec{b}^{\top}\vec{c})^{-1}
\end{align}
Now substituting the values
\begin{align}
	\vec{a}^{\top}\vec{c}&=\myvec{2&2&3} \myvec{3\\1\\0} = 8\\
	\vec{b}^{\top}\vec{c}&=\myvec{-1&2&1} \myvec{3\\1\\0} = -1,
\end{align}
\begin{align}
	\lambda&=-(\vec{a}^{\top}\vec{c})(\vec{b}^{\top}\vec{c})^{-1}\\
	&=-(8)(-1)^{-1}\\
	&=8
\end{align}



\item Show that $\abs {\overrightarrow {a}}\overrightarrow {b}+\abs{\overrightarrow {b}}\overrightarrow {a}$ is perpendicular to $\abs{\overrightarrow {a}} \overrightarrow {b}-\abs{\overrightarrow {b}} \overrightarrow {a}$, for any two nonzero vectors $\overrightarrow {a}$ and $\overrightarrow {b}$.
\item If $\overrightarrow {a}.\overrightarrow {a}$=0 and $\overrightarrow {a}.\overrightarrow {b}$=0, then what can be conculded about the vector $\overrightarrow {b}$?
\item If $\overrightarrow {a},\overrightarrow {b},\overrightarrow {c}$ are unit vectors such that $\overrightarrow {a}+\overrightarrow {b}+\overrightarrow {c}=\overrightarrow {0}$, find the value of $\overrightarrow {a}.\overrightarrow {b}+\overrightarrow {b}.\overrightarrow {c}+\overrightarrow {c}.\overrightarrow {a}$.
\item If either vector $\overrightarrow {a}=0$ or $\overrightarrow {b}=0$, then $\overrightarrow {a}.\overrightarrow {b}$=0. But the converse need not be true .Justify your answer with an example.
\item If the vertices A,B,C of a triangle ABC are (1,2,3),(-1,0,0)(0,1,2), respectively , then find  $\angle{ABC}. [\angle{ABC}$ is the angle between the vectors $\overrightarrow{BA}$ and $\overrightarrow{BC}$].
\item show that the points A(1,2,7),B(2,6,3)and C(3,10,-1) are collinear.
\item show that the vectors $2\hat{i}-\hat{j}+\hat{k},\hat{i}-3\hat{j}-5\hat{k}$ and  $3\hat{i}-4\hat{j}-4\hat{k}$ from the vertices of a right angled triangle.
\item If $\overrightarrow {a}$ is a nonzero vector of magnitude 'a' and $\lambda$ a nonzero scalar , then $\lambda\overrightarrow {a}$ is unit vector if

\begin{enumerate} 
\item $\lambda=1$ 
\item $\lambda=-1$
\item $a=\abs{\lambda}$
\item $a=1/\abs{\lambda}$  
\end{enumerate}
\fi

\end{enumerate}


%
%\subsection{Properties}
%\begin{enumerate}
\item 
\label{chapters/9/8/1/1}
\iffalse
\documentclass[journal,10pt,twocolumn]{article}
\usepackage[margin=0.5in]{geometry}
\usepackage[cmex10]{amsmath}
\usepackage{array}
\usepackage{booktabs}

% The preceding line is only needed to identify funding in the first footnote. If that is unneeded, please comment it out.
\usepackage{cite}
\usepackage{amsmath,amssymb,amsfonts}
\usepackage{graphicx}
\usepackage{textcomp}
\usepackage{xcolor}
\usepackage{graphicx}
\graphicspath{{./figs}}{}
\def\BibTeX{{\rm B\kern-.05em{\sc i\kern-.025em b}\kern-.08em
    T\kern-.1667em\lower.7ex\hbox{E}\kern-.125emX}}

\usepackage{tikz}
\usetikzlibrary{shapes.geometric}
\usetikzlibrary{shapes.geometric,angles,quotes}


\begin{document}



\newtheorem{theorem}{Theorem}[section]
\newtheorem{problem}{Problem}
\newtheorem{proposition}{Proposition}[section]
\newtheorem{lemma}{Lemma}[section]
\newtheorem{corollary}[theorem]{Corollary}
\newtheorem{example}{Example}[section]
\newtheorem{definition}[problem]{Definition}
%\newtheorem{thm}{Theorem}[section] 
%\newtheorem{defn}[thm]{Definition}
%\newtheorem{algorithm}{Algorithm}[section]
%\newtheorem{cor}{Corollary}
\newcommand{\BEQA}{\begin{eqnarray}}
\newcommand{\EEQA}{\end{eqnarray}}
\newcommand{\define}{\stackrel{\triangle}{=}}
\newcommand*\circled[1]{\tikz[baseline=(char.base)]{
    \node[shape=circle,draw,inner sep=2pt] (char) {#1};}}

\bibliographystyle{article}
%\bibliographystyle{ieeetr}


\providecommand{\mbf}{\mathbf}
\providecommand{\pr}[1]{\ensuremath{\Pr\left(#1\right)}}
\providecommand{\re}[1]{\ensuremath{\text{Re}\left(#1\right)}}
\providecommand{\im}[1]{\ensuremath{\text{Im}\left(#1\right)}}
\providecommand{\qfunc}[1]{\ensuremath{Q\left(#1\right)}}
\providecommand{\sbrak}[1]{\ensuremath{{}\left[#1\right]}}
\providecommand{\lsbrak}[1]{\ensuremath{{}\left[#1\right.}}
\providecommand{\rsbrak}[1]{\ensuremath{{}\left.#1\right]}}
\providecommand{\brak}[1]{\ensuremath{\left(#1\right)}}
\providecommand{\lbrak}[1]{\ensuremath{\left(#1\right.}}
\providecommand{\rbrak}[1]{\ensuremath{\left.#1\right)}}
\providecommand{\cbrak}[1]{\ensuremath{\left\{#1\right\}}}
\providecommand{\lcbrak}[1]{\ensuremath{\left\{#1\right.}}
\providecommand{\rcbrak}[1]{\ensuremath{\left.#1\right\}}}

\newcommand{\sgn}{\mathop{\mathrm{sgn}}}

%\providecommand{\hilbert}{\overset{\mathcal{H}}{ \rightleftharpoons}}
\providecommand{\system}{\overset{\mathcal{H}}{ \longleftrightarrow}}
	%\newcommand{\solution}[2]{\textbf{Solution:}{#1}}
\newcommand{\solution}{\noindent \textbf{Solution: }}
\newcommand{\cosec}{\,\text{cosec}\,}
\providecommand{\dec}[2]{\ensuremath{\overset{#1}{\underset{#2}{\gtrless}}}}
\newcommand{\myvec}[1]{\ensuremath{\begin{pmatrix}#1\end{pmatrix}}}
\newcommand{\mydet}[1]{\ensuremath{\begin{vmatrix}#1\end{vmatrix}}}
	\newcommand*{\permcomb}[4][0mu]{{{}^{#3}\mkern#1#2_{#4}}}
\newcommand*{\perm}[1][-3mu]{\permcomb[#1]{P}}
\newcommand*{\comb}[1][-1mu]{\permcomb[#1]{C}}

%\numberwithin{equation}{section}
\numberwithin{equation}{subsection}
%\numberwithin{problem}{section}
%\numberwithin{definition}{section}

\let\vec\mathbf


\title{
{Quadrilateral with angles \\
Using lines}\\
}
\author{Suresh Beere}
\maketitle
\tableofcontents
\section{Problem statement}
\fi
The angles of quadrilateral are in the ratio 3:5:9:13. Find all the angles of the quadrilateral.
\iffalse

\section{Considerations}
\vspace{0.2cm}
\begin{flushleft}
As per given data, the following table has been prepared.\\
\end{flushleft}
\vspace{0.3cm}
\begin{table}[htbp]
    \centering
\setlength\extrarowheight{2pt}
\begin{tabular}{|c|c|c|} \hline
      \textbf{Symbol}           &   \textbf{Value}   & \textbf{Description}\\ \hline
	x &  & constant\\  \hline
	a & 3x & Angle A\\ \hline
	b & 5x & Angle B\\ \hline
    c & 9x & Angle C \\ \hline
    d & 13x & Angle D \\ \hline
\end{tabular}
\caption{\label{tab:widgets}Considerations}
\end{table}

\section{Plot of Quadrilateral}
\vspace{0.25cm}
Plot of the quadrilateral is shown in the figure 1.
\begin{figure}[h]
\includegraphics[width=1\columnwidth]{line.png}
\caption{Plot of Quadrilateral}
\end{figure}

\section{Solution}
\begin{flushleft}


Let angle in the ratio 3:5:9:13 be a,b,c,d\\
\vspace{0.25cm}
Let a=3x,b=5x,,c=9x,d=13x\\
\vspace{0.25cm}
where x is any number\\
\vspace{0.25cm}
We know that\\
\vspace{0.25cm}
Sum of angle of quadrilateral is 360\textdegree\\
\vspace{0.25cm}
a+b+c+d=360$\textdegree$   [Angle sum property of quadrilateral]\\
\vspace{0.25cm}
3x+5x+9x+13x=360\textdegree \\
\vspace{0.25cm}
30x=360\textdegree\\
\vspace{0.25cm}
x=360/30\\
\vspace{0.25cm}
x=12$\textdegree$\\
\vspace{0.25cm}
Hence the angles of Quadrilateral are\\
\vspace{0.25cm}
a=3x=3×12=36\textdegree\\
\vspace{0.25cm}
b=5x=5×12=60\textdegree\\
\vspace{0.25cm}
c=9x=9×12=108\textdegree\\
\vspace{0.25cm}
d=13x=13×12=156\textdegree
\end{flushleft}




\section{Software}
\begin{flushleft}
Download the codes given in the link below and execute them.\\
\end{flushleft}

\begin{table}[h]
\centering
\begin{tabular}{|c|} \hline
\rule{0pt}{10pt} 
https://github.com/sureshoye/line-assignment/blob \\
/main/codes/line.py\\
\\\hline
 \end{tabular}
\end{table}




\section{Conclusion}
\begin{flushleft}
The angles of Quadrilateral are\\
\vspace{0.25cm}
a=3x=3×12=36\textdegree\\
\vspace{0.25cm}
b=5x=5×12=60\textdegree\\
\vspace{0.25cm}
c=9x=9×12=108\textdegree\\
\vspace{0.25cm}
d=13x=13×12=156\textdegree
\end{flushleft}
\endcenter
\end{document}
\fi

\item 
\label{chapters/9/8/1/2}
\iffalse
 

\def\mytitle{MATRICES USING PYTHON}
\def\myauthor{THOUTU RAHUL RAJ}
\def\contact{rdj4648@gmail.com}
\def\mymodule{Future Wireless Communication (FWC)}
\documentclass[10pt, a4paper]{article}
\usepackage[a4paper,outer=1.5cm,inner=1.5cm,top=1.75cm,bottom=1.5cm]{geometry}
\twocolumn
\usepackage{graphicx}
\graphicspath{{./images/}}
\usepackage[colorlinks,linkcolor={black},citecolor={blue!80!black},urlcolor={blue!80!black}]{hyperref}
\usepackage[parfill]{parskip}
\usepackage{lmodern}
\usepackage{amsmath,amsfonts,amssymb,amsthm}
\usepackage{tikz}
	\usepackage{physics}
%\documentclass[tikz, border=2mm]{standalone}
\usepackage{karnaugh-map}
%\documentclass{article}
\usepackage{tabularx}
\usepackage{circuitikz}
\usetikzlibrary{calc}
\usepackage{amsmath}
\usepackage{amssymb}
\renewcommand*\familydefault{\sfdefault}
\usepackage{watermark}
\usepackage{lipsum}
\usepackage{xcolor}
\usepackage{listings}
\usepackage{float}
\usepackage{titlesec}
\providecommand{\norm}[1]{\left\lVert#1\right\rVert}
\providecommand{\sbrak}[1]{\ensuremath{{}\left[#1\right]}}
\providecommand{\lsbrak}[1]{\ensuremath{{}\left[#1\right.}}
\providecommand{\rsbrak}[1]{\ensuremath{{}\left.#1\right]}}
\providecommand{\brak}[1]{\ensuremath{\left(#1\right)}}
\providecommand{\lbrak}[1]{\ensuremath{\left(#1\right.}}
\providecommand{\rbrak}[1]{\ensuremath{\left.#1\right)}}
\providecommand{\cbrak}[1]{\ensuremath{\left\{#1\right\}}}
\providecommand{\lcbrak}[1]{\ensuremath{\left\{#1\right.}}
\providecommand{\rcbrak}[1]{\ensuremath{\left.#1\right\}}}
\newcommand{\myvec}[1]{\ensuremath{\begin{pmatrix}#1\end{pmatrix}}}
\let\vec\mathbf
\providecommand{\mtx}[1]{\mathbf{#1}}
\titlespacing{\subsection}{1pt}{\parskip}{3pt}
\titlespacing{\subsubsection}{0pt}{\parskip}{-\parskip}
\titlespacing{\paragraph}{0pt}{\parskip}{\parskip}
\newcommand{\figuremacro}[5]

\begin{document}

\title{\mytitle}
\author{\myauthor\hspace{1em}\\\contact\\FWC22008\hspace{6.5em}IITH\hspace{0.5em}\mymodule\hspace{6em}ASSIGN-4}
\date{}
	\maketitle
		
	\tableofcontents
\vspace{5mm}
\fi
If diagonals of a parallelogram are equal then show that it is a rectangle.

	\begin{figure}[!h]
		\centering
		\includegraphics[width=\columnwidth]{chapters/9/8/1/2/fig.pdf}
     %\includegraphics[scale=0.5]{fig.pdf} 
		\caption{}
		\label{fig:9/8/1/2}
  	\end{figure}
	\solution 
   See Fig. 
		\ref{fig:9/8/1/2}.
   From 
	  \eqref{eq:two-pgm}, 
\begin{align}
	  \label{eq:two-pgm-def} 
 \vec{B} - \vec{A}= \vec{C}-\vec{D}
 \\
\implies  \vec{B} - \vec{C}= \vec{A}-\vec{D}
	\end{align}
	Also, it is given that the diagonals of $ABCD$ are equal.  Hence, 
\begin{align}
	\norm{\vec{C} - \vec{A}}^2&= \norm{\vec{D}-\vec{B}}^2
 \\
	\implies 
	\norm{(\vec{C}-\vec{B}) + (\vec{B}-\vec{A})}^2 &= \norm{(\vec{D}-\vec{C}) + (\vec{C}-\vec{B}}^2
\end{align}
which can be expressed as
\begin{multline}
	\norm{\vec{C}-\vec{B}}^2 + \norm{\vec{B}-\vec{A}}^2 + 2(\vec{C}-\vec{B})^{\top} (\vec{B}-\vec{A}) 
	\\
	= \norm{\vec{D}-\vec{C}}^2 + \norm{\vec{C}-\vec{B}}^2+2(\vec{D}-\vec{C})^{\top} (\vec{C}-\vec{B}) 
\end{multline}
which, can be simplified to obtain 
\begin{align}
	(\vec{C}-\vec{B})^{\top} (\vec{B}-\vec{A})&=(\vec{D}-\vec{C})^{\top} (\vec{C}-\vec{B}) 
\end{align}
since 
\begin{align}
\norm{\vec{D}-\vec{C}} =   
\norm{\vec{B}-\vec{A}}   
\end{align}
yielding 
\begin{align}
	(\vec{A}-\vec{B})^{\top} (\vec{B}-\vec{C})=\vec{0}
\end{align}
	  from \eqref{eq:two-pgm-def}.  

\item 
\label{chapters/9/8/1/3}
\iffalse
\documentclass[a4paper,12pt,twocolumn]{article}
\usepackage{graphicx}
\usepackage[margin=0.5in]{geometry}
\usepackage[cmex10]{amsmath}
\usepackage{array}
\usepackage{gensymb}
\usepackage{booktabs}
\title{Line Assignment}

\author{Ravi Sumanth Muppana- FWC22003}
\date{September 2022}
\providecommand{\norm}[1]{\left\lVert#1\right\rVert}
\providecommand{\abs}[1]{\left\vert#1\right\vert}
\let\vec\mathbf
\newcommand{\myvec}[1]{\ensuremath{\begin{pmatrix}#1\end{pmatrix}}}	
\newcommand{\mydet}[1]{\ensuremath{\begin{vmatrix}#1\end{vmatrix}}}
\providecommand{\brak}[1]{\ensuremath{\left((#1\right)}}
\begin{document}
\maketitle
\section{Problem:}
\fi
Show that if the diagonals of a quadrilateral bisect each other at right angles, then it is a rhombus.
\begin{figure}[!h]
	\centering
	\includegraphics[width=\columnwidth]{chapters/9/8/1/3/figs/rhombus.png}
	\caption{Rhombus}
	\label{fig:9/8/1/3}
\end{figure}
\\
\solution See Fig. 
	\ref{fig:9/8/1/3}.
\iffalse
\maketitle
\section{Solution:}
\subsection{Theory:}
\fi
From the given information,
\begin{align}
	\label{eq:9/8/1/3-mid}
	\frac{\vec{B}+\vec{D}}{2}
	&=	
	\frac{\vec{A}+\vec{C}}{2}
	\\
	\brak{\vec{B}-\vec{D}}^{\top}&
	\brak{\vec{A}-\vec{C}} = 0
	\label{eq:9/8/1/3-orth}
\end{align}
From 
	\eqref{eq:9/8/1/3-mid},
\begin{align}
	\vec{B}-\vec{A}
	=	
	\vec{C}-\vec{D}
	\label{eq:9/8/1/3-pgm}
\end{align}
which, from  
	  \eqref{eq:two-pgm}, 
is the definition of  a parallelogram.
Further, substituting
\begin{align}
	\vec{B}-\vec{D} &= \brak{\vec{B}-\vec{A}} +  
	\brak{\vec{A}-\vec{D}}
	\\
	\vec{A}-\vec{C} &= \brak{\vec{A}-\vec{B}} +  
	\brak{\vec{B}-\vec{C}}
\end{align}
in 
	\eqref{eq:9/8/1/3-orth},  
\begin{multline}
	\sbrak{\brak{\vec{B}-\vec{A}} +  
	\brak{\vec{A}-\vec{D}}}^{\top}
	\sbrak{ \brak{\vec{A}-\vec{B}} +  
	\brak{\vec{B}-\vec{C}}} = 0
	\\
	\implies 
-\norm{\vec{B}-\vec{A}}^2 + \brak{\vec{B}-\vec{A}}^{\top}\brak{\vec{B}-\vec{C}} + 
\\
	\brak{\vec{A}-\vec{D}}^{\top}\brak{\vec{A}-\vec{B}} + 
\brak{\vec{A}-\vec{D}}^{\top}
\brak{\vec{B}-\vec{C}} = 0
	\label{eq:9/8/1/3-org}
\end{multline}
From
	\eqref{eq:9/8/1/3-pgm},
\begin{align}
	\vec{B}-\vec{C}
	=	
	\vec{A}-\vec{D}
	\\
	\implies \brak{\vec{B}-\vec{A}}^{\top}\brak{\vec{B}-\vec{C}} +
	 \brak{\vec{A}-\vec{D}}^{\top}\brak{\vec{A}-\vec{B}} =\vec{0}
	\label{eq:9/8/1/3-orth-pf}
\end{align}
and 
\begin{align}
\brak{\vec{A}-\vec{D}}^{\top}
\brak{\vec{B}-\vec{C}} = \norm{\vec{B}-\vec{C}}^2
	\label{eq:9/8/1/3-orth-eq}
\end{align}
Substituting from

	\eqref{eq:9/8/1/3-orth-pf}
and
	\eqref{eq:9/8/1/3-orth-eq}
in
	\eqref{eq:9/8/1/3-org},

\begin{align}
\norm{\vec{A}-\vec{B}}^{2}
= \norm{\vec{B}-\vec{C}}^2
\end{align}
which means that the adjacent sides of the parallelogram are equal. Thus, the quadrilateral is a rhombus
\iffalse


Since
Let us assume two vectors $\vec{C}-\vec{B}-\vec{A}$ and $\vec{C}-\vec{B}$ for sides $BA$ and $CB$. The diagonals $AC,BD$ are the addition and subtraction of the two vectors:
\begin{align}
	&\vec{(B-A)} = \vec{C}-\vec{B}-\vec{A}\\
	&\vec{(C-B)} = \vec{C}-\vec{B}\\
	&\vec{(C-A)} = \vec{(C-B)} +\vec{(B-A)}\\
	&\vec{(C-A)} = \vec{b+a}\\
	&\vec{(D-B)} = \vec{b-a}\\
\end{align}
%\subsection{Mathematical Calculation:}
Let the two diagonals be $\vec{a+b}$, $\vec{b-a}$. Since the diagonals are at right angle to each other,
\begin{align}
&0 = \vec{(a+b)^T}\vec{(b-a)}\\	
&||\vec{C}-\vec{B}||^2 - ||\vec{C}-\vec{B}-\vec{A}||^2 = 0\\
&||\vec{C}-\vec{B}|| = ||\vec{C}-\vec{B}-\vec{A}||\\
\end{align}
Hence, the two sides of the quadrilateral are equal. We need to prove the third side is also equal.
Now,
In triangle BOA and AOD;
\begin{align}
	&\vec{B-O} = \vec{p}\\
	&\vec{D-O} = \vec{-p}\\
	&\vec{A-O} = \vec{r}\\
\end{align}
\begin{align}
	&\vec{C}-\vec{B}-\vec{A} = \vec{(B-O)} - \vec{(A-O)}\\ 
	&\vec{d} = \vec{(D-O)} - \vec{(A-O)}\\
&||\vec{C}-\vec{B}-\vec{A}||^2 = ||\vec{p}||^2 + ||\vec{r}||^2 - 2\vec{p^Tr}\\
&||\vec{d}||^2 = ||\vec{-p}||^2 + ||\vec{r}||^2 + 2\vec{p^Tr}\\
\end{align}
The terms $\vec{p^Tr}$ is equal to zero as they is perpendicular.Therefore,
\begin{align}
	&||\vec{C}-\vec{B}-\vec{A}||^2 = ||\vec{p}||^2 + ||\vec{r}||^2\\
	&||\vec{d}||^2 = ||\vec{p}||^2 + ||\vec{r}||^2\\
	&Clearly, ||\vec{C}-\vec{B}-\vec{A}|| = ||\vec{d}||\\
\end{align}
Hence, all three sides are equal, it's a parallelogram. A parallelogram with it's diagonals as perpendicular bisectors is a rhombus.
\fi
\iffalse
\section{Construction:}
Consider any  three vertices of the rhombus. Using the vertices, find the midpoint of the diagonals, then find the fourth point using the midpoint and remaining vertex. 
\begin{table}[h]
	\centering
\setlength\extrarowheight{2pt}
	\begin{tabular}{|c|c|c|}
		\hline
		\textbf{variable} & \textbf{length/point} & \textbf{Description}\\
		\hline
		A & [3,0] & Vertex A\\
		\hline
		B & [4,5] & Vertex B\\
		\hline
		C & [-1,4] & Vertex C\\
		\hline                   
		D & [D-x,D-y] & Vertex D\\
		\hline
		M & (A+B)/2 & midpoint\\
		\hline
		(D-x,D-y) & (2*M[0]-B[0],2*M[1]-B[1]) & vertex of D\\
		\hline
	\end{tabular}
\end{table}

\end{document}
\fi

\item 
\iffalse
\documentclass[journal,12pt,twocolumn]{article}
\usepackage{graphicx}
\usepackage[none]{hyphenat}
\usepackage[margin=0.5in]{geometry}
\usepackage[cmex10]{amsmath}
\usepackage{array}
\usepackage{booktabs}
\usepackage{gensymb}
\usepackage{textcomp}
\title{\textbf{Line Assignment}}
\author{Manideep Parusha - FWC22004}
\date{\today}

\providecommand{\norm}[1]{\left\lVert#1\right\rVert}
\providecommand{\abs}[1]{\left\vert#1\right\vert}
\let\vec\mathbf
\newcommand{\myvec}[1]{\ensuremath{\begin{pmatrix}#1\end{pmatrix}}}
\newcommand{\mydet}[1]{\ensuremath{\begin{vmatrix}#1\end{vmatrix}}}
\providecommand{\brak}[1]{\ensuremath{\left(#1\right)}}

\begin{document}

\maketitle
\section*{Problem}
\fi
Show that the diagonals of a square are equal and bisect each other at right angles.
\solution This is obvious from Problems
\eqref{chapters/9/8/1/2}
and
\eqref{chapters/9/8/1/3}.

\iffalse
\section*{Solution}

\begin{figure}[h]
\centering
\includegraphics[width=\columnwidth]{figs/sq_plot.png}
\caption{Square generated using python}
\label{fig:sq_py}
\end{figure}

\subsection*{Construction}
Inputs taken for the construction of the Square is 'a' , which is the side length of the square.
\begin{table}[h]
	\centering
\setlength\extrarowheight{2pt}
	\begin{tabular}{|c|c|c|}
		\hline
		\textbf{Symbol} & \textbf{Value} & \textbf{Description} \\
		\hline
		a & 10 & length of OA\\
		\hline
		O & (0,0) & point O\\
		\hline
		A & (a,0) & point A\\
		\hline
		B & (0,a) & point B\\
		\hline
		C & A+B & point C\\
		\hline
		M & $\frac{C}{2}$ & point M\\
		\hline
	\end{tabular}
\end{table}


Let OABC is a Square. Length of all sides are equal for a square and all interior angles equal to 90\degree.
O at the origin and vectors A, B \& C represent other vertices of the square.
\begin{align}
	\norm{\boldsymbol{OA}} = \norm{\boldsymbol{OB}} = \norm{\boldsymbol{BC}} = \norm{\boldsymbol{AC}} \\
\angle OAC = \angle OBC = \angle BCA = \angle AOB = 90\degree
\label{eq-1}
\end{align}
Here, $D_1$ and $D_2$ are the diagonals of the square and we can compute $D_1$ and $D_2$ as
\begin{equation}
	\boldsymbol{D_1 = (A + B) } \\
	\label{D1eq}
\end{equation}
\begin{equation}
	\boldsymbol{D_2 = (A - B) }
	\label{D2eq}
\end{equation}

To prove that the diagonals of the square are equal, we can find the length of the two diagonals and compare. Hence,
\begin{equation}
\norm {\boldsymbol{D_1}}  = \norm {\boldsymbol{A + B}}\\
	\label{D1_len}
\end{equation}
\begin{equation}
\norm {\boldsymbol{D_2}}  = \norm {\boldsymbol{A - B}}
	\label{D2_len}
\end{equation}
For finding length of $D_1$, we can write from equation \eqref{D1_len},  
\begin{align}
	\norm{\boldsymbol{A + B}} = \sqrt{\norm{\boldsymbol{A}}^2 + \norm{\boldsymbol{B}}^2 + 2\boldsymbol{A}^T\boldsymbol{B}}
	\label{extend_D1}
\end{align}
But, for a square we know that length of all sides are equal.
\begin{equation}
	\norm {\boldsymbol{A}} = \norm{\boldsymbol{B}}
	\label{equal_sides}
\end{equation}
and, the angle between two adjacent sides is 90\degree. The dot product of two vectors which are seperated by 90\degree angle is always '0'. 
\begin{equation}
	\boldsymbol{A}^T\boldsymbol{B} = 0 
	\label{dot_product_is_0}
\end{equation}
So the equation \eqref{extend_D1} becomes 
\begin{align}
	\norm {\boldsymbol{A + B}} = \sqrt{2}\norm{\boldsymbol{A}}\\
	\norm {\boldsymbol{D_1}} = \sqrt{2}\norm{\boldsymbol{A}} 
	\label{D1_length}
\end{align}
Similarly, for finding the length of $D_2$ 
\begin{align}
	\norm {\boldsymbol{A - B}} = \sqrt{\norm{\boldsymbol{A}}^2 + \norm{\boldsymbol{B}}^2 - 2\boldsymbol{A}^T\boldsymbol{B}}
\end{align}
But, from \eqref{equal_sides} and \eqref{dot_product_is_0}
\begin{align}
	\norm {\boldsymbol{A - B}} = \sqrt{2}\norm{\boldsymbol{A}}\\
	\norm {\boldsymbol{D_2}} = \sqrt{2}\norm{\boldsymbol{A}} 
	\label{D2_length}
\end{align}
So, from the equations \eqref{D1_length} and \eqref{D2_length}, we can say that the lengths of diagonals $\boldsymbol{D_1}$ and $\boldsymbol{D_2}$ are equal 
\begin{equation}
	\norm{\boldsymbol{D_1}} = \norm{\boldsymbol{D_2}}
\end{equation}





We know that, if the dot product of two vectors is zero then the vectors are perpendicular to each other. \\
So, by taking the dot product of $\boldsymbol{D_1}$ and $\boldsymbol{D_2}$ 
\begin{align}
	\boldsymbol{D_1.D_2} = \boldsymbol{D_1}^T \boldsymbol{D_2} \\
	\boldsymbol{D_1.D_2} = (\boldsymbol{A + B})^T(\boldsymbol{A - B})\\
	\boldsymbol{D_1.D_2} = \norm{\boldsymbol{A}}^2 - \norm{\boldsymbol{B}}^2
\end{align}	
From the equation \eqref{equal_sides}, 
\begin{align}
	\boldsymbol{D_1.D_2} = \norm{\boldsymbol{A}}^2 - \norm{\boldsymbol{A}}^2 \\
	\boldsymbol{D_1.D_2} = 0 
\end{align}	
as the dot product of the diagonals is equal to 0, we can say that both diagonals are perpendicular to each other. \\

Let diagonals $\boldsymbol{D_1}$ and $\boldsymbol{D_2}$ intersect at a point $\boldsymbol{M}$. We have to prove that $\boldsymbol{M}$ is the mid point of $\boldsymbol{D_1}$ and $\boldsymbol{D_2}$, in order to say that both diagonals bisect eachother.
\begin{align}
	\boldsymbol{OM} = x \boldsymbol{D_1}\\
	\boldsymbol{MA} = y \boldsymbol{D_2}
\end{align}
From the equations \eqref{D1eq} and \eqref{D2eq}, the above equations can be written as
\begin{align}
	\boldsymbol{OM} = x\boldsymbol{A + B}\\
	\boldsymbol{MA} = y\boldsymbol{A - B}
\end{align}
Now, if we consider
\begin{align}
	\boldsymbol{OA} = \boldsymbol{OM} + \boldsymbol{MA}\\
	\boldsymbol{A} = x(\boldsymbol{A+B}) + y(\boldsymbol{A-B})\\
	\boldsymbol{A} = x\boldsymbol{A} + x\boldsymbol{B} + y\boldsymbol{A} - y\boldsymbol{B}\\
	\boldsymbol{A} = (x+y)\boldsymbol{A} + (x-y)\boldsymbol{B}
\end{align}
Equating the co-efficient of $\boldsymbol{A}$ and $\boldsymbol{B}$, we get
\begin{align}
	x + y = 1 , x - y = 0\\
	2x = 1\\
	x = \frac{1}{2}\\
	y = \frac{1}{2}
\end{align}

now we can say that
\begin{align}
	\boldsymbol{OM} = \frac{1}{2} \boldsymbol{D_1}\\
	\boldsymbol{MA} = \frac{1}{2} \boldsymbol{D_2}
\end{align}
Hence, M is the mid point of diagonals $D_1$ and $D_2$ and we can say that both diagonals bisect eachother. \\

we have proved that diagonals of a square are equal in length and bisect eachother at right angles.





\end{document}
\fi

\item 
\item 
\iffalse
\documentclass[a4paper,12pt,twocolumn]{article}
\usepackage{graphicx}
\usepackage[margin=0.5in]{geometry}
\usepackage[cmex10]{amsmath}
\usepackage{array}
\usepackage{gensymb}
\usepackage{booktabs}
\usepackage{tabularx}
\title{Line Assignment}

\author{Ginna Shreyani- FWC22006}
\date{September 2022}
\providecommand{\norm}[1]{\left\lVert#1\right\rVert}
\providecommand{\abs}[1]{\left\vert#1\right\vert}
\let\vec\mathbf
\newcommand{\myvec}[1]{\ensuremath{\begin{pmatrix}#1\end{pmatrix}}}	
\newcommand{\mydet}[1]{\ensuremath{\begin{vmatrix}#1\end{vmatrix}}}
\providecommand{\brak}[1]{\ensuremath{\left((#1\right)}}
\begin{document}
\maketitle
\section{Problem:}
\fi
Diagonal AC of a parallelogram ABCD bisects $\angle{A}$ in Fig \eqref{fig:9/8/1/6}. Show that 
\begin{enumerate}
	\item	it bisects $\angle{C}$ also
	\item $ABCD$ is a rhombus
\end{enumerate}
\begin{figure}[!h]
	\centering
	\includegraphics[width=\columnwidth]{chapters/9/8/1/6/figs/parallel.png}
	\caption{}
	\label{fig:9/8/1/6}
\end{figure}
\solution  
\iffalse
are given by 
\begin{proof}
Any point on the angle bisector is equidistant from the lines.  
\end{proof}
\fi	
%\item ({\em Reflection }) Assuming that straight lines work as a plane mirror for a point, find the image of the point $\vec{P}=\myvec{1\\2}$ in the line 
%%
\iffalse
\maketitle
\section{Construction:}
\begin{tabularx}
{0.5\textwidth}{
|>
{\raggedright\arraybackslash}X
|>
{\centering\arraybackslash}X
|>
{\raggedleft\arraybackslash}X
|}
\hline
 Variable & Point/Length & Description\\
\hline
 A  &  $\myvec{0\\0}$ & Vertex A\\
 \hline
 B & $\myvec{4\\0}$ & Vertex B\\
 \hline
 C & $\myvec{6\\6}$ & Vertex C\\
 \hline
 D & $\myvec{2\\6}$ & Vertex D\\
 \hline
\end{tabularx}
\maketitle
\section{Solution:}
\subsection{Theory:}
Given a diagonal of a parallelogram bisects an angle,we need to prove that the same diagonal bisects the opposite angle of the parallelogram by using vector algebra.\\
\subsection{Mathematical Calculation:}
Here the diagonal joining vertices A and C  can be represented as
\begin{equation}
\vec{C}-\vec{A} = \vec{B}-\vec{A}+\vec{C}-\vec{B}
\end{equation}
The other diagonal joining vertices B and D can be represented as
\begin{equation}
\vec{D-B} = \vec{B}-\vec{C}-\vec{B}-\vec{A}
\end{equation}
$\boldsymbol{(i)}$ Let the $\angle{CAB}$  be $\theta_1$ and the $\angle{DAC}$  be $\theta_2$ and the $\angle{DCA}$  be $\theta_3$ and $\angle{ACB}$ be $\theta_4$\\
Since the diagonal $\vec{C}-\vec{A}$ bisects $\angle{A}$, $\angle{CAB} = \angle{DAC}$,therefore we get $\theta_1=\theta_2$\\
\fi
\begin{enumerate}
	\item From 
    \eqref{eq:angle2d},
\begin{align}
	 \label{eq:9/8/1/6/bis}
	\angle{BAC}
	&= \angle{DAC}
	\\
\implies  \frac{(\vec{A}-\vec{B})^T(\vec{A}-\vec{C})}{\norm{\vec{A}-\vec{B}}\norm{\vec{A}-\vec{C}}}
	& = \frac{(\vec{A}-\vec{D})^T(\vec{A}-\vec{C})}{\norm{\vec{A}-\vec{D}}\norm{\vec{A}-\vec{C}}}
%
%&cos\theta_3 = \frac{\vec{-(B-A)}^T\vec{-(C-A)}}{\vec{||-(B-A)||}.\vec{||-(C-A)||}}\\
%&cos\theta_3 = \frac{\brak{vec{B}-\vec{A}}^T\brak{vec{C}-\vec{A}}}{\vec{||B-A||}.\vec{||C-A||}}\\
%&cos\theta_1 = cos\theta_3\\
%&\theta_1=\theta_3
\end{align}
Also, 
\begin{align}
\cos	\angle{ACD}
	 = \frac{(\vec{C}-\vec{D})^T(\vec{C}-\vec{A})}{\norm{\vec{C}-\vec{D}}\norm{\vec{C}-\vec{A}}}
	 \label{eq:9/8/1/6/cang}
%
\end{align}
From Appendix
	  \ref{eq:two-pgm}, 
  \begin{align}
	  \vec{B}-\vec{A} &= \vec{C} -\vec{D}
	  \\
	  \implies 
	  \frac{(\vec{C}-\vec{D})^T(\vec{C}-\vec{A})}{\norm{\vec{C}-\vec{D}}\norm{\vec{C}-\vec{A}}}
	  &= \frac{(\vec{B}-\vec{A})^T(\vec{C}-\vec{A})}{\norm{\vec{B}-\vec{A}}\norm{\vec{C}-\vec{A}}}
	 \label{eq:9/8/1/6/rh}
  \end{align}
  upon substituting in 
	 \eqref{eq:9/8/1/6/cang}. Thus, from 
	 \eqref{eq:9/8/1/6/cang}
	 and 
	 \eqref{eq:9/8/1/6/bis}, 
\begin{align}
	\angle{BAC}
	= \angle{DAC}
	=
	\angle{ACD}
  \end{align}
  Similarly, it can be shown that 
  \begin{align}
	\angle{ACD}
	=
	\angle{ACB}
  \end{align}
  \item 
  \iffalse
From 
	 \eqref{eq:9/8/1/6/rh}, 
  \begin{align}
	  \frac{(\vec{C}-\vec{D})^T(\vec{C}-\vec{A})}{\norm{\vec{C}-\vec{D}}\norm{\vec{C}-\vec{A}}}
	  &= \frac{(\vec{B}-\vec{A})^T(\vec{C}-\vec{A})}{\norm{\vec{B}-\vec{A}}\norm{\vec{C}-\vec{A}}}
  \end{align}
The equation of the bisector of $\angle BAD$ is given by Appendix  
	\ref{prob:ang-bisect} as
\begin{align}
	 \label{eq:9/8/1/6/ac}
	\frac{\vec{n}_1^{\top}\vec{x} - c_1}{\norm{\vec{n}_1}}
	= 
	\frac{\vec{n}_2^{\top}\vec{x} - c_2}{\norm{\vec{n}_2}}
\end{align}
where the equations of $AB, AD$ are respectively given by 
\begin{align}
	\vec{n}_1^{\top}\vec{x} &= c_1
	\\
	\vec{n}_2^{\top}\vec{x} &= c_2
\end{align}
From 
    \eqref{eq:dir_vec}
    and 
    \eqref{eq:normal_vec}, 
\begin{align}
	 \label{eq:9/8/1/6/orth}
	\vec{n}_1^{\top}\brak{\vec{A}-\vec{B}} &= 0
	\\
	\vec{n}_2^{\top}\brak{\vec{A}-\vec{D}}&= 0
\end{align}
%
From 
	 \eqref{eq:9/8/1/6/ac}, the normal vector of $AC$ is 
\begin{align}
	\frac{\vec{n}_1}{\norm{\vec{n}_1}}
	&- 
	\frac{\vec{n}_2}{\norm{\vec{n}_2}}
	\\
	\implies
	\brak{\frac{\vec{n}_1}{\norm{\vec{n}_1}}
	- 
	\frac{\vec{n}_2}{\norm{\vec{n}_2}}}^{\top}
\brak{\vec{B}-\vec{D}}
	&= 
	\brak{\frac{\vec{n}_1}{\norm{\vec{n}_1}}
	- 
	\frac{\vec{n}_2}{\norm{\vec{n}_2}}}^{\top}
	\sbrak{\brak{\vec{B}-\vec{A}}+
	\brak{\vec{A}-\vec{D}}}
\end{align}
which can be expressed as 
\begin{align}
	{\frac{\vec{n}_1^{\top}}{\norm{\vec{n}_1}}}\brak{\vec{A}-\vec{D}}
	- 
	\frac{\vec{n}_2^{\top}}{\norm{\vec{n}_2}}
	\brak{\vec{B}-\vec{A}}
\end{align}
upon substituting from 
	 \eqref{eq:9/8/1/6/orth}.
	 \fi

\end{enumerate}


\iffalse
Therefore,$\angle{CAB}=\angle{DAC}=\angle{DCA}$\\
Similarly, applying the same process to $\angle{DAC}$ and $\angle{ACB}$, we get $\theta_2=\theta_4$ and as result $\angle{DAC}=\angle{ACB}$.\\
Therefore, $\angle{DCA} = \angle{ACB}$\\
Since both the angles $\angle{DCA}$ and $\angle{ACB}$ are equal, we can conclude that the diagonal $\vec{C}-\vec{A}$ bisects the $\angle{C}$.\\ 
\end{document}
\fi

\item 
\def\mytitle{PARALLELOGRAM}
\def\myauthor{VUNNAVA SRAVANI}
\def\contact{sravani21vunnava@gmail.com}
\def\mymodule{Future Wireless Communication (FWC)}
\documentclass[10pt, a4paper]{article}
\usepackage[a4paper,outer=1.5cm,inner=1.5cm,top=1.75cm,bottom=1.5cm]{geometry}
\twocolumn
\usepackage{setspace}
\doublespacing
\usepackage{graphicx}
\graphicspath{{./images/}}
\usepackage[colorlinks,linkcolor={black},citecolor={blue!80!black},urlcolor={blue!80!black}]{hyperref}
\usepackage[parfill]{parskip}
\usepackage{lmodern}
\usepackage{tikz}
	\usepackage{physics}
%\documentclass[tikz, border=2mm]{standalone}
\usepackage{karnaugh-map}
%\documentclass{article}
\usepackage{tabularx}
\usepackage{circuitikz}
\usetikzlibrary{calc}
\usepackage{amsmath}
\usepackage{amssymb}
\renewcommand*\familydefault{\sfdefault}
\usepackage{watermark}
\usepackage{lipsum}
\usepackage{xcolor}
\usepackage{listings}
\usepackage{float}
\usepackage{titlesec}
\providecommand{\mtx}[1]{\mathbf{#1}}
\titlespacing{\subsection}{1pt}{\parskip}{3pt}
\titlespacing{\subsubsection}{0pt}{\parskip}{-\parskip}
\titlespacing{\paragraph}{0pt}{\parskip}{\parskip}
\newcommand{\figuremacro}[5]{
    \begin{figure}[#1]
        \centering
        \includegraphics[width=#5\columnwidth]{#2}
        \caption[#3]{\textbf{#3}#4}
        \label{fig:#2}
    \end{figure}
}
\newcommand{\myvec}[1]{\ensuremath{\begin{pmatrix}#1\end{pmatrix}}}
\let\vec\mathbf
\lstset{
frame=single, 
breaklines=true,
columns=fullflexible
}

%\thiswatermark{\centering \put(181,-119.0){\includegraphics[scale=0.13]{IIT_logo.png}} }
\title{\mytitle}
\author{\myauthor\hspace{1em}\\\contact\\FWC22012\hspace{6.5em}IITH\hspace{0.5em}\mymodule\hspace{6em}ASSIGN-5}
\date{}
\begin{document}
	\maketitle
	\tableofcontents
   \section{Problem}
  ABCD is a parallelogram and AP and CQ are
perpendiculars from vertices A and C on diagonal
BD . Show that \\
(i) $\Delta APB \cong \Delta CQD$  \\       
(ii) AP = CQ

	   % \includegraphics[scale=1.0]{diag_1.png}
   \section{Solution}

The input parameters for this construction are 
\begin{center}
\begin{tabular}{|c|c|}
	\hline
	\textbf{Symbol}&\textbf{Value}\\
	\hline
	b&6\\
	\hline
	r&5\\
	\hline
	$\theta$&$\frac{\pi}{3}$\\
	\hline
\end{tabular}
\begin{center}
$\vec{A}=\myvec{0\\0}$\\
$\vec{D}=\myvec{r\cos\theta \\ r\sin\theta}$\\
$\vec{B}=\myvec{0\\b}$\\
$\vec{C} = \vec{B}+\vec{C}$
\end{center}
\end{center}
\textbf{To Prove:} AP = CQ
		\begin{center}
		The line equation for diagonal BD is $x = \vec{B}+\lambda\vec{m}$
		\\
		where $\vec{m} = \vec{B}-\vec{D}$\\
		
		then,\\
		
		$\vec{P} = \vec{B} - \frac{\vec{m}^T \vec{B}}{\norm{\vec{m}}^2}\vec{m}$
	\\
	
	$\vec{Q} = \vec{B} - \frac{\vec{m}^T \vec{B-C}}{\norm{\vec{m}}^2}\vec{m}$\\
	\end{center}
	
	distance between A and P is $\norm{\vec{A-P}}$\\
	distance between C and Q is $\norm{\vec{C-Q}}$\\
	if $\norm{\vec{A-P}}$ =  $\norm{\vec{C-Q}}$\\
	then AP = CQ..........(1)
	
	\textbf{To Prove:}  $\Delta APB \cong \Delta$ CQD\\
	to prove $\angle {APD}=\angle {CQD}=90^{\circ}$\\
	$\vec{m1} = \vec{A-P}$\\
	$\vec{m2} = \vec{P-B}$\\
	$\theta= \angle {APD}$ \\
	 $\cos\theta$ = $\frac{\vec{m1}^T \vec{m2}}{\norm{\vec{m1}}\norm{\vec{m2}}}$\\
	$\theta = 90^{\circ}, cos\theta$ = 0\\
	$\therefore m1^T m2 = 0$\\
	$\vec{n1} = \vec{C-Q}$\\
	$\vec{n2} = \vec{Q-D}$\\
	$\theta = \angle{CQD}$\\
	$cos\theta$ = $\frac{\vec{n1}^T \vec{n2}}{\norm{\vec{n1}}\norm{\vec{n2}}}$\\
	f $\theta$ = 90$^{\circ}, cos\theta$ = 0\\
	$\therefore n1^T n2 = 0$\\
	\begin{center}
	if 	$m1^T m2 = n1^T n2$ = 0\\
	then, $\angle {APD} = \angle {CQD} = 90^{\circ}$..........(2)\\
	\end{center}
	to prove $\angle {ABP}=\angle {CDQ}$ \\
	$\vec{m2} = \vec{P-B}$\\
	$\vec{m3} = \vec{A-B}$\\
	$\theta1 = \angle {ABP}$\\
	$\theta1 = \cos^-1\frac{\vec{m2} \cdot \vec{m3}}{\norm{\vec{m2}}\norm{\vec{m3}}}$\\
	$\vec{n2} = \vec{C-D}$\\
	$\vec{n3} = \vec{Q-D}$\\
	$\theta2 = \angle {CDQ}$\\
	$\theta2 = \cos^-1\frac{\vec{n2} \cdot \vec{n3}}{\norm{\vec{n2}}\norm{\vec{n3}}}$\\
	\begin{center}
	 		if $\theta1 = \theta2$\\
	 		then $\angle {ABP} = \angle {CQD}$..........(3)
	\end{center}
	\begin{center}
$\therefore$ from (1),(2) and (3)
$\Delta APB \cong \Delta CQD$ 
	\end{center}
The below python code realizes the above construction:	\\
\begin{lstlisting}
https://github.com/sravani21vunnava/sravani21vunnava/blob/main/Matrices_line/codes/matrix_line.py
\end{lstlisting}
 
\section{Construction}
\includegraphics[scale=0.66]{matrix_line.png}
 	
\bibliographystyle{ieeetr}
\end{document}
\item 
\item 
\iffalse
\def\mytitle{PYTHON PROGRAMMING ON MATRICES}
\def\myauthor{K.Pavan Kumar}
\def\contact{r170850@rguktrkv.ac.in}
\def\mymodule{Future Wireless Communication (FWC)}
\documentclass[10pt, a4paper]{article}
\usepackage[a4paper,outer=1.5cm,inner=1.5cm,top=1.75cm,bottom=1.5cm]{geometry}
\twocolumn
\usepackage{graphicx}
\graphicspath{{./images/}}
\usepackage[colorlinks,linkcolor={black},citecolor={blue!80!black},urlcolor={blue!80!black}]{hyperref}
\usepackage[parfill]{parskip}
\usepackage{lmodern}
\usepackage{tikz}
	\usepackage{physics}
\usepackage{tabularx}
\usepackage{enumitem}
\usetikzlibrary{calc}
\usepackage{amsmath}
\usepackage{amssymb}
\renewcommand*\familydefault{\sfdefault}
\usepackage{watermark}
\usepackage{lipsum}
\usepackage{xcolor}
\usepackage{listings}
\usepackage{float}
\usepackage{titlesec}
\providecommand{\mtx}[1]{\mathbf{#1}}
\titlespacing{\subsection}{1pt}{\parskip}{3pt}
\titlespacing{\subsubsection}{0pt}{\parskip}{-\parskip}
\titlespacing{\paragraph}{0pt}{\parskip}{\parskip}


\newcommand{\myvec}[1]{\ensuremath{\begin{pmatrix}#1\end{pmatrix}}}
\let\vec\mathbf
\lstset{
frame=single, 
breaklines=true,
columns=fullflexible
}
\thiswatermark{\centering \put(0,-110.0){\includegraphics[scale=0.3]{logo.png}} }
\title{\mytitle}
\author{\myauthor\hspace{1em}\\\contact\\FWC22011\hspace{6.5em}IITH\hspace{0.5em}\mymodule\hspace{6em}Matrix:Line}
\date{}
\begin{document}
	\maketitle
	\tableofcontents
	\fi

In parallelogram $ABCD$, two points $\vec{P}$ and $\vec{Q}$ are
taken on diagonal $BD$ such that $DP = BQ$. Show that 
\begin{enumerate}
	\item  $\triangle APD \cong \triangle CQB$         
	\item  $AP = CQ$
	\item $\triangle AQB \cong \triangle CPD$     
	\item  $AQ = CP$   
	\item  $APCQ$ is a parallelogram 
\end{enumerate}
\solution 
See Fig. 
		\ref{fig:9/8/1/9}.
 	\begin{figure}
		\centering
 \includegraphics[width=\columnwidth]{chapters/9/8/1/9/figs/output.pdf}
		\caption{}
		\label{fig:9/8/1/9}
  	\end{figure}

\iffalse

\section{Construction}
  	\begin{center}
  Figure of construction
  	\end{center}

   
  \section{Solution}
\begin{center}
The input parameters for this construction are
\begin{tabular}{|c|c|}
	\hline
	\textbf{Symbol}&\textbf{Value}\\
	\hline
	r&5\\
	\hline
	k&3\\
	\hline
    b&4\\
	\hline
	$\theta$&$\frac{pi}{3}$\\
	\hline
\end{tabular}
\end{center}

\begin{align*}
\vec{A}=\begin{pmatrix} r\cos\theta\\ r\sin\theta\ \end{pmatrix} \\
\vec{B}=\begin{pmatrix} 0\\ 0\ \end{pmatrix} \\
\vec{C}=\begin{pmatrix} b\\ 0\ \end{pmatrix} \\
\vec{D}={\vec{A}+\vec{C}-\vec{B}} \\
\vec{P} =  \frac{\vec{B} +K\times \vec{D}}{1+K} \\
\vec{Q} =  \frac{K\times\vec{B} +\vec{D}}{1+K} \\ 
\end{align*}


\textbf{Theorem}\\
A quadrilateral is a parallelogram if a pair of opposite sides
is equal and parallel.

"If two directional vectors are equal,implies their magnitude as well as direction are equal to each other."

Two vectors are parallel if they have the same direction (or) are in exactly opposite directions.

\paragraph{Given} ABCD is a parallelogram.
 ,the two points $\vec{P}$ and $\vec{Q}$ are taken on diagonal BD such that DP = BQ.

\fi
From 
    \eqref{eq:angle2d} and the given information,

\begin{align}
		\label{fig:9/8/1/9/pgm}
	\vec{A}-\vec{B} &=\vec{D}-\vec{C} \\
	\implies    \vec{A}-\vec{D} &=\vec{B}-\vec{C}\\
	\vec{B}-\vec{Q} &=\vec{P}-\vec{D} \quad \text{(given)}
		\label{fig:9/8/1/9/newpgm}
\end{align}

From 
		\eqref{fig:9/8/1/9/pgm}
		and
		\eqref{fig:9/8/1/9/newpgm}
		\iffalse
\begin{align}
\begin{split}
    \vec{A}+\vec{C} =\vec{B}+\vec{D}\\
    \vec{P}+\vec{Q} =\vec{B}+\vec{D}
\end{split}
\end{align}
%From (4)
\begin{align}
    \vec{A}+\vec{C} =\vec{P}+\vec{Q}
\end{align}

From (5)
\fi
\begin{align}
%     \implies  \vec{A}-\vec{Q} =\vec{P}-\vec{C}\\
    \vec{A}-\vec{P} =\vec{Q}-\vec{C}
		\label{fig:9/8/1/9/newpgmsol}
\end{align}


\begin{enumerate}
    \item From 
		\eqref{fig:9/8/1/9/pgm}, 
		\eqref{fig:9/8/1/9/newpgm}
		and 
		\eqref{fig:9/8/1/9/newpgmsol}
		taking the norms of the respective sides, 
    \begin{align}
        \triangle APD \cong \triangle CQB
    \end{align}
    
    \item From 
		\eqref{fig:9/8/1/9/newpgmsol}, taking the norm,
	    \begin{align}
AP=CQ
    \end{align}
    
    \item From 
		\eqref{fig:9/8/1/9/pgm}, 
		\eqref{fig:9/8/1/9/newpgm}
		and 
		\eqref{fig:9/8/1/9/newpgmsol}
		taking the norms of the respective sides, 
    \begin{align}
        \triangle AQB \cong \triangle CPD
    \end{align}

    \item From 
		\eqref{fig:9/8/1/9/newpgmsol}, 
	    \begin{align}
AQ=CP
    \end{align}
\end{enumerate}

\iffalse
     \item Equation (6) and (7) 
      $\implies$  Quadrilateral APCQ is a parallelogram.


\textbf{termux commands :}
\begin{lstlisting}
bash lines.sh............using shell command
\end{lstlisting}
\begin{center}
Below python code realizes the above construction :
\fbox{\parbox{8.5cm}{\url{https://github.com/pavan170850/Fwciith2022/blob/main/matrices/line/code/Line.py}}}
\end{center}
\end{document}
\fi




\end{enumerate}

%\subsection{Mid Point Theorem}
%\begin{enumerate}[label=\thesection.\arabic*,ref=\thesection.\theenumi]
\numberwithin{equation}{enumi}
\numberwithin{figure}{enumi}
\numberwithin{table}{enumi}

\item 
\label{chapters/12/8/2/1}

\documentclass[10pt,a4paper]{report}
%\usepackage[latin1]{inputenc}
\usepackage[utf8]{inputenc}
\usepackage{amsmath}
\usepackage{amsfonts}
\usepackage{amssymb}
\usepackage{graphicx}
\usepackage{multicol}
\usepackage{tabularx}
\usepackage{tikz}
\usetikzlibrary{arrows,shapes,automata,petri,positioning,calc}
\usepackage{hyperref}
\usepackage{tikz}
\usetikzlibrary{matrix,calc}
\usepackage[margin=0.5in]{geometry}
% ---- power functions -----% 
\newcommand{\myvec}[1]{\ensuremath{\begin{pmatrix}#1\end{pmatrix}}}
\let\vec\mathbf

\providecommand{\norm}[1]{\left\lVert#1\right\rVert}
\providecommand{\abs}[1]{\left\vert#1\right\vert}
\let\vec\mathbf

\newcommand{\mydet}[1]{\ensuremath{\begin{vmatrix}#1\end{vmatrix}}}
\providecommand{\brak}[1]{\ensuremath{\left(#1\right)}}
\providecommand{\lbrak}[1]{\ensuremath{\left(#1\right.}}
\providecommand{\rbrak}[1]{\ensuremath{\left.#1\right)}}
\providecommand{\sbrak}[1]{\ensuremath{{}\left[#1\right]}}
%-------end power functions----%
\newenvironment{Figure}
  {\par\medskip\noindent\minipage{\linewidth}}
  {\endminipage\par\medskip}
\begin{document}
%--------------------logo figure-------------------------%
\begin{figure*}[!tbp]
  \centering
  \begin{minipage}[b]{0.4\textwidth}
    \includegraphics[scale = 0.05]{iitlogo.jpg}
  \end{minipage}
  \hfill
  \vspace{5mm}\begin{minipage}[b]{0.4\textwidth}
\raggedleft  \includegraphics[scale = 0.10]{nrc.png}\

  \end{minipage}\vspace{0.2cm}
\end{figure*}
%--------------------name & rollno-----------------------
\raggedright \textbf{Name}:\hspace{1mm} Chirag Shah\hspace{3cm} \Large \textbf{Assignment-6}\hspace{2.5cm} % 
\normalsize \textbf{Roll No.} :\hspace{1mm} FWC22053\vspace{1cm}
\begin{multicols}{2}

%----------------problem statement--------------%
\raggedright \textbf{Problem Statement:}\vspace{2mm}
\raggedright \\ Find the area of the circle $4x^2+4y^2=9$ which is interior to the parabola $x^2=4y$.\\
\vspace{5mm}
%-----------------------------solution---------------------------
\raggedright \textbf{SOLUTION}:\vspace{2mm}\\

%---------given----------------%
\raggedright \textbf{Given}:\vspace{2mm}\\
Equation of circle is \\\vspace{1mm}
\begin{align}
4x^2+4y^2=9
\end{align}
Equation of Parabola is \\ \vspace{1mm}
\begin{align}
x^2=4y 
\end{align}
From (2) we can say that Parabola is concave towards positive y axis.\\ \vspace{2mm}
From equation (1) radius of circle is,\\ \vspace{1mm}
\begin{align}
r= \frac{3}{2}
\end{align}

%-------------To find ------------------%
\textbf{To Find }\vspace{2mm}\\
To find the intersection points and area of shaded region shown in figure\vspace{2mm}  \\ 
%--------------steps----------------------%
\textbf{STEP-1}\vspace{2mm}\\
The given circle and parabola can be expressed as conics with parameters,\\ \vspace{1mm}
For circle,\\ \vspace{1mm}
\begin{align}
\vec{V}_1=4\vec{I}
\end{align}
So, \\
\begin{align}
\vec{V}_1=\myvec{
4 & 0\\
0 & 4
}
\end{align} 

\begin{align}
\vec{u_1}=0
\end{align} 
\begin{align}
f_1=-9
\end{align} \vspace{2mm}

For Parabola,\\\vspace{1mm}
\begin{align}
\vec{V}_2=\myvec{
1 & 0\\
0 & 0
}
\end{align} 

\begin{align}
\vec{u_2}= -\myvec{
0\\
2
}
\end{align} 
\begin{align}
f_2=0
\end{align} \vspace{2mm}

\textbf{STEP-2}\vspace{2mm}\\
The intersection of the given conics is obtained
as\\
\begin{align}
	\vec{x}^{\top}\brak{\vec{V}_1 + \mu\vec{V}_2}\vec{x}+2 \brak{\vec{u}_1+\mu \vec{u}_2}^{\top} \vec{x} 
	\\
	+ \brak{f_1+\mu f_2}= 0
    \end{align}
    
\begin{align}
\vec{V}_1+\mu\vec{V}_2= \myvec{
\mu+4 & 0\\
0 & 4
}
\end{align}
\begin{align}
\vec{u}_1+\mu\vec{u}_2= -\myvec{
0\\
2\mu
}
\end{align}
\begin{align}
f_1+\mu f_2= -9
\end{align}
This conic is a single straight line if and only if, \\ \vspace{1mm}
\begin{align}
\mydet{\vec{V}_1 + \mu\vec{V}_2 & \vec{u}_1+\mu \vec{u}_2\\ \brak{\vec{u}_1+\mu \vec{u}_2}^{\top} & f_1 + \mu f_2} &= 0
\end{align}
And,\\
\begin{align}
\mydet{\vec{V}_1 + \mu\vec{V}_2} &= 0
\end{align}
Substituting equation (13),(14) and (15) in equation (16)\\ \vspace{1mm}
We get,\\ \vspace{1mm}
\begin{align}
\implies \mydet{\mu+4 & 0 & 0\\ 
0 & 4 & -2\mu \\
0 & -2\mu & -9
} &= 0
\end{align}
Solving the above equation we get,\\ \vspace{1mm}
\begin{align}
\mu^3 + 4\mu^2 + 9\mu + 36=0
\end{align}
gives,\\ \vspace{1mm}
\begin{align}
    \mu = -4
\end{align}
 Thus, the parameters for a straight line can be expressed as\\ \vspace{1mm}
 \begin{align}
	\vec{V} &= 
\vec{V}_1 + \mu\vec{V}_2
=\myvec{ 0 & 0 \\ 0 & 4},
\\
	\vec{u} &=
\vec{u}_1+\mu \vec{u}_2
	= \myvec{
0\\
8
    }
\\
	f&=-9,
	\\
	\implies \vec{D} &= \vec{V}, \vec{P} = \vec{I}
    \end{align}
Thus, the desired pair of straight lines are \\ 
\begin{align} 
	\myvec{\sqrt{\abs{\lambda_1}} & \pm \sqrt{\abs{\lambda_2}}}\vec{P}^{\top}\brak{\vec{x}-\vec{c}} &= 0
\end{align}
\begin{align}
	\implies\myvec{0 & \pm 2}\vec{x}-\vec{c} &= 0
\end{align}
\begin{align}
	\text{or, }\vec{x} =\vec{c} + \kappa \myvec{\pm 2 \\ 0}
\end{align} 
\textbf{STEP-3}\vspace{2mm}\\
The points of intersection of the line is given by, \\ 
\begin{align}
L: \quad \vec{x} = \vec{q} + \kappa \vec{m} \quad \kappa \in \mathbb{R}
\end{align}
with the conic section, \\ 
\begin{align}
	\vec{x}^{\top}\vec{V}\vec{x} + 2\vec{u}^{\top} \vec{x} + f = 0
\end{align}
are given by \\
\begin{align}
\vec{x}_i = \vec{q} + \kappa_i \vec{m}
\end{align}
where, \\
{\tiny
\begin{multline}
\kappa_i = \frac{1}
{
\vec{m}^T\vec{V}\vec{m}
}
\lbrak{-\vec{m}^T\brak{\vec{V}\vec{q}+\vec{u}}}
\\
\pm
\rbrak{\sqrt{
\sbrak{
\vec{m}^T\brak{\vec{V}\vec{q}+\vec{u}}
}^2
-
\brak
{
\vec{q}^T\vec{V}\vec{q} + 2\vec{u}^T\vec{q} +f
}
\brak{\vec{m}^T\vec{V}\vec{m}}
}
}
\end{multline}
}
On substituting\\
\begin{align}
\vec{q} &= \myvec{
0\\
0.5
} 
\end{align}
\begin{align}
\vec{m} = \myvec{2 \\ 0}
\end{align}
With the given Parabola,\\ 
\begin{align}
	\vec{V} &= \myvec{
1 & 0\\
0 & 0
    }
\end{align}
\begin{align}
	\vec{u} = -\myvec{2 \\0}
 \end{align}
 \begin{align}
  f = 0
 \end{align}
The value of $\kappa$ ,\\
\begin{align}
    \kappa = \sqrt{2},-\sqrt{2}
\end{align}
The points of intersection with Parabola along circle are \\
\begin{align}
    \vec{A}=\myvec{
\sqrt{2}\\
0.5
    }
\end{align}
\begin{align}
    \vec{B}=\myvec{
-\sqrt{2}\\
0.5
    }
\end{align}
\textbf{Result}
\begin{center}
 \includegraphics[width=0.5\textwidth]{conic.jpg}  
 \end{center}\vspace{1mm}
 From the figure,\\ \vspace{1mm}
Total area of portion is given by, \\ \vspace{1mm}
\begin{align}
 A=  \int_{-\sqrt{2}}^{\sqrt{2}} g(x)-f(x) \,dx 
\end{align}
Where g(x) is area of circle and f(x) is the area of parabola around the points\\ \vspace{1mm}
\begin{align}
A= \int_{-\sqrt{2}}^{\sqrt{2}} \frac{\sqrt{9-4x^2}}{2}-\frac{x^2}{4} \,dx 
\end{align}
Area A is,\\ 
\begin{align}
    A= 3.0053609 \,m^2
\end{align}
 \vspace{2mm} \textbf{Construction}
\begin{center}
\setlength{\arrayrulewidth}{0.5mm}
\setlength{\tabcolsep}{6pt}
\renewcommand{\arraystretch}{1.5}
    \begin{tabular}{|l|c|}
    \hline 
    \textbf{Points} & \textbf{coordinates} \\ \hline
   $\vec{A}$ & $\myvec{
   \sqrt{2}\\
   0.5
   } $ \\\hline
   $\vec{B}$ & $\myvec{
   -\sqrt{2}\\
   0.5
   } $ \\\hline
      \end{tabular}
  \end{center}

\raggedright  Download the code \\
Github link: \href{https://github.com/chiragshah1244/FWC/blob/main/assignments/assignment_6/code_conic/conic.py}{Assignment-6}.
  \end{multicols}
\end{document}

\item 
\label{chapters/12/8/2/2}
%\iffalse
\documentclass[journal,10pt,twocolumn]{article}
\usepackage{graphicx}
\usepackage[margin=0.5in]{geometry}
\usepackage[cmex10]{amsmath}
\usepackage{array}
\usepackage{booktabs}
\usepackage{mathtools}
\title{\textbf{Conic section Assignment}}
\author{Jyothsna Paluchuri}
\date{September 2022}


\providecommand{\norm}[1]{\left\lVert#1\right\rVert}
\providecommand{\abs}[1]{\left\vert#1\right\vert}
\let\vec\mathbf
\newcommand{\myvec}[1]{\ensuremath{\begin{pmatrix}#1\end{pmatrix}}}
\newcommand{\mydet}[1]{\ensuremath{\begin{vmatrix}#1\end{vmatrix}}}
\providecommand{\brak}[1]{\ensuremath{\left(#1\right)}}
\providecommand{\lbrak}[1]{\ensuremath{\left(#1\right.}}
\providecommand{\rbrak}[1]{\ensuremath{\left.#1\right)}}
\providecommand{\sbrak}[1]{\ensuremath{{}\left[#1\right]}}

\begin{document}

\maketitle
\paragraph{\textit{Problem Statement} -
\fi
Find the area of the region bounded by the curve $x^2=4y$ and the lines y=2 and y=4 and the y-axis in the first quadrant.
\\
\solution
	\begin{figure}[!h]
		\centering
 \includegraphics[width=\columnwidth]{chapters/12/8/3/3/figs/conic.png}
		\caption{}
		\label{fig:12/8/3/3}
  	\end{figure}
\iffalse

\section*{\large Solution}

\begin{figure}[h]
\centering
\includegraphics[width=1\columnwidth]{conic.png}

\caption{The parabola formed by the curve $x^2 = 4y$ and the lines y=2 and y=4}
\label{fig:parabola}
\end{figure}

The given equation of parabola $x^2 = 4y$ can be written in the general quadratic form as
\begin{align}
    \label{eq:conic_quad_form}
    \vec{x}^{\top}\vec{V}\vec{x}+2\vec{u}^{\top}\vec{x}+f=0
    \end{align}
where
\fi
The conic parameters are
\begin{align}
	\vec{V} = \myvec{1 & 0\\0 & 0},
	\vec{u} = \myvec{0\\-2},
	f = 0
	%\\
\end{align}
\iffalse
The point of intersection of the lines y=2 and y=4 to the parabola is given by



The points of intersection of the line 
\begin{align}
	L: \quad \vec{x} = \vec{q} + \mu \vec{m} \quad \mu \in \mathbf{R}
\label{eq:conic_tangent}
\end{align}
with the conic section are given by
\begin{align}
\vec{x}_i = \vec{q} + \mu_i \vec{m}
\label{eq:conic_tangent_pts}
\end{align}
%
where
{\tiny
\begin{multline}
\mu_i = \frac{1}
{
\vec{m}^T\vec{V}\vec{m}
}
\lbrak{-\vec{m}^T\brak{\vec{V}\vec{q}+\vec{u}}}
\\
\pm
\rbrak{\sqrt{
\sbrak{
\vec{m}^T\brak{\vec{V}\vec{q}+\vec{u}}
}^2
-
\brak
{
\vec{q}^T\vec{V}\vec{q} + 2\vec{u}^T\vec{q} +f
}
\brak{\vec{m}^T\vec{V}\vec{m}}
}
}
\label{eq:tangent_roots}
\end{multline}
}


\fi
The vector parameters of 
$y-4=0$
are
\begin{align}
	\vec{h}_1=\myvec{0\\4},
	\vec{m}_1=\myvec{1\\0}
\end{align}
Substituting the above in \eqref{eq:tangent_roots},
\begin{align}
\mu_i=4,-4
\end{align}
yielding
the points of intersection with the parabola as
\begin{align}
\vec{a}_0=\myvec{4\\4},
\vec{a}_1=\myvec{-4\\4}
\end{align}
Similarly, for 
the line $y-2=0$, the vector parameters are
\begin{align}
\vec{h}_2=\myvec{0\\2},
\vec{m}_2=\myvec{1\\0}
\end{align}
yielding 
\begin{align}
\mu_i=2.8,-2.8
\end{align}
and the points of intersection
\begin{align}
\vec{a}_2=\myvec{2.8\\2},
\vec{a}_3=\myvec{-2.8\\2}
\end{align}
From Fig.
		\ref{fig:12/8/3/3},
the area of the parabola between the lines $y=2$ and $y=4$ is given by
\begin{align}
\int_{0}^{4} \ 2\sqrt{y} \,dy-\int_{0}^{2} \ 2\sqrt{y} \,dy
=6.895 
\end{align}
\iffalse


\section*{\large Construction}

{
\setlength\extrarowheight{5pt}
\begin{tabular}{|l|c|}
    \hline 
    \textbf{Points} & \textbf{intersection points} \\ \hline
	a0 & $\myvec{
   -2.8\\
   2
   } $ \\\hline
	a1 & $\myvec{
   2.8\\
   2
   } $ \\\hline
    
	a3 & $\myvec{
   -4\\
   4
   } $ \\\hline
	a2 & $\myvec{
   4\\
   4
   } $ \\\hline
      
      \end{tabular}
}

\end{document}
\fi

\item 
\label{chapters/12/8/2/3}
\iffalse
\documentclass[journal,10pt,twocolumn]{article}
\usepackage{graphicx}
\usepackage[margin=0.5in]{geometry}
\usepackage[cmex10]{amsmath}
\usepackage{array}
\usepackage{booktabs}
\usepackage{mathtools}
\title{\textbf{Conic section Assignment}}
\author{Jyothsna Paluchuri}
\date{September 2022}


\providecommand{\norm}[1]{\left\lVert#1\right\rVert}
\providecommand{\abs}[1]{\left\vert#1\right\vert}
\let\vec\mathbf
\newcommand{\myvec}[1]{\ensuremath{\begin{pmatrix}#1\end{pmatrix}}}
\newcommand{\mydet}[1]{\ensuremath{\begin{vmatrix}#1\end{vmatrix}}}
\providecommand{\brak}[1]{\ensuremath{\left(#1\right)}}
\providecommand{\lbrak}[1]{\ensuremath{\left(#1\right.}}
\providecommand{\rbrak}[1]{\ensuremath{\left.#1\right)}}
\providecommand{\sbrak}[1]{\ensuremath{{}\left[#1\right]}}

\begin{document}

\maketitle
\paragraph{\textit{Problem Statement} -
\fi
Find the area of the region bounded by the curve $x^2=4y$ and the lines y=2 and y=4 and the y-axis in the first quadrant.
\\
\solution
	\begin{figure}[!h]
		\centering
 \includegraphics[width=\columnwidth]{chapters/12/8/3/3/figs/conic.png}
		\caption{}
		\label{fig:12/8/3/3}
  	\end{figure}
\iffalse

\section*{\large Solution}

\begin{figure}[h]
\centering
\includegraphics[width=1\columnwidth]{conic.png}

\caption{The parabola formed by the curve $x^2 = 4y$ and the lines y=2 and y=4}
\label{fig:parabola}
\end{figure}

The given equation of parabola $x^2 = 4y$ can be written in the general quadratic form as
\begin{align}
    \label{eq:conic_quad_form}
    \vec{x}^{\top}\vec{V}\vec{x}+2\vec{u}^{\top}\vec{x}+f=0
    \end{align}
where
\fi
The conic parameters are
\begin{align}
	\vec{V} = \myvec{1 & 0\\0 & 0},
	\vec{u} = \myvec{0\\-2},
	f = 0
	%\\
\end{align}
\iffalse
The point of intersection of the lines y=2 and y=4 to the parabola is given by



The points of intersection of the line 
\begin{align}
	L: \quad \vec{x} = \vec{q} + \mu \vec{m} \quad \mu \in \mathbf{R}
\label{eq:conic_tangent}
\end{align}
with the conic section are given by
\begin{align}
\vec{x}_i = \vec{q} + \mu_i \vec{m}
\label{eq:conic_tangent_pts}
\end{align}
%
where
{\tiny
\begin{multline}
\mu_i = \frac{1}
{
\vec{m}^T\vec{V}\vec{m}
}
\lbrak{-\vec{m}^T\brak{\vec{V}\vec{q}+\vec{u}}}
\\
\pm
\rbrak{\sqrt{
\sbrak{
\vec{m}^T\brak{\vec{V}\vec{q}+\vec{u}}
}^2
-
\brak
{
\vec{q}^T\vec{V}\vec{q} + 2\vec{u}^T\vec{q} +f
}
\brak{\vec{m}^T\vec{V}\vec{m}}
}
}
\label{eq:tangent_roots}
\end{multline}
}


\fi
The vector parameters of 
$y-4=0$
are
\begin{align}
	\vec{h}_1=\myvec{0\\4},
	\vec{m}_1=\myvec{1\\0}
\end{align}
Substituting the above in \eqref{eq:tangent_roots},
\begin{align}
\mu_i=4,-4
\end{align}
yielding
the points of intersection with the parabola as
\begin{align}
\vec{a}_0=\myvec{4\\4},
\vec{a}_1=\myvec{-4\\4}
\end{align}
Similarly, for 
the line $y-2=0$, the vector parameters are
\begin{align}
\vec{h}_2=\myvec{0\\2},
\vec{m}_2=\myvec{1\\0}
\end{align}
yielding 
\begin{align}
\mu_i=2.8,-2.8
\end{align}
and the points of intersection
\begin{align}
\vec{a}_2=\myvec{2.8\\2},
\vec{a}_3=\myvec{-2.8\\2}
\end{align}
From Fig.
		\ref{fig:12/8/3/3},
the area of the parabola between the lines $y=2$ and $y=4$ is given by
\begin{align}
\int_{0}^{4} \ 2\sqrt{y} \,dy-\int_{0}^{2} \ 2\sqrt{y} \,dy
=6.895 
\end{align}
\iffalse


\section*{\large Construction}

{
\setlength\extrarowheight{5pt}
\begin{tabular}{|l|c|}
    \hline 
    \textbf{Points} & \textbf{intersection points} \\ \hline
	a0 & $\myvec{
   -2.8\\
   2
   } $ \\\hline
	a1 & $\myvec{
   2.8\\
   2
   } $ \\\hline
    
	a3 & $\myvec{
   -4\\
   4
   } $ \\\hline
	a2 & $\myvec{
   4\\
   4
   } $ \\\hline
      
      \end{tabular}
}

\end{document}
\fi

\item 
\label{chapters/12/8/2/4}
%\iffalse
\documentclass[journal,10pt,twocolumn]{article}
\usepackage{graphicx}
\usepackage[margin=0.5in]{geometry}
\usepackage[cmex10]{amsmath}
\usepackage{array}
\usepackage{booktabs}
\usepackage{mathtools}
\title{\textbf{Conic section Assignment}}
\author{Jyothsna Paluchuri}
\date{September 2022}


\providecommand{\norm}[1]{\left\lVert#1\right\rVert}
\providecommand{\abs}[1]{\left\vert#1\right\vert}
\let\vec\mathbf
\newcommand{\myvec}[1]{\ensuremath{\begin{pmatrix}#1\end{pmatrix}}}
\newcommand{\mydet}[1]{\ensuremath{\begin{vmatrix}#1\end{vmatrix}}}
\providecommand{\brak}[1]{\ensuremath{\left(#1\right)}}
\providecommand{\lbrak}[1]{\ensuremath{\left(#1\right.}}
\providecommand{\rbrak}[1]{\ensuremath{\left.#1\right)}}
\providecommand{\sbrak}[1]{\ensuremath{{}\left[#1\right]}}

\begin{document}

\maketitle
\paragraph{\textit{Problem Statement} -
\fi
Find the area of the region bounded by the curve $x^2=4y$ and the lines y=2 and y=4 and the y-axis in the first quadrant.
\\
\solution
	\begin{figure}[!h]
		\centering
 \includegraphics[width=\columnwidth]{chapters/12/8/3/3/figs/conic.png}
		\caption{}
		\label{fig:12/8/3/3}
  	\end{figure}
\iffalse

\section*{\large Solution}

\begin{figure}[h]
\centering
\includegraphics[width=1\columnwidth]{conic.png}

\caption{The parabola formed by the curve $x^2 = 4y$ and the lines y=2 and y=4}
\label{fig:parabola}
\end{figure}

The given equation of parabola $x^2 = 4y$ can be written in the general quadratic form as
\begin{align}
    \label{eq:conic_quad_form}
    \vec{x}^{\top}\vec{V}\vec{x}+2\vec{u}^{\top}\vec{x}+f=0
    \end{align}
where
\fi
The conic parameters are
\begin{align}
	\vec{V} = \myvec{1 & 0\\0 & 0},
	\vec{u} = \myvec{0\\-2},
	f = 0
	%\\
\end{align}
\iffalse
The point of intersection of the lines y=2 and y=4 to the parabola is given by



The points of intersection of the line 
\begin{align}
	L: \quad \vec{x} = \vec{q} + \mu \vec{m} \quad \mu \in \mathbf{R}
\label{eq:conic_tangent}
\end{align}
with the conic section are given by
\begin{align}
\vec{x}_i = \vec{q} + \mu_i \vec{m}
\label{eq:conic_tangent_pts}
\end{align}
%
where
{\tiny
\begin{multline}
\mu_i = \frac{1}
{
\vec{m}^T\vec{V}\vec{m}
}
\lbrak{-\vec{m}^T\brak{\vec{V}\vec{q}+\vec{u}}}
\\
\pm
\rbrak{\sqrt{
\sbrak{
\vec{m}^T\brak{\vec{V}\vec{q}+\vec{u}}
}^2
-
\brak
{
\vec{q}^T\vec{V}\vec{q} + 2\vec{u}^T\vec{q} +f
}
\brak{\vec{m}^T\vec{V}\vec{m}}
}
}
\label{eq:tangent_roots}
\end{multline}
}


\fi
The vector parameters of 
$y-4=0$
are
\begin{align}
	\vec{h}_1=\myvec{0\\4},
	\vec{m}_1=\myvec{1\\0}
\end{align}
Substituting the above in \eqref{eq:tangent_roots},
\begin{align}
\mu_i=4,-4
\end{align}
yielding
the points of intersection with the parabola as
\begin{align}
\vec{a}_0=\myvec{4\\4},
\vec{a}_1=\myvec{-4\\4}
\end{align}
Similarly, for 
the line $y-2=0$, the vector parameters are
\begin{align}
\vec{h}_2=\myvec{0\\2},
\vec{m}_2=\myvec{1\\0}
\end{align}
yielding 
\begin{align}
\mu_i=2.8,-2.8
\end{align}
and the points of intersection
\begin{align}
\vec{a}_2=\myvec{2.8\\2},
\vec{a}_3=\myvec{-2.8\\2}
\end{align}
From Fig.
		\ref{fig:12/8/3/3},
the area of the parabola between the lines $y=2$ and $y=4$ is given by
\begin{align}
\int_{0}^{4} \ 2\sqrt{y} \,dy-\int_{0}^{2} \ 2\sqrt{y} \,dy
=6.895 
\end{align}
\iffalse


\section*{\large Construction}

{
\setlength\extrarowheight{5pt}
\begin{tabular}{|l|c|}
    \hline 
    \textbf{Points} & \textbf{intersection points} \\ \hline
	a0 & $\myvec{
   -2.8\\
   2
   } $ \\\hline
	a1 & $\myvec{
   2.8\\
   2
   } $ \\\hline
    
	a3 & $\myvec{
   -4\\
   4
   } $ \\\hline
	a2 & $\myvec{
   4\\
   4
   } $ \\\hline
      
      \end{tabular}
}

\end{document}
\fi

\item 
\label{chapters/12/8/2/5}
%\iffalse
\documentclass[journal,10pt,twocolumn]{article}
\usepackage{graphicx}
\usepackage[margin=0.5in]{geometry}
\usepackage[cmex10]{amsmath}
\usepackage{array}
\usepackage{booktabs}
\usepackage{mathtools}
\title{\textbf{Conic section Assignment}}
\author{Jyothsna Paluchuri}
\date{September 2022}


\providecommand{\norm}[1]{\left\lVert#1\right\rVert}
\providecommand{\abs}[1]{\left\vert#1\right\vert}
\let\vec\mathbf
\newcommand{\myvec}[1]{\ensuremath{\begin{pmatrix}#1\end{pmatrix}}}
\newcommand{\mydet}[1]{\ensuremath{\begin{vmatrix}#1\end{vmatrix}}}
\providecommand{\brak}[1]{\ensuremath{\left(#1\right)}}
\providecommand{\lbrak}[1]{\ensuremath{\left(#1\right.}}
\providecommand{\rbrak}[1]{\ensuremath{\left.#1\right)}}
\providecommand{\sbrak}[1]{\ensuremath{{}\left[#1\right]}}

\begin{document}

\maketitle
\paragraph{\textit{Problem Statement} -
\fi
Find the area of the region bounded by the curve $x^2=4y$ and the lines y=2 and y=4 and the y-axis in the first quadrant.
\\
\solution
	\begin{figure}[!h]
		\centering
 \includegraphics[width=\columnwidth]{chapters/12/8/3/3/figs/conic.png}
		\caption{}
		\label{fig:12/8/3/3}
  	\end{figure}
\iffalse

\section*{\large Solution}

\begin{figure}[h]
\centering
\includegraphics[width=1\columnwidth]{conic.png}

\caption{The parabola formed by the curve $x^2 = 4y$ and the lines y=2 and y=4}
\label{fig:parabola}
\end{figure}

The given equation of parabola $x^2 = 4y$ can be written in the general quadratic form as
\begin{align}
    \label{eq:conic_quad_form}
    \vec{x}^{\top}\vec{V}\vec{x}+2\vec{u}^{\top}\vec{x}+f=0
    \end{align}
where
\fi
The conic parameters are
\begin{align}
	\vec{V} = \myvec{1 & 0\\0 & 0},
	\vec{u} = \myvec{0\\-2},
	f = 0
	%\\
\end{align}
\iffalse
The point of intersection of the lines y=2 and y=4 to the parabola is given by



The points of intersection of the line 
\begin{align}
	L: \quad \vec{x} = \vec{q} + \mu \vec{m} \quad \mu \in \mathbf{R}
\label{eq:conic_tangent}
\end{align}
with the conic section are given by
\begin{align}
\vec{x}_i = \vec{q} + \mu_i \vec{m}
\label{eq:conic_tangent_pts}
\end{align}
%
where
{\tiny
\begin{multline}
\mu_i = \frac{1}
{
\vec{m}^T\vec{V}\vec{m}
}
\lbrak{-\vec{m}^T\brak{\vec{V}\vec{q}+\vec{u}}}
\\
\pm
\rbrak{\sqrt{
\sbrak{
\vec{m}^T\brak{\vec{V}\vec{q}+\vec{u}}
}^2
-
\brak
{
\vec{q}^T\vec{V}\vec{q} + 2\vec{u}^T\vec{q} +f
}
\brak{\vec{m}^T\vec{V}\vec{m}}
}
}
\label{eq:tangent_roots}
\end{multline}
}


\fi
The vector parameters of 
$y-4=0$
are
\begin{align}
	\vec{h}_1=\myvec{0\\4},
	\vec{m}_1=\myvec{1\\0}
\end{align}
Substituting the above in \eqref{eq:tangent_roots},
\begin{align}
\mu_i=4,-4
\end{align}
yielding
the points of intersection with the parabola as
\begin{align}
\vec{a}_0=\myvec{4\\4},
\vec{a}_1=\myvec{-4\\4}
\end{align}
Similarly, for 
the line $y-2=0$, the vector parameters are
\begin{align}
\vec{h}_2=\myvec{0\\2},
\vec{m}_2=\myvec{1\\0}
\end{align}
yielding 
\begin{align}
\mu_i=2.8,-2.8
\end{align}
and the points of intersection
\begin{align}
\vec{a}_2=\myvec{2.8\\2},
\vec{a}_3=\myvec{-2.8\\2}
\end{align}
From Fig.
		\ref{fig:12/8/3/3},
the area of the parabola between the lines $y=2$ and $y=4$ is given by
\begin{align}
\int_{0}^{4} \ 2\sqrt{y} \,dy-\int_{0}^{2} \ 2\sqrt{y} \,dy
=6.895 
\end{align}
\iffalse


\section*{\large Construction}

{
\setlength\extrarowheight{5pt}
\begin{tabular}{|l|c|}
    \hline 
    \textbf{Points} & \textbf{intersection points} \\ \hline
	a0 & $\myvec{
   -2.8\\
   2
   } $ \\\hline
	a1 & $\myvec{
   2.8\\
   2
   } $ \\\hline
    
	a3 & $\myvec{
   -4\\
   4
   } $ \\\hline
	a2 & $\myvec{
   4\\
   4
   } $ \\\hline
      
      \end{tabular}
}

\end{document}
\fi

\item 
\label{chapters/12/8/2/6}
\documentclass[journal,12pt,twocolumn]{IEEEtran}
\usepackage{graphicx}
\usepackage{listings}
\usepackage[utf8]{inputenc}
\usepackage{caption}
\usepackage{hyperref}
\usepackage[cmex10]{amsmath}
\usepackage{array}
\usepackage{gensymb}
\usepackage{booktabs}
\usepackage{etoolbox}
\usepackage{amssymb}
\patchcmd{\section}{\centering}{}{}{}
\providecommand{\norm}[1]{\left\lVert#1\right\rVert}
\providecommand{\abs}[1]{\left\vert#1\right\vert}
\let\vec\mathbf

\makeatletter
\newcommand\xleftrightarrow[2][]{%
  \ext@arrow 9999{\longleftrightarrowfill@}{#1}{#2}}
\newcommand\longleftrightarrowfill@{%
  \arrowfill@\leftarrow\relbar\rightarrow}
\makeatother
\title{Matrix Problems \textbf{\\Conics }}
\author{Manoj Chavva} 
\newcommand{\myvec}[1]{\ensuremath{\begin{pmatrix}#1\end{pmatrix}}}
\newcommand{\mydet}[1]{\ensuremath{\begin{vmatrix}#1\end{vmatrix}}}
\providecommand{\brak}[1]{\ensuremath{\left(#1\right)}}
\providecommand{\lbrak}[1]{\ensuremath{\left(#1\right.}}
\providecommand{\rbrak}[1]{\ensuremath{\left.#1\right)}}
\providecommand{\sbrak}[1]{\ensuremath{{}\left[#1\right]}}

\begin{document}
\maketitle
\section{Problem Statement}

\noindent Smaller area enclosed by the circle $x^2 + y^2 = 4$ and the line $x + y = 2$. 
\begin{enumerate}
\item $2(\pi -2)$
\item $\pi -2$
\item $2\pi -1$
\item $2(\pi +2)$
\end{enumerate}


\begin{figure}[h]
\includegraphics[width=1\columnwidth]{./figs/conic.png}
\caption{Smaller region between Circle and Line}
\label{fig:conic}
\end{figure}

\raggedright \textbf{Given}: \\
Equation of circle is  
\begin{equation} x^2 + y^2 = 4
\end{equation}
Equation of line is 
\begin{equation}
x+y=2
\end{equation}
\textbf{To Find:} \\
To find the intersection points and area of shaded region shown in figure\
\section{Construction}

\begin{table}[h!]
\begin{center}
\setlength{\arrayrulewidth}{0.5mm}
\renewcommand{\arraystretch}{1.5}
    \begin{tabular}{|l|c|}
    \hline 
    \textbf{Points} & \textbf{coordinates} \\ \hline
   $\vec{A}$ & $\myvec{
   0\\
   2
   } $ \\\hline
   $\vec{B}$ & $\myvec{
   2\\
   0
   } $ \\\hline
      \end{tabular}
  \end{center}
\end{table}
\newpage
\section{solution}
The given circle can be expressed as conics with parameters,
\begin{equation}
\vec{V}=\myvec{
4 & 0\\
0 & 4
}
\end{equation}
\begin{equation}
\vec{u}=0 
\end{equation}
\begin{equation}
f=-16
\end{equation}

The given line equation can be written as\\ 
\begin{align} 
	\vec{x}=\begin{pmatrix}2 \\ 0 \\ \end{pmatrix}+k\begin{pmatrix}\frac{1}{2} \\ -\frac{1}{2} \\ \end{pmatrix}
\end{align}
The points of intersection of the line, \\ 
\begin{equation}
L: \quad \vec{x} = \vec{q} + \kappa \vec{m} \quad \kappa \in \mathbb{R}
\end{equation}

with the conic section, \\ 
\begin{align}
	\vec{x}^{\top}\vec{V}\vec{x} + 2\vec{u}^{\top} \vec{x} + f = 0
\end{align}
are given by \\
\begin{align}
\vec{x}_i = \vec{q} + \kappa_i \vec{m}
\end{align}
where, \\

\begin{equation*}
\kappa_i = \frac{1}
{
\vec{m}^T\vec{V}\vec{m}
}
\lbrak{-\vec{m}^T\brak{\vec{V}\vec{q}+\vec{u}}}
\pm
\end{equation*}
\begin{equation}
\rbrak{\sqrt{
\sbrak{
\vec{m}^T\brak{\vec{V}\vec{q}+\vec{u}}
}^2
-
\brak
{
\vec{q}^T\vec{V}\vec{q} + 2\vec{u}^T\vec{q} +f
}
\brak{\vec{m}^T\vec{V}\vec{m}}
}
}
\end{equation}
On substituting\\
\begin{align}
\vec{q} &= \myvec{
2\\
0
} 
\end{align}
\begin{align}
\vec{m} = \myvec{\frac{1}{2} \\ -\frac{1}{2}}
\end{align}
With the given as in eq(3),(4),(5),\\ 

The value of $\kappa$ ,\\
\begin{equation}
\kappa =0,-4
\end{equation}
    
By substituting eq(13) in eq(6) we get the
points of intersection of line with circle \\
\begin{align}
    \vec{A}=\myvec{
0\\
2
    }
\end{align}
\begin{align}
    \vec{B}=\myvec{
2\\
0
    }
\end{align}
From the figure \\
Total area of portion is given by,\\ 
Total Area=(area of circle in first quadrant)-(area of a triangle \textbf{AOB})

\subsection*{Area of triangle}

\begin{align}
\implies A_1=\int_{0}^{2} (2-x) \,dx
\end{align}
By solving the above equation we get area of triangle as 2 units
\subsection*{Area of circle}

\begin{align} 
\implies A_2=\int_{0}^{2}\sqrt{4-x^2} \,dx 
\end{align}
By solving the above equation we get area of circle $\pi$

The total area is
$\implies \vec{A}=\pi - 2$


\begin{table}[h]
\large
\begin{tabular}{lll}
\multicolumn{3}{l}{Get Python Code for image from}                                                 \\ \hline
\multicolumn{3}{|l|}{\url{https://github.com/ManojChavva/FWC/blob/main/Matrix/conics/code/conic.py}} \\ 
 \hline
\multicolumn{3}{l}{Get LaTex code from}                                                            \\ \hline
\multicolumn{3}{|l|}{\url{https://github.com/ManojChavva/FWC/blob/main/Matrix/conics/conic.tex}}            \\ \hline
\end{tabular}
\end{table}



\end{document}





\item 
\label{chapters/12/8/2/7}
%\iffalse
\documentclass[journal,10pt,twocolumn]{article}
\usepackage{graphicx}
\usepackage[margin=0.5in]{geometry}
\usepackage[cmex10]{amsmath}
\usepackage{array}
\usepackage{booktabs}
\usepackage{mathtools}
\title{\textbf{Conic section Assignment}}
\author{Jyothsna Paluchuri}
\date{September 2022}


\providecommand{\norm}[1]{\left\lVert#1\right\rVert}
\providecommand{\abs}[1]{\left\vert#1\right\vert}
\let\vec\mathbf
\newcommand{\myvec}[1]{\ensuremath{\begin{pmatrix}#1\end{pmatrix}}}
\newcommand{\mydet}[1]{\ensuremath{\begin{vmatrix}#1\end{vmatrix}}}
\providecommand{\brak}[1]{\ensuremath{\left(#1\right)}}
\providecommand{\lbrak}[1]{\ensuremath{\left(#1\right.}}
\providecommand{\rbrak}[1]{\ensuremath{\left.#1\right)}}
\providecommand{\sbrak}[1]{\ensuremath{{}\left[#1\right]}}

\begin{document}

\maketitle
\paragraph{\textit{Problem Statement} -
\fi
Find the area of the region bounded by the curve $x^2=4y$ and the lines y=2 and y=4 and the y-axis in the first quadrant.
\\
\solution
	\begin{figure}[!h]
		\centering
 \includegraphics[width=\columnwidth]{chapters/12/8/3/3/figs/conic.png}
		\caption{}
		\label{fig:12/8/3/3}
  	\end{figure}
\iffalse

\section*{\large Solution}

\begin{figure}[h]
\centering
\includegraphics[width=1\columnwidth]{conic.png}

\caption{The parabola formed by the curve $x^2 = 4y$ and the lines y=2 and y=4}
\label{fig:parabola}
\end{figure}

The given equation of parabola $x^2 = 4y$ can be written in the general quadratic form as
\begin{align}
    \label{eq:conic_quad_form}
    \vec{x}^{\top}\vec{V}\vec{x}+2\vec{u}^{\top}\vec{x}+f=0
    \end{align}
where
\fi
The conic parameters are
\begin{align}
	\vec{V} = \myvec{1 & 0\\0 & 0},
	\vec{u} = \myvec{0\\-2},
	f = 0
	%\\
\end{align}
\iffalse
The point of intersection of the lines y=2 and y=4 to the parabola is given by



The points of intersection of the line 
\begin{align}
	L: \quad \vec{x} = \vec{q} + \mu \vec{m} \quad \mu \in \mathbf{R}
\label{eq:conic_tangent}
\end{align}
with the conic section are given by
\begin{align}
\vec{x}_i = \vec{q} + \mu_i \vec{m}
\label{eq:conic_tangent_pts}
\end{align}
%
where
{\tiny
\begin{multline}
\mu_i = \frac{1}
{
\vec{m}^T\vec{V}\vec{m}
}
\lbrak{-\vec{m}^T\brak{\vec{V}\vec{q}+\vec{u}}}
\\
\pm
\rbrak{\sqrt{
\sbrak{
\vec{m}^T\brak{\vec{V}\vec{q}+\vec{u}}
}^2
-
\brak
{
\vec{q}^T\vec{V}\vec{q} + 2\vec{u}^T\vec{q} +f
}
\brak{\vec{m}^T\vec{V}\vec{m}}
}
}
\label{eq:tangent_roots}
\end{multline}
}


\fi
The vector parameters of 
$y-4=0$
are
\begin{align}
	\vec{h}_1=\myvec{0\\4},
	\vec{m}_1=\myvec{1\\0}
\end{align}
Substituting the above in \eqref{eq:tangent_roots},
\begin{align}
\mu_i=4,-4
\end{align}
yielding
the points of intersection with the parabola as
\begin{align}
\vec{a}_0=\myvec{4\\4},
\vec{a}_1=\myvec{-4\\4}
\end{align}
Similarly, for 
the line $y-2=0$, the vector parameters are
\begin{align}
\vec{h}_2=\myvec{0\\2},
\vec{m}_2=\myvec{1\\0}
\end{align}
yielding 
\begin{align}
\mu_i=2.8,-2.8
\end{align}
and the points of intersection
\begin{align}
\vec{a}_2=\myvec{2.8\\2},
\vec{a}_3=\myvec{-2.8\\2}
\end{align}
From Fig.
		\ref{fig:12/8/3/3},
the area of the parabola between the lines $y=2$ and $y=4$ is given by
\begin{align}
\int_{0}^{4} \ 2\sqrt{y} \,dy-\int_{0}^{2} \ 2\sqrt{y} \,dy
=6.895 
\end{align}
\iffalse


\section*{\large Construction}

{
\setlength\extrarowheight{5pt}
\begin{tabular}{|l|c|}
    \hline 
    \textbf{Points} & \textbf{intersection points} \\ \hline
	a0 & $\myvec{
   -2.8\\
   2
   } $ \\\hline
	a1 & $\myvec{
   2.8\\
   2
   } $ \\\hline
    
	a3 & $\myvec{
   -4\\
   4
   } $ \\\hline
	a2 & $\myvec{
   4\\
   4
   } $ \\\hline
      
      \end{tabular}
}

\end{document}
\fi

\end{enumerate}

%\subsection{Parallelograms}
%\subsection{Triangles and Parallelograms}
%\begin{enumerate}
\item 
\item 
\item 
\label{chapters/9/9/4/3}

\documentclass[10pt,a4paper]{report}
\usepackage[utf8]{inputenc}
\usepackage{amsmath}
\usepackage{amsfonts}
\usepackage{amssymb}
\usepackage{graphicx}
\usepackage{multicol}
\usepackage{tabularx}
\usepackage{tikz}
\newcommand{\myvec}[1]{\ensuremath{\begin{pmatrix}#1\end{pmatrix}}}
\let\vec\mathbf
\let\myvec\bf
\providecommand{\mtx}[1]{\mathbf{#1}}
\newcommand{\mydet}[1]{\ensuremath{\begin{vmatrix}#1\end{vmatrix}}}
\usetikzlibrary{arrows,shapes,automata,petri,positioning,calc}
\usepackage{hyperref}
\usepackage{tikz}
\usetikzlibrary{matrix,calc}
\usepackage[margin=0.5in]{geometry}
\providecommand{\norm}[1]{\left\lVert#1\right\rVert}
\let\vec\mathbf
\newenvironment{Figure}
  {\par\medskip\noindent\minipage{\linewidth}}
  {\endminipage\par\medskip}
    
 
 
\begin{document}
%--------------------logo figure-------------------------%
\begin{figure*}[!tbp]
  \centering
  \begin{minipage}[b]{0.4\textwidth}
    \end{minipage}
  \hfill
  \vspace{5mm}\begin{minipage}[b]{0.4\textwidth}

  \end{minipage}\vspace{0.2cm}
\end{figure*}
%--------------------name & rollno-----------------------
\raggedright \textbf{Name}:\hspace{1mm} Varsha Reddy\hspace{3cm} \Large \textbf{Assignment-4}\hspace{2.5cm} % 
\normalsize \textbf{Roll No.} :\hspace{1mm} FWC22038\vspace{1cm}
\begin{multicols}{2}

\raggedright \textbf{PROBLEM:}\vspace{2mm}\\
\textbf{ABCD,DCFE and ABFE are parallelograms.Show that   ar(ADE) = ar(BCF)}
\vspace{0.5cm}\raggedright \\
Theory:
Parallelograms on the same base and in between the same parallels are equal in area.\\
Given: ABCD,DCFE and ABFE are parallelograms.
\vspace{2mm} \\ 
%----------------Solution  statement--------------%
\raggedright \textbf{Solution Statement:}\vspace{2mm}
\raggedright \\We can see that the sides of a triangle ADE and BCF are also the opposite sides of a given parallelogram. Now we can show both the triangles are congruent using congruency property. We know that congruent triangles are equal areas.  \\
\vspace{5mm}
\section{Construction}
  \begin{center}
   %\includegraphics[scale=1]{assignment4/ABC.pdf}
     \includegraphics[scale=0.5]{ABC.pdf} 
     Figure of Construction
   \end{center}
   \vspace{5mm}

\section{Table :}

The input parameters for this construction are 
\begin{center}
\begin{tabular}{|c|c|c|}
	\hline
	\textbf{Symbol}&\textbf{Value}&\textbf{Description}\\
	\hline
	a&3&EA\\
	\hline
	b&4.5&EF\\
	\hline
	c&2&ED\\
	\hline
	${\theta}_1$& 1$\pi/3$&$ \angle $AEF\\ 
	\hline
	${\theta}_2$& 2$\pi/3$&$ \angle $DEF\\ 
	    \hline
	E&$\
	\begin{pmatrix}
		0 \\
		0 \\
	\end{pmatrix}$%
	&Point E\\
	\hline
\end{tabular}
\end{center}

1. Considering point 'E' as origin.\\
2. From E,with some angle of 60 degrees,mark the point 'A'.\\
3. From E,with some angle of 120 degrees,mark the point 'D'.\\
4. With the distance of 'b' locate the point 'F'.\\
5. To locate a point 'B'\\
\begin{center}
    \myvec{ B = A+F-E }
\end{center}
6. To locate a point 'C'\\
\begin{center}
    \myvec{ C = D+F-E }
\end{center}
7. Joining all the lines from the figure.
\vspace{0.5cm}\\
%-----------------------------solution---------------------------
 \section{Solution}
% \vspace{2mm}\\
n ABCD,
\begin{equation}
\vec{A-B}=\vec{D-C}
\end{equation}
\begin{equation}
\vec{A-D}=\vec{B-C}
\end{equation}
\\
In DEFC,
\begin{equation}
\vec{D-C}=\vec{E-F}
\end{equation}
\begin{equation}
\vec{D-E}=\vec{C-F}
\end{equation}
\\
In ABEF,
\begin{equation}
\vec{A-B}=\vec{E-F}
\end{equation}
\begin{equation}
\vec{A-E}=\vec{B-F}
\end{equation}

\vspace{1mm}

\textbf{To Prove:} 
  \begin{center}
  \begin{equation}
      \vec{Ar(ADE)=Ar(BCF)}
  \end{equation}
 \vspace{3mm}
 Area of the triangle $\Delta$ADE is given by \\
 Ar($\Delta$ADE) 
\begin{equation} 
 = \frac{1}{2}\norm{\vec{A-D}\times\vec{D-E}}
\end{equation}
  %\vspace{3mm}
 \vspace{3mm}
  Area of the triangle $\Delta$BCF is given by \\
  Ar($\Delta$BCF)
   \begin{equation}
   =\frac{1}{2}\norm{\vec{B-C}\times\vec{C-F}}
   \end{equation}
 \end{center}
 substituiting (2) and (4) in (9),
 \begin{equation}
   =\frac{1}{2}\norm{\vec{A-D}\times\vec{D-E}}
   \end{equation}
   from (8) and (10),
 \begin{center}
\begin{equation}
    \vec{Ar(ADE)=Ar(BCF)}
\end{equation}
\end{center}
The below python code realizes the above construction: \\
\url{https://github.com/9705701645/FWC/blob/main/lines4.py}
\bibliographystyle{ieeetr}
\end{multicols}
\end{document}


\item 
\label{chapters/9/9/4/4}
\iffalse
\documentclass[journal,10pt,twocolumn]{article}
\usepackage{graphicx}
\usepackage[margin=0.5in]{geometry}
\usepackage[cmex10]{amsmath}
\usepackage{array}
\usepackage{booktabs}
\usepackage{listings}
\title{\textbf{Line Assignment}}
\author{Bhavani Kanike}
\date{October 2022}

\providecommand{\norm}[1]{\left\lVert#1\right\rVert}
\providecommand{\abs}[1]{\left\vert#1\right\vert}
\let\vec\mathbf
\newcommand{\myvec}[1]{\ensuremath{\begin{pmatrix}#1\end{pmatrix}}}
\newcommand{\mydet}[1]{\ensuremath{\begin{vmatrix}#1\end{vmatrix}}}
\providecommand{\brak}[1]{\ensuremath{\left(#1\right)}}

\begin{document}

\maketitle
\paragraph{\textit{Problem Statement} 
\fi
ABCD is a quadrilateral in which $\vec{P}, \vec{Q}, \vec{R}$ and $\vec{S}$ are mid-points of the sides AB, BC, CD and DA (see Fig \ref{fig:9/8/2/1}). AC is a diagonal. 
		
Show that 
\begin{enumerate}
	\item $SR \parallel AC$ and $SR =\frac{1}{2} AC$
\item $PQ = SR$
\item $PQRS$ is a parallelogram.
\end{enumerate}
 	\begin{figure}
		\centering
 \includegraphics[width=\columnwidth]{chapters/9/8/2/1/figs/line1.pdf}
		\caption{}
		\label{fig:9/8/2/1}
  	\end{figure}
	\solution 
	Using 
	  \eqref{eq:section_formula},
	\begin{align}
		\label{eq:9/8/2/1}
		\begin{split}
		\vec{P} &= \frac{\vec{A}+\vec{B}}{2}\\
 \vec{Q} &= \frac{\vec{C}+\vec{B}}{2}\\
 \vec{R} &= \frac{\vec{C}+\vec{D}}{2}\\
 \vec{S} &= \frac{\vec{D}+\vec{A}}{2}
		\end{split}
	\end{align}
\begin{enumerate}
	\item
	Consequently, 
	\begin{align}
\vec{R}
		-\vec{S} &= \frac{\vec{C}-\vec{A}}{2}
		\\
		\implies SR &\parallel AC
	\end{align}
	Also, 
	\begin{align}
		\norm{\vec{R}
		-\vec{S}} &= \frac{\norm{\vec{C}-\vec{A}}}{2}
		\\
		\implies SR &= \frac{1}{2}AC
	\end{align}
\item 	From 
		\eqref{eq:9/8/2/1},
	\begin{align}
\vec{R}
		-\vec{S} = \vec{Q}-\vec{P}
	\end{align}
	which means that $PQRS$ is a parallelogram and $PQ = SR$.
\end{enumerate}
%
\iffalse
\begin{figure}[h]
\centering
\includegraphics[width=1\columnwidth]
\caption{Figure}
\label{fig:triangle}
\end{figure}

\section*{Solution}

$\boldsymbol Given :$  ABCD is a Quadrilateral P,Q,R and S are the midpoints of line AB,BC,CD,DA.We can obtain the points P,Q,R and S from A,B,C and D and are given by\\\\
\boldmath
\unboldmath
(3) To prove that PQRS is a parallelogram we need to prove  PQ // SR
To prove SR $\parallel$ PQ\\
Direction vector of line SR  $\boldsymbol {(R-S) =  \frac{(C-A)}{2}}$\\\\
Direction vector of line PQ  $\boldsymbol {(Q-P)= \frac{(C-A)}{2}}$\\\\
\begin{equation}
	\boldsymbol {(R-S) = (Q-P) = \frac{(C-A)}{2}}\\
\end{equation}
Since the direction vectors of line SR and PQ are in same direction\\\\
$SR \parallel PQ$\\
Therefore,
$\boldsymbol{ PQRS }$ is a parallelogram\\\\

	
(1)  Directional vector of line SR  = $\boldsymbol {(R-S)}$ = $\frac{\boldsymbol{(C-A)}}{2} $\\
Directional vector of line AC  = $\boldsymbol {(C-A)}$\\

It is observed that the constant k is $\frac{1}{2}$

Therefore
\begin{equation}
	SR \parallel AC
\end{equation} 

and from equation 1 
\begin{equation}
	\boldsymbol {SR = \frac{1}{2}AC}    
\end{equation}\\


(2)   To prove PQ = SR\\ 
		From euqation 1\\\\
\begin{equation}
		\boldsymbol{ (Q-P) = (R-S) = \frac{(C-A)}{2}}
\end{equation}
	 



\section{Execution}
The below python code realizes the construction:
\begin{lstlisting}
https://github.com/bhavani360/FWC_assignments
\end{lstlisting}
	
\section*{Construction}
The dimensions of the Quadrilateral ABCD are taken as below\\
{
\setlength\extrarowheight{2pt}
\centering
	\begin{tabular}{|c|c|}
	\hline
	\textbf{symbol}&\textbf{value}\\
	\hline
	r&8\\
	\hline
	$\theta$&pi/2.5\\
	\hline
	d&7\\
	\hline
	A&(0,0)\\
	\hline
	B&(d,0)\\
	\hline
	D&(rcos$\theta$,rsin$\theta$)\\
	\hline
	C&(D/1.5)+B\\
	\hline
\end{tabular}
}
\end{document}
\fi

\item 
\label{chapters/9/9/4/5}
\iffalse
\def\mytitle{MATRICES USING PYTHON}
\def\myauthor{VAMSI SUNKARI}
\def\contact{vamsisunkari9849@gmail.com}
\def\mymodule{Future Wireless Communication (FWC)}
\documentclass[10pt, a4paper]{article}
\usepackage[a4paper,outer=1.5cm,inner=1.5cm,top=1.75cm,bottom=1.5cm]{geometry}
\twocolumn
\usepackage{graphicx}
\graphicspath{{./images/}}
\usepackage[colorlinks,linkcolor={black},citecolor={blue!80!black},urlcolor={blue!80!black}]{hyperref}
\usepackage[parfill]{parskip}
\usepackage{lmodern}
\usepackage{tikz}
 \usepackage{physics}
\usepackage{karnaugh-map}
\usepackage{setspace}
\doublespacing
%\documentclass{article}
\usepackage{tabularx}
%\usepackage{circuitikz}
\usetikzlibrary{calc}
\usepackage{amsmath}
\usepackage{amssymb}
\renewcommand*\familydefault{\sfdefault}
%\usepackage{watermark}
\usepackage{lipsum}
\usepackage{xcolor}
\usepackage{listings}
\usepackage{float}
\usepackage{titlesec}
\providecommand{\mtx}[1]{\mathbf{#1}}
\titlespacing{\subsection}{1pt}{\parskip}{3pt}
\titlespacing{\subsubsection}{0pt}{\parskip}{-\parskip}
\titlespacing{\paragraph}{0pt}{\parskip}{\parskip}
\newcommand{\figuremacro}[5]{
    \begin{figure}[#1]
        \centering
        \includegraphics[width=#5\columnwidth]{#2}
        \caption[#3]{\textbf{#3}#4}
        \label{fig:#2}
    \end{figure}
}
\newcommand{\myvec}[1]{\ensuremath{\begin{pmatrix}#1\end{pmatrix}}}
\let\vec\mathbf
\lstset{
frame=single, 
breaklines=true,
columns=fullflexible
}
\title{\mytitle}
\author{\myauthor\hspace{1em}\\\contact\\FWC22040\hspace{6.5em}IITH\hspace{0.5em}\mymodule\hspace{6em}ASSIGN-5}
\date{}
\begin{document}
 \maketitle
 \tableofcontents
   \section{Problem}
   \fi
   In Fig. 
		\ref{fig:9/9/4/5},
	\begin{figure}[!h]
		\centering
 \includegraphics[width=\columnwidth]{chapters/9/9/4/5/figs/matrix.png}
		\caption{}
		\label{fig:9/9/4/5}
  	\end{figure}
$  ABC$ and $BDE$ are two equilateral
triangles such that $\vec{D}$ is the mid-point of $BC$. If $AE$
intersects $BC$ at $\vec{F}$, show that
\begin{align}
	ar(BDE) &=\frac{1}{4} ar(ABC)\\
	ar(BDE) &=\frac{1}{2} ar(BAE)\\
ar(ABC) &=2 ar(BEC)\\
ar(BFE) &= ar(AFD)\\
ar(BFE) &=2ar(FED)\\
	ar(FED) &=\frac{1}{8} ar(AFC)
\end{align}
\iffalse


\begin{proof}
From Appendix
    \ref{prop:two-isosc}, considering $\triangle$s $ABC$ and $EBD$,
\begin{align}
		\label{fig:9/9/4/5/b}
	\brak{\vec{B}-\vec{C}}^{\top}
	\brak{\vec{A}-\vec{D}} &=\vec{0}
	%\brak{\vec{A}-\frac{\vec{B}+\vec{C}}{2}}&=\vec{0}
	\\
	\brak{\vec{B}-\vec{D}}^{\top}
	\brak{\vec{E}-\frac{\vec{B}+\vec{D}}{2}}&=\vec{0}
		\label{fig:9/9/4/5/e}
\end{align}
		\eqref{fig:9/9/4/5/e}
		can be expressed as
\begin{align}
	\brak{\vec{B}-\vec{D}}^{\top}
	%\brak{\vec{B}-\brak{\frac{\vec{B}+\vec{C}}{2}}}^{\top}
	\brak{\vec{E}-\frac{\vec{B}+\vec{D}}{2}}&=\vec{0}
	%\brak{\vec{E}-\frac{\vec{B}+\brak{\frac{\vec{B}+\vec{C}}{2}}}{2}}&=\vec{0}
	\\
	\implies \brak{\vec{B}-\vec{C}}^{\top}
	\brak{\vec{E}-\frac{\vec{B}+\vec{D}}{2}}&=\vec{0}
%	\brak{\vec{E}-\frac{3\vec{B}+2\vec{C}}{2}}&=\vec{0}
		\label{fig:9/9/4/5/e/2}
\end{align}
upon substituting
\begin{align}
	\vec{D}=\frac{\vec{B}+\vec{C}}{2}
\end{align}
From 
		\eqref{fig:9/9/4/5/b},
		\eqref{fig:9/9/4/5/e/2}
		and 
	  Appendix \ref{prop:two-orth-para},
\begin{align}
	\vec{E}-\frac{\vec{B}+\vec{D}}{2}&=
	k\brak{\vec{A}-\vec{D}}
	%\vec{E}-\frac{3\vec{B}+2\vec{C}}{2}&=
	%k\brak{\vec{A}-\brak{\frac{\vec{B}+\vec{C}}{2}}}
	\\
\implies
	\vec{E} &= \frac{\vec{B}+\vec{D}}{2}+k\brak{\vec{A}-\vec{D}}
	%\frac{3\vec{B}+2\vec{C}}{2}+	k\brak{\vec{A}-\brak{\frac{\vec{B}+\vec{C}}{2}}}
	%\vec{E} &= \frac{3\vec{B}+2\vec{C}}{2}+	k\brak{\vec{A}-\brak{\frac{\vec{B}+\vec{C}}{2}}}
		\label{fig:9/9/4/5/ef}
\end{align}
Since
\begin{align}
	\vec{E} \times \vec{B} &=\frac{\vec{D}\times \vec{B}}{2}+k\brak{\vec{A}-\vec{D}} \times \vec{B}
	\\
	&=\brak{\frac{1}{2}-k}\vec{D}\times \vec{B}+k\vec{A} \times \vec{B}
	%\vec{E} \times \vec{B} &=\brak{\frac{3\vec{B}+2\vec{C}}{2}+k\brak{\vec{A}-\brak{\frac{\vec{B}+\vec{C}}{2}}}} \times \vec{B}
%	\\
%	&=\brak{\vec{C}+k\brak{\vec{A}-\brak{\frac{\vec{C}}{2}}}} \times \vec{B}
%	\\
%	\vec{B} \times \vec{D} &=\vec{B} \times \frac{\vec{B}+\vec{C}}{2}
%	\\
%	&=\vec{B} \times \frac{\vec{C}}{2}
	\\
	\vec{D} \times \vec{E}&=\frac{\vec{D}\times\vec{B}}{2}+k \vec{D}\times\vec{A}
	%\vec{D} \times \vec{E}&=\brak{\frac{\vec{B}+\vec{C}}{2}}\times\brak{\frac{3\vec{B}+2\vec{C}}{2}+	k\brak{\vec{A}-\brak{\frac{\vec{B}+\vec{C}}{2}}}}
\end{align}
	upon substituting from 
		\eqref{fig:9/9/4/5/ef},
\begin{align}
		\label{fig:9/9/4/5/e/area}
{\vec{E} \times \vec{B}+\vec{B} \times \vec{D}+\vec{D} \times \vec{E}}
 \\
= 
k\brak{\vec{A} \times \vec{B}+\vec{B} \times \vec{D}+\vec{D} \times \vec{A}}
%	\\
%	+\vec{B} \times \vec{D}+\vec{D} \times \brak{\frac{3\vec{B}+2\vec{C}}{2}+k\brak{\vec{A}-\brak{\frac{\vec{B}+\vec{C}}{2}}}}
\end{align}
%upon  substituting from 
%		\eqref{fig:9/9/4/5/ef}
		%in
		%\eqref{fig:9/9/4/5/e/area},
\end{proof}
%\iffalse
From 
	  \eqref{eq:two-tri-indep},
\begin{align}
	  \brak{p+q}\vec{E}- p\vec{B} -q\vec{D} &= 0
	  \\
	  \implies
	  \brak{p+q}\brak{\frac{\vec{B}+\vec{D}}{2}+k\brak{\vec{A}-\vec{D}}}- p\vec{B} -q\vec{D} &= 0
	  \\
	  \text{or, }
	  \brak{\frac{q-p}{2}}\vec{B}+\brak{\frac{p-q}{2}-k\brak{p+q}}\vec{D}+k\vec{A}&= 0
\end{align}
Since $\vec{D}$ is the mid point of $BC$, substituting 
in the above,
\begin{align}
	\brak{\frac{q-p}{2}}\vec{B}+\brak{\frac{p-q}{2}-k\brak{p+q}}\brak{\frac{\vec{B}+\vec{C}}{2}}+k\vec{A}&= 0
%	\brak{\frac{q-p}{2}}\vec{B}-\brak{q+k\brak{{\frac{p+q}{2}}}}\frac{\vec{B}+\vec{C}}{2}+k\brak{p+q}\vec{A}&= 0
%	  \\
%	  \implies
%k\brak{p+q}\vec{A}+
%	\brak{\frac{q-p}{2}-\brak{q+k\brak{{\frac{p+q}{2}}}}}\vec{B}-\brak{q+k\brak{{\frac{p+q}{2}}}}\frac{\vec{B}+\vec{C}}{2}+k\brak{p+q}\vec{A}&= 0
	\\
	\implies 
	k\vec{A}+\brak{\frac{q-p}{2}}\vec{B}+\brak{\frac{p-q}{2}-k\brak{p+q}}\brak{\frac{\vec{B}+\vec{C}}{2}}&= 0
\end{align}
in the above, 

[Hint : Join EC and AD. Show that BE  AC abd DE AB, etc.]

   %  \includegraphics[scale=1.0]{diag_1.png}
   \section{Solution}
   \textbf{Theory:}\\
   \textbf{To Prove:} ar(BDE)=1/4 ar(ABC) \\
   ABC and BDE are two equilateral triangles such that D is the mid-point of BC.If AE intersets BC at F


\textbf{termux commands :}
\begin{lstlisting}
python3 matrix.py
\end{lstlisting}


The input parameters for this construction are 
\begin{center}
\begin{tabular}{|c|c|c|}
 \hline
 \textbf{Symbol}&\textbf{Value}&\textbf{Description}\\
 \hline
 r&5&AB\\
 \hline

 $\vec{A}$&$r\myvec{\cos\theta \\ \sin\theta}$%
 &Point A\\
 \hline
 $\vec{E}$&$\myvec{\frac{r}{2}\cos\theta \\ \frac{r}{2}\sin\theta}$
 &Point E\\
 \hline
 $\vec{B}$&$\myvec{0 \\ 0}$%
 &Point B\\
 \hline
 $\vec{C}$&$\myvec{r \\ 0}$%
 &Point C\\
 \hline
 $\vec{D}$&$\myvec{r/2 \\ 0}$%
 &Point D\\
 \hline
 $\vec{F}$&$\myvec{\frac{2}{3}r\cos\theta \\ 0}$%
 &Point F\\
 \hline
 ${\theta}_1$& $\pi/3$&$ \angle $ABC\\ 
 \hline
\end{tabular}
\end{center}
\textbf{To Prove:} ar(BDE)=1/4 ar(ABC)
  \begin{center}
 
 $\vec{v1}=\vec{A-B}$\\
 $\vec{v2}=\vec{A-C}$\\

$ar(\Delta ABC) =\frac{1}{2}\norm{\vec{v1}\times\vec{v2}}$............(1)\\
$\vec{v3}=\vec{B-D}$\\
 $\vec{v4}=\vec{B-E}$\\
 
 $ar(\Delta BDE) =\frac{1}{2}\norm{\vec{v3}\times\vec{v4}}$...............(2)\\
 $ar(BDE)= \frac{1}{4} ar(ABC)$
 \end{center}
 \textbf{To Prove:}  ar(BDE)=1/2 ar(BAE) 
 \begin{center}
$\vec{v5}=\vec{A-E}$\\
$\vec{v6}=\vec{A-B}$\\

 $ar(\Delta BAE) =\frac{1}{2}\norm{\vec{v5}\times\vec{v6}}$...............(3)\\
 $ar(BDE)=\frac{1}{2} ar(BAE)$
\end{center}
 \textbf{To Prove:}  ar(ABC)=2 ar(BEC) 
 \begin{center}
 $\vec{v7}=\vec{B-C}$\\
$\vec{v8}=\vec{B-E}$\\

 $ar(\Delta BEC) =\frac{1}{2}\norm{\vec{v7}\times\vec{v8}}$...............(4)\\
 $ar(ABC)=2 ar(BEC)$
 \end{center}
 \textbf{To Prove:}  ar(BFE)=ar(AFD) 
 \begin{center}
 $\vec{v9}=\vec{B-F}$\\
$\vec{v10}=\vec{B-E}$\\
$\vec{v11}=\vec{A-D}$\\
$\vec{v12}=\vec{A-F}$\\
 $ar(\Delta BFE) =\frac{1}{2}\norm{\vec{v9}\times\vec{v10}}$...............(4)\\
 $ar(\Delta AFD) =\frac{1}{2}\norm{\vec{v11}\times\vec{v12}}$...............(4)\\
 ar(BFE)=ar(AFD)
 \end{center}
 \textbf{To prove} : ar(BFE) =2ar(FED)
\begin{center}
$\vec{v13}=\vec{F-E}$\\
$\vec{v14}=\vec{F-D}$\\
 $ar(\Delta FED) =\frac{1}{2}\norm{\vec{v13}\times\vec{v14}}$...............(4)\\
 $ar(BFE) =2ar(FED)$   
\end{center}
\textbf{To prove} : $ar(FED) =\frac{1}{8} ar(AFC)$
\begin{center}
$\vec{v15}=\vec{A-F}$\\
$\vec{v16}=\vec{A-C}$\\
$ar(\Delta AFC) =\frac{1}{2}\norm{\vec{v15}\times\vec{v16}}$...............(4)\\ 
 $ar(FED) =\frac{1}{8} ar(AFC)$   
\end{center}
The below python code realizes the above construction: 
\begin{lstlisting}
https://github.com/Vamsi9849/iithfwc/blob/main/Matrix_line/codes/matrix.py
\end{lstlisting}
 \section{Construction}
  \begin{center}
  \includegraphics[scale=0.39]{par.pdf}  
  Figure of construction
   \end{center}   
\bibliographystyle{ieeetr}
\end{document}
\fi

\item 
\item 
\item 
\end{enumerate}

%--------------------------------------------------------

%
\section{Conics}
\subsection{Circle}
\begin{enumerate}[label=\thesection.\arabic*,ref=\thesection.\theenumi]
  \item Find the equation of the circle passing through the points $(4,1)$ and $(6,5)$ and whose centre is on the line $ 4x+y=16. $
\label{chapters/11/11/1/10}
\\
\solution
\iffalse
\documentclass[12pt]{article}
\usepackage{graphicx}
\usepackage{amsmath}
\usepackage{mathtools}
\usepackage{gensymb}

\newcommand{\mydet}[1]{\ensuremath{\begin{vmatrix}#1\end{vmatrix}}}
\providecommand{\brak}[1]{\ensuremath{\left(#1\right)}}
\providecommand{\norm}[1]{\left\lVert#1\right\rVert}
\newcommand{\solution}{\noindent \textbf{Solution: }}
\newcommand{\myvec}[1]{\ensuremath{\begin{pmatrix}#1\end{pmatrix}}}
\let\vec\mathbf

\begin{document}
\begin{center}
\textbf\large{CHAPTER-11 \\ CIRCLES}

\end{center}
\section*{Excercise 11.1}

Q4.Find the equation of the circle with centre $(1,1)$ and radius $\sqrt{2}$.

\solution
\fi
Given
\begin{align}
	\vec{c} &= \myvec{1\\1} \text{ and } r = \sqrt{2},
	\\
	\vec{u}&=\vec{-c}
	 = \myvec{-1\\-1}\\
	 \\
	f &= \norm{\vec{u}}^2 - r^2
	  =0	
\end{align}
Thus, the equation of circle is 
\begin{align}
	\norm{\vec{x}}^2 -2\myvec{1&1}\vec{x} = 0       		       
\end{align}	
See Fig. 
\ref{fig:chapters/11/11/1/4/Fig1}.
\begin{figure}[!h]
	\begin{center} 
	  \includegraphics[width=\columnwidth]{chapters/11/11/1/4/figs/circ.png}
	\end{center}
\caption{}
\label{fig:chapters/11/11/1/4/Fig1}
\end{figure}


  \item Find the equation of the circle passing through the points $(2,3)$ and $(-1,1)$ and whose centre is on the line $x-3y-11=0$.
\label{chapters/11/11/1/11}
\\
\iffalse
\documentclass[12pt]{article}
\usepackage{graphicx}
\usepackage{amsmath}
\usepackage{mathtools}
\usepackage{gensymb}

\newcommand{\mydet}[1]{\ensuremath{\begin{vmatrix}#1\end{vmatrix}}}
\providecommand{\brak}[1]{\ensuremath{\left(#1\right)}}
\providecommand{\norm}[1]{\left\lVert#1\right\rVert}
\newcommand{\solution}{\noindent \textbf{Solution: }}
\newcommand{\myvec}[1]{\ensuremath{\begin{pmatrix}#1\end{pmatrix}}}
\let\vec\mathbf

\begin{document}
\begin{center}
\textbf\large{CHAPTER-11 \\ CIRCLES}

\end{center}
\section*{Excercise 11.1}

Q4.Find the equation of the circle with centre $(1,1)$ and radius $\sqrt{2}$.

\solution
\fi
Given
\begin{align}
	\vec{c} &= \myvec{1\\1} \text{ and } r = \sqrt{2},
	\\
	\vec{u}&=\vec{-c}
	 = \myvec{-1\\-1}\\
	 \\
	f &= \norm{\vec{u}}^2 - r^2
	  =0	
\end{align}
Thus, the equation of circle is 
\begin{align}
	\norm{\vec{x}}^2 -2\myvec{1&1}\vec{x} = 0       		       
\end{align}	
See Fig. 
\ref{fig:chapters/11/11/1/4/Fig1}.
\begin{figure}[!h]
	\begin{center} 
	  \includegraphics[width=\columnwidth]{chapters/11/11/1/4/figs/circ.png}
	\end{center}
\caption{}
\label{fig:chapters/11/11/1/4/Fig1}
\end{figure}

  \item Find the equation of the circle with radius 5 whose centre lies on $x$-axis and passes through the point $(2,3)$.
\label{chapters/11/11/1/12}
\\
\iffalse
\documentclass[a4paper,12pt,twocolumn]{article}
\usepackage{graphicx}
\usepackage[margin=0.5in]{geometry}
\usepackage[cmex10]{amsmath}
\usepackage{array}
\usepackage{gensymb}
\usepackage{booktabs}
\title{Conic Assignment}

\author{Ravi Sumanth Muppana- FWC22003}
\date{September 2022}
\providecommand{\norm}[1]{\left\lVert#1\right\rVert}
\providecommand{\abs}[1]{\left\vert#1\right\vert}
\let\vec\mathbf
\newcommand{\myvec}[1]{\ensuremath{\begin{pmatrix}#1\end{pmatrix}}}
\newcommand{\mydet}[1]{\ensuremath{\begin{vmatrix}#1\end{vmatrix}}}
\providecommand{\brak}[1]{\ensuremath{\left((#1\right)}}
\begin{document}
\maketitle
\section{Problem:}
<<<<<<< HEAD
\fi
Find the equation of circle with radius $5$ whose center lies on x-axis and passes through point $\brak{2,3}$.
\\
\solution 
See Fig. 
		\ref{fig:11/11/1/12}.
	\begin{figure}[!ht]
		\centering
 \includegraphics[width=\columnwidth]{chapters/11/11/1/12/figs/conic.png}
		\caption{}
		\label{fig:11/11/1/12}
  	\end{figure}
\iffalse
=======
Find the equation of circle passing with radius $5$ whose center lies on x-axis and passes through point $(2,3)$.
>>>>>>> f531642 (Created codes and figs folder)
\maketitle
\section{Solution:}
\begin{figure}[h]
	\includegraphics[width=\linewidth]{conic.png}
\caption{Circle}
\end{figure}
\subsection{Theory:}
The circle equation when it's center and radius are given is
\begin{align}
<<<<<<< HEAD
	&\vec{(x-a)^2} + \vec{(y-b)^2} = \vec{r^2}\\
\end{align}
where the centre of the circle is $\myvec{a\\b}$.
\subsection{Mathematical Calculation:}
Given the radius of circle is $5$. The circle passes through a point $\myvec{2\\3}$. Also, the center of circle is assumed as $\myvec{a\\0}$. Substitute $\myvec{a\\0}$ in eq.$1$ we get,
\begin{align}
	&\vec{(x-a)^2} + \vec{(y)^2} = \vec{25}\\
\end{align}
As the point $\myvec{2\\3}$ passes through the circle, substitute $\myvec{2\\3}$ in the equation, we get,
\begin{align}
	&\vec{(2-a)^2} + \vec{(3)^2} = \vec{25}\\
	&\vec{4+a^2-2a} + \vec{9} = \vec{25}\\
	&\vec{a^2-2a+13} = \vec{25}\\
	&\vec{a^2-2a-12} = 0\\
\end{align}
The roots of the equation will be $(6,-2)$. Hence, the center of the circle can be $\myvec{6\\0}$ or $\myvec{-2\\0}$.
The equation of circle will therefore be,
\begin{align}
	&\vec{(x-6)^2} + \vec{y^2} = 25\\
	&\vec{(x+2)^2} + \vec{y^2} = 25
=======
	&\vec{x^Tx} + \vec{2u^Tx} + c = 0\\
	&\vec{x} = \myvec{x\\y}\\\vec{u} = \myvec{g\\f}\\
\end{align}
where the centre of the circle is $\myvec{-g\\-f}$.
\subsection{Mathematical Calculation:}
Given the radius of circle is 5, the center lies on x-axis. The circle passes through a point $\vec{P} = \myvec{2\\3}$. We can get below equations,
\fi
From the given information, the following equations can be formulated
using 
	\eqref{eq:circ-eq}.
\begin{align}
		\label{eq:11/11/1/12/1}
	\norm{\vec{P}}^2 + 2 \vec{u}^{\top}\vec{P} + f &= 0
	\\
		\label{eq:11/11/1/12/2}
	\vec{u} &= k\vec{e}_1
	\\
		\label{eq:11/11/1/12/3}
	\norm{\vec{u}}^2 - f &= r^2
\end{align}
where 
\begin{align}
	\vec{P} = \myvec{2\\3} \text{ and } r = 5
\end{align}
From 
		\eqref{eq:11/11/1/12/1}
		and 
		\eqref{eq:11/11/1/12/3},
\begin{align}
	\norm{\vec{P}}^2 + 2 \vec{u}^{\top}\vec{P} + \norm{\vec{u}}^2 &= r^2
\end{align}
Substituting from 
		\eqref{eq:11/11/1/12/2} in the above, 
\begin{align}
	k^2  + 2k \vec{e}_1^{\top}\vec{P} + \norm{\vec{P}}^2- r^2 = 0
\end{align}
resulting in 
\begin{align}
	k =  - \vec{e}_1^{\top}\vec{P} \pm \sqrt{\brak{{ \vec{e}_1^{\top}\vec{P}  }}^2 + r^2 - \norm{\vec{P}}^2 } 
\end{align}
Substituting numerical values, 
\begin{align}
	k = 2, -6
\end{align}
resulting in circles with centre
\begin{align}
	-\vec{u} = \myvec{-2 \\ 0} \text{ or } \myvec{6 \\ 0}.
\end{align}
This is verified in Fig. 
		\eqref{fig:11/11/1/12}.
\iffalse
Now,
\begin{align}
	&\myvec{0*u_x\\u_y} = \myvec{0\\0}\\
	&u_y=0\\
	&13+2\myvec{u_x\\0}\myvec{2 &3}+f = 0\\
	&\myvec{u_x\\0}\myvec{u_x &0} - f = 25
\end{align}
Solving the above yield us to the points $\myvec{-2\\0}$ and $\myvec{6\\0}$.
\begin{align}
	&\vec{x^Tx} + \vec{2u^Tx} + c = 0
\end{align}
We know that $\vec{x} = \myvec{2\\3}$ and centre $\vec{u} = \myvec{-2\\0},\myvec{6\\0}$. Substitute them.
\begin{align}
	&\vec{x^Tx} + 2\myvec{-6 &0}\vec{x} + c1 = 0\\
	&\vec{x^Tx} + 2\myvec{2 &0}\vec{x} + c2 = 0
\end{align}
The value of c is $c = g^2+f^2-r^2$. Hence c can be $11,-21$. On substitution we get, the circle equations as,
\begin{align}
	&\vec{x^Tx} + \myvec{-6 &0}\vec{x} + 11 = 0\\
	&\vec{x^Tx} + \myvec{2 &0}\vec{x} + -21 = 0\\
>>>>>>> f531642 (Created codes and figs folder)
\end{align}
\section{Construction:}

\begin{table}[h]
        \centering
\setlength\extrarowheight{2pt}
        \begin{tabular}{|c|c|c|}
                \hline
                \textbf{variable} & \textbf{length/point} & \textbf{Description}\\
                \hline
		A & np.roots(coeff) & coeff = (1,-4,-12)\\
		\hline
		c & $(a-A[0])^2+b^2-r^2$ & Circle Eqn\\
		\hline
        \end{tabular}
\end{table}
\end{document}
\fi


  \item Find the equation of the circle passing through $(0,0)$ and making intercepts $a$ and $b$ on the coordinate axes.

  \item Find the equation of a circle with centre $(2,2)$ and passes through the point $(4,5)$.
\label{chapters/11/11/1/14}
\\
\iffalse
\documentclass[journal,12pt,twocolumn]{IEEEtran}
\usepackage{setspace}
\usepackage{gensymb}
\singlespacing
\usepackage[cmex10]{amsmath}
\usepackage{amsthm}
\usepackage{mathrsfs}
\usepackage{txfonts}
\usepackage{stfloats}
\usepackage{bm}
\usepackage{cite}
\usepackage{cases}
\usepackage{subfig}
\usepackage{longtable}
\usepackage{multirow}
\usepackage{enumitem}
\usepackage{mathtools}
\usepackage{steinmetz}
\usepackage{tikz}
\usepackage{circuitikz}
\usepackage{verbatim}
\usepackage{tfrupee}
\usepackage[breaklinks=true]{hyperref}
\usepackage{tkz-euclide}
\usetikzlibrary{calc,math}
\usepackage{listings}
    \usepackage{color}                                            %%
    \usepackage{array}                                            %%
    \usepackage{longtable}                                        %%
    \usepackage{calc}                                             %%
    \usepackage{multirow}                                         %%
    \usepackage{hhline}                                           %%
    \usepackage{ifthen}                                           %%
  %optionally (for landscape tables embedded in another document): %%
    \usepackage{lscape}     
\usepackage{multicol}
\usepackage{chngcntr}
\DeclareMathOperator*{\Res}{Res}
\renewcommand\thesection{\arabic{section}}
\renewcommand\thesubsection{\thesection.\arabic{subsection}}
\renewcommand\thesubsubsection{\thesubsection.\arabic{subsubsection}}

\renewcommand\thesectiondis{\arabic{section}}
\renewcommand\thesubsectiondis{\thesectiondis.\arabic{subsection}}
\renewcommand\thesubsubsectiondis{\thesubsectiondis.\arabic{subsubsection}}

% correct bad hyphenation here
\hyphenation{op-tical net-works semi-conduc-tor}
\def\inputGnumericTable{}                                 %%

\lstset{
frame=single, 
breaklines=true,
columns=fullflexible
}

\begin{document}


\newtheorem{theorem}{Theorem}[section]
\newtheorem{problem}{Problem}
\newtheorem{proposition}{Proposition}[section]
\newtheorem{lemma}{Lemma}[section]
\newtheorem{corollary}[theorem]{Corollary}
\newtheorem{example}{Example}[section]
\newtheorem{definition}[problem]{Definition}
\newcommand{\BEQA}{\begin{eqnarray}}
\newcommand{\EEQA}{\end{eqnarray}}
\newcommand{\define}{\stackrel{\triangle}{=}}

\bibliographystyle{IEEEtran}
\providecommand{\mbf}{\mathbf}
\providecommand{\pr}[1]{\ensuremath{\Pr\left(#1\right)}}
\providecommand{\qfunc}[1]{\ensuremath{Q\left(#1\right)}}
\providecommand{\sbrak}[1]{\ensuremath{{}\left[#1\right]}}
\providecommand{\lsbrak}[1]{\ensuremath{{}\left[#1\right.}}
\providecommand{\rsbrak}[1]{\ensuremath{{}\left.#1\right]}}
\providecommand{\brak}[1]{\ensuremath{\left(#1\right)}}
\providecommand{\lbrak}[1]{\ensuremath{\left(#1\right.}}
\providecommand{\rbrak}[1]{\ensuremath{\left.#1\right)}}
\providecommand{\cbrak}[1]{\ensuremath{\left\{#1\right\}}}
\providecommand{\lcbrak}[1]{\ensuremath{\left\{#1\right.}}
\providecommand{\rcbrak}[1]{\ensuremath{\left.#1\right\}}}
\theoremstyle{remark}
\newtheorem{rem}{Remark}
\newcommand{\sgn}{\mathop{\mathrm{sgn}}}
\providecommand{\abs}[1]{\left\vert#1\right\vert}
\providecommand{\res}[1]{\Res\displaylimits_{#1}} 
\providecommand{\norm}[1]{\left\lVert#1\right\rVert}
\providecommand{\mtx}[1]{\mathbf{#1}}
\providecommand{\mean}[1]{E\left[ #1 \right]}
\providecommand{\fourier}{\overset{\mathcal{F}}{ \rightleftharpoons}}
\providecommand{\system}{\overset{\mathcal{H}}{ \longleftrightarrow}}
\newcommand{\solution}{\noindent \textbf{Solution: }}
\newcommand{\cosec}{\,\text{cosec}\,}
\providecommand{\dec}[2]{\ensuremath{\overset{#1}{\underset{#2}{\gtrless}}}}
\newcommand{\myvec}[1]{\ensuremath{\begin{pmatrix}#1\end{pmatrix}}}
\newcommand{\mydet}[1]{\ensuremath{\begin{vmatrix}#1\end{vmatrix}}}
\numberwithin{equation}{subsection}
\makeatletter
\@addtoreset{figure}{problem}
\makeatother

\let\StandardTheFigure\thefigure
\let\vec\mathbf
\renewcommand{\thefigure}{\theproblem}



\def\putbox#1#2#3{\makebox[0in][l]{\makebox[#1][l]{}\raisebox{\baselineskip}[0in][0in]{\raisebox{#2}[0in][0in]{#3}}}}
     \def\rightbox#1{\makebox[0in][r]{#1}}
     \def\centbox#1{\makebox[0in]{#1}}
     \def\topbox#1{\raisebox{-\baselineskip}[0in][0in]{#1}}
     \def\midbox#1{\raisebox{-0.5\baselineskip}[0in][0in]{#1}}

\vspace{3cm}


\title{Assignment 1}
\author{Jaswanth Chowdary Madala}





% make the title area
\maketitle

\newpage

%\tableofcontents

\bigskip

\renewcommand{\thefigure}{\theenumi}
\renewcommand{\thetable}{\theenumi}


\begin{enumerate}
\item Find the equation of a circle with centre \brak{2,2} and passes through the point \brak{4,5}.

\textbf{Solution:}
We know that the equation to the circle is given as
\begin{align}
	\norm{\vec{x}}^2+2\vec{u}^\top\vec{x}+f = 0 
\end{align}
Given the centre is \brak{2,2} and a point \brak{4,5} lies on circle
\fi
From the given information
\begin{align}
	\vec{u} = -\myvec{2\\2}, \, \vec{A} &= \myvec{4\\5}\\
\implies	\norm{\vec{A}}^2+2\vec{u}^\top\vec{A}+f &= 0\\
\implies	f = -\norm{\vec{A}}^2 - 2\vec{u}^\top\vec{A}
	&= -5
\end{align}
Hence the equation of circle is 
\begin{align}
	\norm{\vec{x}}^2+2\myvec{-2&-2}\vec{x}-5 = 0 	
\end{align}
See Fig. 
\ref{fig:chapters/11/11/1/14/1}.
\begin{figure}[ht]
\centering
\includegraphics[width = \columnwidth]{chapters/11/11/1/14/figs/fig.png}
\caption{}
\label{fig:chapters/11/11/1/14/1}
\end{figure}








  \item Does the point $(-2.5,3.5)$ lie inside, outside or on the circle $x^{2}+y^{2}=25?$
\\
\solution
\iffalse
\documentclass[12pt]{article}
\usepackage{graphicx}
\usepackage{amsmath}
\usepackage{mathtools}
\usepackage{gensymb}

\newcommand{\mydet}[1]{\ensuremath{\begin{vmatrix}#1\end{vmatrix}}}
\providecommand{\brak}[1]{\ensuremath{\left(#1\right)}}
\providecommand{\norm}[1]{\left\lVert#1\right\rVert}
\newcommand{\solution}{\noindent \textbf{Solution: }}
\newcommand{\myvec}[1]{\ensuremath{\begin{pmatrix}#1\end{pmatrix}}}
\let\vec\mathbf

\begin{document}
\begin{center}
\textbf\large{CHAPTER-11 \\ CIRCLES}

\end{center}
\section*{Excercise 11.1}

Q4.Find the equation of the circle with centre $(1,1)$ and radius $\sqrt{2}$.

\solution
\fi
Given
\begin{align}
	\vec{c} &= \myvec{1\\1} \text{ and } r = \sqrt{2},
	\\
	\vec{u}&=\vec{-c}
	 = \myvec{-1\\-1}\\
	 \\
	f &= \norm{\vec{u}}^2 - r^2
	  =0	
\end{align}
Thus, the equation of circle is 
\begin{align}
	\norm{\vec{x}}^2 -2\myvec{1&1}\vec{x} = 0       		       
\end{align}	
See Fig. 
\ref{fig:chapters/11/11/1/4/Fig1}.
\begin{figure}[!h]
	\begin{center} 
	  \includegraphics[width=\columnwidth]{chapters/11/11/1/4/figs/circ.png}
	\end{center}
\caption{}
\label{fig:chapters/11/11/1/4/Fig1}
\end{figure}

\item Find the centre of a circle passing though the points $(6,-6), (3,-7)$ and $(3,3)$. \\ 
\label{chapters/10/7/4/3}
\\
\iffalse
\documentclass[12pt]{article}
\usepackage{graphicx}
\usepackage[none]{hyphenat}
\usepackage{graphicx}
\usepackage{listings}
\usepackage[english]{babel}
\usepackage{graphicx}
\usepackage{caption} 
\usepackage{booktabs}
\usepackage{array}
\usepackage{amssymb} % for \because
\usepackage{amsmath}   % for having text in math mode
\usepackage{extarrows} % for Row operations arrows
\usepackage{listings}
\lstset{
  frame=single,
  breaklines=true
}
\usepackage{hyperref}
  
%Following 2 lines were added to remove the blank page at the beginning
\usepackage{atbegshi}% http://ctan.org/pkg/atbegshi
\AtBeginDocument{\AtBeginShipoutNext{\AtBeginShipoutDiscard}}


%New macro definitions
\newcommand{\mydet}[1]{\ensuremath{\begin{vmatrix}#1\end{vmatrix}}}
\providecommand{\brak}[1]{\ensuremath{\left(#1\right)}}
\providecommand{\norm}[1]{\left\lVert#1\right\rVert}
\providecommand{\abs}[1]{\left\vert#1\right\vert}
\newcommand{\solution}{\noindent \textbf{Solution: }}
\newcommand{\myvec}[1]{\ensuremath{\begin{pmatrix}#1\end{pmatrix}}}
\let\vec\mathbf


\begin{document}

\begin{center}
\title{\textbf{Circles}}
\date{\vspace{-5ex}} %Not to print date automatically
\maketitle
\end{center}
\setcounter{page}{1}

\section{11$^{th}$ Maths - Chapter 10}
This is Problem-3 from Exercise 10.4
\begin{enumerate}
\fi
\solution 
The equation of the circle is given by 
\begin{align}
	\label{eq:10/7/4/3circEq1}
	\norm{\vec{x}}^2+2\vec{x}^\top\vec{u}+f = 0 
\end{align}
where
\begin{align}
	\vec{u} = -\vec{c} \text{ and } \\
        \label{eq:10/7/4/3fRelation}
	f = \norm{\vec{c}}^2 - r^2
\end{align}
Given points are 
\begin{align}
	\label{eq:10/7/4/3circPoints}
     \vec{x_1} = \myvec{6 \\ -6} , \vec{x_2} = \myvec{3 \\-7}, \vec{x_3}= \myvec{3 \\ 3}
\end{align}
Substituting points from \eqref{eq:10/7/4/3circPoints} into \eqref{eq:10/7/4/3circEq1}
\begin{align}
	\brak{6^2 + \brak{-6}^2}+2\myvec{6 & -6}\vec{u}+f = 0 \\ 
	\implies 2\myvec{6 & -6}\vec{u} + f = -72 \\ 
	\brak{3^2 + \brak{-7}^2}+2\myvec{3 & -7}\vec{u}+f = 0 \\ 
	\implies 2\myvec{3 & -7}\vec{u} + f = -58 \\
	\brak{3^2 + 3^2}+2\myvec{3 & 3}\vec{u}+f = 0 \\ 
	\implies 2\myvec{3 & 3}\vec{u} + f = -18 
\end{align}
Representing the above system of equations in matrix form
\begin{align}
 \myvec{6 & -14 & 1 \\
	12 & -12 & 1 \\
	6 & 6 & 1
	} \myvec {\vec{u} \\
	           f 
		}  = \myvec{-58 \\ -72 \\ -18 }
\end{align}

The augmented matrix is expressed as
\begin{align}
	\myvec{6 & -14 & 1 & \vrule & -58 \\ 
	      12 & -12 & 1 & \vrule & -72 \\
	       6 &  6  & 1 & \vrule & -18 
	     }  
\end{align}
Performing sequence of row operations to transform into an Echelon form
\begin{align}
	\xleftrightarrow[]{{R_2\rightarrow R_2-2R_1}}  
	\myvec{6 & -14 & 1 & \vrule & -58 \\ 
	       0 &  16 & -1 & \vrule & 44 \\
	       6 &  6  & 1 & \vrule & -18 
	     }  \\ 
	\xleftrightarrow[]{{R_3\rightarrow R_3-R_1}}  
	\myvec{6 & -14 & 1 & \vrule & -58 \\ 
	       0 &  16 & -1 & \vrule & 44 \\
	       0 &  20  & 0 & \vrule & 40 
	     }  
\end{align}
\begin{align}
	\xleftrightarrow[]{{R_3\rightarrow R_3-\frac{20}{16}R_2}}  
	\myvec{6 & -14 & 1 & \vrule & -58 \\ 
	       0 &  16 & -1 & \vrule & 44 \\
	       0 &  0  &  \frac{20}{16} & \vrule & -15 
	     }  \\ 
	\xleftrightarrow[R_2\rightarrow \frac{1}{16}R_2 \text{,} R_3\rightarrow \frac{16}{20}R_3]{{R_1\rightarrow \frac{1}{6}R_1}}  
	\myvec{1 & -\frac{14}{6} & \frac{1}{6} & \vrule & -\frac{58}{6} \\ 
	       0 &  1 & -\frac{1}{16} & \vrule & \frac{44}{16} \\
	       0 &  0  &  1  & \vrule & -12 
	     }   
\end{align}
\begin{align}
	\xleftrightarrow[R_2\rightarrow R_2+\frac{1}{16}R_3]{{R_1\rightarrow R_1-\frac{1}{6}R_3}}  
	\myvec{1 & -\frac{14}{6} & 0 & \vrule & -\frac{46}{6} \\ 
	       0 &  1 & 0 & \vrule & 2 \\
	       0 &  0  &  1  & \vrule & -12 
	     }  \\ 
	\label{eq:10/7/4/3Solution}
	\xleftrightarrow[]{{R_1\rightarrow R_1+\frac{14}{6}R_2}}  
	\myvec{1 &  0 & 0 & \vrule & -3\\ 
	       0 &  1 & 0 & \vrule & 2 \\
	       0 &  0 & 1 & \vrule & -12 
	     }  
\end{align}
So, from  \eqref{eq:10/7/4/3Solution} 
\begin{align}
	\vec{u} = \myvec{-3 \\ 2} \\ 
	f = -12 
\end{align}
Since $\vec{u} = -\vec{c}$ , 
\begin{align}
	\vec{c} &= \myvec{ 3 \\ -2} \\
	\eqref{eq:10/7/4/3fRelation} \implies r^2 &= \brak{3^2 + \brak{-2}^2} + 12 \\
	 r &= 5
\end{align}
Therefore, the equation of the circle is 
\begin{align}
	\norm{\vec{x}-\myvec{3 \\ -2}}  = 5 
\end{align}
The relevant diagram is shown in Figure \ref{fig:10/7/4/3Fig1}
\begin{figure}[!h]
	\begin{center}
		\includegraphics[width=\columnwidth]{chapters/10/7/4/3/figs/problem3.pdf}
	\end{center}
\caption{}
\label{fig:10/7/4/3Fig1}
\end{figure}

\end{enumerate}
In each of the following exercises, find the equation of the circle with the following parameters
\begin{enumerate}[label=\thesection.\arabic*,ref=\thesection.\theenumi,resume*]
 \item centre $(0,2)$ and radius $2$
	 \\
		\solution
\label{chapters/11/11/1/1}
\iffalse
\documentclass[12pt]{article}
\usepackage{graphicx}
\usepackage{amsmath}
\usepackage{mathtools}
\usepackage{gensymb}

\newcommand{\mydet}[1]{\ensuremath{\begin{vmatrix}#1\end{vmatrix}}}
\providecommand{\brak}[1]{\ensuremath{\left(#1\right)}}
\providecommand{\norm}[1]{\left\lVert#1\right\rVert}
\newcommand{\solution}{\noindent \textbf{Solution: }}
\newcommand{\myvec}[1]{\ensuremath{\begin{pmatrix}#1\end{pmatrix}}}
\let\vec\mathbf

\begin{document}
\begin{center}
\textbf\large{CHAPTER-11 \\ CIRCLES}

\end{center}
\section*{Excercise 11.1}

Q4.Find the equation of the circle with centre $(1,1)$ and radius $\sqrt{2}$.

\solution
\fi
Given
\begin{align}
	\vec{c} &= \myvec{1\\1} \text{ and } r = \sqrt{2},
	\\
	\vec{u}&=\vec{-c}
	 = \myvec{-1\\-1}\\
	 \\
	f &= \norm{\vec{u}}^2 - r^2
	  =0	
\end{align}
Thus, the equation of circle is 
\begin{align}
	\norm{\vec{x}}^2 -2\myvec{1&1}\vec{x} = 0       		       
\end{align}	
See Fig. 
\ref{fig:chapters/11/11/1/4/Fig1}.
\begin{figure}[!h]
	\begin{center} 
	  \includegraphics[width=\columnwidth]{chapters/11/11/1/4/figs/circ.png}
	\end{center}
\caption{}
\label{fig:chapters/11/11/1/4/Fig1}
\end{figure}

%
  \item centre $(-2,3)$ and radius 4
	 \\
		\solution
\label{chapters/11/11/1/2}
\iffalse
\documentclass[12pt]{article}
\usepackage{graphicx}
\usepackage{amsmath}
\usepackage{mathtools}
\usepackage{gensymb}

\newcommand{\mydet}[1]{\ensuremath{\begin{vmatrix}#1\end{vmatrix}}}
\providecommand{\brak}[1]{\ensuremath{\left(#1\right)}}
\providecommand{\norm}[1]{\left\lVert#1\right\rVert}
\newcommand{\solution}{\noindent \textbf{Solution: }}
\newcommand{\myvec}[1]{\ensuremath{\begin{pmatrix}#1\end{pmatrix}}}
\let\vec\mathbf

\begin{document}
\begin{center}
\textbf\large{CLASS-11\\CHAPTER-11 \\ CIRCLES}

\end{center}
\section*{Excercise 11.1}

Q2. Find the equation of the circle with centre $(-2,3)$ and radius 4.

\solution
\\
\fi
Given
\begin{align}
	\vec{u} = -\myvec{-2\\3} \text{ and } r = 4
\end{align}
Hence, 
\begin{align}
	f = \norm{\vec{u}}^2 - r^2= -3
\end{align}
The equation of the circle is then obtained as
\begin{align}
	\norm{\vec{x}}^2 + 2\myvec{2&-3}\vec{x} -3=0     		       
\end{align}	
See Fig. 
\ref{fig:chapters/11/11/1/2/Fig1}.
\begin{figure}[!h]
	\begin{center} 
	    \includegraphics[width=\columnwidth]{chapters/11/11/1/2/figs/circle.png}
	\end{center}
\caption{}
\label{fig:chapters/11/11/1/2/Fig1}
\end{figure}



  \item centre $\left(\frac{1}{2}, \frac{1}{4}\right)$ and radius $\frac{1}{12}$
\label{chapters/11/11/1/3}
	 \\
		\solution
\iffalse
\documentclass[12pt]{article}
\usepackage{graphicx}
\usepackage{amsmath}
\usepackage{mathtools}
\usepackage{gensymb}

\newcommand{\mydet}[1]{\ensuremath{\begin{vmatrix}#1\end{vmatrix}}}
\providecommand{\brak}[1]{\ensuremath{\left(#1\right)}}
\providecommand{\norm}[1]{\left\lVert#1\right\rVert}
\newcommand{\solution}{\noindent \textbf{Solution: }}
\newcommand{\myvec}[1]{\ensuremath{\begin{pmatrix}#1\end{pmatrix}}}
\let\vec\mathbf

\begin{document}
\begin{center}
\textbf\large{CHAPTER-11 \\ CIRCLES}

\end{center}
\section*{Excercise 11.1}

Q4.Find the equation of the circle with centre $(1,1)$ and radius $\sqrt{2}$.

\solution
\fi
Given
\begin{align}
	\vec{c} &= \myvec{1\\1} \text{ and } r = \sqrt{2},
	\\
	\vec{u}&=\vec{-c}
	 = \myvec{-1\\-1}\\
	 \\
	f &= \norm{\vec{u}}^2 - r^2
	  =0	
\end{align}
Thus, the equation of circle is 
\begin{align}
	\norm{\vec{x}}^2 -2\myvec{1&1}\vec{x} = 0       		       
\end{align}	
See Fig. 
\ref{fig:chapters/11/11/1/4/Fig1}.
\begin{figure}[!h]
	\begin{center} 
	  \includegraphics[width=\columnwidth]{chapters/11/11/1/4/figs/circ.png}
	\end{center}
\caption{}
\label{fig:chapters/11/11/1/4/Fig1}
\end{figure}

  \item centre $(1,1)$ and radius $\sqrt{2}$
	 \\
		\solution
\iffalse
\documentclass[12pt]{article}
\usepackage{graphicx}
\usepackage{amsmath}
\usepackage{mathtools}
\usepackage{gensymb}

\newcommand{\mydet}[1]{\ensuremath{\begin{vmatrix}#1\end{vmatrix}}}
\providecommand{\brak}[1]{\ensuremath{\left(#1\right)}}
\providecommand{\norm}[1]{\left\lVert#1\right\rVert}
\newcommand{\solution}{\noindent \textbf{Solution: }}
\newcommand{\myvec}[1]{\ensuremath{\begin{pmatrix}#1\end{pmatrix}}}
\let\vec\mathbf

\begin{document}
\begin{center}
\textbf\large{CHAPTER-11 \\ CIRCLES}

\end{center}
\section*{Excercise 11.1}

Q4.Find the equation of the circle with centre $(1,1)$ and radius $\sqrt{2}$.

\solution
\fi
Given
\begin{align}
	\vec{c} &= \myvec{1\\1} \text{ and } r = \sqrt{2},
	\\
	\vec{u}&=\vec{-c}
	 = \myvec{-1\\-1}\\
	 \\
	f &= \norm{\vec{u}}^2 - r^2
	  =0	
\end{align}
Thus, the equation of circle is 
\begin{align}
	\norm{\vec{x}}^2 -2\myvec{1&1}\vec{x} = 0       		       
\end{align}	
See Fig. 
\ref{fig:chapters/11/11/1/4/Fig1}.
\begin{figure}[!h]
	\begin{center} 
	  \includegraphics[width=\columnwidth]{chapters/11/11/1/4/figs/circ.png}
	\end{center}
\caption{}
\label{fig:chapters/11/11/1/4/Fig1}
\end{figure}


  \item centre $(-a,-b)$ and radius $\sqrt{a^{2}-b^{2}}$.
	 \\
		\solution
\label{chapters/11/11/1/5}
\iffalse
\documentclass[12pt]{article}
\usepackage{graphicx}
\usepackage{amsmath}
\usepackage{mathtools}
\usepackage{gensymb}

\newcommand{\mydet}[1]{\ensuremath{\begin{vmatrix}#1\end{vmatrix}}}
\providecommand{\brak}[1]{\ensuremath{\left(#1\right)}}
\providecommand{\norm}[1]{\left\lVert#1\right\rVert}
\newcommand{\solution}{\noindent \textbf{Solution: }}
\newcommand{\myvec}[1]{\ensuremath{\begin{pmatrix}#1\end{pmatrix}}}
\let\vec\mathbf

\begin{document}
\begin{center}
\textbf\large{CHAPTER-11 \\ CIRCLES}

\end{center}
\section{Excercise 11.1}
Find the equation of the circle with centre $(-a,-b)$ and radius $\sqrt{a^2-b^2}$.

\section{SOLUTION}
\fi
Since
\begin{align}
	\vec{c} &= \myvec{-a\\-b} \text{ and } r = \sqrt{a^2-b^2}
	\\
	\vec{u} &= \myvec{a\\b},\,
	f = \norm{\vec{u}}^2 - r^2
	  =2b^2
\end{align}
Thus, the equation of circle is 
\begin{align}
	\norm{\vec{x}}^2 +2 \myvec{a&b}\vec{x}+2b^2 &= 0       		       
\end{align}	
See Fig.
		\ref{fig:chapters/11/11/1/5/Figure} for 
\begin{align}
	a = -3, b = -2
\end{align} 
\begin{figure}[h]
\centering
\includegraphics[width=\columnwidth]{chapters/11/11/1/5/figs/circle.png}
\caption{circle}
		\label{fig:chapters/11/11/1/5/Figure}
\end{figure}

\end{enumerate}

In each of the following exercises,  find the centre and radius of the circles.
\begin{enumerate}[label=\thesection.\arabic*,ref=\thesection.\theenumi,resume*]
\item  $x^2+y^2 +10x -6y -2=0$. 
	 \\
		\solution
\label{chapters/11/11/1/6}
\iffalse
\documentclass[12pt]{article}
\usepackage{graphicx}
\usepackage{amsmath}
\usepackage{mathtools}
\usepackage{gensymb}

\newcommand{\mydet}[1]{\ensuremath{\begin{vmatrix}#1\end{vmatrix}}}
\providecommand{\brak}[1]{\ensuremath{\left(#1\right)}}
\providecommand{\norm}[1]{\left\lVert#1\right\rVert}
\newcommand{\solution}{\noindent \textbf{Solution: }}
\newcommand{\myvec}[1]{\ensuremath{\begin{pmatrix}#1\end{pmatrix}}}
\let\vec\mathbf

\begin{document}
\begin{center}
\textbf\large{CHAPTER-11 \\ CIRCLES}

\end{center}
\section*{Excercise 11.1}

Q4.Find the equation of the circle with centre $(1,1)$ and radius $\sqrt{2}$.

\solution
\fi
Given
\begin{align}
	\vec{c} &= \myvec{1\\1} \text{ and } r = \sqrt{2},
	\\
	\vec{u}&=\vec{-c}
	 = \myvec{-1\\-1}\\
	 \\
	f &= \norm{\vec{u}}^2 - r^2
	  =0	
\end{align}
Thus, the equation of circle is 
\begin{align}
	\norm{\vec{x}}^2 -2\myvec{1&1}\vec{x} = 0       		       
\end{align}	
See Fig. 
\ref{fig:chapters/11/11/1/4/Fig1}.
\begin{figure}[!h]
	\begin{center} 
	  \includegraphics[width=\columnwidth]{chapters/11/11/1/4/figs/circ.png}
	\end{center}
\caption{}
\label{fig:chapters/11/11/1/4/Fig1}
\end{figure}

\item  $x^{2}+y^{2}-4 x-8 y-45=0$
	 \\
		\solution
\label{chapters/11/11/1/7}
\iffalse
\documentclass[12pt]{article}
\usepackage{graphicx}
\usepackage{amsmath}
\usepackage{mathtools}
\usepackage{gensymb}

\newcommand{\mydet}[1]{\ensuremath{\begin{vmatrix}#1\end{vmatrix}}}
\providecommand{\brak}[1]{\ensuremath{\left(#1\right)}}
\providecommand{\norm}[1]{\left\lVert#1\right\rVert}
\newcommand{\solution}{\noindent \textbf{Solution: }}
\newcommand{\myvec}[1]{\ensuremath{\begin{pmatrix}#1\end{pmatrix}}}
\let\vec\mathbf

\begin{document}
\begin{center}
\textbf\large{CHAPTER-11 \\ CIRCLES}

\end{center}
\section*{Excercise 11.1}

Q4.Find the equation of the circle with centre $(1,1)$ and radius $\sqrt{2}$.

\solution
\fi
Given
\begin{align}
	\vec{c} &= \myvec{1\\1} \text{ and } r = \sqrt{2},
	\\
	\vec{u}&=\vec{-c}
	 = \myvec{-1\\-1}\\
	 \\
	f &= \norm{\vec{u}}^2 - r^2
	  =0	
\end{align}
Thus, the equation of circle is 
\begin{align}
	\norm{\vec{x}}^2 -2\myvec{1&1}\vec{x} = 0       		       
\end{align}	
See Fig. 
\ref{fig:chapters/11/11/1/4/Fig1}.
\begin{figure}[!h]
	\begin{center} 
	  \includegraphics[width=\columnwidth]{chapters/11/11/1/4/figs/circ.png}
	\end{center}
\caption{}
\label{fig:chapters/11/11/1/4/Fig1}
\end{figure}

\item  $x^{2}+y^{2}-8 x+10 y-12=0$ 
	 \\
		\solution
\label{chapters/11/11/1/8}
\iffalse
\documentclass[12pt]{article}
\usepackage{graphicx}
%\documentclass[journal,12pt,twocolumn]{IEEEtran}
\usepackage[none]{hyphenat}
\usepackage{graphicx}
\usepackage{listings}
\usepackage[english]{babel}
\usepackage{graphicx}
\usepackage{caption} 
\usepackage{hyperref}
\usepackage{booktabs}
\usepackage{commath}
\usepackage{gensymb}
\usepackage{array}
\usepackage{amsmath}   % for having text in math mode
\usepackage{listings}
\lstset{
  frame=single,
  breaklines=true
}
  
%Following 2 lines were added to remove the blank page at the beginning
\usepackage{atbegshi}% http://ctan.org/pkg/atbegshi
\AtBeginDocument{\AtBeginShipoutNext{\AtBeginShipoutDiscard}}
%


%New macro definitions
\newcommand{\mydet}[1]{\ensuremath{\begin{vmatrix}#1\end{vmatrix}}}
\providecommand{\brak}[1]{\ensuremath{\left(#1\right)}}
\providecommand{\norm}[1]{\left\lVert#1\right\rVert}
\newcommand{\solution}{\noindent \textbf{Solution: }}
\newcommand{\myvec}[1]{\ensuremath{\begin{pmatrix}#1\end{pmatrix}}}
\let\vec\mathbf
\begin{document}
\begin{center}
\title{\textbf{Circles}}
\date{\vspace{-5ex}} %Not to print date automatically
\maketitle
\end{center}
\setcounter{page}{1}
\section{11$^{th}$ Maths - Exercise 11.1.8}

\begin{enumerate}
\item Find the centre and radius of the given circle $x^2+y^2-8x+10y-12=0$
\section{Solution}
\fi
From the given informtion,
\begin{align}
 \vec{u}=\myvec{-4\\5},\,
 f&=-12\\
\implies \vec{c}&=\myvec{4 \\ -5},\\
	r=\sqrt{\norm{\vec{u}}^2-f}
&=\sqrt{53}
\end{align}
See Fig. 
\ref{fig:chapters/11/11/1/8/Fig1}.
\begin{figure}[!h]
	\begin{center} 
	   \includegraphics[width=\columnwidth]{chapters/11/11/1/8/figs/11.1.8.png}
	\end{center}
\caption{}
\label{fig:chapters/11/11/1/8/Fig1}
\end{figure}

\item  $2 x^{2}+2 y^{2}-x=0$
	 \\
		\solution
\label{chapters/11/11/1/9}
\iffalse
\documentclass[12pt]{article}
\usepackage{graphicx}
%\documentclass[journal,12pt,twocolumn]{IEEEtran}
\usepackage[none]{hyphenat}
\usepackage{graphicx}
\usepackage{listings}
\usepackage[english]{babel}
\usepackage{graphicx}
\usepackage{caption} 
\usepackage{hyperref}
\usepackage{booktabs}
\usepackage{commath}
\usepackage{gensymb}
\usepackage{array}
\usepackage{amsmath}   % for having text in math mode
\usepackage{listings}
\lstset{
  frame=single,
  breaklines=true
}
  
%Following 2 lines were added to remove the blank page at the beginning
\usepackage{atbegshi}% http://ctan.org/pkg/atbegshi
\AtBeginDocument{\AtBeginShipoutNext{\AtBeginShipoutDiscard}}
%


%New macro definitions
\newcommand{\mydet}[1]{\ensuremath{\begin{vmatrix}#1\end{vmatrix}}}
\providecommand{\brak}[1]{\ensuremath{\left(#1\right)}}
\providecommand{\norm}[1]{\left\lVert#1\right\rVert}
\newcommand{\solution}{\noindent \textbf{Solution: }}
\newcommand{\myvec}[1]{\ensuremath{\begin{pmatrix}#1\end{pmatrix}}}
\let\vec\mathbf
\begin{document}
\begin{center}
\title{\textbf{Circles}}
\date{\vspace{-5ex}} %Not to print date automatically
\maketitle
\end{center}
\setcounter{page}{1}
\section{11$^{th}$ Maths - Exercise 11.1.9}

\begin{enumerate}
\item Find the centre and radius of the given circle $2x^2+2y^2-x=0$
\section{Solution}
\fi
The given equation can be expressed as 
\begin{align}
	x^2+y^2-\frac{x}{2}&=0
	\\
\implies 	\norm{\vec{x}}^2+2\myvec{\frac{-1}{4} & 0}\vec{x}&=0
\end{align}	
The centre of circle is then given by 
\begin{align}
	\vec{u} = -\vec{c} 
=\myvec{\frac{1}{4}\\0}
\end{align}
and the radius of circle is obtained as
\begin{align}
	r=\sqrt{\norm{\vec{u}}^2 -f}
=\frac{1}{4}
\end{align}
See Fig. 
  \ref{fig:chapters/11/11/1/9/Figure}.
\begin{figure}[h]
\includegraphics[width=\columnwidth]{chapters/11/11/1/9/figs/fig.png}
\caption{}
  \label{fig:chapters/11/11/1/9/Figure}
\end{figure}

\item
Find the equation of the circle with radius 5 whose centre lies on $x$-axis and passes through the point $\brak{2,3}$.

\textbf{Solution :}
\begin{table}[H]
    \centering
        \begin{tabular}{|c|c|c|}
    \hline
         \textbf{Input parameters}& \textbf{Description}&\textbf{Value} \\
         \hline
         $r$ & Radius&$5$ \\
        \hline
        $\vec{O}$ & Center&$x\vec{e_1}$ \\
        \hline
       $\vec{A}$&Point &$\myvec{2\\3}$ \\
       \hline
    \end{tabular}

        \caption{Table of input parameters}
    \label{tab:11.11.1.13}
\end{table}
The general formula of the circle is
\begin{align}
\norm{\vec{x}}^2 + 2\vec{u}^{\top}\vec{x}+f&=0\\
	where,   \vec{u}&=-x\vec{e_1}\\
	f&=\norm{\vec{O}}-r^2\\
f&=x-r^2\\
\norm{\vec{A}}^2 + 2\vec{u}^{\top}\vec{A}+f&=0\\
13-4x+x-r^2&=0\\
or,x&=-4\\
or,f&=-29
\end{align}
Therefore the equations of the circle are
\begin{align}
   \norm{\vec{x}}^2 - 2\myvec{-4&0}\vec{x}-29&=0\\
\end{align}    
\begin{figure}[H]
    \centering
	\includegraphics[width=\columnwidth]{chapters/11/11/1/13/fig/11.11.1.13.png}
    \caption{}
    \label{fig:11.11.1.13}
\end{figure}



\end{enumerate}

\subsection{Exercises}
\begin{enumerate}[label=\thesection.\arabic*,ref=\thesection.\theenumi]
\item The area of the circle centred ot (1,2) and passing through (4,6) is
\begin{enumerate}
\item 5$\mu$ 
\item 10$\mu$
 \item 25$\mu$ 
\item none of these
\end{enumerate}
\item equation of the circle with centre on the Y-axis and passing through the orgin and the point (2,3) is
\begin{enumerate}
\item $x^2+y^2+6x+6y+3=0$ 
\item $x^2+y^2-6x-6y-9=0$
\item $x^2+y^2-6x-6y+9=0$
\item none of these
\end{enumerate}
\item equation of the circle with centre on the  y-axis and passing through the origin and the point (2,3) is  
\begin{enumerate}
\item $x^2+y^2+13y=0$
\item $3x^2+3y^2+13x+3=0$
\item $6x^2+6y^2-13x=0$
\item $x^2+y^2+13x+3=0$
\end{enumerate}
\item The equation of a circle with origin as centre and passing through the vertices of an equilateral triangle whose median is of lengh 3 a is
\begin{enumerate}
\item $x^2+y^2=9a^2$
\item $x^2+y^2=16a^2$
\item $x^2+y^2=4a^2$
\item $x^2+y^2=a^2$
	[Hint: centroid of the triangle caincdes with the centre of the circle and the redius of the circle is $\frac{2}{3}$of the length of the median]
\end{enumerate}
\end{enumerate}
In each of the following exercises, find the equation of the circle with the following parameters
\begin{enumerate}[label=\thesection.\arabic*,ref=\thesection.\theenumi,resume*]
\numberwithin{equation}{enumi}
\numberwithin{figure}{enumi}
\numberwithin{table}{enumi}
 \item Find the equation of a circle concentric with the circle $x^2+y^2-6x+12y+15=0$ and has double of its area.
 \item If one end of a diameter of the circle $x^2+y^2-4x-6y+11 =0$ is (3,4), then find the coordinate of the other end of the diameter.
 \item Find the equation of the circle having (1,-2) as its centre  and passing through $3x+y=14, 2x+5y=18$.
 \item Show that the point $(x,y)$ given by $x=\frac{2at}{1+t^2}$ and $y=\frac{a(1-t^2)}{1+t^2}$ lies on a circle for all real values of $t$ such that -1$\le t \le $1 where $a$ is any given real numbers. 
 \item lf a circle passes through the point $(0,0),(a,0) and (0,b)$ then find the coordinates of its centre.
\item If the lines 2x-3y=5 and 3x-4y=7 are the diameters of a circie of area 154 squar untits, then obtian the equation of the circle.
\item Find the equation of the circle which passes through the points (2,3) and (4,5) and the centre lies on the straight line y-4x+3=0.
\item Find the equation of a circle whose centre is (3,1) and which cuts offachord of length  6 units on the  line 2x-5y+18=0
[Hint:To detemine the radius of the circle, find the perpendicular distance from the centre to the given line]
\item Find the equation of a circle of redius 5 which is touching another circle $x^2+y^2-2x-4y-20=0$ at (5,5).
\item Find the equation of a circle passing through the point (7,3) having radius 3 units and whose centre lies on the line y=x-1
\end{enumerate}
State whether the statements are True or False 
\begin{enumerate}[label=\thesection.\arabic*,ref=\thesection.\theenumi,resume*]
\item The line $x^2+3y=0$ is a diameter of the circle $x^2+y^2+6x+2y=0$.
\item The point (1,2) lies inside the circle $x^2+y^2-2x+6y+1=0$,
\end{enumerate}
Fill in the Blanks
\begin{enumerate}[label=\thesection.\arabic*,ref=\thesection.\theenumi,resume*]
\item The equation of the circle having centre at (3,-4) and touching the line 5x+12y-12=0 is \makebox[1cm]{\hrulefill}                     
[Hint: To determine radius find the perpendicular distance  from the centre of the circle to the line.]
\item The equation of the circle circumscribing the triangle whose sides are the lines y=x+2,3y=4x,2y=3x is  \makebox[1cm]{\hrulefill}         
\end{enumerate}

\subsection{Construction}
\begin{enumerate}[label=\thesection.\arabic*,ref=\thesection.\theenumi]
\numberwithin{equation}{enumi}
\numberwithin{figure}{enumi}
\numberwithin{table}{enumi}
\item 
\label{chapters/9/10/4/1}
\iffalse
\documentclass[12pt]{article}
\usepackage{graphicx}
\usepackage{amsmath}
\usepackage{mathtools}
\usepackage{gensymb}

\newcommand{\mydet}[1]{\ensuremath{\begin{vmatrix}#1\end{vmatrix}}}
\providecommand{\brak}[1]{\ensuremath{\left(#1\right)}}
\providecommand{\norm}[1]{\left\lVert#1\right\rVert}
\newcommand{\solution}{\noindent \textbf{Solution: }}
\newcommand{\myvec}[1]{\ensuremath{\begin{pmatrix}#1\end{pmatrix}}}
\let\vec\mathbf

\begin{document}
\begin{center}
\textbf\large{CHAPTER-11 \\ CIRCLES}

\end{center}
\section*{Excercise 11.1}

Q4.Find the equation of the circle with centre $(1,1)$ and radius $\sqrt{2}$.

\solution
\fi
Given
\begin{align}
	\vec{c} &= \myvec{1\\1} \text{ and } r = \sqrt{2},
	\\
	\vec{u}&=\vec{-c}
	 = \myvec{-1\\-1}\\
	 \\
	f &= \norm{\vec{u}}^2 - r^2
	  =0	
\end{align}
Thus, the equation of circle is 
\begin{align}
	\norm{\vec{x}}^2 -2\myvec{1&1}\vec{x} = 0       		       
\end{align}	
See Fig. 
\ref{fig:chapters/11/11/1/4/Fig1}.
\begin{figure}[!h]
	\begin{center} 
	  \includegraphics[width=\columnwidth]{chapters/11/11/1/4/figs/circ.png}
	\end{center}
\caption{}
\label{fig:chapters/11/11/1/4/Fig1}
\end{figure}

\item If two equal chords of a circle intersect within the circle, prove 
that the segments of one chord are equal to corresponding segments of the other 
chord.
\label{chapters/9/10/4/2}
\\
\solution 
\iffalse
\documentclass[10pt]{article}
       \usepackage[latin1]{inputenc}
       \usepackage{fullpage}
       \usepackage{color}
       \usepackage{array}
       \usepackage{longtable}
       \usepackage{calc}
       \usepackage{multirow}
       \usepackage{hhline}
       \usepackage{ifthen}
\usepackage{graphicx}
\def\inputGnumericTable{}
\usepackage[none]{hyphenat}
\usepackage{graphicx}
\usepackage{listings}
\usepackage[english]{babel}
\usepackage{graphicx}
\usepackage{caption} 
\usepackage{booktabs}
\usepackage{gensymb}
\usepackage{array}
\usepackage{amssymb} % for \because
\usepackage{amsmath}   % for having text in math mode
\usepackage{extarrows} % for Row operations arrows
\usepackage{listings}
\lstset{
  frame=single,
  breaklines=true
}
\usepackage{hyperref}
%Following 2 lines were added to remove the blank page at the beginning
\usepackage{atbegshi}% http://ctan.org/pkg/atbegshi
\AtBeginDocument{\AtBeginShipoutNext{\AtBeginShipoutDiscard}}
%New macro definitions
\newcommand{\mydet}[1]{\ensuremath{\begin{vmatrix}#1\end{vmatrix}}}
\providecommand{\brak}[1]{\ensuremath{\left(#1\right)}}
\providecommand{\norm}[1]{\left\lVert#1\right\rVert}
\newcommand{\solution}{\noindent \textbf{Solution: }}
\newcommand{\myvec}[1]{\ensuremath{\begin{pmatrix}#1\end{pmatrix}}}
\providecommand{\abs}[1]{\left\vert#1\right\vert}
\let\vec\mathbf
\begin{document}
\begin{center}
\title{\textbf{Properties of Circle}}
\date{\vspace{-5ex}} %Not to print date automatically
\maketitle
\end{center}
\setcounter{page}{1}
\section{9$^{th}$ Maths - Chapter 10}
       This is Problem-2 from Exercise 10.4
\begin{enumerate}
\item If two equal chords of a circle intersect within the circle, prove that the segments of one chord are equal to corresponding segments of other chord.

\solution:
\fi
\begin{figure}[h!]
	\begin{center} 
	  \includegraphics[width=\columnwidth]{chapters/9/10/4/2/figs/c.png}
	\end{center}
\caption{Two equal chords intersecting in a circle}
\label{fig:chapters/9/10/4/2/Fig1}
\end{figure} 
See Table 
\ref{tab:chapters/9/10/4/2/}
for the input  parameters.
\begin{table}[h!]
	\begin{tabular}{|c|c|p{5cm}|}
\hline
\textbf{Symbol} & \textbf{Value} & \textbf{Description} \\
\hline
$\theta$ & $30\degree$ & $\angle{BAP} = \angle{BAQ}$ \\
\hline
$a$ & $9$ & $AB$ \\
\hline
$c$ & $8$ & $AQ$ \\
\hline
$\vec{e}_1$ & $\myvec{1\\0}$ & Basis vector \\
\hline
\end{tabular}

\caption{}
\label{tab:chapters/9/10/4/2/}
\end{table}
Consider
\begin{align}
\vec{P}=\myvec{\cos \theta_1\\\sin \theta_1},\,
\vec{Q}=\myvec{\cos \theta_2\\\sin \theta_2},\,
\vec{R}=\myvec{\cos \theta_3\\\sin \theta_3},\,
\vec{S}=\myvec{\cos \theta_4\\\sin \theta_4}
\label{eq:chapters/9/10/4/2/table1}
\end{align}
such that 
\begin{align}
	\vec{P}-\vec{Q}&=\myvec{\cos\theta_1-\cos\theta_2 \\ \sin \theta_1-\sin \theta_2}
	\\
	\implies \norm{\vec{P}-\vec{Q}}^2 &= 
	\brak{\cos \theta_1-\cos \theta_2}^2+\brak{\sin \theta_1-\sin \theta_2}^2=d^2\\
\end{align}
yielding
\begin{align}
	\brak{\frac{\theta_1-\theta_2}{2}}=\sin^{-1}\brak{\frac{d}{2}}.
	\end{align}
	Similarly, 
\begin{align}
\brak{\frac{\theta_3-\theta_4}{2}}=\sin^{-1}\brak{\frac{d}{2}}.
\end{align}
The equations of $PQ$ and $RS$ are obtained using 
\begin{align}
\vec{{n}_1^{\top}}\brak{\vec{x}-\vec{P}}&=0\\
\vec{{n}_2^{\top}}\brak{\vec{x}-\vec{R}}&=0
\end{align}
where 
\begin{align}
\vec{n}_1
&=\myvec{\sin \theta_1-\sin \theta_2\\\cos \theta_2-\cos \theta_1}\\
\vec{n}_2
&=\myvec{\sin \theta_3-\sin \theta_4\\\cos \theta_4-\cos \theta_3}
\end{align}
Substiuting numerical values, the 
point of intersection of lines $PQ,RS$ is 
\begin{align}
\vec{T}=\myvec{0.68341409\\-0.04288508}
\end{align}
Thus, 
\begin{align}
\norm{\vec{P}-\vec{T}}=
\norm{\vec{S}-\vec{T}}&=0.5727
\end{align}

\item If a line intersects two concentric circles (circles
with the same centre) with centre $\vec{O}$ at $\vec{A}$, $\vec{B}$, $\vec{C}$ and $\vec{D}$, prove that $AB = CD$.
		\label{chapters/9/10/4/4/}
\\
\solution 
\iffalse
\documentclass[12pt]{article}
\usepackage{graphicx}
\usepackage{amsmath}
\usepackage{mathtools}
\usepackage{gensymb}

\newcommand{\mydet}[1]{\ensuremath{\begin{vmatrix}#1\end{vmatrix}}}
\providecommand{\brak}[1]{\ensuremath{\left(#1\right)}}
\providecommand{\norm}[1]{\left\lVert#1\right\rVert}
\newcommand{\solution}{\noindent \textbf{Solution: }}
\newcommand{\myvec}[1]{\ensuremath{\begin{pmatrix}#1\end{pmatrix}}}
\let\vec\mathbf

\begin{document}
\begin{center}
\textbf\large{CHAPTER-11 \\ CIRCLES}

\end{center}
\section*{Excercise 11.1}

Q4.Find the equation of the circle with centre $(1,1)$ and radius $\sqrt{2}$.

\solution
\fi
Given
\begin{align}
	\vec{c} &= \myvec{1\\1} \text{ and } r = \sqrt{2},
	\\
	\vec{u}&=\vec{-c}
	 = \myvec{-1\\-1}\\
	 \\
	f &= \norm{\vec{u}}^2 - r^2
	  =0	
\end{align}
Thus, the equation of circle is 
\begin{align}
	\norm{\vec{x}}^2 -2\myvec{1&1}\vec{x} = 0       		       
\end{align}	
See Fig. 
\ref{fig:chapters/11/11/1/4/Fig1}.
\begin{figure}[!h]
	\begin{center} 
	  \includegraphics[width=\columnwidth]{chapters/11/11/1/4/figs/circ.png}
	\end{center}
\caption{}
\label{fig:chapters/11/11/1/4/Fig1}
\end{figure}

\item If a line intersects two concentric circles (circles with the same 
centre) with centre $\vec{O}$ at $\vec{A}, \vec{B}, \vec{C}, \vec{D}$, prove 
that $AB = CD$ (see Fig. 
		\ref{fig:chapters/9/10/41} ).
\begin{figure}[!ht]
    \centering
    \includegraphics[width=\columnwidth]{chapters/9/10/4/figs/fig1.jpg}
    \caption{}
    \label{fig:chapters/9/10/41}
\end{figure}
\item Three girls Reshma, Salma and Mandip are playing a game by standing on 
a circle of radius 5m drawn in a park. Reshma throws a ball to Salma, Salma to 
Mandip, Mandip to Reshma. If the distance between Reshma and Salma and between 
Salma and Mandip is 6m each, what is the distance between Reshma and Mandip?
\\
\solution 
\iffalse
\documentclass[journal,12pt,twocolumn]{IEEEtran}
%
\usepackage{setspace}
\usepackage{gensymb}
%\doublespacing
\singlespacing

%\usepackage{graphicx}
%\usepackage{amssymb}
%\usepackage{relsize}
\usepackage[cmex10]{amsmath}
%\usepackage{amsthm}
%\interdisplaylinepenalty=2500
%\savesymbol{iint}
%\usepackage{txfonts}
%\restoresymbol{TXF}{iint}
%\usepackage{wasysym}
\usepackage{amsthm}
%\usepackage{iithtlc}
\usepackage{mathrsfs}
\usepackage{txfonts}
\usepackage{stfloats}
\usepackage{bm}
\usepackage{cite}
\usepackage{cases}
\usepackage{subfig}
%\usepackage{xtab}
\usepackage{longtable}
\usepackage{multirow}
%\usepackage{algorithm}
%\usepackage{algpseudocode}
\usepackage{enumitem}
\usepackage{mathtools}
\usepackage{steinmetz}
\usepackage{tikz}
\usepackage{circuitikz}
\usepackage{verbatim}
\usepackage{tfrupee}
\usepackage[breaklinks=true]{hyperref}
%\usepackage{stmaryrd}
\usepackage{tkz-euclide} % loads  TikZ and tkz-base
%\usetkzobj{all}
\usetikzlibrary{calc,math}
\usepackage{listings}
    \usepackage{color}                                            %%
    \usepackage{array}                                            %%
    \usepackage{longtable}                                        %%
    \usepackage{calc}                                             %%
    \usepackage{multirow}                                         %%
    \usepackage{hhline}                                           %%
    \usepackage{ifthen}                                           %%
  %optionally (for landscape tables embedded in another document): %%
    \usepackage{lscape}     
\usepackage{multicol}
\usepackage{chngcntr}
%\usepackage{enumerate}

%\usepackage{wasysym}
%\newcounter{MYtempeqncnt}
\DeclareMathOperator*{\Res}{Res}
%\renewcommand{\baselinestretch}{2}
\renewcommand\thesection{\arabic{section}}
\renewcommand\thesubsection{\thesection.\arabic{subsection}}
\renewcommand\thesubsubsection{\thesubsection.\arabic{subsubsection}}

\renewcommand\thesectiondis{\arabic{section}}
\renewcommand\thesubsectiondis{\thesectiondis.\arabic{subsection}}
\renewcommand\thesubsubsectiondis{\thesubsectiondis.\arabic{subsubsection}}

% correct bad hyphenation here
\hyphenation{op-tical net-works semi-conduc-tor}
\def\inputGnumericTable{}                                 %%

\lstset{
%language=C,
frame=single, 
breaklines=true,
columns=fullflexible
}
%\lstset{
%language=tex,
%frame=single, 
%breaklines=true
%}

\begin{document}
%


\newtheorem{theorem}{Theorem}[section]
\newtheorem{problem}{Problem}
\newtheorem{proposition}{Proposition}[section]
\newtheorem{lemma}{Lemma}[section]
\newtheorem{corollary}[theorem]{Corollary}
\newtheorem{example}{Example}[section]
\newtheorem{definition}[problem]{Definition}
%\newtheorem{thm}{Theorem}[section] 
%\newtheorem{defn}[thm]{Definition}
%\newtheorem{algorithm}{Algorithm}[section]
%\newtheorem{cor}{Corollary}
\newcommand{\BEQA}{\begin{eqnarray}}
\newcommand{\EEQA}{\end{eqnarray}}
\newcommand{\define}{\stackrel{\triangle}{=}}

\bibliographystyle{IEEEtran}
%\bibliographystyle{ieeetr}


\providecommand{\mbf}{\mathbf}
\providecommand{\pr}[1]{\ensuremath{\Pr\left(#1\right)}}
\providecommand{\qfunc}[1]{\ensuremath{Q\left(#1\right)}}
\providecommand{\sbrak}[1]{\ensuremath{{}\left[#1\right]}}
\providecommand{\lsbrak}[1]{\ensuremath{{}\left[#1\right.}}
\providecommand{\rsbrak}[1]{\ensuremath{{}\left.#1\right]}}
\providecommand{\brak}[1]{\ensuremath{\left(#1\right)}}
\providecommand{\lbrak}[1]{\ensuremath{\left(#1\right.}}
\providecommand{\rbrak}[1]{\ensuremath{\left.#1\right)}}
\providecommand{\cbrak}[1]{\ensuremath{\left\{#1\right\}}}
\providecommand{\lcbrak}[1]{\ensuremath{\left\{#1\right.}}
\providecommand{\rcbrak}[1]{\ensuremath{\left.#1\right\}}}
\theoremstyle{remark}
\newtheorem{rem}{Remark}
\newcommand{\sgn}{\mathop{\mathrm{sgn}}}
\providecommand{\abs}[1]{\left\vert#1\right\vert}
\providecommand{\res}[1]{\Res\displaylimits_{#1}} 
\providecommand{\norm}[1]{\left\lVert#1\right\rVert}
%\providecommand{\norm}[1]{\lVert#1\rVert}
\providecommand{\mtx}[1]{\mathbf{#1}}
\providecommand{\mean}[1]{E\left[ #1 \right]}
\providecommand{\fourier}{\overset{\mathcal{F}}{ \rightleftharpoons}}
%\providecommand{\hilbert}{\overset{\mathcal{H}}{ \rightleftharpoons}}
\providecommand{\system}{\overset{\mathcal{H}}{ \longleftrightarrow}}
	%\newcommand{\solution}[2]{\textbf{Solution:}{#1}}
\newcommand{\solution}{\noindent \textbf{Solution: }}
\newcommand{\cosec}{\,\text{cosec}\,}
\providecommand{\dec}[2]{\ensuremath{\overset{#1}{\underset{#2}{\gtrless}}}}
\newcommand{\myvec}[1]{\ensuremath{\begin{pmatrix}#1\end{pmatrix}}}
\newcommand{\mydet}[1]{\ensuremath{\begin{vmatrix}#1\end{vmatrix}}}
%\numberwithin{equation}{section}
\numberwithin{equation}{subsection}
%\numberwithin{problem}{section}
%\numberwithin{definition}{section}
\makeatletter
\@addtoreset{figure}{problem}
\makeatother

\let\StandardTheFigure\thefigure
\let\vec\mathbf
%\renewcommand{\thefigure}{\theproblem.\arabic{figure}}
\renewcommand{\thefigure}{\theproblem}
%\setlist[enumerate,1]{before=\renewcommand\theequation{\theenumi.\arabic{equation}}
%\counterwithin{equation}{enumi}


%\renewcommand{\theequation}{\arabic{subsection}.\arabic{equation}}

\def\putbox#1#2#3{\makebox[0in][l]{\makebox[#1][l]{}\raisebox{\baselineskip}[0in][0in]{\raisebox{#2}[0in][0in]{#3}}}}
     \def\rightbox#1{\makebox[0in][r]{#1}}
     \def\centbox#1{\makebox[0in]{#1}}
     \def\topbox#1{\raisebox{-\baselineskip}[0in][0in]{#1}}
     \def\midbox#1{\raisebox{-0.5\baselineskip}[0in][0in]{#1}}

\vspace{3cm}


\title{Question: 9.10.4.5}
\author{Nikam Pratik Balasaheb (EE21BTECH11037)}





% make the title area
\maketitle

\newpage

%\tableofcontents

\bigskip

\renewcommand{\thefigure}{\theenumi}
\renewcommand{\thetable}{\theenumi}
%\renewcommand{\theequation}{\theenumi}

\section{Problem}
Three girls Reshma, Salma and Mandip are playing a game by standing on a circle of radius 5m drawn in a park. Reshma throws a ball to Salma, Salma to Mandip, Mandip to Reshma. If the distance between Reshma and Salma and
between Salma and Mandip is 6m each, what is the distance between Reshma and Mandip?

\section{Solution}
\fi
Consider Reshma, Salma and Mandip be standing at $\vec{A}$, $\vec{B}$ and $\vec{C}$ respectively, and the center the of the circle $\vec{O}$.
The input parameters are listed in Table 
\ref{tab:chapters/9/10/4/5/}.
Let 
\begin{align}
	\vec{B} = \myvec{0\\0},\,
	\vec{O} = \myvec{5\\0}
\end{align}
Therefore, the equation of the cicle is given by 
\begin{align}
	\norm{\vec{x}-\vec{O}}^2 &= 25\\
\implies 	\norm{\vec{x}}^2 - 2 \vec{O}^{\top}\vec{x} + \norm{\vec{O}}^2 - 25 &= 0\\
\implies	\norm{\vec{x}}^2 - 2 \myvec{5 & 0}\vec{x} &= 0
	\label{eq:chapters/9/10/4/5/1}
\end{align}
Also, $\vec{A}$ and $\vec{C}$ are equidistant (6m) from $\vec{B}$, we can say that they lie on the circle having $\vec{B}$ as center and radius 6m. Equation of this circle is given by
\begin{align}
	\norm{\vec{x}}^2 - 2\vec{B}^{\top} \vec{x} + \norm{\vec{B}}^2 - 36 &= 0\\
	\norm{\vec{x}}^2 &= 36\\
	i.e., \vec{u} = \myvec{0\\0}\;, \; f = -36
	\label{eq:chapters/9/10/4/5/2}
\end{align}
From \eqref{eq:chapters/9/10/4/5/1} and \eqref{eq:chapters/9/10/4/5/2}, the line passing through $\vec{A}$ and $\vec{C}$  is
\begin{align}
	\myvec{5&0} \vec{x} &= 18\\
\implies 	\vec{x} &= \myvec{\frac{18}{5}\\[1pt] 0} + \mu \myvec{0\\1}\\
	i.e., \; \vec{h} = \myvec{\frac{18}{5}\\[1pt] 0} \; , \; \vec{m} = \myvec{0\\1}
\end{align}
For the circle,$\vec{V} = \vec{I}$
\begin{multline}
	\mu_i = \frac{1}{\vec{m}^{\top}\vec{V}\vec{m}} \brak{-m^{\top}\brak{\vec{V}\vec{h}+\vec{u}} \pm \sqrt{\brak{\vec{m}^{\top}\brak{\vec{V}\vec{h}+\vec{u}}}^2 - \text{g}\brak{\vec{h}}\brak{\vec{m}^{\top}\vec{V}\vec{m}}}}
\end{multline}
where,
\begin{align}
	\text{g}\brak{\vec{h}} = \vec{h}^{\top}\vec{V}\vec{h} + 2\vec{u}^{\top}\vec{h} +f
\end{align}
yielding
\begin{align}
	\mu_i = \pm \frac{24}{5}
\end{align}
Therefore,
\begin{align}
	\vec{A} = \myvec{\frac{18}{5}\\[1pt] \frac{24}{5}},\,
	\vec{C} = \myvec{\frac{18}{5}\\[1pt] -\frac{24}{5}}
\end{align}
and the 
distance between Reshma and Mandip is
\begin{align}
	\norm{\vec{A} - \vec{C}} = \norm{\myvec{0\\[1pt] \frac{48}{5}}}
	= \frac{48}{5}
\end{align}
\begin{table}[h!]
\begin{center}
	%%%%%%%%%%%%%%%%%%%%%%%%%%%%%%%%%%%%%%%%%%%%%%%%%%%%%%%%%%%%%%%%%%%%%%
%%                                                                  %%
%%  This is a LaTeX2e table fragment exported from Gnumeric.        %%
%%                                                                  %%
%%%%%%%%%%%%%%%%%%%%%%%%%%%%%%%%%%%%%%%%%%%%%%%%%%%%%%%%%%%%%%%%%%%%%%

\begin{tabular}[]{|c|c|c|}
\hline
$\vec{O}$	& $\myvec{5\\0}$ &Center of the given circle \\ \hline
$\vec{B}$	& $\myvec{0\\0}$ &Point where Salma is standing\\ \hline
$r$		& 5 & radius of given circle \\ \hline
$d$ 		& 6 & distance AB and BC\\ \hline
\end{tabular}

\end{center}
\caption{}
\label{tab:chapters/9/10/4/5/}
\end{table}
See Fig. 
    \ref{fig:chapters/9/10/4/5/}.
\begin{figure}[h!]
  \centering
    \includegraphics[width=\columnwidth]{chapters/9/10/4/5/figs/Figure_1.png}
    \caption{}
    \label{fig:chapters/9/10/4/5/}
\end{figure}





\item A circular park of radius 20m is situated in a colony. Three boys Ankur,
Syed and David are sitting at equal distance on its boundary each having a toy 
telephone in his hands to talk each other. Find the length of the string of each 
phone.
\item If two equal chords of a circle intersect within the circle, prove 
that the line joining the point of intersection to the centre makes equal 
angles with the chords.
\label{chapters/9/10/4/6}
%\label{chapters/9/10/4/6}
\\
\solution 
%\iffalse
\documentclass[12pt]{article}
\usepackage{graphicx}
\usepackage{amsmath}
\usepackage{mathtools}
\usepackage{gensymb}

\newcommand{\mydet}[1]{\ensuremath{\begin{vmatrix}#1\end{vmatrix}}}
\providecommand{\brak}[1]{\ensuremath{\left(#1\right)}}
\providecommand{\norm}[1]{\left\lVert#1\right\rVert}
\newcommand{\solution}{\noindent \textbf{Solution: }}
\newcommand{\myvec}[1]{\ensuremath{\begin{pmatrix}#1\end{pmatrix}}}
\let\vec\mathbf

\begin{document}
\begin{center}
\textbf\large{CHAPTER-11 \\ CIRCLES}

\end{center}
\section*{Excercise 11.1}

Q4.Find the equation of the circle with centre $(1,1)$ and radius $\sqrt{2}$.

\solution
\fi
Given
\begin{align}
	\vec{c} &= \myvec{1\\1} \text{ and } r = \sqrt{2},
	\\
	\vec{u}&=\vec{-c}
	 = \myvec{-1\\-1}\\
	 \\
	f &= \norm{\vec{u}}^2 - r^2
	  =0	
\end{align}
Thus, the equation of circle is 
\begin{align}
	\norm{\vec{x}}^2 -2\myvec{1&1}\vec{x} = 0       		       
\end{align}	
See Fig. 
\ref{fig:chapters/11/11/1/4/Fig1}.
\begin{figure}[!h]
	\begin{center} 
	  \includegraphics[width=\columnwidth]{chapters/11/11/1/4/figs/circ.png}
	\end{center}
\caption{}
\label{fig:chapters/11/11/1/4/Fig1}
\end{figure}

\iffalse
\documentclass[12pt]{article}
\usepackage{graphicx}
\usepackage{amsmath}
\usepackage{mathtools}
\usepackage{gensymb}

\newcommand{\mydet}[1]{\ensuremath{\begin{vmatrix}#1\end{vmatrix}}}
\providecommand{\brak}[1]{\ensuremath{\left(#1\right)}}
\providecommand{\norm}[1]{\left\lVert#1\right\rVert}
\newcommand{\solution}{\noindent \textbf{Solution: }}
\newcommand{\myvec}[1]{\ensuremath{\begin{pmatrix}#1\end{pmatrix}}}
\let\vec\mathbf

\begin{document}
\begin{center}
\textbf\large{CHAPTER-11 \\ CIRCLES}

\end{center}
\section*{Excercise 11.1}

Q4.Find the equation of the circle with centre $(1,1)$ and radius $\sqrt{2}$.

\solution
\fi
Given
\begin{align}
	\vec{c} &= \myvec{1\\1} \text{ and } r = \sqrt{2},
	\\
	\vec{u}&=\vec{-c}
	 = \myvec{-1\\-1}\\
	 \\
	f &= \norm{\vec{u}}^2 - r^2
	  =0	
\end{align}
Thus, the equation of circle is 
\begin{align}
	\norm{\vec{x}}^2 -2\myvec{1&1}\vec{x} = 0       		       
\end{align}	
See Fig. 
\ref{fig:chapters/11/11/1/4/Fig1}.
\begin{figure}[!h]
	\begin{center} 
	  \includegraphics[width=\columnwidth]{chapters/11/11/1/4/figs/circ.png}
	\end{center}
\caption{}
\label{fig:chapters/11/11/1/4/Fig1}
\end{figure}

\item  $\vec{A},\vec{B},\vec{C}$ are the three points on a circle with centre $\vec{O}$ such that $\angle$BOC=30\degree and $\angle$AOB=60\degree. If $\vec{D}$ is a point on the circle other than the arc ABC, find $\angle$ADC.
\label{chapters/9/10/5/1}
\\
\solution
\iffalse
\documentclass[12pt]{article}
\usepackage{graphicx}
\usepackage{amsmath}
\usepackage{mathtools}
\usepackage{gensymb}

\newcommand{\mydet}[1]{\ensuremath{\begin{vmatrix}#1\end{vmatrix}}}
\providecommand{\brak}[1]{\ensuremath{\left(#1\right)}}
\providecommand{\norm}[1]{\left\lVert#1\right\rVert}
\newcommand{\solution}{\noindent \textbf{Solution: }}
\newcommand{\myvec}[1]{\ensuremath{\begin{pmatrix}#1\end{pmatrix}}}
\let\vec\mathbf
\def\inputGnumericTable{}
\usepackage{color}                                            %%
    \usepackage{array}                                            %%
    \usepackage{longtable}                                        %%
    \usepackage{calc}                                             %%
    \usepackage{multirow}                                         %%
    \usepackage{hhline}                                           %%
    \usepackage{ifthen}
\usepackage{array}
\usepackage{amsmath}   % for having text in math mode
\usepackage{listings}
\lstset{
language=tex,
frame=single, 
breaklines=true
}
\begin{document}
\begin{center}
\textbf\large{CLASS-9\\CHAPTER-10 \\ CIRCLES}

\end{center}
\section*{Excercise 10.5}

Q1. section*{\large Solution}
\fi
\begin{figure}[h!]
\centering
\includegraphics[width=\columnwidth]{chapters/9/10/5/1/figs/circle1.pdf}
\caption{}
\label{fig:chapters/9/10/5/1/Fig1}
\end{figure}
%
\begin{table}[h!]
	\centering
	%\subimport{../chapters/9/10/5/1/tables/}{table1.tex}
     \begin{tabular}{|c|c|c|}
  \hline
  \textbf{Symbol}&\textbf{Value}&\textbf{Description}\\
  \hline
  $a$ & 8 & $BC$\\
  \hline
	$\angle{B}$ & 45$\degree{}$ & $\angle{B}$ in $\triangle$$ABC$ \\
  \hline
	$k$ & 3.5 & $AB-AC$ i.e $c-b$ \\
  \hline 
	$\vec{e_2}$ & $\myvec{
			0\\
			1\\
			}$ & Basis vector\\
 \hline			
\end{tabular}

%	\caption{}
	\label{table:chapters/9/10/5/1/table1}
	\end{table}
The input parameters are available in Table
	\ref{table:chapters/9/10/5/1/table1} yielding
\begin{align}
	\vec{C} =\vec{e}_1= \myvec{1\\0},\,
	\vec{A} = \myvec{\cos(\alpha+\beta)\\\sin(\alpha+\beta)},\,
	\vec{D} = \myvec{\cos\gamma\\\sin\gamma}.
\end{align}
Since
\begin{align}
	 \vec{A-D}& = \myvec{\cos(\alpha+\beta) - \cos\gamma\\\sin(\alpha+\beta) - \sin\gamma},
	 \vec{C-D} &= \myvec{1 - \cos\gamma\\-\sin\gamma},
	 \norm{\vec{A-D}}\norm{\vec{C-D}}& = 4 \sin\frac{\alpha+\gamma}2\sin\frac{\beta+\gamma}2,
	 \\
	\cos(\angle ADC) &= \frac{\vec{(A-D)^\top(C-D)}}{\norm{\vec{A-D}}\norm{\vec{C-D}}},
	\label{eq:2}
	\\
	&= 4\sin\frac{\alpha+\gamma}2\sin\frac{\beta+\gamma}2\cos\frac{\alpha+\beta}2
 = \cos\frac{\alpha+\beta}{2}
	\label{eq:7}
\end{align}
Substituting $\alpha$ and $\beta$ in \eqref{eq:7}
\begin{align}
\angle ADC = \frac{\alpha+\beta}{2}=\frac{(30\degree + 60\degree )}{2}=45\degree
\end{align}
See Fig. 
\ref{fig:chapters/9/10/5/1/Fig1}.



\item A chord of a circle is equal to the radius of the circle. Find the angle subtended by the chord at a point on the minor arc and also at a point on the major arc.
\label{chapters/9/10/5/2}
\\
\solution

\iffalse
\documentclass[12pt]{article}
\usepackage{graphicx}
\usepackage{amsmath}
\usepackage{mathtools}
\usepackage{gensymb}
\usepackage[latin1]{inputenc}
\usepackage{fullpage}
\usepackage{color}
\usepackage{array}
\usepackage{longtable}
\usepackage{calc}
\usepackage{multirow}
\usepackage{hhline}
\usepackage{ifthen}
\usepackage{booktabs}
\usepackage{graphicx}
\def\inputGnumericTable{}

\newcommand{\mydet}[1]{\ensuremath{\begin{vmatrix}#1\end{vmatrix}}}
\providecommand{\brak}[1]{\ensuremath{\left(#1\right)}}
\providecommand{\norm}[1]{\left\lVert#1\right\rVert}
\newcommand{\solution}{\noindent \textbf{Solution: }}
\newcommand{\myvec}[1]{\ensuremath{\begin{pmatrix}#1\end{pmatrix}}}
\let\vec\mathbf

\begin{document}
\begin{center}
\textbf\large{CHAPTER-9 \\ CIRCLES}

\end{center}
\begin{enumerate}
\section{EXERCISE-10.5}
\section{SOLUTION}
\fi
The input parameters are listed in Table
\ref{tab:chapters/9/10/5/2/}.	
\begin{table}[h!]
	\begin{tabular}{|c|c|p{5cm}|}
\hline
\textbf{Symbol} & \textbf{Value} & \textbf{Description} \\
\hline
$\theta$ & $30\degree$ & $\angle{BAP} = \angle{BAQ}$ \\
\hline
$a$ & $9$ & $AB$ \\
\hline
$c$ & $8$ & $AQ$ \\
\hline
$\vec{e}_1$ & $\myvec{1\\0}$ & Basis vector \\
\hline
\end{tabular}

\caption{}
\label{tab:chapters/9/10/5/2/}	
\end{table}
Take three points Q,R and P on a unit circle  at angles $\theta,\alpha,\text{ and }\beta$. Then
\begin{align}
	\vec{Q} &= \myvec{\cos\theta\\ \sin\theta},\,
	\vec{R} = \myvec{\cos\alpha\\ \sin\alpha},\,
	\vec{S} = \myvec{\cos\beta\\ \sin\beta}
	\\
	\cos\angle QRP&= \frac{\brak{\vec{Q}-\vec{R}}^{\top}\brak{\vec{P}-\vec{R}}}{\norm{\vec{Q}-\vec{R}}\norm{\vec{P}-\vec{R}}}\label{eq:chapters/9/10/5/2/2}
\end{align}
where
\begin{align}
\brak{\vec{Q}-\vec{R}}^{\top}\brak{\vec{P}-\vec{R}}&= \brak{\cos\alpha-\cos\theta}\cos\alpha+\brak{\sin\theta-\sin\alpha}\label{eq:chapters/9/10/5/2/6}
\end{align}
and 
\begin{align}
\norm{\vec{Q}-\vec{R}}^2\norm{\vec{P}-\vec{R}}^2 =\brak{2-2\cos\theta\cos\alpha-2\sin\theta\sin\alpha}\brak{2-\cos\alpha}\label{eq:chapters/9/10/5/2/8}
\end{align}
Substituing \eqref{eq:chapters/9/10/5/2/6} and \eqref{eq:chapters/9/10/5/2/8} in \eqref{eq:chapters/9/10/5/2/2},
\begin{align}
\cos\angle QRP \implies \angle QRP&=62\degree
\end{align}
Similarly, 
\begin{align}
\brak{\vec{Q}-\vec{S}}^{\top}\brak{\vec{P}-\vec{S}}=\brak{\cos\beta-\cos\theta}\cos\beta+\brak{\sin\theta-\sin\beta}\label{eq:chapters/9/10/5/2/16}
\end{align}
and
\begin{align}
\norm{\vec{Q}-\vec{S}}^2\norm{\vec{P}-\vec{S}}^2 =\brak{2-2\cos\theta\cos\beta-2\sin\theta\sin\beta}\brak{2-\cos\beta}\label{eq:chapters/9/10/5/2/18}
\end{align}
Substituing \eqref{eq:chapters/9/10/5/2/16} and \eqref{eq:chapters/9/10/5/2/18} in \eqref{eq:chapters/9/10/5/2/18},
\begin{align}
\cos\angle QSP&=\frac{1.048}{1.098}\\
\implies \angle QSP&=17\degree
\end{align}
See Fig. 
		\ref{fig:chapters/9/10/5/2/Figure}.
\begin{figure}[h]
\centering
\includegraphics[width=\columnwidth]{chapters/9/10/5/2/figs/circle.png}
\caption{}
		\label{fig:chapters/9/10/5/2/Figure}
\end{figure}


\item  $\angle{PQR} = 100\degree$, where $\vec{P}, \vec{Q}$ and $\vec{R}$ are points on a circle with centre $\vec{O}$. Find $\angle{OPR}$
\label{chapters/9/10/5/3}
\\
\solution
%\iffalse
\documentclass[12pt]{article}
\usepackage{graphicx}
\usepackage{amsmath}
\usepackage{mathtools}
\usepackage{gensymb}

\newcommand{\mydet}[1]{\ensuremath{\begin{vmatrix}#1\end{vmatrix}}}
\providecommand{\brak}[1]{\ensuremath{\left(#1\right)}}
\providecommand{\norm}[1]{\left\lVert#1\right\rVert}
\newcommand{\solution}{\noindent \textbf{Solution: }}
\newcommand{\myvec}[1]{\ensuremath{\begin{pmatrix}#1\end{pmatrix}}}
\let\vec\mathbf

\begin{document}
\begin{center}
\textbf\large{CHAPTER-11 \\ CIRCLES}

\end{center}
\section*{Excercise 11.1}

Q4.Find the equation of the circle with centre $(1,1)$ and radius $\sqrt{2}$.

\solution
\fi
Given
\begin{align}
	\vec{c} &= \myvec{1\\1} \text{ and } r = \sqrt{2},
	\\
	\vec{u}&=\vec{-c}
	 = \myvec{-1\\-1}\\
	 \\
	f &= \norm{\vec{u}}^2 - r^2
	  =0	
\end{align}
Thus, the equation of circle is 
\begin{align}
	\norm{\vec{x}}^2 -2\myvec{1&1}\vec{x} = 0       		       
\end{align}	
See Fig. 
\ref{fig:chapters/11/11/1/4/Fig1}.
\begin{figure}[!h]
	\begin{center} 
	  \includegraphics[width=\columnwidth]{chapters/11/11/1/4/figs/circ.png}
	\end{center}
\caption{}
\label{fig:chapters/11/11/1/4/Fig1}
\end{figure}

\item If diagonals of a cyclic quadrilateral are diameters of the circle through the vertices of quadrilateral,prove that it is a rectangle.\\
\label{chapters/9/10/5/7}
\solution
\iffalse
\documentclass[10pt]{article}
\usepackage{graphicx}
\def\inputGnumericTable{}
\usepackage[latin1]{inputenc}
\usepackage{fullpage}
\usepackage{color}
\usepackage{array}
\usepackage{longtable}
\usepackage{calc}
\usepackage{multirow}
\usepackage{hhline}
\usepackage{ifthen}
\usepackage{amsmath}
\usepackage[none]{hyphenat}
\usepackage{listings}
\usepackage[english]{babel}
\usepackage{siunitx}
\usepackage{caption}
\usepackage{booktabs}
\usepackage{array}
\usepackage{extarrows}
\usepackage{enumerate}
\usepackage{enumitem}
\usepackage{amsmath}
\usepackage{commath}
\usepackage{gensymb}
\usepackage{amssymb}
\usepackage{multicol}
%\usepackage[utf8]{inputenc}
\lstset{
 frame=single,
 breaklines=true
}
\usepackage{hyperref}
\usepackage[margin=0.65in]{geometry}	 
%\usepackage{exsheets}% also loads the `tasks' package
\usepackage{atbegshi}
\AtBeginDocument{\AtBeginShipoutNext{\AtBeginShipoutDiscard}}

%new macro definitions
\renewcommand{\labelenumi}{(\roman{enumi})}
\newcommand{\mydet}[1]{\ensuremath{\begin{vmatrix}#1\end{vmatrix}}}
\providecommand{\brak}[1]{\ensuremath{\left(#1\right)}}
\newcommand{\solution}{\noindent \textbf{Solution: }}
\newcommand{\myvec}[1]{\ensuremath{\begin{pmatrix}#1\end{pmatrix}}}
\newenvironment{amatrix}[1]{%
	\left(\begin{array}{@{}*{#1}{c}|c@{}}
}{%
	\end{array}\right)
}

\newcommand{\myaugvec}[2]{\ensuremath{\begin{amatrix}{#1}#2\end{amatrix}}}
\providecommand{\norm}[1]{\left\1Vert#1\right\rVert}
\let\vec\mathbf{}


%\SetEnumitemKey{twocol}{
% before=\raggedcolumns\begin{multicols}{2},
% after=\end{multicols}}
%\SetEnumitemKey{fourcol}{
% before=\raggedcolumns\begin{multicols}{4},
% after=\end{multicols}} 


\begin{document}
\begin{center}
\title{\textbf{CIRCLES}}
\date{\vspace{-5ex}}
\maketitle
\end{center}
\section*{9$^{th}$Math - Chapter 10}
This is Problem-7 from Exercise 10.5\\\\
\fi
\begin{figure}[!ht]
	\begin{center}
		\includegraphics[width=\columnwidth]{./chapters/9/10/5/7/figs/fig.pdf}
	\end{center}
\caption{}
\label{fig:chapters/9/10/5/7/1}
\end{figure}
The input parameters for construction
are available in Table
	\ref{tab:chapters/9/10/5/7/1}.
\begin{table}[ht!]
	\centering
	%\subimport{../chapters/9/10/5/7/tables/}{table.tex}
     \begin{tabular}{|c|c|p{5cm}|}
\hline
\textbf{Symbol} & \textbf{Value} & \textbf{Description} \\
\hline
$\theta$ & $30\degree$ & $\angle{BAP} = \angle{BAQ}$ \\
\hline
$a$ & $9$ & $AB$ \\
\hline
$c$ & $8$ & $AQ$ \\
\hline
$\vec{e}_1$ & $\myvec{1\\0}$ & Basis vector \\
\hline
\end{tabular}

%	\caption{}
	\label{tab:chapters/9/10/5/7/1}
\end{table}
From the given information,
\begin{align}
	\vec{A}&=r\myvec{\cos0\\ \sin0}=\myvec{2\\0}\\
	\vec{B}&=r\myvec{\cos\frac{\pi}{3}\\ \sin\frac{\pi}{3}}=\myvec{1\\\sqrt{3}}\\
	\vec{C}&=2\vec{O}-\vec{A}=\myvec{-2\\0}\\
	\vec{D}&=2\vec{O}-\vec{B}=\myvec{-1\\-\sqrt{3}}
\end{align}
Consider a circle of radius 2 units. Let $AC$ and $DB$ be diameters of circle which are diagonals of cyclic quadrilateral.
Then, from the above equations,
\begin{align}
	\vec{A}-\vec{C} &= \vec{D}-\vec{B}
\end{align}
\begin{enumerate}
\item \label{itm:chapters/9/10/5/7/1} $AB$ and $DC$ are parellel to each other
\begin{align}
	\vec{A}-\vec{B} &= \myvec{2\\0} - \myvec{1\\\sqrt{3}}\\
	&=\myvec{1\\-\sqrt{3}}\\
	\vec{D}-\vec{C} &= \myvec{-1\\-\sqrt{3}} - \myvec{-2\\0}\\
	&=\myvec{1\\-\sqrt{3}}
\end{align}
	Thus,  $ABCD$ is parallelogram.
\item \label{itm:chapters/9/10/5/7/2} Let's check the angle between adjacent sides of this quadrilateral,$AB$ and $BC$
\begin{align}
	\brak{\vec{A}-\vec{B}}^{\top}\brak{\vec{B}-\vec{C}} &=\myvec{1 & -\sqrt{3}}\myvec{3\\\sqrt{3}}\\	
	&= 0\\
	\implies \angle ABC &= 90\degree
\end{align}
from \ref{itm:chapters/9/10/5/7/1} and \ref{itm:chapters/9/10/5/7/2} , Hence the quadrilateral $ABCD$ is rectangle.
\end{enumerate}
See Fig. 
\ref{fig:chapters/9/10/5/7/1}.

\item If circles are drawn taking two sides of a triangle as diameters, prove that the point of intersection of these circles lie on the third side.
\label{chapters/9/10/5/10}
\\
\solution
\iffalse
\documentclass[journal,12pt,twocolumn]{IEEEtran}
\usepackage{setspace}
\usepackage{gensymb}
\singlespacing
\usepackage[cmex10]{amsmath}
\usepackage{amsthm}
\usepackage{mathrsfs}
\usepackage{txfonts}
\usepackage{stfloats}
\usepackage{bm}
\usepackage{cite}
\usepackage{cases}
\usepackage{subfig}
\usepackage{longtable}
\usepackage{multirow}
\usepackage{enumitem}
\usepackage{mathtools}
\usepackage{steinmetz}
\usepackage{tikz}
\usepackage{circuitikz}
\usepackage{verbatim}
\usepackage{tfrupee}
\usepackage[breaklinks=true]{hyperref}
\usepackage{tkz-euclide}
\usetikzlibrary{calc,math}
\usepackage{listings}
    \usepackage{color}                                            %%
    \usepackage{array}                                            %%
    \usepackage{longtable}                                        %%
    \usepackage{calc}                                             %%
    \usepackage{multirow}                                         %%
    \usepackage{hhline}                                           %%
    \usepackage{ifthen}                                           %%
  %optionally (for landscape tables embedded in another document): %%
    \usepackage{lscape}     
\usepackage{multicol}
\usepackage{chngcntr}
\DeclareMathOperator*{\Res}{Res}
\renewcommand\thesection{\arabic{section}}
\renewcommand\thesubsection{\thesection.\arabic{subsection}}
\renewcommand\thesubsubsection{\thesubsection.\arabic{subsubsection}}

\renewcommand\thesectiondis{\arabic{section}}
\renewcommand\thesubsectiondis{\thesectiondis.\arabic{subsection}}
\renewcommand\thesubsubsectiondis{\thesubsectiondis.\arabic{subsubsection}}

% correct bad hyphenation here
\hyphenation{op-tical net-works semi-conduc-tor}
\def\inputGnumericTable{}                                 %%

\lstset{
frame=single, 
breaklines=true,
columns=fullflexible
}

\begin{document}


\newtheorem{theorem}{Theorem}[section]
\newtheorem{problem}{Problem}
\newtheorem{proposition}{Proposition}[section]
\newtheorem{lemma}{Lemma}[section]
\newtheorem{corollary}[theorem]{Corollary}
\newtheorem{example}{Example}[section]
\newtheorem{definition}[problem]{Definition}
\newcommand{\BEQA}{\begin{eqnarray}}
\newcommand{\EEQA}{\end{eqnarray}}
\newcommand{\define}{\stackrel{\triangle}{=}}

\bibliographystyle{IEEEtran}
\providecommand{\mbf}{\mathbf}
\providecommand{\pr}[1]{\ensuremath{\Pr\left(#1\right)}}
\providecommand{\qfunc}[1]{\ensuremath{Q\left(#1\right)}}
\providecommand{\sbrak}[1]{\ensuremath{{}\left[#1\right]}}
\providecommand{\lsbrak}[1]{\ensuremath{{}\left[#1\right.}}
\providecommand{\rsbrak}[1]{\ensuremath{{}\left.#1\right]}}
\providecommand{\brak}[1]{\ensuremath{\left(#1\right)}}
\providecommand{\lbrak}[1]{\ensuremath{\left(#1\right.}}
\providecommand{\rbrak}[1]{\ensuremath{\left.#1\right)}}
\providecommand{\cbrak}[1]{\ensuremath{\left\{#1\right\}}}
\providecommand{\lcbrak}[1]{\ensuremath{\left\{#1\right.}}
\providecommand{\rcbrak}[1]{\ensuremath{\left.#1\right\}}}
\theoremstyle{remark}
\newtheorem{rem}{Remark}
\newcommand{\sgn}{\mathop{\mathrm{sgn}}}
\providecommand{\abs}[1]{\left\vert#1\right\vert}
\providecommand{\res}[1]{\Res\displaylimits_{#1}} 
\providecommand{\norm}[1]{\left\lVert#1\right\rVert}
\providecommand{\mtx}[1]{\mathbf{#1}}
\providecommand{\mean}[1]{E\left[ #1 \right]}
\providecommand{\fourier}{\overset{\mathcal{F}}{ \rightleftharpoons}}
\providecommand{\system}{\overset{\mathcal{H}}{ \longleftrightarrow}}
\newcommand{\solution}{\noindent \textbf{Solution: }}
\newcommand{\cosec}{\,\text{cosec}\,}
\providecommand{\dec}[2]{\ensuremath{\overset{#1}{\underset{#2}{\gtrless}}}}
\newcommand{\myvec}[1]{\ensuremath{\begin{pmatrix}#1\end{pmatrix}}}
\newcommand{\mydet}[1]{\ensuremath{\begin{vmatrix}#1\end{vmatrix}}}
\numberwithin{equation}{subsection}
\makeatletter
\@addtoreset{figure}{problem}
\makeatother

\let\StandardTheFigure\thefigure
\let\vec\mathbf
\renewcommand{\thefigure}{\theproblem}



\def\putbox#1#2#3{\makebox[0in][l]{\makebox[#1][l]{}\raisebox{\baselineskip}[0in][0in]{\raisebox{#2}[0in][0in]{#3}}}}
     \def\rightbox#1{\makebox[0in][r]{#1}}
     \def\centbox#1{\makebox[0in]{#1}}
     \def\topbox#1{\raisebox{-\baselineskip}[0in][0in]{#1}}
     \def\midbox#1{\raisebox{-0.5\baselineskip}[0in][0in]{#1}}

\vspace{3cm}


\title{Assignment 1}
\author{Jaswanth Chowdary Madala}





% make the title area
\maketitle

\newpage

%\tableofcontents

\bigskip

\renewcommand{\thefigure}{\theenumi}
\renewcommand{\thetable}{\theenumi}


\begin{enumerate}

\textbf{Solution:}
\fi
The input parameters are available in Table 
\ref{tab:chapters/9/10/5/10/1}.
\begin{table}[h]
\centering
%%%%%%%%%%%%%%%%%%%%%%%%%%%%%%%%%%%%%%%%%%%%%%%%%%%%%%%%%%%%%%%%%%%%%%
%%                                                                  %%
%%  This is a LaTeX2e table fragment exported from Gnumeric.        %%
%%                                                                  %%
%%%%%%%%%%%%%%%%%%%%%%%%%%%%%%%%%%%%%%%%%%%%%%%%%%%%%%%%%%%%%%%%%%%%%%
\begin{center}
\begin{tabular}{|c|c|c|}
\hline
\textbf{Parameter}	&\textbf{Description}& \textbf{Value}\\ \hline
$\vec{A}	$ & vertex of the triangle &	$\myvec{0\\4}$ 	\\ \hline
$\vec{B}	$ & vertex of the triangle &	$\myvec{0\\-4}$	\\ \hline
$\vec{C}$ &vertex of the triangle  &	$\myvec{6\\6}$\\ \hline
\end{tabular}
\end{center}
\caption{}
\label{tab:chapters/9/10/5/10/1}
\end{table}
The equation of circle taking $AB$ as diameter is given by,
\begin{align}
\norm{\vec{x}}^2 + 2\vec{u_1}^\top\vec{x} + f_1 &= 0 \\
\implies 
\norm{\vec{x}}^2 -16 &= 0
\label{eq:chapters/9/10/5/10/1}
\end{align}
where
\begin{align}
\vec{u_1} = -\brak{\frac{\vec{A}+\vec{B}}{2}}
&= \myvec{0\\0}\\
r_1 = \frac{\norm{\vec{A}-\vec{B}}}{2}
&= 4\\
f_1 = \norm{\vec{u_1}}^2 - r_1^2
&= -4
\end{align}
The equation of circle taking $AC$ as diameter is given by,
\begin{align}
\norm{\vec{x}}^2 + 2\vec{u_2}^\top\vec{x} + f_2 &= 0 \\
\implies \norm{\vec{x}}^2 -2\myvec{3&5}\vec{x}+24 &= 0
\label{eq:chapters/9/10/5/10/2}
\end{align}
where
\begin{align}
\vec{u_2} = -\brak{\frac{\vec{A}+\vec{C}}{2}}
&= -\myvec{3\\5}\\
r_2 = \frac{\norm{\vec{A}-\vec{C}}}{2}
&= \sqrt{10}\\
f_2 = \norm{\vec{u_2}}^2 - r_2^2
&= 24
\end{align}
Let the intersection of circles \eqref{eq:chapters/9/10/5/10/1} and \eqref{eq:chapters/9/10/5/10/2} be $\vec{P}$. The equation of the common chord of intersection of two circles, $AP$ is given by,
\begin{align}
2\vec{u_1}^\top\vec{x}-2\vec{u_2}^\top\vec{x}+f_1 - f_2 &= 0\\
\implies
2\myvec{3 & 5}\vec{x}-16-24&= 0\\
\myvec{3&5}\vec{x} &= 20
\label{eq:chapters/9/10/5/10/3}
\end{align}
\eqref{eq:chapters/9/10/5/10/3} can be written in parametric form as,
\begin{align}
	\vec{x} =  \vec{h}+ \mu \vec{m}, \text{ where }
\vec{h} = \myvec{0\\4}, \, \vec{m} = \myvec{-5\\3}
\end{align}
and $\mu$ 
is given by 
\begin{align}
\mu^2\vec{m}^{\top}\vec{V}\vec{m} + 2 \mu\vec{m}^{\top}\brak{\vec{V}\vec{h}+\vec{u}} + \text{g}\brak{\vec{h}} &=0
\label{eq:chapters/9/10/5/10/6}
\end{align}
with
\begin{align}
\text{g}\brak{\vec{h}} &= \vec{h}^{\top}\vec{V}\vec{h}+2\vec{u}^{\top}\vec{h}+f
\label{eq:chapters/9/10/5/10/5}
\end{align}
Substituting
\begin{align}
\vec{V} = \vec{I}, \, \vec{u} = \myvec{0\\0}, \, f = 16
\vec{h} = \myvec{0\\4}, \, \vec{m} = \myvec{-5\\3}
\end{align}
\begin{align}
34\mu^2 + 24 \mu = 0 \implies
\mu = 0, -\frac{12}{17}
\end{align}
where
$\mu = 0$ corresponds to point $\vec{A}$.
Thus, 
\begin{align}
\vec{P} = \myvec{0\\4} -\frac{12}{17} \myvec{-5\\3}
 = \myvec{\frac{60}{17}\\\\\frac{32}{17}}
\end{align}
The direction vector of $BC$ is given by,
\begin{align}
\vec{m} = \vec{C}-\vec{B}
= \myvec{3\\5}
\implies \vec{n} = \myvec{-5\\3}
\end{align}
yielding the equation 
\begin{align}
\vec{n}^\top\vec{x} &= \vec{n}^\top\vec{B}
\\
\implies \myvec{-5&3}\vec{x} &= \myvec{-5&3}\myvec{0\\-4}
= -12
\label{eq:chapters/9/10/5/10/7}
\end{align}
It is clear that $\vec{P}$ satisfies the equation of ${BC}$ in \eqref{eq:chapters/9/10/5/10/7}. Hence, the point of intersection of the circles drawn by taking two sides of a triangle as diameters lies on the third side.
See Fig. 
\ref{fig:chapters/9/10/5/10/1}.
\begin{figure}[ht]
\centering
\includegraphics[width = \columnwidth]{chapters/9/10/5/10/figs/fig.png}
\caption{Graph}
\label{fig:chapters/9/10/5/10/1}
\end{figure}


    \item Prove that a cyclic paralellogram is a rectangle.
\label{chapters/9/10/5/12}
\\
\solution
\iffalse
\documentclass[12pt]{article}
\usepackage{graphicx}
\usepackage{amsmath}
\usepackage{mathtools}
\usepackage{gensymb}

\newcommand{\mydet}[1]{\ensuremath{\begin{vmatrix}#1\end{vmatrix}}}
\providecommand{\brak}[1]{\ensuremath{\left(#1\right)}}
\providecommand{\norm}[1]{\left\lVert#1\right\rVert}
\newcommand{\solution}{\noindent \textbf{Solution: }}
\newcommand{\myvec}[1]{\ensuremath{\begin{pmatrix}#1\end{pmatrix}}}
\let\vec\mathbf

\begin{document}
\begin{center}
\textbf\large{CHAPTER-11 \\ CIRCLES}

\end{center}
\section*{Excercise 11.1}

Q4.Find the equation of the circle with centre $(1,1)$ and radius $\sqrt{2}$.

\solution
\fi
Given
\begin{align}
	\vec{c} &= \myvec{1\\1} \text{ and } r = \sqrt{2},
	\\
	\vec{u}&=\vec{-c}
	 = \myvec{-1\\-1}\\
	 \\
	f &= \norm{\vec{u}}^2 - r^2
	  =0	
\end{align}
Thus, the equation of circle is 
\begin{align}
	\norm{\vec{x}}^2 -2\myvec{1&1}\vec{x} = 0       		       
\end{align}	
See Fig. 
\ref{fig:chapters/11/11/1/4/Fig1}.
\begin{figure}[!h]
	\begin{center} 
	  \includegraphics[width=\columnwidth]{chapters/11/11/1/4/figs/circ.png}
	\end{center}
\caption{}
\label{fig:chapters/11/11/1/4/Fig1}
\end{figure}

\item Prove that the line of centres of two intersecting circles subtends equal angles at the two points of intersection.
\\
    \solution 
\label{chapters/9/10/6/1}
\documentclass[12pt]{article}
\usepackage{graphicx}
\usepackage{amsmath}
\usepackage{mathtools}
\usepackage{gensymb}
\usepackage{amssymb}
\usepackage{tikz}
\usetikzlibrary{arrows,shapes,automata,petri,positioning,calc}
\usepackage{hyperref}
\usepackage{tikz}
\usetikzlibrary{matrix,calc}
\usepackage[margin=0.5in]{geometry}

\providecommand{\norm}[1]{\left\lVert#1\right\rVert}
\newcommand{\myvec}[1]{\ensuremath{\begin{pmatrix}#1\end{pmatrix}}}
\let\vec\mathbf
%\providecommand $${\norm}[1]{\left\lVert#1\right\rVert}$$
\providecommand{\abs}[1]{\left\vert#1\right\vert}
\let\vec\mathbf

\newcommand{\mydet}[1]{\ensuremath{\begin{vmatrix}#1\end{vmatrix}}}
\providecommand{\brak}[1]{\ensuremath{\left(#1\right)}}
\providecommand{\lbrak}[1]{\ensuremath{\left(#1\right.}}
\providecommand{\rbrak}[1]{\ensuremath{\left.#1\right)}}
\providecommand{\sbrak}[1]{\ensuremath{{}\left[#1\right]}}

\providecommand{\brak}[1]{\ensuremath{\left(#1\right)}}
\providecommand{\norm}[1]{\left\lVert#1\right\rVert}
\newcommand{\solution}{\noindent \textbf{Solution: }}

\let\vec\mathbf
\def\inputGnumericTable{}
\usepackage{color}                                            %%
    \usepackage{array}                                            %%
    \usepackage{longtable}                                        %%
    \usepackage{calc}                                             %%
    \usepackage{multirow}                                         %%
    \usepackage{hhline}                                           %%
    \usepackage{ifthen}
\usepackage{array}
\usepackage{amsmath}   % for having text in math mode
\usepackage{listings}
\lstset{
language=tex,
frame=single, 
breaklines=true
}
\newenvironment{Figure}
  {\par\medskip\noindent\minipage{\linewidth}}
  {\endminipage\par\medskip}
\begin{document}
\begin{center}
\textbf\large{CLASS-9\\CHAPTER-10 \\ CIRCLES}

\end{center}
\section*{Excercise 10.6}

Q1. Prove that the line of centres of two intersecting circles subtends equal angles at the two points of intersection.
\section*{\large Solution}:
\begin{figure}[h!]
\centering
\includegraphics[width=\columnwidth]{figs/circle3.png}
\caption{}
\label{fig:Fig1}
\end{figure}


\section*{\large Construction}:

\begin{table}[h!]
	\small
	\centering
	%\subimport{../tables/}{table1.tex}
     \begin{tabular}{|c|c|c|}
  \hline
  \textbf{Symbol}&\textbf{Value}&\textbf{Description}\\
  \hline
  $a$ & 8 & $BC$\\
  \hline
	$\angle{B}$ & 45$\degree{}$ & $\angle{B}$ in $\triangle$$ABC$ \\
  \hline
	$k$ & 3.5 & $AB-AC$ i.e $c-b$ \\
  \hline 
	$\vec{e_2}$ & $\myvec{
			0\\
			1\\
			}$ & Basis vector\\
 \hline			
\end{tabular}

%	\caption{}
	\label{table:table1}
\end{table}


\section*{\large Verification:}

 The two circle equations are given by:
\begin{align}
\label{eq:1}
	\norm{x}^2-9=0\\
	\norm{x}^2-8\vec{e}_1+12=0
\end{align}
Equation of two conics is given by:
 \begin{align}
 \vec{x}^\top\vec{V}_i\vec{x}+2\vec{u}_i^\top\vec{x}+f_i=0, \quad i=1,2
 \label{eq:3}
 \end{align}
 Represent the two circles in conic form:
 \begin{align}
	\vec{x}^\top\vec{x}-9=0\\
	\vec{x}^\top\vec{x}+2\myvec{-4&0}+12=0
\end{align}
On comparing above two equations with \eqref{eq:3}, we get:
 \begin{align}
	  \vec{V}_1&=\vec{I},\vec{u}_1=\myvec{0\\0},f_1=-9\\
	  \vec{V}_2&=\vec{I},\vec{u}_2=\myvec{-4\\0},f_2=12
\end{align}
The intersection of the given conics is obtained
as
\begin{align}
	\label{eq:8}
\vec{V}_1+\mu\vec{V}_2&= \myvec{
\mu+1 & 0\\
0 & \mu+1
}
\\ \label{eq:9}
\vec{u}_1+\mu\vec{u}_2&= \myvec{
4\\
0
}
\\ \label{eq:10}
f_1+\mu f_2&= -21
\end{align}
This conic is a single straight line if and only if, 
\begin{align}
\mydet{\vec{V}_1 + \mu\vec{V}_2 & \vec{u}_1+\mu \vec{u}_2\\ (\vec{u}_1+\mu \vec{u}_2)^{\top} & f_1 + \mu f_2} &= 0
\label{eq:11}
\end{align}
Substituting equation \eqref{eq:8},\eqref{eq:9} and \eqref{eq:10} in equation \eqref{eq:11}:
\begin{align}
\implies \mydet{1+\mu& 0 & -4\mu\\ 
0 & 1+\mu & 0 \\
-4\mu & 0 & -9+12\mu
} &= 0
\end{align}
Solving the above equation we get,
\begin{align}
    \mu = -1
\end{align}
Thus, the parameters for a straight line can be expressed as
 \begin{align}
 \label{eq:14}
	\vec{V} &= 
\vec{V}_1 + \mu\vec{V}_2
=\myvec{ 0 & 0 \\ 0 & 0},
\\ \label{eq:15}
	\vec{u} &=
\vec{u}_1+\mu \vec{u}_2
	= \myvec{
4\\
0
},
\\ \label{eq:16}
f&=f_1 + \mu f_2=-21
\end{align}
The conic equation is given by:
 \begin{align}
 \vec{x}^\top\vec{V}\vec{x}+2\vec{u}^\top\vec{x}+f=0, 
  \label{eq:17}
 \end{align}
By substituting \eqref{eq:14},\eqref{eq:15} and \eqref{eq:16} in conic equation \eqref{eq:17}, we get staright line between the intersection of two circles:
\begin{align}
\myvec{1&0}\vec{x}&=\frac{21}{8}\\
\vec{x}&=\myvec{\frac{21}{8}\\\lambda}\\
\vec{x}&=\myvec{\frac{21}{8}\\0}+\lambda\myvec{0\\1}\label{eq:20}
\end{align}
		Equation \eqref{eq:20} can be expressed in the form of parametric equation
\begin{align}
	\vec{x}=\vec{q}+\lambda\vec{m}\label{eq:21}
\end{align}
The distance form origin to point $\vec{x}$ is given by
\begin{align}
	\norm{\vec{x}}^2&=d^2\label{eq:22}
\end{align}
		Then substituting \eqref{eq:21} in \eqref{eq:22} yeilds,
\begin{align}
	&\implies\brak{\vec{q}+\lambda\vec{m}}^{\top}\brak{\vec{q}+\lambda\vec{m}}=d^2\\
	&\implies \vec{q}^{\top}\vec{q}+\brak{\lambda\vec{m}}^{\top}\lambda\vec{m}+\vec{q}^{\top}\lambda\vec{m}+\brak{\lambda\vec{m}}^{\top}\vec{q}=d^2\\
	&\implies \norm{\vec{q}}^2+\lambda^2\norm{\vec{m}}^2+2\lambda\vec{q}^{\top}\vec{m}=d^2\\
	&\implies \lambda^2\norm{\vec{m}}^2+2\lambda\vec{q}^{\top}\vec{m}+\norm{\vec{q}}^2=d^2\label{eq:26}
\end{align}
where
\begin{align}
	\vec{q}=\myvec{\frac{21}{8}\\0},\vec{m}=\myvec{0\\1} \text{ and } d=r_1=3
	\label{eq:27}
\end{align}
		substituting the values of \eqref{eq:27} in \eqref{eq:26} gives
\begin{align}
	&\implies\lambda^2(1)+2\lambda\myvec{\frac{21}{8}&0}\myvec{0\\1}+\frac{441}{64}=9\\
	&\implies\lambda^2=\frac{135}{64}\\
	&\implies\lambda_i=\pm\frac{3\sqrt{5}}{8}
\end{align}
The intersecting points $\vec{C}$ and $\vec{D}$ are given by:
\begin{align}
    \vec{C}&=\vec{q}+\lambda_1\vec{m}=\myvec{\frac{21}{8}\\[2pt]-\frac{3\sqrt{5}}{8}}\\
    \vec{D}&=\vec{q}+\lambda_2\vec{m}=\myvec{\frac{21}{8}\\[2pt]\frac{3\sqrt{5}}{8}}
\end{align}
		Check whether the intersection angles $\angle$ADB and $\angle$ACB are equal or not:
\begin{enumerate}
\item Finding $\angle$ADB:
	\begin{align}
		 \vec{A-D} = \myvec{-\frac{21}{8}\\[2pt]-\frac{3\sqrt{5}}{8}},
		\vec{B-D}& = \myvec{\frac{11}{8}\\[2pt]-\frac{3\sqrt{5}}{8}}\\
	 \vec{(A-D)^\top(B-D)}&= -\frac{3}{2}\\
	 \norm{\vec{A-D}}\norm{\vec{C-D}}& = 6\\
		\cos(\angle ADB)& = \frac{\vec{(A-D)^\top(B-D)}}{\norm{\vec{A-D}}\norm{\vec{B-D}}}\\
		\angle ADB&=104\degree
\end{align}
\item Finding $\angle$ACB:
\begin{align}
	\vec{A-C} = \myvec{-\frac{21}{8}\\[2pt]\frac{3\sqrt{5}}{8}},
	 \vec{B-C}& = \myvec{\frac{11}{8}\\[2pt]\frac{3\sqrt{5}}{8}}\\
	 \vec{(A-C)^\top(B-C)}&= -\frac{3}{2}\\
	 \norm{\vec{A-C}}\norm{\vec{B-C}}& = 6\\
	 \cos(\angle ACB) &= \frac{\vec{(A-C)^\top(B-C)}}{\norm{\vec{A-C}}\norm{\vec{B-C}}}\\
	 \angle ACB&=104\degree
\end{align}
\end{enumerate}
Hence, both the intersecting angles are equal to each other, which satisfies the above condition.


\end{document}

\item  $AC$ and $BD$ are chords of a circle which bisect each other. Prove that 
	\begin{enumerate}
		\item  $AC$ and $BD$ are diameters, 
		\item  $ABCD$ is a rectangle.
	\end{enumerate}
    \solution 
\label{chapters/9/10/6/7}
\iffalse
\documentclass[journal,12pt,twocolumn]{IEEEtran}
\def\inputGnumericTable{}
\usepackage{setspace}
\usepackage{gensymb}
\usepackage{xcolor}
\usepackage{caption}
\singlespacing
\usepackage{siunitx}
\usepackage[cmex10]{amsmath}
\usepackage{mathtools}
\usepackage{hyperref}
\usepackage{amsthm}
\usepackage{mathrsfs}
\usepackage{txfonts}
\usepackage{stfloats}
\usepackage{cite}
\usepackage{cases}
\usepackage{subfig}
\usepackage{longtable}
\usepackage{multirow}
\usepackage{enumitem}
\usepackage{mathtools}
\usepackage{listings}
\usepackage{tikz}
\usetikzlibrary{shapes,arrows,positioning}
\usepackage{circuitikz}
       \usepackage[latin1]{inputenc}
       \usepackage{fullpage}
       \usepackage{color}
       \usepackage{array}
       \usepackage{longtable}
       \usepackage{calc}
       \usepackage{multirow}
       \usepackage{hhline}
       \usepackage{ifthen}
	   \usepackage{setspace}
\let\vec\mathbf
\DeclareMathOperator*{\Res}{Res}
\renewcommand\thesection{\arabic{section}}
\renewcommand\thesubsection{\thesection.\arabic{subsection}}
\renewcommand\thesubsubsection{\thesubsection.\arabic{subsubsection}}

\renewcommand\thesectiondis{\arabic{section}}
\renewcommand\thesubsectiondis{\thesectiondis.\arabic{subsection}}
\renewcommand\thesubsubsectiondis{\thesubsectiondis.\arabic{subsubsection}}
\hyphenation{op-tical net-works semi-conduc-tor}

\lstset{
language=Python,
frame=single, 
breaklines=true,
columns=fullflexible
}
\begin{document}
\theoremstyle{definition}
\newtheorem{theorem}{Theorem}[section]
\newtheorem{problem}{Problem}
\newtheorem{proposition}{Proposition}[section]
\newtheorem{lemma}{Lemma}[section]
\newtheorem{corollary}[theorem]{Corollary}
\newtheorem{example}{Example}[section]
\newtheorem{definition}{Definition}[section]
\newcommand{\BEQA}{\begin{eqnarray}}
        \newcommand{\EEQA}{\end{eqnarray}}
\newcommand{\define}{\stackrel{\triangle}{=}}
\newcommand{\myvec}[1]{\ensuremath{\begin{pmatrix}#1\end{pmatrix}}}
\newcommand{\mydet}[1]{\ensuremath{\begin{vmatrix}#1\end{vmatrix}}}

\bibliographystyle{IEEEtran}
\providecommand{\nCr}[2]{\,^{#1}C_{#2}} % nCr
\providecommand{\nPr}[2]{\,^{#1}P_{#2}} % nPr
\providecommand{\mbf}{\mathbf}
\providecommand{\pr}[1]{\ensuremath{\Pr\left(#1\right)}}
\providecommand{\qfunc}[1]{\ensuremath{Q\left(#1\right)}}
\providecommand{\sbrak}[1]{\ensuremath{{}\left[#1\right]}}
\providecommand{\lsbrak}[1]{\ensuremath{{}\left[#1\right.}}
\providecommand{\rsbrak}[1]{\ensuremath{{}\left.#1\right]}}
\providecommand{\brak}[1]{\ensuremath{\left(#1\right)}}
\providecommand{\lbrak}[1]{\ensuremath{\left(#1\right.}}
\providecommand{\rbrak}[1]{\ensuremath{\left.#1\right)}}
\providecommand{\cbrak}[1]{\ensuremath{\left\{#1\right\}}}
\providecommand{\lcbrak}[1]{\ensuremath{\left\{#1\right.}}
\providecommand{\rcbrak}[1]{\ensuremath{\left.#1\right\}}}
\theoremstyle{remark}
\newtheorem{rem}{Remark}
\newcommand{\sgn}{\mathop{\mathrm{sgn}}}
\newcommand{\rect}{\mathop{\mathrm{rect}}}
\newcommand{\sinc}{\mathop{\mathrm{sinc}}}
\providecommand{\abs}[1]{\left\vert#1\right\vert}
\providecommand{\res}[1]{\Res\displaylimits_{#1}}
\providecommand{\norm}[1]{\lVert#1\rVert}
\providecommand{\mtx}[1]{\mathbf{#1}}
\providecommand{\mean}[1]{E\left[ #1 \right]}
\providecommand{\fourier}{\overset{\mathcal{F}}{ \rightleftharpoons}}
\providecommand{\ztrans}{\overset{\mathcal{Z}}{ \rightleftharpoons}}
\providecommand{\system}[1]{\overset{\mathcal{#1}}{ \longleftrightarrow}}
\newcommand{\solution}{\noindent \textbf{Solution: }}
\providecommand{\dec}[2]{\ensuremath{\overset{#1}{\underset{#2}{\gtrless}}}}
\let\StandardTheFigure\thefigure
\def\putbox#1#2#3{\makebox[0in][l]{\makebox[#1][l]{}\raisebox{\baselineskip}[0in][0in]{\raisebox{#2}[0in][0in]{#3}}}}
\def\rightbox#1{\makebox[0in][r]{#1}}
\def\centbox#1{\makebox[0in]{#1}}
\def\topbox#1{\raisebox{-\baselineskip}[0in][0in]{#1}}
\def\midbox#1{\raisebox{-0.5\baselineskip}[0in][0in]{#1}}

\vspace{3cm}
\title{\LaTeX\ 9.10.6.7}
\author{Lokesh Surana}
\maketitle
\section*{Class 9, Chapter, 10, Exercse 6.7}


\solution
\fi
Consider a unit circle with center at origin.
Let $AC$ and $BD$ be the diameters of the circle.
The points on circle that we consider are available in Table \eqref{tab:chapters/9/10/6/7/points}.
%
\begin{table}[ht!]
\begin{tabular}{|c|c|p{5cm}|}
\hline
\textbf{Symbol} & \textbf{Value} & \textbf{Description} \\
\hline
$\theta$ & $30\degree$ & $\angle{BAP} = \angle{BAQ}$ \\
\hline
$a$ & $9$ & $AB$ \\
\hline
$c$ & $8$ & $AQ$ \\
\hline
$\vec{e}_1$ & $\myvec{1\\0}$ & Basis vector \\
\hline
\end{tabular}

\caption{}
\label{tab:chapters/9/10/6/7/points}
\end{table}
%
\begin{enumerate}
    \item $AC$ and $BD$ are diameters of the circle. Let's check if they bisect each other,
    \begin{align}
        \vec{A} + \vec{C} &= \myvec{1 \\ 0} + \myvec{-1 \\ 0}\\
        \label{eq:chapters/9/10/6/7/1} &= \myvec{0 \\ 0} \\
        \vec{B} + \vec{D} &= \myvec{1 \\ 0} + \myvec{-1 \\ 0}\\
        \label{eq:chapters/9/10/6/7/2} &= \myvec{0 \\ 0}
    \end{align}
    From equation \eqref{eq:chapters/9/10/6/7/1} and \eqref{eq:chapters/9/10/6/7/2} $AC$ and $BD$ bisect each other.
    Hence, we can say that if two chords bisect each other then they are diameters.
%
    \item Let's check if $ABCD$ is a rectangle.
    The sides of a rectangle are parallel to each other. Let's check if $AB$ and $BC$ are parallel to each other.
    \begin{align}
        \vec{A} - \vec{B} &= \myvec{1 \\ 0} - \myvec{0 \\ 1}\\
        \label{eq:chapters/9/10/6/7/4} &= \myvec{1 \\ -1} \\
        \vec{D} - \vec{C} &= \myvec{0\\ -1} - \myvec{-1 \\ 0}\\
        \label{eq:chapters/9/10/6/7/5} &= \myvec{1 \\ -1} 
    \end{align}
    From equation \eqref{eq:chapters/9/10/6/7/4} and \eqref{eq:chapters/9/10/6/7/5}, $AB$ and $DC$ are parallel to each other.
    $\implies ABCD$ is a parallelogram.

    Now let's check if its a rectangle.
    Let's check the angle between adjacent sides of this quadrilateral, i.e. $AB$ and $BC$.
    \begin{align}
        \label{eq:chapters/9/10/6/7/6} \brak{\vec{A} - \vec{B}}^\top \brak{\vec{B} - \vec{C}} &=  \myvec{1 & -1} \myvec{1 \\ 1} \\
        &= 0
    \end{align}
    From equation \eqref{eq:chapters/9/10/6/7/6}, we can say that the angle between $AB$ and $BC$ is $90\degree$.
    Hence, the quadrilateral $ABCD$ is a rectangle.
\end{enumerate}

\begin{figure}[!htb]
    \centering
    \includegraphics[width=\columnwidth]{chapters/9/10/6/7/figs/circle.png}
    \caption{circle}
    \label{fig:chapters/9/10/6/7/circle}
\end{figure}


\item Bisectors of angles $A,B$ and $C$ of a triangle $ABC$ intersect its circumcircle at $D,E$ and $F$ respectively. Prove that the angles of triangle $DEF$ are $90{\degree} - \frac{A}{2}$, $90{\degree}-\frac{B}{2}$ and $90{\degree} - \frac{C}{2}$.
\label{chapters/9/10/6/8}
\\
    \solution 
\iffalse
\documentclass[journal,12pt,twocolumn]{IEEEtran}
%
\usepackage{setspace}
\usepackage{gensymb}
%\doublespacing
\singlespacing

%\usepackage{graphicx}
%\usepackage{amssymb}
%\usepackage{relsize}
\usepackage[cmex10]{amsmath}
%\usepackage{amsthm}
%\interdisplaylinepenalty=2500
%\savesymbol{iint}
%\usepackage{txfonts}
%\restoresymbol{TXF}{iint}
%\usepackage{wasysym}
\usepackage{amsthm}
%\usepackage{iithtlc}
\usepackage{mathrsfs}
\usepackage{txfonts}
\usepackage{stfloats}
\usepackage{bm}
\usepackage{cite}
\usepackage{cases}
\usepackage{subfig}
%\usepackage{xtab}
\usepackage{longtable}
\usepackage{multirow}
%\usepackage{algorithm}
%\usepackage{algpseudocode}
\usepackage{enumitem}
\usepackage{mathtools}
\usepackage{steinmetz}
\usepackage{tikz}
\usepackage{circuitikz}
\usepackage{verbatim}
\usepackage{tfrupee}
\usepackage[breaklinks=true]{hyperref}
%\usepackage{stmaryrd}
\usepackage{tkz-euclide} % loads  TikZ and tkz-base
%\usetkzobj{all}
\usetikzlibrary{calc,math}
\usepackage{listings}
    \usepackage{color}                                            %%
    \usepackage{array}                                            %%
    \usepackage{longtable}                                        %%
    \usepackage{calc}                                             %%
    \usepackage{multirow}                                         %%
    \usepackage{hhline}                                           %%
    \usepackage{ifthen}                                           %%
  %optionally (for landscape tables embedded in another document): %%
    \usepackage{lscape}     
\usepackage{multicol}
\usepackage{chngcntr}
%\usepackage{enumerate}

%\usepackage{wasysym}
%\newcounter{MYtempeqncnt}
\DeclareMathOperator*{\Res}{Res}
%\renewcommand{\baselinestretch}{2}
\renewcommand\thesection{\arabic{section}}
\renewcommand\thesubsection{\thesection.\arabic{subsection}}
\renewcommand\thesubsubsection{\thesubsection.\arabic{subsubsection}}

\renewcommand\thesectiondis{\arabic{section}}
\renewcommand\thesubsectiondis{\thesectiondis.\arabic{subsection}}
\renewcommand\thesubsubsectiondis{\thesubsectiondis.\arabic{subsubsection}}

% correct bad hyphenation here
\hyphenation{op-tical net-works semi-conduc-tor}
\def\inputGnumericTable{}                                 %%

\lstset{
%language=C,
frame=single, 
breaklines=true,
columns=fullflexible
}
%\lstset{
%language=tex,
%frame=single, 
%breaklines=true
%}

\begin{document}
%


\newtheorem{theorem}{Theorem}[section]
\newtheorem{problem}{Problem}
\newtheorem{proposition}{Proposition}[section]
\newtheorem{lemma}{Lemma}[section]
\newtheorem{corollary}[theorem]{Corollary}
\newtheorem{example}{Example}[section]
\newtheorem{definition}[problem]{Definition}
%\newtheorem{thm}{Theorem}[section] 
%\newtheorem{defn}[thm]{Definition}
%\newtheorem{algorithm}{Algorithm}[section]
%\newtheorem{cor}{Corollary}
\newcommand{\BEQA}{\begin{eqnarray}}
\newcommand{\EEQA}{\end{eqnarray}}
\newcommand{\define}{\stackrel{\triangle}{=}}

\bibliographystyle{IEEEtran}
%\bibliographystyle{ieeetr}


\providecommand{\mbf}{\mathbf}
\providecommand{\pr}[1]{\ensuremath{\Pr\left(#1\right)}}
\providecommand{\qfunc}[1]{\ensuremath{Q\left(#1\right)}}
\providecommand{\sbrak}[1]{\ensuremath{{}\left[#1\right]}}
\providecommand{\lsbrak}[1]{\ensuremath{{}\left[#1\right.}}
\providecommand{\rsbrak}[1]{\ensuremath{{}\left.#1\right]}}
\providecommand{\brak}[1]{\ensuremath{\left(#1\right)}}
\providecommand{\lbrak}[1]{\ensuremath{\left(#1\right.}}
\providecommand{\rbrak}[1]{\ensuremath{\left.#1\right)}}
\providecommand{\cbrak}[1]{\ensuremath{\left\{#1\right\}}}
\providecommand{\lcbrak}[1]{\ensuremath{\left\{#1\right.}}
\providecommand{\rcbrak}[1]{\ensuremath{\left.#1\right\}}}
\theoremstyle{remark}
\newtheorem{rem}{Remark}
\newcommand{\sgn}{\mathop{\mathrm{sgn}}}
\providecommand{\abs}[1]{\left\vert#1\right\vert}
\providecommand{\res}[1]{\Res\displaylimits_{#1}} 
\providecommand{\norm}[1]{\left\lVert#1\right\rVert}
%\providecommand{\norm}[1]{\lVert#1\rVert}
\providecommand{\mtx}[1]{\mathbf{#1}}
\providecommand{\mean}[1]{E\left[ #1 \right]}
\providecommand{\fourier}{\overset{\mathcal{F}}{ \rightleftharpoons}}
%\providecommand{\hilbert}{\overset{\mathcal{H}}{ \rightleftharpoons}}
\providecommand{\system}{\overset{\mathcal{H}}{ \longleftrightarrow}}
	%\newcommand{\solution}[2]{\textbf{Solution:}{#1}}
\newcommand{\solution}{\noindent \textbf{Solution: }}
\newcommand{\cosec}{\,\text{cosec}\,}
\providecommand{\dec}[2]{\ensuremath{\overset{#1}{\underset{#2}{\gtrless}}}}
\newcommand{\myvec}[1]{\ensuremath{\begin{pmatrix}#1\end{pmatrix}}}
\newcommand{\mydet}[1]{\ensuremath{\begin{vmatrix}#1\end{vmatrix}}}
%\numberwithin{equation}{section}
\numberwithin{equation}{subsection}
%\numberwithin{problem}{section}
%\numberwithin{definition}{section}
\makeatletter
\@addtoreset{figure}{problem}
\makeatother

\let\StandardTheFigure\thefigure
\let\vec\mathbf
%\renewcommand{\thefigure}{\theproblem.\arabic{figure}}
\renewcommand{\thefigure}{\theproblem}
%\setlist[enumerate,1]{before=\renewcommand\theequation{\theenumi.\arabic{equation}}
%\counterwithin{equation}{enumi}


%\renewcommand{\theequation}{\arabic{subsection}.\arabic{equation}}

\def\putbox#1#2#3{\makebox[0in][l]{\makebox[#1][l]{}\raisebox{\baselineskip}[0in][0in]{\raisebox{#2}[0in][0in]{#3}}}}
     \def\rightbox#1{\makebox[0in][r]{#1}}
     \def\centbox#1{\makebox[0in]{#1}}
     \def\topbox#1{\raisebox{-\baselineskip}[0in][0in]{#1}}
     \def\midbox#1{\raisebox{-0.5\baselineskip}[0in][0in]{#1}}

\vspace{3cm}


\title{Problem: 9.10.6.8}
\author{Nikam Pratik Balasaheb (EE21BTECH11037)}





% make the title area
\maketitle

\newpage

%\tableofcontents

\bigskip

\renewcommand{\thefigure}{\theenumi}
\renewcommand{\thetable}{\theenumi}
%\renewcommand{\theequation}{\theenumi}

\section{Problem}

Bisector of angles A,B and C of a triangle ABC intersect its circumcircle at D,E,and F respectively. Prove that the angles of triangle DEF are $90{\degree} - \frac{A}{2}$, $90{\degree}-\frac{B}{2}$ and $90{\degree} - \frac{C}{2}$.
\section{Solution}
\fi
The input parameters are listed in 
        \tabref{tab:chapters/9/10/6/8/}.
\begin{table}[h!]
\centering
        %%%%%%%%%%%%%%%%%%%%%%%%%%%%%%%%%%%%%%%%%%%%%%%%%%%%%%%%%%%%%%%%%%%%%%
%%                                                                  %%
%%  This is a LaTeX2e table fragment exported from Gnumeric.        %%
%%                                                                  %%
%%%%%%%%%%%%%%%%%%%%%%%%%%%%%%%%%%%%%%%%%%%%%%%%%%%%%%%%%%%%%%%%%%%%%%

\begin{tabular}[]{|c|c|c|}
\hline
Parameter	& Value	& Desription \\ \hline
$a$	& 5 & length of side opposite to Vertex $\vec{A}$\\ \hline
	$c$	& 5 & length of side opposite to vertex $\vec{C}$\\ \hline
	$\theta$		& $60\degree$ & $\angle{ABC}$\\ \hline
\end{tabular}

        \caption{Table}
        \label{tab:chapters/9/10/6/8/}
\end{table}
%
Let the vertices of the triangle ABC be
	\begin{align}
		\vec{B} = \myvec{0\\0},\,
		\vec{C} = \myvec{a\\0},\,
		\vec{A} = c\myvec{\cos{\theta}\\ \sin{\theta}}
	\end{align}
\begin{enumerate}
%
\item Circumcenter: From 
	\eqref{eq:circumcirc-eq}, the circumcentre is given by
		\begin{align}
			\myvec{ a & 0\\ c \cos{\theta} & c \sin{\theta}} \vec{O} &= \myvec{ \frac{a^2}{2} \\ \frac{c^2}{2}}
		\end{align}
and the circumradius is 
	\begin{align}
		R = \norm{\vec{A}-\vec{O}}
	\end{align}
%
\item From 
	\eqref{eq:ang-bisect-dir}, the equation of the angle bisector at $\vec{A}$ is given by 
\begin{align}
\vec{x} = \vec{A}+\mu \vec{m}
\end{align}
where 
\begin{align}
				\vec{m} 
					= \vec{B} + \vec{C} -2\vec{A}
\end{align}
%
Thus, $\vec{D}$ is obtained from
	\eqref{eq:chord-pts} using the parameters of the angle bisector at $\vec{A}$.  Similarly, 
$\vec{E}, \vec{F}$  are found using above method and are available in \tabref{tab:chapters/9/10/6/8/POIs}. 
\begin{table}[h!]
\centering
	%%%%%%%%%%%%%%%%%%%%%%%%%%%%%%%%%%%%%%%%%%%%%%%%%%%%%%%%%%%%%%%%%%%%%%
%%                                                                  %%
%%  This is a LaTeX2e table fragment exported from Gnumeric.        %%
%%                                                                  %%
%%%%%%%%%%%%%%%%%%%%%%%%%%%%%%%%%%%%%%%%%%%%%%%%%%%%%%%%%%%%%%%%%%%%%%

\begin{tabular}[]{|c|c|}
\hline
	$\vec{D}$	 & $\myvec{2.5\\ -1.44}$ \\ \hline
	$\vec{E}$	 & $\myvec{5\\ 2.886} $\\ \hline
	$\vec{F}$	& $\myvec{0\\2.886}$\\ \hline
\end{tabular}

        \caption{Table}		\label{tab:chapters/9/10/6/8/POIs}
\end{table}
%
\item From
  \eqref{eq:dot2d},
	\begin{align}
		\cos\brak{\angle DEF} &= \frac{ \brak{\vec{D}-\vec{E}}^{\top} \brak{ \vec{F}-\vec{E}} } { \norm{\vec{D}-\vec{E}} \norm{\vec{F}-\vec{E}} } = 60 \degree
	\end{align}
%
\begin{figure}[h]
  \centering
   \includegraphics[width=\linewidth,height = \linewidth]{chapters/9/10/6/8/figs/Figure_1.png}
   \caption{Figure}
   \label{fig:chapters/9/10/6/8/}
\end{figure}
%
\end{enumerate}



\item Two congruent circles intersect each other at points $\vec{A}$ and $\vec{B}$. Through $\vec{A}$ any line segment $\vec{PAQ}$ is drawn so that $\vec{P}$, $\vec{Q}$ lie on the two circles. Prove that $BP = BQ$.
\label{chapters/9/10/6/9}
\\
    \solution 
\iffalse
\documentclass[journal,12pt,twocolumn]{IEEEtran}
\usepackage{setspace}
\usepackage{gensymb}
\singlespacing
\usepackage[cmex10]{amsmath}
\usepackage{amsthm}
\usepackage{mathrsfs}
\usepackage{txfonts}
\usepackage{stfloats}
\usepackage{bm}
\usepackage{cite}
\usepackage{cases}
\usepackage{subfig}
\usepackage{longtable}
\usepackage{multirow}
\usepackage{enumitem}
\usepackage{mathtools}
\usepackage{steinmetz}
\usepackage{tikz}
\usepackage{circuitikz}
\usepackage{verbatim}
\usepackage{tfrupee}
\usepackage[breaklinks=true]{hyperref}
\usepackage{tkz-euclide}
\usetikzlibrary{calc,math}
\usepackage{listings}
    \usepackage{color}                                            %%
    \usepackage{array}                                            %%
    \usepackage{longtable}                                        %%
    \usepackage{calc}                                             %%
    \usepackage{multirow}                                         %%
    \usepackage{hhline}                                           %%
    \usepackage{ifthen}                                           %%
  %optionally (for landscape tables embedded in another document): %%
    \usepackage{lscape}     
\usepackage{multicol}
\usepackage{chngcntr}
\DeclareMathOperator*{\Res}{Res}
\renewcommand\thesection{\arabic{section}}
\renewcommand\thesubsection{\thesection.\arabic{subsection}}
\renewcommand\thesubsubsection{\thesubsection.\arabic{subsubsection}}

\renewcommand\thesectiondis{\arabic{section}}
\renewcommand\thesubsectiondis{\thesectiondis.\arabic{subsection}}
\renewcommand\thesubsubsectiondis{\thesubsectiondis.\arabic{subsubsection}}

% correct bad hyphenation here
\hyphenation{op-tical net-works semi-conduc-tor}
\def\inputGnumericTable{}                                 %%

\lstset{
frame=single, 
breaklines=true,
columns=fullflexible
}

\begin{document}


\newtheorem{theorem}{Theorem}[section]
\newtheorem{problem}{Problem}
\newtheorem{proposition}{Proposition}[section]
\newtheorem{lemma}{Lemma}[section]
\newtheorem{corollary}[theorem]{Corollary}
\newtheorem{example}{Example}[section]
\newtheorem{definition}[problem]{Definition}
\newcommand{\BEQA}{\begin{eqnarray}}
\newcommand{\EEQA}{\end{eqnarray}}
\newcommand{\define}{\stackrel{\triangle}{=}}

\bibliographystyle{IEEEtran}
\providecommand{\mbf}{\mathbf}
\providecommand{\pr}[1]{\ensuremath{\Pr\left(#1\right)}}
\providecommand{\qfunc}[1]{\ensuremath{Q\left(#1\right)}}
\providecommand{\sbrak}[1]{\ensuremath{{}\left[#1\right]}}
\providecommand{\lsbrak}[1]{\ensuremath{{}\left[#1\right.}}
\providecommand{\rsbrak}[1]{\ensuremath{{}\left.#1\right]}}
\providecommand{\brak}[1]{\ensuremath{\left(#1\right)}}
\providecommand{\lbrak}[1]{\ensuremath{\left(#1\right.}}
\providecommand{\rbrak}[1]{\ensuremath{\left.#1\right)}}
\providecommand{\cbrak}[1]{\ensuremath{\left\{#1\right\}}}
\providecommand{\lcbrak}[1]{\ensuremath{\left\{#1\right.}}
\providecommand{\rcbrak}[1]{\ensuremath{\left.#1\right\}}}
\theoremstyle{remark}
\newtheorem{rem}{Remark}
\newcommand{\sgn}{\mathop{\mathrm{sgn}}}
\providecommand{\abs}[1]{\left\vert#1\right\vert}
\providecommand{\res}[1]{\Res\displaylimits_{#1}} 
\providecommand{\norm}[1]{\left\lVert#1\right\rVert}
\providecommand{\mtx}[1]{\mathbf{#1}}
\providecommand{\mean}[1]{E\left[ #1 \right]}
\providecommand{\fourier}{\overset{\mathcal{F}}{ \rightleftharpoons}}
\providecommand{\system}{\overset{\mathcal{H}}{ \longleftrightarrow}}
\newcommand{\solution}{\noindent \textbf{Solution: }}
\newcommand{\cosec}{\,\text{cosec}\,}
\providecommand{\dec}[2]{\ensuremath{\overset{#1}{\underset{#2}{\gtrless}}}}
\newcommand{\myvec}[1]{\ensuremath{\begin{pmatrix}#1\end{pmatrix}}}
\newcommand{\mydet}[1]{\ensuremath{\begin{vmatrix}#1\end{vmatrix}}}
\numberwithin{equation}{subsection}
\makeatletter
\@addtoreset{figure}{problem}
\makeatother

\let\StandardTheFigure\thefigure
\let\vec\mathbf
\renewcommand{\thefigure}{\theproblem}



\def\putbox#1#2#3{\makebox[0in][l]{\makebox[#1][l]{}\raisebox{\baselineskip}[0in][0in]{\raisebox{#2}[0in][0in]{#3}}}}
     \def\rightbox#1{\makebox[0in][r]{#1}}
     \def\centbox#1{\makebox[0in]{#1}}
     \def\topbox#1{\raisebox{-\baselineskip}[0in][0in]{#1}}
     \def\midbox#1{\raisebox{-0.5\baselineskip}[0in][0in]{#1}}

\vspace{3cm}


\title{Assignment 1}
\author{Jaswanth Chowdary Madala}





% make the title area
\maketitle

\newpage

%\tableofcontents

\bigskip

\renewcommand{\thefigure}{\theenumi}
\renewcommand{\thetable}{\theenumi}


\begin{enumerate}

\textbf{Solution:}
\fi
The parameters used in the construction are shown in Table \ref{tab:chapters/9/10/6/9/1}.
Let the equations of these circles be given by
\begin{align}
\norm{\vec{x}}^2 + 2\vec{u_1}^\top\vec{x} + f &= 0 
\label{eq:chapters/9/10/6/9/1}\\
\norm{\vec{x}}^2 + 2\vec{u_2}^\top\vec{x} + f &= 0 
\label{eq:chapters/9/10/6/9/2}\\
\vec{u_1} = -\myvec{2\\0}, \, \vec{u_2} &= -\myvec{-2\\0}\\
f = -4. 
\end{align}
The common chord of the circles is given by
\begin{align}
2\vec{u_1}^\top\vec{x}-2\vec{u_2}^\top\vec{x}+f_1 - f_2 &= 0\\
\implies
\myvec{1&0}\vec{x} &= 0
\label{eq:chapters/9/10/6/9/3}
\end{align}
\eqref{eq:chapters/9/10/6/9/3} can be written in parametric form as
\begin{align}
	\vec{h} &= \myvec{0\\0}, \, \vec{m} = \myvec{0\\1}\\
	\vec{x} &= \vec{h}+ \mu \vec{m}
\label{eq:chapters/9/10/6/9/4}
\end{align} 
The parameter $\mu$ for the points of intersection of the above line with the given conic
		%line \eqref{eq:chapters/9/10/6/9/5} with the conic section \eqref{eq:chapters/9/10/6/9/6}
is given by the equation 
\begin{align}
\mu^2\vec{m}^{\top}\vec{V}\vec{m} + 2 \mu\vec{m}^{\top}\brak{\vec{V}\vec{h}+\vec{u}} + \text{g}\brak{\vec{h}} &=0
\label{eq:chapters/9/10/6/9/7}
\end{align}
Substituting numerical values,
\begin{align}
\vec{V} = \vec{I}, \, \vec{u} &= \myvec{-2\\0}, \, f = -4,\,
\vec{h} = \myvec{0\\0}, \, \vec{m} &= \myvec{0\\1}\\
\end{align}
we obtain 
\begin{align}
\mu^2 - 4 =0
\implies \mu = \pm 2
\end{align}
From \eqref{eq:chapters/9/10/6/9/4}, 
\begin{align}
\vec{A} = \myvec{0\\2},\,
\vec{B} = \myvec{0\\-2} 
\end{align} 
Since 
\begin{align}
\vec{m} = \myvec{2\\1}
\label{eq:chapters/9/10/6/9/8}
\end{align} 
The equation of the line passing through the point $\vec{A}$ with the direction vector \eqref{eq:chapters/9/10/6/9/8} in parametric form is given by,
\begin{align}
\vec{x} = \myvec{0\\2} + \alpha\myvec{2\\1}
\label{eq:chapters/9/10/6/9/9}
\end{align}
Substituing 
\begin{align}
\vec{V} = \vec{I}, \, \vec{u} &= \myvec{-2\\0}, \, f = -4
\vec{h} = \myvec{0\\2}, 
\end{align}
the intersection of the line in 
\ref{eq:chapters/9/10/6/9/9}
with one circle is given by 
\begin{align}
5\alpha^2 - 4 \alpha &=0
\implies \alpha &= 0, \frac{4}{5}
\end{align}
Thus,
\begin{align}
\vec{P} &= \myvec{\frac{8}{5}\\\\ \frac{14}{5}}
\end{align}
Similarly, the intersection with the second circle is given by 
\begin{align}
	5\beta^2 + 12 \beta &=0,\,
\implies \beta = 0, -\frac{12}{5}\\
	\text{and } \vec{Q} &= \myvec{-\frac{24}{5}\\\\-\frac{2}{5}}
\end{align}
Thus, 
\begin{align}
	\norm{\vec{B}-\vec{P}} &=  
\norm{\vec{B}-\vec{Q}} =  \frac{8\sqrt{10}}{5}
\\
\implies 
	BP &= BQ.
\end{align}
See Fig.
\ref{fig:chapters/9/10/6/9/1}.
\begin{table}[h]
\centering
\begin{tabular}{|c|c|c|}
  \hline
  \textbf{Symbol}&\textbf{Value}&\textbf{Description}\\
  \hline
  $a$ & 8 & $BC$\\
  \hline
	$\angle{B}$ & 45$\degree{}$ & $\angle{B}$ in $\triangle$$ABC$ \\
  \hline
	$k$ & 3.5 & $AB-AC$ i.e $c-b$ \\
  \hline 
	$\vec{e_2}$ & $\myvec{
			0\\
			1\\
			}$ & Basis vector\\
 \hline			
\end{tabular}

\caption{}
\label{tab:chapters/9/10/6/9/1}
\end{table}
\begin{figure}[ht]
\centering
\includegraphics[width = \columnwidth]{chapters/9/10/6/9/figs/fig.png}
\caption{}
\label{fig:chapters/9/10/6/9/1}
\end{figure}

\item In any $\triangle ABC$, if the angle bisector of $\angle A$ and 
    perpendicular bisector of $BC$ intersect, prove that they intersect on 
    the circumcircle of $\triangle ABC$.
\\
    \solution 
\label{chapters/9/10/6/10}
\iffalse
\documentclass[12pt]{article}
\usepackage{graphicx}
\usepackage{amsmath}
\usepackage{mathtools}
\usepackage{gensymb}

\newcommand{\mydet}[1]{\ensuremath{\begin{vmatrix}#1\end{vmatrix}}}
\providecommand{\brak}[1]{\ensuremath{\left(#1\right)}}
\providecommand{\norm}[1]{\left\lVert#1\right\rVert}
\newcommand{\solution}{\noindent \textbf{Solution: }}
\newcommand{\myvec}[1]{\ensuremath{\begin{pmatrix}#1\end{pmatrix}}}
\let\vec\mathbf

\begin{document}
\begin{center}
\textbf\large{CHAPTER-11 \\ CIRCLES}

\end{center}
\section*{Excercise 11.1}

Q4.Find the equation of the circle with centre $(1,1)$ and radius $\sqrt{2}$.

\solution
\fi
Given
\begin{align}
	\vec{c} &= \myvec{1\\1} \text{ and } r = \sqrt{2},
	\\
	\vec{u}&=\vec{-c}
	 = \myvec{-1\\-1}\\
	 \\
	f &= \norm{\vec{u}}^2 - r^2
	  =0	
\end{align}
Thus, the equation of circle is 
\begin{align}
	\norm{\vec{x}}^2 -2\myvec{1&1}\vec{x} = 0       		       
\end{align}	
See Fig. 
\ref{fig:chapters/11/11/1/4/Fig1}.
\begin{figure}[!h]
	\begin{center} 
	  \includegraphics[width=\columnwidth]{chapters/11/11/1/4/figs/circ.png}
	\end{center}
\caption{}
\label{fig:chapters/11/11/1/4/Fig1}
\end{figure}



\end{enumerate}

\subsection{Exercises}
\begin{enumerate}[label=\thesection.\arabic*,ref=\thesection.\theenumi]
\numberwithin{equation}{enumi}
\numberwithin{figure}{enumi}
\numberwithin{table}{enumi}

\item $AD$ is a diameter of a circle and $AB$ is a chord. If $AD = 34 cm$, $AB = 30 cm$, the distance of $AB$ from the centre of the circle is:
\begin{enumerate}
\item 17cm
\item 15cm
\item 4cm
\item 8cm
\end{enumerate}
\item In Fig. \ref{fig:exemplar/9.10.1/10.3}, if $OA = 5cm, AB = 8cm$ and $OD$ is perpendicular to $AB$, then $CD$ is equal to:
\begin{figure}[H]
\centering
\includegraphics[width=\columnwidth]{exemplar/9.10.1/figs/10.3.jpg}
\caption{}
\label{fig:exemplar/9.10.1/10.3}
\end{figure}
\begin{enumerate}
\item 2cm
\item 3cm
\item 4cm
\item 5cm
\end{enumerate}
\item If $AB = 12 cm, BC = 16 cm$ and $AB$ is perpendicular to $BC$, then the radius of the circle passing through the points $\vec{A},\vec{B}$ and $\vec{C}$ is:
\begin{enumerate}
\item 6cm
\item 8cm
\item 10cm
\item 12cm
\end{enumerate}
\item In Fig. \ref{fig:exemplar/9.10.1/10.4}, if $\angle ABC = 20\degree$ , then $\angle AOC$ is equal to: 
\begin{figure}[H]
\centering
\includegraphics[width=\columnwidth]{exemplar/9.10.1/figs/10.4.jpg}
\caption{}
\label{fig:exemplar/9.10.1/10.4}
\end{figure}
\begin{enumerate}
\item $20\degree$
\item $40\degree$
\item $60\degree$
\item $10\degree$
\end{enumerate}
\item In Fig. \ref{fig:exemplar/9.10.1/10.5}, if $AOB$ is a diameter of the circle and $AC = BC$,then $\angle CAB$ is equal to:
\begin{figure}[H]
\centering
\includegraphics[width=\columnwidth]{exemplar/9.10.1/figs/10.5.jpg}
\caption{}
\label{fig:exemplar/9.10.1/10.5}
\end{figure}
\begin{enumerate}
\item $30\degree$
\item $60\degree$
\item $90\degree$
\item $45\degree$
\end{enumerate}
\item In Fig. \ref{fig:exemplar/9.10.1/10.6}, if $\angle OAB = 40\degree$ , then $\angle ACB$ is equal to:  
\begin{figure}[H]
\centering
\includegraphics[width=\columnwidth]{exemplar/9.10.1/figs/10.6.jpg}
\caption{}
\label{fig:exemplar/9.10.1/10.6}
\end{figure}
\begin{enumerate}
\item $50\degree$
\item $40\degree$
\item $60\degree$
\item $70\degree$
\end{enumerate}
\item In Fig. \ref{fig:exemplar/9.10.1/10.7}, if $\angle DAB = 60\degree , \angle ABD = 50\degree$ , then $\angle ACB$ is equal to:
\begin{figure}[H]
\centering
\includegraphics[width=\columnwidth]{exemplar/9.10.1/figs/10.7.jpg}
\caption{}
\label{fig:exemplar/9.10.1/10.7}
\end{figure}
\begin{enumerate}
\item $60\degree$
\item $50\degree$
\item $70\degree$
\item $80\degree$
\end{enumerate}
\item $ABCD$ is a cyclic quadrilateral such that $AB$ is a diameter of the circle circumscribing it and $\angle ADC = 140\degree$ , then $\angle BAC$ is equal to:
\begin{enumerate}
\item $80\degree$
\item $50\degree$
\item $40\degree$
\item $30\degree$
\end{enumerate}
\item In Fig. \ref{fig:exemplar/9.10.1/10.8}, $BC$ is a diameter of the circle and $\angle BAO = 60\degree$. Then $\angle ADC$ is equal to:
\begin{figure}[H]
\centering
\includegraphics[width=\columnwidth]{exemplar/9.10.1/figs/10.8.jpg}
\caption{}
\label{fig:exemplar/9.10.1/10.8}
\end{figure}
\begin{enumerate}
\item $30\degree$
\item $45\degree$
\item $60\degree$
\item $120\degree$
\end{enumerate}
\item In Fig. \ref{fig:exemplar/9.10.1/10.9}, $\angle AOB = 90\degree$ and $\angle ABC = 30\degree$ , then $\angle CAO$ is equal to:            
\begin{figure}[H]
\centering
\includegraphics[width=\columnwidth]{exemplar/9.10.1/figs/10.9.jpg}
\caption{}
\label{fig:exemplar/9.10.1/10.9}
\end{figure}
\begin{enumerate}
\item $30\degree$
\item $45\degree$
\item $90\degree$
\item $60\degree$
\end{enumerate}
\item Two chords $AB$ and $CD$ of a circle are each at distances $4 cm$ from the centre The $AB=CD$.
\item Two chords $AB$ and $AC$ of a circle with centre $O$ are an the opposite sides of $OA$ Then $\angle 0AB = \angle 0AC$
\item Two congruent circles with centres 0 and 0 intersect at two points $A$ and $B$ Then $\angle AOB= \angle AOB$
\item Through three collinear points a circle can be drawn
\item A circle of radius $3 cm$ can be drawn through two points $A,B$ such that $AB= 6cm$
\item If AOB is a diameter of a circle and $C$ is a point on the circle, then $AC^2+B^2=AB^2.$
\item $ABCD$ is a cyclic quadrilateral such that $\angle A=90\degree ,\angle B=70\degree, \angle C=95\degree \text{ and }\angle D=105\degree$
\item If A,B,C,D are four points such that $\angle BAC=30\degree, \angle BDC=60\degree$, then D is the centre of the circle through A,B  and C.
\item If A,B,C and D are four points such that $\angle BAC =45\degree \text{ and }\angle BDC=45\degree$, then A,B,C,D are concyclic.
\item In fig \ref{fig:exemplar/9.10.2/1} if AOB is a diameter and $\angle ADC=120\degree \text{ then }\angle CAB=30\degree$
\begin{figure}[h!]
 \begin{center} 
	 \includegraphics[width=\columnwidth]{exemplar/9.10.2/figs/image.jpg}
 \end{center}
\caption{}
	\label{fig:exemplar/9.10.2/1}
\end{figure}
\item If two equal chords of a circle intersect prove that the parts of one chord are separately equal to the parts of the other chord.
\item If non-parallel sides of a trapezium are equal. Prove that it is cyclic
\item If $\vec{P},\vec{Q}$ and $\vec{R}$ are the mid-points of the sides $BC$, $CA$ and $AB$ of a triangle and $AD$ is the perpendicular from $\vec{A}$ on $BC$. Prove that $\vec{P},\vec{Q},\vec{R}$ and $\vec{D}$ are concyclic.
\item $ABCD$ is a parallelogram. A circle through $\vec{A}$, $\vec{B}$ is so drawn that it intersects $AD$ at $\vec{P}$ and $BC$ at $\vec{Q}$. Prove that $\vec{P}$, $\vec{Q}$, $\vec{R}$ and $\vec{D}$ are concyclic.
\item Prove that angle bisector of any angle of a triangle and perpendicular bisector of the opposite side if intersect, they will intersent on the circumcircle of the triangle.
\item If two chords $AB$ and $CD$ of a circle AYDZBWCX intersect at right angles see Fig.\ref{fig:exemplar/9.10.4/1}. Prove that
	\begin{align}
		arc\brak{CXA}+arc\brak{DZB}&=arc\brak{AYD}+arc\brak{AYD}+arc\brak{BWC}\\&=semi-circle
	\end{align}
\begin{figure}[h!]                                   \includegraphics[width=\columnwidth]{exemplar/9.10.4/figs/image1.jpg}                            \caption{}                                       \label{fig:exemplar/9.10.4/1}                    \end{figure}
	\item If $ABC$ is an equilateral triangle inscribed in a circle and $\vec{P}$ be any point on the minor arc $BC$ which does not coincide with $\vec{B}$ or $\vec{C}$. Prove that $PA$ is angle bisector of $\angle BPC$.
	\item In Fig.\ref{fig:exemplar/9.10.4/2}, $AB$ and $CD$ are two chords of a circle intersecting each other at point $\vec{E}$. Prove that 
		\begin{align}
\angle AEC=\frac{1}{2} (\text{Angle subtended by arc CXA at centre}\\ + \text{angle subtended by arc DYB at the centre}).
		\end{align}
	\begin{figure}[h!]                                   \includegraphics[width=\columnwidth]{exemplar/9.10.4/figs/image2.jpg}                           \caption{}                                       \label{fig:exemplar/9.10.4/2}                    \end{figure}
	\item If bisectors of opposite angles of a cyclic quadrilateral $ABCD$ intersect the circle, circumscribing it at the points $\vec{P}$ and $\vec{Q}$. Prove that $PQ$ is a diameter of the circle.
\item A circle has radius $\sqrt{442}$ cm it is divided into two segments by a chord of length 2cm. Prove that the angle subtended by the chord at a point in major segment is $45\degree$.
\item Two equal chords $AB$ and $CD$ of a circle when produced intersect at a point $\vec{P}$. Prove that $PB=PD$
\item $AB$ and $AC$ are two chords of a circle of radius r such that $AB=2AC$. If $\vec{P}$ and $\vec{Q}$ are the distances of $AB$ and $AC$ from the centre. Prove that $4q^2=p^2+3r^2$.
\item In Fig.\ref{fig:exemplar/9.10.4/3}, $\vec{O}$ is the centre of the circle, $\angle BCO=30\degree$. Find $x$ and $y$.
	\begin{figure}[h!]                                   \includegraphics[width=\columnwidth]{exemplar/9.10.4/figs/image3.jpg}                            \caption{}                                       \label{fig:exemplar/9.10.4/3}                    \end{figure}
	\item In Fig.\ref{fig:exemplar/9.10.4/4}, $\vec{O}$ is the centre of the circle, $BD=OD$ and $CD \bot AB$. Find $\angle CAB$.
		\begin{figure}[h!]                      \includegraphics[width=\columnwidth]{exemplar/9.10.4/figs/image4.jpg}                              \caption{}                                       \label{fig:exemplar/9.10.4/4}                    \end{figure}
\end{enumerate}

\subsection{Parabola}

  \label{app:parab}
	Using 
\eqref{eq:conic_affine}
%such that 
\eqref{eq:conic_quad_form} can be expressed as

%\item  
%Substituting \eqref{eq:conic_affine} in \eqref{eq:conic_quad_form}
\begin{align}
\brak{\vec{P}\vec{y}+\vec{c}}^{\top}\vec{V}\brak{\vec{P}\vec{y}+\vec{c}}+2\vec{u}^{\top}\brak{\vec{P}\vec{y}+\vec{c}}+ f
	= 0, 
\end{align}
yielding 
\begin{align}
\vec{y}^{\top}\vec{P}^{\top}\vec{V}\vec{P}\vec{y}+2\brak{\vec{V}\vec{c}+\vec{u}}^{\top}\vec{P}\vec{y}
+  \vec{c}^{\top}\vec{V}\vec{c} + 2\vec{u}^{\top}\vec{c} + f= 0
\label{eq:conic_simp_one}
\end{align}
%
From \eqref{eq:conic_simp_one} and \eqref{eq:conic_parmas_eig_def},
\begin{align}
\vec{y}^{\top}\vec{D}\vec{y}+2\brak{\vec{V}\vec{c}+\vec{u}}^{\top}\vec{P}\vec{y}
+  \vec{c}^{\top}\brak{\vec{V}\vec{c} + \vec{u}}+ \vec{u}^{\top}\vec{c} + f= 0
\label{eq:conic_simp}
\end{align}
When $\vec{V}^{-1}$ exists, choosing
\begin{align}
%\begin{split}
\vec{V}\vec{c}+\vec{u} &= \vec{0}, \quad \text{or}, \vec{c} = -\vec{V}^{-1}\vec{u},
\label{eq:conic_parmas_c_def}
\end{align}
%
%%From \eqref{eq:conic_parmas_k_def} and 
%%
and substituting \eqref{eq:conic_parmas_c_def}
in \eqref{eq:conic_simp}
yields \eqref{eq:conic_simp_temp_nonparab}. 
	\subsection{}
%\item  
When $\abs{\vec{V}} = 0, \lambda_1 = 0$ and 
\begin{align}
\vec{V}\vec{p}_1 = 0, 
\vec{V}\vec{p}_2 = \lambda_2\vec{p}_2.
\label{eq:conic_parab_eig_prop} 
\end{align}
where $\vec{p}_1,\vec{p}_2$ are the eigenvectors of $\vec{V}$ such that  \eqref{eq:conic_parmas_eig_def}
%
\begin{align}
\vec{P} = \myvec{\vec{p}_1 & \vec{p}_2},
\label{eq:eig_matrix}
\end{align}
Substituting \eqref{eq:eig_matrix}
in \eqref{eq:conic_simp},
\begin{align}
	\vec{y}^{\top}\vec{D}\vec{y}+2\brak{\vec{c}^{\top}\vec{V}+\vec{u}^{\top}}\myvec{\vec{p}_1 & \vec{p}_2}\vec{y}
	+  \vec{c}^{\top}\brak{\vec{V}\vec{c} + \vec{u}}+ \vec{u}^{\top}\vec{c} + f&= 0
\\
\implies \vec{y}^{\top}\vec{D}\vec{y}
+2\myvec{\brak{\vec{c}^{\top}\vec{V}+\vec{u}^{\top}}\vec{p}_1  \brak{\vec{c}^{\top}\vec{V}+\vec{u}^{\top}}\vec{p}_2}\vec{y}
	+  \vec{c}^{\top}\brak{\vec{V}\vec{c} + \vec{u}}+ \vec{u}^{\top}\vec{c} + f&= 0
\\
\implies \vec{y}^{\top}\vec{D}\vec{y}
+2\myvec{\vec{u}^{\top}\vec{p}_1 & \brak{\lambda_2\vec{c}^{\top}+\vec{u}^{\top}}\vec{p}_2}\vec{y}
	+  \vec{c}^{\top}\brak{\vec{V}\vec{c} + \vec{u}}+ \vec{u}^{\top}\vec{c} + f&= 0
\end{align}
upon substituting from 
 \eqref{eq:conic_parab_eig_prop} yielding
\begin{align}
\lambda_2y_2^2+2\brak{\vec{u}^{\top}\vec{p}_1}y_1+  2y_2\brak{\lambda_2\vec{c}+\vec{u}}^{\top}\vec{p}_2
	+  \vec{c}^{\top}\brak{\vec{V}\vec{c} + \vec{u}}+ \vec{u}^{\top}\vec{c} + f= 0
\label{eq:conic_parab_foc_len_temp} 
\end{align}
%which is the equation of a parabola. 
Thus, \eqref{eq:conic_parab_foc_len_temp} 
can be expressed as \eqref{eq:conic_simp_temp_parab} by choosing
\begin{align}
%\label{eq:eta}
\eta = 2\vec{u}^{\top}\vec{p}_1
\end{align}
%Choosing 
%\begin{align}
%\vec{u} + \lambda_2\vec{c} = 0,
%\vec{c}^{\top}\brak{\vec{V}\vec{c} + \vec{u}}+ \vec{u}^{\top}\vec{c} + f = 0,
%\end{align}
% the above equation becomes
%\begin{align}
%y_2^2= -\frac{2\vec{u}^{\top}\vec{p}_1}{ \lambda_2} \brak{y_1
%+  \frac{\vec{u}^{\top}\vec{V}\vec{u} - 2\lambda_2\vec{u}^{\top}\vec{u} + f\lambda_2^2}{2\vec{u}^{\top}\vec{p}_1\lambda_2^2}}
%\\
%or \eta = 2\vec{u}^{\top}\vec{p}_1
%%\label{eq:conic_simp_parab_new}
%\end{align}
and $\vec{c}$ in \eqref{eq:conic_simp} such that
\begin{align}
\label{eq:conic_parab_one}
2\vec{P}^{\top}\brak{\vec{V}\vec{c}+\vec{u}} &= \eta\myvec{1\\0}
\\
\vec{c}^{\top}\brak{\vec{V}\vec{c} + \vec{u}}+ \vec{u}^{\top}\vec{c} + f&= 0
\label{eq:conic_parab_two}
\end{align}
%we obtain  \eqref{eq:conic_simp_temp_parab}.
$\because
\vec{P}^{\top}\vec{P} = \vec{I}$,
multiplying \eqref{eq:conic_parab_one} by $\vec{P}$ yields
\begin{align}
\label{eq:conic_parab_one_eig}
	\brak{\vec{V}\vec{c}+\vec{u}} &= \frac{\eta}{2}\vec{p}_1,
\end{align}
which, upon substituting in \eqref{eq:conic_parab_two}
results in 
\begin{align}
\frac{\eta}{2}\vec{c}^{\top}\vec{p}_1 + \vec{u}^{\top}\vec{c} + f&= 0
\label{eq:conic_parab_two_eig}
\end{align}
\eqref{eq:conic_parab_one_eig} and \eqref{eq:conic_parab_two_eig} can be clubbed together to obtain \eqref{eq:conic_parab_c}.

\subsection{Exercises}
\begin{enumerate}[label=\thesection.\arabic*,ref=\thesection.\theenumi]
\item If the focus of a parabola is (0,-3) and its directrix is y=3, then its equation is
\begin{enumerate}
\item $x^2=-12y$
\item $x^2=12y$
\item $y^2=-12x$
\item $y^2=12x$
\end{enumerate}
\item If the parabola $y^2=4ax$ passes through the point (3,2), then the length of its latus rectum is
\begin{enumerate}
\item 2$\pm$3
\item 4$\pm$4
\item 1$\pm$3
\item 4
\end{enumerate}
\item If the vertex of the parabola is the point (-3,0) and the directrix is the line x+5=0, then its equation is
\begin{enumerate}
\item $y^2=8(x+3)$
\item $x^2=8(y+3)$
\item $y^2=-8(x+3)$
\item $y^2=8(x+5)$
\end{enumerate}
 \item Find the coordinates of a point on the parabla $y^2=8x$ whose focal distance is 4.
 \item Find the length of the line-segment joining the vertex of the parabola $y^2=4ax$ and a point on the parabola where the line - segment makes an angle 0 to the x-axis.
\item If the points (0,4) and (0,2) are respectively the vertex and focus of a parabola. then find the equation of the parabola
\end{enumerate}
Find the equation of each of the following parabolas
\begin{enumerate}[label=\thesection.\arabic*,ref=\thesection.\theenumi,resume*]
\item  Directrix x=0. focus ot (6,0)
\item  vertex  ot (0,4), focus at (0,2)
\item  Focus at (-1,2), directrix x-2y+3=0
\end{enumerate}
Fill in the Blanks
\begin{enumerate}[label=\thesection.\arabic*,ref=\thesection.\theenumi,resume*]
\item The equation of the parabola having focus at (-1,-2) and the directrix x-2y+3=0 is \makebox[1cm]{\hrulefill}   
\end{enumerate}

\subsection{Ellipse}
\iffalse
\documentclass[journal,12pt,twocolumn]{IEEEtran}
\usepackage{setspace}
\usepackage{gensymb}
\singlespacing
\usepackage[cmex10]{amsmath}
\usepackage{amsthm}
\usepackage{mathrsfs}
\usepackage{txfonts}
\usepackage{stfloats}
\usepackage{bm}
\usepackage{cite}
\usepackage{cases}
\usepackage{subfig}
\usepackage{longtable}
\usepackage{multirow}
\usepackage{enumitem}
\usepackage{mathtools}
\usepackage{steinmetz}
\usepackage{tikz}
\usepackage{circuitikz}
\usepackage{verbatim}
\usepackage{tfrupee}
\usepackage[breaklinks=true]{hyperref}
\usepackage{tkz-euclide}
\usetikzlibrary{calc,math}
\usepackage{listings}
    \usepackage{color}                                            %%
    \usepackage{array}                                            %%
    \usepackage{longtable}                                        %%
    \usepackage{calc}                                             %%
    \usepackage{multirow}                                         %%
    \usepackage{hhline}                                           %%
    \usepackage{ifthen}                                           %%
  %optionally (for landscape tables embedded in another document): %%
    \usepackage{lscape}     
\usepackage{multicol}
\usepackage{chngcntr}
\DeclareMathOperator*{\Res}{Res}
\renewcommand\thesection{\arabic{section}}
\renewcommand\thesubsection{\thesection.\arabic{subsection}}
\renewcommand\thesubsubsection{\thesubsection.\arabic{subsubsection}}

\renewcommand\thesectiondis{\arabic{section}}
\renewcommand\thesubsectiondis{\thesectiondis.\arabic{subsection}}
\renewcommand\thesubsubsectiondis{\thesubsectiondis.\arabic{subsubsection}}

% correct bad hyphenation here
\hyphenation{op-tical net-works semi-conduc-tor}
\def\inputGnumericTable{}                                 %%

\lstset{
frame=single, 
breaklines=true,
columns=fullflexible
}

\begin{document}


\newtheorem{theorem}{Theorem}[section]
\newtheorem{problem}{Problem}
\newtheorem{proposition}{Proposition}[section]
\newtheorem{lemma}{Lemma}[section]
\newtheorem{corollary}[theorem]{Corollary}
\newtheorem{example}{Example}[section]
\newtheorem{definition}[problem]{Definition}
\newcommand{\BEQA}{\begin{eqnarray}}
\newcommand{\EEQA}{\end{eqnarray}}
\newcommand{\define}{\stackrel{\triangle}{=}}

\bibliographystyle{IEEEtran}
\providecommand{\mbf}{\mathbf}
\providecommand{\pr}[1]{\ensuremath{\Pr\left(#1\right)}}
\providecommand{\qfunc}[1]{\ensuremath{Q\left(#1\right)}}
\providecommand{\sbrak}[1]{\ensuremath{{}\left[#1\right]}}
\providecommand{\lsbrak}[1]{\ensuremath{{}\left[#1\right.}}
\providecommand{\rsbrak}[1]{\ensuremath{{}\left.#1\right]}}
\providecommand{\brak}[1]{\ensuremath{\left(#1\right)}}
\providecommand{\lbrak}[1]{\ensuremath{\left(#1\right.}}
\providecommand{\rbrak}[1]{\ensuremath{\left.#1\right)}}
\providecommand{\cbrak}[1]{\ensuremath{\left\{#1\right\}}}
\providecommand{\lcbrak}[1]{\ensuremath{\left\{#1\right.}}
\providecommand{\rcbrak}[1]{\ensuremath{\left.#1\right\}}}
\theoremstyle{remark}
\newtheorem{rem}{Remark}
\newcommand{\sgn}{\mathop{\mathrm{sgn}}}
\providecommand{\abs}[1]{\left\vert#1\right\vert}
\providecommand{\res}[1]{\Res\displaylimits_{#1}} 
\providecommand{\norm}[1]{\left\lVert#1\right\rVert}
\providecommand{\mtx}[1]{\mathbf{#1}}
\providecommand{\mean}[1]{E\left[ #1 \right]}
\providecommand{\fourier}{\overset{\mathcal{F}}{ \rightleftharpoons}}
\providecommand{\system}{\overset{\mathcal{H}}{ \longleftrightarrow}}
\newcommand{\solution}{\noindent \textbf{Solution: }}
\newcommand{\cosec}{\,\text{cosec}\,}
\providecommand{\dec}[2]{\ensuremath{\overset{#1}{\underset{#2}{\gtrless}}}}
\newcommand{\myvec}[1]{\ensuremath{\begin{pmatrix}#1\end{pmatrix}}}
\newcommand{\mydet}[1]{\ensuremath{\begin{vmatrix}#1\end{vmatrix}}}
\numberwithin{equation}{subsection}
\makeatletter
\@addtoreset{figure}{problem}
\makeatother

\let\StandardTheFigure\thefigure
\let\vec\mathbf
\renewcommand{\thefigure}{\theproblem}



\def\putbox#1#2#3{\makebox[0in][l]{\makebox[#1][l]{}\raisebox{\baselineskip}[0in][0in]{\raisebox{#2}[0in][0in]{#3}}}}
     \def\rightbox#1{\makebox[0in][r]{#1}}
     \def\centbox#1{\makebox[0in]{#1}}
     \def\topbox#1{\raisebox{-\baselineskip}[0in][0in]{#1}}
     \def\midbox#1{\raisebox{-0.5\baselineskip}[0in][0in]{#1}}

\vspace{3cm}


\title{Assignment 1}
\author{Jaswanth Chowdary Madala}





% make the title area
\maketitle

\newpage

%\tableofcontents

\bigskip

\renewcommand{\thefigure}{\theenumi}
\renewcommand{\thetable}{\theenumi}


\begin{enumerate}
\item Find the equation of the ellipse that satisfies the conditions - Centre at $\brak{0,0}$, major axis on the y-axis and passes through the points $\brak{3,2}$ and $\brak{1,6}$.
%\begin{figure}[ht]
%\centering
%\includegraphics[width = \columnwidth]{"./chapters/11/11/3/19/figs/fig.png"}
%\caption{Graph}
%\label{fig:chapters/11/11/3/19/1}
%\end{figure}

\textbf{Solution:}
\fi
The input parameters are listed in 
\ref{tab:chapters/11/11/3/19/1}.
The equation of the conic with focus $\vec{F}$, directrix $\vec{n}^\top\vec{x} = c$ and eccentricity $e$ is given by
\begin{align}
\vec{x}^\top\vec{V}\vec{x} + 2\vec{u}^\top\vec{x} + f = 0
\label{eq:chapters/11/11/3/19/1}
\end{align}
where
\begin{align}
\vec{V} &\triangleq \norm{\vec{n}}^2\vec{I} - e^2\vec{n}\vec{n}^\top \label{eq:chapters/11/11/3/19/2} \\
\vec{u} &\triangleq ce^2\vec{n} - \norm{\vec{n}}^2\vec{F} \label{eq:chapters/11/11/3/19/3} \\
f &\triangleq \norm{\vec{n}}^2\norm{\vec{F}}^2 - c^2e^2 \label{eq:chapters/11/11/3/19/4}
\end{align}
Given that the conic is an ellipse with major axis along the $y$-axis,
\begin{align}
\vec{n} = \myvec{0\\1}
\end{align}
Thus,
\begin{align}
\vec{V} = \myvec{1&0\\0&1-e^2} \label{eq:chapters/11/11/3/19/5} \\
\vec{u} = ce^2\myvec{0\\1} - \vec{F} \label{eq:chapters/11/11/3/19/6} \\
f = \norm{\vec{F}}^2 - c^2e^2 \label{eq:chapters/11/11/3/19/7}
\end{align}
The centre of the conic is given by
\begin{align}
\vec{c} = -\vec{V}^{-1}\vec{u}
\label{eq:chapters/11/11/3/19/8}
\end{align}
Since $\vec{c} = \vec{0}$ and $\vec{V}^{-1} \neq \vec{0}$, it follows from \eqref{eq:chapters/11/11/3/19/8} that 
\begin{align}
\vec{u} = \vec{0}
\end{align}
Thus, from \eqref{eq:chapters/11/11/3/19/6},
\begin{align}
\vec{F} = \myvec{0\\ce^2}
\label{eq:chapters/11/11/3/19/9}
\end{align}
and so,
\begin{align}
f = c^2e^2\brak{e^2-1}
%\label{eq:chapters/11/11/3/19/9}
\end{align}
Given that the conic passes through 
\begin{align}
\vec{P} = \myvec{3\\2},
\end{align}
putting $\vec{x} = \vec{P}$ in \eqref{eq:chapters/11/11/3/19/1} we get,
\begin{align}
\myvec{3&2}\myvec{1&0\\0&1-e^2}\myvec{3\\2} + f &= 0 \\
\implies 4e^2 - f = 13 \label{eq:chapters/11/11/3/19/10}
\end{align}
Given that the conic passes through
\begin{align}
\vec{Q} = \myvec{1\\6},
\end{align}
putting $\vec{x} = \vec{Q}$ in \eqref{eq:chapters/11/11/3/19/1}, we get
\begin{align}
\myvec{1&6}\myvec{1&0\\0&1-e^2}\myvec{1\\6} + f &= 0 \\
\implies 36e^2 - f = 37 \label{eq:chapters/11/11/3/19/11}
\end{align}
The equations \eqref{eq:chapters/11/11/3/19/10} and \eqref{eq:chapters/11/11/3/19/11} can be formulated as the following matrix equation
\begin{align}
\myvec{4&-1\\36&-1}\myvec{e^2\\f} = \myvec{13\\37}
\label{eq:chapters/11/11/3/19/12}
\end{align}
The augmented matrix is given by,
\begin{align}
\myvec{4&-1&\vline&13\\36&-1&\vline&37}
\end{align}
yielding
\begin{align}
\xleftrightarrow[]{R_1\leftarrow R_1-R_2} &\myvec{-32&0&\vline&-24\\36&-1&\vline&37} \\
\xleftrightarrow[]{R_1\leftarrow-\frac{R_1}{8}}& \myvec{4&0&\vline&3\\36&-1&\vline&37} \\
\xleftrightarrow[]{R_2\leftarrow R_2-9R_1}
&\myvec{4&0&\vline&3\\0&-1&\vline&10} \\
\xleftrightarrow[R_2\leftarrow -R_2]{R_1\leftarrow \frac{R_1}{4}}
&\myvec{1&0&\vline&\frac{3}{4}\\0&1&\vline&-10}
\end{align}
Thus,
\begin{align}
e^2 = \frac{3}{4},\ f = -10
\end{align}
and the equation of the conic is given by
\begin{align}
\vec{x}^\top\myvec{1&0\\0&\frac{1}{4}}\vec{x} - 10 = 0
\end{align}
See Fig. 
\ref{fig:chapters/11/11/3/19/1}.
\begin{figure}[ht]
\centering
\includegraphics[width = \columnwidth]{chapters/11/11/3/19/figs/fig1.png}
\caption{Graph}
\label{fig:chapters/11/11/3/19/1}
\end{figure}
\begin{table}[h]
\centering
%%%%%%%%%%%%%%%%%%%%%%%%%%%%%%%%%%%%%%%%%%%%%%%%%%%%%%%%%%%%%%%%%%%%%%
%%                                                                  %%
%%  This is a LaTeX2e table fragment exported from Gnumeric.        %%
%%                                                                  %%
%%%%%%%%%%%%%%%%%%%%%%%%%%%%%%%%%%%%%%%%%%%%%%%%%%%%%%%%%%%%%%%%%%%%%%

\begin{center}
\begin{tabular}{|c|c|c|}
\hline
\textbf{Parameter}& \textbf{Description} &\textbf{Value}\\ \hline
$\vec{C}$		 &	center of the ellipse&$\myvec{0\\0}$\\ \hline
$\vec{m}$		 &	Direction vector of major axis &$\myvec{0\\1}$\\ \hline
$\vec{P}$		 &  Point on the ellipse&$\myvec{3\\2}$ \\ \hline
$\vec{Q}$		 &  Point on the ellipse&$\myvec{1\\6}$ \\ \hline
\end{tabular}
\end{center}

\caption{}
\label{tab:chapters/11/11/3/19/1}
\end{table}

\subsection{Exercises}
\iffalse
\documentclass[journal,12pt,twocolumn]{IEEEtran}
\usepackage{setspace}
\usepackage{gensymb}
\singlespacing
\usepackage[cmex10]{amsmath}
\usepackage{amsthm}
\usepackage{mathrsfs}
\usepackage{txfonts}
\usepackage{stfloats}
\usepackage{bm}
\usepackage{cite}
\usepackage{cases}
\usepackage{subfig}
\usepackage{longtable}
\usepackage{multirow}
\usepackage{enumitem}
\usepackage{mathtools}
\usepackage{steinmetz}
\usepackage{tikz}
\usepackage{circuitikz}
\usepackage{verbatim}
\usepackage{tfrupee}
\usepackage[breaklinks=true]{hyperref}
\usepackage{tkz-euclide}
\usetikzlibrary{calc,math}
\usepackage{listings}
    \usepackage{color}                                            %%
    \usepackage{array}                                            %%
    \usepackage{longtable}                                        %%
    \usepackage{calc}                                             %%
    \usepackage{multirow}                                         %%
    \usepackage{hhline}                                           %%
    \usepackage{ifthen}                                           %%
  %optionally (for landscape tables embedded in another document): %%
    \usepackage{lscape}     
\usepackage{multicol}
\usepackage{chngcntr}
\DeclareMathOperator*{\Res}{Res}
\renewcommand\thesection{\arabic{section}}
\renewcommand\thesubsection{\thesection.\arabic{subsection}}
\renewcommand\thesubsubsection{\thesubsection.\arabic{subsubsection}}

\renewcommand\thesectiondis{\arabic{section}}
\renewcommand\thesubsectiondis{\thesectiondis.\arabic{subsection}}
\renewcommand\thesubsubsectiondis{\thesubsectiondis.\arabic{subsubsection}}

% correct bad hyphenation here
\hyphenation{op-tical net-works semi-conduc-tor}
\def\inputGnumericTable{}                                 %%

\lstset{
frame=single, 
breaklines=true,
columns=fullflexible
}

\begin{document}


\newtheorem{theorem}{Theorem}[section]
\newtheorem{problem}{Problem}
\newtheorem{proposition}{Proposition}[section]
\newtheorem{lemma}{Lemma}[section]
\newtheorem{corollary}[theorem]{Corollary}
\newtheorem{example}{Example}[section]
\newtheorem{definition}[problem]{Definition}
\newcommand{\BEQA}{\begin{eqnarray}}
\newcommand{\EEQA}{\end{eqnarray}}
\newcommand{\define}{\stackrel{\triangle}{=}}

\bibliographystyle{IEEEtran}
\providecommand{\mbf}{\mathbf}
\providecommand{\pr}[1]{\ensuremath{\Pr\left(#1\right)}}
\providecommand{\qfunc}[1]{\ensuremath{Q\left(#1\right)}}
\providecommand{\sbrak}[1]{\ensuremath{{}\left[#1\right]}}
\providecommand{\lsbrak}[1]{\ensuremath{{}\left[#1\right.}}
\providecommand{\rsbrak}[1]{\ensuremath{{}\left.#1\right]}}
\providecommand{\brak}[1]{\ensuremath{\left(#1\right)}}
\providecommand{\lbrak}[1]{\ensuremath{\left(#1\right.}}
\providecommand{\rbrak}[1]{\ensuremath{\left.#1\right)}}
\providecommand{\cbrak}[1]{\ensuremath{\left\{#1\right\}}}
\providecommand{\lcbrak}[1]{\ensuremath{\left\{#1\right.}}
\providecommand{\rcbrak}[1]{\ensuremath{\left.#1\right\}}}
\theoremstyle{remark}
\newtheorem{rem}{Remark}
\newcommand{\sgn}{\mathop{\mathrm{sgn}}}
\providecommand{\abs}[1]{\left\vert#1\right\vert}
\providecommand{\res}[1]{\Res\displaylimits_{#1}} 
\providecommand{\norm}[1]{\left\lVert#1\right\rVert}
\providecommand{\mtx}[1]{\mathbf{#1}}
\providecommand{\mean}[1]{E\left[ #1 \right]}
\providecommand{\fourier}{\overset{\mathcal{F}}{ \rightleftharpoons}}
\providecommand{\system}{\overset{\mathcal{H}}{ \longleftrightarrow}}
\newcommand{\solution}{\noindent \textbf{Solution: }}
\newcommand{\cosec}{\,\text{cosec}\,}
\providecommand{\dec}[2]{\ensuremath{\overset{#1}{\underset{#2}{\gtrless}}}}
\newcommand{\myvec}[1]{\ensuremath{\begin{pmatrix}#1\end{pmatrix}}}
\newcommand{\mydet}[1]{\ensuremath{\begin{vmatrix}#1\end{vmatrix}}}
\numberwithin{equation}{subsection}
\makeatletter
\@addtoreset{figure}{problem}
\makeatother

\let\StandardTheFigure\thefigure
\let\vec\mathbf
\renewcommand{\thefigure}{\theproblem}



\def\putbox#1#2#3{\makebox[0in][l]{\makebox[#1][l]{}\raisebox{\baselineskip}[0in][0in]{\raisebox{#2}[0in][0in]{#3}}}}
     \def\rightbox#1{\makebox[0in][r]{#1}}
     \def\centbox#1{\makebox[0in]{#1}}
     \def\topbox#1{\raisebox{-\baselineskip}[0in][0in]{#1}}
     \def\midbox#1{\raisebox{-0.5\baselineskip}[0in][0in]{#1}}

\vspace{3cm}


\title{Assignment 1}
\author{Jaswanth Chowdary Madala}





% make the title area
\maketitle

\newpage

%\tableofcontents

\bigskip

\renewcommand{\thefigure}{\theenumi}
\renewcommand{\thetable}{\theenumi}


\begin{enumerate}
\item Find the equation of the ellipse that satisfies the conditions - Centre at $\brak{0,0}$, major axis on the y-axis and passes through the points $\brak{3,2}$ and $\brak{1,6}$.
%\begin{figure}[ht]
%\centering
%\includegraphics[width = \columnwidth]{"./chapters/11/11/3/19/figs/fig.png"}
%\caption{Graph}
%\label{fig:chapters/11/11/3/19/1}
%\end{figure}

\textbf{Solution:}
\fi
The input parameters are listed in 
\ref{tab:chapters/11/11/3/19/1}.
The equation of the conic with focus $\vec{F}$, directrix $\vec{n}^\top\vec{x} = c$ and eccentricity $e$ is given by
\begin{align}
\vec{x}^\top\vec{V}\vec{x} + 2\vec{u}^\top\vec{x} + f = 0
\label{eq:chapters/11/11/3/19/1}
\end{align}
where
\begin{align}
\vec{V} &\triangleq \norm{\vec{n}}^2\vec{I} - e^2\vec{n}\vec{n}^\top \label{eq:chapters/11/11/3/19/2} \\
\vec{u} &\triangleq ce^2\vec{n} - \norm{\vec{n}}^2\vec{F} \label{eq:chapters/11/11/3/19/3} \\
f &\triangleq \norm{\vec{n}}^2\norm{\vec{F}}^2 - c^2e^2 \label{eq:chapters/11/11/3/19/4}
\end{align}
Given that the conic is an ellipse with major axis along the $y$-axis,
\begin{align}
\vec{n} = \myvec{0\\1}
\end{align}
Thus,
\begin{align}
\vec{V} = \myvec{1&0\\0&1-e^2} \label{eq:chapters/11/11/3/19/5} \\
\vec{u} = ce^2\myvec{0\\1} - \vec{F} \label{eq:chapters/11/11/3/19/6} \\
f = \norm{\vec{F}}^2 - c^2e^2 \label{eq:chapters/11/11/3/19/7}
\end{align}
The centre of the conic is given by
\begin{align}
\vec{c} = -\vec{V}^{-1}\vec{u}
\label{eq:chapters/11/11/3/19/8}
\end{align}
Since $\vec{c} = \vec{0}$ and $\vec{V}^{-1} \neq \vec{0}$, it follows from \eqref{eq:chapters/11/11/3/19/8} that 
\begin{align}
\vec{u} = \vec{0}
\end{align}
Thus, from \eqref{eq:chapters/11/11/3/19/6},
\begin{align}
\vec{F} = \myvec{0\\ce^2}
\label{eq:chapters/11/11/3/19/9}
\end{align}
and so,
\begin{align}
f = c^2e^2\brak{e^2-1}
%\label{eq:chapters/11/11/3/19/9}
\end{align}
Given that the conic passes through 
\begin{align}
\vec{P} = \myvec{3\\2},
\end{align}
putting $\vec{x} = \vec{P}$ in \eqref{eq:chapters/11/11/3/19/1} we get,
\begin{align}
\myvec{3&2}\myvec{1&0\\0&1-e^2}\myvec{3\\2} + f &= 0 \\
\implies 4e^2 - f = 13 \label{eq:chapters/11/11/3/19/10}
\end{align}
Given that the conic passes through
\begin{align}
\vec{Q} = \myvec{1\\6},
\end{align}
putting $\vec{x} = \vec{Q}$ in \eqref{eq:chapters/11/11/3/19/1}, we get
\begin{align}
\myvec{1&6}\myvec{1&0\\0&1-e^2}\myvec{1\\6} + f &= 0 \\
\implies 36e^2 - f = 37 \label{eq:chapters/11/11/3/19/11}
\end{align}
The equations \eqref{eq:chapters/11/11/3/19/10} and \eqref{eq:chapters/11/11/3/19/11} can be formulated as the following matrix equation
\begin{align}
\myvec{4&-1\\36&-1}\myvec{e^2\\f} = \myvec{13\\37}
\label{eq:chapters/11/11/3/19/12}
\end{align}
The augmented matrix is given by,
\begin{align}
\myvec{4&-1&\vline&13\\36&-1&\vline&37}
\end{align}
yielding
\begin{align}
\xleftrightarrow[]{R_1\leftarrow R_1-R_2} &\myvec{-32&0&\vline&-24\\36&-1&\vline&37} \\
\xleftrightarrow[]{R_1\leftarrow-\frac{R_1}{8}}& \myvec{4&0&\vline&3\\36&-1&\vline&37} \\
\xleftrightarrow[]{R_2\leftarrow R_2-9R_1}
&\myvec{4&0&\vline&3\\0&-1&\vline&10} \\
\xleftrightarrow[R_2\leftarrow -R_2]{R_1\leftarrow \frac{R_1}{4}}
&\myvec{1&0&\vline&\frac{3}{4}\\0&1&\vline&-10}
\end{align}
Thus,
\begin{align}
e^2 = \frac{3}{4},\ f = -10
\end{align}
and the equation of the conic is given by
\begin{align}
\vec{x}^\top\myvec{1&0\\0&\frac{1}{4}}\vec{x} - 10 = 0
\end{align}
See Fig. 
\ref{fig:chapters/11/11/3/19/1}.
\begin{figure}[ht]
\centering
\includegraphics[width = \columnwidth]{chapters/11/11/3/19/figs/fig1.png}
\caption{Graph}
\label{fig:chapters/11/11/3/19/1}
\end{figure}
\begin{table}[h]
\centering
%%%%%%%%%%%%%%%%%%%%%%%%%%%%%%%%%%%%%%%%%%%%%%%%%%%%%%%%%%%%%%%%%%%%%%
%%                                                                  %%
%%  This is a LaTeX2e table fragment exported from Gnumeric.        %%
%%                                                                  %%
%%%%%%%%%%%%%%%%%%%%%%%%%%%%%%%%%%%%%%%%%%%%%%%%%%%%%%%%%%%%%%%%%%%%%%

\begin{center}
\begin{tabular}{|c|c|c|}
\hline
\textbf{Parameter}& \textbf{Description} &\textbf{Value}\\ \hline
$\vec{C}$		 &	center of the ellipse&$\myvec{0\\0}$\\ \hline
$\vec{m}$		 &	Direction vector of major axis &$\myvec{0\\1}$\\ \hline
$\vec{P}$		 &  Point on the ellipse&$\myvec{3\\2}$ \\ \hline
$\vec{Q}$		 &  Point on the ellipse&$\myvec{1\\6}$ \\ \hline
\end{tabular}
\end{center}

\caption{}
\label{tab:chapters/11/11/3/19/1}
\end{table}

\subsection{Hyperbola}
\begin{enumerate}[label=\thesection.\arabic*,ref=\thesection.\theenumi]
\numberwithin{equation}{enumi}
\numberwithin{figure}{enumi}
\numberwithin{table}{enumi}

	\item Find the coordinates of the focii, the vertices, the eccentricity and the length of the latus rectum of a hyperbola whose equation is given by $\frac{x^2}{16}-\frac{y^2}{9} = 1$. \\ 
		\solution
		\iffalse
\documentclass[journal,12pt,twocolumn]{IEEEtran}
\usepackage{setspace}
\usepackage{gensymb}
\singlespacing
\usepackage[cmex10]{amsmath}
\usepackage{amsthm}
\usepackage{mathrsfs}
\usepackage{txfonts}
\usepackage{stfloats}
\usepackage{bm}
\usepackage{cite}
\usepackage{cases}
\usepackage{subfig}
\usepackage{longtable}
\usepackage{multirow}
\usepackage{enumitem}
\usepackage{mathtools}
\usepackage{steinmetz}
\usepackage{tikz}
\usepackage{circuitikz}
\usepackage{verbatim}
\usepackage{tfrupee}
\usepackage[breaklinks=true]{hyperref}
\usepackage{tkz-euclide}
\usetikzlibrary{calc,math}
\usepackage{listings}
    \usepackage{color}                                            %%
    \usepackage{array}                                            %%
    \usepackage{longtable}                                        %%
    \usepackage{calc}                                             %%
    \usepackage{multirow}                                         %%
    \usepackage{hhline}                                           %%
    \usepackage{ifthen}                                           %%
  %optionally (for landscape tables embedded in another document): %%
    \usepackage{lscape}     
\usepackage{multicol}
\usepackage{chngcntr}
\DeclareMathOperator*{\Res}{Res}
\renewcommand\thesection{\arabic{section}}
\renewcommand\thesubsection{\thesection.\arabic{subsection}}
\renewcommand\thesubsubsection{\thesubsection.\arabic{subsubsection}}

\renewcommand\thesectiondis{\arabic{section}}
\renewcommand\thesubsectiondis{\thesectiondis.\arabic{subsection}}
\renewcommand\thesubsubsectiondis{\thesubsectiondis.\arabic{subsubsection}}

% correct bad hyphenation here
\hyphenation{op-tical net-works semi-conduc-tor}
\def\inputGnumericTable{}                                 %%

\lstset{
frame=single, 
breaklines=true,
columns=fullflexible
}

\begin{document}


\newtheorem{theorem}{Theorem}[section]
\newtheorem{problem}{Problem}
\newtheorem{proposition}{Proposition}[section]
\newtheorem{lemma}{Lemma}[section]
\newtheorem{corollary}[theorem]{Corollary}
\newtheorem{example}{Example}[section]
\newtheorem{definition}[problem]{Definition}
\newcommand{\BEQA}{\begin{eqnarray}}
\newcommand{\EEQA}{\end{eqnarray}}
\newcommand{\define}{\stackrel{\triangle}{=}}

\bibliographystyle{IEEEtran}
\providecommand{\mbf}{\mathbf}
\providecommand{\pr}[1]{\ensuremath{\Pr\left(#1\right)}}
\providecommand{\qfunc}[1]{\ensuremath{Q\left(#1\right)}}
\providecommand{\sbrak}[1]{\ensuremath{{}\left[#1\right]}}
\providecommand{\lsbrak}[1]{\ensuremath{{}\left[#1\right.}}
\providecommand{\rsbrak}[1]{\ensuremath{{}\left.#1\right]}}
\providecommand{\brak}[1]{\ensuremath{\left(#1\right)}}
\providecommand{\lbrak}[1]{\ensuremath{\left(#1\right.}}
\providecommand{\rbrak}[1]{\ensuremath{\left.#1\right)}}
\providecommand{\cbrak}[1]{\ensuremath{\left\{#1\right\}}}
\providecommand{\lcbrak}[1]{\ensuremath{\left\{#1\right.}}
\providecommand{\rcbrak}[1]{\ensuremath{\left.#1\right\}}}
\theoremstyle{remark}
\newtheorem{rem}{Remark}
\newcommand{\sgn}{\mathop{\mathrm{sgn}}}
\providecommand{\abs}[1]{\left\vert#1\right\vert}
\providecommand{\res}[1]{\Res\displaylimits_{#1}} 
\providecommand{\norm}[1]{\left\lVert#1\right\rVert}
\providecommand{\mtx}[1]{\mathbf{#1}}
\providecommand{\mean}[1]{E\left[ #1 \right]}
\providecommand{\fourier}{\overset{\mathcal{F}}{ \rightleftharpoons}}
\providecommand{\system}{\overset{\mathcal{H}}{ \longleftrightarrow}}
\newcommand{\solution}{\noindent \textbf{Solution: }}
\newcommand{\cosec}{\,\text{cosec}\,}
\providecommand{\dec}[2]{\ensuremath{\overset{#1}{\underset{#2}{\gtrless}}}}
\newcommand{\myvec}[1]{\ensuremath{\begin{pmatrix}#1\end{pmatrix}}}
\newcommand{\mydet}[1]{\ensuremath{\begin{vmatrix}#1\end{vmatrix}}}
\numberwithin{equation}{subsection}
\makeatletter
\@addtoreset{figure}{problem}
\makeatother

\let\StandardTheFigure\thefigure
\let\vec\mathbf
\renewcommand{\thefigure}{\theproblem}



\def\putbox#1#2#3{\makebox[0in][l]{\makebox[#1][l]{}\raisebox{\baselineskip}[0in][0in]{\raisebox{#2}[0in][0in]{#3}}}}
     \def\rightbox#1{\makebox[0in][r]{#1}}
     \def\centbox#1{\makebox[0in]{#1}}
     \def\topbox#1{\raisebox{-\baselineskip}[0in][0in]{#1}}
     \def\midbox#1{\raisebox{-0.5\baselineskip}[0in][0in]{#1}}

\vspace{3cm}


\title{Assignment 1}
\author{Jaswanth Chowdary Madala}





% make the title area
\maketitle

\newpage

%\tableofcontents

\bigskip

\renewcommand{\thefigure}{\theenumi}
\renewcommand{\thetable}{\theenumi}


\begin{enumerate}
%\begin{figure}[ht]
%\centering
%\includegraphics[width = \columnwidth]{"./chapters/11/11/4/13/figs/fig.png"}
%\caption{Graph}
%\label{fig:chapters/11/11/4/13/1}
%\end{figure}

\textbf{Solution:}
\fi
The equation of the conic with focus $\vec{F}$, directrix $\vec{n}^\top\vec{x} = c$ and eccentricity $e$ is given by
\begin{align}
\vec{x}^\top\vec{V}\vec{x} + 2\vec{u}^\top\vec{x} + f = 0
\label{eq:chapters/11/11/4/13/1}
\end{align}
where
\begin{align}
\vec{V} &\triangleq \norm{\vec{n}}^2\vec{I} - e^2\vec{n}\vec{n}^\top \label{eq:chapters/11/11/4/13/2} \\
\vec{u} &\triangleq ce^2\vec{n} - \norm{\vec{n}}^2\vec{F} \label{eq:chapters/11/11/4/13/3} \\
f &\triangleq \norm{\vec{n}}^2\norm{\vec{F}}^2 - c^2e^2 \label{eq:chapters/11/11/4/13/4}
\end{align}
also
\begin{align}
f_0 &= \vec{u}^\top\vec{V}^{-1}\vec{u} - f\\
l &= 2\frac{\sqrt{\abs{f_0\lambda_2}}}{\lambda_1}
\end{align}

\begin{enumerate}
\item $\vec{n}$: Given that the conic has foci as
\begin{align}
\vec{F_1} &= \myvec{4\\0}\\
\vec{F_2} &= \myvec{-4\\0}
\end{align}
The direction vector of $F_1F_2$ is given by
\begin{align}
\vec{m} &= \vec{F_1} - \vec{F_2}\\
&= \myvec{1\\0}
\end{align}
Hence the normal to the directrix is given by,
\begin{align}
\vec{n} = \myvec{1\\0}
\end{align}

\item $\vec{u}$: The centre of the conic is given by
\begin{align}
\vec{c} &= \frac{\vec{F_1} + \vec{F_2}}{2}
\end{align}
\begin{align}
\vec{c} &= \myvec{0\\0}\\
\vec{c} &= -\vec{V}^{-1}\vec{u}
\label{eq:chapters/11/11/4/13/5}
\end{align}
Since $\vec{c} = \vec{0}$ and $\vec{V}^{-1} \neq \vec{0}$, it follows from \eqref{eq:chapters/11/11/4/13/5} that 
\begin{align}
\vec{u} = \myvec{0\\0}
\end{align}

From the above expressions we get
\begin{align}
\vec{V} &= \myvec{1-e^2&0\\0&1} \label{eq:chapters/11/11/4/13/6} \\
\vec{F} &= \myvec{ce^2\\0} \label{eq:chapters/11/11/4/13/7}\\
f &= c^2e^2\brak{e^2-1} \label{eq:chapters/11/11/4/13/8}\\
f_0 &= - f \label{eq:chapters/11/11/4/13/9}\\
l &= 2\frac{\sqrt{\abs{f\lambda_2}}}{\lambda_1}\label{eq:chapters/11/11/4/13/10}
\end{align}

From equation \eqref{eq:chapters/11/11/4/13/6} the eigen values of matrix $\vec{V}$ - $\lambda_1, \lambda_2$ are given by,
\begin{align}
\lambda_1 &= 1-e^2
\label{eq:chapters/11/11/4/13/11}\\
\lambda_2 &= 1
\label{eq:chapters/11/11/4/13/12}
\end{align}

From equation \eqref{eq:chapters/11/11/4/13/7} we get,
\begin{align}
ce^2 = 4 \label{eq:chapters/11/11/4/13/13}
\end{align}
\item Eccentricity: Given that the conic has the latus rectum length 12. Substituting the expressions of $\lambda_1,\lambda_2$ from the equations \eqref{eq:chapters/11/11/4/13/11}, \eqref{eq:chapters/11/11/4/13/12} in \eqref{eq:chapters/11/11/4/13/10} gives
\begin{align}
l = \frac{2ce}{\sqrt{e^2-1}} &= 12\\
\frac{ce}{\sqrt{e^2-1}} &= 6
\end{align}
Substitute the expression of $c$ from \eqref{eq:chapters/11/11/4/13/13} gives,
\begin{align}
\frac{4}{e\sqrt{e^2-1}} &= 6
\end{align}
Squaring on both sides gives,
\begin{align}
9e^2\brak{e^2-1} &= 4\\
9e^4-9e^2-4 &= 0
\label{eq:chapters/11/11/4/13/14}
\end{align}
The equation \eqref{eq:chapters/11/11/4/13/14} is a quadratic equation in $e^2$.
Solving it gives two roots one of which is negative, as $e^2$ is positive we have
\begin{align}
%e^2 &= \frac{-\brak{-9}\pm\sqrt{\brak{-9}^2-4\times9\times\brak{-4}}}{2\times9}\\
e^2 &= \frac{4}{3}
\end{align}
From equation \eqref{eq:chapters/11/11/4/13/8}, \eqref{eq:chapters/11/11/4/13/13}, we get
\begin{align}
f = 4
\end{align} 
\end{enumerate}
The equation of the conic is given by
\begin{align}
\vec{x}^\top\myvec{-\frac{1}{3}&0\\0&1}\vec{x} +4 = 0
\end{align}
\begin{figure}[ht]
\centering
\includegraphics[width = \columnwidth]{chapters/11/11/4/13/figs/fig1.png}
\caption{Graph}
\label{fig:chapters/11/11/4/13/1}
\end{figure}
\begin{table}[h]
\centering
\begin{tabular}{|c|c|c|}
  \hline
  \textbf{Symbol}&\textbf{Value}&\textbf{Description}\\
  \hline
  $a$ & 8 & $BC$\\
  \hline
	$\angle{B}$ & 45$\degree{}$ & $\angle{B}$ in $\triangle$$ABC$ \\
  \hline
	$k$ & 3.5 & $AB-AC$ i.e $c-b$ \\
  \hline 
	$\vec{e_2}$ & $\myvec{
			0\\
			1\\
			}$ & Basis vector\\
 \hline			
\end{tabular}

\caption{}
\label{tab:chapters/11/11/4/13/1}
\end{table}

	\item Find the coordinates of the focii, the vertices, the eccentricity and the length of the latus rectum of a hyperbola whose equation is given by $\frac{y^2}{9}-\frac{x^2}{27}=1$.
		\\
		\solution
		\\
		\iffalse
\documentclass[journal,12pt,twocolumn]{IEEEtran}
\usepackage{setspace}
\usepackage{gensymb}
\singlespacing
\usepackage[cmex10]{amsmath}
\usepackage{amsthm}
\usepackage{mathrsfs}
\usepackage{txfonts}
\usepackage{stfloats}
\usepackage{bm}
\usepackage{cite}
\usepackage{cases}
\usepackage{subfig}
\usepackage{longtable}
\usepackage{multirow}
\usepackage{enumitem}
\usepackage{mathtools}
\usepackage{steinmetz}
\usepackage{tikz}
\usepackage{circuitikz}
\usepackage{verbatim}
\usepackage{tfrupee}
\usepackage[breaklinks=true]{hyperref}
\usepackage{tkz-euclide}
\usetikzlibrary{calc,math}
\usepackage{listings}
    \usepackage{color}                                            %%
    \usepackage{array}                                            %%
    \usepackage{longtable}                                        %%
    \usepackage{calc}                                             %%
    \usepackage{multirow}                                         %%
    \usepackage{hhline}                                           %%
    \usepackage{ifthen}                                           %%
  %optionally (for landscape tables embedded in another document): %%
    \usepackage{lscape}     
\usepackage{multicol}
\usepackage{chngcntr}
\DeclareMathOperator*{\Res}{Res}
\renewcommand\thesection{\arabic{section}}
\renewcommand\thesubsection{\thesection.\arabic{subsection}}
\renewcommand\thesubsubsection{\thesubsection.\arabic{subsubsection}}

\renewcommand\thesectiondis{\arabic{section}}
\renewcommand\thesubsectiondis{\thesectiondis.\arabic{subsection}}
\renewcommand\thesubsubsectiondis{\thesubsectiondis.\arabic{subsubsection}}

% correct bad hyphenation here
\hyphenation{op-tical net-works semi-conduc-tor}
\def\inputGnumericTable{}                                 %%

\lstset{
frame=single, 
breaklines=true,
columns=fullflexible
}

\begin{document}


\newtheorem{theorem}{Theorem}[section]
\newtheorem{problem}{Problem}
\newtheorem{proposition}{Proposition}[section]
\newtheorem{lemma}{Lemma}[section]
\newtheorem{corollary}[theorem]{Corollary}
\newtheorem{example}{Example}[section]
\newtheorem{definition}[problem]{Definition}
\newcommand{\BEQA}{\begin{eqnarray}}
\newcommand{\EEQA}{\end{eqnarray}}
\newcommand{\define}{\stackrel{\triangle}{=}}

\bibliographystyle{IEEEtran}
\providecommand{\mbf}{\mathbf}
\providecommand{\pr}[1]{\ensuremath{\Pr\left(#1\right)}}
\providecommand{\qfunc}[1]{\ensuremath{Q\left(#1\right)}}
\providecommand{\sbrak}[1]{\ensuremath{{}\left[#1\right]}}
\providecommand{\lsbrak}[1]{\ensuremath{{}\left[#1\right.}}
\providecommand{\rsbrak}[1]{\ensuremath{{}\left.#1\right]}}
\providecommand{\brak}[1]{\ensuremath{\left(#1\right)}}
\providecommand{\lbrak}[1]{\ensuremath{\left(#1\right.}}
\providecommand{\rbrak}[1]{\ensuremath{\left.#1\right)}}
\providecommand{\cbrak}[1]{\ensuremath{\left\{#1\right\}}}
\providecommand{\lcbrak}[1]{\ensuremath{\left\{#1\right.}}
\providecommand{\rcbrak}[1]{\ensuremath{\left.#1\right\}}}
\theoremstyle{remark}
\newtheorem{rem}{Remark}
\newcommand{\sgn}{\mathop{\mathrm{sgn}}}
\providecommand{\abs}[1]{\left\vert#1\right\vert}
\providecommand{\res}[1]{\Res\displaylimits_{#1}} 
\providecommand{\norm}[1]{\left\lVert#1\right\rVert}
\providecommand{\mtx}[1]{\mathbf{#1}}
\providecommand{\mean}[1]{E\left[ #1 \right]}
\providecommand{\fourier}{\overset{\mathcal{F}}{ \rightleftharpoons}}
\providecommand{\system}{\overset{\mathcal{H}}{ \longleftrightarrow}}
\newcommand{\solution}{\noindent \textbf{Solution: }}
\newcommand{\cosec}{\,\text{cosec}\,}
\providecommand{\dec}[2]{\ensuremath{\overset{#1}{\underset{#2}{\gtrless}}}}
\newcommand{\myvec}[1]{\ensuremath{\begin{pmatrix}#1\end{pmatrix}}}
\newcommand{\mydet}[1]{\ensuremath{\begin{vmatrix}#1\end{vmatrix}}}
\numberwithin{equation}{subsection}
\makeatletter
\@addtoreset{figure}{problem}
\makeatother

\let\StandardTheFigure\thefigure
\let\vec\mathbf
\renewcommand{\thefigure}{\theproblem}



\def\putbox#1#2#3{\makebox[0in][l]{\makebox[#1][l]{}\raisebox{\baselineskip}[0in][0in]{\raisebox{#2}[0in][0in]{#3}}}}
     \def\rightbox#1{\makebox[0in][r]{#1}}
     \def\centbox#1{\makebox[0in]{#1}}
     \def\topbox#1{\raisebox{-\baselineskip}[0in][0in]{#1}}
     \def\midbox#1{\raisebox{-0.5\baselineskip}[0in][0in]{#1}}

\vspace{3cm}


\title{Assignment 1}
\author{Jaswanth Chowdary Madala}





% make the title area
\maketitle

\newpage

%\tableofcontents

\bigskip

\renewcommand{\thefigure}{\theenumi}
\renewcommand{\thetable}{\theenumi}


\begin{enumerate}
%\begin{figure}[ht]
%\centering
%\includegraphics[width = \columnwidth]{"./chapters/11/11/4/13/figs/fig.png"}
%\caption{Graph}
%\label{fig:chapters/11/11/4/13/1}
%\end{figure}

\textbf{Solution:}
\fi
The equation of the conic with focus $\vec{F}$, directrix $\vec{n}^\top\vec{x} = c$ and eccentricity $e$ is given by
\begin{align}
\vec{x}^\top\vec{V}\vec{x} + 2\vec{u}^\top\vec{x} + f = 0
\label{eq:chapters/11/11/4/13/1}
\end{align}
where
\begin{align}
\vec{V} &\triangleq \norm{\vec{n}}^2\vec{I} - e^2\vec{n}\vec{n}^\top \label{eq:chapters/11/11/4/13/2} \\
\vec{u} &\triangleq ce^2\vec{n} - \norm{\vec{n}}^2\vec{F} \label{eq:chapters/11/11/4/13/3} \\
f &\triangleq \norm{\vec{n}}^2\norm{\vec{F}}^2 - c^2e^2 \label{eq:chapters/11/11/4/13/4}
\end{align}
also
\begin{align}
f_0 &= \vec{u}^\top\vec{V}^{-1}\vec{u} - f\\
l &= 2\frac{\sqrt{\abs{f_0\lambda_2}}}{\lambda_1}
\end{align}

\begin{enumerate}
\item $\vec{n}$: Given that the conic has foci as
\begin{align}
\vec{F_1} &= \myvec{4\\0}\\
\vec{F_2} &= \myvec{-4\\0}
\end{align}
The direction vector of $F_1F_2$ is given by
\begin{align}
\vec{m} &= \vec{F_1} - \vec{F_2}\\
&= \myvec{1\\0}
\end{align}
Hence the normal to the directrix is given by,
\begin{align}
\vec{n} = \myvec{1\\0}
\end{align}

\item $\vec{u}$: The centre of the conic is given by
\begin{align}
\vec{c} &= \frac{\vec{F_1} + \vec{F_2}}{2}
\end{align}
\begin{align}
\vec{c} &= \myvec{0\\0}\\
\vec{c} &= -\vec{V}^{-1}\vec{u}
\label{eq:chapters/11/11/4/13/5}
\end{align}
Since $\vec{c} = \vec{0}$ and $\vec{V}^{-1} \neq \vec{0}$, it follows from \eqref{eq:chapters/11/11/4/13/5} that 
\begin{align}
\vec{u} = \myvec{0\\0}
\end{align}

From the above expressions we get
\begin{align}
\vec{V} &= \myvec{1-e^2&0\\0&1} \label{eq:chapters/11/11/4/13/6} \\
\vec{F} &= \myvec{ce^2\\0} \label{eq:chapters/11/11/4/13/7}\\
f &= c^2e^2\brak{e^2-1} \label{eq:chapters/11/11/4/13/8}\\
f_0 &= - f \label{eq:chapters/11/11/4/13/9}\\
l &= 2\frac{\sqrt{\abs{f\lambda_2}}}{\lambda_1}\label{eq:chapters/11/11/4/13/10}
\end{align}

From equation \eqref{eq:chapters/11/11/4/13/6} the eigen values of matrix $\vec{V}$ - $\lambda_1, \lambda_2$ are given by,
\begin{align}
\lambda_1 &= 1-e^2
\label{eq:chapters/11/11/4/13/11}\\
\lambda_2 &= 1
\label{eq:chapters/11/11/4/13/12}
\end{align}

From equation \eqref{eq:chapters/11/11/4/13/7} we get,
\begin{align}
ce^2 = 4 \label{eq:chapters/11/11/4/13/13}
\end{align}
\item Eccentricity: Given that the conic has the latus rectum length 12. Substituting the expressions of $\lambda_1,\lambda_2$ from the equations \eqref{eq:chapters/11/11/4/13/11}, \eqref{eq:chapters/11/11/4/13/12} in \eqref{eq:chapters/11/11/4/13/10} gives
\begin{align}
l = \frac{2ce}{\sqrt{e^2-1}} &= 12\\
\frac{ce}{\sqrt{e^2-1}} &= 6
\end{align}
Substitute the expression of $c$ from \eqref{eq:chapters/11/11/4/13/13} gives,
\begin{align}
\frac{4}{e\sqrt{e^2-1}} &= 6
\end{align}
Squaring on both sides gives,
\begin{align}
9e^2\brak{e^2-1} &= 4\\
9e^4-9e^2-4 &= 0
\label{eq:chapters/11/11/4/13/14}
\end{align}
The equation \eqref{eq:chapters/11/11/4/13/14} is a quadratic equation in $e^2$.
Solving it gives two roots one of which is negative, as $e^2$ is positive we have
\begin{align}
%e^2 &= \frac{-\brak{-9}\pm\sqrt{\brak{-9}^2-4\times9\times\brak{-4}}}{2\times9}\\
e^2 &= \frac{4}{3}
\end{align}
From equation \eqref{eq:chapters/11/11/4/13/8}, \eqref{eq:chapters/11/11/4/13/13}, we get
\begin{align}
f = 4
\end{align} 
\end{enumerate}
The equation of the conic is given by
\begin{align}
\vec{x}^\top\myvec{-\frac{1}{3}&0\\0&1}\vec{x} +4 = 0
\end{align}
\begin{figure}[ht]
\centering
\includegraphics[width = \columnwidth]{chapters/11/11/4/13/figs/fig1.png}
\caption{Graph}
\label{fig:chapters/11/11/4/13/1}
\end{figure}
\begin{table}[h]
\centering
\begin{tabular}{|c|c|c|}
  \hline
  \textbf{Symbol}&\textbf{Value}&\textbf{Description}\\
  \hline
  $a$ & 8 & $BC$\\
  \hline
	$\angle{B}$ & 45$\degree{}$ & $\angle{B}$ in $\triangle$$ABC$ \\
  \hline
	$k$ & 3.5 & $AB-AC$ i.e $c-b$ \\
  \hline 
	$\vec{e_2}$ & $\myvec{
			0\\
			1\\
			}$ & Basis vector\\
 \hline			
\end{tabular}

\caption{}
\label{tab:chapters/11/11/4/13/1}
\end{table}

	\item Find the coordinates of the foci and the vertices, the eccentricity and the length of the latus rectum of the hyperbolas, whose equation is given by $5{y^2}-9{x^2}=36$.
		\\
		\solution
		\\
		\iffalse
\documentclass[journal,12pt,twocolumn]{IEEEtran}
\usepackage{setspace}
\usepackage{gensymb}
\singlespacing
\usepackage[cmex10]{amsmath}
\usepackage{amsthm}
\usepackage{mathrsfs}
\usepackage{txfonts}
\usepackage{stfloats}
\usepackage{bm}
\usepackage{cite}
\usepackage{cases}
\usepackage{subfig}
\usepackage{longtable}
\usepackage{multirow}
\usepackage{enumitem}
\usepackage{mathtools}
\usepackage{steinmetz}
\usepackage{tikz}
\usepackage{circuitikz}
\usepackage{verbatim}
\usepackage{tfrupee}
\usepackage[breaklinks=true]{hyperref}
\usepackage{tkz-euclide}
\usetikzlibrary{calc,math}
\usepackage{listings}
    \usepackage{color}                                            %%
    \usepackage{array}                                            %%
    \usepackage{longtable}                                        %%
    \usepackage{calc}                                             %%
    \usepackage{multirow}                                         %%
    \usepackage{hhline}                                           %%
    \usepackage{ifthen}                                           %%
  %optionally (for landscape tables embedded in another document): %%
    \usepackage{lscape}     
\usepackage{multicol}
\usepackage{chngcntr}
\DeclareMathOperator*{\Res}{Res}
\renewcommand\thesection{\arabic{section}}
\renewcommand\thesubsection{\thesection.\arabic{subsection}}
\renewcommand\thesubsubsection{\thesubsection.\arabic{subsubsection}}

\renewcommand\thesectiondis{\arabic{section}}
\renewcommand\thesubsectiondis{\thesectiondis.\arabic{subsection}}
\renewcommand\thesubsubsectiondis{\thesubsectiondis.\arabic{subsubsection}}

% correct bad hyphenation here
\hyphenation{op-tical net-works semi-conduc-tor}
\def\inputGnumericTable{}                                 %%

\lstset{
frame=single, 
breaklines=true,
columns=fullflexible
}

\begin{document}


\newtheorem{theorem}{Theorem}[section]
\newtheorem{problem}{Problem}
\newtheorem{proposition}{Proposition}[section]
\newtheorem{lemma}{Lemma}[section]
\newtheorem{corollary}[theorem]{Corollary}
\newtheorem{example}{Example}[section]
\newtheorem{definition}[problem]{Definition}
\newcommand{\BEQA}{\begin{eqnarray}}
\newcommand{\EEQA}{\end{eqnarray}}
\newcommand{\define}{\stackrel{\triangle}{=}}

\bibliographystyle{IEEEtran}
\providecommand{\mbf}{\mathbf}
\providecommand{\pr}[1]{\ensuremath{\Pr\left(#1\right)}}
\providecommand{\qfunc}[1]{\ensuremath{Q\left(#1\right)}}
\providecommand{\sbrak}[1]{\ensuremath{{}\left[#1\right]}}
\providecommand{\lsbrak}[1]{\ensuremath{{}\left[#1\right.}}
\providecommand{\rsbrak}[1]{\ensuremath{{}\left.#1\right]}}
\providecommand{\brak}[1]{\ensuremath{\left(#1\right)}}
\providecommand{\lbrak}[1]{\ensuremath{\left(#1\right.}}
\providecommand{\rbrak}[1]{\ensuremath{\left.#1\right)}}
\providecommand{\cbrak}[1]{\ensuremath{\left\{#1\right\}}}
\providecommand{\lcbrak}[1]{\ensuremath{\left\{#1\right.}}
\providecommand{\rcbrak}[1]{\ensuremath{\left.#1\right\}}}
\theoremstyle{remark}
\newtheorem{rem}{Remark}
\newcommand{\sgn}{\mathop{\mathrm{sgn}}}
\providecommand{\abs}[1]{\left\vert#1\right\vert}
\providecommand{\res}[1]{\Res\displaylimits_{#1}} 
\providecommand{\norm}[1]{\left\lVert#1\right\rVert}
\providecommand{\mtx}[1]{\mathbf{#1}}
\providecommand{\mean}[1]{E\left[ #1 \right]}
\providecommand{\fourier}{\overset{\mathcal{F}}{ \rightleftharpoons}}
\providecommand{\system}{\overset{\mathcal{H}}{ \longleftrightarrow}}
\newcommand{\solution}{\noindent \textbf{Solution: }}
\newcommand{\cosec}{\,\text{cosec}\,}
\providecommand{\dec}[2]{\ensuremath{\overset{#1}{\underset{#2}{\gtrless}}}}
\newcommand{\myvec}[1]{\ensuremath{\begin{pmatrix}#1\end{pmatrix}}}
\newcommand{\mydet}[1]{\ensuremath{\begin{vmatrix}#1\end{vmatrix}}}
\numberwithin{equation}{subsection}
\makeatletter
\@addtoreset{figure}{problem}
\makeatother

\let\StandardTheFigure\thefigure
\let\vec\mathbf
\renewcommand{\thefigure}{\theproblem}



\def\putbox#1#2#3{\makebox[0in][l]{\makebox[#1][l]{}\raisebox{\baselineskip}[0in][0in]{\raisebox{#2}[0in][0in]{#3}}}}
     \def\rightbox#1{\makebox[0in][r]{#1}}
     \def\centbox#1{\makebox[0in]{#1}}
     \def\topbox#1{\raisebox{-\baselineskip}[0in][0in]{#1}}
     \def\midbox#1{\raisebox{-0.5\baselineskip}[0in][0in]{#1}}

\vspace{3cm}


\title{Assignment 1}
\author{Jaswanth Chowdary Madala}





% make the title area
\maketitle

\newpage

%\tableofcontents

\bigskip

\renewcommand{\thefigure}{\theenumi}
\renewcommand{\thetable}{\theenumi}


\begin{enumerate}
%\begin{figure}[ht]
%\centering
%\includegraphics[width = \columnwidth]{"./chapters/11/11/4/13/figs/fig.png"}
%\caption{Graph}
%\label{fig:chapters/11/11/4/13/1}
%\end{figure}

\textbf{Solution:}
\fi
The equation of the conic with focus $\vec{F}$, directrix $\vec{n}^\top\vec{x} = c$ and eccentricity $e$ is given by
\begin{align}
\vec{x}^\top\vec{V}\vec{x} + 2\vec{u}^\top\vec{x} + f = 0
\label{eq:chapters/11/11/4/13/1}
\end{align}
where
\begin{align}
\vec{V} &\triangleq \norm{\vec{n}}^2\vec{I} - e^2\vec{n}\vec{n}^\top \label{eq:chapters/11/11/4/13/2} \\
\vec{u} &\triangleq ce^2\vec{n} - \norm{\vec{n}}^2\vec{F} \label{eq:chapters/11/11/4/13/3} \\
f &\triangleq \norm{\vec{n}}^2\norm{\vec{F}}^2 - c^2e^2 \label{eq:chapters/11/11/4/13/4}
\end{align}
also
\begin{align}
f_0 &= \vec{u}^\top\vec{V}^{-1}\vec{u} - f\\
l &= 2\frac{\sqrt{\abs{f_0\lambda_2}}}{\lambda_1}
\end{align}

\begin{enumerate}
\item $\vec{n}$: Given that the conic has foci as
\begin{align}
\vec{F_1} &= \myvec{4\\0}\\
\vec{F_2} &= \myvec{-4\\0}
\end{align}
The direction vector of $F_1F_2$ is given by
\begin{align}
\vec{m} &= \vec{F_1} - \vec{F_2}\\
&= \myvec{1\\0}
\end{align}
Hence the normal to the directrix is given by,
\begin{align}
\vec{n} = \myvec{1\\0}
\end{align}

\item $\vec{u}$: The centre of the conic is given by
\begin{align}
\vec{c} &= \frac{\vec{F_1} + \vec{F_2}}{2}
\end{align}
\begin{align}
\vec{c} &= \myvec{0\\0}\\
\vec{c} &= -\vec{V}^{-1}\vec{u}
\label{eq:chapters/11/11/4/13/5}
\end{align}
Since $\vec{c} = \vec{0}$ and $\vec{V}^{-1} \neq \vec{0}$, it follows from \eqref{eq:chapters/11/11/4/13/5} that 
\begin{align}
\vec{u} = \myvec{0\\0}
\end{align}

From the above expressions we get
\begin{align}
\vec{V} &= \myvec{1-e^2&0\\0&1} \label{eq:chapters/11/11/4/13/6} \\
\vec{F} &= \myvec{ce^2\\0} \label{eq:chapters/11/11/4/13/7}\\
f &= c^2e^2\brak{e^2-1} \label{eq:chapters/11/11/4/13/8}\\
f_0 &= - f \label{eq:chapters/11/11/4/13/9}\\
l &= 2\frac{\sqrt{\abs{f\lambda_2}}}{\lambda_1}\label{eq:chapters/11/11/4/13/10}
\end{align}

From equation \eqref{eq:chapters/11/11/4/13/6} the eigen values of matrix $\vec{V}$ - $\lambda_1, \lambda_2$ are given by,
\begin{align}
\lambda_1 &= 1-e^2
\label{eq:chapters/11/11/4/13/11}\\
\lambda_2 &= 1
\label{eq:chapters/11/11/4/13/12}
\end{align}

From equation \eqref{eq:chapters/11/11/4/13/7} we get,
\begin{align}
ce^2 = 4 \label{eq:chapters/11/11/4/13/13}
\end{align}
\item Eccentricity: Given that the conic has the latus rectum length 12. Substituting the expressions of $\lambda_1,\lambda_2$ from the equations \eqref{eq:chapters/11/11/4/13/11}, \eqref{eq:chapters/11/11/4/13/12} in \eqref{eq:chapters/11/11/4/13/10} gives
\begin{align}
l = \frac{2ce}{\sqrt{e^2-1}} &= 12\\
\frac{ce}{\sqrt{e^2-1}} &= 6
\end{align}
Substitute the expression of $c$ from \eqref{eq:chapters/11/11/4/13/13} gives,
\begin{align}
\frac{4}{e\sqrt{e^2-1}} &= 6
\end{align}
Squaring on both sides gives,
\begin{align}
9e^2\brak{e^2-1} &= 4\\
9e^4-9e^2-4 &= 0
\label{eq:chapters/11/11/4/13/14}
\end{align}
The equation \eqref{eq:chapters/11/11/4/13/14} is a quadratic equation in $e^2$.
Solving it gives two roots one of which is negative, as $e^2$ is positive we have
\begin{align}
%e^2 &= \frac{-\brak{-9}\pm\sqrt{\brak{-9}^2-4\times9\times\brak{-4}}}{2\times9}\\
e^2 &= \frac{4}{3}
\end{align}
From equation \eqref{eq:chapters/11/11/4/13/8}, \eqref{eq:chapters/11/11/4/13/13}, we get
\begin{align}
f = 4
\end{align} 
\end{enumerate}
The equation of the conic is given by
\begin{align}
\vec{x}^\top\myvec{-\frac{1}{3}&0\\0&1}\vec{x} +4 = 0
\end{align}
\begin{figure}[ht]
\centering
\includegraphics[width = \columnwidth]{chapters/11/11/4/13/figs/fig1.png}
\caption{Graph}
\label{fig:chapters/11/11/4/13/1}
\end{figure}
\begin{table}[h]
\centering
\begin{tabular}{|c|c|c|}
  \hline
  \textbf{Symbol}&\textbf{Value}&\textbf{Description}\\
  \hline
  $a$ & 8 & $BC$\\
  \hline
	$\angle{B}$ & 45$\degree{}$ & $\angle{B}$ in $\triangle$$ABC$ \\
  \hline
	$k$ & 3.5 & $AB-AC$ i.e $c-b$ \\
  \hline 
	$\vec{e_2}$ & $\myvec{
			0\\
			1\\
			}$ & Basis vector\\
 \hline			
\end{tabular}

\caption{}
\label{tab:chapters/11/11/4/13/1}
\end{table}

	\item Find the equation of the hyperbola whose foci is $\brak{0,\pm 8}$ and vertices $\brak{0,\pm 5}$.
\\
\solution
\item Find the equations of hyperbola having Vertices $\myvec{0\\\pm 3}$ and Foci $\myvec{0\\\pm5}$
	\\
\solution
		\iffalse
\documentclass[journal,12pt,twocolumn]{IEEEtran}
\usepackage{setspace}
\usepackage{gensymb}
\singlespacing
\usepackage[cmex10]{amsmath}
\usepackage{amsthm}
\usepackage{mathrsfs}
\usepackage{txfonts}
\usepackage{stfloats}
\usepackage{bm}
\usepackage{cite}
\usepackage{cases}
\usepackage{subfig}
\usepackage{longtable}
\usepackage{multirow}
\usepackage{enumitem}
\usepackage{mathtools}
\usepackage{steinmetz}
\usepackage{tikz}
\usepackage{circuitikz}
\usepackage{verbatim}
\usepackage{tfrupee}
\usepackage[breaklinks=true]{hyperref}
\usepackage{tkz-euclide}
\usetikzlibrary{calc,math}
\usepackage{listings}
    \usepackage{color}                                            %%
    \usepackage{array}                                            %%
    \usepackage{longtable}                                        %%
    \usepackage{calc}                                             %%
    \usepackage{multirow}                                         %%
    \usepackage{hhline}                                           %%
    \usepackage{ifthen}                                           %%
  %optionally (for landscape tables embedded in another document): %%
    \usepackage{lscape}     
\usepackage{multicol}
\usepackage{chngcntr}
\DeclareMathOperator*{\Res}{Res}
\renewcommand\thesection{\arabic{section}}
\renewcommand\thesubsection{\thesection.\arabic{subsection}}
\renewcommand\thesubsubsection{\thesubsection.\arabic{subsubsection}}

\renewcommand\thesectiondis{\arabic{section}}
\renewcommand\thesubsectiondis{\thesectiondis.\arabic{subsection}}
\renewcommand\thesubsubsectiondis{\thesubsectiondis.\arabic{subsubsection}}

% correct bad hyphenation here
\hyphenation{op-tical net-works semi-conduc-tor}
\def\inputGnumericTable{}                                 %%

\lstset{
frame=single, 
breaklines=true,
columns=fullflexible
}

\begin{document}


\newtheorem{theorem}{Theorem}[section]
\newtheorem{problem}{Problem}
\newtheorem{proposition}{Proposition}[section]
\newtheorem{lemma}{Lemma}[section]
\newtheorem{corollary}[theorem]{Corollary}
\newtheorem{example}{Example}[section]
\newtheorem{definition}[problem]{Definition}
\newcommand{\BEQA}{\begin{eqnarray}}
\newcommand{\EEQA}{\end{eqnarray}}
\newcommand{\define}{\stackrel{\triangle}{=}}

\bibliographystyle{IEEEtran}
\providecommand{\mbf}{\mathbf}
\providecommand{\pr}[1]{\ensuremath{\Pr\left(#1\right)}}
\providecommand{\qfunc}[1]{\ensuremath{Q\left(#1\right)}}
\providecommand{\sbrak}[1]{\ensuremath{{}\left[#1\right]}}
\providecommand{\lsbrak}[1]{\ensuremath{{}\left[#1\right.}}
\providecommand{\rsbrak}[1]{\ensuremath{{}\left.#1\right]}}
\providecommand{\brak}[1]{\ensuremath{\left(#1\right)}}
\providecommand{\lbrak}[1]{\ensuremath{\left(#1\right.}}
\providecommand{\rbrak}[1]{\ensuremath{\left.#1\right)}}
\providecommand{\cbrak}[1]{\ensuremath{\left\{#1\right\}}}
\providecommand{\lcbrak}[1]{\ensuremath{\left\{#1\right.}}
\providecommand{\rcbrak}[1]{\ensuremath{\left.#1\right\}}}
\theoremstyle{remark}
\newtheorem{rem}{Remark}
\newcommand{\sgn}{\mathop{\mathrm{sgn}}}
\providecommand{\abs}[1]{\left\vert#1\right\vert}
\providecommand{\res}[1]{\Res\displaylimits_{#1}} 
\providecommand{\norm}[1]{\left\lVert#1\right\rVert}
\providecommand{\mtx}[1]{\mathbf{#1}}
\providecommand{\mean}[1]{E\left[ #1 \right]}
\providecommand{\fourier}{\overset{\mathcal{F}}{ \rightleftharpoons}}
\providecommand{\system}{\overset{\mathcal{H}}{ \longleftrightarrow}}
\newcommand{\solution}{\noindent \textbf{Solution: }}
\newcommand{\cosec}{\,\text{cosec}\,}
\providecommand{\dec}[2]{\ensuremath{\overset{#1}{\underset{#2}{\gtrless}}}}
\newcommand{\myvec}[1]{\ensuremath{\begin{pmatrix}#1\end{pmatrix}}}
\newcommand{\mydet}[1]{\ensuremath{\begin{vmatrix}#1\end{vmatrix}}}
\numberwithin{equation}{subsection}
\makeatletter
\@addtoreset{figure}{problem}
\makeatother

\let\StandardTheFigure\thefigure
\let\vec\mathbf
\renewcommand{\thefigure}{\theproblem}



\def\putbox#1#2#3{\makebox[0in][l]{\makebox[#1][l]{}\raisebox{\baselineskip}[0in][0in]{\raisebox{#2}[0in][0in]{#3}}}}
     \def\rightbox#1{\makebox[0in][r]{#1}}
     \def\centbox#1{\makebox[0in]{#1}}
     \def\topbox#1{\raisebox{-\baselineskip}[0in][0in]{#1}}
     \def\midbox#1{\raisebox{-0.5\baselineskip}[0in][0in]{#1}}

\vspace{3cm}


\title{Assignment 1}
\author{Jaswanth Chowdary Madala}





% make the title area
\maketitle

\newpage

%\tableofcontents

\bigskip

\renewcommand{\thefigure}{\theenumi}
\renewcommand{\thetable}{\theenumi}


\begin{enumerate}
%\begin{figure}[ht]
%\centering
%\includegraphics[width = \columnwidth]{"./chapters/11/11/4/13/figs/fig.png"}
%\caption{Graph}
%\label{fig:chapters/11/11/4/13/1}
%\end{figure}

\textbf{Solution:}
\fi
The equation of the conic with focus $\vec{F}$, directrix $\vec{n}^\top\vec{x} = c$ and eccentricity $e$ is given by
\begin{align}
\vec{x}^\top\vec{V}\vec{x} + 2\vec{u}^\top\vec{x} + f = 0
\label{eq:chapters/11/11/4/13/1}
\end{align}
where
\begin{align}
\vec{V} &\triangleq \norm{\vec{n}}^2\vec{I} - e^2\vec{n}\vec{n}^\top \label{eq:chapters/11/11/4/13/2} \\
\vec{u} &\triangleq ce^2\vec{n} - \norm{\vec{n}}^2\vec{F} \label{eq:chapters/11/11/4/13/3} \\
f &\triangleq \norm{\vec{n}}^2\norm{\vec{F}}^2 - c^2e^2 \label{eq:chapters/11/11/4/13/4}
\end{align}
also
\begin{align}
f_0 &= \vec{u}^\top\vec{V}^{-1}\vec{u} - f\\
l &= 2\frac{\sqrt{\abs{f_0\lambda_2}}}{\lambda_1}
\end{align}

\begin{enumerate}
\item $\vec{n}$: Given that the conic has foci as
\begin{align}
\vec{F_1} &= \myvec{4\\0}\\
\vec{F_2} &= \myvec{-4\\0}
\end{align}
The direction vector of $F_1F_2$ is given by
\begin{align}
\vec{m} &= \vec{F_1} - \vec{F_2}\\
&= \myvec{1\\0}
\end{align}
Hence the normal to the directrix is given by,
\begin{align}
\vec{n} = \myvec{1\\0}
\end{align}

\item $\vec{u}$: The centre of the conic is given by
\begin{align}
\vec{c} &= \frac{\vec{F_1} + \vec{F_2}}{2}
\end{align}
\begin{align}
\vec{c} &= \myvec{0\\0}\\
\vec{c} &= -\vec{V}^{-1}\vec{u}
\label{eq:chapters/11/11/4/13/5}
\end{align}
Since $\vec{c} = \vec{0}$ and $\vec{V}^{-1} \neq \vec{0}$, it follows from \eqref{eq:chapters/11/11/4/13/5} that 
\begin{align}
\vec{u} = \myvec{0\\0}
\end{align}

From the above expressions we get
\begin{align}
\vec{V} &= \myvec{1-e^2&0\\0&1} \label{eq:chapters/11/11/4/13/6} \\
\vec{F} &= \myvec{ce^2\\0} \label{eq:chapters/11/11/4/13/7}\\
f &= c^2e^2\brak{e^2-1} \label{eq:chapters/11/11/4/13/8}\\
f_0 &= - f \label{eq:chapters/11/11/4/13/9}\\
l &= 2\frac{\sqrt{\abs{f\lambda_2}}}{\lambda_1}\label{eq:chapters/11/11/4/13/10}
\end{align}

From equation \eqref{eq:chapters/11/11/4/13/6} the eigen values of matrix $\vec{V}$ - $\lambda_1, \lambda_2$ are given by,
\begin{align}
\lambda_1 &= 1-e^2
\label{eq:chapters/11/11/4/13/11}\\
\lambda_2 &= 1
\label{eq:chapters/11/11/4/13/12}
\end{align}

From equation \eqref{eq:chapters/11/11/4/13/7} we get,
\begin{align}
ce^2 = 4 \label{eq:chapters/11/11/4/13/13}
\end{align}
\item Eccentricity: Given that the conic has the latus rectum length 12. Substituting the expressions of $\lambda_1,\lambda_2$ from the equations \eqref{eq:chapters/11/11/4/13/11}, \eqref{eq:chapters/11/11/4/13/12} in \eqref{eq:chapters/11/11/4/13/10} gives
\begin{align}
l = \frac{2ce}{\sqrt{e^2-1}} &= 12\\
\frac{ce}{\sqrt{e^2-1}} &= 6
\end{align}
Substitute the expression of $c$ from \eqref{eq:chapters/11/11/4/13/13} gives,
\begin{align}
\frac{4}{e\sqrt{e^2-1}} &= 6
\end{align}
Squaring on both sides gives,
\begin{align}
9e^2\brak{e^2-1} &= 4\\
9e^4-9e^2-4 &= 0
\label{eq:chapters/11/11/4/13/14}
\end{align}
The equation \eqref{eq:chapters/11/11/4/13/14} is a quadratic equation in $e^2$.
Solving it gives two roots one of which is negative, as $e^2$ is positive we have
\begin{align}
%e^2 &= \frac{-\brak{-9}\pm\sqrt{\brak{-9}^2-4\times9\times\brak{-4}}}{2\times9}\\
e^2 &= \frac{4}{3}
\end{align}
From equation \eqref{eq:chapters/11/11/4/13/8}, \eqref{eq:chapters/11/11/4/13/13}, we get
\begin{align}
f = 4
\end{align} 
\end{enumerate}
The equation of the conic is given by
\begin{align}
\vec{x}^\top\myvec{-\frac{1}{3}&0\\0&1}\vec{x} +4 = 0
\end{align}
\begin{figure}[ht]
\centering
\includegraphics[width = \columnwidth]{chapters/11/11/4/13/figs/fig1.png}
\caption{Graph}
\label{fig:chapters/11/11/4/13/1}
\end{figure}
\begin{table}[h]
\centering
\begin{tabular}{|c|c|c|}
  \hline
  \textbf{Symbol}&\textbf{Value}&\textbf{Description}\\
  \hline
  $a$ & 8 & $BC$\\
  \hline
	$\angle{B}$ & 45$\degree{}$ & $\angle{B}$ in $\triangle$$ABC$ \\
  \hline
	$k$ & 3.5 & $AB-AC$ i.e $c-b$ \\
  \hline 
	$\vec{e_2}$ & $\myvec{
			0\\
			1\\
			}$ & Basis vector\\
 \hline			
\end{tabular}

\caption{}
\label{tab:chapters/11/11/4/13/1}
\end{table}

	\iffalse
\documentclass[12pt]{article}
\usepackage{graphicx}
\usepackage{amsmath}
\usepackage{mathtools}
\usepackage{gensymb}

\newcommand{\mydet}[1]{\ensuremath{\begin{vmatrix}#1\end{vmatrix}}}
\providecommand{\brak}[1]{\ensuremath{\left(#1\right)}}
\providecommand{\norm}[1]{\left\lVert#1\right\rVert}
\providecommand{\abs}[1]{\left\vert#1\right\vert}
\newcommand{\solution}{\noindent \textbf{Solution: }}
\newcommand{\myvec}[1]{\ensuremath{\begin{pmatrix}#1\end{pmatrix}}}
\let\vec\mathbf

\begin{document}
\begin{center}
\textbf\large{CONIC SECTIONS}

\end{center}
\section*{Excercise 11.4}
\fi
Given
\begin{align}
	\vec{F} = \myvec{0\\\pm 8}, \vec{V} = \myvec{0\\\pm 5} 
\end{align}
\begin{enumerate}
\item We know the vertex is given as
\begin{align}
	\vec{V} = \pm\myvec{0\\\sqrt{\frac{f_0}{\lambda_2}}} = \pm\myvec{0\\5}\\
	\label{eq:chapters/11/11/4/8/eq1}
	\implies f_0 = 25\lambda_2
\end{align}
\item We know the Focii is given as
\begin{align}
	\vec{F} &= \pm \frac{\brak{\frac{1}{e\sqrt{1-e^2}}}\brak{e^2}\sqrt{\frac{\lambda_1}{f_0}}}{\frac{\lambda_1}{f_0}}\vec{e}_2\\
	        &= \frac{\frac{e}{\sqrt{1-e^2}}}{\sqrt{\frac{\lambda_1}{f_0}}}\vec{e}_2
\end{align}
Substituting \eqref{eq:chapters/11/11/4/8/eq1} we get
\begin{align}
	\vec{F} &= 5e\vec{e}_2\\
	\myvec{0\\8} &= 5e\vec{e}_2\\
	\implies e &= \frac{8}{5}
\end{align}
\item Now we know the eccentricity is given as
\begin{align}
	e = \sqrt{1-\frac{\lambda_2}{\lambda_1}}\\
	\label{eq:chapters/11/11/4/8/eq2}
	\implies \frac{\lambda_2}{\lambda_1} = -\frac{39}{25}
\end{align}
\item Now we know from the standard equation
\begin{align}
	\label{eq:chapters/11/11/4/8/eq3}
	f = \norm{\vec{n}}^2 \norm{\vec{F}}^2 - c^2 e^2
\end{align}
Calculating $\vec{n} \text{ and } c$
\begin{align}
	\vec{n} &= \sqrt{\frac{\lambda_1}{f_0}}\vec{e}_2 = \frac{1}{5}\sqrt{\frac{\lambda_1}{\lambda_2}}\vec{e}_2\\
	        &= \frac{1}{\sqrt{-39}}\vec{e}_2\\
	c &= \frac{1}{e\sqrt{1-e^2}} = \frac{25}{8\sqrt{-39}}	
\end{align}
Now
\begin{align}
	\norm{\vec{n}}^2 &= -\frac{1}{39}\\
	\norm{\vec{F}}^2 &= 64
\end{align}
Substituting all the values in \eqref{eq:chapters/11/11/4/8/eq3} we get
\begin{align}
	f &= -\brak{\frac{1}{39}}\brak{64} + \brak{\frac{25}{8}}^2 \brak{\frac{1}{39}} \brak{\frac{64}{25}}\\
	  &= -1\\
	\label{eq:chapters/11/11/4/8/eq4}  
	f_0  &= -f = 1
\end{align}
substituting \eqref{eq:chapters/11/11/4/8/eq4} in \eqref{eq:chapters/11/11/4/8/eq1} we get
\begin{align}
	\label{eq:chapters/11/11/4/8/eq5}
	\lambda_2 = \frac{1}{25} 
\end{align}
Substituting \eqref{eq:chapters/11/11/4/8/eq5} in \eqref{eq:chapters/11/11/4/8/eq2} we get
\begin{align}
	\lambda_1 = -\frac{1}{39}
\end{align}
\end{enumerate}
Therefore the equation of the hyperbola is given as
\begin{align}
	g\brak{\vec{x}}=\vec{x}^\top \vec{V} \vec{x} + 2\vec{u}^\top \vec{x} + f = 0
\end{align}
where
\begin{align}
	\vec{V} &= \myvec{\lambda_1&0\\0&\lambda_2} = \myvec{-\frac{1}{39}&0\\0&\frac{1}{25}}\\
	\vec{u} &= \vec{0}\\
	f &= -1
\end{align}
See Fig. \ref{fig:chapters/11/11/4/8/Fig1}.
\begin{figure}[!h]
	\begin{center} 
	    \includegraphics[width=\columnwidth]{chapters/11/11/4/8/figs/hyperbola2}
	\end{center}
\caption{}
\label{fig:chapters/11/11/4/8/Fig1}
\end{figure}


\item Find the equation of the hyperbola that satisfies the conditions - Foci \brak{\pm 4, 0}, the latus rectum is of length 12.
\\
\solution
		\iffalse
\documentclass[journal,12pt,twocolumn]{IEEEtran}
\usepackage{setspace}
\usepackage{gensymb}
\singlespacing
\usepackage[cmex10]{amsmath}
\usepackage{amsthm}
\usepackage{mathrsfs}
\usepackage{txfonts}
\usepackage{stfloats}
\usepackage{bm}
\usepackage{cite}
\usepackage{cases}
\usepackage{subfig}
\usepackage{longtable}
\usepackage{multirow}
\usepackage{enumitem}
\usepackage{mathtools}
\usepackage{steinmetz}
\usepackage{tikz}
\usepackage{circuitikz}
\usepackage{verbatim}
\usepackage{tfrupee}
\usepackage[breaklinks=true]{hyperref}
\usepackage{tkz-euclide}
\usetikzlibrary{calc,math}
\usepackage{listings}
    \usepackage{color}                                            %%
    \usepackage{array}                                            %%
    \usepackage{longtable}                                        %%
    \usepackage{calc}                                             %%
    \usepackage{multirow}                                         %%
    \usepackage{hhline}                                           %%
    \usepackage{ifthen}                                           %%
  %optionally (for landscape tables embedded in another document): %%
    \usepackage{lscape}     
\usepackage{multicol}
\usepackage{chngcntr}
\DeclareMathOperator*{\Res}{Res}
\renewcommand\thesection{\arabic{section}}
\renewcommand\thesubsection{\thesection.\arabic{subsection}}
\renewcommand\thesubsubsection{\thesubsection.\arabic{subsubsection}}

\renewcommand\thesectiondis{\arabic{section}}
\renewcommand\thesubsectiondis{\thesectiondis.\arabic{subsection}}
\renewcommand\thesubsubsectiondis{\thesubsectiondis.\arabic{subsubsection}}

% correct bad hyphenation here
\hyphenation{op-tical net-works semi-conduc-tor}
\def\inputGnumericTable{}                                 %%

\lstset{
frame=single, 
breaklines=true,
columns=fullflexible
}

\begin{document}


\newtheorem{theorem}{Theorem}[section]
\newtheorem{problem}{Problem}
\newtheorem{proposition}{Proposition}[section]
\newtheorem{lemma}{Lemma}[section]
\newtheorem{corollary}[theorem]{Corollary}
\newtheorem{example}{Example}[section]
\newtheorem{definition}[problem]{Definition}
\newcommand{\BEQA}{\begin{eqnarray}}
\newcommand{\EEQA}{\end{eqnarray}}
\newcommand{\define}{\stackrel{\triangle}{=}}

\bibliographystyle{IEEEtran}
\providecommand{\mbf}{\mathbf}
\providecommand{\pr}[1]{\ensuremath{\Pr\left(#1\right)}}
\providecommand{\qfunc}[1]{\ensuremath{Q\left(#1\right)}}
\providecommand{\sbrak}[1]{\ensuremath{{}\left[#1\right]}}
\providecommand{\lsbrak}[1]{\ensuremath{{}\left[#1\right.}}
\providecommand{\rsbrak}[1]{\ensuremath{{}\left.#1\right]}}
\providecommand{\brak}[1]{\ensuremath{\left(#1\right)}}
\providecommand{\lbrak}[1]{\ensuremath{\left(#1\right.}}
\providecommand{\rbrak}[1]{\ensuremath{\left.#1\right)}}
\providecommand{\cbrak}[1]{\ensuremath{\left\{#1\right\}}}
\providecommand{\lcbrak}[1]{\ensuremath{\left\{#1\right.}}
\providecommand{\rcbrak}[1]{\ensuremath{\left.#1\right\}}}
\theoremstyle{remark}
\newtheorem{rem}{Remark}
\newcommand{\sgn}{\mathop{\mathrm{sgn}}}
\providecommand{\abs}[1]{\left\vert#1\right\vert}
\providecommand{\res}[1]{\Res\displaylimits_{#1}} 
\providecommand{\norm}[1]{\left\lVert#1\right\rVert}
\providecommand{\mtx}[1]{\mathbf{#1}}
\providecommand{\mean}[1]{E\left[ #1 \right]}
\providecommand{\fourier}{\overset{\mathcal{F}}{ \rightleftharpoons}}
\providecommand{\system}{\overset{\mathcal{H}}{ \longleftrightarrow}}
\newcommand{\solution}{\noindent \textbf{Solution: }}
\newcommand{\cosec}{\,\text{cosec}\,}
\providecommand{\dec}[2]{\ensuremath{\overset{#1}{\underset{#2}{\gtrless}}}}
\newcommand{\myvec}[1]{\ensuremath{\begin{pmatrix}#1\end{pmatrix}}}
\newcommand{\mydet}[1]{\ensuremath{\begin{vmatrix}#1\end{vmatrix}}}
\numberwithin{equation}{subsection}
\makeatletter
\@addtoreset{figure}{problem}
\makeatother

\let\StandardTheFigure\thefigure
\let\vec\mathbf
\renewcommand{\thefigure}{\theproblem}



\def\putbox#1#2#3{\makebox[0in][l]{\makebox[#1][l]{}\raisebox{\baselineskip}[0in][0in]{\raisebox{#2}[0in][0in]{#3}}}}
     \def\rightbox#1{\makebox[0in][r]{#1}}
     \def\centbox#1{\makebox[0in]{#1}}
     \def\topbox#1{\raisebox{-\baselineskip}[0in][0in]{#1}}
     \def\midbox#1{\raisebox{-0.5\baselineskip}[0in][0in]{#1}}

\vspace{3cm}


\title{Assignment 1}
\author{Jaswanth Chowdary Madala}





% make the title area
\maketitle

\newpage

%\tableofcontents

\bigskip

\renewcommand{\thefigure}{\theenumi}
\renewcommand{\thetable}{\theenumi}


\begin{enumerate}
%\begin{figure}[ht]
%\centering
%\includegraphics[width = \columnwidth]{"./chapters/11/11/4/13/figs/fig.png"}
%\caption{Graph}
%\label{fig:chapters/11/11/4/13/1}
%\end{figure}

\textbf{Solution:}
\fi
The equation of the conic with focus $\vec{F}$, directrix $\vec{n}^\top\vec{x} = c$ and eccentricity $e$ is given by
\begin{align}
\vec{x}^\top\vec{V}\vec{x} + 2\vec{u}^\top\vec{x} + f = 0
\label{eq:chapters/11/11/4/13/1}
\end{align}
where
\begin{align}
\vec{V} &\triangleq \norm{\vec{n}}^2\vec{I} - e^2\vec{n}\vec{n}^\top \label{eq:chapters/11/11/4/13/2} \\
\vec{u} &\triangleq ce^2\vec{n} - \norm{\vec{n}}^2\vec{F} \label{eq:chapters/11/11/4/13/3} \\
f &\triangleq \norm{\vec{n}}^2\norm{\vec{F}}^2 - c^2e^2 \label{eq:chapters/11/11/4/13/4}
\end{align}
also
\begin{align}
f_0 &= \vec{u}^\top\vec{V}^{-1}\vec{u} - f\\
l &= 2\frac{\sqrt{\abs{f_0\lambda_2}}}{\lambda_1}
\end{align}

\begin{enumerate}
\item $\vec{n}$: Given that the conic has foci as
\begin{align}
\vec{F_1} &= \myvec{4\\0}\\
\vec{F_2} &= \myvec{-4\\0}
\end{align}
The direction vector of $F_1F_2$ is given by
\begin{align}
\vec{m} &= \vec{F_1} - \vec{F_2}\\
&= \myvec{1\\0}
\end{align}
Hence the normal to the directrix is given by,
\begin{align}
\vec{n} = \myvec{1\\0}
\end{align}

\item $\vec{u}$: The centre of the conic is given by
\begin{align}
\vec{c} &= \frac{\vec{F_1} + \vec{F_2}}{2}
\end{align}
\begin{align}
\vec{c} &= \myvec{0\\0}\\
\vec{c} &= -\vec{V}^{-1}\vec{u}
\label{eq:chapters/11/11/4/13/5}
\end{align}
Since $\vec{c} = \vec{0}$ and $\vec{V}^{-1} \neq \vec{0}$, it follows from \eqref{eq:chapters/11/11/4/13/5} that 
\begin{align}
\vec{u} = \myvec{0\\0}
\end{align}

From the above expressions we get
\begin{align}
\vec{V} &= \myvec{1-e^2&0\\0&1} \label{eq:chapters/11/11/4/13/6} \\
\vec{F} &= \myvec{ce^2\\0} \label{eq:chapters/11/11/4/13/7}\\
f &= c^2e^2\brak{e^2-1} \label{eq:chapters/11/11/4/13/8}\\
f_0 &= - f \label{eq:chapters/11/11/4/13/9}\\
l &= 2\frac{\sqrt{\abs{f\lambda_2}}}{\lambda_1}\label{eq:chapters/11/11/4/13/10}
\end{align}

From equation \eqref{eq:chapters/11/11/4/13/6} the eigen values of matrix $\vec{V}$ - $\lambda_1, \lambda_2$ are given by,
\begin{align}
\lambda_1 &= 1-e^2
\label{eq:chapters/11/11/4/13/11}\\
\lambda_2 &= 1
\label{eq:chapters/11/11/4/13/12}
\end{align}

From equation \eqref{eq:chapters/11/11/4/13/7} we get,
\begin{align}
ce^2 = 4 \label{eq:chapters/11/11/4/13/13}
\end{align}
\item Eccentricity: Given that the conic has the latus rectum length 12. Substituting the expressions of $\lambda_1,\lambda_2$ from the equations \eqref{eq:chapters/11/11/4/13/11}, \eqref{eq:chapters/11/11/4/13/12} in \eqref{eq:chapters/11/11/4/13/10} gives
\begin{align}
l = \frac{2ce}{\sqrt{e^2-1}} &= 12\\
\frac{ce}{\sqrt{e^2-1}} &= 6
\end{align}
Substitute the expression of $c$ from \eqref{eq:chapters/11/11/4/13/13} gives,
\begin{align}
\frac{4}{e\sqrt{e^2-1}} &= 6
\end{align}
Squaring on both sides gives,
\begin{align}
9e^2\brak{e^2-1} &= 4\\
9e^4-9e^2-4 &= 0
\label{eq:chapters/11/11/4/13/14}
\end{align}
The equation \eqref{eq:chapters/11/11/4/13/14} is a quadratic equation in $e^2$.
Solving it gives two roots one of which is negative, as $e^2$ is positive we have
\begin{align}
%e^2 &= \frac{-\brak{-9}\pm\sqrt{\brak{-9}^2-4\times9\times\brak{-4}}}{2\times9}\\
e^2 &= \frac{4}{3}
\end{align}
From equation \eqref{eq:chapters/11/11/4/13/8}, \eqref{eq:chapters/11/11/4/13/13}, we get
\begin{align}
f = 4
\end{align} 
\end{enumerate}
The equation of the conic is given by
\begin{align}
\vec{x}^\top\myvec{-\frac{1}{3}&0\\0&1}\vec{x} +4 = 0
\end{align}
\begin{figure}[ht]
\centering
\includegraphics[width = \columnwidth]{chapters/11/11/4/13/figs/fig1.png}
\caption{Graph}
\label{fig:chapters/11/11/4/13/1}
\end{figure}
\begin{table}[h]
\centering
\begin{tabular}{|c|c|c|}
  \hline
  \textbf{Symbol}&\textbf{Value}&\textbf{Description}\\
  \hline
  $a$ & 8 & $BC$\\
  \hline
	$\angle{B}$ & 45$\degree{}$ & $\angle{B}$ in $\triangle$$ABC$ \\
  \hline
	$k$ & 3.5 & $AB-AC$ i.e $c-b$ \\
  \hline 
	$\vec{e_2}$ & $\myvec{
			0\\
			1\\
			}$ & Basis vector\\
 \hline			
\end{tabular}

\caption{}
\label{tab:chapters/11/11/4/13/1}
\end{table}

    \item Find the equation of the hyperbola whose eccentricity is $e = \frac{4}{3}$
    and whose vertices are
    \begin{align}
        \vec{P_1} = \myvec{7\\0},\ \vec{P_2} = \myvec{-7\\0}
        \label{eq:chapters/11/11/4/14/vert}
    \end{align}
\\
\solution
		\iffalse
\documentclass[journal,12pt,twocolumn]{IEEEtran}
\usepackage{setspace}
\usepackage{gensymb}
\usepackage{xcolor}
\usepackage{caption}
\singlespacing
\usepackage{siunitx}
\usepackage[cmex10]{amsmath}
\usepackage{mathtools}
\usepackage{hyperref}
\usepackage{amsthm}
\usepackage{mathrsfs}
\usepackage{txfonts}
\usepackage{stfloats}
\usepackage{cite}
\usepackage{cases}
\usepackage{subfig}
\usepackage{longtable}
\usepackage{multirow}
\usepackage{enumitem}
\usepackage{bm}
\usepackage{mathtools}
\usepackage{listings}
\usepackage{tikz}
\usetikzlibrary{shapes,arrows,positioning}
\usepackage{circuitikz}
\renewcommand{\vec}[1]{\boldsymbol{\mathbf{#1}}}
\DeclareMathOperator*{\Res}{Res}
\renewcommand\thesection{\arabic{section}}
\renewcommand\thesubsection{\thesection.\arabic{subsection}}
\renewcommand\thesubsubsection{\thesubsection.\arabic{subsubsection}}

\renewcommand\thesectiondis{\arabic{section}}
\renewcommand\thesubsectiondis{\thesectiondis.\arabic{subsection}}
\renewcommand\thesubsubsectiondis{\thesubsectiondis.\arabic{subsubsection}}
\hyphenation{op-tical net-works semi-conduc-tor}

\lstset{
language=Python,
frame=single, 
breaklines=true,
columns=fullflexible
}
\begin{document}
\theoremstyle{definition}
\newtheorem{theorem}{Theorem}[section]
\newtheorem{problem}{Problem}
\newtheorem{proposition}{Proposition}[section]
\newtheorem{lemma}{Lemma}[section]
\newtheorem{corollary}[theorem]{Corollary}
\newtheorem{example}{Example}[section]
\newtheorem{definition}{Definition}[section]
\newcommand{\BEQA}{\begin{eqnarray}}
\newcommand{\EEQA}{\end{eqnarray}}
\newcommand{\define}{\stackrel{\triangle}{=}}
\newcommand{\myvec}[1]{\ensuremath{\begin{pmatrix}#1\end{pmatrix}}}
\newcommand{\mydet}[1]{\ensuremath{\begin{vmatrix}#1\end{vmatrix}}}
\bibliographystyle{IEEEtran}
\providecommand{\nCr}[2]{\,^{#1}C_{#2}} % nCr
\providecommand{\nPr}[2]{\,^{#1}P_{#2}} % nPr
\providecommand{\mbf}{\mathbf}
\providecommand{\pr}[1]{\ensuremath{\Pr\left(#1\right)}}
\providecommand{\qfunc}[1]{\ensuremath{Q\left(#1\right)}}
\providecommand{\sbrak}[1]{\ensuremath{{}\left[#1\right]}}
\providecommand{\lsbrak}[1]{\ensuremath{{}\left[#1\right.}}
\providecommand{\rsbrak}[1]{\ensuremath{{}\left.#1\right]}}
\providecommand{\brak}[1]{\ensuremath{\left(#1\right)}}
\providecommand{\lbrak}[1]{\ensuremath{\left(#1\right.}}
\providecommand{\rbrak}[1]{\ensuremath{\left.#1\right)}}
\providecommand{\cbrak}[1]{\ensuremath{\left\{#1\right\}}}
\providecommand{\lcbrak}[1]{\ensuremath{\left\{#1\right.}}
\providecommand{\rcbrak}[1]{\ensuremath{\left.#1\right\}}}
\theoremstyle{remark}
\newtheorem{rem}{Remark}
\newcommand{\sgn}{\mathop{\mathrm{sgn}}}
\newcommand{\rect}{\mathop{\mathrm{rect}}}
\newcommand{\sinc}{\mathop{\mathrm{sinc}}}
\providecommand{\abs}[1]{\left\vert#1\right\vert}
\providecommand{\res}[1]{\Res\displaylimits_{#1}} 
\providecommand{\norm}[1]{\lVert#1\rVert}
\providecommand{\mtx}[1]{\mathbf{#1}}
\providecommand{\mean}[1]{E\left[ #1 \right]}
\providecommand{\fourier}{\overset{\mathcal{F}}{ \rightleftharpoons}}
\providecommand{\ztrans}{\overset{\mathcal{Z}}{ \rightleftharpoons}}
\providecommand{\system}[1]{\overset{\mathcal{#1}}{ \longleftrightarrow}}
\newcommand{\solution}{\noindent \textbf{Solution: }}
\providecommand{\dec}[2]{\ensuremath{\overset{#1}{\underset{#2}{\gtrless}}}}
\let\StandardTheFigure\thefigure
\def\putbox#1#2#3{\makebox[0in][l]{\makebox[#1][l]{}\raisebox{\baselineskip}[0in][0in]{\raisebox{#2}[0in][0in]{#3}}}}
     \def\rightbox#1{\makebox[0in][r]{#1}}
     \def\centbox#1{\makebox[0in]{#1}}
     \def\topbox#1{\raisebox{-\baselineskip}[0in][0in]{#1}}
     \def\midbox#1{\raisebox{-0.5\baselineskip}[0in][0in]{#1}}

\vspace{3cm}
\title{Conic Assignment}
\author{Gautam Singh}
\maketitle
\bigskip

\begin{abstract}
    This document contains the solution to Question 14 of Exercise 4 in Chapter
    11 of the class 11 NCERT textbook.
\end{abstract}

\begin{enumerate}
\fi
		Let the equation of the conic with focus $\vec{F}$, directrix
    $\vec{n}^\top\vec{x} = c$ and eccentricity $e$ be
    \begin{align}
        \vec{x}^\top\vec{V}\vec{x} + 2\vec{u}^\top\vec{x} + f = 0
        \label{eq:chapters/11/11/4/14/conic-def}
    \end{align}
    where
    \begin{align}
        \vec{V} &\triangleq \norm{\vec{n}}^2\vec{I} - e^2\vec{n}\vec{n}^\top \label{eq:chapters/11/11/4/14/V-def} \\
        \vec{u} &\triangleq ce^2\vec{n} - \norm{\vec{n}}^2\vec{F} \label{eq:chapters/11/11/4/14/u-def} \\
        f &\triangleq \norm{\vec{n}}^2\norm{\vec{F}}^2 - c^2e^2 \label{eq:chapters/11/11/4/14/f-def}
    \end{align}
    The major axis of a conic is the chord which passes through the vertices of the conic.
    The direction vector of the major axis in this case is
    \begin{align}
        \vec{P_2}-\vec{P_1} = \myvec{14\\0}
    \end{align}
    Hence, the normal to the major axis $P_1P_2$ is
    \begin{align}
        \vec{n_M} = \vec{e_2} = \myvec{0\\1}
    \end{align}
    Thus, the equation of the major axis is
    \begin{align}
        \vec{e_2}^\top\vec{x} = \vec{e_2}^\top\myvec{7\\0} = 0
    \end{align}
    which is clearly the $x$-axis.

    Since the conic is a hyperbola whose vertices are given by \eqref{eq:chapters/11/11/4/14/vert}
    and the major axis is the $x$-axis, the directrix is parallel to the $y$-axis.
    Hence,
    \begin{align}
        \vec{n} = \myvec{1\\0}
    \end{align}
    Thus,
    \begin{align}
        \vec{V} = \myvec{1-e^2&0\\0&1} \label{eq:chapters/11/11/4/14/V-val} \\
        \vec{u} = ce^2\myvec{1\\0} - \vec{F} \label{eq:chapters/11/11/4/14/u-val} \\
        f = \norm{\vec{F}}^2 - c^2e^2 \label{eq:chapters/11/11/4/14/f-val}
    \end{align}
    Substituting $\vec{P_1}$ and $\vec{P_2}$ in \eqref{eq:chapters/11/11/4/14/conic-def},
    \begin{align}
        \vec{P_1}^\top\vec{VP_1} + 2\vec{u}^\top\vec{P_1} + f &= 0 \label{eq:chapters/11/11/4/14/ep1} \\
        \vec{P_2}^\top\vec{VP_2} + 2\vec{u}^\top\vec{P_2} + f &= 0 \label{eq:chapters/11/11/4/14/ep2}
    \end{align}
    Subtracting \eqref{eq:chapters/11/11/4/14/ep2} from \eqref{eq:chapters/11/11/4/14/ep1}, and noting that $\vec{P_2} = -\vec{P_1}$,
    \begin{align}
        \vec{u}^\top\vec{P_1} = 0
        \label{eq:chapters/11/11/4/14/u-exp}
    \end{align}
    Hence, from \eqref{eq:chapters/11/11/4/14/vert}, we see that $\vec{u}$ lies on the $y$-axis.
    The general expression of the centre of a conic is given by
    \begin{align}
        \vec{c} &= -\vec{V}^{-1}\vec{u} \\
                &= \frac{1}{e^2-1}\myvec{1&0\\0&1-e^2}\vec{u}
        \label{eq:chapters/11/11/4/14/center}
    \end{align}
    We let $\vec{u} \triangleq \myvec{0\\u}$ and obtain from \eqref{eq:chapters/11/11/4/14/center}
    \begin{align}
        \vec{c} = \myvec{0\\-u} = -\vec{u}
        \label{eq:chapters/11/11/4/14/u-c-0}
    \end{align}
    Since the major axis of the hyperbola is the $x$-axis, we see that $\vec{c}$
    lies on the $x$-axis. Thus, \eqref{eq:chapters/11/11/4/14/u-c-0} implies $\vec{c} = -\vec{u} 
    = \vec{0}$. Thus, from \eqref{eq:chapters/11/11/4/14/u-val},
    \begin{align}
        \vec{F} = \myvec{ce^2\\0}
        \label{eq:chapters/11/11/4/14/F-c-e}
    \end{align}
    and so,
    \begin{align}
        f = c^2e^2\brak{e^2-1}
        \label{eq:chapters/11/11/4/14/f-c-e}
    \end{align}
    Putting $\vec{x} = \vec{P_1}$ or $\vec{x} = \vec{P_2}$ in \eqref{eq:chapters/11/11/4/14/conic-def} 
    and using \eqref{eq:chapters/11/11/4/14/F-c-e} and \eqref{eq:chapters/11/11/4/14/f-c-e},
    \begin{align}
        \myvec{\pm7&0}\myvec{1-e^2&0\\0&1}\myvec{0\\\pm7} + f &= 0 \\
        \implies 49e^2 - f = 49 \label{eq:chapters/11/11/4/14/e1}
    \end{align}
    Since $e = \frac{4}{3}$, \eqref{eq:chapters/11/11/4/14/e1} implies
    \begin{align}
        f = 49\brak{e^2-1} = \frac{343}{9}
    \end{align}
    Therefore, the equation of the conic is
    \begin{align}
        \vec{x}^\top\myvec{-\frac{7}{9}&0\\0&1}\vec{x} + \frac{343}{9} = 0
    \end{align}
    The situation is illustrated in Fig. \ref{fig:chapters/11/11/4/14/hyperbola}.
    \begin{figure}[!ht]
        \centering
        \includegraphics[width=\columnwidth]{chapters/11/11/4/14/figs/hyperbola.png}
        \caption{Locus of the required hyperbola.}
        \label{fig:chapters/11/11/4/14/hyperbola}
    \end{figure}


\end{enumerate}

\subsection{Exercises}
\iffalse
\documentclass[journal,12pt,twocolumn]{IEEEtran}
\usepackage{setspace}
\usepackage{gensymb}
\singlespacing
\usepackage[cmex10]{amsmath}
\usepackage{amsthm}
\usepackage{mathrsfs}
\usepackage{txfonts}
\usepackage{stfloats}
\usepackage{bm}
\usepackage{cite}
\usepackage{cases}
\usepackage{subfig}
\usepackage{longtable}
\usepackage{multirow}
\usepackage{enumitem}
\usepackage{mathtools}
\usepackage{steinmetz}
\usepackage{tikz}
\usepackage{circuitikz}
\usepackage{verbatim}
\usepackage{tfrupee}
\usepackage[breaklinks=true]{hyperref}
\usepackage{tkz-euclide}
\usetikzlibrary{calc,math}
\usepackage{listings}
    \usepackage{color}                                            %%
    \usepackage{array}                                            %%
    \usepackage{longtable}                                        %%
    \usepackage{calc}                                             %%
    \usepackage{multirow}                                         %%
    \usepackage{hhline}                                           %%
    \usepackage{ifthen}                                           %%
  %optionally (for landscape tables embedded in another document): %%
    \usepackage{lscape}     
\usepackage{multicol}
\usepackage{chngcntr}
\DeclareMathOperator*{\Res}{Res}
\renewcommand\thesection{\arabic{section}}
\renewcommand\thesubsection{\thesection.\arabic{subsection}}
\renewcommand\thesubsubsection{\thesubsection.\arabic{subsubsection}}

\renewcommand\thesectiondis{\arabic{section}}
\renewcommand\thesubsectiondis{\thesectiondis.\arabic{subsection}}
\renewcommand\thesubsubsectiondis{\thesubsectiondis.\arabic{subsubsection}}

% correct bad hyphenation here
\hyphenation{op-tical net-works semi-conduc-tor}
\def\inputGnumericTable{}                                 %%

\lstset{
frame=single, 
breaklines=true,
columns=fullflexible
}

\begin{document}


\newtheorem{theorem}{Theorem}[section]
\newtheorem{problem}{Problem}
\newtheorem{proposition}{Proposition}[section]
\newtheorem{lemma}{Lemma}[section]
\newtheorem{corollary}[theorem]{Corollary}
\newtheorem{example}{Example}[section]
\newtheorem{definition}[problem]{Definition}
\newcommand{\BEQA}{\begin{eqnarray}}
\newcommand{\EEQA}{\end{eqnarray}}
\newcommand{\define}{\stackrel{\triangle}{=}}

\bibliographystyle{IEEEtran}
\providecommand{\mbf}{\mathbf}
\providecommand{\pr}[1]{\ensuremath{\Pr\left(#1\right)}}
\providecommand{\qfunc}[1]{\ensuremath{Q\left(#1\right)}}
\providecommand{\sbrak}[1]{\ensuremath{{}\left[#1\right]}}
\providecommand{\lsbrak}[1]{\ensuremath{{}\left[#1\right.}}
\providecommand{\rsbrak}[1]{\ensuremath{{}\left.#1\right]}}
\providecommand{\brak}[1]{\ensuremath{\left(#1\right)}}
\providecommand{\lbrak}[1]{\ensuremath{\left(#1\right.}}
\providecommand{\rbrak}[1]{\ensuremath{\left.#1\right)}}
\providecommand{\cbrak}[1]{\ensuremath{\left\{#1\right\}}}
\providecommand{\lcbrak}[1]{\ensuremath{\left\{#1\right.}}
\providecommand{\rcbrak}[1]{\ensuremath{\left.#1\right\}}}
\theoremstyle{remark}
\newtheorem{rem}{Remark}
\newcommand{\sgn}{\mathop{\mathrm{sgn}}}
\providecommand{\abs}[1]{\left\vert#1\right\vert}
\providecommand{\res}[1]{\Res\displaylimits_{#1}} 
\providecommand{\norm}[1]{\left\lVert#1\right\rVert}
\providecommand{\mtx}[1]{\mathbf{#1}}
\providecommand{\mean}[1]{E\left[ #1 \right]}
\providecommand{\fourier}{\overset{\mathcal{F}}{ \rightleftharpoons}}
\providecommand{\system}{\overset{\mathcal{H}}{ \longleftrightarrow}}
\newcommand{\solution}{\noindent \textbf{Solution: }}
\newcommand{\cosec}{\,\text{cosec}\,}
\providecommand{\dec}[2]{\ensuremath{\overset{#1}{\underset{#2}{\gtrless}}}}
\newcommand{\myvec}[1]{\ensuremath{\begin{pmatrix}#1\end{pmatrix}}}
\newcommand{\mydet}[1]{\ensuremath{\begin{vmatrix}#1\end{vmatrix}}}
\numberwithin{equation}{subsection}
\makeatletter
\@addtoreset{figure}{problem}
\makeatother

\let\StandardTheFigure\thefigure
\let\vec\mathbf
\renewcommand{\thefigure}{\theproblem}



\def\putbox#1#2#3{\makebox[0in][l]{\makebox[#1][l]{}\raisebox{\baselineskip}[0in][0in]{\raisebox{#2}[0in][0in]{#3}}}}
     \def\rightbox#1{\makebox[0in][r]{#1}}
     \def\centbox#1{\makebox[0in]{#1}}
     \def\topbox#1{\raisebox{-\baselineskip}[0in][0in]{#1}}
     \def\midbox#1{\raisebox{-0.5\baselineskip}[0in][0in]{#1}}

\vspace{3cm}


\title{Assignment 1}
\author{Jaswanth Chowdary Madala}





% make the title area
\maketitle

\newpage

%\tableofcontents

\bigskip

\renewcommand{\thefigure}{\theenumi}
\renewcommand{\thetable}{\theenumi}


\begin{enumerate}
%\begin{figure}[ht]
%\centering
%\includegraphics[width = \columnwidth]{"./chapters/11/11/4/13/figs/fig.png"}
%\caption{Graph}
%\label{fig:chapters/11/11/4/13/1}
%\end{figure}

\textbf{Solution:}
\fi
The equation of the conic with focus $\vec{F}$, directrix $\vec{n}^\top\vec{x} = c$ and eccentricity $e$ is given by
\begin{align}
\vec{x}^\top\vec{V}\vec{x} + 2\vec{u}^\top\vec{x} + f = 0
\label{eq:chapters/11/11/4/13/1}
\end{align}
where
\begin{align}
\vec{V} &\triangleq \norm{\vec{n}}^2\vec{I} - e^2\vec{n}\vec{n}^\top \label{eq:chapters/11/11/4/13/2} \\
\vec{u} &\triangleq ce^2\vec{n} - \norm{\vec{n}}^2\vec{F} \label{eq:chapters/11/11/4/13/3} \\
f &\triangleq \norm{\vec{n}}^2\norm{\vec{F}}^2 - c^2e^2 \label{eq:chapters/11/11/4/13/4}
\end{align}
also
\begin{align}
f_0 &= \vec{u}^\top\vec{V}^{-1}\vec{u} - f\\
l &= 2\frac{\sqrt{\abs{f_0\lambda_2}}}{\lambda_1}
\end{align}

\begin{enumerate}
\item $\vec{n}$: Given that the conic has foci as
\begin{align}
\vec{F_1} &= \myvec{4\\0}\\
\vec{F_2} &= \myvec{-4\\0}
\end{align}
The direction vector of $F_1F_2$ is given by
\begin{align}
\vec{m} &= \vec{F_1} - \vec{F_2}\\
&= \myvec{1\\0}
\end{align}
Hence the normal to the directrix is given by,
\begin{align}
\vec{n} = \myvec{1\\0}
\end{align}

\item $\vec{u}$: The centre of the conic is given by
\begin{align}
\vec{c} &= \frac{\vec{F_1} + \vec{F_2}}{2}
\end{align}
\begin{align}
\vec{c} &= \myvec{0\\0}\\
\vec{c} &= -\vec{V}^{-1}\vec{u}
\label{eq:chapters/11/11/4/13/5}
\end{align}
Since $\vec{c} = \vec{0}$ and $\vec{V}^{-1} \neq \vec{0}$, it follows from \eqref{eq:chapters/11/11/4/13/5} that 
\begin{align}
\vec{u} = \myvec{0\\0}
\end{align}

From the above expressions we get
\begin{align}
\vec{V} &= \myvec{1-e^2&0\\0&1} \label{eq:chapters/11/11/4/13/6} \\
\vec{F} &= \myvec{ce^2\\0} \label{eq:chapters/11/11/4/13/7}\\
f &= c^2e^2\brak{e^2-1} \label{eq:chapters/11/11/4/13/8}\\
f_0 &= - f \label{eq:chapters/11/11/4/13/9}\\
l &= 2\frac{\sqrt{\abs{f\lambda_2}}}{\lambda_1}\label{eq:chapters/11/11/4/13/10}
\end{align}

From equation \eqref{eq:chapters/11/11/4/13/6} the eigen values of matrix $\vec{V}$ - $\lambda_1, \lambda_2$ are given by,
\begin{align}
\lambda_1 &= 1-e^2
\label{eq:chapters/11/11/4/13/11}\\
\lambda_2 &= 1
\label{eq:chapters/11/11/4/13/12}
\end{align}

From equation \eqref{eq:chapters/11/11/4/13/7} we get,
\begin{align}
ce^2 = 4 \label{eq:chapters/11/11/4/13/13}
\end{align}
\item Eccentricity: Given that the conic has the latus rectum length 12. Substituting the expressions of $\lambda_1,\lambda_2$ from the equations \eqref{eq:chapters/11/11/4/13/11}, \eqref{eq:chapters/11/11/4/13/12} in \eqref{eq:chapters/11/11/4/13/10} gives
\begin{align}
l = \frac{2ce}{\sqrt{e^2-1}} &= 12\\
\frac{ce}{\sqrt{e^2-1}} &= 6
\end{align}
Substitute the expression of $c$ from \eqref{eq:chapters/11/11/4/13/13} gives,
\begin{align}
\frac{4}{e\sqrt{e^2-1}} &= 6
\end{align}
Squaring on both sides gives,
\begin{align}
9e^2\brak{e^2-1} &= 4\\
9e^4-9e^2-4 &= 0
\label{eq:chapters/11/11/4/13/14}
\end{align}
The equation \eqref{eq:chapters/11/11/4/13/14} is a quadratic equation in $e^2$.
Solving it gives two roots one of which is negative, as $e^2$ is positive we have
\begin{align}
%e^2 &= \frac{-\brak{-9}\pm\sqrt{\brak{-9}^2-4\times9\times\brak{-4}}}{2\times9}\\
e^2 &= \frac{4}{3}
\end{align}
From equation \eqref{eq:chapters/11/11/4/13/8}, \eqref{eq:chapters/11/11/4/13/13}, we get
\begin{align}
f = 4
\end{align} 
\end{enumerate}
The equation of the conic is given by
\begin{align}
\vec{x}^\top\myvec{-\frac{1}{3}&0\\0&1}\vec{x} +4 = 0
\end{align}
\begin{figure}[ht]
\centering
\includegraphics[width = \columnwidth]{chapters/11/11/4/13/figs/fig1.png}
\caption{Graph}
\label{fig:chapters/11/11/4/13/1}
\end{figure}
\begin{table}[h]
\centering
\begin{tabular}{|c|c|c|}
  \hline
  \textbf{Symbol}&\textbf{Value}&\textbf{Description}\\
  \hline
  $a$ & 8 & $BC$\\
  \hline
	$\angle{B}$ & 45$\degree{}$ & $\angle{B}$ in $\triangle$$ABC$ \\
  \hline
	$k$ & 3.5 & $AB-AC$ i.e $c-b$ \\
  \hline 
	$\vec{e_2}$ & $\myvec{
			0\\
			1\\
			}$ & Basis vector\\
 \hline			
\end{tabular}

\caption{}
\label{tab:chapters/11/11/4/13/1}
\end{table}


\section{Intersection of Conics}
\subsection{Chords }

		\begin{theorem}[Chord]
  The points of intersection of the line 
\begin{align}
L: \quad \vec{x} = \vec{q} + \mu \vec{m} \quad \mu \in \mathbb{R}
\label{eq:conic_tangent}
\end{align}
with the conic section in \eqref{eq:conic_quad_form} are given by
\begin{align}
\vec{x}_i = \vec{q} + \mu_i \vec{m}
	\label{eq:chord-pts}
\end{align}
%
where
\begin{multline}
\mu_i = \frac{1}
{
\vec{m}^{\top}\vec{V}\vec{m}
}
\lbrak{-\vec{m}^{\top}\brak{\vec{V}\vec{q}+\vec{u}}}
\\
\pm
{\small
\rbrak{\sqrt{
\sbrak{
\vec{m}^{\top}\brak{\vec{V}\vec{q}+\vec{u}}
}^2
-
\brak
{
\vec{q}^{\top}\vec{V}\vec{q} + 2\vec{u}^{\top}\vec{q} +f
}
\brak{\vec{m}^{\top}\vec{V}\vec{m}}
}
}
}
\label{eq:tangent_roots}
\end{multline}


\end{theorem}
\begin{proof}
  Substituting \eqref{eq:conic_tangent}
in \eqref{eq:conic_quad_form}, 
\begin{align}
\brak{\vec{q} + \mu \vec{m}}^{\top}\vec{V}\brak{\vec{q} + \mu \vec{m}}  + 2 \vec{u}^{\top}\brak{\vec{q} + \mu \vec{m}}+f &= 0
\\
\implies \mu^2\vec{m}^{\top}\vec{V}\vec{m} + 2 \mu\vec{m}^{\top}\brak{\vec{V}\vec{q}+\vec{u}} 
+ \vec{q}^{\top}\vec{V}\vec{q} + 2\vec{u}^{\top}\vec{q} +f &= 0
\label{eq:conic_intercept}
\end{align}
Solving the above quadratic in \eqref{eq:conic_intercept}
yields \eqref{eq:tangent_roots}.
\end{proof}
\begin{corollary}
  If $L$ in \eqref{eq:conic_tangent} touches \eqref{eq:conic_quad_form} at exactly one point $\vec{q}$, 
  \begin{align}
  \vec{m}^{\top}\brak{\vec{V}\vec{q}+\vec{u}} = 0
  \label{eq:conic_tangent_mq}
  \end{align}
\end{corollary}
\begin{proof}
  In this case, \eqref{eq:conic_intercept} has exactly one root.  Hence, 
  in \eqref{eq:tangent_roots}
  \begin{align}
  \sbrak{
  \vec{m}^{\top}\brak{\vec{V}\vec{q}+\vec{u}}
  }^2 -\brak{\vec{m}^{\top}\vec{V}\vec{m}}
  \brak
  {
  \vec{q}^{\top}\vec{V}\vec{q} + 2\vec{u}^{\top}\vec{q} +f
  } = 0                                                                                             
  \label{eq:conic_tangent_disc}
  \end{align}                    
  $\because \vec{q}$ is the point of contact, $\vec{q}$ satisfies \eqref{eq:conic_quad_form}
  and 
  \begin{align}
  \vec{q}^{\top}\vec{V}\vec{q} + 2\vec{u}^{\top}\vec{q} +f = 0
  \label{eq:conic_tangent_qquad}
  \end{align}
  Substituting \eqref{eq:conic_tangent_qquad} in \eqref{eq:conic_tangent_disc} and simplifying, we obtain \eqref{eq:conic_tangent_mq}.
\end{proof}
	\begin{theorem}
		The length of the chord in 
\eqref{eq:conic_tangent}
is given by 
\begin{align}
 \frac{2\sqrt{
\sbrak{
\vec{m}^{\top}\brak{\vec{V}\vec{q}+\vec{u}}
}^2
-
\brak
{
\vec{q}^{\top}\vec{V}\vec{q} + 2\vec{u}^{\top}\vec{q} +f
}
\brak{\vec{m}^{\top}\vec{V}\vec{m}}
}
}
{
\vec{m}^{\top}\vec{V}\vec{m}
}\norm{\vec{m}}
\label{eq:chord-len}
  \end{align}
	\end{theorem}
\begin{proof}
The distance between the points in 
	\eqref{eq:chord-pts}
is given by 
\begin{align}
	\norm{\vec{x}_1-\vec{x}_2} =  \abs{\mu_1-\mu_2} \norm{\vec{m}}
\label{eq:conic_tangent_pts_dist}
\end{align}
Substituing $\mu_i$ from 
\eqref{eq:tangent_roots} in
\eqref{eq:conic_tangent_pts_dist}
yields
	\eqref{eq:chord-len}.
\end{proof}
	\begin{theorem}
 The affine transform for the conic section, preserves the norm.  This implies that the length of any chord of a conic
	is invariant to translation and/or rotation.
	\end{theorem}
	\begin{proof}
	Let 
%From \eqref{eq:conic_affine}, 
\begin{align}
\vec{x}_i = \vec{P}\vec{y}_i+\vec{c} 
\label{eq:conic_affine_pts}
\end{align}
be any two points on the conic.  Then the distance between the points is given by 
\begin{align}
	\norm{\vec{x}_1-\vec{x}_2 } &= \norm{\vec{P}\brak{	\vec{y}_1 -\vec{y}_2 }}
\end{align}
which can be expressed as 
\begin{align}
	\norm{\vec{x}_1-\vec{x}_2 }^2 &= 		\brak{\vec{y}_1 -\vec{y}_2 }^{\top}\vec{P}^{\top}\vec{P}\brak{\vec{y}_1 -\vec{y}_2 }
	\\
	&= 		\norm{\vec{y}_1 -\vec{y}_2 }^2
\label{eq:conic_affine_norm_preserve}
\end{align}
since 
\begin{align}
	\vec{P}^{\top}\vec{P} = \vec{I}
\end{align}
	\end{proof}
    \begin{corollary} For the standard hyperbola/ellipse, the length of the major axis is 
  \begin{align}
\label{eq:chord-len-major}
 2\sqrt{\abs{\frac{
f_0}
{\lambda_1}
	  }}
  \end{align}
  and the minor axis is 
  \begin{align}
\label{eq:chord-len-minor}
 2\sqrt{\abs{\frac{
f_0}
{\lambda_2}
	  }}
  \end{align}
%	    $\mydet{\vec{V}} \ne 0$, the lengths of the semi-major and semi-minor axes of the conic in \eqref{eq:conic_quad_form} are given by 
%  \begin{align} 
%    \label{eq:ellipse_axes}
%  %  \begin{aligned}[t]
%    \sqrt{\frac{\vec{u}^{\top}\vec{V}^{-1}\vec{u} -f}{\lambda_1}}, 
%    \sqrt{\frac{\vec{u}^{\top}\vec{V}^{-1}\vec{u} -f}{\lambda_2}}. \quad \brak{\text{ellipse}}
%    \\
%%
%       \sqrt{\frac{\vec{u}^{\top}\vec{V}^{-1}\vec{u} -f}{\lambda_1}}, 
%       \sqrt{\frac{f-\vec{u}^{\top}\vec{V}^{-1}\vec{u}}{\lambda_2}}, \quad \brak{\text{hyperbola }}
%%
%  %\end{aligned}
%  \label{eq:hyper_axes}
%\end{align} 
%\solution For \begin{align} \abs{\vec{V}} > 0, \quad \text{or, } \lambda_1 > 0, \lambda_2 > 0 
%  \end{align} and \eqref{eq:conic_simp_temp_nonparab} becomes 
%  \begin{align} 
%	  \lambda_1y_1^2 +\lambda_2y_2^2 = 
%  \vec{u}^{\top}\vec{V}^{-1}\vec{u} -f 
%	  \label{eq:hyper-pair-cond}
%  \end{align} 
%  yielding        \eqref{eq:ellipse_axes}.  Similarly, \eqref{eq:hyper_axes} can be obtained for 
%  \begin{align} 
%    \label{eq:conic_hyper_cond}
%    \abs{\vec{V}} 
%    < 0, \quad \text{or, } \lambda_1 > 0, \lambda_2 < 0 \end{align}
\end{corollary}
\begin{proof}
	See Appendix \ref{app:major}
\end{proof}
\begin{theorem}[latus rectum]
    The latus rectum of a conic section is the chord that passes through the focus and is perpendicular to the major axis.
	The length of the latus rectum for a conic is given by
		\begin{align}
			l =
			\begin{cases}
				2\frac{\sqrt{\abs{f_0\lambda_1}}}{\lambda_2} & e \ne 1
			\\
			\frac{\eta}{\lambda_2} & e = 1
			\end{cases}
			\label{eq:latus-ellipse}
		\end{align}
\end{theorem}
		\begin{proof}
			See Appendix \ref{app:latus}.
\end{proof}

\subsection{Curves}
\begin{enumerate}[label=\thesection.\arabic*,ref=\thesection.\theenumi]
\numberwithin{equation}{enumi}
\numberwithin{figure}{enumi}
\numberwithin{table}{enumi}

\item 
\label{chapters/12/8/2/1}

\documentclass[10pt,a4paper]{report}
%\usepackage[latin1]{inputenc}
\usepackage[utf8]{inputenc}
\usepackage{amsmath}
\usepackage{amsfonts}
\usepackage{amssymb}
\usepackage{graphicx}
\usepackage{multicol}
\usepackage{tabularx}
\usepackage{tikz}
\usetikzlibrary{arrows,shapes,automata,petri,positioning,calc}
\usepackage{hyperref}
\usepackage{tikz}
\usetikzlibrary{matrix,calc}
\usepackage[margin=0.5in]{geometry}
% ---- power functions -----% 
\newcommand{\myvec}[1]{\ensuremath{\begin{pmatrix}#1\end{pmatrix}}}
\let\vec\mathbf

\providecommand{\norm}[1]{\left\lVert#1\right\rVert}
\providecommand{\abs}[1]{\left\vert#1\right\vert}
\let\vec\mathbf

\newcommand{\mydet}[1]{\ensuremath{\begin{vmatrix}#1\end{vmatrix}}}
\providecommand{\brak}[1]{\ensuremath{\left(#1\right)}}
\providecommand{\lbrak}[1]{\ensuremath{\left(#1\right.}}
\providecommand{\rbrak}[1]{\ensuremath{\left.#1\right)}}
\providecommand{\sbrak}[1]{\ensuremath{{}\left[#1\right]}}
%-------end power functions----%
\newenvironment{Figure}
  {\par\medskip\noindent\minipage{\linewidth}}
  {\endminipage\par\medskip}
\begin{document}
%--------------------logo figure-------------------------%
\begin{figure*}[!tbp]
  \centering
  \begin{minipage}[b]{0.4\textwidth}
    \includegraphics[scale = 0.05]{iitlogo.jpg}
  \end{minipage}
  \hfill
  \vspace{5mm}\begin{minipage}[b]{0.4\textwidth}
\raggedleft  \includegraphics[scale = 0.10]{nrc.png}\

  \end{minipage}\vspace{0.2cm}
\end{figure*}
%--------------------name & rollno-----------------------
\raggedright \textbf{Name}:\hspace{1mm} Chirag Shah\hspace{3cm} \Large \textbf{Assignment-6}\hspace{2.5cm} % 
\normalsize \textbf{Roll No.} :\hspace{1mm} FWC22053\vspace{1cm}
\begin{multicols}{2}

%----------------problem statement--------------%
\raggedright \textbf{Problem Statement:}\vspace{2mm}
\raggedright \\ Find the area of the circle $4x^2+4y^2=9$ which is interior to the parabola $x^2=4y$.\\
\vspace{5mm}
%-----------------------------solution---------------------------
\raggedright \textbf{SOLUTION}:\vspace{2mm}\\

%---------given----------------%
\raggedright \textbf{Given}:\vspace{2mm}\\
Equation of circle is \\\vspace{1mm}
\begin{align}
4x^2+4y^2=9
\end{align}
Equation of Parabola is \\ \vspace{1mm}
\begin{align}
x^2=4y 
\end{align}
From (2) we can say that Parabola is concave towards positive y axis.\\ \vspace{2mm}
From equation (1) radius of circle is,\\ \vspace{1mm}
\begin{align}
r= \frac{3}{2}
\end{align}

%-------------To find ------------------%
\textbf{To Find }\vspace{2mm}\\
To find the intersection points and area of shaded region shown in figure\vspace{2mm}  \\ 
%--------------steps----------------------%
\textbf{STEP-1}\vspace{2mm}\\
The given circle and parabola can be expressed as conics with parameters,\\ \vspace{1mm}
For circle,\\ \vspace{1mm}
\begin{align}
\vec{V}_1=4\vec{I}
\end{align}
So, \\
\begin{align}
\vec{V}_1=\myvec{
4 & 0\\
0 & 4
}
\end{align} 

\begin{align}
\vec{u_1}=0
\end{align} 
\begin{align}
f_1=-9
\end{align} \vspace{2mm}

For Parabola,\\\vspace{1mm}
\begin{align}
\vec{V}_2=\myvec{
1 & 0\\
0 & 0
}
\end{align} 

\begin{align}
\vec{u_2}= -\myvec{
0\\
2
}
\end{align} 
\begin{align}
f_2=0
\end{align} \vspace{2mm}

\textbf{STEP-2}\vspace{2mm}\\
The intersection of the given conics is obtained
as\\
\begin{align}
	\vec{x}^{\top}\brak{\vec{V}_1 + \mu\vec{V}_2}\vec{x}+2 \brak{\vec{u}_1+\mu \vec{u}_2}^{\top} \vec{x} 
	\\
	+ \brak{f_1+\mu f_2}= 0
    \end{align}
    
\begin{align}
\vec{V}_1+\mu\vec{V}_2= \myvec{
\mu+4 & 0\\
0 & 4
}
\end{align}
\begin{align}
\vec{u}_1+\mu\vec{u}_2= -\myvec{
0\\
2\mu
}
\end{align}
\begin{align}
f_1+\mu f_2= -9
\end{align}
This conic is a single straight line if and only if, \\ \vspace{1mm}
\begin{align}
\mydet{\vec{V}_1 + \mu\vec{V}_2 & \vec{u}_1+\mu \vec{u}_2\\ \brak{\vec{u}_1+\mu \vec{u}_2}^{\top} & f_1 + \mu f_2} &= 0
\end{align}
And,\\
\begin{align}
\mydet{\vec{V}_1 + \mu\vec{V}_2} &= 0
\end{align}
Substituting equation (13),(14) and (15) in equation (16)\\ \vspace{1mm}
We get,\\ \vspace{1mm}
\begin{align}
\implies \mydet{\mu+4 & 0 & 0\\ 
0 & 4 & -2\mu \\
0 & -2\mu & -9
} &= 0
\end{align}
Solving the above equation we get,\\ \vspace{1mm}
\begin{align}
\mu^3 + 4\mu^2 + 9\mu + 36=0
\end{align}
gives,\\ \vspace{1mm}
\begin{align}
    \mu = -4
\end{align}
 Thus, the parameters for a straight line can be expressed as\\ \vspace{1mm}
 \begin{align}
	\vec{V} &= 
\vec{V}_1 + \mu\vec{V}_2
=\myvec{ 0 & 0 \\ 0 & 4},
\\
	\vec{u} &=
\vec{u}_1+\mu \vec{u}_2
	= \myvec{
0\\
8
    }
\\
	f&=-9,
	\\
	\implies \vec{D} &= \vec{V}, \vec{P} = \vec{I}
    \end{align}
Thus, the desired pair of straight lines are \\ 
\begin{align} 
	\myvec{\sqrt{\abs{\lambda_1}} & \pm \sqrt{\abs{\lambda_2}}}\vec{P}^{\top}\brak{\vec{x}-\vec{c}} &= 0
\end{align}
\begin{align}
	\implies\myvec{0 & \pm 2}\vec{x}-\vec{c} &= 0
\end{align}
\begin{align}
	\text{or, }\vec{x} =\vec{c} + \kappa \myvec{\pm 2 \\ 0}
\end{align} 
\textbf{STEP-3}\vspace{2mm}\\
The points of intersection of the line is given by, \\ 
\begin{align}
L: \quad \vec{x} = \vec{q} + \kappa \vec{m} \quad \kappa \in \mathbb{R}
\end{align}
with the conic section, \\ 
\begin{align}
	\vec{x}^{\top}\vec{V}\vec{x} + 2\vec{u}^{\top} \vec{x} + f = 0
\end{align}
are given by \\
\begin{align}
\vec{x}_i = \vec{q} + \kappa_i \vec{m}
\end{align}
where, \\
{\tiny
\begin{multline}
\kappa_i = \frac{1}
{
\vec{m}^T\vec{V}\vec{m}
}
\lbrak{-\vec{m}^T\brak{\vec{V}\vec{q}+\vec{u}}}
\\
\pm
\rbrak{\sqrt{
\sbrak{
\vec{m}^T\brak{\vec{V}\vec{q}+\vec{u}}
}^2
-
\brak
{
\vec{q}^T\vec{V}\vec{q} + 2\vec{u}^T\vec{q} +f
}
\brak{\vec{m}^T\vec{V}\vec{m}}
}
}
\end{multline}
}
On substituting\\
\begin{align}
\vec{q} &= \myvec{
0\\
0.5
} 
\end{align}
\begin{align}
\vec{m} = \myvec{2 \\ 0}
\end{align}
With the given Parabola,\\ 
\begin{align}
	\vec{V} &= \myvec{
1 & 0\\
0 & 0
    }
\end{align}
\begin{align}
	\vec{u} = -\myvec{2 \\0}
 \end{align}
 \begin{align}
  f = 0
 \end{align}
The value of $\kappa$ ,\\
\begin{align}
    \kappa = \sqrt{2},-\sqrt{2}
\end{align}
The points of intersection with Parabola along circle are \\
\begin{align}
    \vec{A}=\myvec{
\sqrt{2}\\
0.5
    }
\end{align}
\begin{align}
    \vec{B}=\myvec{
-\sqrt{2}\\
0.5
    }
\end{align}
\textbf{Result}
\begin{center}
 \includegraphics[width=0.5\textwidth]{conic.jpg}  
 \end{center}\vspace{1mm}
 From the figure,\\ \vspace{1mm}
Total area of portion is given by, \\ \vspace{1mm}
\begin{align}
 A=  \int_{-\sqrt{2}}^{\sqrt{2}} g(x)-f(x) \,dx 
\end{align}
Where g(x) is area of circle and f(x) is the area of parabola around the points\\ \vspace{1mm}
\begin{align}
A= \int_{-\sqrt{2}}^{\sqrt{2}} \frac{\sqrt{9-4x^2}}{2}-\frac{x^2}{4} \,dx 
\end{align}
Area A is,\\ 
\begin{align}
    A= 3.0053609 \,m^2
\end{align}
 \vspace{2mm} \textbf{Construction}
\begin{center}
\setlength{\arrayrulewidth}{0.5mm}
\setlength{\tabcolsep}{6pt}
\renewcommand{\arraystretch}{1.5}
    \begin{tabular}{|l|c|}
    \hline 
    \textbf{Points} & \textbf{coordinates} \\ \hline
   $\vec{A}$ & $\myvec{
   \sqrt{2}\\
   0.5
   } $ \\\hline
   $\vec{B}$ & $\myvec{
   -\sqrt{2}\\
   0.5
   } $ \\\hline
      \end{tabular}
  \end{center}

\raggedright  Download the code \\
Github link: \href{https://github.com/chiragshah1244/FWC/blob/main/assignments/assignment_6/code_conic/conic.py}{Assignment-6}.
  \end{multicols}
\end{document}

\item Find the area bounded by the curves $\brak{x-1}^2 + y^2 = 1 \text{ and } x^2+y^2=1$.
\label{chapters/12/8/2/2}
\\
\solution
\iffalse
\documentclass[journal,10pt,twocolumn]{article}
\usepackage{graphicx}
\usepackage[margin=0.5in]{geometry}
\usepackage[cmex10]{amsmath}
\usepackage{array}
\usepackage{booktabs}
\usepackage{mathtools}
\title{\textbf{Conic section Assignment}}
\author{Jyothsna Paluchuri}
\date{September 2022}


\providecommand{\norm}[1]{\left\lVert#1\right\rVert}
\providecommand{\abs}[1]{\left\vert#1\right\vert}
\let\vec\mathbf
\newcommand{\myvec}[1]{\ensuremath{\begin{pmatrix}#1\end{pmatrix}}}
\newcommand{\mydet}[1]{\ensuremath{\begin{vmatrix}#1\end{vmatrix}}}
\providecommand{\brak}[1]{\ensuremath{\left(#1\right)}}
\providecommand{\lbrak}[1]{\ensuremath{\left(#1\right.}}
\providecommand{\rbrak}[1]{\ensuremath{\left.#1\right)}}
\providecommand{\sbrak}[1]{\ensuremath{{}\left[#1\right]}}

\begin{document}

\maketitle
\paragraph{\textit{Problem Statement} -
\fi
Find the area of the region bounded by the curve $x^2=4y$ and the lines y=2 and y=4 and the y-axis in the first quadrant.
\\
\solution
	\begin{figure}[!h]
		\centering
 \includegraphics[width=\columnwidth]{chapters/12/8/3/3/figs/conic.png}
		\caption{}
		\label{fig:12/8/3/3}
  	\end{figure}
\iffalse

\section*{\large Solution}

\begin{figure}[h]
\centering
\includegraphics[width=1\columnwidth]{conic.png}

\caption{The parabola formed by the curve $x^2 = 4y$ and the lines y=2 and y=4}
\label{fig:parabola}
\end{figure}

The given equation of parabola $x^2 = 4y$ can be written in the general quadratic form as
\begin{align}
    \label{eq:conic_quad_form}
    \vec{x}^{\top}\vec{V}\vec{x}+2\vec{u}^{\top}\vec{x}+f=0
    \end{align}
where
\fi
The conic parameters are
\begin{align}
	\vec{V} = \myvec{1 & 0\\0 & 0},
	\vec{u} = \myvec{0\\-2},
	f = 0
	%\\
\end{align}
\iffalse
The point of intersection of the lines y=2 and y=4 to the parabola is given by



The points of intersection of the line 
\begin{align}
	L: \quad \vec{x} = \vec{q} + \mu \vec{m} \quad \mu \in \mathbf{R}
\label{eq:conic_tangent}
\end{align}
with the conic section are given by
\begin{align}
\vec{x}_i = \vec{q} + \mu_i \vec{m}
\label{eq:conic_tangent_pts}
\end{align}
%
where
{\tiny
\begin{multline}
\mu_i = \frac{1}
{
\vec{m}^T\vec{V}\vec{m}
}
\lbrak{-\vec{m}^T\brak{\vec{V}\vec{q}+\vec{u}}}
\\
\pm
\rbrak{\sqrt{
\sbrak{
\vec{m}^T\brak{\vec{V}\vec{q}+\vec{u}}
}^2
-
\brak
{
\vec{q}^T\vec{V}\vec{q} + 2\vec{u}^T\vec{q} +f
}
\brak{\vec{m}^T\vec{V}\vec{m}}
}
}
\label{eq:tangent_roots}
\end{multline}
}


\fi
The vector parameters of 
$y-4=0$
are
\begin{align}
	\vec{h}_1=\myvec{0\\4},
	\vec{m}_1=\myvec{1\\0}
\end{align}
Substituting the above in \eqref{eq:tangent_roots},
\begin{align}
\mu_i=4,-4
\end{align}
yielding
the points of intersection with the parabola as
\begin{align}
\vec{a}_0=\myvec{4\\4},
\vec{a}_1=\myvec{-4\\4}
\end{align}
Similarly, for 
the line $y-2=0$, the vector parameters are
\begin{align}
\vec{h}_2=\myvec{0\\2},
\vec{m}_2=\myvec{1\\0}
\end{align}
yielding 
\begin{align}
\mu_i=2.8,-2.8
\end{align}
and the points of intersection
\begin{align}
\vec{a}_2=\myvec{2.8\\2},
\vec{a}_3=\myvec{-2.8\\2}
\end{align}
From Fig.
		\ref{fig:12/8/3/3},
the area of the parabola between the lines $y=2$ and $y=4$ is given by
\begin{align}
\int_{0}^{4} \ 2\sqrt{y} \,dy-\int_{0}^{2} \ 2\sqrt{y} \,dy
=6.895 
\end{align}
\iffalse


\section*{\large Construction}

{
\setlength\extrarowheight{5pt}
\begin{tabular}{|l|c|}
    \hline 
    \textbf{Points} & \textbf{intersection points} \\ \hline
	a0 & $\myvec{
   -2.8\\
   2
   } $ \\\hline
	a1 & $\myvec{
   2.8\\
   2
   } $ \\\hline
    
	a3 & $\myvec{
   -4\\
   4
   } $ \\\hline
	a2 & $\myvec{
   4\\
   4
   } $ \\\hline
      
      \end{tabular}
}

\end{document}
\fi

\item 
\label{chapters/12/8/3/2}
\documentclass[10pt,a4paper]{report}
\usepackage[latin1]{inputenc}
\usepackage{amsmath}
\usepackage{amsfonts}
\usepackage{amssymb}
\usepackage{graphicx}
\usepackage{hyperref}
\usepackage{multicol}
\usepackage[margin=0.5in]{geometry}
\usepackage{tikz}
\usepackage[document]{ragged2e}
\usepackage{romannum}
\usetikzlibrary{arrows,shapes.gates.logic.US,shapes.gates.logic.IEC,calc}
\usepackage{titlesec}
\titlespacing{\subsection}{1pt}{\parskip}{3pt}
\titlespacing{\subsubsection}{0pt}{\parskip}{-\parskip}
\titlespacing{\paragraph}{0pt}{\parskip}{\parskip}
\newcommand{\myvec}[1]{\ensuremath{\begin{pmatrix}#1\end{pmatrix}}}
\let\vec\mathbf

\newcommand{\mydet}[1]{\ensuremath{\begin{vmatrix}#1\end{vmatrix}}}
\providecommand{\brak}[1]{\ensuremath{\left(#1\right)}}
\providecommand{\lbrak}[1]{\ensuremath{\left(#1\right.}}
\providecommand{\rbrak}[1]{\ensuremath{\left.#1\right)}}
\providecommand{\sbrak}[1]{\ensuremath{{}\left[#1\right]}}

\begin{document}

\begin{multicols}{2}
\raggedright {\includegraphics[scale=0.06]{iith_logo.png}} \vspace{3mm}\\ \raggedleft Hari Venkateswarlu Annam\vspace{2mm}\\ 
\raggedleft  FWC22058\vspace{2mm}\\ 
\raggedleft hariannam99@gmail.com \vspace{2mm}\\ 
\raggedleft Oct 2022 \vspace{5mm}\\
\end{multicols}

\centering \Large \textbf{MATRIX : CONIC ASSIGNMENT} \normalsize \vspace{10mm}

\begin{multicols}{2}

\section{Problem:}  
Find the area between the curves $y=x$ and $y=x^2$.

\section{Solution: }
\raggedright \textbf{Input Parameters :}\\ \vspace{2mm}
\centering Curve Equation : $y=x^2$. \\ \vspace{1mm}
Line Equation : $y=x$.\\
\vspace{3mm}

\raggedright \textbf{To Find :}\\ \vspace{2mm}
\begin{enumerate}
\item Comparing the given curve equation with the standard equation of the conics and finding it's parameters.
\item Finding the required parameters for the line equation.
\item Finding the Point of Intersection of the to the curve.
\item Finding the area between the curve.
\end{enumerate}

\raggedright \textbf{Step - 1 :}\\ \vspace{2mm}
Curve Equation : $y=x^2$. \\ \vspace{1mm}
The standard equation of the conics is given as :
\begin{align}
\vec{x}^{\top}\vec{V}\vec{x}+2\vec{u}^{\top}\vec{x}+f=0
\end{align}
The given curve  can be expressed as conics with \\parameters
\begin{align}
	\vec{V} &= \myvec{1 & 0\\0 & 0}, \vec{u} = \myvec{0 \\-\frac{1}{2}}, f = 0
	\end{align}

\raggedright \textbf{Step - 2 :}\\ \vspace{2mm}
Line Equation : $y=x$. \\ \vspace{1mm}
From the above line equation below vectors are taken
\begin{align}
\vec{q} = \myvec{0 \\0} , \vec{m}=\myvec{1\\1}
\end{align}

\raggedright \textbf{Step - 3 :}\\ \vspace{2mm}
The points of intersection of the line, \\ 
\begin{align}
L: \quad \vec{x} = \vec{q} + \mu \vec{m} \quad \mu \in \mathbb{R}
\end{align}
with the conic section, \\ 
\begin{align}
	\vec{x}^{\top}\vec{V}\vec{x} + 2\vec{u}^{\top} \vec{x} + f = 0
\end{align}
are given by \\
\begin{align}
\vec{x}_i = \vec{q} + \mu_i \vec{m}
\end{align}
where, \\
{\tiny
\begin{multline}
\mu_i = \frac{1}
{
\vec{m}^T\vec{V}\vec{m}
}
\lbrak{-\vec{m}^T\brak{\vec{V}\vec{q}+\vec{u}}}
\\
\pm
\rbrak{\sqrt{
\sbrak{
\vec{m}^T\brak{\vec{V}\vec{q}+\vec{u}}
}^2
-
\brak
{
\vec{q}^T\vec{V}\vec{q} + 2\vec{u}^T\vec{q} +f
}
\brak{\vec{m}^T\vec{V}\vec{m}}
}
}
\end{multline}
}
\raggedright On substituting $\vec{V},\vec{q} ,\vec{m}$ in the above equation,
we get the values of $\mu$. By substituting the values of $\mu$ in eq(6), \\we get the points of intersection of line with the given curve. \\
\centering $i.e., \vec{x_1},\vec{x_2}$\\ 

\begin{align}
\therefore \vec{x_1}=\myvec{0\\0} , \vec{x_2}=\myvec{1\\1}
\end{align}

\raggedright \textbf{Step - 4 :}\\ \vspace{2mm}
The area bounded by the curve $y=x^2$ and line $y=x$ is given by\\

\begin{align}
\implies A=\int_{0}^{1} [x] \,dx
\end{align}

\begin{align}
\implies A=\frac{x^2}{2} \
\end{align}

\centering By solving we get the required area\\
$\therefore A = \frac{1}{6}$ 



\section{Termux Commands :}
\centering bash rncom.sh ..... Using Shell commands.


\section{Plot :} 
\begin{center}
  \includegraphics[scale=0.55]{figure.png}
  	\end{center}

 
\end{multicols}
\end{document}

\item 
\label{chapters/12/8/3/18}
\iffalse
\documentclass[journal,10pt,twocolumn]{article}
\usepackage{graphicx}
\usepackage[margin=0.5in]{geometry}
\usepackage[cmex10]{amsmath}
\usepackage{array}
\usepackage{booktabs}
\usepackage{mathtools}
\title{\textbf{Conic section Assignment}}
\author{Jyothsna Paluchuri}
\date{September 2022}


\providecommand{\norm}[1]{\left\lVert#1\right\rVert}
\providecommand{\abs}[1]{\left\vert#1\right\vert}
\let\vec\mathbf
\newcommand{\myvec}[1]{\ensuremath{\begin{pmatrix}#1\end{pmatrix}}}
\newcommand{\mydet}[1]{\ensuremath{\begin{vmatrix}#1\end{vmatrix}}}
\providecommand{\brak}[1]{\ensuremath{\left(#1\right)}}
\providecommand{\lbrak}[1]{\ensuremath{\left(#1\right.}}
\providecommand{\rbrak}[1]{\ensuremath{\left.#1\right)}}
\providecommand{\sbrak}[1]{\ensuremath{{}\left[#1\right]}}

\begin{document}

\maketitle
\paragraph{\textit{Problem Statement} -
\fi
Find the area of the region bounded by the curve $x^2=4y$ and the lines y=2 and y=4 and the y-axis in the first quadrant.
\\
\solution
	\begin{figure}[!h]
		\centering
 \includegraphics[width=\columnwidth]{chapters/12/8/3/3/figs/conic.png}
		\caption{}
		\label{fig:12/8/3/3}
  	\end{figure}
\iffalse

\section*{\large Solution}

\begin{figure}[h]
\centering
\includegraphics[width=1\columnwidth]{conic.png}

\caption{The parabola formed by the curve $x^2 = 4y$ and the lines y=2 and y=4}
\label{fig:parabola}
\end{figure}

The given equation of parabola $x^2 = 4y$ can be written in the general quadratic form as
\begin{align}
    \label{eq:conic_quad_form}
    \vec{x}^{\top}\vec{V}\vec{x}+2\vec{u}^{\top}\vec{x}+f=0
    \end{align}
where
\fi
The conic parameters are
\begin{align}
	\vec{V} = \myvec{1 & 0\\0 & 0},
	\vec{u} = \myvec{0\\-2},
	f = 0
	%\\
\end{align}
\iffalse
The point of intersection of the lines y=2 and y=4 to the parabola is given by



The points of intersection of the line 
\begin{align}
	L: \quad \vec{x} = \vec{q} + \mu \vec{m} \quad \mu \in \mathbf{R}
\label{eq:conic_tangent}
\end{align}
with the conic section are given by
\begin{align}
\vec{x}_i = \vec{q} + \mu_i \vec{m}
\label{eq:conic_tangent_pts}
\end{align}
%
where
{\tiny
\begin{multline}
\mu_i = \frac{1}
{
\vec{m}^T\vec{V}\vec{m}
}
\lbrak{-\vec{m}^T\brak{\vec{V}\vec{q}+\vec{u}}}
\\
\pm
\rbrak{\sqrt{
\sbrak{
\vec{m}^T\brak{\vec{V}\vec{q}+\vec{u}}
}^2
-
\brak
{
\vec{q}^T\vec{V}\vec{q} + 2\vec{u}^T\vec{q} +f
}
\brak{\vec{m}^T\vec{V}\vec{m}}
}
}
\label{eq:tangent_roots}
\end{multline}
}


\fi
The vector parameters of 
$y-4=0$
are
\begin{align}
	\vec{h}_1=\myvec{0\\4},
	\vec{m}_1=\myvec{1\\0}
\end{align}
Substituting the above in \eqref{eq:tangent_roots},
\begin{align}
\mu_i=4,-4
\end{align}
yielding
the points of intersection with the parabola as
\begin{align}
\vec{a}_0=\myvec{4\\4},
\vec{a}_1=\myvec{-4\\4}
\end{align}
Similarly, for 
the line $y-2=0$, the vector parameters are
\begin{align}
\vec{h}_2=\myvec{0\\2},
\vec{m}_2=\myvec{1\\0}
\end{align}
yielding 
\begin{align}
\mu_i=2.8,-2.8
\end{align}
and the points of intersection
\begin{align}
\vec{a}_2=\myvec{2.8\\2},
\vec{a}_3=\myvec{-2.8\\2}
\end{align}
From Fig.
		\ref{fig:12/8/3/3},
the area of the parabola between the lines $y=2$ and $y=4$ is given by
\begin{align}
\int_{0}^{4} \ 2\sqrt{y} \,dy-\int_{0}^{2} \ 2\sqrt{y} \,dy
=6.895 
\end{align}
\iffalse


\section*{\large Construction}

{
\setlength\extrarowheight{5pt}
\begin{tabular}{|l|c|}
    \hline 
    \textbf{Points} & \textbf{intersection points} \\ \hline
	a0 & $\myvec{
   -2.8\\
   2
   } $ \\\hline
	a1 & $\myvec{
   2.8\\
   2
   } $ \\\hline
    
	a3 & $\myvec{
   -4\\
   4
   } $ \\\hline
	a2 & $\myvec{
   4\\
   4
   } $ \\\hline
      
      \end{tabular}
}

\end{document}
\fi

\end{enumerate}

%
\section{Tangent And Normal}
\subsection{Examples}
%\begin{enumerate}
\begin{enumerate}[label=\thesection.\arabic*.,ref=\thesection.\theenumi]
\item
  Given the point of contact $\vec{q}$, the equation of a tangent to \eqref{eq:conic_quad_form} is 
  \begin{align}
  \brak{\vec{V}\vec{q}+\vec{u}}^{\top}\vec{x}+\vec{u}^{\top}\vec{q}+f = 0
  \label{eq:conic_tangent_final}
  \end{align}

\begin{proof}
  The normal vector is obtained from \eqref{eq:conic_tangent_mq} and \eqref{eq:normal_vec}
  as
  %
  \begin{align}
  \label{eq:conic_normal_vec}
	  \kappa \vec{n} = \vec{V}\vec{q}+\vec{u}, \kappa \in \mathbb{R}
  \end{align}  
  From \eqref{eq:conic_normal_vec} and \eqref{eq:line_norm_eq}, the equation of the tangent is\begin{align}
    \brak{\vec{V}\vec{q}+\vec{u}}^{\top}\brak{\vec{x}-\vec{q}} &=0
    \\
    \implies \brak{\vec{V}\vec{q}+\vec{u}}^{\top}\vec{x}-\vec{q}^{\top}\vec{V}\vec{q}-\vec{u}^{\top}\vec{q} &= 0
    \end{align}
    which, upon substituting from \eqref{eq:conic_tangent_qquad} and simplifying yields 
  \eqref{eq:conic_tangent_final}
%	\eqref{eq:conic_tangent}.
\end{proof}
\item
	\label{eq:conic-p-contact-nonparab}
  If $\vec{V}^{-1}$ exists, given the normal vector $\vec{n}$, the tangent points of contact to \eqref{eq:conic_quad_form} are given by
\begin{align}
  \begin{split}
\vec{q}_i &= \vec{V}^{-1}\brak{\kappa_i \vec{n}-\vec{u}}, i = 1,2
\\
\text{where }\kappa_i &= \pm \sqrt{
\frac{
f_0
%\vec{u}^{\top}\vec{V}^{-1}\vec{u}-f
}
{
\vec{n}^{\top}\vec{V}^{-1}\vec{n}
}
}
  \end{split}
\label{eq:conic_tangent_qk}
\end{align}

\begin{proof}
  From \eqref{eq:conic_normal_vec},
\begin{align}
\label{eq:conic_normal_vec_q}
 \vec{q} = \vec{V}^{-1}\brak{\kappa \vec{n}-\vec{u}}, \quad \kappa \in \mathbb{R}
\end{align}
Substituting \eqref{eq:conic_normal_vec_q}
in \eqref{eq:conic_tangent_qquad},
\begin{align}
\brak{\kappa \vec{n}-\vec{u}}^{\top}\vec{V}^{-1}\brak{\kappa \vec{n}-\vec{u}} 
%\\
+ 2\vec{u}^{\top}\vec{V}^{-1}\brak{\kappa \vec{n}-\vec{u}} +f &= 0
\\
\implies 
\kappa^2 \vec{n}^{\top}\vec{V}^{-1}\vec{n} - \vec{u}^{\top}\vec{V}^{-1}\vec{u} + f &=0
 \\
 \text{or, } \kappa = \pm \sqrt{\frac{
	 %\vec{u}^{\top}\vec{V}^{-1}\vec{u}-f
	f_0 
 }{\vec{n}^{\top}\vec{V}^{-1}\vec{n}}} &
	\label{eq:conic_normal_k}
\end{align}
%
%yileding 
Substituting \eqref{eq:conic_normal_k} in \eqref{eq:conic_normal_vec_q}
yields \eqref{eq:conic_tangent_qk}.
%
\end{proof}


\item
	\label{eq:conic-p-contact-parab}
  If $\vec{V}$ is not invertible,  given the normal vector $\vec{n}$, the point of contact to \eqref{eq:conic_quad_form} is given by the matrix equation
\begin{align}
\label{eq:conic_tangent_q_eigen}
\begin{pmatrix}
\vec{\brak{u+\kappa \vec{n}}}^{\top} \\ \vec{V}
\end{pmatrix}
\vec{q} &= 
\begin{pmatrix}
-f
\\
\kappa\vec{n}-\vec{u}
\end{pmatrix}
\\
\text{where }  \kappa = \frac{\vec{p}_1^{\top}\vec{u}}{\vec{p}_1^{\top}\vec{n}}, \quad \vec{V}\vec{p}_1 &= 0
\label{eq:conic_tangent_qk_eigen}
\end{align}


\begin{proof}
  If $\vec{V}$ is non-invertible, it has a zero eigenvalue.  If the corresponding eigenvector is $\vec{p}_1$, then,
\begin{align}
\vec{V}\vec{p}_1 = 0
\label{eq:conic_zero_eigen}
\end{align}
From \eqref{eq:conic_normal_vec},
\begin{align}
\label{eq:conic_zero_eigen_normal}
\kappa \vec{n} &= \vec{V} \vec{q}+\vec{u}, \quad \kappa \in \mathbb{R}
\\
\implies \kappa \vec{p}_1^{\top}\vec{n} &= \vec{p}_1^{\top}\vec{V} \vec{q}+\vec{p}_1^{\top}\vec{u}
\\
\text{or, } \kappa \vec{p}_1^{\top}\vec{n} &= \vec{p}_1^{\top}\vec{u},  \quad \because \vec{p}_1^{\top} \vec{V} = 0, 
%\\
\quad 
\brak{\text{ from } \eqref{eq:conic_zero_eigen}}
%\label{eq:conic_normal_vec_q}
\end{align}
yielding $\kappa$ in \eqref{eq:conic_tangent_qk_eigen}. From \eqref{eq:conic_zero_eigen_normal},
\begin{align}
\kappa \vec{q}^{\top}\vec{n} &= \vec{q}^{\top}\vec{V} \vec{q}+\vec{q}^{\top}\vec{u}
\\
\implies \kappa \vec{q}^{\top}\vec{n} &= -f-\vec{q}^{\top}\vec{u} \quad \text{from } \eqref{eq:conic_tangent_qquad},
\\
\text{or, } \brak{\kappa \vec{n}+\vec{u}}^{\top}\vec{q} &= -f
\label{eq:conic_zero_eigen_normal_fq}
\end{align}
\eqref{eq:conic_zero_eigen_normal} can be expressed as
\begin{align}
\label{eq:conic_zero_eigen_normal_vq}
\vec{V} \vec{q} = \kappa \vec{n} - \vec{u}.
\end{align}
\eqref{eq:conic_zero_eigen_normal_fq} and \eqref{eq:conic_zero_eigen_normal_vq} clubbed together result in \eqref{eq:conic_tangent_q_eigen}.
\end{proof}
\item
	The normal vectors of the tangents  
	 to the conic in \eqref{eq:conic_quad_form} satisfy
\begin{align}
\vec{n} ^{\top}\vec{V}^{-1}\vec{n}-f_0 = 0
	\label{eq:dual-nf0}
    \end{align}

%
\begin{proof}
From 
  \eqref{eq:conic_tangent_mq}, the normal vector to  the tangent at $\vec{q}$ can be expressed as 
  \begin{align}
  \vec{n} &= \vec{V}\vec{q}+\vec{u} 
  \label{eq:conic_normal_n}
  \\
  \implies \vec{q} &= \vec{V}^{-1}\brak{\vec{n}-\vec{u} }
  \label{eq:conic_normal_q}
  \end{align}
  which upon substituting in \eqref{eq:conic_quad_form} yields
\begin{align}
    \label{eq:conic_quad_form_q}
    \brak{\vec{n}-\vec{u} }^{\top}\vec{V}^{-1}\vec{V}\vec{V}^{-1}\brak{\vec{n}-\vec{u} }+2\vec{u}^{\top}\vec{V}^{-1}\brak{\vec{n}-\vec{u} }+f&=0
	%\vec{u}^{\top}\vec{V}^{-1}\vec{u} +f&=0
    \end{align}
which can be simplified to obtain \eqref{eq:dual-nf0}.
\end{proof}
\item
	The normal vectors of the tangents 
to the conic in \eqref{eq:conic_quad_form} 
	from 
	a point $\vec{h}$ 
	are given by 

\begin{proof}
Let the equation of the tangent be 
\begin{align}
	\vec{n}^{\top}
	\vec{x} = c
	\label{eq:ext-tan}
\end{align}
If $\vec{q}$ be the point of contact,  since $\vec{h}, \vec{q}$ lie on 
	\eqref{eq:ext-tan},
\begin{align}
	\vec{n}^{\top}
	\vec{q} = 
	\vec{n}^{\top}
	\vec{h} = c
\end{align}
From 
  \eqref{eq:conic_normal_n}, 
%  \begin{align}
%	  \vec{n}^{\top}\vec{V}^{-1}\vec{n} &= \vec{n}^{\top}\brak{\vec{q}+\vec{V}^{-1}\vec{u} }
%	  \\
%%	  &= \vec{n}^{\top}\vec{q}+\vec{n}^{\top}\vec{V}^{-1}\vec{u} 
%\end{align}
\end{proof}
\item
	The normal vectors of the tangents 
to the conic in \eqref{eq:conic_quad_form} 
	from 
	a point $\vec{h}$ 
	are given by 
  \begin{align} 
  \label{eq:quad_form_pair_normvecs-sigma}
  \begin{split}
  \vec{n}_1 &= \vec{P}\myvec{\sqrt{\abs{\lambda_1}} \\[2mm]  \sqrt{\abs{\lambda_2}}}
  \\
  \vec{n}_2 &= \vec{P}\myvec{\sqrt{\abs{\lambda_1}} \\[2mm] - \sqrt{\abs{\lambda_2}}}
  \end{split}
  \end{align} 
  where $\lambda_i, \vec{P}$ are the eigenparameters of 
  \begin{align} 
		\bm{\Sigma} &= 
	   \brak{\vec{V}\vec{h}+\vec{u}}
	  \brak{\vec{V}\vec{h}+\vec{u}}^{\top}
   -\vec{V}
  \brak
  {
  \vec{h}^{\top}\vec{V}\vec{h} + 2\vec{u}^{\top}\vec{h} +f
  }.
	  \label{eq:h-tangents-sigma}
  \end{align}                    

\begin{proof}
 From \eqref{eq:tangent_roots},
 and
  \eqref{eq:conic_tangent_disc}
  \begin{align}
  \sbrak{
  \vec{m}^{\top}\brak{\vec{V}\vec{h}+\vec{u}}
  }^2 -\brak{\vec{m}^{\top}\vec{V}\vec{m}}
  \brak
  {
  \vec{h}^{\top}\vec{V}\vec{h} + 2\vec{u}^{\top}\vec{h} +f
  } &= 0                                                                                             
  \\
	  \implies 
  \vec{m}^{\top}  \sbrak{\brak{\vec{V}\vec{h}+\vec{u}}
	  \brak{\vec{V}\vec{h}+\vec{u}}^{\top}
   -\vec{V}
  \brak
  {
  \vec{h}^{\top}\vec{V}\vec{h} + 2\vec{u}^{\top}\vec{h} +f
  }}\vec{m} &= 0                                                                                             
  \label{eq:conic_tangent_disc-h}
  \end{align}                    
  yielding
	  \eqref{eq:h-tangents-sigma}.  Consequently, from 
  \eqref{eq:quad_form_pair_normvecs}, 
  \eqref{eq:quad_form_pair_normvecs-sigma}
  can be obtained.
\end{proof}
%
\end{enumerate}

\subsection{Exercises}
%\begin{enumerate}
\begin{enumerate}[label=\thesection.\arabic*.,ref=\thesection.\theenumi]
\item
  Given the point of contact $\vec{q}$, the equation of a tangent to \eqref{eq:conic_quad_form} is 
  \begin{align}
  \brak{\vec{V}\vec{q}+\vec{u}}^{\top}\vec{x}+\vec{u}^{\top}\vec{q}+f = 0
  \label{eq:conic_tangent_final}
  \end{align}

\begin{proof}
  The normal vector is obtained from \eqref{eq:conic_tangent_mq} and \eqref{eq:normal_vec}
  as
  %
  \begin{align}
  \label{eq:conic_normal_vec}
	  \kappa \vec{n} = \vec{V}\vec{q}+\vec{u}, \kappa \in \mathbb{R}
  \end{align}  
  From \eqref{eq:conic_normal_vec} and \eqref{eq:line_norm_eq}, the equation of the tangent is\begin{align}
    \brak{\vec{V}\vec{q}+\vec{u}}^{\top}\brak{\vec{x}-\vec{q}} &=0
    \\
    \implies \brak{\vec{V}\vec{q}+\vec{u}}^{\top}\vec{x}-\vec{q}^{\top}\vec{V}\vec{q}-\vec{u}^{\top}\vec{q} &= 0
    \end{align}
    which, upon substituting from \eqref{eq:conic_tangent_qquad} and simplifying yields 
  \eqref{eq:conic_tangent_final}
%	\eqref{eq:conic_tangent}.
\end{proof}
\item
	\label{eq:conic-p-contact-nonparab}
  If $\vec{V}^{-1}$ exists, given the normal vector $\vec{n}$, the tangent points of contact to \eqref{eq:conic_quad_form} are given by
\begin{align}
  \begin{split}
\vec{q}_i &= \vec{V}^{-1}\brak{\kappa_i \vec{n}-\vec{u}}, i = 1,2
\\
\text{where }\kappa_i &= \pm \sqrt{
\frac{
f_0
%\vec{u}^{\top}\vec{V}^{-1}\vec{u}-f
}
{
\vec{n}^{\top}\vec{V}^{-1}\vec{n}
}
}
  \end{split}
\label{eq:conic_tangent_qk}
\end{align}

\begin{proof}
  From \eqref{eq:conic_normal_vec},
\begin{align}
\label{eq:conic_normal_vec_q}
 \vec{q} = \vec{V}^{-1}\brak{\kappa \vec{n}-\vec{u}}, \quad \kappa \in \mathbb{R}
\end{align}
Substituting \eqref{eq:conic_normal_vec_q}
in \eqref{eq:conic_tangent_qquad},
\begin{align}
\brak{\kappa \vec{n}-\vec{u}}^{\top}\vec{V}^{-1}\brak{\kappa \vec{n}-\vec{u}} 
%\\
+ 2\vec{u}^{\top}\vec{V}^{-1}\brak{\kappa \vec{n}-\vec{u}} +f &= 0
\\
\implies 
\kappa^2 \vec{n}^{\top}\vec{V}^{-1}\vec{n} - \vec{u}^{\top}\vec{V}^{-1}\vec{u} + f &=0
 \\
 \text{or, } \kappa = \pm \sqrt{\frac{
	 %\vec{u}^{\top}\vec{V}^{-1}\vec{u}-f
	f_0 
 }{\vec{n}^{\top}\vec{V}^{-1}\vec{n}}} &
	\label{eq:conic_normal_k}
\end{align}
%
%yileding 
Substituting \eqref{eq:conic_normal_k} in \eqref{eq:conic_normal_vec_q}
yields \eqref{eq:conic_tangent_qk}.
%
\end{proof}


\item
	\label{eq:conic-p-contact-parab}
  If $\vec{V}$ is not invertible,  given the normal vector $\vec{n}$, the point of contact to \eqref{eq:conic_quad_form} is given by the matrix equation
\begin{align}
\label{eq:conic_tangent_q_eigen}
\begin{pmatrix}
\vec{\brak{u+\kappa \vec{n}}}^{\top} \\ \vec{V}
\end{pmatrix}
\vec{q} &= 
\begin{pmatrix}
-f
\\
\kappa\vec{n}-\vec{u}
\end{pmatrix}
\\
\text{where }  \kappa = \frac{\vec{p}_1^{\top}\vec{u}}{\vec{p}_1^{\top}\vec{n}}, \quad \vec{V}\vec{p}_1 &= 0
\label{eq:conic_tangent_qk_eigen}
\end{align}


\begin{proof}
  If $\vec{V}$ is non-invertible, it has a zero eigenvalue.  If the corresponding eigenvector is $\vec{p}_1$, then,
\begin{align}
\vec{V}\vec{p}_1 = 0
\label{eq:conic_zero_eigen}
\end{align}
From \eqref{eq:conic_normal_vec},
\begin{align}
\label{eq:conic_zero_eigen_normal}
\kappa \vec{n} &= \vec{V} \vec{q}+\vec{u}, \quad \kappa \in \mathbb{R}
\\
\implies \kappa \vec{p}_1^{\top}\vec{n} &= \vec{p}_1^{\top}\vec{V} \vec{q}+\vec{p}_1^{\top}\vec{u}
\\
\text{or, } \kappa \vec{p}_1^{\top}\vec{n} &= \vec{p}_1^{\top}\vec{u},  \quad \because \vec{p}_1^{\top} \vec{V} = 0, 
%\\
\quad 
\brak{\text{ from } \eqref{eq:conic_zero_eigen}}
%\label{eq:conic_normal_vec_q}
\end{align}
yielding $\kappa$ in \eqref{eq:conic_tangent_qk_eigen}. From \eqref{eq:conic_zero_eigen_normal},
\begin{align}
\kappa \vec{q}^{\top}\vec{n} &= \vec{q}^{\top}\vec{V} \vec{q}+\vec{q}^{\top}\vec{u}
\\
\implies \kappa \vec{q}^{\top}\vec{n} &= -f-\vec{q}^{\top}\vec{u} \quad \text{from } \eqref{eq:conic_tangent_qquad},
\\
\text{or, } \brak{\kappa \vec{n}+\vec{u}}^{\top}\vec{q} &= -f
\label{eq:conic_zero_eigen_normal_fq}
\end{align}
\eqref{eq:conic_zero_eigen_normal} can be expressed as
\begin{align}
\label{eq:conic_zero_eigen_normal_vq}
\vec{V} \vec{q} = \kappa \vec{n} - \vec{u}.
\end{align}
\eqref{eq:conic_zero_eigen_normal_fq} and \eqref{eq:conic_zero_eigen_normal_vq} clubbed together result in \eqref{eq:conic_tangent_q_eigen}.
\end{proof}
\item
	The normal vectors of the tangents  
	 to the conic in \eqref{eq:conic_quad_form} satisfy
\begin{align}
\vec{n} ^{\top}\vec{V}^{-1}\vec{n}-f_0 = 0
	\label{eq:dual-nf0}
    \end{align}

%
\begin{proof}
From 
  \eqref{eq:conic_tangent_mq}, the normal vector to  the tangent at $\vec{q}$ can be expressed as 
  \begin{align}
  \vec{n} &= \vec{V}\vec{q}+\vec{u} 
  \label{eq:conic_normal_n}
  \\
  \implies \vec{q} &= \vec{V}^{-1}\brak{\vec{n}-\vec{u} }
  \label{eq:conic_normal_q}
  \end{align}
  which upon substituting in \eqref{eq:conic_quad_form} yields
\begin{align}
    \label{eq:conic_quad_form_q}
    \brak{\vec{n}-\vec{u} }^{\top}\vec{V}^{-1}\vec{V}\vec{V}^{-1}\brak{\vec{n}-\vec{u} }+2\vec{u}^{\top}\vec{V}^{-1}\brak{\vec{n}-\vec{u} }+f&=0
	%\vec{u}^{\top}\vec{V}^{-1}\vec{u} +f&=0
    \end{align}
which can be simplified to obtain \eqref{eq:dual-nf0}.
\end{proof}
\item
	The normal vectors of the tangents 
to the conic in \eqref{eq:conic_quad_form} 
	from 
	a point $\vec{h}$ 
	are given by 

\begin{proof}
Let the equation of the tangent be 
\begin{align}
	\vec{n}^{\top}
	\vec{x} = c
	\label{eq:ext-tan}
\end{align}
If $\vec{q}$ be the point of contact,  since $\vec{h}, \vec{q}$ lie on 
	\eqref{eq:ext-tan},
\begin{align}
	\vec{n}^{\top}
	\vec{q} = 
	\vec{n}^{\top}
	\vec{h} = c
\end{align}
From 
  \eqref{eq:conic_normal_n}, 
%  \begin{align}
%	  \vec{n}^{\top}\vec{V}^{-1}\vec{n} &= \vec{n}^{\top}\brak{\vec{q}+\vec{V}^{-1}\vec{u} }
%	  \\
%%	  &= \vec{n}^{\top}\vec{q}+\vec{n}^{\top}\vec{V}^{-1}\vec{u} 
%\end{align}
\end{proof}
\item
	The normal vectors of the tangents 
to the conic in \eqref{eq:conic_quad_form} 
	from 
	a point $\vec{h}$ 
	are given by 
  \begin{align} 
  \label{eq:quad_form_pair_normvecs-sigma}
  \begin{split}
  \vec{n}_1 &= \vec{P}\myvec{\sqrt{\abs{\lambda_1}} \\[2mm]  \sqrt{\abs{\lambda_2}}}
  \\
  \vec{n}_2 &= \vec{P}\myvec{\sqrt{\abs{\lambda_1}} \\[2mm] - \sqrt{\abs{\lambda_2}}}
  \end{split}
  \end{align} 
  where $\lambda_i, \vec{P}$ are the eigenparameters of 
  \begin{align} 
		\bm{\Sigma} &= 
	   \brak{\vec{V}\vec{h}+\vec{u}}
	  \brak{\vec{V}\vec{h}+\vec{u}}^{\top}
   -\vec{V}
  \brak
  {
  \vec{h}^{\top}\vec{V}\vec{h} + 2\vec{u}^{\top}\vec{h} +f
  }.
	  \label{eq:h-tangents-sigma}
  \end{align}                    

\begin{proof}
 From \eqref{eq:tangent_roots},
 and
  \eqref{eq:conic_tangent_disc}
  \begin{align}
  \sbrak{
  \vec{m}^{\top}\brak{\vec{V}\vec{h}+\vec{u}}
  }^2 -\brak{\vec{m}^{\top}\vec{V}\vec{m}}
  \brak
  {
  \vec{h}^{\top}\vec{V}\vec{h} + 2\vec{u}^{\top}\vec{h} +f
  } &= 0                                                                                             
  \\
	  \implies 
  \vec{m}^{\top}  \sbrak{\brak{\vec{V}\vec{h}+\vec{u}}
	  \brak{\vec{V}\vec{h}+\vec{u}}^{\top}
   -\vec{V}
  \brak
  {
  \vec{h}^{\top}\vec{V}\vec{h} + 2\vec{u}^{\top}\vec{h} +f
  }}\vec{m} &= 0                                                                                             
  \label{eq:conic_tangent_disc-h}
  \end{align}                    
  yielding
	  \eqref{eq:h-tangents-sigma}.  Consequently, from 
  \eqref{eq:quad_form_pair_normvecs}, 
  \eqref{eq:quad_form_pair_normvecs-sigma}
  can be obtained.
\end{proof}
%
\end{enumerate}

\subsection{Construction}
%\begin{enumerate}
\begin{enumerate}[label=\thesection.\arabic*.,ref=\thesection.\theenumi]
\item
  Given the point of contact $\vec{q}$, the equation of a tangent to \eqref{eq:conic_quad_form} is 
  \begin{align}
  \brak{\vec{V}\vec{q}+\vec{u}}^{\top}\vec{x}+\vec{u}^{\top}\vec{q}+f = 0
  \label{eq:conic_tangent_final}
  \end{align}

\begin{proof}
  The normal vector is obtained from \eqref{eq:conic_tangent_mq} and \eqref{eq:normal_vec}
  as
  %
  \begin{align}
  \label{eq:conic_normal_vec}
	  \kappa \vec{n} = \vec{V}\vec{q}+\vec{u}, \kappa \in \mathbb{R}
  \end{align}  
  From \eqref{eq:conic_normal_vec} and \eqref{eq:line_norm_eq}, the equation of the tangent is\begin{align}
    \brak{\vec{V}\vec{q}+\vec{u}}^{\top}\brak{\vec{x}-\vec{q}} &=0
    \\
    \implies \brak{\vec{V}\vec{q}+\vec{u}}^{\top}\vec{x}-\vec{q}^{\top}\vec{V}\vec{q}-\vec{u}^{\top}\vec{q} &= 0
    \end{align}
    which, upon substituting from \eqref{eq:conic_tangent_qquad} and simplifying yields 
  \eqref{eq:conic_tangent_final}
%	\eqref{eq:conic_tangent}.
\end{proof}
\item
	\label{eq:conic-p-contact-nonparab}
  If $\vec{V}^{-1}$ exists, given the normal vector $\vec{n}$, the tangent points of contact to \eqref{eq:conic_quad_form} are given by
\begin{align}
  \begin{split}
\vec{q}_i &= \vec{V}^{-1}\brak{\kappa_i \vec{n}-\vec{u}}, i = 1,2
\\
\text{where }\kappa_i &= \pm \sqrt{
\frac{
f_0
%\vec{u}^{\top}\vec{V}^{-1}\vec{u}-f
}
{
\vec{n}^{\top}\vec{V}^{-1}\vec{n}
}
}
  \end{split}
\label{eq:conic_tangent_qk}
\end{align}

\begin{proof}
  From \eqref{eq:conic_normal_vec},
\begin{align}
\label{eq:conic_normal_vec_q}
 \vec{q} = \vec{V}^{-1}\brak{\kappa \vec{n}-\vec{u}}, \quad \kappa \in \mathbb{R}
\end{align}
Substituting \eqref{eq:conic_normal_vec_q}
in \eqref{eq:conic_tangent_qquad},
\begin{align}
\brak{\kappa \vec{n}-\vec{u}}^{\top}\vec{V}^{-1}\brak{\kappa \vec{n}-\vec{u}} 
%\\
+ 2\vec{u}^{\top}\vec{V}^{-1}\brak{\kappa \vec{n}-\vec{u}} +f &= 0
\\
\implies 
\kappa^2 \vec{n}^{\top}\vec{V}^{-1}\vec{n} - \vec{u}^{\top}\vec{V}^{-1}\vec{u} + f &=0
 \\
 \text{or, } \kappa = \pm \sqrt{\frac{
	 %\vec{u}^{\top}\vec{V}^{-1}\vec{u}-f
	f_0 
 }{\vec{n}^{\top}\vec{V}^{-1}\vec{n}}} &
	\label{eq:conic_normal_k}
\end{align}
%
%yileding 
Substituting \eqref{eq:conic_normal_k} in \eqref{eq:conic_normal_vec_q}
yields \eqref{eq:conic_tangent_qk}.
%
\end{proof}


\item
	\label{eq:conic-p-contact-parab}
  If $\vec{V}$ is not invertible,  given the normal vector $\vec{n}$, the point of contact to \eqref{eq:conic_quad_form} is given by the matrix equation
\begin{align}
\label{eq:conic_tangent_q_eigen}
\begin{pmatrix}
\vec{\brak{u+\kappa \vec{n}}}^{\top} \\ \vec{V}
\end{pmatrix}
\vec{q} &= 
\begin{pmatrix}
-f
\\
\kappa\vec{n}-\vec{u}
\end{pmatrix}
\\
\text{where }  \kappa = \frac{\vec{p}_1^{\top}\vec{u}}{\vec{p}_1^{\top}\vec{n}}, \quad \vec{V}\vec{p}_1 &= 0
\label{eq:conic_tangent_qk_eigen}
\end{align}


\begin{proof}
  If $\vec{V}$ is non-invertible, it has a zero eigenvalue.  If the corresponding eigenvector is $\vec{p}_1$, then,
\begin{align}
\vec{V}\vec{p}_1 = 0
\label{eq:conic_zero_eigen}
\end{align}
From \eqref{eq:conic_normal_vec},
\begin{align}
\label{eq:conic_zero_eigen_normal}
\kappa \vec{n} &= \vec{V} \vec{q}+\vec{u}, \quad \kappa \in \mathbb{R}
\\
\implies \kappa \vec{p}_1^{\top}\vec{n} &= \vec{p}_1^{\top}\vec{V} \vec{q}+\vec{p}_1^{\top}\vec{u}
\\
\text{or, } \kappa \vec{p}_1^{\top}\vec{n} &= \vec{p}_1^{\top}\vec{u},  \quad \because \vec{p}_1^{\top} \vec{V} = 0, 
%\\
\quad 
\brak{\text{ from } \eqref{eq:conic_zero_eigen}}
%\label{eq:conic_normal_vec_q}
\end{align}
yielding $\kappa$ in \eqref{eq:conic_tangent_qk_eigen}. From \eqref{eq:conic_zero_eigen_normal},
\begin{align}
\kappa \vec{q}^{\top}\vec{n} &= \vec{q}^{\top}\vec{V} \vec{q}+\vec{q}^{\top}\vec{u}
\\
\implies \kappa \vec{q}^{\top}\vec{n} &= -f-\vec{q}^{\top}\vec{u} \quad \text{from } \eqref{eq:conic_tangent_qquad},
\\
\text{or, } \brak{\kappa \vec{n}+\vec{u}}^{\top}\vec{q} &= -f
\label{eq:conic_zero_eigen_normal_fq}
\end{align}
\eqref{eq:conic_zero_eigen_normal} can be expressed as
\begin{align}
\label{eq:conic_zero_eigen_normal_vq}
\vec{V} \vec{q} = \kappa \vec{n} - \vec{u}.
\end{align}
\eqref{eq:conic_zero_eigen_normal_fq} and \eqref{eq:conic_zero_eigen_normal_vq} clubbed together result in \eqref{eq:conic_tangent_q_eigen}.
\end{proof}
\item
	The normal vectors of the tangents  
	 to the conic in \eqref{eq:conic_quad_form} satisfy
\begin{align}
\vec{n} ^{\top}\vec{V}^{-1}\vec{n}-f_0 = 0
	\label{eq:dual-nf0}
    \end{align}

%
\begin{proof}
From 
  \eqref{eq:conic_tangent_mq}, the normal vector to  the tangent at $\vec{q}$ can be expressed as 
  \begin{align}
  \vec{n} &= \vec{V}\vec{q}+\vec{u} 
  \label{eq:conic_normal_n}
  \\
  \implies \vec{q} &= \vec{V}^{-1}\brak{\vec{n}-\vec{u} }
  \label{eq:conic_normal_q}
  \end{align}
  which upon substituting in \eqref{eq:conic_quad_form} yields
\begin{align}
    \label{eq:conic_quad_form_q}
    \brak{\vec{n}-\vec{u} }^{\top}\vec{V}^{-1}\vec{V}\vec{V}^{-1}\brak{\vec{n}-\vec{u} }+2\vec{u}^{\top}\vec{V}^{-1}\brak{\vec{n}-\vec{u} }+f&=0
	%\vec{u}^{\top}\vec{V}^{-1}\vec{u} +f&=0
    \end{align}
which can be simplified to obtain \eqref{eq:dual-nf0}.
\end{proof}
\item
	The normal vectors of the tangents 
to the conic in \eqref{eq:conic_quad_form} 
	from 
	a point $\vec{h}$ 
	are given by 

\begin{proof}
Let the equation of the tangent be 
\begin{align}
	\vec{n}^{\top}
	\vec{x} = c
	\label{eq:ext-tan}
\end{align}
If $\vec{q}$ be the point of contact,  since $\vec{h}, \vec{q}$ lie on 
	\eqref{eq:ext-tan},
\begin{align}
	\vec{n}^{\top}
	\vec{q} = 
	\vec{n}^{\top}
	\vec{h} = c
\end{align}
From 
  \eqref{eq:conic_normal_n}, 
%  \begin{align}
%	  \vec{n}^{\top}\vec{V}^{-1}\vec{n} &= \vec{n}^{\top}\brak{\vec{q}+\vec{V}^{-1}\vec{u} }
%	  \\
%%	  &= \vec{n}^{\top}\vec{q}+\vec{n}^{\top}\vec{V}^{-1}\vec{u} 
%\end{align}
\end{proof}
\item
	The normal vectors of the tangents 
to the conic in \eqref{eq:conic_quad_form} 
	from 
	a point $\vec{h}$ 
	are given by 
  \begin{align} 
  \label{eq:quad_form_pair_normvecs-sigma}
  \begin{split}
  \vec{n}_1 &= \vec{P}\myvec{\sqrt{\abs{\lambda_1}} \\[2mm]  \sqrt{\abs{\lambda_2}}}
  \\
  \vec{n}_2 &= \vec{P}\myvec{\sqrt{\abs{\lambda_1}} \\[2mm] - \sqrt{\abs{\lambda_2}}}
  \end{split}
  \end{align} 
  where $\lambda_i, \vec{P}$ are the eigenparameters of 
  \begin{align} 
		\bm{\Sigma} &= 
	   \brak{\vec{V}\vec{h}+\vec{u}}
	  \brak{\vec{V}\vec{h}+\vec{u}}^{\top}
   -\vec{V}
  \brak
  {
  \vec{h}^{\top}\vec{V}\vec{h} + 2\vec{u}^{\top}\vec{h} +f
  }.
	  \label{eq:h-tangents-sigma}
  \end{align}                    

\begin{proof}
 From \eqref{eq:tangent_roots},
 and
  \eqref{eq:conic_tangent_disc}
  \begin{align}
  \sbrak{
  \vec{m}^{\top}\brak{\vec{V}\vec{h}+\vec{u}}
  }^2 -\brak{\vec{m}^{\top}\vec{V}\vec{m}}
  \brak
  {
  \vec{h}^{\top}\vec{V}\vec{h} + 2\vec{u}^{\top}\vec{h} +f
  } &= 0                                                                                             
  \\
	  \implies 
  \vec{m}^{\top}  \sbrak{\brak{\vec{V}\vec{h}+\vec{u}}
	  \brak{\vec{V}\vec{h}+\vec{u}}^{\top}
   -\vec{V}
  \brak
  {
  \vec{h}^{\top}\vec{V}\vec{h} + 2\vec{u}^{\top}\vec{h} +f
  }}\vec{m} &= 0                                                                                             
  \label{eq:conic_tangent_disc-h}
  \end{align}                    
  yielding
	  \eqref{eq:h-tangents-sigma}.  Consequently, from 
  \eqref{eq:quad_form_pair_normvecs}, 
  \eqref{eq:quad_form_pair_normvecs-sigma}
  can be obtained.
\end{proof}
%
\end{enumerate}

\subsection{Exercises}
%\begin{enumerate}
\begin{enumerate}[label=\thesection.\arabic*.,ref=\thesection.\theenumi]
\item
  Given the point of contact $\vec{q}$, the equation of a tangent to \eqref{eq:conic_quad_form} is 
  \begin{align}
  \brak{\vec{V}\vec{q}+\vec{u}}^{\top}\vec{x}+\vec{u}^{\top}\vec{q}+f = 0
  \label{eq:conic_tangent_final}
  \end{align}

\begin{proof}
  The normal vector is obtained from \eqref{eq:conic_tangent_mq} and \eqref{eq:normal_vec}
  as
  %
  \begin{align}
  \label{eq:conic_normal_vec}
	  \kappa \vec{n} = \vec{V}\vec{q}+\vec{u}, \kappa \in \mathbb{R}
  \end{align}  
  From \eqref{eq:conic_normal_vec} and \eqref{eq:line_norm_eq}, the equation of the tangent is\begin{align}
    \brak{\vec{V}\vec{q}+\vec{u}}^{\top}\brak{\vec{x}-\vec{q}} &=0
    \\
    \implies \brak{\vec{V}\vec{q}+\vec{u}}^{\top}\vec{x}-\vec{q}^{\top}\vec{V}\vec{q}-\vec{u}^{\top}\vec{q} &= 0
    \end{align}
    which, upon substituting from \eqref{eq:conic_tangent_qquad} and simplifying yields 
  \eqref{eq:conic_tangent_final}
%	\eqref{eq:conic_tangent}.
\end{proof}
\item
	\label{eq:conic-p-contact-nonparab}
  If $\vec{V}^{-1}$ exists, given the normal vector $\vec{n}$, the tangent points of contact to \eqref{eq:conic_quad_form} are given by
\begin{align}
  \begin{split}
\vec{q}_i &= \vec{V}^{-1}\brak{\kappa_i \vec{n}-\vec{u}}, i = 1,2
\\
\text{where }\kappa_i &= \pm \sqrt{
\frac{
f_0
%\vec{u}^{\top}\vec{V}^{-1}\vec{u}-f
}
{
\vec{n}^{\top}\vec{V}^{-1}\vec{n}
}
}
  \end{split}
\label{eq:conic_tangent_qk}
\end{align}

\begin{proof}
  From \eqref{eq:conic_normal_vec},
\begin{align}
\label{eq:conic_normal_vec_q}
 \vec{q} = \vec{V}^{-1}\brak{\kappa \vec{n}-\vec{u}}, \quad \kappa \in \mathbb{R}
\end{align}
Substituting \eqref{eq:conic_normal_vec_q}
in \eqref{eq:conic_tangent_qquad},
\begin{align}
\brak{\kappa \vec{n}-\vec{u}}^{\top}\vec{V}^{-1}\brak{\kappa \vec{n}-\vec{u}} 
%\\
+ 2\vec{u}^{\top}\vec{V}^{-1}\brak{\kappa \vec{n}-\vec{u}} +f &= 0
\\
\implies 
\kappa^2 \vec{n}^{\top}\vec{V}^{-1}\vec{n} - \vec{u}^{\top}\vec{V}^{-1}\vec{u} + f &=0
 \\
 \text{or, } \kappa = \pm \sqrt{\frac{
	 %\vec{u}^{\top}\vec{V}^{-1}\vec{u}-f
	f_0 
 }{\vec{n}^{\top}\vec{V}^{-1}\vec{n}}} &
	\label{eq:conic_normal_k}
\end{align}
%
%yileding 
Substituting \eqref{eq:conic_normal_k} in \eqref{eq:conic_normal_vec_q}
yields \eqref{eq:conic_tangent_qk}.
%
\end{proof}


\item
	\label{eq:conic-p-contact-parab}
  If $\vec{V}$ is not invertible,  given the normal vector $\vec{n}$, the point of contact to \eqref{eq:conic_quad_form} is given by the matrix equation
\begin{align}
\label{eq:conic_tangent_q_eigen}
\begin{pmatrix}
\vec{\brak{u+\kappa \vec{n}}}^{\top} \\ \vec{V}
\end{pmatrix}
\vec{q} &= 
\begin{pmatrix}
-f
\\
\kappa\vec{n}-\vec{u}
\end{pmatrix}
\\
\text{where }  \kappa = \frac{\vec{p}_1^{\top}\vec{u}}{\vec{p}_1^{\top}\vec{n}}, \quad \vec{V}\vec{p}_1 &= 0
\label{eq:conic_tangent_qk_eigen}
\end{align}


\begin{proof}
  If $\vec{V}$ is non-invertible, it has a zero eigenvalue.  If the corresponding eigenvector is $\vec{p}_1$, then,
\begin{align}
\vec{V}\vec{p}_1 = 0
\label{eq:conic_zero_eigen}
\end{align}
From \eqref{eq:conic_normal_vec},
\begin{align}
\label{eq:conic_zero_eigen_normal}
\kappa \vec{n} &= \vec{V} \vec{q}+\vec{u}, \quad \kappa \in \mathbb{R}
\\
\implies \kappa \vec{p}_1^{\top}\vec{n} &= \vec{p}_1^{\top}\vec{V} \vec{q}+\vec{p}_1^{\top}\vec{u}
\\
\text{or, } \kappa \vec{p}_1^{\top}\vec{n} &= \vec{p}_1^{\top}\vec{u},  \quad \because \vec{p}_1^{\top} \vec{V} = 0, 
%\\
\quad 
\brak{\text{ from } \eqref{eq:conic_zero_eigen}}
%\label{eq:conic_normal_vec_q}
\end{align}
yielding $\kappa$ in \eqref{eq:conic_tangent_qk_eigen}. From \eqref{eq:conic_zero_eigen_normal},
\begin{align}
\kappa \vec{q}^{\top}\vec{n} &= \vec{q}^{\top}\vec{V} \vec{q}+\vec{q}^{\top}\vec{u}
\\
\implies \kappa \vec{q}^{\top}\vec{n} &= -f-\vec{q}^{\top}\vec{u} \quad \text{from } \eqref{eq:conic_tangent_qquad},
\\
\text{or, } \brak{\kappa \vec{n}+\vec{u}}^{\top}\vec{q} &= -f
\label{eq:conic_zero_eigen_normal_fq}
\end{align}
\eqref{eq:conic_zero_eigen_normal} can be expressed as
\begin{align}
\label{eq:conic_zero_eigen_normal_vq}
\vec{V} \vec{q} = \kappa \vec{n} - \vec{u}.
\end{align}
\eqref{eq:conic_zero_eigen_normal_fq} and \eqref{eq:conic_zero_eigen_normal_vq} clubbed together result in \eqref{eq:conic_tangent_q_eigen}.
\end{proof}
\item
	The normal vectors of the tangents  
	 to the conic in \eqref{eq:conic_quad_form} satisfy
\begin{align}
\vec{n} ^{\top}\vec{V}^{-1}\vec{n}-f_0 = 0
	\label{eq:dual-nf0}
    \end{align}

%
\begin{proof}
From 
  \eqref{eq:conic_tangent_mq}, the normal vector to  the tangent at $\vec{q}$ can be expressed as 
  \begin{align}
  \vec{n} &= \vec{V}\vec{q}+\vec{u} 
  \label{eq:conic_normal_n}
  \\
  \implies \vec{q} &= \vec{V}^{-1}\brak{\vec{n}-\vec{u} }
  \label{eq:conic_normal_q}
  \end{align}
  which upon substituting in \eqref{eq:conic_quad_form} yields
\begin{align}
    \label{eq:conic_quad_form_q}
    \brak{\vec{n}-\vec{u} }^{\top}\vec{V}^{-1}\vec{V}\vec{V}^{-1}\brak{\vec{n}-\vec{u} }+2\vec{u}^{\top}\vec{V}^{-1}\brak{\vec{n}-\vec{u} }+f&=0
	%\vec{u}^{\top}\vec{V}^{-1}\vec{u} +f&=0
    \end{align}
which can be simplified to obtain \eqref{eq:dual-nf0}.
\end{proof}
\item
	The normal vectors of the tangents 
to the conic in \eqref{eq:conic_quad_form} 
	from 
	a point $\vec{h}$ 
	are given by 

\begin{proof}
Let the equation of the tangent be 
\begin{align}
	\vec{n}^{\top}
	\vec{x} = c
	\label{eq:ext-tan}
\end{align}
If $\vec{q}$ be the point of contact,  since $\vec{h}, \vec{q}$ lie on 
	\eqref{eq:ext-tan},
\begin{align}
	\vec{n}^{\top}
	\vec{q} = 
	\vec{n}^{\top}
	\vec{h} = c
\end{align}
From 
  \eqref{eq:conic_normal_n}, 
%  \begin{align}
%	  \vec{n}^{\top}\vec{V}^{-1}\vec{n} &= \vec{n}^{\top}\brak{\vec{q}+\vec{V}^{-1}\vec{u} }
%	  \\
%%	  &= \vec{n}^{\top}\vec{q}+\vec{n}^{\top}\vec{V}^{-1}\vec{u} 
%\end{align}
\end{proof}
\item
	The normal vectors of the tangents 
to the conic in \eqref{eq:conic_quad_form} 
	from 
	a point $\vec{h}$ 
	are given by 
  \begin{align} 
  \label{eq:quad_form_pair_normvecs-sigma}
  \begin{split}
  \vec{n}_1 &= \vec{P}\myvec{\sqrt{\abs{\lambda_1}} \\[2mm]  \sqrt{\abs{\lambda_2}}}
  \\
  \vec{n}_2 &= \vec{P}\myvec{\sqrt{\abs{\lambda_1}} \\[2mm] - \sqrt{\abs{\lambda_2}}}
  \end{split}
  \end{align} 
  where $\lambda_i, \vec{P}$ are the eigenparameters of 
  \begin{align} 
		\bm{\Sigma} &= 
	   \brak{\vec{V}\vec{h}+\vec{u}}
	  \brak{\vec{V}\vec{h}+\vec{u}}^{\top}
   -\vec{V}
  \brak
  {
  \vec{h}^{\top}\vec{V}\vec{h} + 2\vec{u}^{\top}\vec{h} +f
  }.
	  \label{eq:h-tangents-sigma}
  \end{align}                    

\begin{proof}
 From \eqref{eq:tangent_roots},
 and
  \eqref{eq:conic_tangent_disc}
  \begin{align}
  \sbrak{
  \vec{m}^{\top}\brak{\vec{V}\vec{h}+\vec{u}}
  }^2 -\brak{\vec{m}^{\top}\vec{V}\vec{m}}
  \brak
  {
  \vec{h}^{\top}\vec{V}\vec{h} + 2\vec{u}^{\top}\vec{h} +f
  } &= 0                                                                                             
  \\
	  \implies 
  \vec{m}^{\top}  \sbrak{\brak{\vec{V}\vec{h}+\vec{u}}
	  \brak{\vec{V}\vec{h}+\vec{u}}^{\top}
   -\vec{V}
  \brak
  {
  \vec{h}^{\top}\vec{V}\vec{h} + 2\vec{u}^{\top}\vec{h} +f
  }}\vec{m} &= 0                                                                                             
  \label{eq:conic_tangent_disc-h}
  \end{align}                    
  yielding
	  \eqref{eq:h-tangents-sigma}.  Consequently, from 
  \eqref{eq:quad_form_pair_normvecs}, 
  \eqref{eq:quad_form_pair_normvecs-sigma}
  can be obtained.
\end{proof}
%
\end{enumerate}



%\include{ch02} 
%\backmatter
%\appendix
\appendices
\section{ Vectors}
\subsection{$2\times 1$ vectors}

%\renewcommand{\theequation}{\theenumi}
%\begin{enumerate}[label=\arabic*.,ref=\theenumi]
\begin{enumerate}[label=\thesection.\arabic*.,ref=\thesection.\theenumi]
%\begin{enumerate}[1.]
%\begin{enumerate}
%\numberwithin{equation}{enumi}
\item Let 
\begin{align}
  \vec{A} \equiv \overrightarrow{A} &= \myvec{a_1\\a_2} 
  \\
  &\equiv a_1\overrightarrow{i}+a_2\overrightarrow{j}, 
  \\
  \vec{B} &= \myvec{b_1\\b_2}, 
\end{align}
be $2 \times 1$ vectors.
Then, the determinant of the $2 \times 2$ matrix 
\begin{align}  
  \vec{M} = \myvec{\vec{A} & \vec{B}}
\end{align}
is defined as
\begin{align}
  \label{eq:det2d}
  \mydet{\vec{M}} &= \mydet{\vec{A} & \vec{B}} 
  \\
  &= \mydet{a_1 & b_1\\a_2 & b_2} = a_1b_2 - a_2 b_1
\end{align}
%
\item The value of the cross product of two vectors is given by  
  \eqref{eq:det2d}.
\item The area of the triangle with vertices $\vec{A}, \vec{B}, \vec{C}$ is given by the absolute value of 
\begin{align}
  \label{eq:area2d}
\frac{1}{2} \mydet{\vec{A-B} & \vec{A-C}}
  \end{align}
  \item  The transpose of $\vec{A}$ is defined as
\begin{align}
  \label{eq:transpose2d}
  \vec{A}^{\top}  = \myvec{a_1 & a_2}
\end{align}
%
\item The {\em inner product} or {\em dot product} is defined as
\begin{align}
  \label{eq:dot2d}
  \vec{A}^{\top} \vec{B} &\equiv \vec{A} \cdot \vec{B} 
  \\
  &= \myvec{a_1 & a_2} \myvec{b_1 \\ b_2}= a_1b_1+a_2b_2 
\end{align}
%
\item {\em norm} of $\vec{A}$ is defined as
\begin{align}
  \label{eq:norm2d}
  \norm{A} &\equiv \mydet{\overrightarrow{A}}
  \\
  &= \sqrt{\vec{A}^{\top} \vec{A}}= \sqrt{a_1^2+a_2^2}
\end{align}
Thus, 
\begin{align}
  \label{eq:norm2d_const}
  \norm{\lambda \vec{A}} &\equiv \mydet{\lambda\overrightarrow{A}}
  \\
  &= \abs{\lambda} \norm{\vec{A}}
\end{align}
\item The distance betwen the points $\vec{A}$ and $\vec{B}$ is given by 
\begin{align}
  \label{eq:norm2d_dist}
\norm{\vec{A}-\vec{B}} 
\end{align}
\item Let $\vec{x}$ be equidistant from the points $\vec{A}$ and $\vec{B}$.  Then 
  \begin{align}
	  \brak{\vec{A}-\vec{B}}^{\top}{\vec{x}} 
	  =  \frac{\norm{\vec{A}}^2 - \norm{\vec{B}}^2}{2}
  \label{eq:norm2d_equidist}
  \end{align}
  \solution 
\begin{align}
	\norm{\vec{x}-\vec{A}} &=
\norm{\vec{A}-\vec{B}} 
\\
	\implies \norm{\vec{x}-\vec{A}}^2 &=
\norm{\vec{x}-\vec{B}}^2 
\end{align}
which can be expressed as 
\begin{multline}
%  \label{eq:norm2d_dist}
	\brak{\vec{x}-\vec{A}}^{\top} \brak{\vec{x}-\vec{A}}=
	\brak{\vec{x}-\vec{B}}^{\top} 
\brak{\vec{x}-\vec{B}}
\\
	\implies	\norm{\vec{x}}^2-2{\vec{x}}^{\top}\vec{A} + \norm{\vec{A}}^2
	\\= \norm{\vec{x}}^2-2{\vec{x}}^{\top}\vec{B} + \norm{\vec{B}}^2
\end{multline}
which can be simplified to obtain
  \eqref{eq:norm2d_equidist}.
\item If $\vec{x}$ lies on the  $x$-axis and is  equidistant from the points $\vec{A}$ and $\vec{B}$, 
  \begin{align}
	  \vec{x} &=
	   x\vec{e}_1
  \end{align}
  where 
  \begin{align}
	  x &=\frac{\norm{\vec{A}}^2 -\norm{\vec{B}}^2 }{2\brak{\vec{A}-\vec{B}}^{\top }\vec{e}_1
}
	  \label{eq:cbse_10_x}
  \end{align}
  \solution 
  From \eqref{eq:norm2d_equidist}.
  \begin{align}
	   x\brak{\vec{A}-\vec{B}}^{\top }\vec{e}_1
		  &=
	  \frac{\norm{\vec{A}}^2 -\norm{\vec{B}}^2 }{2}
   \end{align}
	  yielding \eqref{eq:cbse_10_x}.
  \item The angle between two vectors is given by 
  \begin{align}
    \label{eq:angle2d}
    \theta = \cos^{-1}\frac{\vec{A}^{\top} \vec{B}}{\norm{A}\norm{B}}
  \end{align}
  \item If two vectors are orthogonal (perpendicular), 
  \begin{align}
    \label{eq:angle2d_orth}
\vec{A}^{\top} \vec{B} = 0
  \end{align}

  \item The {\em direction vector} of the line joining two points $\vec{A},\vec{B}$ is given by 
  \begin{align}
    \label{eq:dir_vec}
    \vec{m} = \vec{A}-\vec{B}
  \end{align}
\item The unit vector in the direction of $\vec{m}$ is defined as
\begin{align}
    \frac{\vec{m}}{\norm{\vec{m}}}
\end{align}
\item If the direction vector of a line is expressed as 
		\label{prop:two-dir-vec}
	\begin{align}
		\label{eq:two-dir-vec}
    \vec{m} = \myvec{1\\m},
\end{align}
 the $m$ is defined to be the {\em} slope of the line. 
  \item $AB \parallel CD$ if 
	  \label{prop:two-par-dir-vec}
  \begin{align}
	  \vec{A}- \vec{B}= k\brak{\vec{C}- \vec{D}}
	  \label{eq:two-par-dir-vec}
  \end{align}
  \item The {\em normal vector} to $\vec{m}$ is defined by 
  \begin{align}
    \label{eq:normal_vec}
    \vec{m}^{\top}  \vec{n} = 0
  \end{align}
  \item The point $\vec{P}$ that divides the line segment $AB$ in the ratio $k:1$  is given by 

  \begin{align}
	  \vec{P}&= \frac{k\vec{B}+ \vec{A}}{k+1}
	  \label{eq:section_formula}
  \end{align}
\item  The standard basis vectors are defined as 
	\label{def:matrix-two}

  \begin{align}
  \vec{e}_1&= \myvec{1\\0}, 
  \\
  \vec{e}_2&= \myvec{0\\1}.
  \end{align}
  \item If $ABCD$ be a parallelogram,
	  \label{eq:two-pgm}
  \begin{align}
 \vec{B}-\vec{A} = \vec{C} -\vec{D}
  \end{align}
  \item Points $\vec{A},\vec{B}$ and $\vec{C}$ form a triangle  if 
	  \label{prop:two-tri-indep}
  \begin{align}
	  p\brak{\vec{A}- \vec{B}} +q\brak{\vec{A} -\vec{C}} &= 0
	  \\
	  \label{eq:two-tri-indep}
	  \text{or, }\brak{p+q}\vec{A}- p\vec{B} -q\vec{C} &= 0
	  \\
	  \implies p=0, q=0
  \end{align}
  are linearly independent.
  \item In $\triangle ABC$, if $\vec{D}, \vec{E}$ divide the lines $AB, AC$ in the ratio $k:1$ respectively,  then $DE \parallel BC$.
	  \label{prop:two-tri-bpt}
	  \begin{proof}
		  From 
	  \eqref{eq:section_formula}, 
  \begin{align}
	  \vec{D}&= \frac{k\vec{B}+ \vec{A}}{k+1}
	  \\
	  \vec{E}&= \frac{k\vec{C}+ \vec{A}}{k+1}
	  \\
	  \implies 
	  \vec{D}-	  \vec{E}&= \frac{k}{k+1}\brak{\vec{B}- \vec{C}}
  \end{align}
  Thus, from 
		  Appendix \ref{prop:two-dir-vec}, $DE \parallel BC$.

	  \end{proof}

  \item In $\triangle ABC$, if $DE \parallel BC$, $\vec{D}$ and $\vec{E}$ divide the lines $AB, AC$ in the same ratio.  
	  \label{prop:two-tri-bpt-conv}
	  \begin{proof}
If $DE \parallel BC$,
		  from 
 \eqref{eq:two-par-dir-vec}
  \begin{align}
	  \label{prop:two-tri-bpt-conv-1}
	  \brak{\vec{B}- \vec{C}} = k\brak{\vec{D}-	  \vec{E}}
  \end{align}
Using   
	  \eqref{eq:section_formula}, 
let 
  \begin{align}
	  \vec{D}&= \frac{k_1\vec{B}+ \vec{A}}{k_1+1}
	  \\
	  \vec{E}&= \frac{k_2\vec{C}+ \vec{A}}{k_2+1}
  \end{align}
	  Subtituting the above in 
	  \eqref{prop:two-tri-bpt-conv-1}, after some algebra, we obtain 
	
  \begin{align}
\brak{p+q}\vec{A}- p\vec{B} -q\vec{C} &= 0
  \end{align}
  where
  \begin{align}
	  p = \frac{1}{k} -  \frac{k_1}{k_1+1},
	  q = \frac{1}{k} -  \frac{k_1}{k_1+1}
  \end{align}
  %
From 	  
	  \eqref{eq:two-tri-indep},
  \begin{align}
	p = q = 0
	  \\
	  \implies k_1 = k_2  = \frac{1}{k-1}
  \end{align}

	  \end{proof}
\end{enumerate}

%\include{app01}
%\appendix
\subsection{$3\times 1$ vectors}

%\renewcommand{\theequation}{\theenumi}
%\begin{enumerate}[label=\arabic*.,ref=\theenumi]
\begin{enumerate}[label=\thesection.\arabic*.,ref=\thesection.\theenumi]
%\begin{enumerate}
%\numberwithin{equation}{enumi}

\item Let 
\begin{align}
  \vec{A} &= \myvec{a_1\\a_2 \\ a_3} \equiv a_1\overrightarrow{i}+a_2\overrightarrow{j}+a_3\overrightarrow{j}, 
  \\
  \vec{B} &= \myvec{b_1\\b_2 \\ b_3}, 
\end{align}
and 
\begin{align}
  \vec{A}_{ij} &= \myvec{a_i\\a_j}, 
  \\
  \vec{B}_{ij} &= \myvec{b_i\\b_j}. 
\end{align}

\item The {\em cross product} or {\em vector product} of $\vec{A}, \vec{B}$ is defined as
\begin{align}
  \label{eq:cross3d}
	\vec{A} \times \vec{B} = \myvec{ \mydet{\vec{A}_{23} & \vec{B}_{23}} \\[10pt] \mydet{\vec{A}_{31} & \vec{B}_{31}} \\[10pt] \mydet{\vec{A}_{12}  & \vec{B}_{12}}}
\end{align}
\item Verify that
\begin{align}
  \vec{A} \times \vec{B} = -  \vec{B} \times \vec{A} 
\end{align}
\item The area of a triangle is given by 
\begin{align}
	\frac{1}{2} \norm{  \vec{A} \times \vec{B}}
\end{align}
\item (Cauchy-Schwarz Inequality)
    \begin{align}
        \label{eq:dot-mag-ineq}
	    \abs{\vec{a}^\top\vec{b}} &\le \norm{\vec{a}}\norm{\vec{b}}
    \end{align}
    \solution
	\begin{align}
        \norm{\vec{a}-\frac{\vec{a}^\top\vec{b}}{\norm{\vec{b}}^2}\vec{b}}^2 &\ge 0 \\
        \implies \norm{\vec{a}}^2 - 2\frac{\brak{\vec{a}^\top\vec{b}}^2}{\norm{\vec{b}}^2} + \frac{\brak{\vec{a}^\top\vec{b}}^2}{\norm{\vec{b}}^2} &\ge 0 \\
        \implies \norm{\vec{a}}^2 - \frac{\brak{\vec{a}^\top\vec{b}}^2}{\norm{\vec{b}}^2} &\ge 0 \\
        \implies \norm{\vec{a}}^2\norm{\vec{b}}^2 &\ge \brak{\vec{a}^\top\vec{b}}^2 \\
    \end{align}
    yielding
        \eqref{eq:dot-mag-ineq}.
\item (Triangle Inequality)
    \begin{align}
\norm{\vec{a}+\vec{b}} &\le \norm{\vec{a}}+\norm{\vec{b}}
        \label{eq:triangle-ineq}
    \end{align}
    \solution
    Using \eqref{eq:dot-mag-ineq},
    \begin{align}
        \vec{a}^\top\vec{b} &\le \norm{\vec{a}}\norm{\vec{b}} \\
\implies        \norm{\vec{a}}^2 + 2\vec{a}^\top\vec{b} + \norm{\vec{b}}^2 &\le \norm{\vec{a}}^2 + 2\norm{\vec{a}}\norm{\vec{b}} + \norm{\vec{b}}^2 \\
\implies               \norm{\vec{a}+\vec{b}}^2 &\le \brak{\norm{\vec{a}}+\norm{\vec{b}}}^2 
    \end{align}
    yielding
        \eqref{eq:triangle-ineq}.
\end{enumerate}

\section{Matrices}
\section{Eigenvalues and Eigenvectors}
%\renewcommand{\theequation}{\theenumi}
%\begin{enumerate}[label=\arabic*.,ref=\theenumi]
\begin{enumerate}[label=\thesection.\arabic*.,ref=\thesection.\theenumi]
%\begin{enumerate}
%\numberwithin{equation}{enumi}
\item The eigenvalue $\lambda$ and the eigenvector $\vec{x}$  for a matrix $\vec{A}$ are defined as, 
\begin{align}
  \vec{A} \vec{x} = \lambda \vec{x}
\end{align}
\item The eigenvalues are calculated by solving the
equation
\begin{align}
  \label{eq:chareq}
f\brak{\lambda} = \mydet{\lambda \vec{I}- \vec{A} } =0
\end{align}
The above equation is known as the characteristic equation.
\item According to the Cayley-Hamilton theorem,
\begin{align}
	\label{eq:cayley}
  f(\lambda) = 0 \implies f\brak{\vec{A}} = 0
\end{align}
\item The trace of a square  matrix is defined to be the sum of the diagonal elements.
\begin{align}
	\label{eq:trace}
	\text{tr}\brak{\vec{A}}=\sum_{i=1}^{N}a_{ii}.
\end{align}
	where $a_{ii}$ is the $i$th diagonal element of the matrix $\vec{A}$. 	
\item The trace of a matrix is equal to the sum of the eigenvalues
\begin{align}
	\label{eq:trace_eig}
	\text{tr}\brak{\vec{A}}=\sum_{i=1}^{N}\lambda_i
\end{align}


\end{enumerate}
\section{Determinants}
%\renewcommand{\theequation}{\theenumi}
%\begin{enumerate}[label=\arabic*.,ref=\theenumi]
\begin{enumerate}[label=\thesection.\arabic*.,ref=\thesection.\theenumi]
%\begin{enumerate}
%\numberwithin{equation}{enumi}

\item Let 
\begin{align}
	\vec{A} = \myvec{a_1 & b_1 & c_1  \\ a_2 & b_2 & c_2  \\ a_3 & b_3 & c_3}.
\end{align}
be a $3 \times 3$ matrix. 
Then, 
\begin{multline}
	\mydet{\vec{A}} = a_1 \myvec{ b_2 & c_2 \\  b_3 & c_3} - a_2\myvec{ b_1 & c_1 \\  b_3 & c_3 }  \\ + a_3\myvec{a_1 & b_1 \\ a_2 & b_2 }.
\end{multline}
\item Let $\lambda_1,\lambda_2, \dots, \lambda_n$ be the eigenvalues of a matrix $\vec{A}$.  Then,   the product of the eigenvalues is equal to the determinant of $\vec{A}$.
\begin{align}
	\mydet{\vec{A}} = \prod_{i=1}^{n}\lambda_i
\end{align}
%
\item 
\begin{align}
	\mydet{\vec{A}\vec{B}} = \mydet{\vec{A}}\mydet{\vec{B}}
\end{align}
\item If $\vec{A}$ be an $n \times n$ matrix, 
\begin{align}
	\label{eq:det_kord}
	\mydet{k\vec{A}} = k^n\mydet{\vec{A}}
\end{align}

\end{enumerate}
\section{Rank of a Matrix}
%\renewcommand{\theequation}{\theenumi}
%\begin{enumerate}[label=\arabic*.,ref=\theenumi]
\begin{enumerate}[label=\thesection.\arabic*.,ref=\thesection.\theenumi]
%\begin{enumerate}
%\numberwithin{equation}{enumi}
\item The rank of a matrix is defined as the number of linearly independent rows.  This is also known as the row rank.
\item Row rank = Column rank.
\item The rank of a matrix is obtained as the number of nonzero rows obtained after row reduction.
\item An $n \times n$ matrix is invertible if and only if its rank is $n$.

\item Points $\vec{A}, \vec{B}, \vec{C}$ are on a line if 
\begin{align}
%  \label{eq:line_rank}
  \text{rank}\myvec{\vec{A} \\ \vec{B} \\ \vec{C} }  = 1
\end{align}
\item Points $\vec{A}, \vec{B}, \vec{C}, \vec{D}$ form a paralelogram if 
\begin{align}
%  \label{eq:parallelgm_rank}
  \text{rank}\myvec{\vec{A} \\ \vec{B} \\ \vec{C} \\ \vec{D}  }  = 1, 
  \text{rank}\myvec{\vec{A} \\ \vec{B} \\ \vec{C} }  = 2
\end{align}
\end{enumerate}
\section{Inverse of a Matrix}
%\renewcommand{\theequation}{\theenumi}
%\begin{enumerate}[label=\arabic*.,ref=\theenumi]
\begin{enumerate}[label=\thesection.\arabic*.,ref=\thesection.\theenumi]
%\begin{enumerate}
%\numberwithin{equation}{enumi}
\item For a $2 \times 2$ matrix 
\begin{align}
	\vec{A} = \myvec{a_1 & b_1  \\ a_2 & b_2 },
\end{align}
the inverse is given by 
\begin{align}
	\vec{A}^{-1} = \frac{1}{\mydet{\vec{A}}}\myvec{b_2 & -b_1  \\ -a_2 & a_1 },
\end{align}
\item For higher order matrices, the inverse should be calculated using row operations.
\end{enumerate}
\section{Orthogonality}
%\renewcommand{\theequation}{\theenumi}
%\begin{enumerate}[label=\arabic*.,ref=\theenumi]
\begin{enumerate}[label=\thesection.\arabic*.,ref=\thesection.\theenumi]
%\begin{enumerate}
%\numberwithin{equation}{enumi}
\item The rotation matrix is defined as 
\begin{align}
	\vec{R}_{\theta} = \myvec{\cos \theta & -\sin \theta  \\ \sin \theta  & \cos \theta  }, \quad \theta \in \sbrak{0, 2\pi}
\end{align}
\item The rotation matrix is {\em orthogonal}
\begin{align}
	\vec{R}_{\theta}^{\top}\vec{R}_{\theta} = \vec{R}_{\theta}\vec{R}_{\theta}^{\top} = \vec{I}
\end{align}
\item 
\begin{align}
	\vec{m}^{\top}\vec{n} = 0 \implies \vec{n} = \vec{R}_{\frac{\pi}{2}}\vec{m}
\end{align}
\item 
\begin{align}
	\label{eq:mat-nh}
	\vec{n}^{\top}\vec{h} = 1 \implies \vec{n} = \frac{\vec{e}_1}{\vec{e}_1^{\top}\vec{h}}+\mu\vec{R}_{\frac{\pi}{2}}\vec{h}, \quad \mu \in \mathbb{R}.
\end{align}


\item
	The {\em affine} transformation is given by 
    \begin{align}
	    \vec{x} &= \vec{P}\vec{y}+\vec{c} \quad \text{(Affine Transformation)}
\label{eq:conic_affine}
    \end{align}
	where $\vec{P}$ is invertible.

\item
	The eigenvalue decomposition of a symmetric matrix $\vec{V}$ is given by 
	%\cite{banchoff}
    \begin{align}
      \label{eq:conic_parmas_eig_def}
      \vec{P}^{\top}\vec{V}\vec{P} &= \vec{D}. \quad \text{(Eigenvalue Decomposition)}
      \\
      \vec{D} &= \myvec{\lambda_1 & 0\\ 0 & \lambda_2}, 
      \\
      \vec{P} &= \myvec{\vec{p}_1 & \vec{p}_2}, \quad \vec{P}^{\top}=\vec{P}^{-1},
      \label{eq:eigevecP}
    \end{align}


\end{enumerate}

\section{Singular Value Decomposition}
\renewcommand{\theequation}{\theenumi}
\begin{enumerate}[label=\thesection.\arabic*.,ref=\thesection.\theenumi]
\numberwithin{equation}{enumi}

\item 
\iffalse
	We revisit \eqref{eq:pseudo_mat_eq}
%
\begin{align}
\myvec{
1 & 2
\\
-1 & 1
\\
1 & 2
}
\vec{x} =
\myvec{1 \\ -3 \\ -2}
\end{align}
\fi
\item Consider the rectangular equation
\begin{align}
	\vec{M}^{\top}\vec{x}=\vec{b}
\label{eq:pseudo_mat_eq}
\end{align}

\item Find $\vec{M}^T\vec{M}$ and $\vec{M}\vec{M}^T$.
\item Obtain the eigen decomposition 
\begin{align}
\vec{M}^T\vec{M} = \vec{P}_1\vec{D}_1\vec{P}_1^T
\end{align}
and 
\begin{align}
\vec{M}\vec{M}^T = \vec{P}_2\vec{D}_2\vec{P}_2^T
\end{align}
\item Perform the $QR$ decompositions
\begin{align}
\vec{P}_1 = \vec{U}\vec{R}_1,\,
\vec{P}_2 = \vec{V}\vec{R}_2
\end{align}
\item The singular value decomposition is the given by
\begin{align}
\vec{M} = \vec{U} \Sigma \vec{V}^T,
\end{align}
where $\Sigma$ has the same shape as $\vec{M}$ and
\begin{align}
\Sigma = \myvec{\vec{D}_1 & \vec{0} \\ \vec{0} & \vec{0}}
\end{align}
\item \eqref{eq:pseudo_mat_eq} can then be expressed as
\begin{align}
\vec{U} \Sigma \vec{V}^T \vec{x} &= \vec{b}
\\
\implies \vec{x} & = \vec{V}\Sigma^{-1} \vec{U}^T \vec{b}
\end{align}
%
where $\vec{\Sigma}^{-1}$ is obtained by inverting  only the non-zero elements of $\vec{\Sigma}$.
\iffalse
\item The relevant codes are available at
\begin{lstlisting}
codes/line/skew_builtin.py
codes/line/skew_svd.py
\end{lstlisting}
\fi
\end{enumerate}


\section{Triangle Constructions}
\begin{enumerate}[label=\thesection.\arabic*,ref=\thesection.\theenumi]
\numberwithin{equation}{enumi}
\numberwithin{figure}{enumi}
\numberwithin{table}{enumi}
\item 
	Construct a triangle $ABC$ in which $a, \angle{B}$ and $c + b  = K$ are given.
\label{cons/tri/1}
	\\
	\solution 
\iffalse
\documentclass[journal,10pt,twocolumn]{article}
\usepackage{graphicx}
\usepackage[margin=0.5in]{geometry}
\usepackage[cmex10]{amsmath}
\usepackage{array}
\usepackage{booktabs}
\usepackage{listings}
\title{\textbf{Line Assignment}}
\author{Bhavani Kanike}
\date{October 2022}

\providecommand{\norm}[1]{\left\lVert#1\right\rVert}
\providecommand{\abs}[1]{\left\vert#1\right\vert}
\let\vec\mathbf
\newcommand{\myvec}[1]{\ensuremath{\begin{pmatrix}#1\end{pmatrix}}}
\newcommand{\mydet}[1]{\ensuremath{\begin{vmatrix}#1\end{vmatrix}}}
\providecommand{\brak}[1]{\ensuremath{\left(#1\right)}}

\begin{document}

\maketitle
\paragraph{\textit{Problem Statement} 
\fi
ABCD is a quadrilateral in which $\vec{P}, \vec{Q}, \vec{R}$ and $\vec{S}$ are mid-points of the sides AB, BC, CD and DA (see Fig \ref{fig:9/8/2/1}). AC is a diagonal. 
		
Show that 
\begin{enumerate}
	\item $SR \parallel AC$ and $SR =\frac{1}{2} AC$
\item $PQ = SR$
\item $PQRS$ is a parallelogram.
\end{enumerate}
 	\begin{figure}
		\centering
 \includegraphics[width=\columnwidth]{chapters/9/8/2/1/figs/line1.pdf}
		\caption{}
		\label{fig:9/8/2/1}
  	\end{figure}
	\solution 
	Using 
	  \eqref{eq:section_formula},
	\begin{align}
		\label{eq:9/8/2/1}
		\begin{split}
		\vec{P} &= \frac{\vec{A}+\vec{B}}{2}\\
 \vec{Q} &= \frac{\vec{C}+\vec{B}}{2}\\
 \vec{R} &= \frac{\vec{C}+\vec{D}}{2}\\
 \vec{S} &= \frac{\vec{D}+\vec{A}}{2}
		\end{split}
	\end{align}
\begin{enumerate}
	\item
	Consequently, 
	\begin{align}
\vec{R}
		-\vec{S} &= \frac{\vec{C}-\vec{A}}{2}
		\\
		\implies SR &\parallel AC
	\end{align}
	Also, 
	\begin{align}
		\norm{\vec{R}
		-\vec{S}} &= \frac{\norm{\vec{C}-\vec{A}}}{2}
		\\
		\implies SR &= \frac{1}{2}AC
	\end{align}
\item 	From 
		\eqref{eq:9/8/2/1},
	\begin{align}
\vec{R}
		-\vec{S} = \vec{Q}-\vec{P}
	\end{align}
	which means that $PQRS$ is a parallelogram and $PQ = SR$.
\end{enumerate}
%
\iffalse
\begin{figure}[h]
\centering
\includegraphics[width=1\columnwidth]
\caption{Figure}
\label{fig:triangle}
\end{figure}

\section*{Solution}

$\boldsymbol Given :$  ABCD is a Quadrilateral P,Q,R and S are the midpoints of line AB,BC,CD,DA.We can obtain the points P,Q,R and S from A,B,C and D and are given by\\\\
\boldmath
\unboldmath
(3) To prove that PQRS is a parallelogram we need to prove  PQ // SR
To prove SR $\parallel$ PQ\\
Direction vector of line SR  $\boldsymbol {(R-S) =  \frac{(C-A)}{2}}$\\\\
Direction vector of line PQ  $\boldsymbol {(Q-P)= \frac{(C-A)}{2}}$\\\\
\begin{equation}
	\boldsymbol {(R-S) = (Q-P) = \frac{(C-A)}{2}}\\
\end{equation}
Since the direction vectors of line SR and PQ are in same direction\\\\
$SR \parallel PQ$\\
Therefore,
$\boldsymbol{ PQRS }$ is a parallelogram\\\\

	
(1)  Directional vector of line SR  = $\boldsymbol {(R-S)}$ = $\frac{\boldsymbol{(C-A)}}{2} $\\
Directional vector of line AC  = $\boldsymbol {(C-A)}$\\

It is observed that the constant k is $\frac{1}{2}$

Therefore
\begin{equation}
	SR \parallel AC
\end{equation} 

and from equation 1 
\begin{equation}
	\boldsymbol {SR = \frac{1}{2}AC}    
\end{equation}\\


(2)   To prove PQ = SR\\ 
		From euqation 1\\\\
\begin{equation}
		\boldsymbol{ (Q-P) = (R-S) = \frac{(C-A)}{2}}
\end{equation}
	 



\section{Execution}
The below python code realizes the construction:
\begin{lstlisting}
https://github.com/bhavani360/FWC_assignments
\end{lstlisting}
	
\section*{Construction}
The dimensions of the Quadrilateral ABCD are taken as below\\
{
\setlength\extrarowheight{2pt}
\centering
	\begin{tabular}{|c|c|}
	\hline
	\textbf{symbol}&\textbf{value}\\
	\hline
	r&8\\
	\hline
	$\theta$&pi/2.5\\
	\hline
	d&7\\
	\hline
	A&(0,0)\\
	\hline
	B&(d,0)\\
	\hline
	D&(rcos$\theta$,rsin$\theta$)\\
	\hline
	C&(D/1.5)+B\\
	\hline
\end{tabular}
}
\end{document}
\fi

%
\iffalse
\item 
\label{cons/tri/2}
\iffalse
\documentclass[journal,10pt,twocolumn]{article}
\usepackage{graphicx}
\usepackage[margin=0.5in]{geometry}
\usepackage[cmex10]{amsmath}
\usepackage{array}
\usepackage{booktabs}
\usepackage{listings}
\title{\textbf{Line Assignment}}
\author{Bhavani Kanike}
\date{October 2022}

\providecommand{\norm}[1]{\left\lVert#1\right\rVert}
\providecommand{\abs}[1]{\left\vert#1\right\vert}
\let\vec\mathbf
\newcommand{\myvec}[1]{\ensuremath{\begin{pmatrix}#1\end{pmatrix}}}
\newcommand{\mydet}[1]{\ensuremath{\begin{vmatrix}#1\end{vmatrix}}}
\providecommand{\brak}[1]{\ensuremath{\left(#1\right)}}

\begin{document}

\maketitle
\paragraph{\textit{Problem Statement} 
\fi
ABCD is a quadrilateral in which $\vec{P}, \vec{Q}, \vec{R}$ and $\vec{S}$ are mid-points of the sides AB, BC, CD and DA (see Fig \ref{fig:9/8/2/1}). AC is a diagonal. 
		
Show that 
\begin{enumerate}
	\item $SR \parallel AC$ and $SR =\frac{1}{2} AC$
\item $PQ = SR$
\item $PQRS$ is a parallelogram.
\end{enumerate}
 	\begin{figure}
		\centering
 \includegraphics[width=\columnwidth]{chapters/9/8/2/1/figs/line1.pdf}
		\caption{}
		\label{fig:9/8/2/1}
  	\end{figure}
	\solution 
	Using 
	  \eqref{eq:section_formula},
	\begin{align}
		\label{eq:9/8/2/1}
		\begin{split}
		\vec{P} &= \frac{\vec{A}+\vec{B}}{2}\\
 \vec{Q} &= \frac{\vec{C}+\vec{B}}{2}\\
 \vec{R} &= \frac{\vec{C}+\vec{D}}{2}\\
 \vec{S} &= \frac{\vec{D}+\vec{A}}{2}
		\end{split}
	\end{align}
\begin{enumerate}
	\item
	Consequently, 
	\begin{align}
\vec{R}
		-\vec{S} &= \frac{\vec{C}-\vec{A}}{2}
		\\
		\implies SR &\parallel AC
	\end{align}
	Also, 
	\begin{align}
		\norm{\vec{R}
		-\vec{S}} &= \frac{\norm{\vec{C}-\vec{A}}}{2}
		\\
		\implies SR &= \frac{1}{2}AC
	\end{align}
\item 	From 
		\eqref{eq:9/8/2/1},
	\begin{align}
\vec{R}
		-\vec{S} = \vec{Q}-\vec{P}
	\end{align}
	which means that $PQRS$ is a parallelogram and $PQ = SR$.
\end{enumerate}
%
\iffalse
\begin{figure}[h]
\centering
\includegraphics[width=1\columnwidth]
\caption{Figure}
\label{fig:triangle}
\end{figure}

\section*{Solution}

$\boldsymbol Given :$  ABCD is a Quadrilateral P,Q,R and S are the midpoints of line AB,BC,CD,DA.We can obtain the points P,Q,R and S from A,B,C and D and are given by\\\\
\boldmath
\unboldmath
(3) To prove that PQRS is a parallelogram we need to prove  PQ // SR
To prove SR $\parallel$ PQ\\
Direction vector of line SR  $\boldsymbol {(R-S) =  \frac{(C-A)}{2}}$\\\\
Direction vector of line PQ  $\boldsymbol {(Q-P)= \frac{(C-A)}{2}}$\\\\
\begin{equation}
	\boldsymbol {(R-S) = (Q-P) = \frac{(C-A)}{2}}\\
\end{equation}
Since the direction vectors of line SR and PQ are in same direction\\\\
$SR \parallel PQ$\\
Therefore,
$\boldsymbol{ PQRS }$ is a parallelogram\\\\

	
(1)  Directional vector of line SR  = $\boldsymbol {(R-S)}$ = $\frac{\boldsymbol{(C-A)}}{2} $\\
Directional vector of line AC  = $\boldsymbol {(C-A)}$\\

It is observed that the constant k is $\frac{1}{2}$

Therefore
\begin{equation}
	SR \parallel AC
\end{equation} 

and from equation 1 
\begin{equation}
	\boldsymbol {SR = \frac{1}{2}AC}    
\end{equation}\\


(2)   To prove PQ = SR\\ 
		From euqation 1\\\\
\begin{equation}
		\boldsymbol{ (Q-P) = (R-S) = \frac{(C-A)}{2}}
\end{equation}
	 



\section{Execution}
The below python code realizes the construction:
\begin{lstlisting}
https://github.com/bhavani360/FWC_assignments
\end{lstlisting}
	
\section*{Construction}
The dimensions of the Quadrilateral ABCD are taken as below\\
{
\setlength\extrarowheight{2pt}
\centering
	\begin{tabular}{|c|c|}
	\hline
	\textbf{symbol}&\textbf{value}\\
	\hline
	r&8\\
	\hline
	$\theta$&pi/2.5\\
	\hline
	d&7\\
	\hline
	A&(0,0)\\
	\hline
	B&(d,0)\\
	\hline
	D&(rcos$\theta$,rsin$\theta$)\\
	\hline
	C&(D/1.5)+B\\
	\hline
\end{tabular}
}
\end{document}
\fi

%
\item 
\label{cons/tri/3}
\iffalse
\documentclass[journal,10pt,twocolumn]{article}
\usepackage{graphicx}
\usepackage[margin=0.5in]{geometry}
\usepackage[cmex10]{amsmath}
\usepackage{array}
\usepackage{booktabs}
\usepackage{listings}
\title{\textbf{Line Assignment}}
\author{Bhavani Kanike}
\date{October 2022}

\providecommand{\norm}[1]{\left\lVert#1\right\rVert}
\providecommand{\abs}[1]{\left\vert#1\right\vert}
\let\vec\mathbf
\newcommand{\myvec}[1]{\ensuremath{\begin{pmatrix}#1\end{pmatrix}}}
\newcommand{\mydet}[1]{\ensuremath{\begin{vmatrix}#1\end{vmatrix}}}
\providecommand{\brak}[1]{\ensuremath{\left(#1\right)}}

\begin{document}

\maketitle
\paragraph{\textit{Problem Statement} 
\fi
ABCD is a quadrilateral in which $\vec{P}, \vec{Q}, \vec{R}$ and $\vec{S}$ are mid-points of the sides AB, BC, CD and DA (see Fig \ref{fig:9/8/2/1}). AC is a diagonal. 
		
Show that 
\begin{enumerate}
	\item $SR \parallel AC$ and $SR =\frac{1}{2} AC$
\item $PQ = SR$
\item $PQRS$ is a parallelogram.
\end{enumerate}
 	\begin{figure}
		\centering
 \includegraphics[width=\columnwidth]{chapters/9/8/2/1/figs/line1.pdf}
		\caption{}
		\label{fig:9/8/2/1}
  	\end{figure}
	\solution 
	Using 
	  \eqref{eq:section_formula},
	\begin{align}
		\label{eq:9/8/2/1}
		\begin{split}
		\vec{P} &= \frac{\vec{A}+\vec{B}}{2}\\
 \vec{Q} &= \frac{\vec{C}+\vec{B}}{2}\\
 \vec{R} &= \frac{\vec{C}+\vec{D}}{2}\\
 \vec{S} &= \frac{\vec{D}+\vec{A}}{2}
		\end{split}
	\end{align}
\begin{enumerate}
	\item
	Consequently, 
	\begin{align}
\vec{R}
		-\vec{S} &= \frac{\vec{C}-\vec{A}}{2}
		\\
		\implies SR &\parallel AC
	\end{align}
	Also, 
	\begin{align}
		\norm{\vec{R}
		-\vec{S}} &= \frac{\norm{\vec{C}-\vec{A}}}{2}
		\\
		\implies SR &= \frac{1}{2}AC
	\end{align}
\item 	From 
		\eqref{eq:9/8/2/1},
	\begin{align}
\vec{R}
		-\vec{S} = \vec{Q}-\vec{P}
	\end{align}
	which means that $PQRS$ is a parallelogram and $PQ = SR$.
\end{enumerate}
%
\iffalse
\begin{figure}[h]
\centering
\includegraphics[width=1\columnwidth]
\caption{Figure}
\label{fig:triangle}
\end{figure}

\section*{Solution}

$\boldsymbol Given :$  ABCD is a Quadrilateral P,Q,R and S are the midpoints of line AB,BC,CD,DA.We can obtain the points P,Q,R and S from A,B,C and D and are given by\\\\
\boldmath
\unboldmath
(3) To prove that PQRS is a parallelogram we need to prove  PQ // SR
To prove SR $\parallel$ PQ\\
Direction vector of line SR  $\boldsymbol {(R-S) =  \frac{(C-A)}{2}}$\\\\
Direction vector of line PQ  $\boldsymbol {(Q-P)= \frac{(C-A)}{2}}$\\\\
\begin{equation}
	\boldsymbol {(R-S) = (Q-P) = \frac{(C-A)}{2}}\\
\end{equation}
Since the direction vectors of line SR and PQ are in same direction\\\\
$SR \parallel PQ$\\
Therefore,
$\boldsymbol{ PQRS }$ is a parallelogram\\\\

	
(1)  Directional vector of line SR  = $\boldsymbol {(R-S)}$ = $\frac{\boldsymbol{(C-A)}}{2} $\\
Directional vector of line AC  = $\boldsymbol {(C-A)}$\\

It is observed that the constant k is $\frac{1}{2}$

Therefore
\begin{equation}
	SR \parallel AC
\end{equation} 

and from equation 1 
\begin{equation}
	\boldsymbol {SR = \frac{1}{2}AC}    
\end{equation}\\


(2)   To prove PQ = SR\\ 
		From euqation 1\\\\
\begin{equation}
		\boldsymbol{ (Q-P) = (R-S) = \frac{(C-A)}{2}}
\end{equation}
	 



\section{Execution}
The below python code realizes the construction:
\begin{lstlisting}
https://github.com/bhavani360/FWC_assignments
\end{lstlisting}
	
\section*{Construction}
The dimensions of the Quadrilateral ABCD are taken as below\\
{
\setlength\extrarowheight{2pt}
\centering
	\begin{tabular}{|c|c|}
	\hline
	\textbf{symbol}&\textbf{value}\\
	\hline
	r&8\\
	\hline
	$\theta$&pi/2.5\\
	\hline
	d&7\\
	\hline
	A&(0,0)\\
	\hline
	B&(d,0)\\
	\hline
	D&(rcos$\theta$,rsin$\theta$)\\
	\hline
	C&(D/1.5)+B\\
	\hline
\end{tabular}
}
\end{document}
\fi

%
\item 
\label{cons/tri/5}
\iffalse
\documentclass[journal,10pt,twocolumn]{article}
\usepackage{graphicx}
\usepackage[margin=0.5in]{geometry}
\usepackage[cmex10]{amsmath}
\usepackage{array}
\usepackage{booktabs}
\usepackage{listings}
\title{\textbf{Line Assignment}}
\author{Bhavani Kanike}
\date{October 2022}

\providecommand{\norm}[1]{\left\lVert#1\right\rVert}
\providecommand{\abs}[1]{\left\vert#1\right\vert}
\let\vec\mathbf
\newcommand{\myvec}[1]{\ensuremath{\begin{pmatrix}#1\end{pmatrix}}}
\newcommand{\mydet}[1]{\ensuremath{\begin{vmatrix}#1\end{vmatrix}}}
\providecommand{\brak}[1]{\ensuremath{\left(#1\right)}}

\begin{document}

\maketitle
\paragraph{\textit{Problem Statement} 
\fi
ABCD is a quadrilateral in which $\vec{P}, \vec{Q}, \vec{R}$ and $\vec{S}$ are mid-points of the sides AB, BC, CD and DA (see Fig \ref{fig:9/8/2/1}). AC is a diagonal. 
		
Show that 
\begin{enumerate}
	\item $SR \parallel AC$ and $SR =\frac{1}{2} AC$
\item $PQ = SR$
\item $PQRS$ is a parallelogram.
\end{enumerate}
 	\begin{figure}
		\centering
 \includegraphics[width=\columnwidth]{chapters/9/8/2/1/figs/line1.pdf}
		\caption{}
		\label{fig:9/8/2/1}
  	\end{figure}
	\solution 
	Using 
	  \eqref{eq:section_formula},
	\begin{align}
		\label{eq:9/8/2/1}
		\begin{split}
		\vec{P} &= \frac{\vec{A}+\vec{B}}{2}\\
 \vec{Q} &= \frac{\vec{C}+\vec{B}}{2}\\
 \vec{R} &= \frac{\vec{C}+\vec{D}}{2}\\
 \vec{S} &= \frac{\vec{D}+\vec{A}}{2}
		\end{split}
	\end{align}
\begin{enumerate}
	\item
	Consequently, 
	\begin{align}
\vec{R}
		-\vec{S} &= \frac{\vec{C}-\vec{A}}{2}
		\\
		\implies SR &\parallel AC
	\end{align}
	Also, 
	\begin{align}
		\norm{\vec{R}
		-\vec{S}} &= \frac{\norm{\vec{C}-\vec{A}}}{2}
		\\
		\implies SR &= \frac{1}{2}AC
	\end{align}
\item 	From 
		\eqref{eq:9/8/2/1},
	\begin{align}
\vec{R}
		-\vec{S} = \vec{Q}-\vec{P}
	\end{align}
	which means that $PQRS$ is a parallelogram and $PQ = SR$.
\end{enumerate}
%
\iffalse
\begin{figure}[h]
\centering
\includegraphics[width=1\columnwidth]
\caption{Figure}
\label{fig:triangle}
\end{figure}

\section*{Solution}

$\boldsymbol Given :$  ABCD is a Quadrilateral P,Q,R and S are the midpoints of line AB,BC,CD,DA.We can obtain the points P,Q,R and S from A,B,C and D and are given by\\\\
\boldmath
\unboldmath
(3) To prove that PQRS is a parallelogram we need to prove  PQ // SR
To prove SR $\parallel$ PQ\\
Direction vector of line SR  $\boldsymbol {(R-S) =  \frac{(C-A)}{2}}$\\\\
Direction vector of line PQ  $\boldsymbol {(Q-P)= \frac{(C-A)}{2}}$\\\\
\begin{equation}
	\boldsymbol {(R-S) = (Q-P) = \frac{(C-A)}{2}}\\
\end{equation}
Since the direction vectors of line SR and PQ are in same direction\\\\
$SR \parallel PQ$\\
Therefore,
$\boldsymbol{ PQRS }$ is a parallelogram\\\\

	
(1)  Directional vector of line SR  = $\boldsymbol {(R-S)}$ = $\frac{\boldsymbol{(C-A)}}{2} $\\
Directional vector of line AC  = $\boldsymbol {(C-A)}$\\

It is observed that the constant k is $\frac{1}{2}$

Therefore
\begin{equation}
	SR \parallel AC
\end{equation} 

and from equation 1 
\begin{equation}
	\boldsymbol {SR = \frac{1}{2}AC}    
\end{equation}\\


(2)   To prove PQ = SR\\ 
		From euqation 1\\\\
\begin{equation}
		\boldsymbol{ (Q-P) = (R-S) = \frac{(C-A)}{2}}
\end{equation}
	 



\section{Execution}
The below python code realizes the construction:
\begin{lstlisting}
https://github.com/bhavani360/FWC_assignments
\end{lstlisting}
	
\section*{Construction}
The dimensions of the Quadrilateral ABCD are taken as below\\
{
\setlength\extrarowheight{2pt}
\centering
	\begin{tabular}{|c|c|}
	\hline
	\textbf{symbol}&\textbf{value}\\
	\hline
	r&8\\
	\hline
	$\theta$&pi/2.5\\
	\hline
	d&7\\
	\hline
	A&(0,0)\\
	\hline
	B&(d,0)\\
	\hline
	D&(rcos$\theta$,rsin$\theta$)\\
	\hline
	C&(D/1.5)+B\\
	\hline
\end{tabular}
}
\end{document}
\fi

\fi
%
\item Construct a triangle $ABC$ in which $\angle{B}, \angle{C}$ and  $a+b+c=K$ are given.
\label{cons/tri/4}
\\
\solution
\iffalse
\documentclass[journal,10pt,twocolumn]{article}
\usepackage{graphicx}
\usepackage[margin=0.5in]{geometry}
\usepackage[cmex10]{amsmath}
\usepackage{array}
\usepackage{booktabs}
\usepackage{listings}
\title{\textbf{Line Assignment}}
\author{Bhavani Kanike}
\date{October 2022}

\providecommand{\norm}[1]{\left\lVert#1\right\rVert}
\providecommand{\abs}[1]{\left\vert#1\right\vert}
\let\vec\mathbf
\newcommand{\myvec}[1]{\ensuremath{\begin{pmatrix}#1\end{pmatrix}}}
\newcommand{\mydet}[1]{\ensuremath{\begin{vmatrix}#1\end{vmatrix}}}
\providecommand{\brak}[1]{\ensuremath{\left(#1\right)}}

\begin{document}

\maketitle
\paragraph{\textit{Problem Statement} 
\fi
ABCD is a quadrilateral in which $\vec{P}, \vec{Q}, \vec{R}$ and $\vec{S}$ are mid-points of the sides AB, BC, CD and DA (see Fig \ref{fig:9/8/2/1}). AC is a diagonal. 
		
Show that 
\begin{enumerate}
	\item $SR \parallel AC$ and $SR =\frac{1}{2} AC$
\item $PQ = SR$
\item $PQRS$ is a parallelogram.
\end{enumerate}
 	\begin{figure}
		\centering
 \includegraphics[width=\columnwidth]{chapters/9/8/2/1/figs/line1.pdf}
		\caption{}
		\label{fig:9/8/2/1}
  	\end{figure}
	\solution 
	Using 
	  \eqref{eq:section_formula},
	\begin{align}
		\label{eq:9/8/2/1}
		\begin{split}
		\vec{P} &= \frac{\vec{A}+\vec{B}}{2}\\
 \vec{Q} &= \frac{\vec{C}+\vec{B}}{2}\\
 \vec{R} &= \frac{\vec{C}+\vec{D}}{2}\\
 \vec{S} &= \frac{\vec{D}+\vec{A}}{2}
		\end{split}
	\end{align}
\begin{enumerate}
	\item
	Consequently, 
	\begin{align}
\vec{R}
		-\vec{S} &= \frac{\vec{C}-\vec{A}}{2}
		\\
		\implies SR &\parallel AC
	\end{align}
	Also, 
	\begin{align}
		\norm{\vec{R}
		-\vec{S}} &= \frac{\norm{\vec{C}-\vec{A}}}{2}
		\\
		\implies SR &= \frac{1}{2}AC
	\end{align}
\item 	From 
		\eqref{eq:9/8/2/1},
	\begin{align}
\vec{R}
		-\vec{S} = \vec{Q}-\vec{P}
	\end{align}
	which means that $PQRS$ is a parallelogram and $PQ = SR$.
\end{enumerate}
%
\iffalse
\begin{figure}[h]
\centering
\includegraphics[width=1\columnwidth]
\caption{Figure}
\label{fig:triangle}
\end{figure}

\section*{Solution}

$\boldsymbol Given :$  ABCD is a Quadrilateral P,Q,R and S are the midpoints of line AB,BC,CD,DA.We can obtain the points P,Q,R and S from A,B,C and D and are given by\\\\
\boldmath
\unboldmath
(3) To prove that PQRS is a parallelogram we need to prove  PQ // SR
To prove SR $\parallel$ PQ\\
Direction vector of line SR  $\boldsymbol {(R-S) =  \frac{(C-A)}{2}}$\\\\
Direction vector of line PQ  $\boldsymbol {(Q-P)= \frac{(C-A)}{2}}$\\\\
\begin{equation}
	\boldsymbol {(R-S) = (Q-P) = \frac{(C-A)}{2}}\\
\end{equation}
Since the direction vectors of line SR and PQ are in same direction\\\\
$SR \parallel PQ$\\
Therefore,
$\boldsymbol{ PQRS }$ is a parallelogram\\\\

	
(1)  Directional vector of line SR  = $\boldsymbol {(R-S)}$ = $\frac{\boldsymbol{(C-A)}}{2} $\\
Directional vector of line AC  = $\boldsymbol {(C-A)}$\\

It is observed that the constant k is $\frac{1}{2}$

Therefore
\begin{equation}
	SR \parallel AC
\end{equation} 

and from equation 1 
\begin{equation}
	\boldsymbol {SR = \frac{1}{2}AC}    
\end{equation}\\


(2)   To prove PQ = SR\\ 
		From euqation 1\\\\
\begin{equation}
		\boldsymbol{ (Q-P) = (R-S) = \frac{(C-A)}{2}}
\end{equation}
	 



\section{Execution}
The below python code realizes the construction:
\begin{lstlisting}
https://github.com/bhavani360/FWC_assignments
\end{lstlisting}
	
\section*{Construction}
The dimensions of the Quadrilateral ABCD are taken as below\\
{
\setlength\extrarowheight{2pt}
\centering
	\begin{tabular}{|c|c|}
	\hline
	\textbf{symbol}&\textbf{value}\\
	\hline
	r&8\\
	\hline
	$\theta$&pi/2.5\\
	\hline
	d&7\\
	\hline
	A&(0,0)\\
	\hline
	B&(d,0)\\
	\hline
	D&(rcos$\theta$,rsin$\theta$)\\
	\hline
	C&(D/1.5)+B\\
	\hline
\end{tabular}
}
\end{document}
\fi



\end{enumerate}



\section{Linear Forms}
\subsection{Two Dimensions}

\renewcommand{\theequation}{\theenumi}
%\begin{enumerate}[label=\arabic*.,ref=\theenumi]
\begin{enumerate}[label=\thesubsection.\arabic*.,ref=\thesubsection.\theenumi]
\numberwithin{equation}{enumi}
\item The equation of a line  is given by  
\begin{align}
	\label{eq:normal_line}
   \vec{n}^{\top}\vec{x} = c
\end{align}
		where $\vec{n}$ is the normal vector of the line.
	\item The equation of a line with normal vector $\vec{n}$ and passing through a point $\vec{A}$ 
		is given by 
\begin{align}
    \label{eq:line_norm_eq}
%	\label{eq:normal_line_pt}
	\vec{n}^{\top}\brak{\vec{x}-\vec{A}} =0 
\end{align}
\item The equation of a line $L$ is also given by  
\begin{align}
	\label{eq:normal_line_orig}
   \vec{n}^{\top}\vec{x}  = 
	\begin{cases}
		0  & \vec{0} \in L
		 \\
		1 & \text{otherwise}
	\end{cases}
\end{align}
%	\item The equation of a line with normal vector $\vec{n}$ and passing through a point $\vec{A}$ 
%		is given by 
%\begin{align}
%    \label{eq:line_norm_eq-pt}
%%	\label{eq:normal_line_pt}
%	\vec{n}^{\top}\brak{\vec{x}-\vec{A}} =0 
%\end{align}
\item The parametric equation of a line  is given by  
\begin{align}
	\label{eq:dir_line}
	\vec{x} = \vec{A} + \lambda \vec{m}
\end{align}
		where $\vec{m}$ is the direction vector of the line and $\vec{A}$ is any point on the line.
  \item Let $\vec{A}$ and $\vec{B}$ be two points on a straight line and let $\vec{P}= \myvec{p_1\\p_2}$ be any point on it. If $p_2$ is known, then 
  \begin{align}
	  \vec{P}  &=	  \vec{A} + \frac{p_2 -\vec{e}_2^{\top}  \vec{A} }{\vec{e}_2^{\top}\brak{\vec{B} -\vec{A} }}\brak{\vec{B} -\vec{A} }
	  \label{eq:line-3pt}
  \end{align}
  \solution The equation of the line can be expressed in parametric from as 
  \begin{align}
	  \vec{x}  &=	  \vec{A} + \lambda \brak{\vec{B} -\vec{A} }
	  \\
	  \implies 
	  \vec{P}  &=	  \vec{A} + \lambda \brak{\vec{B} -\vec{A} }
	  \\
	  \implies 	   \vec{e}_2^{\top}\vec{P}  &=	\vec{e}_2^{\top}  \vec{A} + \lambda \vec{e}_2^{\top}\brak{\vec{B} -\vec{A} }
	  \\
	 \implies p_2 &=\vec{e}_2^{\top}  \vec{A} + \lambda \vec{e}_2^{\top}\brak{\vec{B} -\vec{A} }
	 \\
	  \text{or, } \lambda &= \frac{p_2 -\vec{e}_2^{\top}  \vec{A} }{\vec{e}_2^{\top}\brak{\vec{B} -\vec{A} }}
  \end{align}
	  yielding \eqref{eq:line-3pt}.
	\item The distance from a point $\vec{P}$ to the line  in 
	\eqref{eq:normal_line}
	is given by 
\begin{align}
  \label{conics/30/lemma}
%	\label{eq:line_dist_2d}
	d = \frac{\abs{   \vec{n}^{\top}\vec{P}-c }}{\norm{\vec{n}}}	
\end{align}
		\solution Without loss of generality, let $\vec{A}$ be the foot of the perpendicular from $\vec{P}$ to the line in 
	\eqref{eq:dir_line}.  The equation of the normal to 
	\eqref{eq:normal_line} can then be expressed as 
\begin{align}
	\label{eq:dir_line_normal_dist}
	\vec{x} &= \vec{A} + \lambda \vec{n}
	\\
	\implies 
	\vec{P}- \vec{A} &=  \lambda \vec{n}
	\label{eq:dir_line_normal_dist_pa}
\end{align}
$\because \vec{P}$ lies on 
		\eqref{eq:dir_line_normal_dist}.
From the above, the desired distance can be expressed as 
\begin{align}
d = 	\norm{\vec{P}- \vec{A}}= \abs{\lambda} \norm{\vec{n}}
	\label{eq:dir_line_normal_dist_pa_d}
\end{align}
From 
	\eqref{eq:dir_line_normal_dist_pa},
\begin{align}
	\vec{n}^{\top}
	\brak{\vec{P}- \vec{A}} &=  \lambda \vec{n}^{\top}\vec{n} = \lambda\norm{\vec{n}}^2
	\\
	\implies \abs{\lambda}&= \frac{\abs{\vec{n}^{\top}
	\brak{\vec{P}- \vec{A}}}}{\norm{\vec{n}}^2} 
\end{align}
	Substituting the above in \eqref{eq:dir_line_normal_dist_pa_d} and using 
	the fact that 
\begin{align}
   \vec{n}^{\top}\vec{A} = c
\end{align}
from 	\eqref{eq:normal_line}, yields 
  \eqref{conics/30/lemma}
%	\eqref{eq:line_dist_2d}.

	\item The distance from the origin to the line  in 
	\eqref{eq:normal_line}
	is given by 
\begin{align}
	\label{eq:dist_line_2d_orig}
	d = \frac{\abs{   c }}{\norm{\vec{n}}}	
\end{align}
\item The distance between the parallel lines 
\begin{align}
	\label{eq:parallel_lines}
	\begin{split}
		\vec{n}^{\top}\vec{x} &= c_1
		\\
		\vec{n}^{\top}\vec{x} &= c_2
	\end{split}
\end{align}
is given by 
\begin{align}
	\label{eq:dist_lines_2d}
	d = \frac{\abs{   c_1-c_2 }}{\norm{\vec{n}}}	
\end{align}
\item The equation of the line perpendicular to 
	\eqref{eq:normal_line}
		and passing through the point $\vec{P}$ is given by 
\begin{align}
	\vec{m}^{\top}\brak{\vec{x}-\vec{P}}  = 0
\end{align}
\item The foot of the perpendicular from $\vec{P}$ to the line in 
	\eqref{eq:normal_line}
	is given by 
\begin{align}
	\label{eq:normal_line_foot}
	\myvec{ \vec{m} & \vec{n}}^{\top}\vec{x}= \myvec{\vec{m}^{\top}\vec{P}\\ c }  
\end{align}
% 
\solution From
	\eqref{eq:normal_line} and 
\eqref{eq:line_norm_eq}
%	\eqref{eq:normal_line_pt} 
the foot of the perpendicular satisfies the equations 
\begin{align}
	\vec{n}^{\top}\vec{x} &= c
	\\
	\vec{m}^{\top}\brak{\vec{x}-\vec{P} }&=0 
\end{align}
where $\vec{m}$ is the direction vector of the given line.  Combining the above into a matrix equation results in 
	\eqref{eq:normal_line_foot}.

%\item ({\em Reflection }) Assuming that straight lines work as a plane mirror for a point, find the image of the point $\vec{P}=\myvec{1\\2}$ in the line 
%%
%\begin{align}
%L: \quad \myvec{1 & -3}\vec{x}  = -4.
%\end{align}
%\solution From the given equation, the line parameters are
%\begin{align}
%\vec{n} = \myvec{1 \\ -3}, c =  -4, \vec{m} = \myvec{3 \\ -1}
%\end{align}
%
%Let $\vec{R}$ be the reflection of $\vec{P}$ such that $PR$ bisects the line $L$ at $\vec{Q}$. Then $\vec{Q}$ bisects $PR$.  
%This leads to the following equations
%\begin{align}
%\label{eq:reflect_bisect}
%2\vec{Q} &= \vec{P}+\vec{R}
%\\
%\label{eq:reflect_Q}
%\vec{n}^{\top}\vec{Q} &= c \quad \because \vec{Q} \text{ lies on the given line}
%\\
%\label{eq:reflect_R}
%\vec{m}^{\top}\vec{R} &= \vec{m}^{\top}\vec{P} \quad \because \vec{m}\perp \vec{P} - \vec{R}
%\end{align}
%%
%%where 
%%$\vec{m}$ is the direction vector of $L$.  
%From \eqref{eq:reflect_bisect} and \eqref{eq:reflect_Q},
%\begin{align}
%\label{eq:reflect_bisectQ}
%\vec{n}^{\top}\vec{R}  &= 2c - \vec{n}^{\top}\vec{P}
%\end{align}
%%
%From \eqref{eq:reflect_bisectQ} and \eqref{eq:reflect_R},
%\begin{align}
%\label{eq:reflect_bisectQR}
%\myvec{\vec{m} & \vec{n}}^T\vec{R} &= \myvec{\vec{m} & -\vec{n}}^T\vec{P}+ \myvec{0 \\ 2c}
%\end{align}
%%
%Letting 
%\begin{align}
%\label{eq:reflect_mat}
%\vec{V}=  \myvec{\vec{m} & \vec{n}}
%\end{align}
%with the condition that $\vec{m},\vec{n}$ are orthonormal, i.e.
%\begin{align}
%\label{eq:reflect_ortho}
%\vec{V}^T\vec{V}=  \vec{I}
%\end{align}
%%
%Noting that 
%\begin{align}
%\label{eq:reflect_trans}
%\myvec{\vec{m} & -\vec{n}} &= \myvec{\vec{m} & \vec{n}} \myvec{1 & 0 \\ 0 & -1},
%\end{align}
%\eqref{eq:reflect_bisectQR} can be expressed as
%%
%\begin{align}
%\label{eq:reflect_}
%\vec{V}^T\vec{R} &=  \sbrak{\vec{V}\myvec{1 & 0 \\ 0 & -1}}^T\vec{P}+\myvec{0 \\ 2c}
%\\
%\implies \vec{R} &= \sbrak{\vec{V}\myvec{1 & 0 \\ 0 & -1}\vec{V}^{-1}}^T\vec{P}+ \vec{V}\myvec{0 \\ 2c}
%\\
% &=\vec{V}\myvec{1 & 0 \\ 0 & -1}\vec{V}^T \vec{P}+2c \vec{n}
%\label{eq:reflect_mat_final}
%\end{align}
%upon substituting from \eqref{eq:reflect_mat} in \eqref{eq:reflect_mat_final}.
%It can be verified that 
%%\item Show that, for any $\vec{m},\vec{n}$, 
%the reflection is also given by
%\begin{align}
%%\label{eq:reflect_bisect}
%\vec{R} &= \myvec{\vec{m} & \vec{n}}\myvec{1 & 0 \\ 0 & -1}\myvec{\vec{m} & \vec{n}}^T \vec{P}+2c \vec{n}
%\\
% &= \myvec{\vec{m} & -\vec{n}}\myvec{\vec{m}^T \\ \vec{n}^T} \vec{P}+2c \vec{n}
%\\
%\implies \vec{R}&= \brak{\vec{m}\vec{m}^T-\vec{n}\vec{n}^T}\vec{P} + 2c \vec{n} 
%\label{eq:reflect_orth_vec}
%\end{align}
%If $\vec{m}, \vec{n}$ are not orthonormal, \eqref{eq:reflect_orth_vec}
%can be expressed as
%\begin{align}
% \frac{\vec{R}}{2}= \frac{\vec{m}\vec{m}^T-\vec{n}\vec{n}^T}{\vec{m}^T\vec{m}+\vec{n}^T\vec{n}}\vec{P} + c \frac{\vec{n}}{\norm{\vec{n}}^2}
%\label{eq:reflect_non_orth_vec}
%\end{align}
%

\end{enumerate}

\subsection{Three Dimensions}

%\renewcommand{\theequation}{\theenumi}
%\begin{enumerate}[label=\arabic*.,ref=\theenumi]
\begin{enumerate}[label=\thesection.\arabic*.,ref=\thesection.\theenumi]
%\begin{enumerate}
%\numberwithin{equation}{enumi}
%\item The area of a triangle with vertices $\vec{A}, \vec{B}, \vec{C}$ is given by 
%\begin{align}
%  \label{eq:area3d}
% \frac{1}{2} \norm{\brak{\vec{A} - \vec{B}} \times \brak{\vec{A} - \vec{C}}}
%\end{align}

\item Points $\vec{A}, \vec{B}, \vec{C}$ are on a line if 
\begin{align}
  \label{eq:line_rank}
  \text{rank}\myvec{\vec{A} \\ \vec{B} \\ \vec{C} }  = 1
\end{align}
\item Points $\vec{A}, \vec{B}, \vec{C}, \vec{D}$ form a paralelogram if 
\begin{align}
  \label{eq:parallelgm_rank}
  \text{rank}\myvec{\vec{A} \\ \vec{B} \\ \vec{C} \\ \vec{D}  }  = 1, 
  \text{rank}\myvec{\vec{A} \\ \vec{B} \\ \vec{C} }  = 2
\end{align}
\item The equation of a line  is given by  
	\eqref{eq:dir_line}
	\item The equation of a plane is given by
	\eqref{eq:normal_line}
	\item The distance from the origin to the line  in 
	\eqref{eq:normal_line}
	is given by 
	\eqref{eq:dist_line_2d_orig}
\item The distance from a point $\vec{P}$  to the line in 
	\eqref{eq:dir_line} is given by 
\begin{align}
	\label{dist_3d_def_final}
		d = \norm{\vec{A} -\vec{P}}^2 - \frac{\cbrak{\vec{m}^{\top}\brak{\vec{A}-\vec{P} 
	}}^2}{\norm{\vec{m}}^2}
%	d =\norm{\vec{A}  -\vec{P}
% -\frac{\vec{m}^{\top}\brak{\vec{A} 
%			-\vec{P}}}
%			{ \norm{\vec{m}}^2}
%	\vec{m}}
		\end{align}
		\solution
%		\solution{\title{Solution:}
\begin{align}
	\label{dist_3d_def}
	d\brak{\lambda } &=\norm{\vec{A} + \lambda \vec{m}-\vec{P}}
	\\
\implies 	d^2\brak{\lambda } &=\norm{\vec{A} + \lambda \vec{m}-\vec{P}}^2
\end{align}
which can be simplified to obtain 
	\begin{multline}
d^2\brak{\lambda } =\lambda^2 \norm{\vec{m}}^2+2\lambda \vec{m}^{\top}\brak{\vec{A} 
		-\vec{P}}
		\\
		+\norm{\vec{A} -\vec{P}}^2
	\end{multline}
which is of the form 
\begin{align}
	\label{dist_3d_def_quad}
	d^2\brak{\lambda } &=a \lambda^2 + 2b\lambda +c
	\\
	&=a \cbrak{\brak{\lambda+ \frac{b}{a}}^2 +\sbrak{\frac{c}{a}-\brak{\frac{b}{a}}^2 }}
\end{align}
with 
\begin{align}
	\label{dist_3d_def_quad_abc}
	a = \norm{\vec{m}}^2, b = \vec{m}^{\top}\brak{\vec{A} 
		-\vec{P}}, c = 
		\norm{\vec{A} -\vec{P}}^2
\end{align}
which can be expressed as 
%		\begin{multline}
%			d^2\brak{\lambda } =\norm{\vec{m}}^2\brak{\lambda + \frac{\vec{m}^{\top}\brak{\vec{A}-\vec{P} }}{\vec{m}}^2}}^2 +2\lambda \vec{m}^{\top}\brak{\vec{A} 
%			-\vec{P}}
%			\\
%			+\norm{\vec{A} -\vec{P}}^2
%		\end{multline}
		From the above, $d^2\brak{\lambda}$ is smallest when upon substituting from 
	\eqref{dist_3d_def_quad_abc}
\begin{align}
	\label{dist_3d_def_quad_small}
	\lambda+ \frac{b}{2a} &= 0 \implies \lambda = - \frac{b}{2a}
	\\
	&= -\frac{\vec{m}^{\top}\brak{\vec{A} 
			-\vec{P}}}
			{ \norm{\vec{m}}^2}
	%		\label{dist_3d_lam}
\end{align}
and consequently, 
\begin{align}
	d_{\min}\brak{\lambda } &=a \brak{\frac{c}{a}-\brak{\frac{b}{a}}^2 } 
	\\
	&=c - \frac{b^2}{a }
\end{align}
yielding
	\eqref{dist_3d_def_final} after substituting from 
	\eqref{dist_3d_def_quad_abc}.
%From 	\eqref{dist_3d_def} and \eqref{dist_3d_lam}, 
%	\eqref{dist_3d_def} is obtained.
\item The distance between the parallel planes 
	\eqref{eq:parallel_lines}
	is given by 
	\eqref{eq:dist_lines_2d}.
\item The plane 
		\begin{align}
		\vec{n}^{\top}
			\vec{x} = c
			\label{eq:plain_contain}
		\end{align}
		contains the line 
		\begin{align}
			\vec{x} = \vec{A}+\lambda \vec{m}
			\label{eq:line_contain}
		\end{align}
		if 
		\begin{align}
		\vec{m}^{\top}\vec{n} = 0
			\label{eq:line_plain_contain}
		\end{align}
		\solution Any point on the line 
			\eqref{eq:line_contain}
			should also satisfy 
			\eqref{eq:plain_contain}.  Hence, 
		\begin{align}
			\vec{n}^{\top}\brak{\vec{A}+\lambda \vec{m}} &= \vec{n}^{\top}\vec{A}=c
		\end{align}
		which can be simplified to obtain
			\eqref{eq:line_plain_contain}
		\item The foot of the perpendicular from a point $\vec{P}$ to the plane 
		\begin{align}
			\vec{n}^{\top}\vec{x} =c
		\end{align}
		is given by 
\begin{align}
	\vec{x} &= \vec{P} + \frac{c - \vec{n}^{\top}\vec{P}}{\norm{\vec{n}}^2}
\vec{n}
	\label{eq:foot_perp_pt_plane}
\end{align}
		\\
		\solution The equation of the line perpendicular to the given plane and passing through $\vec{P}$ is 
		\begin{align}
			\vec{x} = \vec{P} + \lambda 	\vec{n}
		\end{align}
		From 
	\eqref{eq:dir_line_plane_isect}, the intersection of the above line with the given plane is 
	\eqref{eq:foot_perp_pt_plane}.
	\iffalse
\begin{align}
	\vec{x} &= \vec{P} + \frac{c - \vec{n}^{\top}\vec{P}}{\norm{\vec{n}}^2}
\vec{n}
	\label{eq:foot_perp_pt_plane}
\end{align}
\fi
\item The image of a point $\vec{P}$ with respect to the plane 
		\begin{align}
			\vec{n}^{\top}\vec{x} =c
		\end{align}
		is given by 
		\begin{align}
			\vec{R} &=
	  \vec{P} + 2\frac{c - \vec{n}^{\top}\vec{P}}{\norm{\vec{n}}^2}
			\label{eq:image_pt_plane}
		\end{align}
		\solution Let $\vec{R}$ be the desired image.  Then, subtituting the expression for the  foot of the perpendicular from $\vec{P}$ to the given plane using 
	\eqref{eq:foot_perp_pt_plane},
		\begin{align}
			\frac{\vec{P}+\vec{R}}{2} &=
	  \vec{P} + \frac{c - \vec{n}^{\top}\vec{P}}{\norm{\vec{n}}^2}
		\end{align}
		\item Let a plane pass through the points $\vec{A},\vec{B}$ and be perpendicular to the plane 
		\begin{align}
		\vec{n}^{\top}\vec{x} =c 
			\label{eq:plane_3d_2pt_perp_given}
		\end{align}
		Then the equation of this plane is given by 
		\begin{align}
		\vec{p}^{\top}\vec{x} = 1
			\label{eq:plane_3d_2pt}
		\end{align}
		where
		\begin{align}
			\vec{p} = 		\myvec{\vec{A} & \vec{B} & \vec{n}}^{-\top}  \myvec{1 \\ 1 \\ 0}
			\label{eq:plane_3d_2pt_perp_norm}
		\end{align}
	\solution From the given information, 
		\begin{align}
			\vec{p}^{\top}\vec{A} &=d 
			\\
			\vec{p}^{\top}\vec{B} &=d 
			\\
			\vec{p}^{\top}\vec{n} &= 0
			\label{eq:plane_3d_2pt_perp_system}
		\end{align}
		$\because$ the normal vectors to the two planes will also be perpendicular.  The system of equations in 
			\eqref{eq:plane_3d_2pt_perp_system}
			can be expressed as the matrix equation
		\begin{align}
			\myvec{\vec{A} & \vec{B} & \vec{n}}^{\top}\vec{p} = d\myvec{1 \\ 1 \\ 0}
			\label{eq:plane_3d_2pt_perp_system_temp}
		\end{align}
		which yields 
			\eqref{eq:plane_3d_2pt_perp_norm}
			upon normalising with $d$.
		\item The intersection of the line represented by 
	\eqref{eq:dir_line}
	with the plane represented by 
	\eqref{eq:normal_line}
	is given by 
\begin{align}
	\label{eq:dir_line_plane_isect}
	\vec{x} &= \vec{A} + \frac{c - \vec{n}^{\top}\vec{A}}{\vec{n}^{\top}\vec{m}}
\vec{m}
\end{align}
\solution From 
	\eqref{eq:dir_line}
	and 
	\eqref{eq:normal_line},
\begin{align}
	\vec{x} &= \vec{A} + \lambda \vec{m}
	\\
	\vec{n}^{\top}\vec{x} &= c
	\\
	\implies 
	\vec{n}^{\top}\brak{\vec{A} + \lambda \vec{m}}&= c
	\label{eq:dir_line_plane_inter}
\end{align}
which can be simplified to obtain
\begin{align}
	\vec{n}^{\top}\vec{A} + \lambda 	\vec{n}^{\top}\vec{m}&= c
	\\
	\implies \lambda &= \frac{c - \vec{n}^{\top}\vec{A}}{\vec{n}^{\top}\vec{m}}
\end{align}
Substituting the above in 
	\eqref{eq:dir_line_plane_inter}
	yields
	\eqref{eq:dir_line_plane_isect}.
\item The foot of the perpendicular from the point $\vec{P}$ to the line  represented by 
	\eqref{eq:dir_line}
	is given by 
\begin{align}
	\label{eq:plane_line_foot_ans}
	\vec{x} &= \vec{A} + \frac{ \vec{m}^{\top}\brak{\vec{P} - \vec{A}}}{\norm{\vec{m}}^2}
\vec{m}
\end{align}
\solution  Let the equation of the line be 
\begin{align}
	\label{eq:dir_line_foot}
	\vec{x} &= \vec{A} + \lambda \vec{m}
\end{align}
	The equation of the plane perpendicular to the given line passing through $\vec{P}$ is given by
\begin{align}
	\label{eq:plane_line_foot}
	\vec{m}^{\top}\brak{\vec{x}-\vec{P}}  &= 0
	\\
	\implies \vec{m}^{\top}\vec{x}  &= \vec{m}^{\top}\vec{P}
\end{align}
The desired foot of the perpendicular is the intersection of 
	\eqref{eq:dir_line_foot} with 
	\eqref{eq:plane_line_foot}
	which can be obtained from 
	\eqref{eq:dir_line_plane_isect}
	as 
	\eqref{eq:plane_line_foot_ans}
\item The foot of the perpendicular from a point $\vec{P}$ to a plane is $\vec{Q}$.  The equation of the plane is given by 
\begin{align}
	\label{eq:plane_foot_perp}
	\brak{\vec{P}-\vec{Q}}^{\top}\brak{\vec{x}-\vec{Q}} = 0
\end{align}
	\solution  The normal vector to the plane is given by 
\begin{align}
	\vec{n}= \vec{P}-\vec{Q} 
\end{align}
	Hence, the equation of the plane is
	\eqref{eq:plane_foot_perp}.
\item Let $\vec{A}, \vec{B}, \vec{C}$ be  points on a plane.  The equation of the plane is then given by 	
\begin{align}
	\myvec{	\vec{A} & \vec{B}& \vec{C}}^{\top} \vec{n}= \myvec{1\\1\\1}
	\label{eq:plane_3pt}
\end{align}
\solution Let the equation of the plane be 
\begin{align}
	\vec{n}^{\top}	\vec{x} &= 1
\end{align}
Then 
\begin{align}
	\vec{n}^{\top}	\vec{A} &= 1
	\\
	\vec{n}^{\top}	\vec{B} &= 1
	\\
	\vec{n}^{\top}	\vec{C} &= 1
\end{align}
which can be combined to obtain 
	\eqref{eq:plane_3pt}.
\item The lines 
    \begin{align}
        \vec{x} = \vec{x_1} + \lambda_1\vec{m_1} \label{eq:chapters/12/11/2/16/L1-gen} \\
        \vec{x} = \vec{x_2} + \lambda_2\vec{m_2} \label{eq:chapters/12/11/2/16/L2-gen}
    \end{align}
    intersect if
    \begin{align}
    %    \lambda_1\vec{m_1} - \lambda_2\vec{m_2} &= \vec{x_2} - \vec{x_1} \\
       \vec{M}\bm{\lambda} &= \vec{x_2} - \vec{x_1}
        \label{eq:chapters/12/11/2/16/intersect-cond}
    \end{align}
    where
    \begin{align}
        \vec{M} \triangleq \myvec{\vec{m_1} & \vec{m_2}} \label{eq:chapters/12/11/2/16/M-def} \\
        \bm{\lambda} \triangleq \myvec{\lambda_1\\-\lambda_2}
        \label{eq:chapters/12/11/2/16/lambda-def}
    \end{align}
\item 
    \iffalse
    Let 
    \begin{align}
        \vec{x} = \vec{x_1} + \lambda_1\vec{m_1} \label{eq:chapters/12/11/2/16/L1-gen} \\
        \vec{x} = \vec{x_2} + \lambda_2\vec{m_2} \label{eq:chapters/12/11/2/16/L2-gen}
    \end{align}
be two skew lines. 
\fi
	The closest points on two skew lines are given by 
    \begin{align}
	    \vec{M}^\top \vec{M}\bm{\lambda} = \vec{M}^\top\brak{\vec{x_2}-\vec{x_1}}
        \label{eq:chapters/12/11/2/16/lambda-eqn}
    \end{align}
	\solution
	\iffalse
    If these lines intersect, then
    If the lines are skew,
    \fi
    For the lines defined in \eqref{eq:chapters/12/11/2/16/L1-gen} and \eqref{eq:chapters/12/11/2/16/L2-gen},
Suppose the closest points on both lines are
    \begin{align}
        \vec{A} = \vec{x_1} + \lambda_1\vec{m_1} \label{eq:chapters/12/11/2/16/a-def} \\
        \vec{B} = \vec{x_2} + \lambda_2\vec{m_2}
        \label{eq:chapters/12/11/2/16/b-def}
    \end{align}
    Then, $AB$ is perpendicular to both lines, hence
    \begin{align}
        \vec{m_1}^\top\brak{\vec{A}-\vec{B}} = 0 \\
        \vec{m_2}^\top\brak{\vec{A}-\vec{B}} = 0 \\
        \implies \vec{M}^\top\brak{\vec{A}-\vec{B}} = \vec{O}
        \label{eq:chapters/12/11/2/16/perp-vec}
    \end{align}
    Using \eqref{eq:chapters/12/11/2/16/a-def} and \eqref{eq:chapters/12/11/2/16/b-def} in \eqref{eq:chapters/12/11/2/16/perp-vec},
    \begin{align}
        \vec{M}^\top\brak{\vec{x_1}-\vec{x_2} + \vec{M}\bm{\lambda}} = \vec{0} \\
    \end{align}
    yielding
        \ref{eq:chapters/12/11/2/16/lambda-eqn}.
%\renewcommand{\theequation}{\theenumi}
%%\begin{enumerate}[label=\arabic*.,ref=\theenumi]
%\begin{enumerate}[label=\thesubsection.\arabic*.,ref=\thesubsection.\theenumi]
%\numberwithin{equation}{enumi}
%
\item (Parallelogram Law)  Let $\vec{A}, \vec{B}, \vec{D}$ be three vertices of a parallelogram.  Then the vertex $\vec{C}$ is given by 
\begin{align}
  \label{eq:pgm_law}
  \vec{C} = \vec{B}+\vec{C} - \vec{A}
\end{align}
		\solution Shifting $\vec{A}$ to the origin, we obtain a parallelogram with corresponding vertices 
\begin{align}
  \label{eq:pgm_law_org_vert}
  \vec{0}, \vec{B}-\vec{A}, \vec{D} - \vec{A}
\end{align}
The fourth vertex of this parallelogram is then obtained as 
\begin{align}
  \label{eq:pgm_law_org}
	\brak{\vec{B}-\vec{A}}+\brak{ \vec{D} - \vec{A}} = \vec{D}+ \vec{B} - 2\vec{A}
\end{align}
Shifting the origin to $\vec{A}$, the fourth vertex is obtained as 
\begin{align}
  \label{eq:pgm_law_org_C}
		 \vec{C} &= \vec{D}+ \vec{B} - 2\vec{A}+\vec{A} 
		 \\
	 &=
	 \vec{D}+ \vec{B} - \vec{A} 
\end{align}
\item (Affine Transformation) Let $\vec{A},\vec{C}$, be opposite vertices of a square. The other two points can be obtained as  
\begin{align}
  \label{eq:square_points}
  \vec{B} = \frac{\norm{\vec{A}-\vec{C}}}{\sqrt{2}} \vec{P}\vec{e}_1+\vec{A}
  \\
  \vec{D} = \frac{\norm{\vec{A}-\vec{C}}}{\sqrt{2}} \vec{P}\vec{e}_2+\vec{A}
\end{align}
where 
\begin{align}
	\vec{P} = \myvec{\cos \brak{\theta-\frac{\pi}{4}} & \sin  \brak{\theta-\frac{\pi}{4}} \\ \sin \brak{\theta-\frac{\pi}{4}} & \cos \brak{\theta-\frac{\pi}{4}}}
\end{align}
and 
\begin{align}
	\cos\theta = \frac{\brak{\vec{C}-\vec{A}}^{\top}\vec{e}_1}{\norm{\vec{A}-\vec{C}}\norm{\vec{e}_1}}
\end{align}
\end{enumerate}

\section{Quadratic Forms}
%\numberwithin{equation}{subsubsection}
%\numberwithin{equation}{subsection}
\subsection{Conic equation }

\numberwithin{equation}{subsection}
\subsection{Definitions}
\begin{definition}
	The {\em affine} transformation is given by 
    \begin{align}
	    \vec{x} &= \vec{P}\vec{y}+\vec{c} \quad \text{(Affine Transformation)}
\label{eq:conic_affine}
    \end{align}
	where $\vec{P}$ is invertible.
\end{definition}
\begin{definition}
	The eigenvalue decomposition of a symmetric matrix $\vec{V}$ is given by 
	%\cite{banchoff}
    \begin{align}
      \label{eq:conic_parmas_eig_def}
      \vec{P}^{\top}\vec{V}\vec{P} &= \vec{D}. \quad \text{(Eigenvalue Decomposition)}
      \\
      \vec{D} &= \myvec{\lambda_1 & 0\\ 0 & \lambda_2}, 
      \\
      \vec{P} &= \myvec{\vec{p}_1 & \vec{p}_2}, \quad \vec{P}^{\top}=\vec{P}^{-1},
      \label{eq:eigevecP}
    \end{align}
\end{definition}
\subsection{The Quadratic Form}

\begin{definition}
  Let $\vec{q}$ be a point such that the ratio of its distance from a fixed point $\vec{F}$ and the distance ($d$) from a fixed line 
	\begin{align}
L: \vec{n}^{\top}\vec{x}=c 
	\end{align}
		is constant, given by 
\label{conics/30/def}
\begin{align}
\frac{\norm{\vec{q}-\vec{F}}}{d} = e    
\end{align}
The locus of $\vec{q}$ is known as a conic section. The line $L$ is known as the directrix and the point $\vec{F}$ is the focus. $e$ is defined to be 
the eccentricity of the conic.  
\begin{enumerate}
    \item For $e = 1$, the conic is a parabola
    \item For $e < 1$, the conic is an ellipse
    \item For $e > 1$, the conic is a hyperbola
\end{enumerate}
\end{definition}
\begin{theorem}
The equation of  a conic with directrix $\vec{n}^{\top}\vec{x} = c$, eccentricity $e$ and focus $\vec{F}$ is given by 
\begin{align}
    \label{eq:conic_quad_form}
    \vec{x}^{\top}\vec{V}\vec{x}+2\vec{u}^{\top}\vec{x}+f=0
    \end{align}
where     
\begin{align}
  \label{eq:conic_quad_form_v}
\vec{V} &=\norm{\vec{n}}^2\vec{I}-e^2\vec{n}\vec{n}^{\top}, 
\\
\label{eq:conic_quad_form_u}
\vec{u} &= ce^2\vec{n}-\norm{\vec{n}}^2\vec{F}, 
\\
\label{eq:conic_quad_form_f}
f &= \norm{\vec{n}}^2\norm{\vec{F}}^2-c^2e^2
%\\
    \end{align}
    
% \begin{align}
% \vec{x}^{\top}(t\vec{I}-\vec{n}\vec{n}^{\top})\vec{x}+2(c\vec{n}-t\vec{F})^{\top}\vec{x}+t\norm{\vec{F}}^2-c^2&=0
% \end{align}
%
%and 
% where 
% \begin{align}
%     %t=\frac{\norm{\vec{n}}^2}{e^2}
%     \norm{\vec{n}} = 1
% \end{align}
%\end{theorem}
\end{theorem}
\begin{proof}
  Using Definition \ref{conics/30/def} and Lemma \ref{conics/30/lemma},  for any point $\vec{x}$ on the conic,
\begin{align}
\norm{\vec{x}-\vec{F}}^2&=e^2 \frac{\brak{{\vec{n}^{\top}\vec{x} - c}}^2}{\norm{\vec{n}}^2}\label{conics/30/eq:1} \\
\implies \norm{\vec{n}}^2\brak{\vec{x}-\vec{F}}^{\top}\brak{\vec{x}-\vec{F}}&=e^2\brak{\vec{n}^{\top}\vec{x} - c}^2
\\
\implies \norm{\vec{n}}^2\brak{\vec{x}^{\top}\vec{x}-2\vec{F}^{\top}\vec{x}+\norm{\vec{F}}^2}&=e^2\brak{c^2+\brak{\vec{n}^{\top}\vec{x} }^2-2c\vec{n}^{\top}\vec{x}} \\
&=e^2\brak{c^2+\brak{\vec{x}^{\top}\vec{n}\vec{n}^{\top}\vec{x} }-2c\vec{n}^{\top}\vec{x}}
% t\vec{x}^{\top}\vec{x}-(\vec{n}^{\top}\vec{x} )^2-2t\vec{F}^{\top}\vec{x}+2c\vec{n}^{\top}\vec{x}=c^2-t\norm{\vec{F}}^2\\
% t\vec{x}^{\top}\vec{I}\vec{x}-\vec{n}^{\top}\vec{x} \vec{n}^{\top}\vec{x}+2(c\vec{n}-t\vec{F})^{\top}\vec{x}=c^2-t\norm{\vec{F}}^2\\
% \vec{x}^{\top}(t\vec{I}-\vec{n}\vec{n}^{\top})\vec{x}+2(c\vec{n}-t\vec{F})^{\top}\vec{x}+t\norm{\vec{F}}^2-c^2=0
\end{align}
%
which can be expressed as \eqref{eq:conic_quad_form} after simplification.

% See Appendix \ref{app:conicdef}
\end{proof}
\begin{theorem}
  The eccentricity, directrices and foci of \eqref{eq:conic_quad_form} are given by 
%  \eqref{eq:conic_quad_form_e} -
%  \eqref{eq:conic_quad_form_F} 
%  \begin{figure*}[!hb]
%	  \centering
%	  \hrule
\begin{align}
  \label{eq:conic_quad_form_e} 
  e&= \sqrt{1-\frac{\lambda_1}{\lambda_2}}
\\
\label{eq:conic_quad_form_nc} 
	\begin{split}
  \vec{n}&= \sqrt{\lambda_2}\vec{p}_1,  
  \\
	c &= 
  \begin{cases}
    \frac{e\vec{u}^{\top}\vec{n} \pm \sqrt{e^2\brak{\vec{u}^{\top}\vec{n}}^2-\lambda_2\brak{e^2-1}\brak{\norm{\vec{u}}^2 - \lambda_2 f}}}{\lambda_2e\brak{e^2-1}} & e \ne 1
    \\
    \frac{\norm{\vec{u}}^2 - \lambda_2 f   }{2\vec{u}^{\top}\vec{n}} & e = 1
  \end{cases}
	\end{split}
  \\
  \label{eq:conic_quad_form_F} 
  \vec{F}  &= \frac{ce^2\vec{n}-\vec{u}}{\lambda_2}
\end{align}  
%  \end{figure*}
\end{theorem}
\begin{proof}
	See Appendix \ref{app:conic-parameters}
\end{proof}

\begin{theorem}
\eqref{eq:conic_quad_form} represents 
	\begin{enumerate}
		\item a parabola for $\mydet{\vec{V}} = 0 $,
		\item ellipse for $\mydet{\vec{V}} > 0 $ and 
		\item hyperbola for $\mydet{\vec{V}} < 0 $.
	\end{enumerate}
%		\item a pair of straight lines if
%\begin{align}
%\mydet{
%\vec{V}&\vec{u}
%\\
%\vec{u}^{\top}&f
%}
%=  0, \quad \mydet{\vec{V}} < 0
%\label{eq:quad_forms_pair_det}
%\end{align}
%			else, it represents
\end{theorem}
\begin{proof}
  From \eqref{eq:conic_quad_form_e},
\begin{align}
  \frac{\lambda_1}{\lambda_2} = 1 - e^2
\end{align}
Also, 
\begin{align}
	\mydet{\vec{V}} =   \lambda_1\lambda_2 
\end{align}
	yielding Table \ref{table:det}
\begin{table}[!h]
\centering
\input{tables/det.tex}
	\caption{}
\label{table:det}
\end{table}
\end{proof}

\subsection{Circles}
%\subsection{The Quadratic Form}
%\numberwithin{equation}{subsection}
%\begin{enumerate}
\begin{enumerate}[label=\thesection.\arabic*.,ref=\thesection.\theenumi]
	\item The equation of a circle is given by 
	\label{prop:circ-eq}
\begin{align}
	\norm{\vec{x}}^2 + 2 \vec{u}^{\top}\vec{x} + f = 0
	\label{eq:circ-eq}
\end{align}
\item For a circle with centre $\vec{c}$ and radius r,
\begin{align}
	\vec{u} = -\vec{c}, f = \norm{\vec{u}}^2 - r^2
	\label{eq:circ-cr}
\end{align}
\item The equation of the common chord of intersection of two  circles is given by 
\begin{align}
	   \vec{u}_1^{\top}\vec{x} 
	   -\vec{u}_2^{\top}\vec{x} + f_1 f_2 = 0
	\label{eq:circ-chord}
\end{align}

\end{enumerate}


%\subsection{Standard Form}
\begin{enumerate}
\item
Using the affine transformation in
\eqref{eq:conic_affine},
	the conic in     \eqref{eq:conic_quad_form} can be expressed in standard form 
	%(centre/vertex at the origin, major axis - $x$ axis)
	as
  \begin{align}
    %\begin{aligned}
    \label{eq:conic_simp_temp_nonparab}
	    \vec{y}^{\top}\brak{\frac{\vec{D}}{f_0}}\vec{y} &= 1   &  \abs{\vec{V}} &\ne 0
    \\
	    \vec{y}^{\top}\vec{D}\vec{y} &=  -\eta\vec{e}_1^{\top}\vec{y}   & \abs{\vec{V}} &= 0
    \label{eq:conic_simp_temp_parab}
    %\end{aligned}
    \end{align}
    where
  \begin{align}
      %\begin{split}
      \label{eq:f0}
	  f_0 &=\vec{u}^{\top}\vec{V}^{-1}\vec{u} -f \ne 0
	  \\
      \label{eq:eta}
       \eta &=2\vec{u}^{\top}\vec{p}_1
       \\
       \vec{e}_1 &=\myvec{1 \\ 0}
      \end{align}
     
    
\begin{proof}
  \label{app:parab}
	Using 
\eqref{eq:conic_affine}
%such that 
\eqref{eq:conic_quad_form} can be expressed as

%\item  
%Substituting \eqref{eq:conic_affine} in \eqref{eq:conic_quad_form}
\begin{align}
\brak{\vec{P}\vec{y}+\vec{c}}^{\top}\vec{V}\brak{\vec{P}\vec{y}+\vec{c}}+2\vec{u}^{\top}\brak{\vec{P}\vec{y}+\vec{c}}+ f
	= 0, 
\end{align}
yielding 
\begin{align}
\vec{y}^{\top}\vec{P}^{\top}\vec{V}\vec{P}\vec{y}+2\brak{\vec{V}\vec{c}+\vec{u}}^{\top}\vec{P}\vec{y}
+  \vec{c}^{\top}\vec{V}\vec{c} + 2\vec{u}^{\top}\vec{c} + f= 0
\label{eq:conic_simp_one}
\end{align}
%
From \eqref{eq:conic_simp_one} and \eqref{eq:conic_parmas_eig_def},
\begin{align}
\vec{y}^{\top}\vec{D}\vec{y}+2\brak{\vec{V}\vec{c}+\vec{u}}^{\top}\vec{P}\vec{y}
+  \vec{c}^{\top}\brak{\vec{V}\vec{c} + \vec{u}}+ \vec{u}^{\top}\vec{c} + f= 0
\label{eq:conic_simp}
\end{align}
When $\vec{V}^{-1}$ exists, choosing
\begin{align}
%\begin{split}
\vec{V}\vec{c}+\vec{u} &= \vec{0}, \quad \text{or}, \vec{c} = -\vec{V}^{-1}\vec{u},
\label{eq:conic_parmas_c_def}
\end{align}
%
%%From \eqref{eq:conic_parmas_k_def} and 
%%
and substituting \eqref{eq:conic_parmas_c_def}
in \eqref{eq:conic_simp}
yields \eqref{eq:conic_simp_temp_nonparab}. 
  %See Appendix \ref{app:parab}.
When $\abs{\vec{V}} = 0, \lambda_1 = 0$ and 
\begin{align}
\vec{V}\vec{p}_1 = 0, 
\vec{V}\vec{p}_2 = \lambda_2\vec{p}_2.
\label{eq:conic_parab_eig_prop} 
\end{align}
where $\vec{p}_1,\vec{p}_2$ are the eigenvectors of $\vec{V}$ such that  \eqref{eq:conic_parmas_eig_def}
%
\begin{align}
\vec{P} = \myvec{\vec{p}_1 & \vec{p}_2},
\label{eq:eig_matrix}
\end{align}
Substituting \eqref{eq:eig_matrix}
in \eqref{eq:conic_simp},
\begin{align}
	\vec{y}^{\top}\vec{D}\vec{y}+2\brak{\vec{c}^{\top}\vec{V}+\vec{u}^{\top}}\myvec{\vec{p}_1 & \vec{p}_2}\vec{y}
	+  \vec{c}^{\top}\brak{\vec{V}\vec{c} + \vec{u}}+ \vec{u}^{\top}\vec{c} + f&= 0
\\
\implies \vec{y}^{\top}\vec{D}\vec{y}
+2\myvec{\brak{\vec{c}^{\top}\vec{V}+\vec{u}^{\top}}\vec{p}_1  \brak{\vec{c}^{\top}\vec{V}+\vec{u}^{\top}}\vec{p}_2}\vec{y}
	+  \vec{c}^{\top}\brak{\vec{V}\vec{c} + \vec{u}}+ \vec{u}^{\top}\vec{c} + f&= 0
\\
\implies \vec{y}^{\top}\vec{D}\vec{y}
+2\myvec{\vec{u}^{\top}\vec{p}_1 & \brak{\lambda_2\vec{c}^{\top}+\vec{u}^{\top}}\vec{p}_2}\vec{y}
	+  \vec{c}^{\top}\brak{\vec{V}\vec{c} + \vec{u}}+ \vec{u}^{\top}\vec{c} + f&= 0
\end{align}
upon substituting from 
 \eqref{eq:conic_parab_eig_prop} yielding
\begin{align}
\lambda_2y_2^2+2\brak{\vec{u}^{\top}\vec{p}_1}y_1+  2y_2\brak{\lambda_2\vec{c}+\vec{u}}^{\top}\vec{p}_2
	+  \vec{c}^{\top}\brak{\vec{V}\vec{c} + \vec{u}}+ \vec{u}^{\top}\vec{c} + f= 0
\label{eq:conic_parab_foc_len_temp} 
\end{align}
%which is the equation of a parabola. 
Thus, \eqref{eq:conic_parab_foc_len_temp} 
can be expressed as \eqref{eq:conic_simp_temp_parab} by choosing
\begin{align}
%\label{eq:eta}
\eta = 2\vec{u}^{\top}\vec{p}_1
\end{align}
%Choosing 
%\begin{align}
%\vec{u} + \lambda_2\vec{c} = 0,
%\vec{c}^{\top}\brak{\vec{V}\vec{c} + \vec{u}}+ \vec{u}^{\top}\vec{c} + f = 0,
%\end{align}
% the above equation becomes
%\begin{align}
%y_2^2= -\frac{2\vec{u}^{\top}\vec{p}_1}{ \lambda_2} \brak{y_1
%+  \frac{\vec{u}^{\top}\vec{V}\vec{u} - 2\lambda_2\vec{u}^{\top}\vec{u} + f\lambda_2^2}{2\vec{u}^{\top}\vec{p}_1\lambda_2^2}}
%\\
%or \eta = 2\vec{u}^{\top}\vec{p}_1
%%\label{eq:conic_simp_parab_new}
%\end{align}
and $\vec{c}$ in \eqref{eq:conic_simp} such that
\begin{align}
\label{eq:conic_parab_one}
2\vec{P}^{\top}\brak{\vec{V}\vec{c}+\vec{u}} &= \eta\myvec{1\\0}
\\
\vec{c}^{\top}\brak{\vec{V}\vec{c} + \vec{u}}+ \vec{u}^{\top}\vec{c} + f&= 0
\label{eq:conic_parab_two}
\end{align}
%we obtain  \eqref{eq:conic_simp_temp_parab}.
$\because
\vec{P}^{\top}\vec{P} = \vec{I}$,
multiplying \eqref{eq:conic_parab_one} by $\vec{P}$ yields
\begin{align}
\label{eq:conic_parab_one_eig}
	\brak{\vec{V}\vec{c}+\vec{u}} &= \frac{\eta}{2}\vec{p}_1,
\end{align}
which, upon substituting in \eqref{eq:conic_parab_two}
results in 
\begin{align}
\frac{\eta}{2}\vec{c}^{\top}\vec{p}_1 + \vec{u}^{\top}\vec{c} + f&= 0
\label{eq:conic_parab_two_eig}
\end{align}
\eqref{eq:conic_parab_one_eig} and \eqref{eq:conic_parab_two_eig} can be clubbed together to obtain \eqref{eq:conic_parab_c}.
  \end{proof}
	  \item
		For the standard conic, 
				\begin{align}
					\label{eq:std-prm-P}
					\vec{P} &= \vec{I}
					\\
					\vec{u} &= 
				\begin{cases}
				0 & e \ne 1
       \\
				\frac{\eta}{2} \vec{e}_1 & e = 1
				\end{cases}
				\label{eq:std-prm-u}
				\\
				\lambda_1 &  
					\begin{cases}
						=0 & e = 1
						\\
						\ne 0 & e \ne 1
					\end{cases}
				\label{eq:std-prm-lam1}
				\end{align}
				where 
				\begin{align}
					\vec{I} = \myvec{\vec{e}_1 & \vec{e}_2}
				\end{align}
				is the identity matrix.
	  
    \item\leavevmode
		\begin{enumerate}
			\item The directrices for the  standard conic are given by 
				\begin{align}
					\label{eq:dx-ell-hyp}
					\vec{e}_1^{\top}\vec{y} &=  
					%\pm\sqrt{\abs{\frac{f_0\lambda_2}{\lambda_1\brak{\lambda_2-\lambda_1}}}} & e \ne 1
					\pm \frac{1}{e}\sqrt{\frac{\abs{f_0}}{\lambda_2\brak{1-e^2}}} & e \ne 1
					\\
					\vec{e}_1^{\top}\vec{y} &= \frac{\eta}{2\lambda_2} & e = 1
					\label{eq:dx-parab}
				\end{align}
    \item The foci of the standard ellipse and hyperbola are given by 
				\begin{align}
					\label{eq:F-ell-hyp-parab}
					\vec{F} 
=
					\begin{cases}
						\pm e\sqrt{\frac{\abs{f_0}}{\lambda_2\brak{1-e^2}}}\vec{e}_1 & e \ne 1
					%	\pm \sqrt{\abs{\frac{f_0}{\lambda_1}\brak{1 - \frac{\lambda_1}{\lambda_2}}}}\vec{e}_1 & e \ne 1
						\\
						 -\frac{\eta}{4\lambda_2}\vec{e}_1 & e = 1
					\end{cases}
				\end{align}
	
		\end{enumerate}
	%	where, without loss of generality, $f_0 < 0$ for the hyperbola.
    
	\begin{proof}%\leavevmode
  \label{app:foc-dir}
%  \renewcommand{\theequation}{\theenumi}
\begin{enumerate}[label=\thesection.\arabic*.,ref=\thesection.\theenumi]
\numberwithin{equation}{enumi}

\item Substituting \eqref{eq:conic_affine} in \eqref{eq:conic_quad_form}
\begin{align}
\brak{\vec{P}\vec{y}+\vec{c}}^T\vec{V}\brak{\vec{P}\vec{y}+\vec{c}}+2\vec{u}^T\brak{\vec{P}\vec{y}+\vec{c}}+ f = 0, 
\end{align}
which can be expressed as
\begin{multline}
\vec{y}^T\vec{P}^T\vec{V}\vec{P}\vec{y}+2\brak{\vec{V}\vec{c}+\vec{u}}^T\vec{P}\vec{y}
\\
+  \vec{c}^T\vec{V}\vec{c} + 2\vec{u}^T\vec{c} + f= 0
\label{eq:conic_simp_one}
\end{multline}
%
From \eqref{eq:conic_simp_one} and \eqref{eq:conic_parmas_eig_def},
\begin{multline}
\vec{y}^T\vec{D}\vec{y}+2\brak{\vec{V}\vec{c}+\vec{u}}^T\vec{P}\vec{y}
\\
+  \vec{c}^T\brak{\vec{V}\vec{c} + \vec{u}}+ \vec{u}^T\vec{c} + f= 0
\label{eq:conic_simp}
\end{multline}
When $\vec{V}^{-1}$ exists,
\begin{align}
%\begin{split}
\vec{V}\vec{c}+\vec{u} &= \vec{0}, \quad \text{or}, \vec{c} = -\vec{V}^{-1}\vec{u},
\label{eq:conic_parmas_c_def}
\end{align}
%
%%From \eqref{eq:conic_parmas_k_def} and 
%%
and substituting \eqref{eq:conic_parmas_c_def}
in \eqref{eq:conic_simp}
yields \eqref{eq:conic_simp_temp_nonparab}. 
\item 
When $\abs{\vec{V}} = 0, \lambda_1 = 0$ and 
\begin{align}
\vec{V}\vec{p}_1 = 0, 
\vec{V}\vec{p}_2 = \lambda_2\vec{p}_2.
\label{eq:conic_parab_eig_prop} 
\end{align}
where $\vec{p}_1,\vec{p}_2$ are the eigenvectors of $\vec{V}$ such that  \eqref{eq:conic_parmas_eig_def}
%
\begin{align}
\vec{P} = \myvec{\vec{p}_1 & \vec{p}_2},
\label{eq:eig_matrix}
\end{align}
Substituting \eqref{eq:eig_matrix}
in \eqref{eq:conic_simp},
\begin{multline}
\vec{y}^T\vec{D}\vec{y}+2\brak{\vec{c}^T\vec{V}+\vec{u}^T}\myvec{\vec{p}_1 & \vec{p}_2}\vec{y}
\\
+  \vec{c}^T\brak{\vec{V}\vec{c} + \vec{u}}+ \vec{u}^T\vec{c} + f= 0
\\
\implies \vec{y}^T\vec{D}\vec{y}
\\
+2\myvec{\brak{\vec{c}^T\vec{V}+\vec{u}^T}\vec{p}_1 & \brak{\vec{c}^T\vec{V}+\vec{u}^T}\vec{p}_2}\vec{y}
\\
+  \vec{c}^T\brak{\vec{V}\vec{c} + \vec{u}}+ \vec{u}^T\vec{c} + f= 0
\\
\implies \vec{y}^T\vec{D}\vec{y}
\\
+2\myvec{\vec{u}^T\vec{p}_1 & \brak{\lambda_2\vec{c}^T+\vec{u}^T}\vec{p}_2}\vec{y}
\\
+  \vec{c}^T\brak{\vec{V}\vec{c} + \vec{u}}+ \vec{u}^T\vec{c} + f= 0
\\
\text{ from } \eqref{eq:conic_parab_eig_prop} 
\\
\implies \lambda_2y_2^2+2\brak{\vec{u}^T\vec{p}_1}y_1+  2y_2\brak{\lambda_2\vec{c}+\vec{u}}^T\vec{p}_2
\\
+  \vec{c}^T\brak{\vec{V}\vec{c} + \vec{u}}+ \vec{u}^T\vec{c} + f= 0
\label{eq:conic_parab_foc_len_temp} 
\end{multline}
which is the equation of a parabola. From \eqref{eq:conic_parab_foc_len_temp}, by comparing the coefficients of $y_2^2$ and $y_1$, the focal length of the parabola is obtained as     \ref{eq:conic_parab_foc_len} 
\begin{align}
    \mydet{\frac{2\eta}{\lambda_2}} = \mydet{\frac{2\vec{u}^T\vec{p}_1}{\lambda_2}}.
    \label{eq:conic_parab_foc_len} 
    \end{align}    
  %
Thus, \eqref{eq:conic_parab_foc_len_temp} 
can be expressed as \eqref{eq:conic_simp_temp_parab} by choosing
\begin{align}
\label{eq:eta}
\eta = \vec{u}^T\vec{p}_1
\end{align}
%Choosing 
%\begin{align}
%\vec{u} + \lambda_2\vec{c} = 0,
%\vec{c}^T\brak{\vec{V}\vec{c} + \vec{u}}+ \vec{u}^T\vec{c} + f = 0,
%\end{align}
% the above equation becomes
%\begin{align}
%y_2^2= -\frac{2\vec{u}^T\vec{p}_1}{ \lambda_2} \brak{y_1
%+  \frac{\vec{u}^T\vec{V}\vec{u} - 2\lambda_2\vec{u}^T\vec{u} + f\lambda_2^2}{2\vec{u}^T\vec{p}_1\lambda_2^2}}
%\\
%or \eta = 2\vec{u}^T\vec{p}_1
%%\label{eq:conic_simp_parab_new}
%\end{align}
and $\vec{c}$ in \eqref{eq:conic_simp} such that
\begin{align}
\label{eq:conic_parab_one}
\vec{P}^{T}\brak{\vec{V}\vec{c}+\vec{u}} &= \eta\myvec{1\\0}
\\
\vec{c}^T\brak{\vec{V}\vec{c} + \vec{u}}+ \vec{u}^T\vec{c} + f&= 0
\label{eq:conic_parab_two}
\end{align}
%we obtain  \eqref{eq:conic_simp_temp_parab}.
are satisfied.  Multiplying \eqref{eq:conic_parab_one} by $\vec{P}$ yields
\begin{align}
\label{eq:conic_parab_one_eig}
\brak{\vec{V}\vec{c}+\vec{u}} &= \eta\vec{p}_1,
\end{align}
which, upon substituting in \eqref{eq:conic_parab_two}
results in 
\begin{align}
\eta\vec{c}^T\vec{p}_1 + \vec{u}^T\vec{c} + f&= 0
\label{eq:conic_parab_two_eig}
\end{align}
\eqref{eq:conic_parab_one_eig} and \eqref{eq:conic_parab_two_eig} can be clubbed together to obtain \eqref{eq:conic_parab_c}.

\end{enumerate}

		\begin{enumerate}
			\item For the standard hyperbola/ellipse in \eqref{eq:conic_simp_temp_nonparab}, from 
					\eqref{eq:std-prm-P},
\eqref{eq:conic_quad_form_nc}
and 
					\eqref{eq:std-prm-u},
				\begin{align}
\label{eq:n-ell-hyp}
					\vec{n} &= \sqrt{\frac{\lambda_2}{f_0}} \vec{e}_1 
					\\
					c &= 
					%\pm \frac{\sqrt{-\lambda_2\brak{e^2-1}\brak{\lambda_2 f_0}}}{\lambda_2e\brak{e^2-1}}
					\pm \frac{\sqrt{-\frac{\lambda_2}{f_0}\brak{e^2-1}\brak{\frac{\lambda_2}{ f_0}}}}{\frac{\lambda_2}{f_0}e\brak{e^2-1}}
					\\
					&=\pm \frac{1}{e\sqrt{1-e^2}}
%					\\
%					&=\pm\sqrt{\abs{\frac{f_0}{\brak{1 - \frac{\lambda_1}{\lambda_2}}\frac{\lambda_1}{\lambda_2}}}}
\label{eq:c-ell-hyp}
				\end{align}
				yielding 
					\eqref{eq:dx-ell-hyp} upon substituting from 
\eqref{eq:conic_quad_form_e} and simplifying.
For the standard parabola in \eqref{eq:conic_simp_temp_parab},  from 
					\eqref{eq:std-prm-P},
\eqref{eq:conic_quad_form_nc}
and 
					\eqref{eq:std-prm-u}, noting that $f = 0$,

				\begin{align}
\label{eq:n-parab}
					\vec{n} &= \sqrt{\lambda_2} \vec{e}_1 
					\\
					c &=
	\frac{\norm{\frac{\eta}{2} \vec{e}_1}^2   }{2\vec{\brak{\frac{\eta}{2}} \brak{\vec{e}_1}^{\top}\vec{n}}} 
\\
					\\
					&= \frac{\eta}{4\sqrt{\lambda_2}}
\label{eq:c-parab}
				\end{align}
				yielding 
					\eqref{eq:dx-parab}.

				\item 	For the standard ellipse/hyperbola, substituting from
\eqref{eq:c-ell-hyp},
\eqref{eq:n-ell-hyp},
\eqref{eq:std-prm-u}
and \eqref{eq:conic_quad_form_e}
in \eqref{eq:conic_quad_form_F},
				\begin{align}
					\vec{F} &= \pm \frac{\brak{\frac{1}{e\sqrt{1-e^2}}}\brak{e^2}\sqrt{\frac{\lambda_2}{f_0}}\vec{e}_1}{\frac{\lambda_2}{f_0}}
					%\pm\sqrt{\abs{\frac{f_0}{\brak{1 - \frac{\lambda_1}{\lambda_2}}\frac{\lambda_1}{\lambda_2}}}}
					%\brak{1 - \frac{\lambda_1}{\lambda_2}}\frac{\sqrt{\lambda_2}}{\lambda_2}\vec{e}_1
 			\end{align}
			yielding
					\eqref{eq:F-ell-hyp-parab}
					after simplification.
					For the standard parabola, substituting from 
\eqref{eq:c-parab},
\eqref{eq:n-parab},
\eqref{eq:std-prm-u}
and \eqref{eq:conic_quad_form_e}
in \eqref{eq:conic_quad_form_F},			
				\begin{align}
	\vec{F}  &= \frac{\brak{\frac{\eta}{4\sqrt{\lambda_2}}}\sqrt{\lambda_2}\vec{e}_1-\vec{\frac{\eta}{2} \vec{e}_1}}{\lambda_2}
\\
				\end{align}
				yielding 
					\eqref{eq:F-ell-hyp-parab} after simplification.

		\end{enumerate}
%		See Appendix \ref{app:foc-dir}.
	\end{proof}
	\end{enumerate}

\section{Conic Parameters}
\subsection{Standard Form}

	\begin{corollary}
			\label{corr:center}
			The center of the standard ellipse/hyperbola, defined to be the mid point of the line joining the foci, is the origin.
	\end{corollary}
	\begin{corollary}
		\label{corr:axis}
			The principal (major) axis of the standard ellipse/hyperbola, defined to be the line joining the two foci   is the $x$-axis.  
	\end{corollary}
	\begin{proof}
		From 	\eqref{eq:F-ell-hyp-parab}, it is obvious that the line joining the foci passes through the origin.  Also, the direction vector of this line is $\vec{e}_1$.  Thus, the principal axis is the $x$-axis. 
	\end{proof}
	\begin{corollary}
		\label{corr:minor-axis}
			The minor axis of the standard ellipse/hyperbola, defined to be the line orthogonal to the $x$-axis is the $y$-axis. 
	\end{corollary}


	\begin{corollary}
			The axis of symmetry of the standard parabola, defined to be the line perpendicular to the directrix and passing through the focus,  is the $x$- axis.
	\end{corollary}
	\begin{proof}
	From \eqref{eq:n-parab} and 	
					\eqref{eq:F-ell-hyp-parab}, 
					the axis of the parabola  can be expressed using 
    \eqref{eq:line_norm_eq} as 
		\begin{align}
			\vec{e}_2^{\top}\brak{\vec{y}  
			+\frac{\eta}{4\lambda_2}\vec{e}_1} &= 0
			\\
			\implies \vec{e}_2^{\top}\vec{y} &= 0
					\label{eq:axis-std-parab}, 
		\end{align}
		which is the equation of the $x$-axis.
	\end{proof}


	\begin{corollary}
			\label{corr:center-parab}
 The point where the parabola intersects its axis of symmetry is called the vertex. For the standard parabola, the vertex is the origin.
	\end{corollary}
	\begin{proof}
					\eqref{eq:axis-std-parab} can be expressed as 
    \begin{align}
			\vec{y}= \alpha \vec{e}_1 
					\label{eq:axis-std-parab-dir}, 
    \end{align}
    using 
    \eqref{eq:line_norm_eq}.
					Substituting \eqref{eq:axis-std-parab-dir} in 
    \eqref{eq:conic_simp_temp_parab}, 
    \begin{align}
	     \alpha^2 \vec{e}_1^{\top}\vec{D} \vec{e}_1 &=  -\eta\alpha \vec{e}_1^{\top} \vec{e}_1   
	     \\
	     \implies \alpha &=0, \text{ or, } \vec{y} = \vec{0}.
    %\end{aligned}
    \end{align}
	\end{proof}
	\begin{corollary}
			\label{corr:foclen}
	 The {\em focal length} of the standard parabola, , defined to be the distance between the vertex and the focus, measured along the axis of symmetry, is $\abs{\frac{\eta}{4 \lambda_2}}$
	\end{corollary}

\subsection{Quadratic Form }

	\begin{corollary}
		The center/vertex of a conic section are given by
  \begin{align}
    \label{eq:conic_nonparab_c}
	    \vec{c} &= - \vec{V}^{-1}\vec{u}  & \mydet{\vec{V}} \ne 0
    \\
	    \myvec{ \vec{u}^{\top}+\frac{\eta}{2}\vec{p}_1^{\top} \\ \vec{v}}\vec{c} &= \myvec{-f \\ \frac{\eta}{2}\vec{p}_1-\vec{u}}  
& \mydet{\vec{V}} = 0
    \label{eq:conic_parab_c}
    \end{align}	
	\end{corollary}
		\begin{proof}
			In 
			\eqref{eq:conic_affine}, substituting $\vec{y} = \vec{0}$, the center/vertex for the quadratic form is obtained as
    \begin{align}
	    \vec{x} = \vec{c}, 
    \end{align}
			where $\vec{c}$ is derived as 
    \eqref{eq:conic_nonparab_c}
    and 
    \eqref{eq:conic_parab_c}
in Appendix  \ref{app:parab}.
		\end{proof}

%
    \begin{corollary} The equation of the minor and major  axes for the ellipse/hyperbola are respectively given by 
  \begin{align}
\vec{p}_i^{\top}\brak{\vec{x}-\vec{c}} = 0, i = 1,2
	  \label{eq:major-minor-axis-quad}
  \end{align}
  The axis of symmetry for the parabola is also given by 
	  \eqref{eq:major-minor-axis-quad}.
\end{corollary}
		\begin{proof}
From		\eqref{corr:axis}, the major/symmetry axis for the hyperbola/ellipse/parabola can be expressed using 
\eqref{eq:conic_affine} as
  \begin{align}
	  \vec{e}_2^{\top}
		  \vec{P}^{\top}\brak{\vec{x}-\vec{c}} &= 0
		  \\
	  \implies 		  \brak{\vec{P}\vec{e}_2}^{\top}\brak{\vec{x}-\vec{c}} &= 0
  \end{align}
yielding	  \eqref{eq:major-minor-axis-quad}, and the proof for the minor axis is similar.
		\end{proof}


\section{Conic Lines}
\subsection{Pair of Straight Lines}
%

\begin{lemma}[Asymptotes]
	The asymptotes of the hyperbola in 
    \eqref{eq:conic_simp_temp_nonparab}, defined to be the lines that do not intersect the hyperbola, are given by 
    \begin{align} 
    \label{eq:pair-std}
    \myvec{\sqrt{\abs{\lambda_1}} & \pm \sqrt{\abs{\lambda_2}}}\vec{y} = 0
    \end{align} 
%  \begin{align}
%	  \myvec{\lambda_1 & \pm \lambda_2}\vec{y} = 0   
%  \end{align}
  \end{lemma}
  \begin{proof}
	  From 
\eqref{eq:conic_simp_temp_nonparab},
it is obvious that 
the pair of lines represented by 
  \begin{align}
	    \vec{y}^{\top}\vec{D}\vec{y} = 0   
      \label{eq:pair-conic}
  \end{align}
  do not intersect the conic 
  \begin{align}
	    \vec{y}^{\top}\vec{D}\vec{y} =  f_0  
  \end{align}
  Thus, 
      \eqref{eq:pair-conic}
      represents the asysmptotes of the hyperbola in 
\eqref{eq:conic_simp_temp_nonparab} and can be expressed as 
  \begin{align} 
    \lambda_1y_1^2 +\lambda_2y_1^2 = 0, 
    \label{eq:quad_form_hyper}
    \end{align}
%    \eqref{eq:quad_form_hyper}
which can then be simplified to obtain 
    \eqref{eq:pair-std}.

  \end{proof}
  \begin{corollary}
\eqref{eq:conic_quad_form} represents a pair of straight lines if 
  \begin{align} 
	  \label{eq:pair-cond}
%	  \lambda_1y_1^2 +\lambda_2y_2^2 = 
  \vec{u}^{\top}\vec{V}^{-1}\vec{u} -f  = 0
  \end{align} 
  \end{corollary}
  \begin{theorem}
	  \label{them:pair-mat-sing}
\eqref{eq:conic_quad_form} represents a pair of straight lines if 
the matrix 
  \begin{align} 
	  \myvec{\vec{V} & \vec{u}\\ \vec{u} & f}  
	  \label{eq:pair-mat-sing}
  \end{align} 
  is singular.
  \end{theorem}
  \begin{proof}
Let 
  \begin{align} 
	  \myvec{\vec{V} & \vec{u}\\ \vec{u} & f}  \vec{x} =\vec{0}
  \end{align} 
  Expressing 
  \begin{align} 
	  \vec{x} =\myvec{\vec{y} \\ y_3}, 
  \end{align} 
  \begin{align} 
	  \myvec{\vec{V} & \vec{u}\\ \vec{u}^{\top} & f}   
	  \myvec{\vec{y} \\ y_3} &= \vec{0}
	  \\
	  \implies
	  \label{eq:pair-mat-sing-1}
	  \vec{V} \vec{y} + y_3\vec{u} &= \vec{0} \quad \text{and}
	  \\
	  \vec{u}^{\top}\vec{y} + fy_3 &=0
	  \label{eq:pair-mat-sing-2}
  \end{align} 
  From 
	  \eqref{eq:pair-mat-sing-1} we obtain,
  \begin{align} 
	  \vec{y}^{\top}  \vec{V} \vec{y} + y_3\vec{y}^{\top}\vec{u} &= \vec{0} 
	  \\
	  \implies 
	  \vec{y}^{\top}  \vec{V} \vec{y} + y_3\vec{u}^{\top}\vec{y} &= \vec{0} 
  \end{align} 
  yielding 
	  \eqref{eq:pair-cond} upon substituting from 
	  \eqref{eq:pair-mat-sing-2}.
  \end{proof}
  \begin{corollary}
	  Using the affine transformation, 
    \eqref{eq:pair-std}
 can be expressed as the lines 
%
\begin{align} 
\label{eq:quad_form_pair}
\myvec{\sqrt{\abs{\lambda_1}} & \pm \sqrt{\abs{\lambda_2}}}\vec{P}^{\top}\brak{\vec{x}-\vec{c}} = 0
\end{align} 
  \end{corollary}
   \begin{corollary}
	   The angle between the asymptotes can be expressed as
\begin{align} 
\label{eq:quad_form_pair_ang}
\cos\theta=\frac{\abs{\lambda_1}-\abs{\lambda_2}}
{\abs{\lambda_1}+\abs{\lambda_2}}
\end{align} 
  \end{corollary}
  \begin{proof}
The normal vectors of the lines in \eqref{eq:quad_form_pair} are 
  \begin{align} 
  \label{eq:quad_form_pair_normvecs}
  \begin{split}
  \vec{n}_1 &= \vec{P}\myvec{\sqrt{\abs{\lambda_1}} \\[2mm]  \sqrt{\abs{\lambda_2}}}
  \\
  \vec{n}_2 &= \vec{P}\myvec{\sqrt{\abs{\lambda_1}} \\[2mm] - \sqrt{\abs{\lambda_2}}}
  \end{split}
  \end{align} 
  The angle between the asymptotes is given by 
\begin{align} 
\label{eq:quad_form_pair_ang_exp}
\cos\theta=\frac{\vec{n_1}^{\top}\vec{n_2}}{\norm{\vec{n_1}}\norm{\vec{n_2}}}
\end{align} 
The orthogonal matrix $\vec{P}$ preserves the norm, i.e.
\begin{align} 
	\norm{\vec{n_1}} &= \norm{\vec{P}\myvec{\sqrt{\abs{\lambda_1}} \\[2mm]  \sqrt{\abs{\lambda_2}}}}
	=\norm{\myvec{\sqrt{\abs{\lambda_1}} \\[2mm]  \sqrt{\abs{\lambda_2}}}}
	\\
	&=\sqrt{\abs{\lambda_1}+\abs{\lambda_2}} = \norm{\vec{n_2}}
\end{align} 
It is easy to verify that 
\begin{align} 
\vec{n_1}^{\top}\vec{n_2} = \abs{\lambda_1}-\abs{\lambda_2}
\end{align} 
%
Thus, the angle between the asymptotes is obtained from \eqref{eq:quad_form_pair_ang_exp} as \eqref{eq:quad_form_pair_ang}.
  \end{proof}

\subsection{Intersubsection of Conics}
\begin{enumerate}[label=\thesection.\arabic*,ref=\thesection.\theenumi]
\numberwithin{equation}{enumi}
\numberwithin{figure}{enumi}
\numberwithin{table}{enumi}

\item  Point $\vec{P}(0,2)$ is the point of intersection of $y$-axis and perpendicular bisector of line segment joining the points $\vec{A}(-1,1) \text{ and } \vec{B}(3,3)$

\item Prove that the line through A$(0,-1,-1)$ and B$(4,5,1)$ intersects the line through C$(3,9,4)$ and D$(-4,4,4)$.
\item Show the lines
$$\frac{x-1}{2}=\frac{y-2}{3}=\frac{z-3}{4}$$
$$\text{ and } \frac{x-4}{5}=\frac{y-1}{2}=z  \text{ intersect }.$$
 Also, find their point of intersection.
\item The area of the region bounded by the curve $y = x + 1$ and the lines $x = 2\text{ and }x = 3$ is
\begin{enumerate}
\item $\frac{7}{2}$ sq units
\item $\frac{9}{2}$ sq units
\item $\frac{11}{2}$ sq units
\item $\frac{13}{2}$ sq units
\end{enumerate}   
\item The area of the region bounded by the curve $x = 2 + 3$ and the $y$ lines $y = 1\text{ and }y = - 1$ is
\begin{enumerate}
\item 4 sq units 
\item $\frac{3}{2}$ sq units
\item 6 sq units
\item 8 sq units
\end{enumerate}
\item Compute the area bounded by the line $x + 2y = 2$, $y - x = 1\text{ and }2x + y = 7$.
\item Find the area bounded by the lines $y = 4x + 5$, $y = 5 - x\text{ and }4y = x + 5$.
\end{enumerate}

\subsection{ Chords of a Conic}
%\begin{enumerate}
\begin{enumerate}[label=\thesection.\arabic*.,ref=\thesection.\theenumi]
		\item
  The points of intersection of the line 
\begin{align}
L: \quad \vec{x} = \vec{h} + \mu \vec{m} \quad \mu \in \mathbb{R}
\label{eq:conic_tangent}
\end{align}
with the conic section in \eqref{eq:conic_quad_form} are given by
\begin{align}
\vec{x}_i = \vec{h} + \mu_i \vec{m}
	\label{eq:chord-pts}
\end{align}
%
where
\begin{multline}
\mu_i = \frac{1}
{
\vec{m}^{\top}\vec{V}\vec{m}
}
\lbrak{-\vec{m}^{\top}\brak{\vec{V}\vec{h}+\vec{u}}}
\\
\pm
{\small
\rbrak{\sqrt{
\sbrak{
\vec{m}^{\top}\brak{\vec{V}\vec{h}+\vec{u}}
}^2
	-\text{g}
\brak
{\vec{h}
%\vec{h}^{\top}\vec{V}\vec{h} + 2\vec{u}^{\top}\vec{h} +f
}
\brak{\vec{m}^{\top}\vec{V}\vec{m}}
}
}
}
\label{eq:tangent_roots}
\end{multline}



\begin{proof}
  Substituting \eqref{eq:conic_tangent}
in \eqref{eq:conic_quad_form}, 
\begin{align}
\brak{\vec{h} + \mu \vec{m}}^{\top}\vec{V}\brak{\vec{h} + \mu \vec{m}}  + 2 \vec{u}^{\top}\brak{\vec{h} + \mu \vec{m}}+f &= 0
\\
\implies \mu^2\vec{m}^{\top}\vec{V}\vec{m} + 2 \mu\vec{m}^{\top}\brak{\vec{V}\vec{h}+\vec{u}} 
+ \vec{h}^{\top}\vec{V}\vec{h} + 2\vec{u}^{\top}\vec{h} +f &= 0
	\\
	\text{or, }
\mu^2\vec{m}^{\top}\vec{V}\vec{m} + 2 \mu\vec{m}^{\top}\brak{\vec{V}\vec{h}+\vec{u}} 
	+ \text{g}\brak{\vec{h}} &=0
	%^{\top}\vec{V}\vec{h} + 2\vec{u}^{\top}\vec{h} +f &= 0
\label{eq:conic_intercept}
\end{align}
for g defined in \eqref{eq:conic_quad_form}.
Solving the above quadratic in \eqref{eq:conic_intercept}
yields \eqref{eq:tangent_roots}.
\end{proof}
\item
  If $L$ in \eqref{eq:conic_tangent} touches \eqref{eq:conic_quad_form} at exactly one point $\vec{q}$, 
  \begin{align}
  \vec{m}^{\top}\brak{\vec{V}\vec{q}+\vec{u}} = 0
  \label{eq:conic_tangent_mq}
  \end{align}

\begin{proof}
  In this case, \eqref{eq:conic_intercept} has exactly one root.  Hence, 
  in \eqref{eq:tangent_roots}
  \begin{align}
  \sbrak{
  \vec{m}^{\top}\brak{\vec{V}\vec{q}+\vec{u}}
  }^2 -\brak{\vec{m}^{\top}\vec{V}\vec{m}}
	  \text{g}\brak
  {
  \vec{q}
%  \vec{q}^{\top}\vec{V}\vec{q} + 2\vec{u}^{\top}\vec{q} +f
  } = 0                                                                                             
  \label{eq:conic_tangent_disc}
  \end{align}                    
  $\because \vec{q}$ is the point of contact,
	%$\vec{q}$ satisfies \eqref{eq:conic_quad_form}
%  and 
  \begin{align}
	  \text{g}\brak{  \vec{q}} = 0
%  \vec{q}^{\top}\vec{V}\vec{q} + 2\vec{u}^{\top}\vec{q} +f = 0
  \label{eq:conic_tangent_qquad}
  \end{align}
  Substituting \eqref{eq:conic_tangent_qquad} in \eqref{eq:conic_tangent_disc} and simplifying, we obtain \eqref{eq:conic_tangent_mq}.
\end{proof}
	\item
		The length of the chord in 
\eqref{eq:conic_tangent}
is given by 
\begin{align}
 \frac{2\sqrt{
\sbrak{
\vec{m}^{\top}\brak{\vec{V}\vec{h}+\vec{u}}
}^2
-
\brak
{
\vec{h}^{\top}\vec{V}\vec{h} + 2\vec{u}^{\top}\vec{h} +f
}
\brak{\vec{m}^{\top}\vec{V}\vec{m}}
}
}
{
\vec{m}^{\top}\vec{V}\vec{m}
}\norm{\vec{m}}
\label{eq:chord-len}
  \end{align}
	
\begin{proof}
The distance between the points in 
	\eqref{eq:chord-pts}
is given by 
\begin{align}
	\norm{\vec{x}_1-\vec{x}_2} =  \abs{\mu_1-\mu_2} \norm{\vec{m}}
\label{eq:conic_tangent_pts_dist}
\end{align}
Substituing $\mu_i$ from 
\eqref{eq:tangent_roots} in
\eqref{eq:conic_tangent_pts_dist}
yields
	\eqref{eq:chord-len}.
\end{proof}
	\item
 The affine transform for the conic section, preserves the norm.  This implies that the length of any chord of a conic
	is invariant to translation and/or rotation.
	
	\begin{proof}
	Let 
%From \eqref{eq:conic_affine}, 
\begin{align}
\vec{x}_i = \vec{P}\vec{y}_i+\vec{c} 
\label{eq:conic_affine_pts}
\end{align}
be any two points on the conic.  Then the distance between the points is given by 
\begin{align}
	\norm{\vec{x}_1-\vec{x}_2 } &= \norm{\vec{P}\brak{	\vec{y}_1 -\vec{y}_2 }}
\end{align}
which can be expressed as 
\begin{align}
	\norm{\vec{x}_1-\vec{x}_2 }^2 &= 		\brak{\vec{y}_1 -\vec{y}_2 }^{\top}\vec{P}^{\top}\vec{P}\brak{\vec{y}_1 -\vec{y}_2 }
	\\
	&= 		\norm{\vec{y}_1 -\vec{y}_2 }^2
\label{eq:conic_affine_norm_preserve}
\end{align}
since 
\begin{align}
	\vec{P}^{\top}\vec{P} = \vec{I}
\end{align}
	\end{proof}
    \item For the standard hyperbola/ellipse, the length of the major axis is 
  \begin{align}
\label{eq:chord-len-major}
 2\sqrt{\abs{\frac{
f_0}
{\lambda_1}
	  }}
  \end{align}
  and the minor axis is 
  \begin{align}
\label{eq:chord-len-minor}
 2\sqrt{\abs{\frac{
f_0}
{\lambda_2}
	  }}
  \end{align}
%	    $\mydet{\vec{V}} \ne 0$, the lengths of the semi-major and semi-minor axes of the conic in \eqref{eq:conic_quad_form} are given by 
%  \begin{align} 
%    \label{eq:ellipse_axes}
%  %  \begin{aligned}[t]
%    \sqrt{\frac{\vec{u}^{\top}\vec{V}^{-1}\vec{u} -f}{\lambda_1}}, 
%    \sqrt{\frac{\vec{u}^{\top}\vec{V}^{-1}\vec{u} -f}{\lambda_2}}. \quad \brak{\text{ellipse}}
%    \\
%%
%       \sqrt{\frac{\vec{u}^{\top}\vec{V}^{-1}\vec{u} -f}{\lambda_1}}, 
%       \sqrt{\frac{f-\vec{u}^{\top}\vec{V}^{-1}\vec{u}}{\lambda_2}}, \quad \brak{\text{hyperbola }}
%%
%  %\end{aligned}
%  \label{eq:hyper_axes}
%\end{align} 
%\solution For \begin{align} \abs{\vec{V}} > 0, \quad \text{or, } \lambda_1 > 0, \lambda_2 > 0 
%  \end{align} and \eqref{eq:conic_simp_temp_nonparab} becomes 
%  \begin{align} 
%	  \lambda_1y_1^2 +\lambda_2y_2^2 = 
%  \vec{u}^{\top}\vec{V}^{-1}\vec{u} -f 
%	  \label{eq:hyper-pair-cond}
%  \end{align} 
%  yielding        \eqref{eq:ellipse_axes}.  Similarly, \eqref{eq:hyper_axes} can be obtained for 
%  \begin{align} 
%    \label{eq:conic_hyper_cond}
%    \abs{\vec{V}} 
%    < 0, \quad \text{or, } \lambda_1 > 0, \lambda_2 < 0 \end{align}

\begin{proof}
%	See Appendix \ref{app:major}
		\label{app:major}
		Since the major axis passes through the origin, 
  \begin{align}
	  \vec{q} =			\vec{0} 
\end{align}  
Further, from Corollary  
		\eqref{corr:axis},
  \begin{align}
  \vec{m}&= \vec{e}_2,  
\end{align} and
from 
    \eqref{eq:conic_simp_temp_nonparab},
  \begin{align}
	  \vec{V} =     \frac{\vec{D} }{f_0}, 
	   \vec{u} = 0, 
	   f = -1
	    \label{eq:latus_rectum_ellipse_param}
\end{align}  
Substituting the above in
\eqref{eq:chord-len}, 
\begin{align}
 \frac{2\sqrt{
\vec{e}_1^{\top}\frac{\vec{D}}{f_0}\vec{e}_1
}
}
{
\vec{e}_1^{\top}\frac{\vec{D}}{f_0}\vec{e}_1
}\norm{\vec{e}_1}
  \end{align}
  yielding 
\eqref{eq:chord-len-major}.
Similarly, for the minor axis, the only different parameter is 
  \begin{align}
  \vec{m}&= \vec{e}_2,  
\end{align} 
Substituting the above in
\eqref{eq:chord-len}, 
\begin{align}
 \frac{2\sqrt{
\vec{e}_2^{\top}\frac{\vec{D}}{f_0}\vec{e}_2
}
}
{
\vec{e}_2^{\top}\frac{\vec{D}}{f_0}\vec{e}_2
}\norm{\vec{e}_2}
  \end{align}
  yielding 
\eqref{eq:chord-len-minor}.

\end{proof}
\item
    The latus rectum of a conic section is the chord that passes through the focus and is perpendicular to the major axis.
	The length of the latus rectum for a conic is given by
		\begin{align}
			l =
			\begin{cases}
				2\frac{\sqrt{\abs{f_0\lambda_1}}}{\lambda_2} & e \ne 1
			\\
			\frac{\eta}{\lambda_2} & e = 1
			\end{cases}
			\label{eq:latus-ellipse}
		\end{align}

		\begin{proof}
%			See Appendix \ref{app:latus}.
		%\section{}
		\label{app:latus}
			The latus rectum is perpendicular to the major axis for the standard conic.  Hence, from Corollary  
		\eqref{corr:axis},
  \begin{align}
  \vec{m}&= \vec{e}_2,  
\end{align}  
Since it passes through the focus, from 
					\eqref{eq:F-ell-hyp-parab}
  \begin{align}
	  \vec{q} =			\vec{F} 
=
					 \pm e\sqrt{\frac{f_0}{\lambda_2\brak{1-e^2}}} \vec{e }_1
%					 \frac{e}{\sqrt{f_0\lambda_2\brak{1-e^2}}}\vec{e }_1
\end{align}  
for the standard hyperbola/ellipse.  Also, 
from 
    \eqref{eq:conic_simp_temp_nonparab},
  \begin{align}
	  \vec{V} =     \frac{\vec{D} }{f_0}, 
	   \vec{u} = 0, 
	   f = -1
	    \label{eq:latus_rectum_ellipse_param-new}
\end{align}  
Substituting the above in
\eqref{eq:chord-len}, 
\begin{align}
 \frac{2\sqrt{
\sbrak{
\vec{e}_2^{\top}\brak{\frac{\vec{D}}{f_0} e\sqrt{\frac{f_0}{\lambda_2\brak{1-e^2}}} \vec{e }_1}
}^2
-
\brak
{
 e\sqrt{\frac{f_0}{\lambda_2\brak{1-e^2}}} \vec{e }_1^{\top}\frac{\vec{D}}{f_0} e\sqrt{\frac{f_0}{\lambda_2\brak{1-e^2}}} \vec{e }_1 -1 
}
\brak{\vec{e}_2^{\top}\frac{\vec{D}}{f_0}\vec{e}_2}
}
}
{
\vec{e}_2^{\top}\frac{\vec{D}}{f_0}\vec{e}_2
}\norm{\vec{e}_2}
\label{eq:chord-len-sub-ell}
  \end{align}
  Since 
  \begin{align}
\vec{e}_2^{\top}\vec{D}\vec{e}_1 = 0, 
%\vec{e}_2^{\top}\vec{e}_2 = 0,
\vec{e}_1^{\top}\vec{D}\vec{e}_1 = \lambda_1,
\vec{e}_1^{\top}\vec{e}_1 = 1,
	  \norm{\vec{e}_2} = 1,
\vec{e}_2^{\top}\vec{D}\vec{e}_2 = \lambda_2,
  \end{align}
\eqref{eq:chord-len-sub-ell} can be expressed as 
  \begin{align}
	&		\frac{2\sqrt{\brak{1-\frac{\lambda_1e^2}{{\lambda_2\brak{1-e^2}}}}\brak{\frac{\lambda_2}{f_0}}}}
{
	\frac{\lambda_2}{f_0}
	} 	
	\\
	&=		2\frac{\sqrt{
		f_0\lambda_1}}{\lambda_2}
 & \brak{ \because e^2 = 1-\frac{\lambda_1}{\lambda_2}}
		   \end{align}
For the standard parabola, the parameters in 
\eqref{eq:chord-len} are
\begin{align}  
	\vec{q} =\vec{F} =  -\frac{\eta}{4\lambda_2}\vec{e}_1, \vec{m} = \vec{e}_1, \vec{V} = \vec{D},
	\vec{u} = \frac{\eta}{2}\vec{e}_1^{\top}, f = 0
\end{align}  

Substituting the above in
\eqref{eq:chord-len}, 
%			from \eqref{eq:conic_simp_temp_nonparab},  
%					from \eqref{eq:F-ell-hyp-parab}
%and 						 \\
the length of the latus rectum  can be expressed as
{\footnotesize
\begin{align}
 \frac{2\sqrt{
\sbrak{
\vec{e}_2^{\top}\brak{\vec{D}\brak{-\frac{\eta}{4\lambda_2}\vec{e}_1}+\frac{\eta}{2}\vec{e}_1}
}^2
-
\brak
{
\brak{-\frac{\eta}{4\lambda_2}\vec{e}_1}^{\top}\vec{D}\brak{-\frac{\eta}{4\lambda_2}\vec{e}_1} + 2\frac{\eta}{2}\vec{e}_1^{\top}\brak{-\frac{\eta}{4\lambda_2}\vec{e}_1} 
}
\brak{\vec{e}_2^{\top}\vec{D}\vec{e}_2}
}
}
{
\vec{e}_2^{\top}\vec{D}\vec{e}_2
}\norm{\vec{e}_2}
\label{eq:chord-len-sub}
  \end{align}
  }
  Since 
  \begin{align}
\vec{e}_2^{\top}\vec{D}\vec{e}_1 = 0, 
\vec{e}_2^{\top}\vec{e}_2 = 0,
\vec{e}_1^{\top}\vec{D}\vec{e}_1 = 0,
\vec{e}_1^{\top}\vec{e}_1 = 1,
	  \norm{\vec{e}_1} = 1,
\vec{e}_2^{\top}\vec{D}\vec{e}_2 = \lambda_2,
  \end{align}
\eqref{eq:chord-len-sub} can be expressed as 
  \begin{align}
	  2 \frac{\sqrt{\frac{\eta^2}{4\lambda_2}\lambda_2}}{\lambda_2}
	  = \frac{\eta}{\lambda_2}
  \end{align}
\end{proof}
\end{enumerate}

\subsection{ Tangent and Normal}
%\begin{enumerate}
\begin{enumerate}[label=\thesection.\arabic*.,ref=\thesection.\theenumi]
\item
  Given the point of contact $\vec{q}$, the equation of a tangent to \eqref{eq:conic_quad_form} is 
  \begin{align}
  \brak{\vec{V}\vec{q}+\vec{u}}^{\top}\vec{x}+\vec{u}^{\top}\vec{q}+f = 0
  \label{eq:conic_tangent_final}
  \end{align}

\begin{proof}
  The normal vector is obtained from \eqref{eq:conic_tangent_mq} and \eqref{eq:normal_vec}
  as
  %
  \begin{align}
  \label{eq:conic_normal_vec}
	  \kappa \vec{n} = \vec{V}\vec{q}+\vec{u}, \kappa \in \mathbb{R}
  \end{align}  
  From \eqref{eq:conic_normal_vec} and \eqref{eq:line_norm_eq}, the equation of the tangent is\begin{align}
    \brak{\vec{V}\vec{q}+\vec{u}}^{\top}\brak{\vec{x}-\vec{q}} &=0
    \\
    \implies \brak{\vec{V}\vec{q}+\vec{u}}^{\top}\vec{x}-\vec{q}^{\top}\vec{V}\vec{q}-\vec{u}^{\top}\vec{q} &= 0
    \end{align}
    which, upon substituting from \eqref{eq:conic_tangent_qquad} and simplifying yields 
  \eqref{eq:conic_tangent_final}
%	\eqref{eq:conic_tangent}.
\end{proof}
\item
	\label{eq:conic-p-contact-nonparab}
  If $\vec{V}^{-1}$ exists, given the normal vector $\vec{n}$, the tangent points of contact to \eqref{eq:conic_quad_form} are given by
\begin{align}
  \begin{split}
\vec{q}_i &= \vec{V}^{-1}\brak{\kappa_i \vec{n}-\vec{u}}, i = 1,2
\\
\text{where }\kappa_i &= \pm \sqrt{
\frac{
f_0
%\vec{u}^{\top}\vec{V}^{-1}\vec{u}-f
}
{
\vec{n}^{\top}\vec{V}^{-1}\vec{n}
}
}
  \end{split}
\label{eq:conic_tangent_qk}
\end{align}

\begin{proof}
  From \eqref{eq:conic_normal_vec},
\begin{align}
\label{eq:conic_normal_vec_q}
 \vec{q} = \vec{V}^{-1}\brak{\kappa \vec{n}-\vec{u}}, \quad \kappa \in \mathbb{R}
\end{align}
Substituting \eqref{eq:conic_normal_vec_q}
in \eqref{eq:conic_tangent_qquad},
\begin{align}
\brak{\kappa \vec{n}-\vec{u}}^{\top}\vec{V}^{-1}\brak{\kappa \vec{n}-\vec{u}} 
%\\
+ 2\vec{u}^{\top}\vec{V}^{-1}\brak{\kappa \vec{n}-\vec{u}} +f &= 0
\\
\implies 
\kappa^2 \vec{n}^{\top}\vec{V}^{-1}\vec{n} - \vec{u}^{\top}\vec{V}^{-1}\vec{u} + f &=0
 \\
 \text{or, } \kappa = \pm \sqrt{\frac{
	 %\vec{u}^{\top}\vec{V}^{-1}\vec{u}-f
	f_0 
 }{\vec{n}^{\top}\vec{V}^{-1}\vec{n}}} &
	\label{eq:conic_normal_k}
\end{align}
%
%yileding 
Substituting \eqref{eq:conic_normal_k} in \eqref{eq:conic_normal_vec_q}
yields \eqref{eq:conic_tangent_qk}.
%
\end{proof}


\item
	\label{eq:conic-p-contact-parab}
  If $\vec{V}$ is not invertible,  given the normal vector $\vec{n}$, the point of contact to \eqref{eq:conic_quad_form} is given by the matrix equation
\begin{align}
\label{eq:conic_tangent_q_eigen}
\begin{pmatrix}
\vec{\brak{u+\kappa \vec{n}}}^{\top} \\ \vec{V}
\end{pmatrix}
\vec{q} &= 
\begin{pmatrix}
-f
\\
\kappa\vec{n}-\vec{u}
\end{pmatrix}
\\
\text{where }  \kappa = \frac{\vec{p}_1^{\top}\vec{u}}{\vec{p}_1^{\top}\vec{n}}, \quad \vec{V}\vec{p}_1 &= 0
\label{eq:conic_tangent_qk_eigen}
\end{align}


\begin{proof}
  If $\vec{V}$ is non-invertible, it has a zero eigenvalue.  If the corresponding eigenvector is $\vec{p}_1$, then,
\begin{align}
\vec{V}\vec{p}_1 = 0
\label{eq:conic_zero_eigen}
\end{align}
From \eqref{eq:conic_normal_vec},
\begin{align}
\label{eq:conic_zero_eigen_normal}
\kappa \vec{n} &= \vec{V} \vec{q}+\vec{u}, \quad \kappa \in \mathbb{R}
\\
\implies \kappa \vec{p}_1^{\top}\vec{n} &= \vec{p}_1^{\top}\vec{V} \vec{q}+\vec{p}_1^{\top}\vec{u}
\\
\text{or, } \kappa \vec{p}_1^{\top}\vec{n} &= \vec{p}_1^{\top}\vec{u},  \quad \because \vec{p}_1^{\top} \vec{V} = 0, 
%\\
\quad 
\brak{\text{ from } \eqref{eq:conic_zero_eigen}}
%\label{eq:conic_normal_vec_q}
\end{align}
yielding $\kappa$ in \eqref{eq:conic_tangent_qk_eigen}. From \eqref{eq:conic_zero_eigen_normal},
\begin{align}
\kappa \vec{q}^{\top}\vec{n} &= \vec{q}^{\top}\vec{V} \vec{q}+\vec{q}^{\top}\vec{u}
\\
\implies \kappa \vec{q}^{\top}\vec{n} &= -f-\vec{q}^{\top}\vec{u} \quad \text{from } \eqref{eq:conic_tangent_qquad},
\\
\text{or, } \brak{\kappa \vec{n}+\vec{u}}^{\top}\vec{q} &= -f
\label{eq:conic_zero_eigen_normal_fq}
\end{align}
\eqref{eq:conic_zero_eigen_normal} can be expressed as
\begin{align}
\label{eq:conic_zero_eigen_normal_vq}
\vec{V} \vec{q} = \kappa \vec{n} - \vec{u}.
\end{align}
\eqref{eq:conic_zero_eigen_normal_fq} and \eqref{eq:conic_zero_eigen_normal_vq} clubbed together result in \eqref{eq:conic_tangent_q_eigen}.
\end{proof}
\item
	The normal vectors of the tangents  
	 to the conic in \eqref{eq:conic_quad_form} satisfy
\begin{align}
\vec{n} ^{\top}\vec{V}^{-1}\vec{n}-f_0 = 0
	\label{eq:dual-nf0}
    \end{align}

%
\begin{proof}
From 
  \eqref{eq:conic_tangent_mq}, the normal vector to  the tangent at $\vec{q}$ can be expressed as 
  \begin{align}
  \vec{n} &= \vec{V}\vec{q}+\vec{u} 
  \label{eq:conic_normal_n}
  \\
  \implies \vec{q} &= \vec{V}^{-1}\brak{\vec{n}-\vec{u} }
  \label{eq:conic_normal_q}
  \end{align}
  which upon substituting in \eqref{eq:conic_quad_form} yields
\begin{align}
    \label{eq:conic_quad_form_q}
    \brak{\vec{n}-\vec{u} }^{\top}\vec{V}^{-1}\vec{V}\vec{V}^{-1}\brak{\vec{n}-\vec{u} }+2\vec{u}^{\top}\vec{V}^{-1}\brak{\vec{n}-\vec{u} }+f&=0
	%\vec{u}^{\top}\vec{V}^{-1}\vec{u} +f&=0
    \end{align}
which can be simplified to obtain \eqref{eq:dual-nf0}.
\end{proof}
\item
	The normal vectors of the tangents 
to the conic in \eqref{eq:conic_quad_form} 
	from 
	a point $\vec{h}$ 
	are given by 

\begin{proof}
Let the equation of the tangent be 
\begin{align}
	\vec{n}^{\top}
	\vec{x} = c
	\label{eq:ext-tan}
\end{align}
If $\vec{q}$ be the point of contact,  since $\vec{h}, \vec{q}$ lie on 
	\eqref{eq:ext-tan},
\begin{align}
	\vec{n}^{\top}
	\vec{q} = 
	\vec{n}^{\top}
	\vec{h} = c
\end{align}
From 
  \eqref{eq:conic_normal_n}, 
%  \begin{align}
%	  \vec{n}^{\top}\vec{V}^{-1}\vec{n} &= \vec{n}^{\top}\brak{\vec{q}+\vec{V}^{-1}\vec{u} }
%	  \\
%%	  &= \vec{n}^{\top}\vec{q}+\vec{n}^{\top}\vec{V}^{-1}\vec{u} 
%\end{align}
\end{proof}
\item
	The normal vectors of the tangents 
to the conic in \eqref{eq:conic_quad_form} 
	from 
	a point $\vec{h}$ 
	are given by 
  \begin{align} 
  \label{eq:quad_form_pair_normvecs-sigma}
  \begin{split}
  \vec{n}_1 &= \vec{P}\myvec{\sqrt{\abs{\lambda_1}} \\[2mm]  \sqrt{\abs{\lambda_2}}}
  \\
  \vec{n}_2 &= \vec{P}\myvec{\sqrt{\abs{\lambda_1}} \\[2mm] - \sqrt{\abs{\lambda_2}}}
  \end{split}
  \end{align} 
  where $\lambda_i, \vec{P}$ are the eigenparameters of 
  \begin{align} 
		\bm{\Sigma} &= 
	   \brak{\vec{V}\vec{h}+\vec{u}}
	  \brak{\vec{V}\vec{h}+\vec{u}}^{\top}
   -\vec{V}
  \brak
  {
  \vec{h}^{\top}\vec{V}\vec{h} + 2\vec{u}^{\top}\vec{h} +f
  }.
	  \label{eq:h-tangents-sigma}
  \end{align}                    

\begin{proof}
 From \eqref{eq:tangent_roots},
 and
  \eqref{eq:conic_tangent_disc}
  \begin{align}
  \sbrak{
  \vec{m}^{\top}\brak{\vec{V}\vec{h}+\vec{u}}
  }^2 -\brak{\vec{m}^{\top}\vec{V}\vec{m}}
  \brak
  {
  \vec{h}^{\top}\vec{V}\vec{h} + 2\vec{u}^{\top}\vec{h} +f
  } &= 0                                                                                             
  \\
	  \implies 
  \vec{m}^{\top}  \sbrak{\brak{\vec{V}\vec{h}+\vec{u}}
	  \brak{\vec{V}\vec{h}+\vec{u}}^{\top}
   -\vec{V}
  \brak
  {
  \vec{h}^{\top}\vec{V}\vec{h} + 2\vec{u}^{\top}\vec{h} +f
  }}\vec{m} &= 0                                                                                             
  \label{eq:conic_tangent_disc-h}
  \end{align}                    
  yielding
	  \eqref{eq:h-tangents-sigma}.  Consequently, from 
  \eqref{eq:quad_form_pair_normvecs}, 
  \eqref{eq:quad_form_pair_normvecs-sigma}
  can be obtained.
\end{proof}
%
\end{enumerate}

%\section{Proofs}
%   \subsection{}
%
	\label{app:conic-parameters}
	From \eqref{eq:conic_quad_form_v}, using the fact that $\vec{V}$ is symmetric with $\vec{V} = \vec{V}^{\top}$,
  \begin{align}
	  \vec{V}^{\top} \vec{V}&=\brak{\norm{\vec{n}}^2\vec{I}-e^2\vec{n}\vec{n}^{\top}}^{\top}
	  \brak{\norm{\vec{n}}^2\vec{I}-e^2\vec{n}\vec{n}^{\top}}
    \\
	  \implies \vec{V}^{2} &= \norm{\vec{n}}^4\vec{I}+e^4\vec{n}\vec{n}^{\top}\vec{n}\vec{n}^{\top}
	  -2e^2\norm{\vec{n}}^2\vec{n}\vec{n}^{\top}
    \\
	  &= \norm{\vec{n}}^4\vec{I} + e^4\norm{\vec{n}}^2\vec{n}\vec{n}^{\top}
	%  \\
	  - 2e^2\norm{\vec{n}}^2\vec{n}\vec{n}^{\top}
    \\
	  &= \norm{\vec{n}}^4\vec{I} + e^2\brak{e^2 - 2}\norm{\vec{n}}^2\vec{n}\vec{n}^{\top}
    \\
	  &= \norm{\vec{n}}^4\vec{I} + \brak{e^2 - 2}\norm{\vec{n}}^2\brak{\norm{\vec{n}}^2\vec{I}- \vec{V}}
    \end{align}
%    
which can be expressed as
\begin{align}
  \vec{V}^{2} + \brak{e^2 - 2}\norm{\vec{n}}^2\vec{V} - \brak{e^2 - 1}\norm{\vec{n}}^4\vec{I}=0
  \label{eq:conic_quad_form_e_cayley}
\end{align}
	Using the Cayley-Hamilton theorem,
%	\cite{banchoff}, 
	\eqref{eq:conic_quad_form_e_cayley} results in the characteristic equation, 
\begin{align}
  \lambda^{2} - \brak{2-e^2}\norm{\vec{n}}^2\lambda + \brak{1-e^2 }\norm{\vec{n}}^4=0
\end{align}
which can be expressed as
\begin{align}
\brak{\frac{\lambda}{\norm{\vec{n}}^2}}^2 - \brak{2-e^2 }\brak{\frac{\lambda}{\norm{\vec{n}}^2}} 
	+ \brak{1-e^2 } &= 0
	\\
	\implies \frac{\lambda}{\norm{\vec{n}}^2} &= 1-e^2, 1
  \\
	\text{or, }\lambda_2 = \norm{\vec{n}}^2, \lambda_1 &= \brak{1-e^2}\lambda_2 
  \label{eq:conic_quad_form_lam_cayley}
\end{align}
From   \eqref{eq:conic_quad_form_lam_cayley}, the eccentricity of \eqref{eq:conic_quad_form} is given by 
\eqref{eq:conic_quad_form_e}.   
%
% By inspection, we find that 
% \begin{align}
%   \frac{\lambda}}{\norm{\vec{n}}^2} = 1
%   \label{eq:conic_quad_form_lam2_cayley}
% \end{align}
%satisfies \eqref{eq:conic_quad_form_lam_cayley}.
Multiplying both sides of    \eqref{eq:conic_quad_form_v} by $\vec{n}$,
\begin{align}
\vec{V} \vec{n}&=\norm{\vec{n}}^2\vec{n}-e^2\vec{n}\vec{n}^{\top}\vec{n} 
\\
&=\norm{\vec{n}}^2\brak{1-e^2}\vec{n} 
 \\
% &=\frac{\lambda_1}{\lambda_2}\norm{\vec{n}}^2\vec{n} 
% \end{align}
% upon substituting from \eqref{eq:conic_quad_form_e}  and simplifying.  From the above, it is obvious that $\vec{n}$ is an eigenvector
% of $\vec{V}$.  Choosing 
% \begin{align}
%   \lambda_2 = \norm{\vec{n}}^2,
%   \label{eq:eigevecn_lam2}
% \end{align}  
% we obtain 
% \begin{align}
  &=\lambda_1 \vec{n} 
	\\
  \label{eq:eigevecn}
\end{align}  
from \eqref{eq:conic_quad_form_lam_cayley}.
Thus,  $\lambda_1$ is the corresponding eigenvalue for $\vec{n}$.  From       \eqref{eq:eigevecP} and \eqref{eq:eigevecn}, this implies that 
\begin{align}  
	\vec{p}_1 &= \frac{\vec{n}}{\norm{\vec{n}}} 
	\\
	\text{or, }
   \vec{n}&= \norm{\vec{n}}\vec{p}_1  = \sqrt{\lambda_2}\vec{p}_1 
\end{align}  
from   \eqref{eq:conic_quad_form_lam_cayley} .
From \eqref{eq:conic_quad_form_u} and \eqref{eq:conic_quad_form_lam_cayley},
\begin{align}
%   \label{eq:conic_quad_form_v}
% \vec{V} &=\norm{\vec{n}}^2\vec{I}-e^2\vec{n}\vec{n}^{\top}, 
% \\
%\label{eq:conic_quad_form_u}
\vec{F}  &= \frac{ce^2\vec{n}-\vec{u}}{\lambda_2}
 \\
 \implies \norm{\vec{F}}^2  &= \frac{\brak{ce^2\vec{n}-\vec{u}}^{\top}\brak{ce^2\vec{n}-\vec{u}}}{\lambda_2^2}
 \\
 \implies \lambda_2^2\norm{\vec{F}}^2  &= c^2e^4\lambda_2-2ce^2\vec{u}^{\top}\vec{n}+\norm{\vec{u}}^2
 \label{eq:conic_quad_form_u_temp}
% f &= \norm{\vec{n}}^2\norm{\vec{F}}^2-c^2e^2
% %\\
    \end{align}
    Also, \eqref{eq:conic_quad_form_f} can be expressed as
    \begin{align}
    \lambda_2\norm{\vec{F}}^2 &= f+c^2e^2
    \label{eq:conic_quad_form_f_temp}
\end{align}
From  \eqref{eq:conic_quad_form_u_temp} and     \eqref{eq:conic_quad_form_f_temp},
\begin{align}
c^2e^4\lambda_2-2ce^2\vec{u}^{\top}\vec{n}+\norm{\vec{u}}^2 = \lambda_2\brak{f+c^2e^2}
\end{align}
\begin{align}
\implies \lambda_2e^2\brak{e^2-1}c^2-2ce^2\vec{u}^{\top}\vec{n}
	+\norm{\vec{u}}^2 - \lambda_2 f = 0
\end{align}
yielding
  \eqref{eq:conic_quad_form_F}. 
%\begin{align}
%\text{or, } c = 
%\begin{cases}
%  \frac{e\vec{u}^{\top}\vec{n} \pm \sqrt{e^2\brak{\vec{u}^{\top}\vec{n}}^2-\lambda_2\brak{e^2-1}\brak{\norm{\vec{u}}^2 - \lambda_2 f}}}{\lambda_2e\brak{e^2-1}} & e \ne 1
%  \\
%  \frac{\norm{\vec{u}}^2 - \lambda_2 f   }{2e^2\vec{u}^{\top}\vec{n}} & e = 1
%\end{cases}
%\end{align}


%  \subsection{}
%
  \label{app:parab}
	Using 
\eqref{eq:conic_affine}
%such that 
\eqref{eq:conic_quad_form} can be expressed as

%\item  
%Substituting \eqref{eq:conic_affine} in \eqref{eq:conic_quad_form}
\begin{align}
\brak{\vec{P}\vec{y}+\vec{c}}^{\top}\vec{V}\brak{\vec{P}\vec{y}+\vec{c}}+2\vec{u}^{\top}\brak{\vec{P}\vec{y}+\vec{c}}+ f
	= 0, 
\end{align}
yielding 
\begin{align}
\vec{y}^{\top}\vec{P}^{\top}\vec{V}\vec{P}\vec{y}+2\brak{\vec{V}\vec{c}+\vec{u}}^{\top}\vec{P}\vec{y}
+  \vec{c}^{\top}\vec{V}\vec{c} + 2\vec{u}^{\top}\vec{c} + f= 0
\label{eq:conic_simp_one}
\end{align}
%
From \eqref{eq:conic_simp_one} and \eqref{eq:conic_parmas_eig_def},
\begin{align}
\vec{y}^{\top}\vec{D}\vec{y}+2\brak{\vec{V}\vec{c}+\vec{u}}^{\top}\vec{P}\vec{y}
+  \vec{c}^{\top}\brak{\vec{V}\vec{c} + \vec{u}}+ \vec{u}^{\top}\vec{c} + f= 0
\label{eq:conic_simp}
\end{align}
When $\vec{V}^{-1}$ exists, choosing
\begin{align}
%\begin{split}
\vec{V}\vec{c}+\vec{u} &= \vec{0}, \quad \text{or}, \vec{c} = -\vec{V}^{-1}\vec{u},
\label{eq:conic_parmas_c_def}
\end{align}
%
%%From \eqref{eq:conic_parmas_k_def} and 
%%
and substituting \eqref{eq:conic_parmas_c_def}
in \eqref{eq:conic_simp}
yields \eqref{eq:conic_simp_temp_nonparab}. 
	\subsection{}
%\item  
When $\abs{\vec{V}} = 0, \lambda_1 = 0$ and 
\begin{align}
\vec{V}\vec{p}_1 = 0, 
\vec{V}\vec{p}_2 = \lambda_2\vec{p}_2.
\label{eq:conic_parab_eig_prop} 
\end{align}
where $\vec{p}_1,\vec{p}_2$ are the eigenvectors of $\vec{V}$ such that  \eqref{eq:conic_parmas_eig_def}
%
\begin{align}
\vec{P} = \myvec{\vec{p}_1 & \vec{p}_2},
\label{eq:eig_matrix}
\end{align}
Substituting \eqref{eq:eig_matrix}
in \eqref{eq:conic_simp},
\begin{align}
	\vec{y}^{\top}\vec{D}\vec{y}+2\brak{\vec{c}^{\top}\vec{V}+\vec{u}^{\top}}\myvec{\vec{p}_1 & \vec{p}_2}\vec{y}
	+  \vec{c}^{\top}\brak{\vec{V}\vec{c} + \vec{u}}+ \vec{u}^{\top}\vec{c} + f&= 0
\\
\implies \vec{y}^{\top}\vec{D}\vec{y}
+2\myvec{\brak{\vec{c}^{\top}\vec{V}+\vec{u}^{\top}}\vec{p}_1  \brak{\vec{c}^{\top}\vec{V}+\vec{u}^{\top}}\vec{p}_2}\vec{y}
	+  \vec{c}^{\top}\brak{\vec{V}\vec{c} + \vec{u}}+ \vec{u}^{\top}\vec{c} + f&= 0
\\
\implies \vec{y}^{\top}\vec{D}\vec{y}
+2\myvec{\vec{u}^{\top}\vec{p}_1 & \brak{\lambda_2\vec{c}^{\top}+\vec{u}^{\top}}\vec{p}_2}\vec{y}
	+  \vec{c}^{\top}\brak{\vec{V}\vec{c} + \vec{u}}+ \vec{u}^{\top}\vec{c} + f&= 0
\end{align}
upon substituting from 
 \eqref{eq:conic_parab_eig_prop} yielding
\begin{align}
\lambda_2y_2^2+2\brak{\vec{u}^{\top}\vec{p}_1}y_1+  2y_2\brak{\lambda_2\vec{c}+\vec{u}}^{\top}\vec{p}_2
	+  \vec{c}^{\top}\brak{\vec{V}\vec{c} + \vec{u}}+ \vec{u}^{\top}\vec{c} + f= 0
\label{eq:conic_parab_foc_len_temp} 
\end{align}
%which is the equation of a parabola. 
Thus, \eqref{eq:conic_parab_foc_len_temp} 
can be expressed as \eqref{eq:conic_simp_temp_parab} by choosing
\begin{align}
%\label{eq:eta}
\eta = 2\vec{u}^{\top}\vec{p}_1
\end{align}
%Choosing 
%\begin{align}
%\vec{u} + \lambda_2\vec{c} = 0,
%\vec{c}^{\top}\brak{\vec{V}\vec{c} + \vec{u}}+ \vec{u}^{\top}\vec{c} + f = 0,
%\end{align}
% the above equation becomes
%\begin{align}
%y_2^2= -\frac{2\vec{u}^{\top}\vec{p}_1}{ \lambda_2} \brak{y_1
%+  \frac{\vec{u}^{\top}\vec{V}\vec{u} - 2\lambda_2\vec{u}^{\top}\vec{u} + f\lambda_2^2}{2\vec{u}^{\top}\vec{p}_1\lambda_2^2}}
%\\
%or \eta = 2\vec{u}^{\top}\vec{p}_1
%%\label{eq:conic_simp_parab_new}
%\end{align}
and $\vec{c}$ in \eqref{eq:conic_simp} such that
\begin{align}
\label{eq:conic_parab_one}
2\vec{P}^{\top}\brak{\vec{V}\vec{c}+\vec{u}} &= \eta\myvec{1\\0}
\\
\vec{c}^{\top}\brak{\vec{V}\vec{c} + \vec{u}}+ \vec{u}^{\top}\vec{c} + f&= 0
\label{eq:conic_parab_two}
\end{align}
%we obtain  \eqref{eq:conic_simp_temp_parab}.
$\because
\vec{P}^{\top}\vec{P} = \vec{I}$,
multiplying \eqref{eq:conic_parab_one} by $\vec{P}$ yields
\begin{align}
\label{eq:conic_parab_one_eig}
	\brak{\vec{V}\vec{c}+\vec{u}} &= \frac{\eta}{2}\vec{p}_1,
\end{align}
which, upon substituting in \eqref{eq:conic_parab_two}
results in 
\begin{align}
\frac{\eta}{2}\vec{c}^{\top}\vec{p}_1 + \vec{u}^{\top}\vec{c} + f&= 0
\label{eq:conic_parab_two_eig}
\end{align}
\eqref{eq:conic_parab_one_eig} and \eqref{eq:conic_parab_two_eig} can be clubbed together to obtain \eqref{eq:conic_parab_c}.

%  \subsection{}
%
  \label{app:foc-dir}
%  \renewcommand{\theequation}{\theenumi}
\begin{enumerate}[label=\thesection.\arabic*.,ref=\thesection.\theenumi]
\numberwithin{equation}{enumi}

\item Substituting \eqref{eq:conic_affine} in \eqref{eq:conic_quad_form}
\begin{align}
\brak{\vec{P}\vec{y}+\vec{c}}^T\vec{V}\brak{\vec{P}\vec{y}+\vec{c}}+2\vec{u}^T\brak{\vec{P}\vec{y}+\vec{c}}+ f = 0, 
\end{align}
which can be expressed as
\begin{multline}
\vec{y}^T\vec{P}^T\vec{V}\vec{P}\vec{y}+2\brak{\vec{V}\vec{c}+\vec{u}}^T\vec{P}\vec{y}
\\
+  \vec{c}^T\vec{V}\vec{c} + 2\vec{u}^T\vec{c} + f= 0
\label{eq:conic_simp_one}
\end{multline}
%
From \eqref{eq:conic_simp_one} and \eqref{eq:conic_parmas_eig_def},
\begin{multline}
\vec{y}^T\vec{D}\vec{y}+2\brak{\vec{V}\vec{c}+\vec{u}}^T\vec{P}\vec{y}
\\
+  \vec{c}^T\brak{\vec{V}\vec{c} + \vec{u}}+ \vec{u}^T\vec{c} + f= 0
\label{eq:conic_simp}
\end{multline}
When $\vec{V}^{-1}$ exists,
\begin{align}
%\begin{split}
\vec{V}\vec{c}+\vec{u} &= \vec{0}, \quad \text{or}, \vec{c} = -\vec{V}^{-1}\vec{u},
\label{eq:conic_parmas_c_def}
\end{align}
%
%%From \eqref{eq:conic_parmas_k_def} and 
%%
and substituting \eqref{eq:conic_parmas_c_def}
in \eqref{eq:conic_simp}
yields \eqref{eq:conic_simp_temp_nonparab}. 
\item 
When $\abs{\vec{V}} = 0, \lambda_1 = 0$ and 
\begin{align}
\vec{V}\vec{p}_1 = 0, 
\vec{V}\vec{p}_2 = \lambda_2\vec{p}_2.
\label{eq:conic_parab_eig_prop} 
\end{align}
where $\vec{p}_1,\vec{p}_2$ are the eigenvectors of $\vec{V}$ such that  \eqref{eq:conic_parmas_eig_def}
%
\begin{align}
\vec{P} = \myvec{\vec{p}_1 & \vec{p}_2},
\label{eq:eig_matrix}
\end{align}
Substituting \eqref{eq:eig_matrix}
in \eqref{eq:conic_simp},
\begin{multline}
\vec{y}^T\vec{D}\vec{y}+2\brak{\vec{c}^T\vec{V}+\vec{u}^T}\myvec{\vec{p}_1 & \vec{p}_2}\vec{y}
\\
+  \vec{c}^T\brak{\vec{V}\vec{c} + \vec{u}}+ \vec{u}^T\vec{c} + f= 0
\\
\implies \vec{y}^T\vec{D}\vec{y}
\\
+2\myvec{\brak{\vec{c}^T\vec{V}+\vec{u}^T}\vec{p}_1 & \brak{\vec{c}^T\vec{V}+\vec{u}^T}\vec{p}_2}\vec{y}
\\
+  \vec{c}^T\brak{\vec{V}\vec{c} + \vec{u}}+ \vec{u}^T\vec{c} + f= 0
\\
\implies \vec{y}^T\vec{D}\vec{y}
\\
+2\myvec{\vec{u}^T\vec{p}_1 & \brak{\lambda_2\vec{c}^T+\vec{u}^T}\vec{p}_2}\vec{y}
\\
+  \vec{c}^T\brak{\vec{V}\vec{c} + \vec{u}}+ \vec{u}^T\vec{c} + f= 0
\\
\text{ from } \eqref{eq:conic_parab_eig_prop} 
\\
\implies \lambda_2y_2^2+2\brak{\vec{u}^T\vec{p}_1}y_1+  2y_2\brak{\lambda_2\vec{c}+\vec{u}}^T\vec{p}_2
\\
+  \vec{c}^T\brak{\vec{V}\vec{c} + \vec{u}}+ \vec{u}^T\vec{c} + f= 0
\label{eq:conic_parab_foc_len_temp} 
\end{multline}
which is the equation of a parabola. From \eqref{eq:conic_parab_foc_len_temp}, by comparing the coefficients of $y_2^2$ and $y_1$, the focal length of the parabola is obtained as     \ref{eq:conic_parab_foc_len} 
\begin{align}
    \mydet{\frac{2\eta}{\lambda_2}} = \mydet{\frac{2\vec{u}^T\vec{p}_1}{\lambda_2}}.
    \label{eq:conic_parab_foc_len} 
    \end{align}    
  %
Thus, \eqref{eq:conic_parab_foc_len_temp} 
can be expressed as \eqref{eq:conic_simp_temp_parab} by choosing
\begin{align}
\label{eq:eta}
\eta = \vec{u}^T\vec{p}_1
\end{align}
%Choosing 
%\begin{align}
%\vec{u} + \lambda_2\vec{c} = 0,
%\vec{c}^T\brak{\vec{V}\vec{c} + \vec{u}}+ \vec{u}^T\vec{c} + f = 0,
%\end{align}
% the above equation becomes
%\begin{align}
%y_2^2= -\frac{2\vec{u}^T\vec{p}_1}{ \lambda_2} \brak{y_1
%+  \frac{\vec{u}^T\vec{V}\vec{u} - 2\lambda_2\vec{u}^T\vec{u} + f\lambda_2^2}{2\vec{u}^T\vec{p}_1\lambda_2^2}}
%\\
%or \eta = 2\vec{u}^T\vec{p}_1
%%\label{eq:conic_simp_parab_new}
%\end{align}
and $\vec{c}$ in \eqref{eq:conic_simp} such that
\begin{align}
\label{eq:conic_parab_one}
\vec{P}^{T}\brak{\vec{V}\vec{c}+\vec{u}} &= \eta\myvec{1\\0}
\\
\vec{c}^T\brak{\vec{V}\vec{c} + \vec{u}}+ \vec{u}^T\vec{c} + f&= 0
\label{eq:conic_parab_two}
\end{align}
%we obtain  \eqref{eq:conic_simp_temp_parab}.
are satisfied.  Multiplying \eqref{eq:conic_parab_one} by $\vec{P}$ yields
\begin{align}
\label{eq:conic_parab_one_eig}
\brak{\vec{V}\vec{c}+\vec{u}} &= \eta\vec{p}_1,
\end{align}
which, upon substituting in \eqref{eq:conic_parab_two}
results in 
\begin{align}
\eta\vec{c}^T\vec{p}_1 + \vec{u}^T\vec{c} + f&= 0
\label{eq:conic_parab_two_eig}
\end{align}
\eqref{eq:conic_parab_one_eig} and \eqref{eq:conic_parab_two_eig} can be clubbed together to obtain \eqref{eq:conic_parab_c}.

\end{enumerate}

		\begin{enumerate}
			\item For the standard hyperbola/ellipse in \eqref{eq:conic_simp_temp_nonparab}, from 
					\eqref{eq:std-prm-P},
\eqref{eq:conic_quad_form_nc}
and 
					\eqref{eq:std-prm-u},
				\begin{align}
\label{eq:n-ell-hyp}
					\vec{n} &= \sqrt{\frac{\lambda_2}{f_0}} \vec{e}_1 
					\\
					c &= 
					%\pm \frac{\sqrt{-\lambda_2\brak{e^2-1}\brak{\lambda_2 f_0}}}{\lambda_2e\brak{e^2-1}}
					\pm \frac{\sqrt{-\frac{\lambda_2}{f_0}\brak{e^2-1}\brak{\frac{\lambda_2}{ f_0}}}}{\frac{\lambda_2}{f_0}e\brak{e^2-1}}
					\\
					&=\pm \frac{1}{e\sqrt{1-e^2}}
%					\\
%					&=\pm\sqrt{\abs{\frac{f_0}{\brak{1 - \frac{\lambda_1}{\lambda_2}}\frac{\lambda_1}{\lambda_2}}}}
\label{eq:c-ell-hyp}
				\end{align}
				yielding 
					\eqref{eq:dx-ell-hyp} upon substituting from 
\eqref{eq:conic_quad_form_e} and simplifying.
For the standard parabola in \eqref{eq:conic_simp_temp_parab},  from 
					\eqref{eq:std-prm-P},
\eqref{eq:conic_quad_form_nc}
and 
					\eqref{eq:std-prm-u}, noting that $f = 0$,

				\begin{align}
\label{eq:n-parab}
					\vec{n} &= \sqrt{\lambda_2} \vec{e}_1 
					\\
					c &=
	\frac{\norm{\frac{\eta}{2} \vec{e}_1}^2   }{2\vec{\brak{\frac{\eta}{2}} \brak{\vec{e}_1}^{\top}\vec{n}}} 
\\
					\\
					&= \frac{\eta}{4\sqrt{\lambda_2}}
\label{eq:c-parab}
				\end{align}
				yielding 
					\eqref{eq:dx-parab}.

				\item 	For the standard ellipse/hyperbola, substituting from
\eqref{eq:c-ell-hyp},
\eqref{eq:n-ell-hyp},
\eqref{eq:std-prm-u}
and \eqref{eq:conic_quad_form_e}
in \eqref{eq:conic_quad_form_F},
				\begin{align}
					\vec{F} &= \pm \frac{\brak{\frac{1}{e\sqrt{1-e^2}}}\brak{e^2}\sqrt{\frac{\lambda_2}{f_0}}\vec{e}_1}{\frac{\lambda_2}{f_0}}
					%\pm\sqrt{\abs{\frac{f_0}{\brak{1 - \frac{\lambda_1}{\lambda_2}}\frac{\lambda_1}{\lambda_2}}}}
					%\brak{1 - \frac{\lambda_1}{\lambda_2}}\frac{\sqrt{\lambda_2}}{\lambda_2}\vec{e}_1
 			\end{align}
			yielding
					\eqref{eq:F-ell-hyp-parab}
					after simplification.
					For the standard parabola, substituting from 
\eqref{eq:c-parab},
\eqref{eq:n-parab},
\eqref{eq:std-prm-u}
and \eqref{eq:conic_quad_form_e}
in \eqref{eq:conic_quad_form_F},			
				\begin{align}
	\vec{F}  &= \frac{\brak{\frac{\eta}{4\sqrt{\lambda_2}}}\sqrt{\lambda_2}\vec{e}_1-\vec{\frac{\eta}{2} \vec{e}_1}}{\lambda_2}
\\
				\end{align}
				yielding 
					\eqref{eq:F-ell-hyp-parab} after simplification.

		\end{enumerate}

%		\subsection{}
%
		\label{app:major}
		Since the major axis passes through the origin, 
  \begin{align}
	  \vec{q} =			\vec{0} 
\end{align}  
Further, from Corollary  
		\eqref{corr:axis},
  \begin{align}
  \vec{m}&= \vec{e}_2,  
\end{align} and
from 
    \eqref{eq:conic_simp_temp_nonparab},
  \begin{align}
	  \vec{V} =     \frac{\vec{D} }{f_0}, 
	   \vec{u} = 0, 
	   f = -1
	    \label{eq:latus_rectum_ellipse_param}
\end{align}  
Substituting the above in
\eqref{eq:chord-len}, 
\begin{align}
 \frac{2\sqrt{
\vec{e}_1^{\top}\frac{\vec{D}}{f_0}\vec{e}_1
}
}
{
\vec{e}_1^{\top}\frac{\vec{D}}{f_0}\vec{e}_1
}\norm{\vec{e}_1}
  \end{align}
  yielding 
\eqref{eq:chord-len-major}.
Similarly, for the minor axis, the only different parameter is 
  \begin{align}
  \vec{m}&= \vec{e}_2,  
\end{align} 
Substituting the above in
\eqref{eq:chord-len}, 
\begin{align}
 \frac{2\sqrt{
\vec{e}_2^{\top}\frac{\vec{D}}{f_0}\vec{e}_2
}
}
{
\vec{e}_2^{\top}\frac{\vec{D}}{f_0}\vec{e}_2
}\norm{\vec{e}_2}
  \end{align}
  yielding 
\eqref{eq:chord-len-minor}.
		\section{}
		\label{app:latus}
			The latus rectum is perpendicular to the major axis for the standard conic.  Hence, from Corollary  
		\eqref{corr:axis},
  \begin{align}
  \vec{m}&= \vec{e}_2,  
\end{align}  
Since it passes through the focus, from 
					\eqref{eq:F-ell-hyp-parab}
  \begin{align}
	  \vec{q} =			\vec{F} 
=
					 \pm e\sqrt{\frac{f_0}{\lambda_2\brak{1-e^2}}} \vec{e }_1
%					 \frac{e}{\sqrt{f_0\lambda_2\brak{1-e^2}}}\vec{e }_1
\end{align}  
for the standard hyperbola/ellipse.  Also, 
from 
    \eqref{eq:conic_simp_temp_nonparab},
  \begin{align}
	  \vec{V} =     \frac{\vec{D} }{f_0}, 
	   \vec{u} = 0, 
	   f = -1
	    \label{eq:latus_rectum_ellipse_param-new}
\end{align}  
Substituting the above in
\eqref{eq:chord-len}, 
\begin{align}
 \frac{2\sqrt{
\sbrak{
\vec{e}_2^{\top}\brak{\frac{\vec{D}}{f_0} e\sqrt{\frac{f_0}{\lambda_2\brak{1-e^2}}} \vec{e }_1}
}^2
-
\brak
{
 e\sqrt{\frac{f_0}{\lambda_2\brak{1-e^2}}} \vec{e }_1^{\top}\frac{\vec{D}}{f_0} e\sqrt{\frac{f_0}{\lambda_2\brak{1-e^2}}} \vec{e }_1 -1 
}
\brak{\vec{e}_2^{\top}\frac{\vec{D}}{f_0}\vec{e}_2}
}
}
{
\vec{e}_2^{\top}\frac{\vec{D}}{f_0}\vec{e}_2
}\norm{\vec{e}_2}
\label{eq:chord-len-sub-ell}
  \end{align}
  Since 
  \begin{align}
\vec{e}_2^{\top}\vec{D}\vec{e}_1 = 0, 
%\vec{e}_2^{\top}\vec{e}_2 = 0,
\vec{e}_1^{\top}\vec{D}\vec{e}_1 = \lambda_1,
\vec{e}_1^{\top}\vec{e}_1 = 1,
	  \norm{\vec{e}_2} = 1,
\vec{e}_2^{\top}\vec{D}\vec{e}_2 = \lambda_2,
  \end{align}
\eqref{eq:chord-len-sub-ell} can be expressed as 
  \begin{align}
	&		\frac{2\sqrt{\brak{1-\frac{\lambda_1e^2}{{\lambda_2\brak{1-e^2}}}}\brak{\frac{\lambda_2}{f_0}}}}
{
	\frac{\lambda_2}{f_0}
	} 	
	\\
	&=		2\frac{\sqrt{
		f_0\lambda_1}}{\lambda_2}
 & \brak{ \because e^2 = 1-\frac{\lambda_1}{\lambda_2}}
		   \end{align}
For the standard parabola, the parameters in 
\eqref{eq:chord-len} are
\begin{align}  
	\vec{q} =\vec{F} =  -\frac{\eta}{4\lambda_2}\vec{e}_1, \vec{m} = \vec{e}_1, \vec{V} = \vec{D},
	\vec{u} = \frac{\eta}{2}\vec{e}_1^{\top}, f = 0
\end{align}  

Substituting the above in
\eqref{eq:chord-len}, 
%			from \eqref{eq:conic_simp_temp_nonparab},  
%					from \eqref{eq:F-ell-hyp-parab}
%and 						 \\
the length of the latus rectum  can be expressed as
{\footnotesize
\begin{align}
 \frac{2\sqrt{
\sbrak{
\vec{e}_2^{\top}\brak{\vec{D}\brak{-\frac{\eta}{4\lambda_2}\vec{e}_1}+\frac{\eta}{2}\vec{e}_1}
}^2
-
\brak
{
\brak{-\frac{\eta}{4\lambda_2}\vec{e}_1}^{\top}\vec{D}\brak{-\frac{\eta}{4\lambda_2}\vec{e}_1} + 2\frac{\eta}{2}\vec{e}_1^{\top}\brak{-\frac{\eta}{4\lambda_2}\vec{e}_1} 
}
\brak{\vec{e}_2^{\top}\vec{D}\vec{e}_2}
}
}
{
\vec{e}_2^{\top}\vec{D}\vec{e}_2
}\norm{\vec{e}_2}
\label{eq:chord-len-sub}
  \end{align}
  }
  Since 
  \begin{align}
\vec{e}_2^{\top}\vec{D}\vec{e}_1 = 0, 
\vec{e}_2^{\top}\vec{e}_2 = 0,
\vec{e}_1^{\top}\vec{D}\vec{e}_1 = 0,
\vec{e}_1^{\top}\vec{e}_1 = 1,
	  \norm{\vec{e}_1} = 1,
\vec{e}_2^{\top}\vec{D}\vec{e}_2 = \lambda_2,
  \end{align}
\eqref{eq:chord-len-sub} can be expressed as 
  \begin{align}
	  2 \frac{\sqrt{\frac{\eta^2}{4\lambda_2}\lambda_2}}{\lambda_2}
	  = \frac{\eta}{\lambda_2}
  \end{align}

%
\end{enumerate}

%\latexprintindex

\end{document}



%\renewcommand{\theequation}{\theenumi}
%\subsubsection{Problem}

%\subsection{Chemistry}
%\begin{enumerate}[label=\arabic*.,ref=\thesubsection.\theenumi]
%\numberwithin{equation}{enumi}
\begin{enumerate}[label=\arabic*.,ref=\theenumi]
%\numberwithin{equation}{enumi}
\item Fig. \ref{fig:tri_sss_py} is generated using 
\begin{lstlisting}
math/codes/tri_sss.py
\end{lstlisting}
%
\begin{figure}
\centering
\includegraphics[width=\columnwidth]{./figs/tri_sss.pdf}
\caption{Triangle generated using python}
\label{fig:tri_sss_py}
\end{figure}
%
\item Fig. \ref{fig:tri_sss_tikz} is generated using 
\begin{lstlisting}
math/figs/tri_sss_alone.tex
\end{lstlisting}
\begin{figure}[!ht]
	\begin{center}
		
		\resizebox{\columnwidth}{!}{%Code by GVV Sharma
%December 7, 2019
%released under GNU GPL
%Drawing a triangle given 3 sides

\begin{tikzpicture}
[scale=2,>=stealth,point/.style={draw,circle,fill = black,inner sep=0.5pt},]

%Triangle sides
\def\a{6}
\def\b{5}
\def\c{4}
 
%Coordinates of A
%\def\p{{\a^2+\c^2-\b^2}/{(2*\a)}}
\def\p{2.25}
\def\q{{sqrt(\c^2-\p^2)}}

%Labeling points
\node (A) at (\p,\q)[point,label=above right:$A$] {};
\node (B) at (0, 0)[point,label=below left:$B$] {};
\node (C) at (\a, 0)[point,label=below right:$C$] {};

%Foot of perpendicular

\node (D) at (\p,0)[point,label=above right:$D$] {};

%Drawing triangle ABC
\draw (A) -- node[left] {$\textrm{c}$} (B) -- node[below] {$\textrm{a}$} (C) -- node[above,xshift=2mm] {$\textrm{b}$} (A);

%Drawing altitude AD
\draw (A) -- node[left] {$\textrm{h}$}(D);

%Drawing and marking angles
%\tkzMarkAngle[fill=orange!40,size=0.5cm,mark=](A,C,B)
%\tkzMarkAngle[fill=orange!40,size=0.4cm,mark=](D,B,A)
%\tkzMarkAngle[fill=green!40,size=0.5cm,mark=](B,A,C)
%\tkzMarkAngle[fill=green!40,size=0.5cm,mark=](C,B,D)
\tkzMarkRightAngle[fill=blue!20,size=.2](A,D,B)
%\tkzMarkRightAngle[fill=blue!20,size=.2](B,D,A)
%\tkzLabelAngle[pos=0.65](A,C,B){$\theta$}
%\tkzLabelAngle[pos=0.65](A,B,D){$\theta$}
%\tkzLabelAngle[pos=1](B,A,C){\rotatebox{-45}{$\alpha = 90\degree -\theta$}}
%\tkzLabelAngle[pos=0.65](C,B,D){$\alpha$}

\end{tikzpicture}
}
	\end{center}
	\caption{Triangle generated using \LaTeX Tikz.}
	\label{fig:tri_sss_tikz}	
\end{figure}

\end{enumerate}
\end{document}


