
\begin{enumerate}[label=\thesection.\arabic*,ref=\thesection.\theenumi]
\numberwithin{equation}{enumi}
\numberwithin{figure}{enumi}
\numberwithin{table}{enumi}
\item 
\label{chapters/12/8/1/1}
\iffalse
\documentclass[journal,10pt,twocolumn]{article}
\usepackage{graphicx}
\usepackage[margin=0.5in]{geometry}
\usepackage[cmex10]{amsmath}
\usepackage{array}
\usepackage{booktabs}
\usepackage{mathtools}
\title{\textbf{Conic section Assignment}}
\author{Jyothsna Paluchuri}
\date{September 2022}


\providecommand{\norm}[1]{\left\lVert#1\right\rVert}
\providecommand{\abs}[1]{\left\vert#1\right\vert}
\let\vec\mathbf
\newcommand{\myvec}[1]{\ensuremath{\begin{pmatrix}#1\end{pmatrix}}}
\newcommand{\mydet}[1]{\ensuremath{\begin{vmatrix}#1\end{vmatrix}}}
\providecommand{\brak}[1]{\ensuremath{\left(#1\right)}}
\providecommand{\lbrak}[1]{\ensuremath{\left(#1\right.}}
\providecommand{\rbrak}[1]{\ensuremath{\left.#1\right)}}
\providecommand{\sbrak}[1]{\ensuremath{{}\left[#1\right]}}

\begin{document}

\maketitle
\paragraph{\textit{Problem Statement} -
\fi
Find the area of the region bounded by the curve $x^2=4y$ and the lines y=2 and y=4 and the y-axis in the first quadrant.
\\
\solution
	\begin{figure}[!h]
		\centering
 \includegraphics[width=\columnwidth]{chapters/12/8/3/3/figs/conic.png}
		\caption{}
		\label{fig:12/8/3/3}
  	\end{figure}
\iffalse

\section*{\large Solution}

\begin{figure}[h]
\centering
\includegraphics[width=1\columnwidth]{conic.png}

\caption{The parabola formed by the curve $x^2 = 4y$ and the lines y=2 and y=4}
\label{fig:parabola}
\end{figure}

The given equation of parabola $x^2 = 4y$ can be written in the general quadratic form as
\begin{align}
    \label{eq:conic_quad_form}
    \vec{x}^{\top}\vec{V}\vec{x}+2\vec{u}^{\top}\vec{x}+f=0
    \end{align}
where
\fi
The conic parameters are
\begin{align}
	\vec{V} = \myvec{1 & 0\\0 & 0},
	\vec{u} = \myvec{0\\-2},
	f = 0
	%\\
\end{align}
\iffalse
The point of intersection of the lines y=2 and y=4 to the parabola is given by



The points of intersection of the line 
\begin{align}
	L: \quad \vec{x} = \vec{q} + \mu \vec{m} \quad \mu \in \mathbf{R}
\label{eq:conic_tangent}
\end{align}
with the conic section are given by
\begin{align}
\vec{x}_i = \vec{q} + \mu_i \vec{m}
\label{eq:conic_tangent_pts}
\end{align}
%
where
{\tiny
\begin{multline}
\mu_i = \frac{1}
{
\vec{m}^T\vec{V}\vec{m}
}
\lbrak{-\vec{m}^T\brak{\vec{V}\vec{q}+\vec{u}}}
\\
\pm
\rbrak{\sqrt{
\sbrak{
\vec{m}^T\brak{\vec{V}\vec{q}+\vec{u}}
}^2
-
\brak
{
\vec{q}^T\vec{V}\vec{q} + 2\vec{u}^T\vec{q} +f
}
\brak{\vec{m}^T\vec{V}\vec{m}}
}
}
\label{eq:tangent_roots}
\end{multline}
}


\fi
The vector parameters of 
$y-4=0$
are
\begin{align}
	\vec{h}_1=\myvec{0\\4},
	\vec{m}_1=\myvec{1\\0}
\end{align}
Substituting the above in \eqref{eq:tangent_roots},
\begin{align}
\mu_i=4,-4
\end{align}
yielding
the points of intersection with the parabola as
\begin{align}
\vec{a}_0=\myvec{4\\4},
\vec{a}_1=\myvec{-4\\4}
\end{align}
Similarly, for 
the line $y-2=0$, the vector parameters are
\begin{align}
\vec{h}_2=\myvec{0\\2},
\vec{m}_2=\myvec{1\\0}
\end{align}
yielding 
\begin{align}
\mu_i=2.8,-2.8
\end{align}
and the points of intersection
\begin{align}
\vec{a}_2=\myvec{2.8\\2},
\vec{a}_3=\myvec{-2.8\\2}
\end{align}
From Fig.
		\ref{fig:12/8/3/3},
the area of the parabola between the lines $y=2$ and $y=4$ is given by
\begin{align}
\int_{0}^{4} \ 2\sqrt{y} \,dy-\int_{0}^{2} \ 2\sqrt{y} \,dy
=6.895 
\end{align}
\iffalse


\section*{\large Construction}

{
\setlength\extrarowheight{5pt}
\begin{tabular}{|l|c|}
    \hline 
    \textbf{Points} & \textbf{intersection points} \\ \hline
	a0 & $\myvec{
   -2.8\\
   2
   } $ \\\hline
	a1 & $\myvec{
   2.8\\
   2
   } $ \\\hline
    
	a3 & $\myvec{
   -4\\
   4
   } $ \\\hline
	a2 & $\myvec{
   4\\
   4
   } $ \\\hline
      
      \end{tabular}
}

\end{document}
\fi

\item 
\label{chapters/12/8/1/2}
\iffalse
\documentclass[journal,10pt,twocolumn]{article}
\usepackage{graphicx}
\usepackage[margin=0.5in]{geometry}
\usepackage[cmex10]{amsmath}
\usepackage{array}
\usepackage{booktabs}
\usepackage{mathtools}
\title{\textbf{Conic section Assignment}}
\author{Jyothsna Paluchuri}
\date{September 2022}


\providecommand{\norm}[1]{\left\lVert#1\right\rVert}
\providecommand{\abs}[1]{\left\vert#1\right\vert}
\let\vec\mathbf
\newcommand{\myvec}[1]{\ensuremath{\begin{pmatrix}#1\end{pmatrix}}}
\newcommand{\mydet}[1]{\ensuremath{\begin{vmatrix}#1\end{vmatrix}}}
\providecommand{\brak}[1]{\ensuremath{\left(#1\right)}}
\providecommand{\lbrak}[1]{\ensuremath{\left(#1\right.}}
\providecommand{\rbrak}[1]{\ensuremath{\left.#1\right)}}
\providecommand{\sbrak}[1]{\ensuremath{{}\left[#1\right]}}

\begin{document}

\maketitle
\paragraph{\textit{Problem Statement} -
\fi
Find the area of the region bounded by the curve $x^2=4y$ and the lines y=2 and y=4 and the y-axis in the first quadrant.
\\
\solution
	\begin{figure}[!h]
		\centering
 \includegraphics[width=\columnwidth]{chapters/12/8/3/3/figs/conic.png}
		\caption{}
		\label{fig:12/8/3/3}
  	\end{figure}
\iffalse

\section*{\large Solution}

\begin{figure}[h]
\centering
\includegraphics[width=1\columnwidth]{conic.png}

\caption{The parabola formed by the curve $x^2 = 4y$ and the lines y=2 and y=4}
\label{fig:parabola}
\end{figure}

The given equation of parabola $x^2 = 4y$ can be written in the general quadratic form as
\begin{align}
    \label{eq:conic_quad_form}
    \vec{x}^{\top}\vec{V}\vec{x}+2\vec{u}^{\top}\vec{x}+f=0
    \end{align}
where
\fi
The conic parameters are
\begin{align}
	\vec{V} = \myvec{1 & 0\\0 & 0},
	\vec{u} = \myvec{0\\-2},
	f = 0
	%\\
\end{align}
\iffalse
The point of intersection of the lines y=2 and y=4 to the parabola is given by



The points of intersection of the line 
\begin{align}
	L: \quad \vec{x} = \vec{q} + \mu \vec{m} \quad \mu \in \mathbf{R}
\label{eq:conic_tangent}
\end{align}
with the conic section are given by
\begin{align}
\vec{x}_i = \vec{q} + \mu_i \vec{m}
\label{eq:conic_tangent_pts}
\end{align}
%
where
{\tiny
\begin{multline}
\mu_i = \frac{1}
{
\vec{m}^T\vec{V}\vec{m}
}
\lbrak{-\vec{m}^T\brak{\vec{V}\vec{q}+\vec{u}}}
\\
\pm
\rbrak{\sqrt{
\sbrak{
\vec{m}^T\brak{\vec{V}\vec{q}+\vec{u}}
}^2
-
\brak
{
\vec{q}^T\vec{V}\vec{q} + 2\vec{u}^T\vec{q} +f
}
\brak{\vec{m}^T\vec{V}\vec{m}}
}
}
\label{eq:tangent_roots}
\end{multline}
}


\fi
The vector parameters of 
$y-4=0$
are
\begin{align}
	\vec{h}_1=\myvec{0\\4},
	\vec{m}_1=\myvec{1\\0}
\end{align}
Substituting the above in \eqref{eq:tangent_roots},
\begin{align}
\mu_i=4,-4
\end{align}
yielding
the points of intersection with the parabola as
\begin{align}
\vec{a}_0=\myvec{4\\4},
\vec{a}_1=\myvec{-4\\4}
\end{align}
Similarly, for 
the line $y-2=0$, the vector parameters are
\begin{align}
\vec{h}_2=\myvec{0\\2},
\vec{m}_2=\myvec{1\\0}
\end{align}
yielding 
\begin{align}
\mu_i=2.8,-2.8
\end{align}
and the points of intersection
\begin{align}
\vec{a}_2=\myvec{2.8\\2},
\vec{a}_3=\myvec{-2.8\\2}
\end{align}
From Fig.
		\ref{fig:12/8/3/3},
the area of the parabola between the lines $y=2$ and $y=4$ is given by
\begin{align}
\int_{0}^{4} \ 2\sqrt{y} \,dy-\int_{0}^{2} \ 2\sqrt{y} \,dy
=6.895 
\end{align}
\iffalse


\section*{\large Construction}

{
\setlength\extrarowheight{5pt}
\begin{tabular}{|l|c|}
    \hline 
    \textbf{Points} & \textbf{intersection points} \\ \hline
	a0 & $\myvec{
   -2.8\\
   2
   } $ \\\hline
	a1 & $\myvec{
   2.8\\
   2
   } $ \\\hline
    
	a3 & $\myvec{
   -4\\
   4
   } $ \\\hline
	a2 & $\myvec{
   4\\
   4
   } $ \\\hline
      
      \end{tabular}
}

\end{document}
\fi

\item Find the area of the region bounded by ${x}^2
= 4{y}$, ${y} = 2$, ${y} = 4$ and the y-axis in the
first quadrant.
\label{chapters/12/8/1/3}
\item Find the area of the region bounded by the ellipse \(\frac{{x}^2}{16}\ + \frac{{y}^2}{9} = 1\)
\label{chapters/12/8/1/4}
\item Find the area of the region bounded by the ellipse \(\frac{{x}^2}{4}\ + \frac{{y}^2}{9} = 1\)
\label{chapters/12/8/1/5}
%\item 
%\label{chapters/12/8/1/3}
%%\iffalse
\documentclass[journal,10pt,twocolumn]{article}
\usepackage{graphicx}
\usepackage[margin=0.5in]{geometry}
\usepackage[cmex10]{amsmath}
\usepackage{array}
\usepackage{booktabs}
\usepackage{mathtools}
\title{\textbf{Conic section Assignment}}
\author{Jyothsna Paluchuri}
\date{September 2022}


\providecommand{\norm}[1]{\left\lVert#1\right\rVert}
\providecommand{\abs}[1]{\left\vert#1\right\vert}
\let\vec\mathbf
\newcommand{\myvec}[1]{\ensuremath{\begin{pmatrix}#1\end{pmatrix}}}
\newcommand{\mydet}[1]{\ensuremath{\begin{vmatrix}#1\end{vmatrix}}}
\providecommand{\brak}[1]{\ensuremath{\left(#1\right)}}
\providecommand{\lbrak}[1]{\ensuremath{\left(#1\right.}}
\providecommand{\rbrak}[1]{\ensuremath{\left.#1\right)}}
\providecommand{\sbrak}[1]{\ensuremath{{}\left[#1\right]}}

\begin{document}

\maketitle
\paragraph{\textit{Problem Statement} -
\fi
Find the area of the region bounded by the curve $x^2=4y$ and the lines y=2 and y=4 and the y-axis in the first quadrant.
\\
\solution
	\begin{figure}[!h]
		\centering
 \includegraphics[width=\columnwidth]{chapters/12/8/3/3/figs/conic.png}
		\caption{}
		\label{fig:12/8/3/3}
  	\end{figure}
\iffalse

\section*{\large Solution}

\begin{figure}[h]
\centering
\includegraphics[width=1\columnwidth]{conic.png}

\caption{The parabola formed by the curve $x^2 = 4y$ and the lines y=2 and y=4}
\label{fig:parabola}
\end{figure}

The given equation of parabola $x^2 = 4y$ can be written in the general quadratic form as
\begin{align}
    \label{eq:conic_quad_form}
    \vec{x}^{\top}\vec{V}\vec{x}+2\vec{u}^{\top}\vec{x}+f=0
    \end{align}
where
\fi
The conic parameters are
\begin{align}
	\vec{V} = \myvec{1 & 0\\0 & 0},
	\vec{u} = \myvec{0\\-2},
	f = 0
	%\\
\end{align}
\iffalse
The point of intersection of the lines y=2 and y=4 to the parabola is given by



The points of intersection of the line 
\begin{align}
	L: \quad \vec{x} = \vec{q} + \mu \vec{m} \quad \mu \in \mathbf{R}
\label{eq:conic_tangent}
\end{align}
with the conic section are given by
\begin{align}
\vec{x}_i = \vec{q} + \mu_i \vec{m}
\label{eq:conic_tangent_pts}
\end{align}
%
where
{\tiny
\begin{multline}
\mu_i = \frac{1}
{
\vec{m}^T\vec{V}\vec{m}
}
\lbrak{-\vec{m}^T\brak{\vec{V}\vec{q}+\vec{u}}}
\\
\pm
\rbrak{\sqrt{
\sbrak{
\vec{m}^T\brak{\vec{V}\vec{q}+\vec{u}}
}^2
-
\brak
{
\vec{q}^T\vec{V}\vec{q} + 2\vec{u}^T\vec{q} +f
}
\brak{\vec{m}^T\vec{V}\vec{m}}
}
}
\label{eq:tangent_roots}
\end{multline}
}


\fi
The vector parameters of 
$y-4=0$
are
\begin{align}
	\vec{h}_1=\myvec{0\\4},
	\vec{m}_1=\myvec{1\\0}
\end{align}
Substituting the above in \eqref{eq:tangent_roots},
\begin{align}
\mu_i=4,-4
\end{align}
yielding
the points of intersection with the parabola as
\begin{align}
\vec{a}_0=\myvec{4\\4},
\vec{a}_1=\myvec{-4\\4}
\end{align}
Similarly, for 
the line $y-2=0$, the vector parameters are
\begin{align}
\vec{h}_2=\myvec{0\\2},
\vec{m}_2=\myvec{1\\0}
\end{align}
yielding 
\begin{align}
\mu_i=2.8,-2.8
\end{align}
and the points of intersection
\begin{align}
\vec{a}_2=\myvec{2.8\\2},
\vec{a}_3=\myvec{-2.8\\2}
\end{align}
From Fig.
		\ref{fig:12/8/3/3},
the area of the parabola between the lines $y=2$ and $y=4$ is given by
\begin{align}
\int_{0}^{4} \ 2\sqrt{y} \,dy-\int_{0}^{2} \ 2\sqrt{y} \,dy
=6.895 
\end{align}
\iffalse


\section*{\large Construction}

{
\setlength\extrarowheight{5pt}
\begin{tabular}{|l|c|}
    \hline 
    \textbf{Points} & \textbf{intersection points} \\ \hline
	a0 & $\myvec{
   -2.8\\
   2
   } $ \\\hline
	a1 & $\myvec{
   2.8\\
   2
   } $ \\\hline
    
	a3 & $\myvec{
   -4\\
   4
   } $ \\\hline
	a2 & $\myvec{
   4\\
   4
   } $ \\\hline
      
      \end{tabular}
}

\end{document}
\fi

\item 
\label{chapters/12/8/1/6}
\iffalse
\documentclass[journal,10pt,twocolumn]{article}
\usepackage{graphicx}
\usepackage[margin=0.5in]{geometry}
\usepackage[cmex10]{amsmath}
\usepackage{array}
\usepackage{booktabs}
\usepackage{mathtools}
\title{\textbf{Conic section Assignment}}
\author{Jyothsna Paluchuri}
\date{September 2022}


\providecommand{\norm}[1]{\left\lVert#1\right\rVert}
\providecommand{\abs}[1]{\left\vert#1\right\vert}
\let\vec\mathbf
\newcommand{\myvec}[1]{\ensuremath{\begin{pmatrix}#1\end{pmatrix}}}
\newcommand{\mydet}[1]{\ensuremath{\begin{vmatrix}#1\end{vmatrix}}}
\providecommand{\brak}[1]{\ensuremath{\left(#1\right)}}
\providecommand{\lbrak}[1]{\ensuremath{\left(#1\right.}}
\providecommand{\rbrak}[1]{\ensuremath{\left.#1\right)}}
\providecommand{\sbrak}[1]{\ensuremath{{}\left[#1\right]}}

\begin{document}

\maketitle
\paragraph{\textit{Problem Statement} -
\fi
Find the area of the region bounded by the curve $x^2=4y$ and the lines y=2 and y=4 and the y-axis in the first quadrant.
\\
\solution
	\begin{figure}[!h]
		\centering
 \includegraphics[width=\columnwidth]{chapters/12/8/3/3/figs/conic.png}
		\caption{}
		\label{fig:12/8/3/3}
  	\end{figure}
\iffalse

\section*{\large Solution}

\begin{figure}[h]
\centering
\includegraphics[width=1\columnwidth]{conic.png}

\caption{The parabola formed by the curve $x^2 = 4y$ and the lines y=2 and y=4}
\label{fig:parabola}
\end{figure}

The given equation of parabola $x^2 = 4y$ can be written in the general quadratic form as
\begin{align}
    \label{eq:conic_quad_form}
    \vec{x}^{\top}\vec{V}\vec{x}+2\vec{u}^{\top}\vec{x}+f=0
    \end{align}
where
\fi
The conic parameters are
\begin{align}
	\vec{V} = \myvec{1 & 0\\0 & 0},
	\vec{u} = \myvec{0\\-2},
	f = 0
	%\\
\end{align}
\iffalse
The point of intersection of the lines y=2 and y=4 to the parabola is given by



The points of intersection of the line 
\begin{align}
	L: \quad \vec{x} = \vec{q} + \mu \vec{m} \quad \mu \in \mathbf{R}
\label{eq:conic_tangent}
\end{align}
with the conic section are given by
\begin{align}
\vec{x}_i = \vec{q} + \mu_i \vec{m}
\label{eq:conic_tangent_pts}
\end{align}
%
where
{\tiny
\begin{multline}
\mu_i = \frac{1}
{
\vec{m}^T\vec{V}\vec{m}
}
\lbrak{-\vec{m}^T\brak{\vec{V}\vec{q}+\vec{u}}}
\\
\pm
\rbrak{\sqrt{
\sbrak{
\vec{m}^T\brak{\vec{V}\vec{q}+\vec{u}}
}^2
-
\brak
{
\vec{q}^T\vec{V}\vec{q} + 2\vec{u}^T\vec{q} +f
}
\brak{\vec{m}^T\vec{V}\vec{m}}
}
}
\label{eq:tangent_roots}
\end{multline}
}


\fi
The vector parameters of 
$y-4=0$
are
\begin{align}
	\vec{h}_1=\myvec{0\\4},
	\vec{m}_1=\myvec{1\\0}
\end{align}
Substituting the above in \eqref{eq:tangent_roots},
\begin{align}
\mu_i=4,-4
\end{align}
yielding
the points of intersection with the parabola as
\begin{align}
\vec{a}_0=\myvec{4\\4},
\vec{a}_1=\myvec{-4\\4}
\end{align}
Similarly, for 
the line $y-2=0$, the vector parameters are
\begin{align}
\vec{h}_2=\myvec{0\\2},
\vec{m}_2=\myvec{1\\0}
\end{align}
yielding 
\begin{align}
\mu_i=2.8,-2.8
\end{align}
and the points of intersection
\begin{align}
\vec{a}_2=\myvec{2.8\\2},
\vec{a}_3=\myvec{-2.8\\2}
\end{align}
From Fig.
		\ref{fig:12/8/3/3},
the area of the parabola between the lines $y=2$ and $y=4$ is given by
\begin{align}
\int_{0}^{4} \ 2\sqrt{y} \,dy-\int_{0}^{2} \ 2\sqrt{y} \,dy
=6.895 
\end{align}
\iffalse


\section*{\large Construction}

{
\setlength\extrarowheight{5pt}
\begin{tabular}{|l|c|}
    \hline 
    \textbf{Points} & \textbf{intersection points} \\ \hline
	a0 & $\myvec{
   -2.8\\
   2
   } $ \\\hline
	a1 & $\myvec{
   2.8\\
   2
   } $ \\\hline
    
	a3 & $\myvec{
   -4\\
   4
   } $ \\\hline
	a2 & $\myvec{
   4\\
   4
   } $ \\\hline
      
      \end{tabular}
}

\end{document}
\fi

\item 
\label{chapters/12/8/1/7}
\documentclass[10pt,a4paper]{report}
%\usepackage[latin1]{inputenc}
\usepackage[utf8]{inputenc}
\usepackage{amsmath}
\usepackage{amsfonts}
\usepackage{amssymb}
\usepackage{graphicx}
\usepackage{multicol}
\usepackage{tabularx}
\usepackage{tikz}
\usetikzlibrary{arrows,shapes,automata,petri,positioning,calc}
\usepackage{hyperref}
\usepackage{tikz}
\usetikzlibrary{matrix,calc}
\usepackage[margin=0.5in]{geometry}
% ---- power functions -----% 
\newcommand{\myvec}[1]{\ensuremath{\begin{pmatrix}#1\end{pmatrix}}}
\let\vec\mathbf

\providecommand{\norm}[1]{\left\lVert#1\right\rVert}
\providecommand{\abs}[1]{\left\vert#1\right\vert}
\let\vec\mathbf

\newcommand{\mydet}[1]{\ensuremath{\begin{vmatrix}#1\end{vmatrix}}}
\providecommand{\brak}[1]{\ensuremath{\left(#1\right)}}
\providecommand{\lbrak}[1]{\ensuremath{\left(#1\right.}}
\providecommand{\rbrak}[1]{\ensuremath{\left.#1\right)}}
\providecommand{\sbrak}[1]{\ensuremath{{}\left[#1\right]}}
%-------end power functions----%
\newenvironment{Figure}
  {\par\medskip\noindent\minipage{\linewidth}}
  {\endminipage\par\medskip}
\begin{document}
%--------------------logo figure-------------------------%
\begin{figure*}[!tbp]
  \centering
  \begin{minipage}[b]{0.4\textwidth}
    %\includegraphics[scale=0.05]{fig/iitlogo.jpg} 
  \end{minipage}
  \hfill
  \vspace{5mm}\begin{minipage}[b]{0.4\textwidth}
\raggedleft  \includegraphics[scale=0.05]{iitlogo.jpg}  \

  \end{minipage}\vspace{0.2cm}
\end{figure*}
%--------------------name & rollno-----------------------
\raggedright \textbf{Name}:\hspace{1mm} T.ManasaReddy\hspace{3cm} \Large \textbf{Conic Assignment}\hspace{2.5cm} % 
\normalsize \textbf{Roll No.} :\hspace{1mm} FWC22048\vspace{1cm}
\begin{multicols}{2}

%----------------problem statement--------------%
\raggedright \textbf{Problem Statement:}\vspace{2mm}
\raggedright \\Find the area of the smaller part of the circle $x^2+y^2=a^2 $ cut off by the line x=$\frac{a}{\sqrt{2}}$\\
\vspace{5mm}
%-----------------------------solution---------------------------
\raggedright \textbf{SOLUTION}:\vspace{2mm}\\

%---------given----------------%
\raggedright \textbf{Given}:\vspace{2mm}\\
Equation of Circle is \\\vspace{1mm}
\begin{align}
x^2+y^2=a^2
\end{align}
Equation of line is \\ \vspace{1mm}
\begin{align}
x=\frac{a}{\sqrt{2}}
\end{align}
%-------------To find ------------------%
\textbf{To Find }\vspace{2mm}\\
To find the intersection points and area of shaded region shown in figure\vspace{2mm}  \\ 
%--------------steps----------------------%
\textbf{STEP-1}\vspace{2mm}\\
The given circle can be expressed as conics with parameters,\\ \vspace{1mm}
\begin{align}
\myvec{
1 & 0\\
0 & 1
}
\end{align}

\begin{align}
\vec{u}=0
\end{align} 
\begin{align}
f=-a^2
\end{align} \vspace{2mm}


\textbf{STEP-2}\vspace{2mm}\\
the given line equation can be written as\\ 
\begin{align} 
	\vec{x}=\begin{pmatrix}\frac{a}{\sqrt{2}} \\ 0 \\ \end{pmatrix}+k\begin{pmatrix} 0 \\ -1 \\ \end{pmatrix}
\end{align}

\textbf{STEP-3}\vspace{2mm}\\
The points of intersection of the line, \\ 
\begin{align}
L: \quad \vec{x} = \vec{q} + \kappa \vec{m} \quad \kappa \in \mathbb{R}
\end{align}
with the conic section, \\ 
\begin{align}
	\vec{x}^{\top}\vec{V}\vec{x} + 2\vec{u}^{\top} \vec{x} + f = 0
\end{align}
are given by \\
\begin{align}
\vec{x}_i = \vec{q} + \kappa_i \vec{m}
\end{align}
where, \\
{\tiny
\begin{multline}
\kappa_i = \frac{1}
{
\vec{m}^T\vec{V}\vec{m}
}
\lbrak{-\vec{m}^T\brak{\vec{V}\vec{q}+\vec{u}}}
\\
\pm
\rbrak{\sqrt{
\sbrak{
\vec{m}^T\brak{\vec{V}\vec{q}+\vec{u}}
}^2
-
\brak
{
\vec{q}^T\vec{V}\vec{q} + 2\vec{u}^T\vec{q} +f
}
\brak{\vec{m}^T\vec{V}\vec{m}}
}
}
\end{multline}
}
On substituting\\
\begin{align}
\vec{q} &= \myvec{
\frac{a}{\sqrt{2}}\\
0
} 
\end{align}
\begin{align}
\vec{m} = \myvec{ 0 \\ -1 }
\end{align}
With the given circle  as in eq(3),(4),(5),\\ 

The value of $\kappa$ ,\\
\begin{align}
    \kappa =\frac{a}{\sqrt{2}},\frac{-a}{\sqrt{2}}
\end{align}
by substituting eq(13) in eq(6)we get the
points of intersection of line with Circle \\
\begin{align}
    \vec{A}=\myvec{
\frac{a}{\sqrt{2}}\\
\frac{-a}{\sqrt{2}}
    }
\end{align}
\begin{align}
    \vec{B}=\myvec{
\frac{a}{\sqrt{2}}\\
\frac{a}{\sqrt{2}}
    }
\end{align}
\textbf{Result}
\begin{center}
 \includegraphics[scale=0.5]{conic.png}    
 \end{center}\vspace{1mm}
 From the figure,\\ \vspace{1mm}
Total area of portion is given by, \\ \vspace{1mm}
area\textbf{ APQ}=2* area of \textbf{APR}

\subsection*{Area of APR}
Since \textbf{APR} is in first Quadrant \\
\begin{align}
y=\sqrt{a^2-x^2}
\end{align}
\subsection*{Area of Circle}

\begin{align} 
\implies APQ=2*\int_{0}^{\frac{a}{\sqrt{2}}}\sqrt{a^2-x^2}\,dx 
\end{align}
by solving the above equation we get area of  smaller part of the circle\\



\vspace{3mm}
$\implies APQ=\frac{a^2}{2}[1+\frac{\pi}{2}]$\vspace{3mm}
 \vspace{2mm} \textbf{Construction}
\begin{center}
\setlength{\arrayrulewidth}{0.5mm}
\setlength{\tabcolsep}{6pt}
\renewcommand{\arraystretch}{1.5}
    \begin{tabular}{|l|c|}
    \hline 
    \textbf{Points} & \textbf{coordinates} \\ \hline
   B & $\myvec{
\frac{a}{\sqrt{2}}\\
\frac{a}{\sqrt{2}}
   } $ \\\hline
   A & $
   \vec{A}=\myvec{
\frac{a}{\sqrt{2}}\\
\frac{-a}{\sqrt{2}}
   } $ 
   \\\hline
      \end{tabular}
  \end{center}
  \end{multicols}
 
Get the python code of the figures from

\begin{table}[h]
\large
\centering
\framebox{
\url{https://github.com/manasa/MANASA_FWC/blob/main/conics/code/conic.py}}
\bibliographystyle{ieeetr}
\end{table} 
\end{document}

\item 
\label{chapters/12/8/1/8}
\iffalse
\documentclass[journal,10pt,twocolumn]{article}
\usepackage{graphicx}
\usepackage[margin=0.5in]{geometry}
\usepackage[cmex10]{amsmath}
\usepackage{array}
\usepackage{booktabs}
\usepackage{mathtools}
\title{\textbf{Conic section Assignment}}
\author{Jyothsna Paluchuri}
\date{September 2022}


\providecommand{\norm}[1]{\left\lVert#1\right\rVert}
\providecommand{\abs}[1]{\left\vert#1\right\vert}
\let\vec\mathbf
\newcommand{\myvec}[1]{\ensuremath{\begin{pmatrix}#1\end{pmatrix}}}
\newcommand{\mydet}[1]{\ensuremath{\begin{vmatrix}#1\end{vmatrix}}}
\providecommand{\brak}[1]{\ensuremath{\left(#1\right)}}
\providecommand{\lbrak}[1]{\ensuremath{\left(#1\right.}}
\providecommand{\rbrak}[1]{\ensuremath{\left.#1\right)}}
\providecommand{\sbrak}[1]{\ensuremath{{}\left[#1\right]}}

\begin{document}

\maketitle
\paragraph{\textit{Problem Statement} -
\fi
Find the area of the region bounded by the curve $x^2=4y$ and the lines y=2 and y=4 and the y-axis in the first quadrant.
\\
\solution
	\begin{figure}[!h]
		\centering
 \includegraphics[width=\columnwidth]{chapters/12/8/3/3/figs/conic.png}
		\caption{}
		\label{fig:12/8/3/3}
  	\end{figure}
\iffalse

\section*{\large Solution}

\begin{figure}[h]
\centering
\includegraphics[width=1\columnwidth]{conic.png}

\caption{The parabola formed by the curve $x^2 = 4y$ and the lines y=2 and y=4}
\label{fig:parabola}
\end{figure}

The given equation of parabola $x^2 = 4y$ can be written in the general quadratic form as
\begin{align}
    \label{eq:conic_quad_form}
    \vec{x}^{\top}\vec{V}\vec{x}+2\vec{u}^{\top}\vec{x}+f=0
    \end{align}
where
\fi
The conic parameters are
\begin{align}
	\vec{V} = \myvec{1 & 0\\0 & 0},
	\vec{u} = \myvec{0\\-2},
	f = 0
	%\\
\end{align}
\iffalse
The point of intersection of the lines y=2 and y=4 to the parabola is given by



The points of intersection of the line 
\begin{align}
	L: \quad \vec{x} = \vec{q} + \mu \vec{m} \quad \mu \in \mathbf{R}
\label{eq:conic_tangent}
\end{align}
with the conic section are given by
\begin{align}
\vec{x}_i = \vec{q} + \mu_i \vec{m}
\label{eq:conic_tangent_pts}
\end{align}
%
where
{\tiny
\begin{multline}
\mu_i = \frac{1}
{
\vec{m}^T\vec{V}\vec{m}
}
\lbrak{-\vec{m}^T\brak{\vec{V}\vec{q}+\vec{u}}}
\\
\pm
\rbrak{\sqrt{
\sbrak{
\vec{m}^T\brak{\vec{V}\vec{q}+\vec{u}}
}^2
-
\brak
{
\vec{q}^T\vec{V}\vec{q} + 2\vec{u}^T\vec{q} +f
}
\brak{\vec{m}^T\vec{V}\vec{m}}
}
}
\label{eq:tangent_roots}
\end{multline}
}


\fi
The vector parameters of 
$y-4=0$
are
\begin{align}
	\vec{h}_1=\myvec{0\\4},
	\vec{m}_1=\myvec{1\\0}
\end{align}
Substituting the above in \eqref{eq:tangent_roots},
\begin{align}
\mu_i=4,-4
\end{align}
yielding
the points of intersection with the parabola as
\begin{align}
\vec{a}_0=\myvec{4\\4},
\vec{a}_1=\myvec{-4\\4}
\end{align}
Similarly, for 
the line $y-2=0$, the vector parameters are
\begin{align}
\vec{h}_2=\myvec{0\\2},
\vec{m}_2=\myvec{1\\0}
\end{align}
yielding 
\begin{align}
\mu_i=2.8,-2.8
\end{align}
and the points of intersection
\begin{align}
\vec{a}_2=\myvec{2.8\\2},
\vec{a}_3=\myvec{-2.8\\2}
\end{align}
From Fig.
		\ref{fig:12/8/3/3},
the area of the parabola between the lines $y=2$ and $y=4$ is given by
\begin{align}
\int_{0}^{4} \ 2\sqrt{y} \,dy-\int_{0}^{2} \ 2\sqrt{y} \,dy
=6.895 
\end{align}
\iffalse


\section*{\large Construction}

{
\setlength\extrarowheight{5pt}
\begin{tabular}{|l|c|}
    \hline 
    \textbf{Points} & \textbf{intersection points} \\ \hline
	a0 & $\myvec{
   -2.8\\
   2
   } $ \\\hline
	a1 & $\myvec{
   2.8\\
   2
   } $ \\\hline
    
	a3 & $\myvec{
   -4\\
   4
   } $ \\\hline
	a2 & $\myvec{
   4\\
   4
   } $ \\\hline
      
      \end{tabular}
}

\end{document}
\fi

\item 
\label{chapters/12/8/1/9}
\iffalse
\documentclass[journal,10pt,twocolumn]{article}
\usepackage{graphicx}
\usepackage[margin=0.5in]{geometry}
\usepackage[cmex10]{amsmath}
\usepackage{array}
\usepackage{booktabs}
\usepackage{mathtools}
\title{\textbf{Conic section Assignment}}
\author{Jyothsna Paluchuri}
\date{September 2022}


\providecommand{\norm}[1]{\left\lVert#1\right\rVert}
\providecommand{\abs}[1]{\left\vert#1\right\vert}
\let\vec\mathbf
\newcommand{\myvec}[1]{\ensuremath{\begin{pmatrix}#1\end{pmatrix}}}
\newcommand{\mydet}[1]{\ensuremath{\begin{vmatrix}#1\end{vmatrix}}}
\providecommand{\brak}[1]{\ensuremath{\left(#1\right)}}
\providecommand{\lbrak}[1]{\ensuremath{\left(#1\right.}}
\providecommand{\rbrak}[1]{\ensuremath{\left.#1\right)}}
\providecommand{\sbrak}[1]{\ensuremath{{}\left[#1\right]}}

\begin{document}

\maketitle
\paragraph{\textit{Problem Statement} -
\fi
Find the area of the region bounded by the curve $x^2=4y$ and the lines y=2 and y=4 and the y-axis in the first quadrant.
\\
\solution
	\begin{figure}[!h]
		\centering
 \includegraphics[width=\columnwidth]{chapters/12/8/3/3/figs/conic.png}
		\caption{}
		\label{fig:12/8/3/3}
  	\end{figure}
\iffalse

\section*{\large Solution}

\begin{figure}[h]
\centering
\includegraphics[width=1\columnwidth]{conic.png}

\caption{The parabola formed by the curve $x^2 = 4y$ and the lines y=2 and y=4}
\label{fig:parabola}
\end{figure}

The given equation of parabola $x^2 = 4y$ can be written in the general quadratic form as
\begin{align}
    \label{eq:conic_quad_form}
    \vec{x}^{\top}\vec{V}\vec{x}+2\vec{u}^{\top}\vec{x}+f=0
    \end{align}
where
\fi
The conic parameters are
\begin{align}
	\vec{V} = \myvec{1 & 0\\0 & 0},
	\vec{u} = \myvec{0\\-2},
	f = 0
	%\\
\end{align}
\iffalse
The point of intersection of the lines y=2 and y=4 to the parabola is given by



The points of intersection of the line 
\begin{align}
	L: \quad \vec{x} = \vec{q} + \mu \vec{m} \quad \mu \in \mathbf{R}
\label{eq:conic_tangent}
\end{align}
with the conic section are given by
\begin{align}
\vec{x}_i = \vec{q} + \mu_i \vec{m}
\label{eq:conic_tangent_pts}
\end{align}
%
where
{\tiny
\begin{multline}
\mu_i = \frac{1}
{
\vec{m}^T\vec{V}\vec{m}
}
\lbrak{-\vec{m}^T\brak{\vec{V}\vec{q}+\vec{u}}}
\\
\pm
\rbrak{\sqrt{
\sbrak{
\vec{m}^T\brak{\vec{V}\vec{q}+\vec{u}}
}^2
-
\brak
{
\vec{q}^T\vec{V}\vec{q} + 2\vec{u}^T\vec{q} +f
}
\brak{\vec{m}^T\vec{V}\vec{m}}
}
}
\label{eq:tangent_roots}
\end{multline}
}


\fi
The vector parameters of 
$y-4=0$
are
\begin{align}
	\vec{h}_1=\myvec{0\\4},
	\vec{m}_1=\myvec{1\\0}
\end{align}
Substituting the above in \eqref{eq:tangent_roots},
\begin{align}
\mu_i=4,-4
\end{align}
yielding
the points of intersection with the parabola as
\begin{align}
\vec{a}_0=\myvec{4\\4},
\vec{a}_1=\myvec{-4\\4}
\end{align}
Similarly, for 
the line $y-2=0$, the vector parameters are
\begin{align}
\vec{h}_2=\myvec{0\\2},
\vec{m}_2=\myvec{1\\0}
\end{align}
yielding 
\begin{align}
\mu_i=2.8,-2.8
\end{align}
and the points of intersection
\begin{align}
\vec{a}_2=\myvec{2.8\\2},
\vec{a}_3=\myvec{-2.8\\2}
\end{align}
From Fig.
		\ref{fig:12/8/3/3},
the area of the parabola between the lines $y=2$ and $y=4$ is given by
\begin{align}
\int_{0}^{4} \ 2\sqrt{y} \,dy-\int_{0}^{2} \ 2\sqrt{y} \,dy
=6.895 
\end{align}
\iffalse


\section*{\large Construction}

{
\setlength\extrarowheight{5pt}
\begin{tabular}{|l|c|}
    \hline 
    \textbf{Points} & \textbf{intersection points} \\ \hline
	a0 & $\myvec{
   -2.8\\
   2
   } $ \\\hline
	a1 & $\myvec{
   2.8\\
   2
   } $ \\\hline
    
	a3 & $\myvec{
   -4\\
   4
   } $ \\\hline
	a2 & $\myvec{
   4\\
   4
   } $ \\\hline
      
      \end{tabular}
}

\end{document}
\fi

\item 
\label{chapters/12/8/1/10}
\documentclass[10pt,a4paper]{report}
\usepackage[latin1]{inputenc}
\usepackage{amsmath}
\usepackage{amsfonts}
\usepackage{amssymb}
\usepackage{graphicx}
\usepackage{hyperref}
\usepackage{multicol}
\usepackage[margin=0.5in]{geometry}
\usepackage{tikz}
\usepackage[document]{ragged2e}
\usepackage{romannum}
\usetikzlibrary{arrows,shapes.gates.logic.US,shapes.gates.logic.IEC,calc}
\usepackage{titlesec}
\titlespacing{\subsection}{1pt}{\parskip}{3pt}
\titlespacing{\subsubsection}{0pt}{\parskip}{-\parskip}
\titlespacing{\paragraph}{0pt}{\parskip}{\parskip}
\newcommand{\myvec}[1]{\ensuremath{\begin{pmatrix}#1\end{pmatrix}}}
\let\vec\mathbf

\newcommand{\mydet}[1]{\ensuremath{\begin{vmatrix}#1\end{vmatrix}}}
\providecommand{\brak}[1]{\ensuremath{\left(#1\right)}}
\providecommand{\lbrak}[1]{\ensuremath{\left(#1\right.}}
\providecommand{\rbrak}[1]{\ensuremath{\left.#1\right)}}
\providecommand{\sbrak}[1]{\ensuremath{{}\left[#1\right]}}

\begin{document}

\begin{multicols}{2}
\raggedright {\includegraphics[scale=0.06]{IITH logo.jpg}} \vspace{3mm}\\ \raggedleft Name:SHAIK KHAJA MASTAN AHMED\vspace{2mm}\\ 
\raggedleft Roll No.: FWC22052\vspace{2mm}\\ 
\raggedleft 19pa1a04e9@vishnu.edu.in \vspace{2mm}\\ 
\raggedleft Oct 2022 \vspace{5mm}\\
\end{multicols}

\centering \Large \textbf{MATRIX : CONIC ASSIGNMENT} \normalsize \vspace{10mm}

\begin{multicols}{2}

\section{Problem:}  
Find the area bounded by the curve $x^2=4y$ and the line $x=4y-2$.

\section{Solution: }
\raggedright \textbf{Input Parameters :}\\ \vspace{2mm}
\centering Curve Equation : $x^2=4y$. \\ \vspace{1mm}
Line Equation : $x=4y-2$.\\
\vspace{3mm}

\raggedright \textbf{To Find :}\\ \vspace{2mm}
\begin{enumerate}
\item Comparing the given curve equation with the standard equation of the conics and finding it's parameters.
\item Finding the required parameters for the line equation.
\item Finding the Point of Intersection of the to the curve.
\item Finding the area bounded by the curve and the line.
\end{enumerate}

\raggedright \textbf{Step - 1 :}\\ \vspace{2mm}
Curve Equation : $x^2=4y$. \\ \vspace{1mm}
The standard equation of the conics is given as :
\begin{align}
\vec{x}^{\top}\vec{V}\vec{x}+2\vec{u}^{\top}\vec{x}+f=0
\end{align}
The given curve  can be expressed as conics with \\parameters
\begin{align}
	\vec{V} &= \myvec{1 & 0\\0 & 0}, \vec{u} = \myvec{0 \\-2}, f = 0
	\end{align}

\raggedright \textbf{Step - 2 :}\\ \vspace{2mm}
Line Equation : $x=4y-2$. \\ \vspace{1mm}
From the above line equation below vectors are taken
\begin{align}
\vec{q} = \myvec{-2 \\0} , \vec{m}=\myvec{4\\1}
\end{align}

\raggedright \textbf{Step - 3 :}\\ \vspace{2mm}
The points of intersection of the line, \\ 
\begin{align}
L: \quad \vec{x} = \vec{q} + \mu \vec{m} \quad \mu \in \mathbb{R}
\end{align}
with the conic section, \\ 
\begin{align}
	\vec{x}^{\top}\vec{V}\vec{x} + 2\vec{u}^{\top} \vec{x} + f = 0
\end{align}
are given by \\
\begin{align}
\vec{x}_i = \vec{q} + \mu_i \vec{m}
\end{align}
where, \\
{\tiny
\begin{multline}
\mu_i = \frac{1}
{
\vec{m}^T\vec{V}\vec{m}
}
\lbrak{-\vec{m}^T\brak{\vec{V}\vec{q}+\vec{u}}}
\\
\pm
\rbrak{\sqrt{
\sbrak{
\vec{m}^T\brak{\vec{V}\vec{q}+\vec{u}}
}^2
-
\brak
{
\vec{q}^T\vec{V}\vec{q} + 2\vec{u}^T\vec{q} +f
}
\brak{\vec{m}^T\vec{V}\vec{m}}
}
}
\end{multline}
}
\raggedright On substituting $\vec{V},\vec{q} ,\vec{m}$ in the above equation,
we get the values of $\mu$. By substituting the values of $\mu$ in eq(6), \\we get the points of intersection of line with the given curve. \\
\centering $i.e., \vec{x}_1,\vec{x}_2$\\ 

\begin{align}
\therefore \vec{x}_1=\myvec{2\\1} , \vec{x}_2=\myvec{-1\\ \frac{1}{4}}
\end{align}

\raggedright \textbf{Step - 4 :}\\ \vspace{2mm}
The area bounded by the curve $x^2=4y$ and line $x=4y-2$ is given by\\

\begin{align}
\implies A=\int_{x_2}^{x_1} [f(x)-g(x)] \,dx
\end{align}

\begin{align}
\implies A=\int_{-1}^{2} (\frac{x+2}{4}-\frac{x^2}{4}) \,dx
\end{align}

\centering By solving we get the required area\\
$\therefore A = \frac{9}{8}$ 

\raggedright \textbf{Code Link :}\\ \vspace{2mm}
The below link realises the code of the above construction.\\
\begin{center}
\fbox{\parbox{8.5cm}{\url{https://github.com/19pa1a04e9/FWC-IITH/tree/main/Assignment-1/MATRICES/Conic/codes/conic.py}}}
\end{center}


\section{Termux Commands :}
\centering bash rncom.sh ..... Using Shell commands.


\section{Plot :} 
\begin{center}
  \includegraphics[scale=0.55]{conic.png}
  Figure 1
  	\end{center}

 
\end{multicols}
\end{document}

\item Find the area of the region bounded by the curve ${y}^2
= 4{x}$ and the line ${x} = 3$.
\label{chapters/12/8/1/11}
\end{enumerate}
Choose the correct answer in the following   Exercises 12 and 13.
\begin{enumerate} [resume]
\item Area lying in the first quadrant and bounded by the circle ${x}^2 + {y}^2 = 4$ and the lines ${x} = 0$ and ${x} = 2$ is \break
\label{chapters/12/8/1/12}
\begin{enumerate}[itemsep=+2mm]
\item $\pi$
\item $\dfrac{\pi}{2}$
\item $\dfrac{\pi}{3}$  
\item $\dfrac{\pi}{4}$
\end{enumerate}
\item Find the area of the region bounded by the curve $y^2 = 4x$, y-axis and the line $y = 3$. 
\label{chapters/12/8/1/13}
\\
\solution
\iffalse
\documentclass[12pt]{article}
\usepackage{graphicx}
\usepackage[none]{hyphenat}
\usepackage{graphicx}
\usepackage{listings}
\usepackage[english]{babel}
\usepackage{graphicx}
\usepackage{caption} 
\usepackage{booktabs}
\usepackage{array}
\usepackage{amssymb} % for \because
\usepackage{amsmath}   % for having text in math mode
\usepackage{extarrows} % for Row operations arrows
\usepackage{listings}
\lstset{
  frame=single,
  breaklines=true
}
\usepackage{hyperref}
  
%Following 2 lines were added to remove the blank page at the beginning
\usepackage{atbegshi}% http://ctan.org/pkg/atbegshi
\AtBeginDocument{\AtBeginShipoutNext{\AtBeginShipoutDiscard}}
\usepackage{gensymb}


%New macro definitions
\newcommand{\mydet}[1]{\ensuremath{\begin{vmatrix}#1\end{vmatrix}}}
\providecommand{\brak}[1]{\ensuremath{\left(#1\right)}}
\providecommand{\sbrak}[1]{\ensuremath{{}\left[#1\right]}}
\providecommand{\norm}[1]{\left\lVert#1\right\rVert}
\providecommand{\abs}[1]{\left\vert#1\right\vert}
\newcommand{\solution}{\noindent \textbf{Solution: }}
\newcommand{\myvec}[1]{\ensuremath{\begin{pmatrix}#1\end{pmatrix}}}
\let\vec\mathbf


\begin{document}

\begin{center}
\title{\textbf{Chords}}
\date{\vspace{-5ex}} %Not to print date automatically
\maketitle
\end{center}
\setcounter{page}{1}

\section{12$^{th}$ Maths - Chapter 8}
This is Problem-13 from Exercise 8.1 
\begin{enumerate}

\solution 
\item 
	\fi
	The given equation of the curve can be rearranged as
\begin{align}
	y^2-4x &= 0 \\
        \label{eq:chapters/12/8/1/13/Eq1}
	\implies \vec{x}^\top\myvec{0 & 0 \\ 0 & 1}\vec{x} + 2\myvec{-2 & 0}\vec{x}+0 &= 0 
\end{align}
The above equation can be equated to the generic equation of conic sections
\begin{align}
	\label{eq:chapters/12/8/1/13/Eq2}
	g\brak{\vec{x}} = \vec{x}^T\vec{V}\vec{x} + 2\vec{u}^T\vec{x} + f = 0 
\end{align}
Comparing coefficients of both equations \eqref{eq:chapters/12/8/1/13/Eq1} and \eqref{eq:chapters/12/8/1/13/Eq2} 
\begin{align}
	\vec{V} &= \myvec{ 0 & 0 \\ 0 & 1} \\
	\vec{u} &= \myvec{-2 \\ 0} \\
	f &= 0
\end{align}
For the given line $y=3$, the parameters are
\begin{align}
	\vec{h} = \myvec{0 \\ 3} , \vec{m} = \myvec{1 \\ 0 }
\end{align}
To calculate the point of contact of line with the conic, we use
\begin{align}
	\label{eq:chapters/12/8/1/13/Eq3}
	\mu^2\vec{m}^\top\vec{V}\vec{m}+2\mu\vec{m}^\top\brak{\vec{V}\vec{h}+\vec{u}}+g\brak{\vec{h}}= 0 
\end{align}
\begin{multline}
	g\brak{\vec{h}}=\myvec{0 & 3}\myvec{0 & 0 \\ 0 & 1}\myvec{0 \\3} \\
	+ 2\myvec{-2 & 0}\myvec{0 \\ 3} + 0 \\
	 \implies g\brak{\vec{h}} = \myvec{0 & 3}\myvec{0 \\3} + 2\myvec{0} \\ 
	 \implies g\brak{\vec{h}} = 9 
\end{multline}
\begin{multline}
	\eqref{eq:chapters/12/8/1/13/Eq3} \implies \mu^2\myvec{1 & 0}\myvec{0 & 0 \\ 0 & 1}\myvec{1 \\ 0} \\
	 + 2\mu\myvec{1 & 0}\brak{\myvec{0 & 0 \\ 0 & 1}\myvec{0 \\3}+\myvec{-2 \\ 0}} + 9 = 0 \\
	\implies \mu^2\brak{0}+2\mu\myvec{1 & 0}\myvec{-2 \\3} + 9 = 0 \\
	\implies -4\mu + 9 = 0  \\
	\implies \mu  = \frac{9}{4} 
\end{multline}
The point of contact is given as
\begin{align}
	\vec{a}_0 = \myvec{\frac{9}{4}  \\[1pt] \\ 3}
\end{align}
The desired area of the region is given as
\begin{align}
	\int_{0}^{3} \ \frac{y^2}{4} \,dy &= \frac{1}{12}\sbrak{y^3}_{0}^{3} \\
	&= \frac{1}{12}\brak{27-0} \\
	&= \frac{9}{4} \text{ sq.units}
\end{align}
The relevant diagram is shown in Figure \ref{fig:chapters/12/8/1/13/Fig1}
\begin{figure}[!h]
	\begin{center}
		\includegraphics[width=\columnwidth]{chapters/12/8/1/13/figs/problem13.pdf}
	\end{center}
\caption{}
\label{fig:chapters/12/8/1/13/Fig1}
\end{figure}

\item 
\label{chapters/12/8/3/3}
\iffalse
\documentclass[journal,10pt,twocolumn]{article}
\usepackage{graphicx}
\usepackage[margin=0.5in]{geometry}
\usepackage[cmex10]{amsmath}
\usepackage{array}
\usepackage{booktabs}
\usepackage{mathtools}
\title{\textbf{Conic section Assignment}}
\author{Jyothsna Paluchuri}
\date{September 2022}


\providecommand{\norm}[1]{\left\lVert#1\right\rVert}
\providecommand{\abs}[1]{\left\vert#1\right\vert}
\let\vec\mathbf
\newcommand{\myvec}[1]{\ensuremath{\begin{pmatrix}#1\end{pmatrix}}}
\newcommand{\mydet}[1]{\ensuremath{\begin{vmatrix}#1\end{vmatrix}}}
\providecommand{\brak}[1]{\ensuremath{\left(#1\right)}}
\providecommand{\lbrak}[1]{\ensuremath{\left(#1\right.}}
\providecommand{\rbrak}[1]{\ensuremath{\left.#1\right)}}
\providecommand{\sbrak}[1]{\ensuremath{{}\left[#1\right]}}

\begin{document}

\maketitle
\paragraph{\textit{Problem Statement} -
\fi
Find the area of the region bounded by the curve $x^2=4y$ and the lines y=2 and y=4 and the y-axis in the first quadrant.
\\
\solution
	\begin{figure}[!h]
		\centering
 \includegraphics[width=\columnwidth]{chapters/12/8/3/3/figs/conic.png}
		\caption{}
		\label{fig:12/8/3/3}
  	\end{figure}
\iffalse

\section*{\large Solution}

\begin{figure}[h]
\centering
\includegraphics[width=1\columnwidth]{conic.png}

\caption{The parabola formed by the curve $x^2 = 4y$ and the lines y=2 and y=4}
\label{fig:parabola}
\end{figure}

The given equation of parabola $x^2 = 4y$ can be written in the general quadratic form as
\begin{align}
    \label{eq:conic_quad_form}
    \vec{x}^{\top}\vec{V}\vec{x}+2\vec{u}^{\top}\vec{x}+f=0
    \end{align}
where
\fi
The conic parameters are
\begin{align}
	\vec{V} = \myvec{1 & 0\\0 & 0},
	\vec{u} = \myvec{0\\-2},
	f = 0
	%\\
\end{align}
\iffalse
The point of intersection of the lines y=2 and y=4 to the parabola is given by



The points of intersection of the line 
\begin{align}
	L: \quad \vec{x} = \vec{q} + \mu \vec{m} \quad \mu \in \mathbf{R}
\label{eq:conic_tangent}
\end{align}
with the conic section are given by
\begin{align}
\vec{x}_i = \vec{q} + \mu_i \vec{m}
\label{eq:conic_tangent_pts}
\end{align}
%
where
{\tiny
\begin{multline}
\mu_i = \frac{1}
{
\vec{m}^T\vec{V}\vec{m}
}
\lbrak{-\vec{m}^T\brak{\vec{V}\vec{q}+\vec{u}}}
\\
\pm
\rbrak{\sqrt{
\sbrak{
\vec{m}^T\brak{\vec{V}\vec{q}+\vec{u}}
}^2
-
\brak
{
\vec{q}^T\vec{V}\vec{q} + 2\vec{u}^T\vec{q} +f
}
\brak{\vec{m}^T\vec{V}\vec{m}}
}
}
\label{eq:tangent_roots}
\end{multline}
}


\fi
The vector parameters of 
$y-4=0$
are
\begin{align}
	\vec{h}_1=\myvec{0\\4},
	\vec{m}_1=\myvec{1\\0}
\end{align}
Substituting the above in \eqref{eq:tangent_roots},
\begin{align}
\mu_i=4,-4
\end{align}
yielding
the points of intersection with the parabola as
\begin{align}
\vec{a}_0=\myvec{4\\4},
\vec{a}_1=\myvec{-4\\4}
\end{align}
Similarly, for 
the line $y-2=0$, the vector parameters are
\begin{align}
\vec{h}_2=\myvec{0\\2},
\vec{m}_2=\myvec{1\\0}
\end{align}
yielding 
\begin{align}
\mu_i=2.8,-2.8
\end{align}
and the points of intersection
\begin{align}
\vec{a}_2=\myvec{2.8\\2},
\vec{a}_3=\myvec{-2.8\\2}
\end{align}
From Fig.
		\ref{fig:12/8/3/3},
the area of the parabola between the lines $y=2$ and $y=4$ is given by
\begin{align}
\int_{0}^{4} \ 2\sqrt{y} \,dy-\int_{0}^{2} \ 2\sqrt{y} \,dy
=6.895 
\end{align}
\iffalse


\section*{\large Construction}

{
\setlength\extrarowheight{5pt}
\begin{tabular}{|l|c|}
    \hline 
    \textbf{Points} & \textbf{intersection points} \\ \hline
	a0 & $\myvec{
   -2.8\\
   2
   } $ \\\hline
	a1 & $\myvec{
   2.8\\
   2
   } $ \\\hline
    
	a3 & $\myvec{
   -4\\
   4
   } $ \\\hline
	a2 & $\myvec{
   4\\
   4
   } $ \\\hline
      
      \end{tabular}
}

\end{document}
\fi

\item 
\label{chapters/12/8/3/7}
\documentclass[10pt,a4paper]{report}
%\usepackage[latin1]{inputenc}
\usepackage[utf8]{inputenc}
\usepackage{amsmath}
\usepackage{amsfonts}
\usepackage{amssymb}
\usepackage{graphicx}
\usepackage{multicol}
\usepackage{tabularx}
\usepackage{tikz}
\usetikzlibrary{arrows,shapes,automata,petri,positioning,calc}
\usepackage{hyperref}
\usepackage{tikz}
\usetikzlibrary{matrix,calc}
\usepackage[margin=0.5in]{geometry}
% ---- power functions -----% 
\newcommand{\myvec}[1]{\ensuremath{\begin{pmatrix}#1\end{pmatrix}}}
\let\vec\mathbf

\providecommand{\norm}[1]{\left\lVert#1\right\rVert}
\providecommand{\abs}[1]{\left\vert#1\right\vert}
\let\vec\mathbf

\newcommand{\mydet}[1]{\ensuremath{\begin{vmatrix}#1\end{vmatrix}}}
\providecommand{\brak}[1]{\ensuremath{\left(#1\right)}}
\providecommand{\lbrak}[1]{\ensuremath{\left(#1\right.}}
\providecommand{\rbrak}[1]{\ensuremath{\left.#1\right)}}
\providecommand{\sbrak}[1]{\ensuremath{{}\left[#1\right]}}
%-------end power functions----%
\newenvironment{Figure}
  {\par\medskip\noindent\minipage{\linewidth}}
  {\endminipage\par\medskip}
\begin{document}
%--------------------logo figure-------------------------%
\begin{figure*}[!tbp]
  \centering
  \begin{minipage}[b]{0.4\textwidth}
    \includegraphics[scale=0.05]{iitlogo.jpg} 
  \end{minipage}
  \hfill
  \vspace{5mm}\begin{minipage}[b]{0.4\textwidth}
\raggedleft  \includegraphics[scale=0.05]{nrc.png}  \

  \end{minipage}\vspace{0.2cm}
\end{figure*}
%--------------------name & rollno-----------------------
\raggedright \textbf{Name}:\hspace{1mm} Cheenepalli Chandana\hspace{2cm} \Large \textbf{Conic Assignment}\hspace{2.5cm} % 
\normalsize \textbf{Roll No.} :\hspace{1mm} FWC22062\vspace{1cm}
\begin{multicols}{2}

%----------------problem statement--------------%
\raggedright \textbf{Problem Statement:}\vspace{2mm}
\raggedright \\Find the area enclosed by the parabola $4y=3x^2 $ and the line $2y=3x+12$\\
\vspace{5mm}
%-----------------------------solution---------------------------
\raggedright \textbf{SOLUTION}:\vspace{2mm}\\

%---------given----------------%
\raggedright \textbf{Given}:\vspace{2mm}\\
Equation of parabola is \\\vspace{1mm}
\begin{align}
4y=3x^2
\end{align}
Equation of line is \\ \vspace{1mm}
\begin{align}
2y=3x+12
\end{align}
%-------------To find ------------------%
\textbf{To Find }\vspace{2mm}\\
To find the intersection points and area enclosed by the parabola and line shown in figure\vspace{2mm}  \\ 
%--------------steps----------------------%
\textbf{STEP-1}\vspace{2mm}\\
The given parabola can be expressed as conics with parameters,\\ \vspace{1mm}
\begin{align}
	\vec{x}^{\top}\vec{V}\vec{x} + 2\vec{u}^{\top} \vec{x} + f = 0
\end{align}
\begin{align}
\vec{V}=\myvec{
3 & 0\\
0 & 0
}
\end{align}

\begin{align}
\vec{u}=\myvec{0\\-2}
\end{align} 
\begin{align}
f=0
\end{align} \vspace{2mm}


\textbf{STEP-2}\vspace{2mm}\\
the given line equation can be written as\\ 
\begin{align} 
	\vec{n}^{\top}\vec{X}=c
\end{align}
Where
\begin{align}
\vec{n}=\myvec{-3\\2},\vec{m}=\myvec{2\\3}
\end{align}
\textbf{STEP-3}\vspace{2mm}\\
The points of intersection of the line, \\ 
\begin{align}
L: \quad \vec{x} = \vec{q} + \kappa \vec{m} \quad \kappa \in \mathbb{R}
\end{align}
with the conic section, \\ 
\begin{align}
	\vec{x}^{\top}\vec{V}\vec{x} + 2\vec{u}^{\top} \vec{x} + f = 0
\end{align}
are given by \\
\begin{align}
\vec{x}_i = \vec{q} + \kappa_i \vec{m}
\end{align}
where, \\
{\tiny
\begin{multline}
\kappa_i = \frac{1}
{
\vec{m}^T\vec{V}\vec{m}
}
\lbrak{-\vec{m}^T\brak{\vec{V}\vec{q}+\vec{u}}}
\\
\pm
\rbrak{\sqrt{
\sbrak{
\vec{m}^T\brak{\vec{V}\vec{q}+\vec{u}}
}^2
-
\brak
{
\vec{q}^T\vec{V}\vec{q} + 2\vec{u}^T\vec{q} +f
}
\brak{\vec{m}^T\vec{V}\vec{m}}
}
}
\end{multline}
}
On substituting\\
\begin{align}
\vec{q} &= \myvec{
-2\\
3
} 
\end{align}
\begin{align}
\vec{m} = \myvec{2 \\ 3}
\end{align}
With the given parabola as in eq(3),(4),(5),\\ 

The value of $\kappa$ ,\\
\begin{align}
    \kappa =-2.5,2.7
\end{align}
by substituting eq(13) in eq(6)we get the
points of intersection of line with parabola \\
\begin{align}
    \vec{A}=\myvec{
-2\\
3
    }
\end{align}
\begin{align}
    \vec{B}=\myvec{
4\\
12
    }
\end{align}
\textbf{Result}
\begin{center}
 \includegraphics[scale=0.5]{conic.png}    
 \end{center}\vspace{1mm}
 From the figure,\\ \vspace{1mm}
Total area of portion is given by, \\ \vspace{1mm}
Total Area=(area enclosed by the line)-(area of parabola under the line )

\subsection*{Area Under the line}

\begin{align}
\implies A1=\int_{-2}^{4} \frac{3x+12}{2} \,dx
\end{align}
by solving the above equation we get area of triangle \textbf{$45 m^2$}
\subsection*{Area of enclosed ny the parabola under line}

\begin{align} 
\implies A2=\int_{-2}^{4}\frac{3x^2}{4} \,dx 
\end{align}
by solving the above equation we get area of parabola under the line \textbf{$18 m^2$}

the total area is

\vspace{3mm}
$\implies A=63 m^2 $\vspace{3mm}

The area enclosed by the parabola and line is ,
\begin{align}
\hspace{-4.25cm}A= 27 m^2
\end{align}
 \vspace{2mm} \textbf{Construction}
\begin{center}
\setlength{\arrayrulewidth}{0.5mm}
\setlength{\tabcolsep}{6pt}
\renewcommand{\arraystretch}{1.5}
    \begin{tabular}{|l|c|}
    \hline 
    \textbf{Points} & \textbf{coordinates} \\ \hline
   B & $\myvec{
   4\\
   12
   } $ \\\hline
   A & $\myvec{
   -2\\
   3
   } $ \\\hline
      \end{tabular}
  \end{center}
  \end{multicols}
 
Get the python code of the figures from

\begin{table}[h]
\large
\centering
\framebox{
\url{https://github.com/chandana531/cchandana_fwc/blob/main/conic_assignment/code/conic.py}}
\bibliographystyle{ieeetr}
\end{table} 
 
\end{document}
 
\item 
\label{chapters/12/8/3/8}
\iffalse
\documentclass[journal,10pt,twocolumn]{article}
\usepackage{graphicx}
\usepackage[margin=0.5in]{geometry}
\usepackage[cmex10]{amsmath}
\usepackage{array}
\usepackage{booktabs}
\usepackage{mathtools}
\title{\textbf{Conic section Assignment}}
\author{Jyothsna Paluchuri}
\date{September 2022}


\providecommand{\norm}[1]{\left\lVert#1\right\rVert}
\providecommand{\abs}[1]{\left\vert#1\right\vert}
\let\vec\mathbf
\newcommand{\myvec}[1]{\ensuremath{\begin{pmatrix}#1\end{pmatrix}}}
\newcommand{\mydet}[1]{\ensuremath{\begin{vmatrix}#1\end{vmatrix}}}
\providecommand{\brak}[1]{\ensuremath{\left(#1\right)}}
\providecommand{\lbrak}[1]{\ensuremath{\left(#1\right.}}
\providecommand{\rbrak}[1]{\ensuremath{\left.#1\right)}}
\providecommand{\sbrak}[1]{\ensuremath{{}\left[#1\right]}}

\begin{document}

\maketitle
\paragraph{\textit{Problem Statement} -
\fi
Find the area of the region bounded by the curve $x^2=4y$ and the lines y=2 and y=4 and the y-axis in the first quadrant.
\\
\solution
	\begin{figure}[!h]
		\centering
 \includegraphics[width=\columnwidth]{chapters/12/8/3/3/figs/conic.png}
		\caption{}
		\label{fig:12/8/3/3}
  	\end{figure}
\iffalse

\section*{\large Solution}

\begin{figure}[h]
\centering
\includegraphics[width=1\columnwidth]{conic.png}

\caption{The parabola formed by the curve $x^2 = 4y$ and the lines y=2 and y=4}
\label{fig:parabola}
\end{figure}

The given equation of parabola $x^2 = 4y$ can be written in the general quadratic form as
\begin{align}
    \label{eq:conic_quad_form}
    \vec{x}^{\top}\vec{V}\vec{x}+2\vec{u}^{\top}\vec{x}+f=0
    \end{align}
where
\fi
The conic parameters are
\begin{align}
	\vec{V} = \myvec{1 & 0\\0 & 0},
	\vec{u} = \myvec{0\\-2},
	f = 0
	%\\
\end{align}
\iffalse
The point of intersection of the lines y=2 and y=4 to the parabola is given by



The points of intersection of the line 
\begin{align}
	L: \quad \vec{x} = \vec{q} + \mu \vec{m} \quad \mu \in \mathbf{R}
\label{eq:conic_tangent}
\end{align}
with the conic section are given by
\begin{align}
\vec{x}_i = \vec{q} + \mu_i \vec{m}
\label{eq:conic_tangent_pts}
\end{align}
%
where
{\tiny
\begin{multline}
\mu_i = \frac{1}
{
\vec{m}^T\vec{V}\vec{m}
}
\lbrak{-\vec{m}^T\brak{\vec{V}\vec{q}+\vec{u}}}
\\
\pm
\rbrak{\sqrt{
\sbrak{
\vec{m}^T\brak{\vec{V}\vec{q}+\vec{u}}
}^2
-
\brak
{
\vec{q}^T\vec{V}\vec{q} + 2\vec{u}^T\vec{q} +f
}
\brak{\vec{m}^T\vec{V}\vec{m}}
}
}
\label{eq:tangent_roots}
\end{multline}
}


\fi
The vector parameters of 
$y-4=0$
are
\begin{align}
	\vec{h}_1=\myvec{0\\4},
	\vec{m}_1=\myvec{1\\0}
\end{align}
Substituting the above in \eqref{eq:tangent_roots},
\begin{align}
\mu_i=4,-4
\end{align}
yielding
the points of intersection with the parabola as
\begin{align}
\vec{a}_0=\myvec{4\\4},
\vec{a}_1=\myvec{-4\\4}
\end{align}
Similarly, for 
the line $y-2=0$, the vector parameters are
\begin{align}
\vec{h}_2=\myvec{0\\2},
\vec{m}_2=\myvec{1\\0}
\end{align}
yielding 
\begin{align}
\mu_i=2.8,-2.8
\end{align}
and the points of intersection
\begin{align}
\vec{a}_2=\myvec{2.8\\2},
\vec{a}_3=\myvec{-2.8\\2}
\end{align}
From Fig.
		\ref{fig:12/8/3/3},
the area of the parabola between the lines $y=2$ and $y=4$ is given by
\begin{align}
\int_{0}^{4} \ 2\sqrt{y} \,dy-\int_{0}^{2} \ 2\sqrt{y} \,dy
=6.895 
\end{align}
\iffalse


\section*{\large Construction}

{
\setlength\extrarowheight{5pt}
\begin{tabular}{|l|c|}
    \hline 
    \textbf{Points} & \textbf{intersection points} \\ \hline
	a0 & $\myvec{
   -2.8\\
   2
   } $ \\\hline
	a1 & $\myvec{
   2.8\\
   2
   } $ \\\hline
    
	a3 & $\myvec{
   -4\\
   4
   } $ \\\hline
	a2 & $\myvec{
   4\\
   4
   } $ \\\hline
      
      \end{tabular}
}

\end{document}
\fi

\item 
\label{chapters/12/8/3/9}
\iffalse
\documentclass[journal,10pt,twocolumn]{article}
\usepackage{graphicx}
\usepackage[margin=0.5in]{geometry}
\usepackage[cmex10]{amsmath}
\usepackage{array}
\usepackage{booktabs}
\usepackage{mathtools}
\title{\textbf{Conic section Assignment}}
\author{Jyothsna Paluchuri}
\date{September 2022}


\providecommand{\norm}[1]{\left\lVert#1\right\rVert}
\providecommand{\abs}[1]{\left\vert#1\right\vert}
\let\vec\mathbf
\newcommand{\myvec}[1]{\ensuremath{\begin{pmatrix}#1\end{pmatrix}}}
\newcommand{\mydet}[1]{\ensuremath{\begin{vmatrix}#1\end{vmatrix}}}
\providecommand{\brak}[1]{\ensuremath{\left(#1\right)}}
\providecommand{\lbrak}[1]{\ensuremath{\left(#1\right.}}
\providecommand{\rbrak}[1]{\ensuremath{\left.#1\right)}}
\providecommand{\sbrak}[1]{\ensuremath{{}\left[#1\right]}}

\begin{document}

\maketitle
\paragraph{\textit{Problem Statement} -
\fi
Find the area of the region bounded by the curve $x^2=4y$ and the lines y=2 and y=4 and the y-axis in the first quadrant.
\\
\solution
	\begin{figure}[!h]
		\centering
 \includegraphics[width=\columnwidth]{chapters/12/8/3/3/figs/conic.png}
		\caption{}
		\label{fig:12/8/3/3}
  	\end{figure}
\iffalse

\section*{\large Solution}

\begin{figure}[h]
\centering
\includegraphics[width=1\columnwidth]{conic.png}

\caption{The parabola formed by the curve $x^2 = 4y$ and the lines y=2 and y=4}
\label{fig:parabola}
\end{figure}

The given equation of parabola $x^2 = 4y$ can be written in the general quadratic form as
\begin{align}
    \label{eq:conic_quad_form}
    \vec{x}^{\top}\vec{V}\vec{x}+2\vec{u}^{\top}\vec{x}+f=0
    \end{align}
where
\fi
The conic parameters are
\begin{align}
	\vec{V} = \myvec{1 & 0\\0 & 0},
	\vec{u} = \myvec{0\\-2},
	f = 0
	%\\
\end{align}
\iffalse
The point of intersection of the lines y=2 and y=4 to the parabola is given by



The points of intersection of the line 
\begin{align}
	L: \quad \vec{x} = \vec{q} + \mu \vec{m} \quad \mu \in \mathbf{R}
\label{eq:conic_tangent}
\end{align}
with the conic section are given by
\begin{align}
\vec{x}_i = \vec{q} + \mu_i \vec{m}
\label{eq:conic_tangent_pts}
\end{align}
%
where
{\tiny
\begin{multline}
\mu_i = \frac{1}
{
\vec{m}^T\vec{V}\vec{m}
}
\lbrak{-\vec{m}^T\brak{\vec{V}\vec{q}+\vec{u}}}
\\
\pm
\rbrak{\sqrt{
\sbrak{
\vec{m}^T\brak{\vec{V}\vec{q}+\vec{u}}
}^2
-
\brak
{
\vec{q}^T\vec{V}\vec{q} + 2\vec{u}^T\vec{q} +f
}
\brak{\vec{m}^T\vec{V}\vec{m}}
}
}
\label{eq:tangent_roots}
\end{multline}
}


\fi
The vector parameters of 
$y-4=0$
are
\begin{align}
	\vec{h}_1=\myvec{0\\4},
	\vec{m}_1=\myvec{1\\0}
\end{align}
Substituting the above in \eqref{eq:tangent_roots},
\begin{align}
\mu_i=4,-4
\end{align}
yielding
the points of intersection with the parabola as
\begin{align}
\vec{a}_0=\myvec{4\\4},
\vec{a}_1=\myvec{-4\\4}
\end{align}
Similarly, for 
the line $y-2=0$, the vector parameters are
\begin{align}
\vec{h}_2=\myvec{0\\2},
\vec{m}_2=\myvec{1\\0}
\end{align}
yielding 
\begin{align}
\mu_i=2.8,-2.8
\end{align}
and the points of intersection
\begin{align}
\vec{a}_2=\myvec{2.8\\2},
\vec{a}_3=\myvec{-2.8\\2}
\end{align}
From Fig.
		\ref{fig:12/8/3/3},
the area of the parabola between the lines $y=2$ and $y=4$ is given by
\begin{align}
\int_{0}^{4} \ 2\sqrt{y} \,dy-\int_{0}^{2} \ 2\sqrt{y} \,dy
=6.895 
\end{align}
\iffalse


\section*{\large Construction}

{
\setlength\extrarowheight{5pt}
\begin{tabular}{|l|c|}
    \hline 
    \textbf{Points} & \textbf{intersection points} \\ \hline
	a0 & $\myvec{
   -2.8\\
   2
   } $ \\\hline
	a1 & $\myvec{
   2.8\\
   2
   } $ \\\hline
    
	a3 & $\myvec{
   -4\\
   4
   } $ \\\hline
	a2 & $\myvec{
   4\\
   4
   } $ \\\hline
      
      \end{tabular}
}

\end{document}
\fi

\item 
\label{chapters/12/8/3/10}
\iffalse
\documentclass[journal,10pt,twocolumn]{article}
\usepackage{graphicx}
\usepackage[margin=0.5in]{geometry}
\usepackage[cmex10]{amsmath}
\usepackage{array}
\usepackage{booktabs}
\usepackage{mathtools}
\title{\textbf{Conic section Assignment}}
\author{P.Revathi}
\date{October 2022}


\providecommand{\norm}[1]{\left\lVert#1\right\rVert}
\providecommand{\abs}[1]{\left\vert#1\right\vert}
\let\vec\mathbf
\newcommand{\myvec}[1]{\ensuremath{\begin{pmatrix}#1\end{pmatrix}}}
\newcommand{\mydet}[1]{\ensuremath{\begin{vmatrix}#1\end{vmatrix}}}
\providecommand{\brak}[1]{\ensuremath{\left(#1\right)}}
\providecommand{\lbrak}[1]{\ensuremath{\left(#1\right.}}
\providecommand{\rbrak}[1]{\ensuremath{\left.#1\right)}}
\providecommand{\sbrak}[1]{\ensuremath{{}\left[#1\right]}}

\begin{document}

\maketitle
\paragraph{\textit{Problem Statement} -
\fi
Find the area of the region bounded by the curve $x^2=y$ and the lines $y=x+2$ and the $x$ axis.
\\
\solution 
	\begin{figure}[!h]
		\centering
 \includegraphics[width=\columnwidth]{chapters/12/8/3/10/figs/conics1.png}
		\caption{}
		\label{fig:12/8/3/10}
  	\end{figure}
\iffalse

\section*{\large Solution}

\begin{figure}[h]
\centering
\includegraphics[width=1\columnwidth]{conics1.png}

%\caption{The parabola formed by the curve $y^2 = 9x$ and the lines x=2 and x=4}
\label{fig:parabola}
\end{figure}

The given equation of parabola $x^2 = y$ can be written in the general quadratic form as
\begin{align}
    \label{eq:conic_quad_form}
    \vec{x}^{\top}\vec{V}\vec{x}+2\vec{u}^{\top}\vec{x}+f=0
    \end{align}
where
The parameters of the given conic are
\begin{align}
 \vec{V} = \myvec{1 & 0\\0 & 0},
 \vec{u} = \myvec{0\\-0.5},
 f = 0
\end{align}
The points of intersection of the line 
\begin{align}
 L: \quad \vec{x} = \vec{q} + \mu \vec{m} \quad \mu \in \mathbf{R}
\label{eq:conic_tangent}
\end{align}
with the conic section are given by
\begin{align}
\vec{x}_i = \vec{q} + \mu_i \vec{m}
\label{eq:conic_tangent_pts}
\end{align}
%
where
{\tiny
\begin{multline}
\mu_i = \frac{1}
{
\vec{m}^T\vec{V}\vec{m}
}
\lbrak{-\vec{m}^T\brak{\vec{V}\vec{q}+\vec{u}}}
\\
\pm
\rbrak{\sqrt{
\sbrak{
\vec{m}^T\brak{\vec{V}\vec{q}+\vec{u}}
}^2
-
\brak
{
\vec{q}^T\vec{V}\vec{q} + 2\vec{u}^T\vec{q} +f
}
\brak{\vec{m}^T\vec{V}\vec{m}}
}
}
\label{eq:tangent_roots}
\end{multline}
}
The parameters of the line $y=x+2$ are
\begin{align}
\vec{h}=\myvec{0\\2},
\vec{m}=\myvec{1\\1}
\end{align}
yielding
\begin{align}
\mu_i=-2
\end{align}
upon substituting in \eqref{eq:tangent_roots}.     The points of intersection of this line with the conic are
\begin{align}
\vec{a_0}=\myvec{2\\4},
\vec{a_1}=\myvec{-1\\1}
\end{align}
Similarly, 
Given line equation y=x+2\\

$$x-y=-2$$\\
$$\vec{n}^{t}\vec{x}=c$$\\
$$\vec{x}=\vec{A}+\lambda \vec{m}$$\\
%\end{centre}


$$\vec{x}=\begin{pmatrix}
-2\\ 
0
\end{pmatrix}+\mu \begin{pmatrix}
1\\ 
1
\end{pmatrix}$$\\

Substitute the x value in the quadratic equation then we get a quadratic equations
\begin{align}
    \label{eq:conic_quad_form}
    \vec{x}^{\top}\vec{V}\vec{x}+2\vec{u}^{\top}\vec{x}+f=0
    \end{align}
    
$$\mu ^{2}-3\mu +2=0\\$$
$$\mu =1,2\\$$
$$\mu ^{2}-\mu\\$$
$$\mu =1,0\\$$
The resultant x values are\\
$$\vec{x}=\begin{pmatrix}
-2\\ 
0
\end{pmatrix}$$\\
$$\vec{x}=\begin{pmatrix}
-1\\ 
1
\end{pmatrix}$$\\
$$\vec{x}=\begin{pmatrix}
0\\ 
2
\end{pmatrix}$$

Area of the parabola in between the lines parabola and y=x+2 is given by
\begin{align}
\implies A_1=\int_{-2}^{-1} \ x+2 \,dx
\end{align}

\begin{align}
\implies A_2=\int_{-1}^{0} \ x^2 \,dx
\end{align}
\begin{align}
\implies A_1+A_1=\int_{-2}^{-1} \ x+2 \,dx+\int_{-1}^{0} \ x^2 \,dx
\end{align}
\begin{align}
\implies A_1+ A_2=\frac{5}{6}sq units
\end{align}

\end{document}
\fi

\item 
\label{chapters/12/8/3/17}

\def\mytitle{MATRICES USING PYTHON(CONIC)}
\def\myauthor{R.Radhika}
\def\contact{r170234@rguktrkv.ac.in}
\def\mymodule{Future Wireless Communication (FWC)}
\documentclass[10pt, a4paper]{article}
\usepackage[a4paper,outer=1.5cm,inner=1.5cm,top=1.75cm,bottom=1.5cm]{geometry}
\twocolumn
\usepackage{graphicx}
\graphicspath{{./images/}}
\usepackage[colorlinks,linkcolor={black},citecolor={blue!80!black},urlcolor={blue!80!black}]{hyperref}
\usepackage[parfill]{parskip}
\usepackage{lmodern}
\usepackage{tikz}
	\usepackage{physics}
%\documentclass[tikz, border=2mm]{standalone}
%\usepackage{karnaugh-map}
%\documentclass{article}
\usepackage{tabularx}
%\usepackage{circuitikz}
\usepackage{enumitem}
\usetikzlibrary{calc}
\usepackage{amsmath}
\usepackage{amssymb}
\renewcommand*\familydefault{\sfdefault}
\usepackage{watermark}
\usepackage{lipsum}
\usepackage{xcolor}
\usepackage{listings}
\usepackage{float}
\usepackage{titlesec}
\providecommand{\mtx}[1]{\mathbf{#1}}
\titlespacing{\subsection}{1pt}{\parskip}{3pt}
\titlespacing{\subsubsection}{0pt}{\parskip}{-\parskip}
\titlespacing{\paragraph}{0pt}{\parskip}{\parskip}
\newcommand{\figuremacro}[5]{
    \begin{figure}[#1]
        \centering
        \includegraphics[width=#5\columnwidth]{#2}
        \caption[#3]{\textbf{#3}#4}
        \label{fig:#2}
    \end{figure}
}

\newcommand{\myvec}[1]{\ensuremath{\begin{pmatrix}#1\end{pmatrix}}}
\let\vec\mathbf
\lstset{
frame=single, 
breaklines=true,
columns=fullflexible
}

\title{\mytitle}
\author{\myauthor\hspace{1em}\\\contact\\FWC22066\hspace{6.5em}IITH\hspace{0.5em}\mymodule\hspace{6em}Assignment}
\begin{document}
	\maketitle
	\tableofcontents
   \section{Problem}
   The area bounded by the curve y=$x|x|$ x-axis and the ordinates $x$=-1 and $x$=1,is given by
[Hint: y=$x^2$ if $x>0$  and y=$-x^2$ if $x<0$]
   					
\section{Construction}
  \includegraphics[scale=0.47]{conicfig.pdf}
  	\begin{center}
  Figure of construction
  	\end{center}
  \section{Solution}
  
\raggedright\large{ Draw the ordinates by using $x$=1 and $x$=-1. Then we need to draw  two parabolas using  given hint [Hint: y=$x^2$ if $x>0$  and y=$-x^2$ if $x<0$]for that we need to find out the area bounded by the curve  y=$x|x|$  .}
\vspace{2mm}\\
\raggedright\large{Then the limits from -1 to 1  and the points(-1,-1),(1,1)}\vspace{2mm}\\
The standard conic equation\\
\begin{align}
\vec{x}^{\top}\vec{V}\vec{x}+2\vec{u}^{\top}\vec{x}+f=0
\end{align}
\begin{align}
\vec{x}^{\top}\vec{V}\vec{x}+2\vec{u_1}^{\top}\vec{x}+f_1=0
\end{align}
\begin{align}
\vec{V}=\myvec{1&0\\0&0} ,\vec{u_1}=\myvec{0\\-\frac{1}{2}},  f_1=0
\end{align}
\begin{align}
\vec{x}^{\top}\vec{V}\vec{x}+2\vec{u_2}^{\top}\vec{x}+f_2=0
\end{align}
\begin{align}
\vec{V}=\myvec{-1&0\\0&0} ,\vec{u_2}=\myvec{0\\-\frac{1}{2}},  f_2=0
\end{align}
The intersection of two conics\\
\begin{align}
|\myvec{\vec{V_1}+\mu{\vec{V_2}}&\vec{u_1}+\mu{\vec{u_2}}\\\vec{u_1}+\mu{\vec{u_2}}&0}|
\end{align}
substitute eq 3 and 4 in eq 6\\
\begin{align}
\myvec{1-\mu&0&0\\0&0&-\frac{1}{2}-\frac{\mu}{2}\\0&-\frac{1}{2}-\frac{\mu}{2}&0}
\end{align}
by solving eq-7 \\
yielding,
\begin{align}
\mu^3+\mu^2-\mu-1=0
\end{align}
After solving eq-8\\
we get\\
$\mu=-1,1,1$
\begin{align}
|\vec{V_1}+\mu\vec{V_2}|<0
\end{align}
substitute $\vec{V_1}$ and $\vec{V_2}$ in eq-9\\
we get 0\\
\begin{align}
\vec{x}=\vec{q}+\mu{\vec{m}}
\end{align}
\begin{align}
q=\vec{V^{-1}}(k\vec{n}-\vec{u})
\end{align}
\begin{align}
k=\pm\sqrt{\frac{\norm{\vec{u_2}}^2\vec{V}-f}{\vec{n}^{\top}\vec{V^{-1}}\vec{n}}}
\end{align}
$\vec{n}=\myvec{1\\-1}$\\
$\vec{m}=\myvec{1\\1}$\\

by solving eq 10 and 11 we get\\
\begin{align}
\vec{q}=\myvec{0\\0}
\end{align}


 Given equation :  y=$x|x|$\\

We know that \\
\begin{equation}
   |x| =
    \begin{cases}
      x, & {x\geq0}\\
      -x & {x<0}\\
    \end{cases}       
\end{equation}

	Therefore,
\begin{equation}
   y=x|x| =
    \begin{cases}
      xx, & {x\geq0}\\
      x(-x) & {x<0}\\
    \end{cases}       
\end{equation}
\begin{equation}
   y =
    \begin{cases}
      x^2, & {x\geq0}\\
      -x^2 & {x<0}\\
    \end{cases}       
\end{equation}
Area Required=Area ABO+Area DCO\\
\textbf{Area of DCO}

Area  : \[ \int_{-1}^{1} y \,dx \]

Here, y=$x|x|$

Therefore Area DCO: \[ \int_{-1}^{0} -x^2 \,dx \]

 yielding ,\\
 
   -1/3 \\
 
 $|(-1/3)|$=1/3\\
 
 Area of DCO= 1/3

\textbf{Area  of ABO}: \[ \int_{0}^{1} x^2 \,dx \]

    yielding 1/3\\
   
     
     Area of ABO= 1/3
     

\textbf{Required Area=ABO+DCO}:
  1/3+1/3=2/3
Below python code realizes the above construction 

\begin{table}[h!]
    \begin{tabular}{|c|}
    \hline
         https://github.com/Radhikarkv/fwcproject.git\\
	\hline
    \end{tabular}
\end{table}
\end{document}

\item 
\label{chapters/12/8/2/3}
\iffalse
\documentclass[journal,10pt,twocolumn]{article}
\usepackage{graphicx}
\usepackage[margin=0.5in]{geometry}
\usepackage[cmex10]{amsmath}
\usepackage{array}
\usepackage{booktabs}
\usepackage{mathtools}
\title{\textbf{Conic section Assignment}}
\author{Jyothsna Paluchuri}
\date{September 2022}


\providecommand{\norm}[1]{\left\lVert#1\right\rVert}
\providecommand{\abs}[1]{\left\vert#1\right\vert}
\let\vec\mathbf
\newcommand{\myvec}[1]{\ensuremath{\begin{pmatrix}#1\end{pmatrix}}}
\newcommand{\mydet}[1]{\ensuremath{\begin{vmatrix}#1\end{vmatrix}}}
\providecommand{\brak}[1]{\ensuremath{\left(#1\right)}}
\providecommand{\lbrak}[1]{\ensuremath{\left(#1\right.}}
\providecommand{\rbrak}[1]{\ensuremath{\left.#1\right)}}
\providecommand{\sbrak}[1]{\ensuremath{{}\left[#1\right]}}

\begin{document}

\maketitle
\paragraph{\textit{Problem Statement} -
\fi
Find the area of the region bounded by the curve $x^2=4y$ and the lines y=2 and y=4 and the y-axis in the first quadrant.
\\
\solution
	\begin{figure}[!h]
		\centering
 \includegraphics[width=\columnwidth]{chapters/12/8/3/3/figs/conic.png}
		\caption{}
		\label{fig:12/8/3/3}
  	\end{figure}
\iffalse

\section*{\large Solution}

\begin{figure}[h]
\centering
\includegraphics[width=1\columnwidth]{conic.png}

\caption{The parabola formed by the curve $x^2 = 4y$ and the lines y=2 and y=4}
\label{fig:parabola}
\end{figure}

The given equation of parabola $x^2 = 4y$ can be written in the general quadratic form as
\begin{align}
    \label{eq:conic_quad_form}
    \vec{x}^{\top}\vec{V}\vec{x}+2\vec{u}^{\top}\vec{x}+f=0
    \end{align}
where
\fi
The conic parameters are
\begin{align}
	\vec{V} = \myvec{1 & 0\\0 & 0},
	\vec{u} = \myvec{0\\-2},
	f = 0
	%\\
\end{align}
\iffalse
The point of intersection of the lines y=2 and y=4 to the parabola is given by



The points of intersection of the line 
\begin{align}
	L: \quad \vec{x} = \vec{q} + \mu \vec{m} \quad \mu \in \mathbf{R}
\label{eq:conic_tangent}
\end{align}
with the conic section are given by
\begin{align}
\vec{x}_i = \vec{q} + \mu_i \vec{m}
\label{eq:conic_tangent_pts}
\end{align}
%
where
{\tiny
\begin{multline}
\mu_i = \frac{1}
{
\vec{m}^T\vec{V}\vec{m}
}
\lbrak{-\vec{m}^T\brak{\vec{V}\vec{q}+\vec{u}}}
\\
\pm
\rbrak{\sqrt{
\sbrak{
\vec{m}^T\brak{\vec{V}\vec{q}+\vec{u}}
}^2
-
\brak
{
\vec{q}^T\vec{V}\vec{q} + 2\vec{u}^T\vec{q} +f
}
\brak{\vec{m}^T\vec{V}\vec{m}}
}
}
\label{eq:tangent_roots}
\end{multline}
}


\fi
The vector parameters of 
$y-4=0$
are
\begin{align}
	\vec{h}_1=\myvec{0\\4},
	\vec{m}_1=\myvec{1\\0}
\end{align}
Substituting the above in \eqref{eq:tangent_roots},
\begin{align}
\mu_i=4,-4
\end{align}
yielding
the points of intersection with the parabola as
\begin{align}
\vec{a}_0=\myvec{4\\4},
\vec{a}_1=\myvec{-4\\4}
\end{align}
Similarly, for 
the line $y-2=0$, the vector parameters are
\begin{align}
\vec{h}_2=\myvec{0\\2},
\vec{m}_2=\myvec{1\\0}
\end{align}
yielding 
\begin{align}
\mu_i=2.8,-2.8
\end{align}
and the points of intersection
\begin{align}
\vec{a}_2=\myvec{2.8\\2},
\vec{a}_3=\myvec{-2.8\\2}
\end{align}
From Fig.
		\ref{fig:12/8/3/3},
the area of the parabola between the lines $y=2$ and $y=4$ is given by
\begin{align}
\int_{0}^{4} \ 2\sqrt{y} \,dy-\int_{0}^{2} \ 2\sqrt{y} \,dy
=6.895 
\end{align}
\iffalse


\section*{\large Construction}

{
\setlength\extrarowheight{5pt}
\begin{tabular}{|l|c|}
    \hline 
    \textbf{Points} & \textbf{intersection points} \\ \hline
	a0 & $\myvec{
   -2.8\\
   2
   } $ \\\hline
	a1 & $\myvec{
   2.8\\
   2
   } $ \\\hline
    
	a3 & $\myvec{
   -4\\
   4
   } $ \\\hline
	a2 & $\myvec{
   4\\
   4
   } $ \\\hline
      
      \end{tabular}
}

\end{document}
\fi

\item 
\label{chapters/12/8/2/6}
\documentclass[journal,12pt,twocolumn]{IEEEtran}
\usepackage{graphicx}
\usepackage{listings}
\usepackage[utf8]{inputenc}
\usepackage{caption}
\usepackage{hyperref}
\usepackage[cmex10]{amsmath}
\usepackage{array}
\usepackage{gensymb}
\usepackage{booktabs}
\usepackage{etoolbox}
\usepackage{amssymb}
\patchcmd{\section}{\centering}{}{}{}
\providecommand{\norm}[1]{\left\lVert#1\right\rVert}
\providecommand{\abs}[1]{\left\vert#1\right\vert}
\let\vec\mathbf

\makeatletter
\newcommand\xleftrightarrow[2][]{%
  \ext@arrow 9999{\longleftrightarrowfill@}{#1}{#2}}
\newcommand\longleftrightarrowfill@{%
  \arrowfill@\leftarrow\relbar\rightarrow}
\makeatother
\title{Matrix Problems \textbf{\\Conics }}
\author{Manoj Chavva} 
\newcommand{\myvec}[1]{\ensuremath{\begin{pmatrix}#1\end{pmatrix}}}
\newcommand{\mydet}[1]{\ensuremath{\begin{vmatrix}#1\end{vmatrix}}}
\providecommand{\brak}[1]{\ensuremath{\left(#1\right)}}
\providecommand{\lbrak}[1]{\ensuremath{\left(#1\right.}}
\providecommand{\rbrak}[1]{\ensuremath{\left.#1\right)}}
\providecommand{\sbrak}[1]{\ensuremath{{}\left[#1\right]}}

\begin{document}
\maketitle
\section{Problem Statement}

\noindent Smaller area enclosed by the circle $x^2 + y^2 = 4$ and the line $x + y = 2$. 
\begin{enumerate}
\item $2(\pi -2)$
\item $\pi -2$
\item $2\pi -1$
\item $2(\pi +2)$
\end{enumerate}


\begin{figure}[h]
\includegraphics[width=1\columnwidth]{./figs/conic.png}
\caption{Smaller region between Circle and Line}
\label{fig:conic}
\end{figure}

\raggedright \textbf{Given}: \\
Equation of circle is  
\begin{equation} x^2 + y^2 = 4
\end{equation}
Equation of line is 
\begin{equation}
x+y=2
\end{equation}
\textbf{To Find:} \\
To find the intersection points and area of shaded region shown in figure\
\section{Construction}

\begin{table}[h!]
\begin{center}
\setlength{\arrayrulewidth}{0.5mm}
\renewcommand{\arraystretch}{1.5}
    \begin{tabular}{|l|c|}
    \hline 
    \textbf{Points} & \textbf{coordinates} \\ \hline
   $\vec{A}$ & $\myvec{
   0\\
   2
   } $ \\\hline
   $\vec{B}$ & $\myvec{
   2\\
   0
   } $ \\\hline
      \end{tabular}
  \end{center}
\end{table}
\newpage
\section{solution}
The given circle can be expressed as conics with parameters,
\begin{equation}
\vec{V}=\myvec{
4 & 0\\
0 & 4
}
\end{equation}
\begin{equation}
\vec{u}=0 
\end{equation}
\begin{equation}
f=-16
\end{equation}

The given line equation can be written as\\ 
\begin{align} 
	\vec{x}=\begin{pmatrix}2 \\ 0 \\ \end{pmatrix}+k\begin{pmatrix}\frac{1}{2} \\ -\frac{1}{2} \\ \end{pmatrix}
\end{align}
The points of intersection of the line, \\ 
\begin{equation}
L: \quad \vec{x} = \vec{q} + \kappa \vec{m} \quad \kappa \in \mathbb{R}
\end{equation}

with the conic section, \\ 
\begin{align}
	\vec{x}^{\top}\vec{V}\vec{x} + 2\vec{u}^{\top} \vec{x} + f = 0
\end{align}
are given by \\
\begin{align}
\vec{x}_i = \vec{q} + \kappa_i \vec{m}
\end{align}
where, \\

\begin{equation*}
\kappa_i = \frac{1}
{
\vec{m}^T\vec{V}\vec{m}
}
\lbrak{-\vec{m}^T\brak{\vec{V}\vec{q}+\vec{u}}}
\pm
\end{equation*}
\begin{equation}
\rbrak{\sqrt{
\sbrak{
\vec{m}^T\brak{\vec{V}\vec{q}+\vec{u}}
}^2
-
\brak
{
\vec{q}^T\vec{V}\vec{q} + 2\vec{u}^T\vec{q} +f
}
\brak{\vec{m}^T\vec{V}\vec{m}}
}
}
\end{equation}
On substituting\\
\begin{align}
\vec{q} &= \myvec{
2\\
0
} 
\end{align}
\begin{align}
\vec{m} = \myvec{\frac{1}{2} \\ -\frac{1}{2}}
\end{align}
With the given as in eq(3),(4),(5),\\ 

The value of $\kappa$ ,\\
\begin{equation}
\kappa =0,-4
\end{equation}
    
By substituting eq(13) in eq(6) we get the
points of intersection of line with circle \\
\begin{align}
    \vec{A}=\myvec{
0\\
2
    }
\end{align}
\begin{align}
    \vec{B}=\myvec{
2\\
0
    }
\end{align}
From the figure \\
Total area of portion is given by,\\ 
Total Area=(area of circle in first quadrant)-(area of a triangle \textbf{AOB})

\subsection*{Area of triangle}

\begin{align}
\implies A_1=\int_{0}^{2} (2-x) \,dx
\end{align}
By solving the above equation we get area of triangle as 2 units
\subsection*{Area of circle}

\begin{align} 
\implies A_2=\int_{0}^{2}\sqrt{4-x^2} \,dx 
\end{align}
By solving the above equation we get area of circle $\pi$

The total area is
$\implies \vec{A}=\pi - 2$


\begin{table}[h]
\large
\begin{tabular}{lll}
\multicolumn{3}{l}{Get Python Code for image from}                                                 \\ \hline
\multicolumn{3}{|l|}{\url{https://github.com/ManojChavva/FWC/blob/main/Matrix/conics/code/conic.py}} \\ 
 \hline
\multicolumn{3}{l}{Get LaTex code from}                                                            \\ \hline
\multicolumn{3}{|l|}{\url{https://github.com/ManojChavva/FWC/blob/main/Matrix/conics/conic.tex}}            \\ \hline
\end{tabular}
\end{table}



\end{document}





\end{enumerate}
