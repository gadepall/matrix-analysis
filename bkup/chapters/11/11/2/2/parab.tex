\iffalse
\documentclass[12pt]{article}
\usepackage{graphicx}
\usepackage{amsmath}
\usepackage{mathtools}
\usepackage{gensymb}

\newcommand{\mydet}[1]{\ensuremath{\begin{vmatrix}#1\end{vmatrix}}}
\providecommand{\brak}[1]{\ensuremath{\left(#1\right)}}
\providecommand{\norm}[1]{\left\lVert#1\right\rVert}
\newcommand{\solution}{\noindent \textbf{Solution: }}
\newcommand{\myvec}[1]{\ensuremath{\begin{pmatrix}#1\end{pmatrix}}}
\let\vec\mathbf

\begin{document}
\begin{center}
\textbf\large{CONIC SECTIONS}

\end{center}
\section*{Excercise 11.2}

Q2.Find the coordinates of the focus, axis of the parabola, the equation of the directrix and the length of the latus rectum of a parabola whose equation is given by $x^2=6y$.

\solution
\fi
The given equation of the parabola can be rearranged as
\begin{align}
	\label{eq:chapters/11/11/2/2/parabolaEq1}
	x^2-6y=0
\end{align}
The above equation can be equated to the generic equation of conic sections
\begin{align}
	\label{eq:chapters/11/11/2/2/parabolaEq2}
	g\brak{\vec{x}}=\vec{x}^\top \vec{V}\vec{x}+2\vec{u}^\top \vec{x}+f=0
\end{align}
Comparing the coefficients of both equations \eqref{eq:chapters/11/11/2/2/parabolaEq1} and \eqref{eq:chapters/11/11/2/2/parabolaEq2}
\begin{align}
	\label{eq:chapters/11/11/2/2/eqV}
	\vec{V} &= \myvec{1&0\\0&0}\\
	\label{eq:chapters/11/11/2/2/eqU}
	\vec{u} &= -\myvec{0\\3}\\
	\label{eq:chapters/11/11/2/2/eqF}
	f &= 0
\end{align}
\begin{enumerate}
\item From equation \eqref{eq:chapters/11/11/2/2/eqV}, since $\vec{V}$ is already diagonalized, the Eigen values $\lambda_1 \text{ and } \lambda_2$ are given as
\begin{align}
	\label{eq:chapters/11/11/2/2/eqEigen1}
	\lambda_1 &= 1\\
	\label{eq:chapters/11/11/2/2/eqEigen2}
	\lambda_2 &= 0
\end{align}
And the corresponding eigen vector matrix $\vec{P}$ is indentity, so the Eigen vector $\vec{p}_2$ corresponding to Eigen value $\lambda_2$ is
\begin{align}
	\vec{p}_2 &= \myvec{0\\1}\\
	\vec{n} &= \sqrt{\lambda_1}\vec{p}_2\\
		&= \sqrt{1}\myvec{0\\1}\\
		&= \myvec{0\\1}
\end{align}
Now,
\begin{align}
	\label{eq:chapters/11/11/2/2/eqC}
	c = \frac{\norm{\vec{u}}^2 - \lambda_1 f}{2\vec{u}^\top \vec{n}}
\end{align}
Substituting values of $\vec{u},\vec{n},\lambda_1 \text{ and } f$ in \eqref{eq:chapters/11/11/2/2/eqC}
\begin{align}
	c = \frac{3^2-1\brak{0}}{-2\myvec{0&3}\myvec{0\\1}} = -\frac{3}{2}
\end{align}
The focus $\vec{F}$ of parabola is expressed as
\begin{align}
	\vec{F} &= \frac{ce^2 \vec{n}-\vec{u}}{\lambda_1}\\
		&= \frac{-\frac{3}{2}\brak{1}^2 \myvec{0\\1}+\myvec{0\\3}}{1}\\
		&= \myvec{0\\\frac{3}{2}}
\end{align}
\item Equation of directrix is given as
\begin{align}
	\vec{n}^\top \vec{x} &= c\\
	\myvec{0&1}\vec{x} &= -\frac{3}{2}
\end{align}
\item The equation for the axis of parabola passing through $\vec{F}$ and orthogonal to the directrix is given as
\begin{align}
	\label{eq:chapters/11/11/2/2/eqM}
	\vec{m}^\top \brak{\vec{x}-\vec{F}} = 0
\end{align}
where $\vec{m}$ is the normal vector to the axis and also the slope of the directrix. Now since
\begin{align}
	\vec{n} = \myvec{0\\1}\\
	\vec{m} = \myvec{1\\0}
\end{align}
Substituting in \eqref{eq:chapters/11/11/2/2/eqM}
\begin{align}
	\myvec{1&0}\myvec{\vec{x}-\myvec{0\\\frac{3}{2}}}&=0\\
	\myvec{1&0}\vec{x} &= 0
\end{align}
\item The latus rectum of a parabola is given by
\begin{align}
	l&=\frac{\eta}{\lambda_1}\\
	 &=\frac{2\vec{u}^\top \vec{p}_2}{\lambda_1}\\
	 &=\frac{2\myvec{0&3}\myvec{0\\1}}{1}\\
	 &=6 \text{ units }
\end{align}
See Fig. \ref{fig:chapters/11/11/2/2/Fig1}
\begin{figure}[!h]
	\begin{center} 
	    \includegraphics[width=\columnwidth]{chapters/11/11/2/2/figs/parabola}
	\end{center}
\caption{}
\label{fig:chapters/11/11/2/2/Fig1}
\end{figure}
\end{enumerate}




