\iffalse
\documentclass[12pt]{article}
\usepackage{graphicx}
%\documentclass[journal,12pt,twocolumn]{IEEEtran}
\usepackage[none]{hyphenat}
\usepackage{graphicx}
\usepackage{listings}
\usepackage[english]{babel}
\usepackage{graphicx}
\usepackage{caption}
\usepackage[parfill]{parskip}
\usepackage{hyperref}
\usepackage{gensymb}
\usepackage{booktabs}
%\usepackage{setspace}\doublespacing\pagestyle{plain}
\def\inputGnumericTable{}
\usepackage{color}                                            %%
    \usepackage{array}                                            %%
    \usepackage{longtable}                                        %%
    \usepackage{calc}                                             %%
    \usepackage{multirow}                                         %%
    \usepackage{hhline}                                           %%
    \usepackage{ifthen}
\usepackage{array}
\usepackage{amsmath}   % for having text in math mode
\usepackage{parallel,enumitem}
\usepackage{listings}
\lstset{
language=tex,
frame=single,
breaklines=true
}
 
%Following 2 lines were added to remove the blank page at the beginning
\usepackage{atbegshi}% http://ctan.org/pkg/atbegshi
\AtBeginDocument{\AtBeginShipoutNext{\AtBeginShipoutDiscard}}
%
%New macro definitions
\newcommand{\mydet}[1]{\ensuremath{\begin{vmatrix}#1\end{vmatrix}}}
\providecommand{\brak}[1]{\ensuremath{\left(#1\right)}}
\providecommand{\norm}[1]{\left\lVert#1\right\rVert}
\newcommand{\solution}{\noindent \textbf{Solution: }}
\newcommand{\myvec}[1]{\ensuremath{\begin{pmatrix}#1\end{pmatrix}}}
\providecommand{\abs}[1]{\left\vert#1\right\vert}
\let\vec\mathbf
\begin{document}
\begin{center}
\enlargethispage{-4cm}
\title{\textbf{Straight Lines}}
\date{\vspace{-5ex}} %Not to print date automatically
\maketitle
\end{center}
\setcounter{page}{1}
\section*{11$^{th}$ Maths - Chapter 10}
This is Problem-8 from Exercise 2
\begin{enumerate}

\solution The equation of a line is given by
		\begin{align}                                                                                          \vec{n}^\top\vec{x}=c \label{eq:chapters/11/10/2/8/1}                                                                                  \end{align}
			\fi
			From the given information, 
the normal vector of the line is
\begin{align}
	\vec{n}=\myvec{\cos{30}\degree\\\sin{30}\degree}
\end{align}
The distance from the origin to the line is given by
\begin{align}
	d=\frac{\abs{c}}{\norm{\vec{n}}}
	\implies c = \pm d = \pm 5
\label{eq:chapters/11/10/2/8/3}
\end{align}
since
	\begin{align}
		\norm{\vec{n}} =1,
	\end{align}
		Thus, the equation of lines are
\begin{align}
	\myvec{\frac{\sqrt{3}}{2}& \frac{1}{2}}\vec{x}=\pm5
\end{align}
See Fig. 
\ref{fig:chapters/11/10/2/8/Fig1}.
\begin{figure}[!h]
\begin{center}
\includegraphics[width=\columnwidth]{chapters/11/10/2/8/figs/fig.pdf}
\end{center}
\caption{}
\label{fig:chapters/11/10/2/8/Fig1}
\end{figure}
