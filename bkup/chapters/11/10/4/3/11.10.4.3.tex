\iffalse
\documentclass[10pt]{article}
\usepackage[latin1]{inputenc}
\usepackage{fullpage}
\usepackage{color}
\usepackage{array}
\usepackage{longtable}
\usepackage{calc}
\usepackage{multirow}
\usepackage{hhline}
\usepackage{ifthen}
\usepackage{graphicx}
\def\inputGnumericTable{}
\usepackage[none]{hyphenat}
\usepackage{graphicx}
\usepackage{listings}
\usepackage[english]{babel}
\usepackage{graphicx}
\usepackage{caption} 
\usepackage{booktabs}
\usepackage{gensymb}
\usepackage{array}
\usepackage{amssymb} % for \because
\usepackage{amsmath}   % for having text in math mode
\usepackage{extarrows} % for Row operations arrows
\usepackage{listings}
\lstset{
  frame=single,
  breaklines=true
}
\usepackage{hyperref}
%Following 2 lines were added to remove the blank page at the beginning
\usepackage{atbegshi}% http://ctan.org/pkg/atbegshi
\AtBeginDocument{\AtBeginShipoutNext{\AtBeginShipoutDiscard}}
%New macro definitions
\newcommand{\mydet}[1]{\ensuremath{\begin{vmatrix}#1\end{vmatrix}}}
\providecommand{\brak}[1]{\ensuremath{\left(#1\right)}}
\providecommand{\norm}[1]{\left\lVert#1\right\rVert}
\newcommand{\solution}{\noindent \textbf{Solution: }}
\newcommand{\myvec}[1]{\ensuremath{\begin{pmatrix}#1\end{pmatrix}}}
\providecommand{\abs}[1]{\left\vert#1\right\vert}
\let\vec\mathbf
\begin{document}

\begin{center}
\title{\textbf{LINES}}
\date{\vspace{-5ex}} %Not to print date automatically
\maketitle
\end{center}

\section*{11$^{th}$ Maths - EXERCISE-10.4}


\solution
Let the intercepts of $x$ and $y$ are $a$ and $b$\\

\begin{table}[!h]
\centering
\begin{tabular}{|c|c|p{5cm}|}
\hline
\textbf{Symbol} & \textbf{Value} & \textbf{Description} \\
\hline
$\theta$ & $30\degree$ & $\angle{BAP} = \angle{BAQ}$ \\
\hline
$a$ & $9$ & $AB$ \\
\hline
$c$ & $8$ & $AQ$ \\
\hline
$\vec{e}_1$ & $\myvec{1\\0}$ & Basis vector \\
\hline
\end{tabular}

\caption{}
\label{Inputs}
\end{table}
Given
\begin{align}
a+b&=1
\label{eq1}\\
ab&=-6
\label{eq2}
\end{align} 
on solving \eqref{eq1} and \eqref{eq2} we get\\

\begin{align}
\implies a=3,b=-2 \text{ or }   a=-2,b=3
\end{align}
Thus,the possible intercepts are\\
\begin{align}
\myvec{3\\ 0 \\} ,\myvec{0\\ -2 \\} ,\myvec{-2\\ 0 \\}, \myvec{0\\ 3 \\}
\end{align}
\begin{align}
\vec{m}&=\vec{a}-\vec{b}\\
&=\myvec{3\\0}-\myvec{0\\-2}\\
&=\myvec{3\\2}\\
&=\myvec{3\\2} \text{ or } \myvec{-2\\-3}\\
\end{align}
\begin{enumerate}
\item For
\begin{align}
\vec{n} = \myvec{ 2\\-3}\\
\end{align}
the equation of the line will be\\
\begin{align}
\vec{n}^{\top}\brak{\vec{x}-\vec{A}}=0\\
\myvec{2&-3}\vec{x}=-6
\end{align}
\item For
\begin{align}
\vec{n}=\myvec{-3\\2}\\
\end{align}
the equation of the line will be\\
\begin{align}
\vec{n}^{\top}\brak{\vec{x}-\vec{B}}=0\\
\myvec{-3&2}\vec{x}=-6
\end{align}
\end{enumerate}
\begin{figure}[!h]
	\begin{center}
		\includegraphics[width=\columnwidth]{./figs/fig.pdf}
	\end{center}
\caption{}
\label{figure}
\end{figure}


