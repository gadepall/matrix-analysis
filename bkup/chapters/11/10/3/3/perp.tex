\iffalse
\documentclass[12pt]{article}
\usepackage{graphicx}
\usepackage{amsmath}
\usepackage{mathtools}
\usepackage{gensymb}
\usepackage[utf8]{inputenc}
\usepackage{float}
\newcommand{\mydet}[1]{\ensuremath{\begin{vmatrix}#1\end{vmatrix}}}
\providecommand{\brak}[1]{\ensuremath{\left(#1\right)}}
\providecommand{\norm}[1]{\left\lVert#1\right\rVert}
\newcommand{\solution}{\noindent \textbf{Solution: }}
\newcommand{\myvec}[1]{\ensuremath{\begin{pmatrix}#1\end{pmatrix}}}
\let\vec\mathbf

\begin{document}
\begin{center}
\textbf\large{CLASS-11 \\ CHAPTER-10 \\ STRAIGHT LINES}
\end{center}
\section*{Excercise 10.3}

\solution
\fi
\begin{enumerate}
\item The given equation can be expressed as
		\begin{align}
			\myvec{ 1 & -\sqrt{3}}\vec{x}= -8
			\end{align}
			yielding
\begin{align}
			\vec{n} = \myvec{ 1 & -\sqrt{3}}, c = -8
\end{align}
From the above, the	angle between perpendicular and the positive $x$-axis is given by
		\begin{align}
			\tan^{-1}\brak{-\sqrt{3}} = \frac{2\pi}{3}
		\end{align}
	The perpendicular distance from the origin to the line is given by
		\begin{align}
			d=\frac{\abs{c}}{\norm{\vec{n}}}=4
		\end{align}
\item In this case, the given equation becomes
          \begin{align}
		  \myvec{0 & 1}\vec{x} = 2
          \end{align}
	  yielding
                  \begin{align}
			  \vec{n}=\myvec{0\\1}, c = 2
                          \end{align}
          Angle between perpendicular and the positive $x$-axis is given by:
		\begin{align}  
			\tan^{-1}\infty = \frac{\pi}{2}
                \end{align}      
and  the perpendicular distance from the origin to the line is given by    
                                      \begin{align}
					      d=\frac{|c|}{\norm{\vec{n}}}=2             
                  \end{align}
\item   The given equation can be expressed as
                  \begin{align}
      \myvec{-1 & 1}\vec{x} = 4
                          \end{align}
			  yielding
                  \begin{align}
			  \vec{n}=\myvec{1\\-1}, c = 4
                          \end{align}
          Angle between perpendicular and the positive $x$-axis is given by
		\begin{align}   
			\tan^{-1}\brak{-1} = \frac{3\pi}{4}
                \end{align}                                                                           The perpendicular distance from the origin to the line is given by
		\begin{align}
			d=\frac{|c|}{\norm{\vec{n}}}=\frac{4}{\sqrt{2}}=2\sqrt{2}                    
                  \end{align}
\end{enumerate}
 
