\documentclass[journal,10pt,twocolumn]{article}
\usepackage{graphicx, float}
\usepackage[margin=0.5in]{geometry}
\usepackage{amsmath, bm}
\usepackage{array}
\usepackage{booktabs}
\usepackage[utf8]{inputenc}
\usepackage{amsfonts}
\usepackage{amssymb}
\usepackage{graphicx}
\usepackage{multicol}
\usepackage{tabularx}
\usepackage{hyperref}
\usepackage{mathtools}
\DeclareUnicodeCharacter{2212}{-}
\providecommand{\norm}[1]{\left\lVert#1\right\rVert}
\providecommand{\abs}[1]{\left\vert#1\right\vert}
\let\vec\mathbf
\newcommand{\myvec}[1]{\ensuremath{\begin{pmatrix}#1\end{pmatrix}}}
\newcommand{\mydet}[1]{\ensuremath{\begin{vmatrix}#1\end{vmatrix}}}
\providecommand{\brak}[1]{\ensuremath{\left(#1\right)}}
\providecommand{\lbrak}[1]{\ensuremath{\left(#1\right.}}
\providecommand{\rbrak}[1]{\ensuremath{\left.#1\right)}}
\providecommand{\sbrak}[1]{\ensuremath{{}\left[#1\right]}}
%\providecommand{\norm}[1]{\left\lVert#1\right\rVert}
%\providecommand{\sbrak}[1]{\ensuremath{{}\left[#1\right]}}
%\providecommand{\lsbrak}[1]{\ensuremath{{}\left[#1\right.}}
%\providecommand{\rsbrak}[1]{\ensuremath{{}\left.#1\right]}}
%\providecommand{\brak}[1]{\ensuremath{\left(#1\right)}}
%\providecommand{\lbrak}[1]{\ensuremath{\left(#1\right.}}
%\providecommand{\rbrak}[1]{\ensuremath{\left.#1\right)}}
%\providecommand{\cbrak}[1]{\ensuremath{\left\{#1\right\}}}
%\providecommand{\lcbrak}[1]{\ensuremath{\left\{#1\right.}}
%\providecommand{\rcbrak}[1]{\ensuremath{\left.#1\right\}}}
%\newcommand{\myvec}[1]{\ensuremath{\begin{pmatrix}#1\end{pmatrix}}}
%\let\vec\mathbf

\title{\textbf{Conic Assignment}}
\author{Srinivas Dulla \hspace{9cm} FWC22041}
\date{October 2022}

\begin{document}

\maketitle
\paragraph{\textit{Problem Statement} - The normal at the point (1,2) on the curve \(2y+x^2=3\) :\\
(a)x+y=0  \hspace{2cm} (b)x-y=0\\ 
(c)x+y+1=0 \hspace{2cm}  (d)x-y=1\\}

\section*{\large Solution}

\begin{figure}[H]
\centering
\includegraphics[width=1\columnwidth]{cone.jpg}
\caption{Tangents from A to circle through B, C and D}
\label{fig:triangle}
\end{figure}

+	The given equation of parabola $ 2y + $x^2 = 3$ can be written in the general quadratic form as
\begin{align}
    \label{eq:conic_quad_form}
    \vec{x}^{\top}\vec{V}\vec{x}+2\vec{u}^{\top}\vec{x}+f=0
    \end{align}
where
\begin{align}
	\label{eq:V_matrix}
	\vec{V} &= \myvec{1 & 0\\0 & 0},
	\\
	\label{eq:u_vector}
	\vec{u} &= \myvec{0\\1},
	\\
	\label{eq:f_value}
	f &= -3
	%\\
\end{align}

Let the point from which normals are drawn be $\vec{h}$. Then, the equation of the normal can be written as
\begin{align}
	\vec{x} = \vec{h} + \lambda\vec{m}
	\label{eq:normal_chord}
\end{align}
Say the point of intersection of \eqref{eq:normal_chord} with the conic is $\vec{q}$. A tangent drawn at $\vec{q}$ satisfies the equation
\begin{align}
	\label{eq:tangency_condition}
	\vec{n}^\top(\vec{Vq}+\vec{u}) = 0
\end{align}
Where $\vec{n}$ is the direction vector of the tangent and is perpendicular to $\vec{m}$ in \eqref{eq:normal_chord}.\\\\
In general, the parameter values for points of intersection of a line given by \eqref{eq:normal_chord} with a conic is given by
{\tiny
\begin{multline}
\lambda_i = \frac{1}
{
\vec{m}^T\vec{V}\vec{m}
}
\lbrak{-\vec{m}^T\brak{\vec{V}\vec{h}+\vec{u}}}
\\
\pm
\rbrak{\sqrt{
\sbrak{
\vec{m}^T\brak{\vec{V}\vec{h}+\vec{u}}
}^2
-
\brak
{
\vec{h}^T\vec{V}\vec{h} + 2\vec{u}^T\vec{h} +f
}
\brak{\vec{m}^T\vec{V}\vec{m}}
}
}
\label{eq:tangent_roots}
\end{multline}
}
Using \eqref{eq:tangent_roots} and \eqref{eq:normal_chord}, the intersection point $\vec{q}$ can be written as
\begin{align}
	\label{eq:point_of_tangency}
	\vec{q} = \vec{h} + \lambda_i\vec{m}
\end{align}
Substituting \eqref{eq:point_of_tangency} in \eqref{eq:tangency_condition},
\begin{align}
	\label{eq:normal_simp_1}
	\vec{n}^\top(\vec{V}(\vec{h}+\lambda_i\vec{m})+\vec{u}) = 0\\
	\label{eq:normal_simp_2}
	\implies \lambda_i\vec{n}^\top\vec{V}\vec{m} = -\vec{n}^\top(\vec{Vh}+\vec{u})
\end{align}
Substituting value of $\lambda_i$ from \eqref{eq:tangent_roots} in \eqref{eq:normal_simp_2}
{\tiny
\begin{multline}
	\frac{1}{\vec{m}^\top\vec{V}\vec{m}}\lbrak{-\vec{m}^\top\brak{\vec{Vh}+\vec{u}}} \\ 
	\pm \rbrak{\sqrt{\sbrak{\vec{m}^T\brak{\vec{V}\vec{h}+\vec{u}}}^2-\brak{\vec{h}^T\vec{V}\vec{h} + 2\vec{u}^T\vec{h} +f}\brak{\vec{m}^T\vec{V}\vec{m}}}}\vec{n}^\top\vec{V}\vec{m} \\
	= -\vec{n}^\top\brak{\vec{Vh}+\vec{u}}
	\label{eq:normal_simp_3}
\end{multline}
}
Rearranging the terms,
{\tiny
\begin{multline}
	\pm \sqrt{\sbrak{\vec{m}^T\brak{\vec{V}\vec{h}+\vec{u}}}^2-\brak{\vec{h}^T\vec{V}\vec{h} + 2\vec{u}^T\vec{h} +f}\brak{\vec{m}^T\vec{V}\vec{m}}} \brak{\vec{n}^\top\vec{V}\vec{m}} \\ = \brak{\vec{Vh}+\vec{u}}^\top\brak{\brak{\vec{n}^\top\vec{V}\vec{m}}\vec{m}-\brak{\vec{m}^\top\vec{V}\vec{m}}\vec{n}}
\end{multline}
}
Squaring on both sides
{\tiny
\begin{multline}
	\sbrak{\sbrak{\vec{m}^T\brak{\vec{V}\vec{h}+\vec{u}}}^2-\brak{\vec{h}^T\vec{V}\vec{h} + 2\vec{u}^T\vec{h} +f}\brak{\vec{m}^T\vec{V}\vec{m}}}\brak{\vec{n}^\top\vec{V}\vec{m}}^2 \\ = \sbrak{\brak{\vec{Vh}+\vec{u}}^\top\brak{\brak{\vec{n}^\top\vec{V}\vec{m}}\vec{m}-\brak{\vec{m}^\top\vec{V}\vec{m}}\vec{n}}}^2
	\label{eq:normal_solution}
\end{multline}
}\\
If $\vec{n}$ is taken as $\myvec{-\mu \\ 1}$, then $\vec{m}$ is $\myvec{-1 \\ -\mu}$. Substituting these values in \eqref{eq:normal_solution} and solving for $\mu$, the different possible normals passing through $\vec{h}$ are obtained.\\\\
Thus after solving we get the following values for $\mu$ = {-1, 1/2 - $\sqrt 3*I/2, 1/2 + $\sqrt 3*I/2}\\\\
Taking $\mu$=1 we get,
\begin{center}
$\vec{n} = \myvec{1 \\ 1}$, $\vec{m} = \myvec{-1 \\ 1}$\\
\end{center}
By calculating $\lambda_i$ from \eqref{eq:normal_simp_2}, we get
\begin{center}
    $\lambda_i = -1$
\end{center}
We find out $\vec{q}$ from \eqref{eq:point_of_tangency},
\begin{center}
where $\vec{h} = \myvec{1 \\ 1}$, $\vec{m} = \myvec{-1 \\ 1}$, $\lambda_i = -1$
\end{center}
\begin{center}
    $\vec{q} = \myvec{1 \\ 1} + (-1)\myvec{-1 \\ 1} 
\end{center}
\begin{center}
    Thus $\vec{q}$ satisfies Option(a) i.e. $x+y+1$
\end{center} 

\section*{\large Construction}
{
\setlength\extrarowheight{5pt}
\begin{tabular}{|c|c|c|}
	\hline
	\textbf{Symbol}&\textbf{Value}&\textbf{Description}\\[5pt]
	\hline
	$\vec{h}$&$\myvec{1 \\ 1}$&Given point through which Normal is passing\\[5pt]
	\hline
	$\vec{q}$&$\myvec{2 \\ 1}$&Foot of Normal\\[5pt]
	\hline
	$\vec{m}$ & $\myvec{-1 \\ 1}$ & Direction Vector of Normal\\[5pt]
	\hline
	$\vec{n}$ & $\myvec{1 \\ 1}$ & Direction Vector of Tangent at $\myvec{q}$\\
	\hline
\end{tabular}
}

\end{document}
