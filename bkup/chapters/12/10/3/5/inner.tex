\iffalse
\documentclass[journal,12pt,twocolumn]{IEEEtran}
\usepackage{graphicx}
\graphicspath{{./figs/}}{}
\usepackage{amsmath,amssymb,amsfonts,amsthm}
\newcommand{\myvec}[1]{\ensuremath{\begin{pmatrix}#1\end{pmatrix}}}
\providecommand{\norm}[1]{\lVert#1\rVert}
\usepackage{listings}
\usepackage{watermark}
\usepackage{titlesec}
\usepackage{caption}
\let\vec\mathbf
\lstset{
frame=single, 
breaklines=true,
columns=fullflexible
}
\thiswatermark{\centering \put(0,-105.0){\includegraphics[scale=0.15]{/sdcard/IITH/vectors/12.10.3.5/figs/logo.png}} }
\title{\mytitle}
\title{
Assignment - 12.10.3.5
}
\author{Surajit Sarkar}
\begin{document}
\maketitle
\tableofcontents
\bigskip
\section{\textbf{Problem}}
Show that each of the given three vectors is a unit vector:$\frac{1}{7}\myvec{2\hat{i}+3\hat{j}+6\hat{k}},\frac{1}{7}\myvec{3\hat{i}-6\hat{j}+2\hat{k}},\frac{1}{7}\myvec{6\hat{i}+2\hat{j}-3\hat{k}}$
Also, Show that they are mutually perpendicular to eatch other.
\section{\textbf{Solution}}
\fi
Let
\begin{align}
\vec{A}=\myvec{\frac{2}{7}\\ \frac{3}{7}\\ \frac{6}{7}},\Vec{B}=\myvec{\frac{3}{7}\\ -\frac{6}{7}\\ \frac{2}{7}},\vec{C}=\myvec{\frac{6}{7}\\ \frac{2}{7}\\ -\frac{3}{7}}
\end{align}
Then
\begin{align}
	\norm{\vec{A}}
=
\norm{\vec{B}}
=
\norm{\vec{C}}
=1
\end{align}
Also,
\begin{align}
\vec{A}^{\top}\vec{B}&=\myvec{\frac{2}{7}&\frac{3}{7}&\frac{6}{7}}\myvec{\frac{3}{7}\\-\frac{6}{7}\\ \frac{2}{7}}
=\frac{6}{49}-\frac{18}{49}+\frac{12}{49}
=0
\\
\vec{B}^{\top}\vec{C}&=\myvec{\frac{3}{7}&-\frac{6}{7}&\frac{2}{7}}\myvec{\frac{6}{7}\\ \frac{2}{7}\\-\frac{3}{7}}
=\frac{18}{49}-\frac{12}{49}-\frac{6}{49}
=0
\\
\vec{C}^{\top}\vec{A}&=\myvec{\frac{6}{7}&\frac{2}{7}&-\frac{3}{7}}\myvec{\frac{2}{7}\\ \frac{3}{7}\\ \frac{6}{7}}
=\frac{12}{49}+\frac{6}{49}-\frac{18}{49}
=0
\end{align}

