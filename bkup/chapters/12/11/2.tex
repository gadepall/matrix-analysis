\begin{enumerate}[label=\thesection.\arabic*,ref=\thesection.\theenumi]
\item  Show that the three lines with direction cosines
$\frac{12}{13},\frac{-3}{13},\frac{-4}{13}$; $\frac{4}{13},\frac{12}{13},\frac{3}{13}$; $\frac{3}{13},\frac{-4}{13},\frac{12}{13}$; are mutually perpendicular.\\
    \solution
		\iffalse
\documentclass[12pt]{article}
\usepackage{graphicx}
\usepackage{amsmath}
\usepackage{mathtools}
\usepackage{gensymb}

\newcommand{\mydet}[1]{\ensuremath{\begin{vmatrix}#1\end{vmatrix}}}
\providecommand{\brak}[1]{\ensuremath{\left(#1\right)}}
\providecommand{\norm}[1]{\left\lVert#1\right\rVert}
\newcommand{\solution}{\noindent \textbf{Solution: }}
\newcommand{\myvec}[1]{\ensuremath{\begin{pmatrix}#1\end{pmatrix}}}
\let\vec\mathbf

\begin{document}
\begin{center}
\textbf\large{CHAPTER-7 \\ COORDINATE GEOMETRY}

\end{center}
\section*{Excercise 7.1}

Q6.Name the type of quadilateral formed,if any, by the following points, and give reasons for your answer:
\begin{enumerate}
	\item $\brak{-1,-2}, \brak{1,0}, \brak{-1,2}, \brak{-3,0}$ 
	\item $\brak{-3,5}, \brak{3,1}, \brak{0,3}, \brak{-1,-4}$
	\item $\brak{4,5}, \brak{7,6}, \brak{4,3}, \brak{1,2}$
\end{enumerate}
\solution
\fi
\begin{enumerate}
\item The coordinates are given as
	\begin{align}
	\vec{A} = \myvec{
		-1\\
		-2\\
		},
	\vec{B} = \myvec{
		1\\
		0\\
		},
	\vec{C} = \myvec{
		-1\\
		2\\
		} \text{ and }
	\vec{D} = \myvec{
		-3\\
		0\\
		}
	\end{align}
	\begin{align}
		\vec{B} - \vec{A} &= \myvec{1\\0} - \myvec{-1\\-2} = \myvec{2\\2}\\
		\vec{C} - \vec{B} &= \myvec{-1\\2} - \myvec{1\\0} = \myvec{-2\\2}\\
		\vec{C} - \vec{D} &= \myvec{-1\\2} - \myvec{-3\\0} = \myvec{2\\2}\\
		\vec{D} - \vec{A} &= \myvec{-3\\0} - \myvec{-1\\-2} = \myvec{-2\\2}
	\end{align}
	\begin{align}	
		\vec{C} - \vec{A} &= \myvec{-1\\2} - \myvec{-1\\-2} = \myvec{0\\4}\\
		\vec{D} - \vec{B} &= \myvec{-3\\0} - \myvec{1\\0} = \myvec{-4\\0}
	\end{align}
	\begin{align}	
		\vec{B}-\vec{A} = \vec{C}-\vec{D} \text{ and } \vec{C}-\vec{B} = \vec{D}-\vec{A}.
	\end{align}
	Hence, $ABCD$ is a parallelogram.
	\begin{enumerate}
		\item Now checking if the adjacent sides are orthogonal to each other
	\begin{align}
		(\vec{B}-\vec{A})^\top (\vec{C}-\vec{B}) = \myvec{2&2} \myvec{-2\\2} = -4+4 = 0
	\end{align}
		\item Now checking if the diagonals are also orthogonal then it is a square else a rectangle.
	\end{enumerate}	
	\begin{align}
		(\vec{C}-\vec{A})^\top (\vec{D}-\vec{B}) = \myvec{0&4} \myvec{-4\\0} = 0
	\end{align}
	Hence the diagonals are orthogonal to each other.

	So, we can conclude that $ABCD$ is a square.

	As shown in Figure \ref{fig:10/7/1/6/Fig1} we can see that $ABCD$ is a square hence we can conclude that our theoritical result is verified.
 
\begin{figure}[!h]
	\begin{center} 
	    \includegraphics[width=\columnwidth]{chapters/10/7/1/6/figs/quad1}
	\end{center}
\caption{}
\label{fig:10/7/1/6/Fig1}
\end{figure}

\item The coordinates are given as
	\begin{align}
	\vec{A} = \myvec{
		-3\\
		5\\
		},
	\vec{B} = \myvec{
		3\\
		1\\
		},
	\vec{C} = \myvec{
		0\\
		3\\
		} \text{ and }
	\vec{D} = \myvec{
		-1\\
		-4\\
		}
	\end{align}
	\begin{align}
		\vec{B} - \vec{A} &= \myvec{3\\1} - \myvec{-3\\5} = \myvec{6\\-4}\\
		\vec{C} - \vec{B} &= \myvec{0\\3} - \myvec{3\\1} = \myvec{-3\\2}\\
		\vec{C} - \vec{D} &= \myvec{0\\3} - \myvec{-1\\-4} = \myvec{1\\7}\\
		\vec{D} - \vec{A} &= \myvec{-1\\-4} - \myvec{-3\\5} = \myvec{2\\-9}
	\end{align}
	\begin{align}
		\vec{C} - \vec{A} &= \myvec{0\\3} - \myvec{-3\\5} = \myvec{3\\-2}\\
		\vec{D} - \vec{B} &= \myvec{-1\\-4} - \myvec{3\\1} = \myvec{-4\\-5}
	\end{align}
	\begin{align}
	\vec{B}-\vec{A} \neq \vec{C}-\vec{D} \text{ and } \vec{C}-\vec{B} \neq \vec{D}-\vec{A},
	\end{align}
	Hence, $ABCD$ is not a parallelogram, it can be a irregular quadilateral.
	\begin{enumerate}
		\item Now to check if any three points are collinear,

	if rank of $\myvec{\vec{B}-\vec{A} & \vec{C}-\vec{B}} = 1$ then points are collinear

	Forming the collinearity matrix
	\begin{align}
		\myvec{6&-3\\-4&2} \xleftrightarrow{R_{2}\rightarrow R_{2}+\frac{2}{3}R_{1}}= \myvec{6&-3\\0&0}
	\end{align}
	\end{enumerate}
	Hence, rank = 1

	Since none of the opposite sides are parallel to each other and three points are collinear so these does not form a quadilateral.

	As shown in Figure \ref{fig:10/7/1/6/Fig2} we can see that $ABCD$ does not form a quadilateral and three points are collinear hence, our theoritical result is verified.
	
\begin{figure}[!h]
	\begin{center} 
	    \includegraphics[width=\columnwidth]{chapters/10/7/1/6/figs/quad2}
	\end{center}
\caption{}
\label{fig:10/7/1/6/Fig2}
\end{figure}
	
\item The coordinates are given as
	\begin{align}
	\vec{A} = \myvec{
		4\\
		5\\
		},
	\vec{B} = \myvec{
		7\\
		6\\
		},
	\vec{C} = \myvec{
		4\\
		3\\
		} \text{ and }
	\vec{D} = \myvec{
		1\\
		2\\
		}
	\end{align}
	\begin{align}
		\vec{B} - \vec{A} &= \myvec{7\\6} - \myvec{4\\5} = \myvec{3\\1}\\
		\vec{C} - \vec{B} &= \myvec{4\\3} - \myvec{7\\6} = \myvec{-3\\-3}\\
		\vec{C} - \vec{D} &= \myvec{4\\3} - \myvec{1\\2} = \myvec{3\\1}\\
		\vec{D} - \vec{A} &= \myvec{1\\2} - \myvec{4\\5} = \myvec{-3\\-3}
	\end{align}
	\begin{align}
		\vec{C} - \vec{A} &= \myvec{4\\3} - \myvec{4\\5} = \myvec{0\\-2}\\
		\vec{D} - \vec{B} &= \myvec{1\\2} - \myvec{7\\6} = \myvec{-6\\-4}
	\end{align}
	\begin{align}
		\vec{B}-\vec{A} = \vec{C}-\vec{D} \text{ and } \vec{C}-\vec{B} = \vec{D}-\vec{A},
	\end{align}
	Hence, $ABCD$ is a parallelogram.
	\begin{enumerate}
		\item Now checking if the adjacent sides are orthogonal to each other
	\begin{align}
		(\vec{B}-\vec{A})^\top (\vec{C}-\vec{B}) = \myvec{3&1} \myvec{-3\\-3} = -9-3 = -12
	\end{align}
	Since inner product is not zero so adjacent sides are not orthogonal.

	Hence, we can say that $ABCD$ is neither a rectangle nor a square.

		\item Now checking if the diagonals are orthogonal then it is a Rhombus.
	\begin{align}
		(\vec{C}- \vec{A})^\top (\vec{D}-\vec{B}) = \myvec{0&-2} \myvec{-6\\-4} = 0+8 = 8
	\end{align}
	\end{enumerate}		
	Hence the diagonals are also not orthogonal so we conclude that $ABCD$ is a parallelogram.

	As shown in Figure \ref{fig:10/7/1/6/Fig3} we can see that $ABCD$ forms a parallelogram hence, our theoritical result is verified.

\begin{figure}[!h]
	\begin{center} 
	    \includegraphics[width=\columnwidth]{chapters/10/7/1/6/figs/quad3}
	\end{center}
\caption{}
\label{fig:10/7/1/6/Fig3}
\end{figure}
\end{enumerate}



\item  Show that the line through the points $(1,-1,2),(3,4,-2 )$ is perpendicular to the line through the points$(0,3,2)$ and$(3,5,6)$.\\
    \solution
		\iffalse
\documentclass[12pt]{article}
\usepackage{graphicx}
\usepackage{amsmath}
\usepackage{mathtools}
\usepackage{gensymb}

\newcommand{\mydet}[1]{\ensuremath{\begin{vmatrix}#1\end{vmatrix}}}
\providecommand{\brak}[1]{\ensuremath{\left(#1\right)}}
\providecommand{\norm}[1]{\left\lVert#1\right\rVert}
\newcommand{\solution}{\noindent \textbf{Solution: }}
\newcommand{\myvec}[1]{\ensuremath{\begin{pmatrix}#1\end{pmatrix}}}
\let\vec\mathbf

\begin{document}
\begin{center}
\textbf\large{CHAPTER-7 \\ COORDINATE GEOMETRY}

\end{center}
\section*{Excercise 7.1}

Q6.Name the type of quadilateral formed,if any, by the following points, and give reasons for your answer:
\begin{enumerate}
	\item $\brak{-1,-2}, \brak{1,0}, \brak{-1,2}, \brak{-3,0}$ 
	\item $\brak{-3,5}, \brak{3,1}, \brak{0,3}, \brak{-1,-4}$
	\item $\brak{4,5}, \brak{7,6}, \brak{4,3}, \brak{1,2}$
\end{enumerate}
\solution
\fi
\begin{enumerate}
\item The coordinates are given as
	\begin{align}
	\vec{A} = \myvec{
		-1\\
		-2\\
		},
	\vec{B} = \myvec{
		1\\
		0\\
		},
	\vec{C} = \myvec{
		-1\\
		2\\
		} \text{ and }
	\vec{D} = \myvec{
		-3\\
		0\\
		}
	\end{align}
	\begin{align}
		\vec{B} - \vec{A} &= \myvec{1\\0} - \myvec{-1\\-2} = \myvec{2\\2}\\
		\vec{C} - \vec{B} &= \myvec{-1\\2} - \myvec{1\\0} = \myvec{-2\\2}\\
		\vec{C} - \vec{D} &= \myvec{-1\\2} - \myvec{-3\\0} = \myvec{2\\2}\\
		\vec{D} - \vec{A} &= \myvec{-3\\0} - \myvec{-1\\-2} = \myvec{-2\\2}
	\end{align}
	\begin{align}	
		\vec{C} - \vec{A} &= \myvec{-1\\2} - \myvec{-1\\-2} = \myvec{0\\4}\\
		\vec{D} - \vec{B} &= \myvec{-3\\0} - \myvec{1\\0} = \myvec{-4\\0}
	\end{align}
	\begin{align}	
		\vec{B}-\vec{A} = \vec{C}-\vec{D} \text{ and } \vec{C}-\vec{B} = \vec{D}-\vec{A}.
	\end{align}
	Hence, $ABCD$ is a parallelogram.
	\begin{enumerate}
		\item Now checking if the adjacent sides are orthogonal to each other
	\begin{align}
		(\vec{B}-\vec{A})^\top (\vec{C}-\vec{B}) = \myvec{2&2} \myvec{-2\\2} = -4+4 = 0
	\end{align}
		\item Now checking if the diagonals are also orthogonal then it is a square else a rectangle.
	\end{enumerate}	
	\begin{align}
		(\vec{C}-\vec{A})^\top (\vec{D}-\vec{B}) = \myvec{0&4} \myvec{-4\\0} = 0
	\end{align}
	Hence the diagonals are orthogonal to each other.

	So, we can conclude that $ABCD$ is a square.

	As shown in Figure \ref{fig:10/7/1/6/Fig1} we can see that $ABCD$ is a square hence we can conclude that our theoritical result is verified.
 
\begin{figure}[!h]
	\begin{center} 
	    \includegraphics[width=\columnwidth]{chapters/10/7/1/6/figs/quad1}
	\end{center}
\caption{}
\label{fig:10/7/1/6/Fig1}
\end{figure}

\item The coordinates are given as
	\begin{align}
	\vec{A} = \myvec{
		-3\\
		5\\
		},
	\vec{B} = \myvec{
		3\\
		1\\
		},
	\vec{C} = \myvec{
		0\\
		3\\
		} \text{ and }
	\vec{D} = \myvec{
		-1\\
		-4\\
		}
	\end{align}
	\begin{align}
		\vec{B} - \vec{A} &= \myvec{3\\1} - \myvec{-3\\5} = \myvec{6\\-4}\\
		\vec{C} - \vec{B} &= \myvec{0\\3} - \myvec{3\\1} = \myvec{-3\\2}\\
		\vec{C} - \vec{D} &= \myvec{0\\3} - \myvec{-1\\-4} = \myvec{1\\7}\\
		\vec{D} - \vec{A} &= \myvec{-1\\-4} - \myvec{-3\\5} = \myvec{2\\-9}
	\end{align}
	\begin{align}
		\vec{C} - \vec{A} &= \myvec{0\\3} - \myvec{-3\\5} = \myvec{3\\-2}\\
		\vec{D} - \vec{B} &= \myvec{-1\\-4} - \myvec{3\\1} = \myvec{-4\\-5}
	\end{align}
	\begin{align}
	\vec{B}-\vec{A} \neq \vec{C}-\vec{D} \text{ and } \vec{C}-\vec{B} \neq \vec{D}-\vec{A},
	\end{align}
	Hence, $ABCD$ is not a parallelogram, it can be a irregular quadilateral.
	\begin{enumerate}
		\item Now to check if any three points are collinear,

	if rank of $\myvec{\vec{B}-\vec{A} & \vec{C}-\vec{B}} = 1$ then points are collinear

	Forming the collinearity matrix
	\begin{align}
		\myvec{6&-3\\-4&2} \xleftrightarrow{R_{2}\rightarrow R_{2}+\frac{2}{3}R_{1}}= \myvec{6&-3\\0&0}
	\end{align}
	\end{enumerate}
	Hence, rank = 1

	Since none of the opposite sides are parallel to each other and three points are collinear so these does not form a quadilateral.

	As shown in Figure \ref{fig:10/7/1/6/Fig2} we can see that $ABCD$ does not form a quadilateral and three points are collinear hence, our theoritical result is verified.
	
\begin{figure}[!h]
	\begin{center} 
	    \includegraphics[width=\columnwidth]{chapters/10/7/1/6/figs/quad2}
	\end{center}
\caption{}
\label{fig:10/7/1/6/Fig2}
\end{figure}
	
\item The coordinates are given as
	\begin{align}
	\vec{A} = \myvec{
		4\\
		5\\
		},
	\vec{B} = \myvec{
		7\\
		6\\
		},
	\vec{C} = \myvec{
		4\\
		3\\
		} \text{ and }
	\vec{D} = \myvec{
		1\\
		2\\
		}
	\end{align}
	\begin{align}
		\vec{B} - \vec{A} &= \myvec{7\\6} - \myvec{4\\5} = \myvec{3\\1}\\
		\vec{C} - \vec{B} &= \myvec{4\\3} - \myvec{7\\6} = \myvec{-3\\-3}\\
		\vec{C} - \vec{D} &= \myvec{4\\3} - \myvec{1\\2} = \myvec{3\\1}\\
		\vec{D} - \vec{A} &= \myvec{1\\2} - \myvec{4\\5} = \myvec{-3\\-3}
	\end{align}
	\begin{align}
		\vec{C} - \vec{A} &= \myvec{4\\3} - \myvec{4\\5} = \myvec{0\\-2}\\
		\vec{D} - \vec{B} &= \myvec{1\\2} - \myvec{7\\6} = \myvec{-6\\-4}
	\end{align}
	\begin{align}
		\vec{B}-\vec{A} = \vec{C}-\vec{D} \text{ and } \vec{C}-\vec{B} = \vec{D}-\vec{A},
	\end{align}
	Hence, $ABCD$ is a parallelogram.
	\begin{enumerate}
		\item Now checking if the adjacent sides are orthogonal to each other
	\begin{align}
		(\vec{B}-\vec{A})^\top (\vec{C}-\vec{B}) = \myvec{3&1} \myvec{-3\\-3} = -9-3 = -12
	\end{align}
	Since inner product is not zero so adjacent sides are not orthogonal.

	Hence, we can say that $ABCD$ is neither a rectangle nor a square.

		\item Now checking if the diagonals are orthogonal then it is a Rhombus.
	\begin{align}
		(\vec{C}- \vec{A})^\top (\vec{D}-\vec{B}) = \myvec{0&-2} \myvec{-6\\-4} = 0+8 = 8
	\end{align}
	\end{enumerate}		
	Hence the diagonals are also not orthogonal so we conclude that $ABCD$ is a parallelogram.

	As shown in Figure \ref{fig:10/7/1/6/Fig3} we can see that $ABCD$ forms a parallelogram hence, our theoritical result is verified.

\begin{figure}[!h]
	\begin{center} 
	    \includegraphics[width=\columnwidth]{chapters/10/7/1/6/figs/quad3}
	\end{center}
\caption{}
\label{fig:10/7/1/6/Fig3}
\end{figure}
\end{enumerate}



\item Show that the line through the points $(4,7,8),(2,3,4)$ is parallel to the line through the points $(-1,-2,1),(1,2,5)$.\\
\item  Find the equation of the line which passes through the point $(1,2,3)$ and is parallel to the vector $3\hat{i}+2\hat{j}-2\hat{k}$\\
\item  Find the equation of the line in vector and in cartesian form that passes through the point with position vector $2\hat{i}-\hat{j}+4\hat{k}$ and is in direction $\hat{i}+2\hat{j}-\hat{k}$.\\
    \solution
		\iffalse
\documentclass[journal,10pt,twocolumn]{article}
\usepackage{graphicx}
\usepackage[margin=0.5in]{geometry}
\usepackage[cmex10]{amsmath}
\usepackage{array}
\usepackage{booktabs}
\usepackage{listings}
\title{\textbf{Line Assignment}}
\author{Bhavani Kanike}
\date{October 2022}

\providecommand{\norm}[1]{\left\lVert#1\right\rVert}
\providecommand{\abs}[1]{\left\vert#1\right\vert}
\let\vec\mathbf
\newcommand{\myvec}[1]{\ensuremath{\begin{pmatrix}#1\end{pmatrix}}}
\newcommand{\mydet}[1]{\ensuremath{\begin{vmatrix}#1\end{vmatrix}}}
\providecommand{\brak}[1]{\ensuremath{\left(#1\right)}}

\begin{document}

\maketitle
\paragraph{\textit{Problem Statement} 
\fi
ABCD is a quadrilateral in which $\vec{P}, \vec{Q}, \vec{R}$ and $\vec{S}$ are mid-points of the sides AB, BC, CD and DA (see Fig \ref{fig:9/8/2/1}). AC is a diagonal. 
		
Show that 
\begin{enumerate}
	\item $SR \parallel AC$ and $SR =\frac{1}{2} AC$
\item $PQ = SR$
\item $PQRS$ is a parallelogram.
\end{enumerate}
 	\begin{figure}
		\centering
 \includegraphics[width=\columnwidth]{chapters/9/8/2/1/figs/line1.pdf}
		\caption{}
		\label{fig:9/8/2/1}
  	\end{figure}
	\solution 
	Using 
	  \eqref{eq:section_formula},
	\begin{align}
		\label{eq:9/8/2/1}
		\begin{split}
		\vec{P} &= \frac{\vec{A}+\vec{B}}{2}\\
 \vec{Q} &= \frac{\vec{C}+\vec{B}}{2}\\
 \vec{R} &= \frac{\vec{C}+\vec{D}}{2}\\
 \vec{S} &= \frac{\vec{D}+\vec{A}}{2}
		\end{split}
	\end{align}
\begin{enumerate}
	\item
	Consequently, 
	\begin{align}
\vec{R}
		-\vec{S} &= \frac{\vec{C}-\vec{A}}{2}
		\\
		\implies SR &\parallel AC
	\end{align}
	Also, 
	\begin{align}
		\norm{\vec{R}
		-\vec{S}} &= \frac{\norm{\vec{C}-\vec{A}}}{2}
		\\
		\implies SR &= \frac{1}{2}AC
	\end{align}
\item 	From 
		\eqref{eq:9/8/2/1},
	\begin{align}
\vec{R}
		-\vec{S} = \vec{Q}-\vec{P}
	\end{align}
	which means that $PQRS$ is a parallelogram and $PQ = SR$.
\end{enumerate}
%
\iffalse
\begin{figure}[h]
\centering
\includegraphics[width=1\columnwidth]
\caption{Figure}
\label{fig:triangle}
\end{figure}

\section*{Solution}

$\boldsymbol Given :$  ABCD is a Quadrilateral P,Q,R and S are the midpoints of line AB,BC,CD,DA.We can obtain the points P,Q,R and S from A,B,C and D and are given by\\\\
\boldmath
\unboldmath
(3) To prove that PQRS is a parallelogram we need to prove  PQ // SR
To prove SR $\parallel$ PQ\\
Direction vector of line SR  $\boldsymbol {(R-S) =  \frac{(C-A)}{2}}$\\\\
Direction vector of line PQ  $\boldsymbol {(Q-P)= \frac{(C-A)}{2}}$\\\\
\begin{equation}
	\boldsymbol {(R-S) = (Q-P) = \frac{(C-A)}{2}}\\
\end{equation}
Since the direction vectors of line SR and PQ are in same direction\\\\
$SR \parallel PQ$\\
Therefore,
$\boldsymbol{ PQRS }$ is a parallelogram\\\\

	
(1)  Directional vector of line SR  = $\boldsymbol {(R-S)}$ = $\frac{\boldsymbol{(C-A)}}{2} $\\
Directional vector of line AC  = $\boldsymbol {(C-A)}$\\

It is observed that the constant k is $\frac{1}{2}$

Therefore
\begin{equation}
	SR \parallel AC
\end{equation} 

and from equation 1 
\begin{equation}
	\boldsymbol {SR = \frac{1}{2}AC}    
\end{equation}\\


(2)   To prove PQ = SR\\ 
		From euqation 1\\\\
\begin{equation}
		\boldsymbol{ (Q-P) = (R-S) = \frac{(C-A)}{2}}
\end{equation}
	 



\section{Execution}
The below python code realizes the construction:
\begin{lstlisting}
https://github.com/bhavani360/FWC_assignments
\end{lstlisting}
	
\section*{Construction}
The dimensions of the Quadrilateral ABCD are taken as below\\
{
\setlength\extrarowheight{2pt}
\centering
	\begin{tabular}{|c|c|}
	\hline
	\textbf{symbol}&\textbf{value}\\
	\hline
	r&8\\
	\hline
	$\theta$&pi/2.5\\
	\hline
	d&7\\
	\hline
	A&(0,0)\\
	\hline
	B&(d,0)\\
	\hline
	D&(rcos$\theta$,rsin$\theta$)\\
	\hline
	C&(D/1.5)+B\\
	\hline
\end{tabular}
}
\end{document}
\fi

\item Find the cartesian equation of the line which passes through the point $(-2,4,-5)$ and parallel to the line given by$ \frac{x+3}{3}=\frac{y-4}{5}=\frac{z+8}{6}$.\\
\item The cartesian equation of a line is $ \frac{x-5}{3}=\frac{y+4}{7}=\frac{z-6}{2}$. Write its vector form.\\
\item Find the vector and the cartesian equations of the lines that passes through the origin and $(5,-2,3)$.\\
    \solution
		\iffalse
\documentclass[journal,10pt,twocolumn]{article}
\usepackage{graphicx}
\usepackage[margin=0.5in]{geometry}
\usepackage[cmex10]{amsmath}
\usepackage{array}
\usepackage{booktabs}
\usepackage{listings}
\title{\textbf{Line Assignment}}
\author{Bhavani Kanike}
\date{October 2022}

\providecommand{\norm}[1]{\left\lVert#1\right\rVert}
\providecommand{\abs}[1]{\left\vert#1\right\vert}
\let\vec\mathbf
\newcommand{\myvec}[1]{\ensuremath{\begin{pmatrix}#1\end{pmatrix}}}
\newcommand{\mydet}[1]{\ensuremath{\begin{vmatrix}#1\end{vmatrix}}}
\providecommand{\brak}[1]{\ensuremath{\left(#1\right)}}

\begin{document}

\maketitle
\paragraph{\textit{Problem Statement} 
\fi
ABCD is a quadrilateral in which $\vec{P}, \vec{Q}, \vec{R}$ and $\vec{S}$ are mid-points of the sides AB, BC, CD and DA (see Fig \ref{fig:9/8/2/1}). AC is a diagonal. 
		
Show that 
\begin{enumerate}
	\item $SR \parallel AC$ and $SR =\frac{1}{2} AC$
\item $PQ = SR$
\item $PQRS$ is a parallelogram.
\end{enumerate}
 	\begin{figure}
		\centering
 \includegraphics[width=\columnwidth]{chapters/9/8/2/1/figs/line1.pdf}
		\caption{}
		\label{fig:9/8/2/1}
  	\end{figure}
	\solution 
	Using 
	  \eqref{eq:section_formula},
	\begin{align}
		\label{eq:9/8/2/1}
		\begin{split}
		\vec{P} &= \frac{\vec{A}+\vec{B}}{2}\\
 \vec{Q} &= \frac{\vec{C}+\vec{B}}{2}\\
 \vec{R} &= \frac{\vec{C}+\vec{D}}{2}\\
 \vec{S} &= \frac{\vec{D}+\vec{A}}{2}
		\end{split}
	\end{align}
\begin{enumerate}
	\item
	Consequently, 
	\begin{align}
\vec{R}
		-\vec{S} &= \frac{\vec{C}-\vec{A}}{2}
		\\
		\implies SR &\parallel AC
	\end{align}
	Also, 
	\begin{align}
		\norm{\vec{R}
		-\vec{S}} &= \frac{\norm{\vec{C}-\vec{A}}}{2}
		\\
		\implies SR &= \frac{1}{2}AC
	\end{align}
\item 	From 
		\eqref{eq:9/8/2/1},
	\begin{align}
\vec{R}
		-\vec{S} = \vec{Q}-\vec{P}
	\end{align}
	which means that $PQRS$ is a parallelogram and $PQ = SR$.
\end{enumerate}
%
\iffalse
\begin{figure}[h]
\centering
\includegraphics[width=1\columnwidth]
\caption{Figure}
\label{fig:triangle}
\end{figure}

\section*{Solution}

$\boldsymbol Given :$  ABCD is a Quadrilateral P,Q,R and S are the midpoints of line AB,BC,CD,DA.We can obtain the points P,Q,R and S from A,B,C and D and are given by\\\\
\boldmath
\unboldmath
(3) To prove that PQRS is a parallelogram we need to prove  PQ // SR
To prove SR $\parallel$ PQ\\
Direction vector of line SR  $\boldsymbol {(R-S) =  \frac{(C-A)}{2}}$\\\\
Direction vector of line PQ  $\boldsymbol {(Q-P)= \frac{(C-A)}{2}}$\\\\
\begin{equation}
	\boldsymbol {(R-S) = (Q-P) = \frac{(C-A)}{2}}\\
\end{equation}
Since the direction vectors of line SR and PQ are in same direction\\\\
$SR \parallel PQ$\\
Therefore,
$\boldsymbol{ PQRS }$ is a parallelogram\\\\

	
(1)  Directional vector of line SR  = $\boldsymbol {(R-S)}$ = $\frac{\boldsymbol{(C-A)}}{2} $\\
Directional vector of line AC  = $\boldsymbol {(C-A)}$\\

It is observed that the constant k is $\frac{1}{2}$

Therefore
\begin{equation}
	SR \parallel AC
\end{equation} 

and from equation 1 
\begin{equation}
	\boldsymbol {SR = \frac{1}{2}AC}    
\end{equation}\\


(2)   To prove PQ = SR\\ 
		From euqation 1\\\\
\begin{equation}
		\boldsymbol{ (Q-P) = (R-S) = \frac{(C-A)}{2}}
\end{equation}
	 



\section{Execution}
The below python code realizes the construction:
\begin{lstlisting}
https://github.com/bhavani360/FWC_assignments
\end{lstlisting}
	
\section*{Construction}
The dimensions of the Quadrilateral ABCD are taken as below\\
{
\setlength\extrarowheight{2pt}
\centering
	\begin{tabular}{|c|c|}
	\hline
	\textbf{symbol}&\textbf{value}\\
	\hline
	r&8\\
	\hline
	$\theta$&pi/2.5\\
	\hline
	d&7\\
	\hline
	A&(0,0)\\
	\hline
	B&(d,0)\\
	\hline
	D&(rcos$\theta$,rsin$\theta$)\\
	\hline
	C&(D/1.5)+B\\
	\hline
\end{tabular}
}
\end{document}
\fi

\item Find the vector and the cartesian equations of the line that passes through the points $(3,-2,-5),(3,-2,6)$.\\
    \solution
		\iffalse
\documentclass[journal,10pt,twocolumn]{article}
\usepackage{graphicx}
\usepackage[margin=0.5in]{geometry}
\usepackage[cmex10]{amsmath}
\usepackage{array}
\usepackage{booktabs}
\usepackage{listings}
\title{\textbf{Line Assignment}}
\author{Bhavani Kanike}
\date{October 2022}

\providecommand{\norm}[1]{\left\lVert#1\right\rVert}
\providecommand{\abs}[1]{\left\vert#1\right\vert}
\let\vec\mathbf
\newcommand{\myvec}[1]{\ensuremath{\begin{pmatrix}#1\end{pmatrix}}}
\newcommand{\mydet}[1]{\ensuremath{\begin{vmatrix}#1\end{vmatrix}}}
\providecommand{\brak}[1]{\ensuremath{\left(#1\right)}}

\begin{document}

\maketitle
\paragraph{\textit{Problem Statement} 
\fi
ABCD is a quadrilateral in which $\vec{P}, \vec{Q}, \vec{R}$ and $\vec{S}$ are mid-points of the sides AB, BC, CD and DA (see Fig \ref{fig:9/8/2/1}). AC is a diagonal. 
		
Show that 
\begin{enumerate}
	\item $SR \parallel AC$ and $SR =\frac{1}{2} AC$
\item $PQ = SR$
\item $PQRS$ is a parallelogram.
\end{enumerate}
 	\begin{figure}
		\centering
 \includegraphics[width=\columnwidth]{chapters/9/8/2/1/figs/line1.pdf}
		\caption{}
		\label{fig:9/8/2/1}
  	\end{figure}
	\solution 
	Using 
	  \eqref{eq:section_formula},
	\begin{align}
		\label{eq:9/8/2/1}
		\begin{split}
		\vec{P} &= \frac{\vec{A}+\vec{B}}{2}\\
 \vec{Q} &= \frac{\vec{C}+\vec{B}}{2}\\
 \vec{R} &= \frac{\vec{C}+\vec{D}}{2}\\
 \vec{S} &= \frac{\vec{D}+\vec{A}}{2}
		\end{split}
	\end{align}
\begin{enumerate}
	\item
	Consequently, 
	\begin{align}
\vec{R}
		-\vec{S} &= \frac{\vec{C}-\vec{A}}{2}
		\\
		\implies SR &\parallel AC
	\end{align}
	Also, 
	\begin{align}
		\norm{\vec{R}
		-\vec{S}} &= \frac{\norm{\vec{C}-\vec{A}}}{2}
		\\
		\implies SR &= \frac{1}{2}AC
	\end{align}
\item 	From 
		\eqref{eq:9/8/2/1},
	\begin{align}
\vec{R}
		-\vec{S} = \vec{Q}-\vec{P}
	\end{align}
	which means that $PQRS$ is a parallelogram and $PQ = SR$.
\end{enumerate}
%
\iffalse
\begin{figure}[h]
\centering
\includegraphics[width=1\columnwidth]
\caption{Figure}
\label{fig:triangle}
\end{figure}

\section*{Solution}

$\boldsymbol Given :$  ABCD is a Quadrilateral P,Q,R and S are the midpoints of line AB,BC,CD,DA.We can obtain the points P,Q,R and S from A,B,C and D and are given by\\\\
\boldmath
\unboldmath
(3) To prove that PQRS is a parallelogram we need to prove  PQ // SR
To prove SR $\parallel$ PQ\\
Direction vector of line SR  $\boldsymbol {(R-S) =  \frac{(C-A)}{2}}$\\\\
Direction vector of line PQ  $\boldsymbol {(Q-P)= \frac{(C-A)}{2}}$\\\\
\begin{equation}
	\boldsymbol {(R-S) = (Q-P) = \frac{(C-A)}{2}}\\
\end{equation}
Since the direction vectors of line SR and PQ are in same direction\\\\
$SR \parallel PQ$\\
Therefore,
$\boldsymbol{ PQRS }$ is a parallelogram\\\\

	
(1)  Directional vector of line SR  = $\boldsymbol {(R-S)}$ = $\frac{\boldsymbol{(C-A)}}{2} $\\
Directional vector of line AC  = $\boldsymbol {(C-A)}$\\

It is observed that the constant k is $\frac{1}{2}$

Therefore
\begin{equation}
	SR \parallel AC
\end{equation} 

and from equation 1 
\begin{equation}
	\boldsymbol {SR = \frac{1}{2}AC}    
\end{equation}\\


(2)   To prove PQ = SR\\ 
		From euqation 1\\\\
\begin{equation}
		\boldsymbol{ (Q-P) = (R-S) = \frac{(C-A)}{2}}
\end{equation}
	 



\section{Execution}
The below python code realizes the construction:
\begin{lstlisting}
https://github.com/bhavani360/FWC_assignments
\end{lstlisting}
	
\section*{Construction}
The dimensions of the Quadrilateral ABCD are taken as below\\
{
\setlength\extrarowheight{2pt}
\centering
	\begin{tabular}{|c|c|}
	\hline
	\textbf{symbol}&\textbf{value}\\
	\hline
	r&8\\
	\hline
	$\theta$&pi/2.5\\
	\hline
	d&7\\
	\hline
	A&(0,0)\\
	\hline
	B&(d,0)\\
	\hline
	D&(rcos$\theta$,rsin$\theta$)\\
	\hline
	C&(D/1.5)+B\\
	\hline
\end{tabular}
}
\end{document}
\fi

\item  Find the angle between the following pairs of lines:
\begin{enumerate}
\item  $\overrightarrow{r}=2\hat{i}-5\hat{j}+\hat{k}+\lambda(3\hat{i}+2\hat{j}+6\hat{k})$ and\\ $\overrightarrow{r}=7\hat{i}-6\hat{k}+\mu(\hat{i}+2\hat{j}+2\hat{k})$
\item   $\overrightarrow{r}=3\hat{i}+\hat{j}-2\hat{k}+\lambda(\hat{i}-\hat{j}-2\hat{k})$ and\\ $\overrightarrow{r}=2\hat{i}-\hat{j}-56\hat{k}+\mu(3\hat{i}-5\hat{j}-4\hat{k})$ 
\end{enumerate}
\item Find the angle between the following pairs of lines:
\begin{enumerate}
\item $ \frac{x-2}{2}=\frac{y-1}{5}=\frac{z+3}{-3}$ and $ \frac{x+2}{-1}=\frac{y-4}{8}=\frac{z-5}{4}$.
\item $ \frac{x}{2}=\frac{y}{2}=\frac{z}{1}$ and $ \frac{x-5}{4}=\frac{y-2}{1}=\frac{z-3}{8}$.
\end{enumerate}
\item Find the values of p so that the lines $ \frac{1-x}{3}=\frac{7y-14}{2p}=\frac{z-3}{2}$ and $ \frac{7-7x}{3p}=\frac{y-5}{1}=\frac{6-z}{5}$ are at right angles.
\item Show that the lines $ \frac{x-5}{7}=\frac{y+2}{-5}=\frac{z}{1}$ and $ \frac{x}{1}=\frac{y}{2}=\frac{z}{3}$ are perpendicular to each other.
	\\
    \solution
		\iffalse
\documentclass[journal,12pt,twocolumn]{IEEEtran}
%
\usepackage{setspace}
\usepackage{gensymb}
%\doublespacing
\singlespacing

%\usepackage{graphicx}
%\usepackage{amssymb}
%\usepackage{relsize}
\usepackage[cmex10]{amsmath}
%\usepackage{amsthm}
%\interdisplaylinepenalty=2500
%\savesymbol{iint}
%\usepackage{txfonts}
%\restoresymbol{TXF}{iint}
%\usepackage{wasysym}
\usepackage{amsthm}
%\usepackage{iithtlc}
\usepackage{mathrsfs}
\usepackage{txfonts}
\usepackage{stfloats}
\usepackage{bm}
\usepackage{cite}
\usepackage{cases}
\usepackage{subfig}
%\usepackage{xtab}
\usepackage{longtable}
\usepackage{multirow}
%\usepackage{algorithm}
%\usepackage{algpseudocode}
\usepackage{enumitem}
\usepackage{mathtools}
\usepackage{steinmetz}
\usepackage{tikz}
\usepackage{circuitikz}
\usepackage{verbatim}
\usepackage{tfrupee}
\usepackage[breaklinks=true]{hyperref}
%\usepackage{stmaryrd}
\usepackage{tkz-euclide} % loads  TikZ and tkz-base
%\usetkzobj{all}
\usetikzlibrary{calc,math}
\usepackage{listings}
    \usepackage{color}                                            %%
    \usepackage{array}                                            %%
    \usepackage{longtable}                                        %%
    \usepackage{calc}                                             %%
    \usepackage{multirow}                                         %%
    \usepackage{hhline}                                           %%
    \usepackage{ifthen}                                           %%
  %optionally (for landscape tables embedded in another document): %%
    \usepackage{lscape}     
\usepackage{multicol}
\usepackage{chngcntr}
%\usepackage{enumerate}

%\usepackage{wasysym}
%\newcounter{MYtempeqncnt}
\DeclareMathOperator*{\Res}{Res}
%\renewcommand{\baselinestretch}{2}
\renewcommand\thesection{\arabic{section}}
\renewcommand\thesubsection{\thesection.\arabic{subsection}}
\renewcommand\thesubsubsection{\thesubsection.\arabic{subsubsection}}

\renewcommand\thesectiondis{\arabic{section}}
\renewcommand\thesubsectiondis{\thesectiondis.\arabic{subsection}}
\renewcommand\thesubsubsectiondis{\thesubsectiondis.\arabic{subsubsection}}

% correct bad hyphenation here
\hyphenation{op-tical net-works semi-conduc-tor}
\def\inputGnumericTable{}                                 %%

\lstset{
%language=C,
frame=single, 
breaklines=true,
columns=fullflexible
}
%\lstset{
%language=tex,
%frame=single, 
%breaklines=true
%}


\begin{document}
%


\newtheorem{theorem}{Theorem}[section]
\newtheorem{problem}{Problem}
\newtheorem{proposition}{Proposition}[section]
\newtheorem{lemma}{Lemma}[section]
\newtheorem{corollary}[theorem]{Corollary}
\newtheorem{example}{Example}[section]
\newtheorem{definition}[problem]{Definition}
%\newtheorem{thm}{Theorem}[section] 
%\newtheorem{defn}[thm]{Definition}
%\newtheorem{algorithm}{Algorithm}[section]
%\newtheorem{cor}{Corollary}
\newcommand{\BEQA}{\begin{eqnarray}}
\newcommand{\EEQA}{\end{eqnarray}}
\newcommand{\define}{\stackrel{\triangle}{=}}

\bibliographystyle{IEEEtran}
%\bibliographystyle{ieeetr}


\providecommand{\mbf}{\mathbf}
\providecommand{\pr}[1]{\ensuremath{\Pr\left(#1\right)}}
\providecommand{\qfunc}[1]{\ensuremath{Q\left(#1\right)}}
\providecommand{\sbrak}[1]{\ensuremath{{}\left[#1\right]}}
\providecommand{\lsbrak}[1]{\ensuremath{{}\left[#1\right.}}
\providecommand{\rsbrak}[1]{\ensuremath{{}\left.#1\right]}}
\providecommand{\brak}[1]{\ensuremath{\left(#1\right)}}
\providecommand{\lbrak}[1]{\ensuremath{\left(#1\right.}}
\providecommand{\rbrak}[1]{\ensuremath{\left.#1\right)}}
\providecommand{\cbrak}[1]{\ensuremath{\left\{#1\right\}}}
\providecommand{\lcbrak}[1]{\ensuremath{\left\{#1\right.}}
\providecommand{\rcbrak}[1]{\ensuremath{\left.#1\right\}}}
\theoremstyle{remark}
\newtheorem{rem}{Remark}
\newcommand{\sgn}{\mathop{\mathrm{sgn}}}
\providecommand{\abs}[1]{\left\vert#1\right\vert}
\providecommand{\res}[1]{\Res\displaylimits_{#1}} 
\providecommand{\norm}[1]{\left\lVert#1\right\rVert}
%\providecommand{\norm}[1]{\lVert#1\rVert}
\providecommand{\mtx}[1]{\mathbf{#1}}
\providecommand{\mean}[1]{E\left[ #1 \right]}
\providecommand{\fourier}{\overset{\mathcal{F}}{ \rightleftharpoons}}
%\providecommand{\hilbert}{\overset{\mathcal{H}}{ \rightleftharpoons}}
\providecommand{\system}{\overset{\mathcal{H}}{ \longleftrightarrow}}
	%\newcommand{\solution}[2]{\textbf{Solution:}{#1}}
\newcommand{\solution}{\noindent \textbf{Solution: }}
\newcommand{\cosec}{\,\text{cosec}\,}
\providecommand{\dec}[2]{\ensuremath{\overset{#1}{\underset{#2}{\gtrless}}}}
\newcommand{\myvec}[1]{\ensuremath{\begin{pmatrix}#1\end{pmatrix}}}
\newcommand{\mydet}[1]{\ensuremath{\begin{vmatrix}#1\end{vmatrix}}}
%\numberwithin{equation}{section}
\numberwithin{equation}{subsection}
%\numberwithin{problem}{section}
%\numberwithin{definition}{section}
\makeatletter
\@addtoreset{figure}{problem}
\makeatother

\let\StandardTheFigure\thefigure
\let\vec\mathbf
%\renewcommand{\thefigure}{\theproblem.\arabic{figure}}
\renewcommand{\thefigure}{\theproblem}
%\setlist[enumerate,1]{before=\renewcommand\theequation{\theenumi.\arabic{equation}}
%\counterwithin{equation}{enumi}


%\renewcommand{\theequation}{\arabic{subsection}.\arabic{equation}}

\def\putbox#1#2#3{\makebox[0in][l]{\makebox[#1][l]{}\raisebox{\baselineskip}[0in][0in]{\raisebox{#2}[0in][0in]{#3}}}}
     \def\rightbox#1{\makebox[0in][r]{#1}}
     \def\centbox#1{\makebox[0in]{#1}}
     \def\topbox#1{\raisebox{-\baselineskip}[0in][0in]{#1}}
     \def\midbox#1{\raisebox{-0.5\baselineskip}[0in][0in]{#1}}

\vspace{3cm}


\title{Quiz 8}
\author{S Nithish}
% make the title area
\maketitle

\newpage

%\tableofcontents

\bigskip

\renewcommand{\thefigure}{\theenumi}
\renewcommand{\thetable}{\theenumi}
%\renewcommand{\theequation}{\theenumi}


\begin{abstract}
This document contains the solution of the question from NCERT 12th standard chapter 11 exercise 11.2 problem 13
\end{abstract}

%Download all python codes 
%
%\begin{lstlisting}
%svn co https://github.com/JayatiD93/trunk/My_solution_design/codes
%\end{lstlisting}

%Download all and latex-tikz codes from 
%
%\begin{lstlisting}
%svn co https://github.com/gadepall/school/trunk/ncert/geometry/figs
%\end{lstlisting}
%
\section{Exercise 11.2}
\begin{enumerate}
\item Show that the lines $\frac{x-5}{7}=\frac{y+2}{-5}=\frac{z}{1}$ and $\frac{x}{1}=\frac{y}{2}=\frac{z}{3}$ are perpendicular to each other.
	\fi
The direction vectors of the lines are  
\begin{align}
	\vec{m_1} = \myvec{7\\-5\\1},\,
	\vec{m_2} = \myvec{1\\2\\3}.
\end{align}
Hence, 
\begin{align}
	\vec{m_1}^{\top}\vec{m_2} = \myvec{7 & -5 & 1}\myvec{1\\2\\3}
				  = 0
\end{align}


\item Find the shortest distance between the lines\\  $\overrightarrow{r}=(\hat{i}+2\hat{j}+\hat{k})+\lambda(\hat{i}-\hat{j}+\hat{k})$ and \\$\overrightarrow{r}=2\hat{i}-\hat{j}-\hat{k}+\mu(2\hat{i}+\hat{j}+2\hat{k})$
\item Find the shortest distance between the lines\\
$ \frac{x+1}{7}=\frac{y+1}{-6}=\frac{z+1}{1}$ and $ \frac{x-3}{1}=\frac{y-5}{-2}=\frac{z-7}{1}$ 
    \solution
		\iffalse
\documentclass[journal,12pt,twocolumn]{IEEEtran}
\usepackage{romannum}
\usepackage{float}
\usepackage{setspace}
\usepackage{gensymb}
\singlespacing
\usepackage[cmex10]{amsmath}
\usepackage{amsthm}
\usepackage{mathrsfs}
\usepackage{txfonts}
\usepackage{stfloats}
\usepackage{bm}
\usepackage{cite}
\usepackage{cases}
\usepackage{subfig}
\usepackage{longtable}
\usepackage{multirow}
\usepackage{enumitem}
\usepackage{mathtools}
\usepackage{steinmetz}
\usepackage{tikz}
\usepackage{circuitikz}
\usepackage{verbatim}
\usepackage{tfrupee}
\usepackage[breaklinks=true]{hyperref}
\usepackage{tkz-euclide}
\usetikzlibrary{calc,math}
\usepackage{listings}
    \usepackage{color}                                            %%
    \usepackage{array}                                            %%
    \usepackage{longtable}                                        %%
    \usepackage{calc}                                             %%
    \usepackage{multirow}                                         %%
    \usepackage{hhline}                                           %%
    \usepackage{ifthen}                                           %%
  %optionally (for landscape tables embedded in another document): %%
    \usepackage{lscape}     
\usepackage{multicol}
\usepackage{chngcntr}
\DeclareMathOperator*{\Res}{Res}
\renewcommand\thesection{\arabic{section}}
\renewcommand\thesubsection{\thesection.\arabic{subsection}}
\renewcommand\thesubsubsection{\thesubsection.\arabic{subsubsection}}

\renewcommand\thesectiondis{\arabic{section}}
\renewcommand\thesubsectiondis{\thesectiondis.\arabic{subsection}}
\renewcommand\thesubsubsectiondis{\thesubsectiondis.\arabic{subsubsection}}

% correct bad hyphenation here
\hyphenation{op-tical net-works semi-conduc-tor}
\def\inputGnumericTable{}                                 %%

\lstset{
frame=single, 
breaklines=true,
columns=fullflexible
}

\begin{document}


\newtheorem{theorem}{Theorem}[section]
\newtheorem{problem}{Problem}
\newtheorem{proposition}{Proposition}[section]
\newtheorem{lemma}{Lemma}[section]
\newtheorem{corollary}[theorem]{Corollary}
\newtheorem{example}{Example}[section]
\newtheorem{definition}[problem]{Definition}
\newcommand{\BEQA}{\begin{eqnarray}}
\newcommand{\EEQA}{\end{eqnarray}}
\newcommand{\define}{\stackrel{\triangle}{=}}

\bibliographystyle{IEEEtran}
\providecommand{\mbf}{\mathbf}
\providecommand{\pr}[1]{\ensuremath{\Pr\left(#1\right)}}
\providecommand{\qfunc}[1]{\ensuremath{Q\left(#1\right)}}
\providecommand{\sbrak}[1]{\ensuremath{{}\left[#1\right]}}
\providecommand{\lsbrak}[1]{\ensuremath{{}\left[#1\right.}}
\providecommand{\rsbrak}[1]{\ensuremath{{}\left.#1\right]}}
\providecommand{\brak}[1]{\ensuremath{\left(#1\right)}}
\providecommand{\lbrak}[1]{\ensuremath{\left(#1\right.}}
\providecommand{\rbrak}[1]{\ensuremath{\left.#1\right)}}
\providecommand{\cbrak}[1]{\ensuremath{\left\{#1\right\}}}
\providecommand{\lcbrak}[1]{\ensuremath{\left\{#1\right.}}
\providecommand{\rcbrak}[1]{\ensuremath{\left.#1\right\}}}
\theoremstyle{remark}
\newtheorem{rem}{Remark}
\newcommand{\sgn}{\mathop{\mathrm{sgn}}}
\providecommand{\abs}[1]{\left\vert#1\right\vert}
\providecommand{\res}[1]{\Res\displaylimits_{#1}} 
\providecommand{\norm}[1]{\left\lVert#1\right\rVert}
\providecommand{\mtx}[1]{\mathbf{#1}}
\providecommand{\mean}[1]{E\left[ #1 \right]}
\providecommand{\fourier}{\overset{\mathcal{F}}{ \rightleftharpoons}}
\providecommand{\system}{\overset{\mathcal{H}}{ \longleftrightarrow}}
\newcommand{\solution}{\noindent \textbf{Solution: }}
\newcommand{\cosec}{\,\text{cosec}\,}
\providecommand{\dec}[2]{\ensuremath{\overset{#1}{\underset{#2}{\gtrless}}}}
\newcommand{\myvec}[1]{\ensuremath{\begin{pmatrix}#1\end{pmatrix}}}
\newcommand{\mydet}[1]{\ensuremath{\begin{vmatrix}#1\end{vmatrix}}}
\numberwithin{equation}{subsection}
\makeatletter
\@addtoreset{figure}{problem}
\makeatother

\let\StandardTheFigure\thefigure
\let\vec\mathbf
\renewcommand{\thefigure}{\theproblem}



\def\putbox#1#2#3{\makebox[0in][l]{\makebox[#1][l]{}\raisebox{\baselineskip}[0in][0in]{\raisebox{#2}[0in][0in]{#3}}}}
     \def\rightbox#1{\makebox[0in][r]{#1}}
     \def\centbox#1{\makebox[0in]{#1}}
     \def\topbox#1{\raisebox{-\baselineskip}[0in][0in]{#1}}
     \def\midbox#1{\raisebox{-0.5\baselineskip}[0in][0in]{#1}}

\vspace{3cm}


\title{Assignment 1}
\author{Jaswanth Chowdary Madala}





% make the title area
\maketitle

\newpage

%\tableofcontents

\bigskip

\renewcommand{\thefigure}{\theenumi}
\renewcommand{\thetable}{\theenumi}

\begin{enumerate}

\textbf{Solution:}
\fi
		The givne lines can be written  in vector form  as
\begin{align}
	\vec{x} &= \myvec{1\\1\\0} + \lambda_1\myvec{2\\-1\\1},
\vec{x} = \myvec{2\\1\\-1} + \lambda_2\myvec{3\\-5\\2}\\
\implies \vec{x_1} = \myvec{1\\1\\0},\, \vec{x_2} &= \myvec{2\\1\\-1}, \,\vec{m_1} = \myvec{2\\-1\\1}, \, \vec{m_2} = \myvec{3\\-5\\2}
\end{align}
%
We first check whether the given lines are skew. The lines 
\begin{align}
\vec{x} = \vec{x_1} + \lambda_1\vec{m_1},\, \vec{x} = \vec{x_2} + \lambda_2\vec{m_2} 
\label{eq:chapters/12/11/2/e11/1}
\end{align}
intersect if
\begin{align}
\vec{M}\bm{\lambda} &= \vec{x_2} - \vec{x_1}\\
\vec{M} &\triangleq \myvec{\vec{m_1} & \vec{m_2}} \\
\bm{\lambda} &\triangleq \myvec{\lambda_1\\-\lambda_2}\\
\end{align}
Here we have,
\begin{align}
\vec{M} &= \myvec{2&3\\-1&-5\\1&2},
\vec{x_2} - \vec{x_1} &= \myvec{1\\0\\-1}
\end{align}
We check whether the equation \eqref{eq:chapters/12/11/2/e11/2} has a solution
\begin{align}
\myvec{2&3\\-1&-5\\1&2}\bm{\lambda} = \myvec{1\\0\\-1}
\label{eq:chapters/12/11/2/e11/2}
\end{align}
The augmented matrix is given by,
\begin{align}
\myvec{2&3&\vrule&1\\-1&-5&\vrule&0\\1&2&\vrule&-1}
\xleftrightarrow[R_3 \leftarrow R_3 - \frac{1}{2}R_1]{R_2 \leftarrow R_2 + \frac{1}{2}R_1}
\myvec{2&3&\vrule&1\\&&\vrule\\0&-\frac{7}{2}&\vrule&\frac{1}{2}\\&&\vrule\\0&\frac{1}{2}&\vrule&-\frac{3}{2}}\\
\xleftrightarrow{R_3 \leftarrow R_3 + 7R_2}
\myvec{2&3&\vrule&1\\&&\vrule\\0&-\frac{7}{2}&\vrule&\frac{1}{2}\\&&\vrule\\0&0&\vrule&-10}
\end{align}
The rank of the matrix is 3. So the given lines are skew.
The closest points on two skew lines defined by \eqref{eq:chapters/12/11/2/e11/1} are given by 
\begin{align}
\vec{M}^\top \vec{M}\bm{\lambda} &= \vec{M}^\top\brak{\vec{x_2}-\vec{x_1}}\\
\implies \myvec{2&-1&1\\3&-5&2} \myvec{2&3\\-1&-5\\1&2}\bm{\lambda} &= \myvec{2&-1&1\\3&-5&2} \myvec{1\\0\\-1}\\
\implies \myvec{6&13\\13&38}\bm{\lambda} &= \myvec{1\\1}
\label{eq:chapters/12/11/2/e11/3}
\end{align}
The augmented matrix of the above equation \eqref{eq:chapters/12/11/2/e11/3} is given by,
\begin{align}
\myvec{6&13&\vrule&1\\13&38&\vrule&1}
\xleftrightarrow{R_2 \leftarrow R_2 - \frac{13}{6}R_1}
\myvec{6&13&\vrule&1 \\&&\vrule\\ 0&\frac{59}{6}&\vrule&-\frac{7}{6}}\\
\xleftrightarrow{R_1 \leftarrow R_1 - \frac{78}{59}R_2}
\myvec{6&0&\vrule&\frac{150}{59} \\&&\vrule\\ 0&\frac{59}{6}&\vrule&-\frac{7}{6}}
\end{align}
So, we get
\begin{align}
\myvec{\lambda_1\\-\lambda_2} &= \myvec{\frac{25}{59}\\\\-\frac{7}{59}}
\end{align}
The closest points $\vec{A}$ on line $l_1$ and $\vec{B}$ on line $l_2$ are given by,
\begin{align}
\vec{A} &= \vec{x_1} + \lambda_1\vec{m_1}
= \frac{1}{59}\myvec{109\\34\\25}\\
\vec{B} &= \vec{x_2} + \lambda_2\vec{m_2}
= \frac{1}{59}\myvec{139\\24\\-45}
\end{align}
The minimum distance between the lines is given by,
\begin{align}
\norm{\vec{B}-\vec{A}} &= \norm{\frac{1}{59}\myvec{30\\-10\\-70}}
= \frac{10}{\sqrt{59}}
\end{align}
See Fig. 
	\ref{fig:chapters/12/11/2/e11/}.
\begin{figure}[!ht]
\centering
\includegraphics[width=\columnwidth]{chapters/12/11/2/e11/figs/skew.png}
\caption{}
	\label{fig:chapters/12/11/2/e11/}
\end{figure}

    \item Find the shortest distance between the lines whose vector equations are
    \begin{align}
        \vec{x} = \myvec{1\\2\\3} + \lambda_1\myvec{1\\-3\\2}
        \label{eq:chapters/12/11/2/16/L1}
    \end{align}
    and
    \begin{align}
        \vec{x} = \myvec{4\\5\\6} + \lambda_2\myvec{2\\3\\1}
        \label{eq:chapters/12/11/2/16/L2}
    \end{align}
    \solution
		\iffalse
\documentclass[journal,12pt,twocolumn]{IEEEtran}
\usepackage{romannum}
\usepackage{float}
\usepackage{setspace}
\usepackage{gensymb}
\singlespacing
\usepackage[cmex10]{amsmath}
\usepackage{amsthm}
\usepackage{mathrsfs}
\usepackage{txfonts}
\usepackage{stfloats}
\usepackage{bm}
\usepackage{cite}
\usepackage{cases}
\usepackage{subfig}
\usepackage{longtable}
\usepackage{multirow}
\usepackage{enumitem}
\usepackage{mathtools}
\usepackage{steinmetz}
\usepackage{tikz}
\usepackage{circuitikz}
\usepackage{verbatim}
\usepackage{tfrupee}
\usepackage[breaklinks=true]{hyperref}
\usepackage{tkz-euclide}
\usetikzlibrary{calc,math}
\usepackage{listings}
    \usepackage{color}                                            %%
    \usepackage{array}                                            %%
    \usepackage{longtable}                                        %%
    \usepackage{calc}                                             %%
    \usepackage{multirow}                                         %%
    \usepackage{hhline}                                           %%
    \usepackage{ifthen}                                           %%
  %optionally (for landscape tables embedded in another document): %%
    \usepackage{lscape}     
\usepackage{multicol}
\usepackage{chngcntr}
\DeclareMathOperator*{\Res}{Res}
\renewcommand\thesection{\arabic{section}}
\renewcommand\thesubsection{\thesection.\arabic{subsection}}
\renewcommand\thesubsubsection{\thesubsection.\arabic{subsubsection}}

\renewcommand\thesectiondis{\arabic{section}}
\renewcommand\thesubsectiondis{\thesectiondis.\arabic{subsection}}
\renewcommand\thesubsubsectiondis{\thesubsectiondis.\arabic{subsubsection}}

% correct bad hyphenation here
\hyphenation{op-tical net-works semi-conduc-tor}
\def\inputGnumericTable{}                                 %%

\lstset{
frame=single, 
breaklines=true,
columns=fullflexible
}

\begin{document}


\newtheorem{theorem}{Theorem}[section]
\newtheorem{problem}{Problem}
\newtheorem{proposition}{Proposition}[section]
\newtheorem{lemma}{Lemma}[section]
\newtheorem{corollary}[theorem]{Corollary}
\newtheorem{example}{Example}[section]
\newtheorem{definition}[problem]{Definition}
\newcommand{\BEQA}{\begin{eqnarray}}
\newcommand{\EEQA}{\end{eqnarray}}
\newcommand{\define}{\stackrel{\triangle}{=}}

\bibliographystyle{IEEEtran}
\providecommand{\mbf}{\mathbf}
\providecommand{\pr}[1]{\ensuremath{\Pr\left(#1\right)}}
\providecommand{\qfunc}[1]{\ensuremath{Q\left(#1\right)}}
\providecommand{\sbrak}[1]{\ensuremath{{}\left[#1\right]}}
\providecommand{\lsbrak}[1]{\ensuremath{{}\left[#1\right.}}
\providecommand{\rsbrak}[1]{\ensuremath{{}\left.#1\right]}}
\providecommand{\brak}[1]{\ensuremath{\left(#1\right)}}
\providecommand{\lbrak}[1]{\ensuremath{\left(#1\right.}}
\providecommand{\rbrak}[1]{\ensuremath{\left.#1\right)}}
\providecommand{\cbrak}[1]{\ensuremath{\left\{#1\right\}}}
\providecommand{\lcbrak}[1]{\ensuremath{\left\{#1\right.}}
\providecommand{\rcbrak}[1]{\ensuremath{\left.#1\right\}}}
\theoremstyle{remark}
\newtheorem{rem}{Remark}
\newcommand{\sgn}{\mathop{\mathrm{sgn}}}
\providecommand{\abs}[1]{\left\vert#1\right\vert}
\providecommand{\res}[1]{\Res\displaylimits_{#1}} 
\providecommand{\norm}[1]{\left\lVert#1\right\rVert}
\providecommand{\mtx}[1]{\mathbf{#1}}
\providecommand{\mean}[1]{E\left[ #1 \right]}
\providecommand{\fourier}{\overset{\mathcal{F}}{ \rightleftharpoons}}
\providecommand{\system}{\overset{\mathcal{H}}{ \longleftrightarrow}}
\newcommand{\solution}{\noindent \textbf{Solution: }}
\newcommand{\cosec}{\,\text{cosec}\,}
\providecommand{\dec}[2]{\ensuremath{\overset{#1}{\underset{#2}{\gtrless}}}}
\newcommand{\myvec}[1]{\ensuremath{\begin{pmatrix}#1\end{pmatrix}}}
\newcommand{\mydet}[1]{\ensuremath{\begin{vmatrix}#1\end{vmatrix}}}
\numberwithin{equation}{subsection}
\makeatletter
\@addtoreset{figure}{problem}
\makeatother

\let\StandardTheFigure\thefigure
\let\vec\mathbf
\renewcommand{\thefigure}{\theproblem}



\def\putbox#1#2#3{\makebox[0in][l]{\makebox[#1][l]{}\raisebox{\baselineskip}[0in][0in]{\raisebox{#2}[0in][0in]{#3}}}}
     \def\rightbox#1{\makebox[0in][r]{#1}}
     \def\centbox#1{\makebox[0in]{#1}}
     \def\topbox#1{\raisebox{-\baselineskip}[0in][0in]{#1}}
     \def\midbox#1{\raisebox{-0.5\baselineskip}[0in][0in]{#1}}

\vspace{3cm}


\title{Assignment 1}
\author{Jaswanth Chowdary Madala}





% make the title area
\maketitle

\newpage

%\tableofcontents

\bigskip

\renewcommand{\thefigure}{\theenumi}
\renewcommand{\thetable}{\theenumi}

\begin{enumerate}

\textbf{Solution:}
\fi
		The givne lines can be written  in vector form  as
\begin{align}
	\vec{x} &= \myvec{1\\1\\0} + \lambda_1\myvec{2\\-1\\1},
\vec{x} = \myvec{2\\1\\-1} + \lambda_2\myvec{3\\-5\\2}\\
\implies \vec{x_1} = \myvec{1\\1\\0},\, \vec{x_2} &= \myvec{2\\1\\-1}, \,\vec{m_1} = \myvec{2\\-1\\1}, \, \vec{m_2} = \myvec{3\\-5\\2}
\end{align}
%
We first check whether the given lines are skew. The lines 
\begin{align}
\vec{x} = \vec{x_1} + \lambda_1\vec{m_1},\, \vec{x} = \vec{x_2} + \lambda_2\vec{m_2} 
\label{eq:chapters/12/11/2/e11/1}
\end{align}
intersect if
\begin{align}
\vec{M}\bm{\lambda} &= \vec{x_2} - \vec{x_1}\\
\vec{M} &\triangleq \myvec{\vec{m_1} & \vec{m_2}} \\
\bm{\lambda} &\triangleq \myvec{\lambda_1\\-\lambda_2}\\
\end{align}
Here we have,
\begin{align}
\vec{M} &= \myvec{2&3\\-1&-5\\1&2},
\vec{x_2} - \vec{x_1} &= \myvec{1\\0\\-1}
\end{align}
We check whether the equation \eqref{eq:chapters/12/11/2/e11/2} has a solution
\begin{align}
\myvec{2&3\\-1&-5\\1&2}\bm{\lambda} = \myvec{1\\0\\-1}
\label{eq:chapters/12/11/2/e11/2}
\end{align}
The augmented matrix is given by,
\begin{align}
\myvec{2&3&\vrule&1\\-1&-5&\vrule&0\\1&2&\vrule&-1}
\xleftrightarrow[R_3 \leftarrow R_3 - \frac{1}{2}R_1]{R_2 \leftarrow R_2 + \frac{1}{2}R_1}
\myvec{2&3&\vrule&1\\&&\vrule\\0&-\frac{7}{2}&\vrule&\frac{1}{2}\\&&\vrule\\0&\frac{1}{2}&\vrule&-\frac{3}{2}}\\
\xleftrightarrow{R_3 \leftarrow R_3 + 7R_2}
\myvec{2&3&\vrule&1\\&&\vrule\\0&-\frac{7}{2}&\vrule&\frac{1}{2}\\&&\vrule\\0&0&\vrule&-10}
\end{align}
The rank of the matrix is 3. So the given lines are skew.
The closest points on two skew lines defined by \eqref{eq:chapters/12/11/2/e11/1} are given by 
\begin{align}
\vec{M}^\top \vec{M}\bm{\lambda} &= \vec{M}^\top\brak{\vec{x_2}-\vec{x_1}}\\
\implies \myvec{2&-1&1\\3&-5&2} \myvec{2&3\\-1&-5\\1&2}\bm{\lambda} &= \myvec{2&-1&1\\3&-5&2} \myvec{1\\0\\-1}\\
\implies \myvec{6&13\\13&38}\bm{\lambda} &= \myvec{1\\1}
\label{eq:chapters/12/11/2/e11/3}
\end{align}
The augmented matrix of the above equation \eqref{eq:chapters/12/11/2/e11/3} is given by,
\begin{align}
\myvec{6&13&\vrule&1\\13&38&\vrule&1}
\xleftrightarrow{R_2 \leftarrow R_2 - \frac{13}{6}R_1}
\myvec{6&13&\vrule&1 \\&&\vrule\\ 0&\frac{59}{6}&\vrule&-\frac{7}{6}}\\
\xleftrightarrow{R_1 \leftarrow R_1 - \frac{78}{59}R_2}
\myvec{6&0&\vrule&\frac{150}{59} \\&&\vrule\\ 0&\frac{59}{6}&\vrule&-\frac{7}{6}}
\end{align}
So, we get
\begin{align}
\myvec{\lambda_1\\-\lambda_2} &= \myvec{\frac{25}{59}\\\\-\frac{7}{59}}
\end{align}
The closest points $\vec{A}$ on line $l_1$ and $\vec{B}$ on line $l_2$ are given by,
\begin{align}
\vec{A} &= \vec{x_1} + \lambda_1\vec{m_1}
= \frac{1}{59}\myvec{109\\34\\25}\\
\vec{B} &= \vec{x_2} + \lambda_2\vec{m_2}
= \frac{1}{59}\myvec{139\\24\\-45}
\end{align}
The minimum distance between the lines is given by,
\begin{align}
\norm{\vec{B}-\vec{A}} &= \norm{\frac{1}{59}\myvec{30\\-10\\-70}}
= \frac{10}{\sqrt{59}}
\end{align}
See Fig. 
	\ref{fig:chapters/12/11/2/e11/}.
\begin{figure}[!ht]
\centering
\includegraphics[width=\columnwidth]{chapters/12/11/2/e11/figs/skew.png}
\caption{}
	\label{fig:chapters/12/11/2/e11/}
\end{figure}

\item Find the shortest distance between the lines whose vector equations are \\
 $\overrightarrow{r}=(1-t)\hat{i}+(t-2)\hat{j}+(3-2t)\hat{k}$     and  \\$\overrightarrow{r}=(s+1)\hat{i}+(2s-1)\hat{j}-(2s+1)\hat{k}$
 \item 
\item Find the shortest distance between the lines $l_1$ and $l_2$ whose vector equations are ${\overrightarrow{r} = \hat{i}+\hat{j}+\lambda(2\hat{i}-\hat{j}+\hat{k})}$ and ${\overrightarrow{r} = 2\hat{i}+\hat{j}-\hat{k}+\mu(3\hat{i}-5\hat{j}+2\hat{k})}$.
    \solution
		\iffalse
\documentclass[journal,12pt,twocolumn]{IEEEtran}
\usepackage{romannum}
\usepackage{float}
\usepackage{setspace}
\usepackage{gensymb}
\singlespacing
\usepackage[cmex10]{amsmath}
\usepackage{amsthm}
\usepackage{mathrsfs}
\usepackage{txfonts}
\usepackage{stfloats}
\usepackage{bm}
\usepackage{cite}
\usepackage{cases}
\usepackage{subfig}
\usepackage{longtable}
\usepackage{multirow}
\usepackage{enumitem}
\usepackage{mathtools}
\usepackage{steinmetz}
\usepackage{tikz}
\usepackage{circuitikz}
\usepackage{verbatim}
\usepackage{tfrupee}
\usepackage[breaklinks=true]{hyperref}
\usepackage{tkz-euclide}
\usetikzlibrary{calc,math}
\usepackage{listings}
    \usepackage{color}                                            %%
    \usepackage{array}                                            %%
    \usepackage{longtable}                                        %%
    \usepackage{calc}                                             %%
    \usepackage{multirow}                                         %%
    \usepackage{hhline}                                           %%
    \usepackage{ifthen}                                           %%
  %optionally (for landscape tables embedded in another document): %%
    \usepackage{lscape}     
\usepackage{multicol}
\usepackage{chngcntr}
\DeclareMathOperator*{\Res}{Res}
\renewcommand\thesection{\arabic{section}}
\renewcommand\thesubsection{\thesection.\arabic{subsection}}
\renewcommand\thesubsubsection{\thesubsection.\arabic{subsubsection}}

\renewcommand\thesectiondis{\arabic{section}}
\renewcommand\thesubsectiondis{\thesectiondis.\arabic{subsection}}
\renewcommand\thesubsubsectiondis{\thesubsectiondis.\arabic{subsubsection}}

% correct bad hyphenation here
\hyphenation{op-tical net-works semi-conduc-tor}
\def\inputGnumericTable{}                                 %%

\lstset{
frame=single, 
breaklines=true,
columns=fullflexible
}

\begin{document}


\newtheorem{theorem}{Theorem}[section]
\newtheorem{problem}{Problem}
\newtheorem{proposition}{Proposition}[section]
\newtheorem{lemma}{Lemma}[section]
\newtheorem{corollary}[theorem]{Corollary}
\newtheorem{example}{Example}[section]
\newtheorem{definition}[problem]{Definition}
\newcommand{\BEQA}{\begin{eqnarray}}
\newcommand{\EEQA}{\end{eqnarray}}
\newcommand{\define}{\stackrel{\triangle}{=}}

\bibliographystyle{IEEEtran}
\providecommand{\mbf}{\mathbf}
\providecommand{\pr}[1]{\ensuremath{\Pr\left(#1\right)}}
\providecommand{\qfunc}[1]{\ensuremath{Q\left(#1\right)}}
\providecommand{\sbrak}[1]{\ensuremath{{}\left[#1\right]}}
\providecommand{\lsbrak}[1]{\ensuremath{{}\left[#1\right.}}
\providecommand{\rsbrak}[1]{\ensuremath{{}\left.#1\right]}}
\providecommand{\brak}[1]{\ensuremath{\left(#1\right)}}
\providecommand{\lbrak}[1]{\ensuremath{\left(#1\right.}}
\providecommand{\rbrak}[1]{\ensuremath{\left.#1\right)}}
\providecommand{\cbrak}[1]{\ensuremath{\left\{#1\right\}}}
\providecommand{\lcbrak}[1]{\ensuremath{\left\{#1\right.}}
\providecommand{\rcbrak}[1]{\ensuremath{\left.#1\right\}}}
\theoremstyle{remark}
\newtheorem{rem}{Remark}
\newcommand{\sgn}{\mathop{\mathrm{sgn}}}
\providecommand{\abs}[1]{\left\vert#1\right\vert}
\providecommand{\res}[1]{\Res\displaylimits_{#1}} 
\providecommand{\norm}[1]{\left\lVert#1\right\rVert}
\providecommand{\mtx}[1]{\mathbf{#1}}
\providecommand{\mean}[1]{E\left[ #1 \right]}
\providecommand{\fourier}{\overset{\mathcal{F}}{ \rightleftharpoons}}
\providecommand{\system}{\overset{\mathcal{H}}{ \longleftrightarrow}}
\newcommand{\solution}{\noindent \textbf{Solution: }}
\newcommand{\cosec}{\,\text{cosec}\,}
\providecommand{\dec}[2]{\ensuremath{\overset{#1}{\underset{#2}{\gtrless}}}}
\newcommand{\myvec}[1]{\ensuremath{\begin{pmatrix}#1\end{pmatrix}}}
\newcommand{\mydet}[1]{\ensuremath{\begin{vmatrix}#1\end{vmatrix}}}
\numberwithin{equation}{subsection}
\makeatletter
\@addtoreset{figure}{problem}
\makeatother

\let\StandardTheFigure\thefigure
\let\vec\mathbf
\renewcommand{\thefigure}{\theproblem}



\def\putbox#1#2#3{\makebox[0in][l]{\makebox[#1][l]{}\raisebox{\baselineskip}[0in][0in]{\raisebox{#2}[0in][0in]{#3}}}}
     \def\rightbox#1{\makebox[0in][r]{#1}}
     \def\centbox#1{\makebox[0in]{#1}}
     \def\topbox#1{\raisebox{-\baselineskip}[0in][0in]{#1}}
     \def\midbox#1{\raisebox{-0.5\baselineskip}[0in][0in]{#1}}

\vspace{3cm}


\title{Assignment 1}
\author{Jaswanth Chowdary Madala}





% make the title area
\maketitle

\newpage

%\tableofcontents

\bigskip

\renewcommand{\thefigure}{\theenumi}
\renewcommand{\thetable}{\theenumi}

\begin{enumerate}

\textbf{Solution:}
\fi
		The givne lines can be written  in vector form  as
\begin{align}
	\vec{x} &= \myvec{1\\1\\0} + \lambda_1\myvec{2\\-1\\1},
\vec{x} = \myvec{2\\1\\-1} + \lambda_2\myvec{3\\-5\\2}\\
\implies \vec{x_1} = \myvec{1\\1\\0},\, \vec{x_2} &= \myvec{2\\1\\-1}, \,\vec{m_1} = \myvec{2\\-1\\1}, \, \vec{m_2} = \myvec{3\\-5\\2}
\end{align}
%
We first check whether the given lines are skew. The lines 
\begin{align}
\vec{x} = \vec{x_1} + \lambda_1\vec{m_1},\, \vec{x} = \vec{x_2} + \lambda_2\vec{m_2} 
\label{eq:chapters/12/11/2/e11/1}
\end{align}
intersect if
\begin{align}
\vec{M}\bm{\lambda} &= \vec{x_2} - \vec{x_1}\\
\vec{M} &\triangleq \myvec{\vec{m_1} & \vec{m_2}} \\
\bm{\lambda} &\triangleq \myvec{\lambda_1\\-\lambda_2}\\
\end{align}
Here we have,
\begin{align}
\vec{M} &= \myvec{2&3\\-1&-5\\1&2},
\vec{x_2} - \vec{x_1} &= \myvec{1\\0\\-1}
\end{align}
We check whether the equation \eqref{eq:chapters/12/11/2/e11/2} has a solution
\begin{align}
\myvec{2&3\\-1&-5\\1&2}\bm{\lambda} = \myvec{1\\0\\-1}
\label{eq:chapters/12/11/2/e11/2}
\end{align}
The augmented matrix is given by,
\begin{align}
\myvec{2&3&\vrule&1\\-1&-5&\vrule&0\\1&2&\vrule&-1}
\xleftrightarrow[R_3 \leftarrow R_3 - \frac{1}{2}R_1]{R_2 \leftarrow R_2 + \frac{1}{2}R_1}
\myvec{2&3&\vrule&1\\&&\vrule\\0&-\frac{7}{2}&\vrule&\frac{1}{2}\\&&\vrule\\0&\frac{1}{2}&\vrule&-\frac{3}{2}}\\
\xleftrightarrow{R_3 \leftarrow R_3 + 7R_2}
\myvec{2&3&\vrule&1\\&&\vrule\\0&-\frac{7}{2}&\vrule&\frac{1}{2}\\&&\vrule\\0&0&\vrule&-10}
\end{align}
The rank of the matrix is 3. So the given lines are skew.
The closest points on two skew lines defined by \eqref{eq:chapters/12/11/2/e11/1} are given by 
\begin{align}
\vec{M}^\top \vec{M}\bm{\lambda} &= \vec{M}^\top\brak{\vec{x_2}-\vec{x_1}}\\
\implies \myvec{2&-1&1\\3&-5&2} \myvec{2&3\\-1&-5\\1&2}\bm{\lambda} &= \myvec{2&-1&1\\3&-5&2} \myvec{1\\0\\-1}\\
\implies \myvec{6&13\\13&38}\bm{\lambda} &= \myvec{1\\1}
\label{eq:chapters/12/11/2/e11/3}
\end{align}
The augmented matrix of the above equation \eqref{eq:chapters/12/11/2/e11/3} is given by,
\begin{align}
\myvec{6&13&\vrule&1\\13&38&\vrule&1}
\xleftrightarrow{R_2 \leftarrow R_2 - \frac{13}{6}R_1}
\myvec{6&13&\vrule&1 \\&&\vrule\\ 0&\frac{59}{6}&\vrule&-\frac{7}{6}}\\
\xleftrightarrow{R_1 \leftarrow R_1 - \frac{78}{59}R_2}
\myvec{6&0&\vrule&\frac{150}{59} \\&&\vrule\\ 0&\frac{59}{6}&\vrule&-\frac{7}{6}}
\end{align}
So, we get
\begin{align}
\myvec{\lambda_1\\-\lambda_2} &= \myvec{\frac{25}{59}\\\\-\frac{7}{59}}
\end{align}
The closest points $\vec{A}$ on line $l_1$ and $\vec{B}$ on line $l_2$ are given by,
\begin{align}
\vec{A} &= \vec{x_1} + \lambda_1\vec{m_1}
= \frac{1}{59}\myvec{109\\34\\25}\\
\vec{B} &= \vec{x_2} + \lambda_2\vec{m_2}
= \frac{1}{59}\myvec{139\\24\\-45}
\end{align}
The minimum distance between the lines is given by,
\begin{align}
\norm{\vec{B}-\vec{A}} &= \norm{\frac{1}{59}\myvec{30\\-10\\-70}}
= \frac{10}{\sqrt{59}}
\end{align}
See Fig. 
	\ref{fig:chapters/12/11/2/e11/}.
\begin{figure}[!ht]
\centering
\includegraphics[width=\columnwidth]{chapters/12/11/2/e11/figs/skew.png}
\caption{}
	\label{fig:chapters/12/11/2/e11/}
\end{figure}

\end{enumerate}

