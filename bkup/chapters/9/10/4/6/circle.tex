\iffalse
\documentclass[journal,12pt,twocolumn]{IEEEtran}
\usepackage{setspace}
\usepackage{gensymb}
\singlespacing
\usepackage[cmex10]{amsmath}
\usepackage{amsthm}
\usepackage{mathrsfs}
\usepackage{txfonts}
\usepackage{stfloats}
\usepackage{bm}
\usepackage{cite}
\usepackage{cases}
\usepackage{subfig}
\usepackage{longtable}
\usepackage{multirow}
\usepackage{enumitem}
\usepackage{mathtools}
\usepackage{tikz}
\usepackage{circuitikz}
\usepackage{verbatim}
\usepackage[breaklinks=true]{hyperref}
\usepackage{tkz-euclide} % loads  TikZ and tkz-base
\usepackage{listings}
\usepackage{color}    
\usepackage{array}    
\usepackage{longtable}
\usepackage{calc}     
\usepackage{multirow} 
\usepackage{hhline}   
\usepackage{ifthen}   
\usepackage{lscape}     
\usepackage{chngcntr}
\DeclareMathOperator*{\Res}{Res}
\renewcommand\thesection{\arabic{section}}
\renewcommand\thesubsection{\thesection.\arabic{subsection}}
\renewcommand\thesubsubsection{\thesubsection.\arabic{subsubsection}}

\renewcommand\thesectiondis{\arabic{section}}
\renewcommand\thesubsectiondis{\thesectiondis.\arabic{subsection}}
\renewcommand\thesubsubsectiondis{\thesubsectiondis.\arabic{subsubsection}}
\renewcommand\thetable{\arabic{table}}
% correct bad hyphenation here
\hyphenation{op-tical net-works semi-conduc-tor}
\def\inputGnumericTable{}                                 %%

\lstset{
%language=C,
frame=single, 
breaklines=true,
columns=fullflexible
}
%\lstset{
%language=tex,
%frame=single, 
%breaklines=true
%}

\begin{document}
\newtheorem{theorem}{Theorem}[section]
\newtheorem{problem}{Problem}
\newtheorem{proposition}{Proposition}[section]
\newtheorem{lemma}{Lemma}[section]
\newtheorem{corollary}[theorem]{Corollary}
\newtheorem{example}{Example}[section]
\newtheorem{definition}[problem]{Definition}
\newcommand{\BEQA}{\begin{eqnarray}}
\newcommand{\EEQA}{\end{eqnarray}}
\newcommand{\define}{\stackrel{\triangle}{=}}
\bibliographystyle{IEEEtran}
\providecommand{\mbf}{\mathbf}
\providecommand{\pr}[1]{\ensuremath{\Pr\left(#1\right)}}
\providecommand{\qfunc}[1]{\ensuremath{Q\left(#1\right)}}
\providecommand{\sbrak}[1]{\ensuremath{{}\left[#1\right]}}
\providecommand{\lsbrak}[1]{\ensuremath{{}\left[#1\right.}}
\providecommand{\rsbrak}[1]{\ensuremath{{}\left.#1\right]}}
\providecommand{\brak}[1]{\ensuremath{\left(#1\right)}}
\providecommand{\lbrak}[1]{\ensuremath{\left(#1\right.}}
\providecommand{\rbrak}[1]{\ensuremath{\left.#1\right)}}
\providecommand{\cbrak}[1]{\ensuremath{\left\{#1\right\}}}
\providecommand{\lcbrak}[1]{\ensuremath{\left\{#1\right.}}
\providecommand{\rcbrak}[1]{\ensuremath{\left.#1\right\}}}
\theoremstyle{remark}
\newtheorem{rem}{Remark}
\newcommand{\sgn}{\mathop{\mathrm{sgn}}}
\providecommand{\abs}[1]{\left\vert#1\right\vert}
\providecommand{\res}[1]{\Res\displaylimits_{#1}} 
\providecommand{\norm}[1]{\left\lVert#1\right\rVert}
\providecommand{\mtx}[1]{\mathbf{#1}}
\providecommand{\mean}[1]{E\left[ #1 \right]}
\providecommand{\fourier}{\overset{\mathcal{F}}{ \rightleftharpoons}}
\providecommand{\system}[1]{\overset{\mathcal{#1}}{ \longleftrightarrow}}
\newcommand{\solution}{\noindent \textbf{Solution: }}
\newcommand{\cosec}{\,\text{cosec}\,}
\providecommand{\dec}[2]{\ensuremath{\overset{#1}{\underset{#2}{\gtrless}}}}
\newcommand{\myvec}[1]{\ensuremath{\begin{pmatrix}#1\end{pmatrix}}}
\newcommand{\mydet}[1]{\ensuremath{\begin{vmatrix}#1\end{vmatrix}}}
\let\vec\mathbf
\def\putbox#1#2#3{\makebox[0in][l]{\makebox[#1][l]{}\raisebox{\baselineskip}[0in][0in]{\raisebox{#2}[0in][0in]{#3}}}}
     \def\rightbox#1{\makebox[0in][r]{#1}}
     \def\centbox#1{\makebox[0in]{#1}}
     \def\topbox#1{\raisebox{-\baselineskip}[0in][0in]{#1}}
     \def\midbox#1{\raisebox{-0.5\baselineskip}[0in][0in]{#1}}

\vspace{3cm}
\title{Circle Assignment}
\author{Gautam Singh}
\maketitle
\bigskip

\begin{abstract}
    This document contains the solution to Question 6 of 
    Exercise 4 in Chapter 10 of the class 9 NCERT textbook.
\end{abstract}

\begin{enumerate}
    \item A circular park of radius 20 m is situated in a colony. Three boys 
    Ankur, Syed and David are sitting at equal distance on its boundary each 
    having a toy telephone in his hands to talk each other. Find the length of 
    the string of each phone.

    \solution 
    \fi
		Let the position vectors of the boys be
    \begin{align}
        \vec{A} = \myvec{r\\0},\ \vec{S} = \myvec{r\cos\beta\\r\sin\beta},\ \vec{D} = \myvec{r\cos\gamma\\r\sin\gamma}
        \label{eq:chapters/9/10/4/6/pos-vec-def}
    \end{align}
    where
    \begin{align}
        \beta, \gamma \in \brak{0,2\pi}
        \label{eq:chapters/9/10/4/6/int}
    \end{align}
    We have,
    \begin{align}
        \norm{\vec{A}-\vec{S}}^2 &= \norm{\vec{A}-\vec{D}}^2 \\
        \implies \vec{A}^\top\vec{S} &= \vec{A}^\top\vec{D} \\
        \implies \cos\beta &= \cos\gamma \\
        \implies \beta = 2n\pi \pm \gamma
        \label{eq:chapters/9/10/4/6/bg-gen}
    \end{align}
    where $n \in \mathbb{Z}$. From \eqref{eq:chapters/9/10/4/6/int}, we force $n = 1$
    and consider the negative sign to get
    \begin{align}
        \beta+\gamma = 2\pi
        \label{eq:chapters/9/10/4/6/bg-sum}
    \end{align}
    Therefore, using \eqref{eq:chapters/9/10/4/6/bg-sum}
    \begin{align}
        \norm{\vec{A}-\vec{S}}^2 &= \norm{\vec{S}-\vec{D}}^2 \\
        \implies \vec{A}^\top\vec{S} &= \vec{D}^\top\vec{S} \\
        \implies \cos\beta &= \cos\brak{\beta-\gamma} \\
        \implies 2\beta-2\pi &= 2m\pi\pm\beta \\
        \implies 2\beta\pm\beta &= 2k\pi
        \label{eq:chapters/9/10/4/6/beta-eqn}
    \end{align}
    where $k \in \mathbb{Z}$. From \eqref{eq:chapters/9/10/4/6/int}, we can only consider
    the plus sign and $k \in \cbrak{1,2}$ to get
    \begin{align}
        \beta,\gamma \in \cbrak{\frac{2\pi}{3},\frac{4\pi}{3}}
        \label{eq:chapters/9/10/4/6/bg-sol}
    \end{align}
    Therefore, the length of the thread from \eqref{eq:chapters/9/10/4/6/bg-sol} is
    \begin{align}
        \norm{\vec{S}-\vec{D}} &= \norm{r\myvec{\cos\beta-\cos\gamma\\\sin\beta-\sin\gamma}} \\
                               &= r\sqrt{3}
    \end{align}
    Here, $r = 20$ m. Thus, the length is $20\sqrt{3}$ m.

    The situation is demonstrated in Fig. \ref{fig:chapters/9/10/4/6/equilateral}, plotted by the 
    Python code \texttt{codes/equilateral.py}. The values used for 
    construction are shown in Table \ref{tab:chapters/9/10/4/6/param}.
    \begin{table}[!ht]
        \centering
        \input{chapters/9/10/4/6/tables/09.10.04.06_1.tex}
        \caption{Parameters used in the construction of Fig. \ref{fig:chapters/9/10/4/6/equilateral}.}
        \label{tab:chapters/9/10/4/6/param}
    \end{table}
    \begin{figure}[!ht]
        \centering
        \includegraphics[width=\columnwidth]{chapters/9/10/4/6/figs/equilateral.png}
        \caption{$ASD$ is an equilateral triangle of side $20\sqrt{3}$ m.}
        \label{fig:chapters/9/10/4/6/equilateral}
    \end{figure}
