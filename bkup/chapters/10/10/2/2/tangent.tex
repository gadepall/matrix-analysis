\iffalse
\documentclass[12pt]{article}
\usepackage{graphicx}
\usepackage{amsmath}
\usepackage{mathtools}
\usepackage{gensymb}
\usepackage{tabularx}
\usepackage{array}
\usepackage[latin1]{inputenc}
\usepackage{fullpage}
\usepackage{color}
\usepackage{array}
\usepackage{longtable}
\usepackage{calc}
\usepackage{multirow}
\usepackage{hhline}
\usepackage{ifthen}
\usepackage{lscape}
\usepackage{float}

\newcommand{\mydet}[1]{\ensuremath{\begin{vmatrix}#1\end{vmatrix}}}
\providecommand{\brak}[1]{\ensuremath{\left(#1\right)}}
\providecommand{\norm}[1]{\left\lVert#1\right\rVert}
\providecommand{\abs}[1]{\left\vert#1\right\vert}
\newcommand{\solution}{\noindent \textbf{Solution: }}
\newcommand{\myvec}[1]{\ensuremath{\begin{pmatrix}#1\end{pmatrix}}}
\let\vec\mathbf

\def\inputGnumericTable{}

\begin{document}
\begin{center}
\textbf\large{TANGENTS AND NORMALS}

\end{center}
\section*{Excercise 10.2}

\solution
\fi
Let the output angle be $\phi$.
The input parameters are given as
\begin{tabular}{|c|c|c|}
  \hline
  \textbf{Symbol}&\textbf{Value}&\textbf{Description}\\
  \hline
  $a$ & 8 & $BC$\\
  \hline
	$\angle{B}$ & 45$\degree{}$ & $\angle{B}$ in $\triangle$$ABC$ \\
  \hline
	$k$ & 3.5 & $AB-AC$ i.e $c-b$ \\
  \hline 
	$\vec{e_2}$ & $\myvec{
			0\\
			1\\
			}$ & Basis vector\\
 \hline			
\end{tabular}

Any point $\vec{X}$ on the circle is given as
\begin{align}
	\vec{X} = \vec{O}+r\myvec{\cos\theta\\\sin\theta}
\end{align}
So points $\vec{P} \text{ and } \vec{Q}$ can be calculated as
\begin{align}
	\vec{P} &= \vec{O}+\myvec{\cos\theta\\\sin\theta} = \myvec{\cos\theta\\\sin\theta}\\
	\vec{Q} &= \vec{e}_1
\end{align}
For tangent $TP$
\begin{align}
	\vec{n}_1 &= \vec{P}-\vec{O}\\
	&= \myvec{\cos\theta\\\sin\theta} =  \myvec{1\\\tan\theta}\\
	\vec{m}_1 &= \myvec{1\\-\cot\theta}
\end{align}
For tangent $TQ$
\begin{align}
	\vec{n}_2 &= \vec{e}_1-\vec{O}\\
	&= \vec{e}_1\\
	\vec{m}_2 &= \vec{e}_2
\end{align}
The equation of $TP$ is given as
\begin{align}
	\vec{n}_1^\top\brak{\vec{x}-\vec{P}} &= 0\\
	\vec{n}_1^\top\brak{\vec{x}-\myvec{\cos\theta\\\sin\theta}} &= 0\\
	\label{eq:chapters/10/10/2/2/eq1}
	\myvec{\cos\theta & \sin\theta}\vec{x} &= 1
\end{align}
The equation of $TQ$ is given as
\begin{align}
	\vec{n}_2^\top\brak{\vec{x}-\vec{e}_1} &= 0\\
	\label{eq:chapters/10/10/2/2/eq2}
	\myvec{1&0}\vec{x} &= 1
\end{align}
The tangent point can be calculated by solving \eqref{eq:chapters/10/10/2/2/eq1} and \eqref{eq:chapters/10/10/2/2/eq2}
\begin{align}
	\myvec{\cos\theta&\sin\theta\\1&0}\myvec{x\\y} &= \myvec{1\\1}\\
	\label{eq:chapters/10/10/2/2/eq3}
	\implies \myvec{x\\y} &= \myvec{1\\\tan{\frac{\theta}{2}}}
\end{align}
Now, $\vec{T}=$\eqref{eq:chapters/10/10/2/2/eq3}, since it is the intersection of $TP \text{ and } TQ$. Hence, it is given as
\begin{align}
	\vec{T} = \myvec{1\\\tan{55}\degree} = \myvec{1\\1.428}	
\end{align}
The angle between two lines with slope $\vec{m}_1 \text{ and } \vec{m}_2$ is given as
\begin{align}
	\cos\phi &= \frac{\vec{m}_1^\top\vec{m}_2}{\norm{\vec{m}_1}\norm{\vec{m}_2}}\\
	&= \frac{\myvec{1&-\cot\theta}\myvec{0\\1}}{\brak{\csc\theta}\brak{1}}\\
	&= -\cos\theta\\
	\implies \cos\phi &= -\cos\theta
\end{align}
Hence,
\begin{align}
	\phi &= \cos^{-1}\brak{\cos{\brak{180\degree-\theta}}}\\
	     &= 180\degree-\theta = 70\degree
\end{align}
Hence, $\angle{PTQ} = 70\degree$. See Fig \ref{fig:chapters/10/10/2/2/Fig1}
\begin{figure}[!h]
	\begin{center} 
	    \includegraphics[width=\columnwidth]{chapters/10/10/2/2/figs/tangent2}
	\end{center}
\caption{}
\label{fig:chapters/10/10/2/2/Fig1}
\end{figure}


















