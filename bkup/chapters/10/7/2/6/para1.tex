\iffalse
\documentclass[12pt]{article}
\usepackage{graphicx}
%\documentclass[journal,12pt,twocolumn]{IEEEtran}
\def\inputGnumericTable{}
\usepackage{color}                                            %%
    \usepackage{array}                                            %%
    \usepackage{longtable}                                        %%
    \usepackage{calc}                                             %%
    \usepackage{multirow}                                         %%
    \usepackage{hhline}                                           %%
    \usepackage{ifthen}
\usepackage[none]{hyphenat}
\usepackage{graphicx}
\usepackage{listings}
\usepackage[english]{babel}
\usepackage{graphicx}
\usepackage{caption} 
\usepackage{hyperref}
\usepackage{booktabs}
\usepackage{array}
\usepackage{amsmath}   % for having text in math mode
\usepackage{listings}
\lstset{
  frame=single,
  breaklines=true
}
  
%Following 2 lines were added to remove the blank page at the beginning
\usepackage{atbegshi}% http://ctan.org/pkg/atbegshi
\AtBeginDocument{\AtBeginShipoutNext{\AtBeginShipoutDiscard}}
%


%New macro definitions
\newcommand{\mydet}[1]{\ensuremath{\begin{vmatrix}#1\end{vmatrix}}}
\providecommand{\brak}[1]{\ensuremath{\left(#1\right)}}
\providecommand{\norm}[1]{\left\lVert#1\right\rVert}
\newcommand{\solution}{\noindent \textbf{Solution: }}
\newcommand{\myvec}[1]{\ensuremath{\begin{pmatrix}#1\end{pmatrix}}}
\let\vec\mathbf

\begin{document}

\begin{center}
\title{\textbf{Properties of Parallelegram}}
\date{\vspace{-5ex}} %Not to print date automatically
\maketitle
\end{center}

\setcounter{page}{1}

\section{10$^{th}$ Maths - Chapter 7}

This is Problem-6 from Exercise 7.2

\begin{enumerate}
\item If $\vec{A}(1, 2),\vec{B}(4, x),\vec{C}(y, 6) \text{and } \vec{D}(3, 5)$ are the vertices of a parallelogram taken in order,find x and y.
\end{enumerate}
\fi

The input parameters for this problem are available in
\ref{table:chapters/10/7/2/6/tables/}.	
\begin{table}[!ht]
	\centering
	\begin{tabular}{|c|c|p{5cm}|}
\hline
\textbf{Symbol} & \textbf{Value} & \textbf{Description} \\
\hline
$\theta$ & $30\degree$ & $\angle{BAP} = \angle{BAQ}$ \\
\hline
$a$ & $9$ & $AB$ \\
\hline
$c$ & $8$ & $AQ$ \\
\hline
$\vec{e}_1$ & $\myvec{1\\0}$ & Basis vector \\
\hline
\end{tabular}

\caption{}
\label{table:chapters/10/7/2/6/tables/}	
\end{table}
From the given information,
\begin{align}
  \label{eq:chapters/10/7/2/6/tables/det2f}
	\vec{B}-\vec{A} &= \myvec{4 \\y } - \myvec{1 \\2 }  = \myvec{3 \\y-2 }\\
	\vec{C}-\vec{D} &= \myvec{x \\6 } - \myvec{3 \\5 }  = \myvec{x-3 \\1}
\end{align}
Since $ABCD$ is a parallellogram,
\begin{align}
	\myvec{3\\y-2}&=\myvec{x-3\\1}\\
	\implies x&=6 ,y=3
\end{align}
Fig. \ref{fig:chapters/10/7/2/6/Fig3}
provides a verification.
\begin{figure}[h!]
	\begin{center}
  \includegraphics[width=\columnwidth]{chapters/10/7/2/6/figs/para.pdf}
	\end{center}
\caption{}
\label{fig:chapters/10/7/2/6/Fig3}
\end{figure}

