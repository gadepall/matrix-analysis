\iffalse
\documentclass[12pt]{article}
\usepackage{graphicx}
%\documentclass[journal,12pt,twocolumn]{IEEEtran}
\usepackage[none]{hyphenat}
\usepackage{graphicx}
\usepackage{listings}
\usepackage[english]{babel}
\usepackage{graphicx}
\usepackage{caption} 
\usepackage{hyperref}
\usepackage{booktabs}
\def\inputGnumericTable{}
\usepackage{color}                                            %%
    \usepackage{array}                                            %%
    \usepackage{longtable}                                        %%
    \usepackage{calc}                                             %%
    \usepackage{multirow}                                         %%
    \usepackage{hhline}                                           %%
    \usepackage{ifthen}
\usepackage{array}
\usepackage{amsmath}   % for having text in math mode
\usepackage{listings}
\lstset{
language=tex,
frame=single, 
breaklines=true
}
  
%Following 2 lines were added to remove the blank page at the beginning
\usepackage{atbegshi}% http://ctan.org/pkg/atbegshi
\AtBeginDocument{\AtBeginShipoutNext{\AtBeginShipoutDiscard}}
%
%New macro definitions
\newcommand{\mydet}[1]{\ensuremath{\begin{vmatrix}#1\end{vmatrix}}}
\providecommand{\brak}[1]{\ensuremath{\left(#1\right)}}
\providecommand{\norm}[1]{\left\lVert#1\right\rVert}
\newcommand{\solution}{\noindent \textbf{Solution: }}
\newcommand{\myvec}[1]{\ensuremath{\begin{pmatrix}#1\end{pmatrix}}}
\let\vec\mathbf
\begin{document}
\begin{center}
\title{\textbf{Coordinate Geometry}}
\date{\vspace{-5ex}} %Not to print date automatically
\maketitle
\end{center}
\setcounter{page}{1}
\section*{10$^{th}$ Maths - Chapter 7}
This is Problem-4 from Exercise 7.3
\begin{enumerate}
\item Find the area of quadrilateral whose vertices, taken in order, are $\myvec{-4 \\ -2}, \myvec{-3\\-5}, \myvec{3\\-2}$ and $\myvec{2\\3}$.\\
	\solution 
\fi
		The input parameters for this problem are available in Table \eqref{tab:10/7/3/4}.
\begin{table}[ht!]\centering
\begin{tabular}{|c|c|p{5cm}|}
\hline
\textbf{Symbol} & \textbf{Value} & \textbf{Description} \\
\hline
$\theta$ & $30\degree$ & $\angle{BAP} = \angle{BAQ}$ \\
\hline
$a$ & $9$ & $AB$ \\
\hline
$c$ & $8$ & $AQ$ \\
\hline
$\vec{e}_1$ & $\myvec{1\\0}$ & Basis vector \\
\hline
\end{tabular}

\caption{}
\label{tab:10/7/3/4}	
\end{table}
By joining $\vec{B}$ to $\vec{D}$, two triangles $\vec{A}\vec{B}\vec{D}$ and $\vec{B}\vec{C}\vec{D}$ are obtained.
Since
\begin{align}
	\vec{A}- \vec{B} &= \myvec{-4\\-2\\}-\myvec{-3\\-5\\}=\myvec{-1\\3\\}\label{eq:chapters/10/7/3/4/2}\\
	  \vec{A}- \vec{D} &= \myvec{-4\\-2\\}-\myvec{2\\3\\}=\myvec{-6\\-5\\}\label{eq:chapters/10/7/3/4/3}
  \end{align}
  \begin{align}
	  ar(ABD)&=\frac{1}{2} \norm{\brak{\vec{A}-\vec{B}}  \times 
   \brak{\vec{A}- \vec{D}}} \label{eq:chapters/10/7/3/4/1} 
   \\
	  &=	\frac{1}{2}\mydet{-1 & 3\\-6 & -5}  
	=	\frac{23}{2}
\end{align}
upon substituting the values of \eqref{eq:chapters/10/7/3/4/2} and \eqref{eq:chapters/10/7/3/4/3} in \eqref{eq:chapters/10/7/3/4/1}.
Similarly,
\begin{align}
	\vec{B}- \vec{C} &= \myvec{-3\\-5\\}-\myvec{3\\-2\\}=\myvec{-6\\-5\\}\label{eq:chapters/10/7/3/4/6} \\
	  \vec{B}- \vec{D} &= \myvec{-3\\-5\\}-\myvec{2\\3\\}=\myvec{-3\\-8\\}\label{eq:chapters/10/7/3/4/7} 
  \end{align}
  yielding
  \begin{align}
	  ar(BCD)&=\frac{1}{2} \norm{\brak{\vec{B}-\vec{C}}  \times 
   \brak{\vec{B}- \vec{D}}} \label{eq:chapters/10/7/3/4/5}
   \\
	  &=	\frac{1}{2}\mydet{-6 & -3\\-5 & -8}  
	=	\frac{33}{2}
\end{align}
	upon 	substituting the values of \eqref{eq:chapters/10/7/3/4/6} and \eqref{eq:chapters/10/7/3/4/7} in \eqref{eq:chapters/10/7/3/4/5}
		Thus, 
\begin{align}
	ar(ABCD)&=  ar(ABD) +  ar(BCD)
	= 28
\end{align}
See Fig. 
\ref{fig:chapters/10/7/3/4/Fig1}
\begin{figure}[!h]
 \begin{center}
  \includegraphics[width=\columnwidth]{chapters/10/7/3/4/figs/fig.pdf}
 \end{center}
\caption{}
\label{fig:chapters/10/7/3/4/Fig1}
\end{figure}
