\renewcommand{\theequation}{\theenumi}
\begin{enumerate}[label=\thesection.\arabic*.,ref=\thesection.\theenumi]
\numberwithin{equation}{enumi}

\item 
{\em Secant: }The points of intersection of the line 
\begin{align}
L: \quad \vec{x} = \vec{q} + \mu \vec{m} \quad \mu \in \mathbb{R}
\label{eq:conic_tangent}
\end{align}
with the conic section in \eqref{eq:conic_quad_form} are given by
\begin{align}
\vec{x}_i = \vec{q} + \mu_i \vec{m}
\end{align}
%
where
\begin{multline}
\mu_i = \frac{1}
{
\vec{m}^T\vec{V}\vec{m}
}
\lbrak{-\vec{m}^T\brak{\vec{V}\vec{q}+\vec{u}}}
\\
\pm
{\small
\rbrak{\sqrt{
\sbrak{
\vec{m}^T\brak{\vec{V}\vec{q}+\vec{u}}
}^2
-
\brak
{
\vec{q}^T\vec{V}\vec{q} + 2\vec{u}^T\vec{q} +f
}
\brak{\vec{m}^T\vec{V}\vec{m}}
}
}
}
\label{eq:tangent_roots}
\end{multline}
\solution
Substituting \eqref{eq:conic_tangent}
in \eqref{eq:conic_quad_form}, 
\begin{multline}
\brak{\vec{q} + \mu \vec{m}}^T\vec{V}\brak{\vec{q} + \mu \vec{m}}  + 2 \vec{u}^T\brak{\vec{q} + \mu \vec{m}}+f = 0
\\
\implies \mu^2\vec{m}^T\vec{V}\vec{m} + 2 \mu\vec{m}^T\brak{\vec{V}\vec{q}+\vec{u}} 
\\
+ \vec{q}^T\vec{V}\vec{q} + 2\vec{u}^T\vec{q} +f = 0
\label{eq:conic_intercept}
\end{multline}
Solving the above quadratic in \eqref{eq:conic_intercept}
yields \eqref{eq:tangent_roots}.
\item 
{\em Tangent: } If $L$ in \eqref{eq:conic_tangent} touches \eqref{eq:conic_quad_form} at exactly one point $\vec{q}$, 
\begin{align}
\vec{m}^T\brak{\vec{V}\vec{q}+\vec{u}} = 0
\label{eq:conic_tangent_mq}
\end{align}
\solution  In this case, \eqref{eq:conic_intercept} has exactly one root.  Hence, 
in \eqref{eq:tangent_roots}
\begin{multline}
\sbrak{
\vec{m}^T\brak{\vec{V}\vec{q}+\vec{u}}
}^2                                                                                                                    \\
-\brak{\vec{m}^T\vec{V}\vec{m}}
\brak
{
\vec{q}^T\vec{V}\vec{q} + 2\vec{u}^T\vec{q} +f
} = 0                                                                                             
\label{eq:conic_tangent_disc}
\end{multline}                    
$\because \vec{q}$ is the point of contact, $\vec{q}$ satisfies \eqref{eq:conic_quad_form}
and 
\begin{align}
\vec{q}^T\vec{V}\vec{q} + 2\vec{u}^T\vec{q} +f = 0
\label{eq:conic_tangent_qquad}
\end{align}
Substituting \eqref{eq:conic_tangent_qquad} in \eqref{eq:conic_tangent_disc} and simplifying, we obtain \eqref{eq:conic_tangent_mq}.
\item 
The normal vector is obtained from \eqref{eq:conic_tangent_mq} and \eqref{eq:line_dir_norm}
as
%
\begin{align}
\label{eq:conic_normal_vec}
\vec{n} = \vec{V}\vec{q}+\vec{u}
\end{align}
\item 
Given the point of contact $\vec{q}$, the equation of a tangent is 
\begin{align}
\brak{\vec{V}\vec{q}+\vec{u}}^T\vec{x}+\vec{u}^T\vec{q}+f = 0
\label{eq:conic_tangent_final}
\end{align}
\solution
 From \eqref{eq:conic_normal_vec} and \eqref{eq:line_norm_eq}, the equation of the tangent is\begin{align}
\brak{\vec{V}\vec{q}+\vec{u}}^T\brak{\vec{x}-\vec{q}} &=0
\\
\implies \brak{\vec{V}\vec{q}+\vec{u}}^T\vec{x}-\vec{q}^T\vec{V}\vec{q}-\vec{u}^T\vec{q} &= 0
\end{align}
which, upon substituting from \eqref{eq:conic_tangent_qquad} and simplifying yields \eqref{eq:conic_tangent}.
\item 
If $\vec{V}^{-1}$ exists, given the normal vector $\vec{n}$, the tangent points of contact to \eqref{eq:conic_quad_form} are given by
\begin{align}
\vec{q}_i &= \vec{V}^{-1}\brak{\kappa_i \vec{n}-\vec{u}}, i = 1,2
\\
\text{where }\kappa_i &= \pm \sqrt{
\frac{
\vec{u}^T\vec{V}^{-1}\vec{u}-f
}
{
\vec{n}^T\vec{V}^{-1}\vec{n}
}
}
\label{eq:conic_tangent_qk}
\end{align}
\solution  From \eqref{eq:conic_normal_vec},
\begin{align}
\label{eq:conic_normal_vec_q}
 \vec{q} = \vec{V}^{-1}\brak{\kappa \vec{n}-\vec{u}}, \quad \kappa \in \mathbb{R}
\end{align}
Substituting \eqref{eq:conic_normal_vec_q}
in \eqref{eq:conic_tangent_qquad},
\begin{multline}
\brak{\kappa \vec{n}-\vec{u}}^T\vec{V}^{-1}\brak{\kappa \vec{n}-\vec{u}} 
\\
+ 2\vec{u}^T\vec{V}^{-1}\brak{\kappa \vec{n}-\vec{u}} +f = 0
\\
\implies 
\kappa^2 \vec{n}^T\vec{V}^{-1}\vec{n} - \vec{u}^T\vec{V}^{-1}\vec{u} + f =0
\\
\text{or, } \kappa = \pm \sqrt{\frac{\vec{u}^T\vec{V}^{-1}\vec{u}-f}{\vec{n}^T\vec{V}^{-1}\vec{n}}}\label{eq:conic_normal_k}
\end{multline}
%
Substituting \eqref{eq:conic_normal_k} in \eqref{eq:conic_normal_vec_q}
yields \eqref{eq:conic_tangent_qk}.
%
\item 
If $\vec{V}$ is not invertible,  given the normal vector $\vec{n}$, the point of contact to \eqref{eq:conic_quad_form} is given by the matrix equation
\begin{align}
\label{eq:conic_tangent_q_eigen}
\begin{pmatrix}
\vec{u+\kappa \vec{n}}^T \\ \vec{V}
\end{pmatrix}
\vec{q} &= 
\begin{pmatrix}
-f
\\
\kappa\vec{n}-\vec{u}
\end{pmatrix}
\\
\text{where }  \kappa = \frac{\vec{p}_1^T\vec{u}}{\vec{p}_1^T\vec{n}}, \quad \vec{V}\vec{p}_1 &= 0
\label{eq:conic_tangent_qk_eigen}
\end{align}
\solution If $\vec{V}$ is non-invertible, it has a zero eigenvalue.  If the corresponding eigenvector is $\vec{p}_1$, then,
\begin{align}
\vec{V}\vec{p}_1 = 0
\label{eq:conic_zero_eigen}
\end{align}
From \eqref{eq:conic_normal_vec},
\begin{align}
\label{eq:conic_zero_eigen_normal}
\kappa \vec{n} &= \vec{V} \vec{q}+\vec{u}, \quad \kappa \in \mathbb{R}
\\
\implies \kappa \vec{p}_1^T\vec{n} &= \vec{p}_1^T\vec{V} \vec{q}+\vec{p}_1^T\vec{u}
\\
\text{or, } \kappa \vec{p}_1^T\vec{n} &= \vec{p}_1^T\vec{u},  \quad \because \vec{p}_1^T \vec{V} = 0, 
\\
\text{ from } \eqref{eq:conic_zero_eigen} &
%\label{eq:conic_normal_vec_q}
\end{align}
yielding $\kappa$ in \eqref{eq:conic_tangent_qk_eigen}. From \eqref{eq:conic_zero_eigen_normal},
\begin{align}
\kappa \vec{q}^T\vec{n} &= \vec{q}^T\vec{V} \vec{q}+\vec{q}^T\vec{u}
\\
\implies \kappa \vec{q}^T\vec{n} &= -f-\vec{q}^T\vec{u} \quad \text{from } \eqref{eq:conic_tangent_qquad},
\\
\text{or, } \brak{\kappa \vec{n}+\vec{u}}\vec{q} &= -f
\label{eq:conic_zero_eigen_normal_fq}
\end{align}
\eqref{eq:conic_zero_eigen_normal} can be expressed as
\begin{align}
\label{eq:conic_zero_eigen_normal_vq}
\vec{V} \vec{q} = \kappa \vec{n} - \vec{u}.
\end{align}
\eqref{eq:conic_zero_eigen_normal_fq} and \eqref{eq:conic_zero_eigen_normal_vq} clubbed together result in \eqref{eq:conic_tangent_q_eigen}.
\item All the results related to conics are summarized in 
Table \ref{table:conics}.  
\begin{table*}[!t]
\centering
\documentclass[10pt,a4paper]{report}
%\usepackage[latin1]{inputenc}
\usepackage[utf8]{inputenc}
\usepackage{amsmath}
\usepackage{amsfonts}
\usepackage{amssymb}
\usepackage{graphicx}
\usepackage{multicol}
\usepackage{tabularx}
\usepackage{tikz}
\usetikzlibrary{arrows,shapes,automata,petri,positioning,calc}
\usepackage{hyperref}
\usepackage{tikz}
\usetikzlibrary{matrix,calc}
\usepackage[margin=0.5in]{geometry}
% ---- power functions -----% 
\newcommand{\myvec}[1]{\ensuremath{\begin{pmatrix}#1\end{pmatrix}}}
\let\vec\mathbf

\providecommand{\norm}[1]{\left\lVert#1\right\rVert}
\providecommand{\abs}[1]{\left\vert#1\right\vert}
\let\vec\mathbf

\newcommand{\mydet}[1]{\ensuremath{\begin{vmatrix}#1\end{vmatrix}}}
\providecommand{\brak}[1]{\ensuremath{\left(#1\right)}}
\providecommand{\lbrak}[1]{\ensuremath{\left(#1\right.}}
\providecommand{\rbrak}[1]{\ensuremath{\left.#1\right)}}
\providecommand{\sbrak}[1]{\ensuremath{{}\left[#1\right]}}
%-------end power functions----%
\newenvironment{Figure}
  {\par\medskip\noindent\minipage{\linewidth}}
  {\endminipage\par\medskip}
\begin{document}
%--------------------logo figure-------------------------%
\begin{figure*}[!tbp]
  \centering
  \begin{minipage}[b]{0.4\textwidth}
    \includegraphics[scale=0.05]{iitlogo.jpg} 
  \end{minipage}
  \hfill
  \vspace{5mm}\begin{minipage}[b]{0.4\textwidth}
\raggedleft  \includegraphics[scale=0.05]{nrc.png}  \

  \end{minipage}\vspace{0.2cm}
\end{figure*}
%--------------------name & rollno-----------------------
\raggedright \textbf{Name}:\hspace{1mm} D. Siva Krishna\hspace{3cm} \Large \textbf{Assignment-6}\hspace{2.5cm} % 
\normalsize \textbf{Roll No.} :\hspace{1mm} FWC22065\vspace{1cm}
\begin{multicols}{2}

%----------------problem statement--------------%
\raggedright \textbf{Problem Statement:}\vspace{2mm}
\raggedright \\Find the area of the smaller region bounded by the ellipse $\frac{x^2}{9}+\frac{y^2}{4}=1$
and the line $\frac{x}{3}+\frac{y}{2}=1$\\
\vspace{5mm}
%-----------------------------solution---------------------------
\raggedright \textbf{SOLUTION}:\vspace{2mm}\\
\section{Without numericals}
%---------given----------------%
\raggedright \textbf{Given}:\vspace{2mm}\\
The general equation of ellipse is \\\vspace{1mm}
\begin{align}
\frac{x^2}{a^2}+\frac{y^2}{b^2}=1
\end{align}
The general equation of line is \\ \vspace{1mm}
\begin{align}
\frac{x}{a}+\frac{y}{b}=1
\end{align}
%-------------To find ------------------%
\textbf{To Find }\vspace{2mm}\\
To find the intersection points and area of shaded region shown in figure\vspace{2mm}  \\ 
%--------------steps----------------------%
\textbf{STEP-1}\vspace{2mm}\\
The given ellipse can be expressed as conics with parameters,\\ \vspace{1mm}
\begin{align}
\vec{V}=\myvec{
b^2 & 0\\
0 & a^2
}
\end{align}

\begin{align}
\vec{u}=0
\end{align} 
\begin{align}
f=-(a^2b^2)
\end{align} \vspace{2mm}


\textbf{STEP-2}\vspace{2mm}\\
the given line equation can be written as\\ 
\begin{align} 
	\vec{x}=\begin{pmatrix}a \\ 0 \\ \end{pmatrix}+k\begin{pmatrix}\frac{1}{b} \\ -\frac{1}{a} \\ \end{pmatrix}
\end{align}

\textbf{STEP-3}\vspace{2mm}\\
The points of intersection of the line, \\ 
\begin{align}
L: \quad \vec{x} = \vec{q} + \kappa \vec{m} \quad \kappa \in \mathbb{R}
\end{align}
with the conic section, \\ 
\begin{align}
	\vec{x}^{\top}\vec{V}\vec{x} + 2\vec{u}^{\top} \vec{x} + f = 0
\end{align}
are given by \\
\begin{align}
\vec{x}_i = \vec{q} + \kappa_i \vec{m}
\end{align}
where, \\
{\tiny
\begin{multline}
\kappa_i = \frac{1}
{
\vec{m}^T\vec{V}\vec{m}
}
\lbrak{-\vec{m}^T\brak{\vec{V}\vec{q}+\vec{u}}}
\\
\pm
\rbrak{\sqrt{
\sbrak{
\vec{m}^T\brak{\vec{V}\vec{q}+\vec{u}}
}^2
-
\brak
{
\vec{q}^T\vec{V}\vec{q} + 2\vec{u}^T\vec{q} +f
}
\brak{\vec{m}^T\vec{V}\vec{m}}
}
}
\end{multline}
}
On substituting\\
\begin{align}
\vec{q} &= \myvec{
a\\
0
} 
\end{align}
\begin{align}
\vec{m} = \myvec{\frac{1}{b} \\ -\frac{1}{a}}
\end{align}
With the given ellipse as in eq(3),(4),(5),\\ 

The value of $\kappa$ ,\\
\begin{align}
    \kappa =0,-6
\end{align}
by substituting eq(13) in eq(6)we get the
points of intersection of line with ellipse \\
\begin{align}
    \vec{A}=\myvec{
a\\
0
    }
\end{align}
\begin{align}
    \vec{B}=\myvec{
0\\
b
    }
\end{align}
\section{Substituting the numericals according to the problem}
\begin{align}
a=3, \vspace{2mm}
b=2\\
\vec{\textbf{V}}=\myvec{4\hspace{2mm}0\\ 0 \hspace{2mm} 9}, \vspace{2mm}
\vec{\textbf{u}}=0, \vspace{2mm}
f = -36\\
\vec{\textbf{q}}=\myvec{3\\0}, \vspace{2mm}
\vec{\textbf{m}}=\myvec{\frac{1}{2}\\-\frac{1}{3}}, \vspace{2mm}\\
\vec{\textbf{A}}=\myvec{0\\2}, \vspace{2mm}
\vec{\textbf{B}}=\myvec{3\\o}
\end{align}
\textbf{Result}
\begin{center}
 \includegraphics[scale=0.5]{conic_fig.png}    
 \end{center}\vspace{1mm}
 From the figure,\\ \vspace{1mm}
Total area of portion is given by, \\ \vspace{1mm}
Total Area=(area of ellipse in first quadrant)-(area of a triangle\textbf{AOB})

\subsection*{Area of ellipse}

\begin{align} 
\implies A2=\int_{0}^{a}\frac{b}{a}\sqrt{a^2-x^2} \,dx 
\\
=\int_{0}^{3}\frac{2}{3}\sqrt{9-x^2} \,dx 
\end{align}
by solving the above equation we get area of ellipse $\frac{3\pi}{2}$

\subsection*{Area of triangle}

\begin{align}
\implies A1=\int_{0}^{a} \frac{b}{a}(a-x) \,dx\\
=\int_{0}^{3} \frac{2}{3}(3-x) \,dx
\end{align}
by solving the above equation we get area of triangle 3 square units.
\\

The total area is

\vspace{3mm}
$\implies A=\frac{3\pi}{2}-3$\vspace{3mm}

The area of the smaller region is ,
\begin{align}
\hspace{-4.25cm}A= 3(\frac{\pi}{2}-1) square units
\end{align}
 \vspace{2mm} \textbf{Construction}
\begin{center}
\setlength{\arrayrulewidth}{0.5mm}
\setlength{\tabcolsep}{6pt}
\renewcommand{\arraystretch}{1.5}
    \begin{tabular}{|l|c|}
    \hline 
    \textbf{Points} & \textbf{coordinates} \\ \hline
   B & $\myvec{
   a\\
   0
   } $ \\\hline
   A & $\myvec{
   0\\
   b
   } $ \\\hline
      \end{tabular}
  \end{center}
  \end{multicols}
 
\textbf{Download code}

\begin{table}[h]
\large
\centering
\framebox{
\url{https://github.com/Siva Krishna/blob/main/conics/code/conic.py}}
\bibliographystyle{ieeetr}
\end{table} 
 
\end{document}

%\documentclass[10pt,a4paper]{report}
%\usepackage[latin1]{inputenc}
\usepackage[utf8]{inputenc}
\usepackage{amsmath}
\usepackage{amsfonts}
\usepackage{amssymb}
\usepackage{graphicx}
\usepackage{multicol}
\usepackage{tabularx}
\usepackage{tikz}
\usetikzlibrary{arrows,shapes,automata,petri,positioning,calc}
\usepackage{hyperref}
\usepackage{tikz}
\usetikzlibrary{matrix,calc}
\usepackage[margin=0.5in]{geometry}
% ---- power functions -----% 
\newcommand{\myvec}[1]{\ensuremath{\begin{pmatrix}#1\end{pmatrix}}}
\let\vec\mathbf

\providecommand{\norm}[1]{\left\lVert#1\right\rVert}
\providecommand{\abs}[1]{\left\vert#1\right\vert}
\let\vec\mathbf

\newcommand{\mydet}[1]{\ensuremath{\begin{vmatrix}#1\end{vmatrix}}}
\providecommand{\brak}[1]{\ensuremath{\left(#1\right)}}
\providecommand{\lbrak}[1]{\ensuremath{\left(#1\right.}}
\providecommand{\rbrak}[1]{\ensuremath{\left.#1\right)}}
\providecommand{\sbrak}[1]{\ensuremath{{}\left[#1\right]}}
%-------end power functions----%
\newenvironment{Figure}
  {\par\medskip\noindent\minipage{\linewidth}}
  {\endminipage\par\medskip}
\begin{document}
%--------------------logo figure-------------------------%
\begin{figure*}[!tbp]
  \centering
  \begin{minipage}[b]{0.4\textwidth}
    \includegraphics[scale=0.05]{iitlogo.jpg} 
  \end{minipage}
  \hfill
  \vspace{5mm}\begin{minipage}[b]{0.4\textwidth}
\raggedleft  \includegraphics[scale=0.05]{nrc.png}  \

  \end{minipage}\vspace{0.2cm}
\end{figure*}
%--------------------name & rollno-----------------------
\raggedright \textbf{Name}:\hspace{1mm} D. Siva Krishna\hspace{3cm} \Large \textbf{Assignment-6}\hspace{2.5cm} % 
\normalsize \textbf{Roll No.} :\hspace{1mm} FWC22065\vspace{1cm}
\begin{multicols}{2}

%----------------problem statement--------------%
\raggedright \textbf{Problem Statement:}\vspace{2mm}
\raggedright \\Find the area of the smaller region bounded by the ellipse $\frac{x^2}{9}+\frac{y^2}{4}=1$
and the line $\frac{x}{3}+\frac{y}{2}=1$\\
\vspace{5mm}
%-----------------------------solution---------------------------
\raggedright \textbf{SOLUTION}:\vspace{2mm}\\
\section{Without numericals}
%---------given----------------%
\raggedright \textbf{Given}:\vspace{2mm}\\
The general equation of ellipse is \\\vspace{1mm}
\begin{align}
\frac{x^2}{a^2}+\frac{y^2}{b^2}=1
\end{align}
The general equation of line is \\ \vspace{1mm}
\begin{align}
\frac{x}{a}+\frac{y}{b}=1
\end{align}
%-------------To find ------------------%
\textbf{To Find }\vspace{2mm}\\
To find the intersection points and area of shaded region shown in figure\vspace{2mm}  \\ 
%--------------steps----------------------%
\textbf{STEP-1}\vspace{2mm}\\
The given ellipse can be expressed as conics with parameters,\\ \vspace{1mm}
\begin{align}
\vec{V}=\myvec{
b^2 & 0\\
0 & a^2
}
\end{align}

\begin{align}
\vec{u}=0
\end{align} 
\begin{align}
f=-(a^2b^2)
\end{align} \vspace{2mm}


\textbf{STEP-2}\vspace{2mm}\\
the given line equation can be written as\\ 
\begin{align} 
	\vec{x}=\begin{pmatrix}a \\ 0 \\ \end{pmatrix}+k\begin{pmatrix}\frac{1}{b} \\ -\frac{1}{a} \\ \end{pmatrix}
\end{align}

\textbf{STEP-3}\vspace{2mm}\\
The points of intersection of the line, \\ 
\begin{align}
L: \quad \vec{x} = \vec{q} + \kappa \vec{m} \quad \kappa \in \mathbb{R}
\end{align}
with the conic section, \\ 
\begin{align}
	\vec{x}^{\top}\vec{V}\vec{x} + 2\vec{u}^{\top} \vec{x} + f = 0
\end{align}
are given by \\
\begin{align}
\vec{x}_i = \vec{q} + \kappa_i \vec{m}
\end{align}
where, \\
{\tiny
\begin{multline}
\kappa_i = \frac{1}
{
\vec{m}^T\vec{V}\vec{m}
}
\lbrak{-\vec{m}^T\brak{\vec{V}\vec{q}+\vec{u}}}
\\
\pm
\rbrak{\sqrt{
\sbrak{
\vec{m}^T\brak{\vec{V}\vec{q}+\vec{u}}
}^2
-
\brak
{
\vec{q}^T\vec{V}\vec{q} + 2\vec{u}^T\vec{q} +f
}
\brak{\vec{m}^T\vec{V}\vec{m}}
}
}
\end{multline}
}
On substituting\\
\begin{align}
\vec{q} &= \myvec{
a\\
0
} 
\end{align}
\begin{align}
\vec{m} = \myvec{\frac{1}{b} \\ -\frac{1}{a}}
\end{align}
With the given ellipse as in eq(3),(4),(5),\\ 

The value of $\kappa$ ,\\
\begin{align}
    \kappa =0,-6
\end{align}
by substituting eq(13) in eq(6)we get the
points of intersection of line with ellipse \\
\begin{align}
    \vec{A}=\myvec{
a\\
0
    }
\end{align}
\begin{align}
    \vec{B}=\myvec{
0\\
b
    }
\end{align}
\section{Substituting the numericals according to the problem}
\begin{align}
a=3, \vspace{2mm}
b=2\\
\vec{\textbf{V}}=\myvec{4\hspace{2mm}0\\ 0 \hspace{2mm} 9}, \vspace{2mm}
\vec{\textbf{u}}=0, \vspace{2mm}
f = -36\\
\vec{\textbf{q}}=\myvec{3\\0}, \vspace{2mm}
\vec{\textbf{m}}=\myvec{\frac{1}{2}\\-\frac{1}{3}}, \vspace{2mm}\\
\vec{\textbf{A}}=\myvec{0\\2}, \vspace{2mm}
\vec{\textbf{B}}=\myvec{3\\o}
\end{align}
\textbf{Result}
\begin{center}
 \includegraphics[scale=0.5]{conic_fig.png}    
 \end{center}\vspace{1mm}
 From the figure,\\ \vspace{1mm}
Total area of portion is given by, \\ \vspace{1mm}
Total Area=(area of ellipse in first quadrant)-(area of a triangle\textbf{AOB})

\subsection*{Area of ellipse}

\begin{align} 
\implies A2=\int_{0}^{a}\frac{b}{a}\sqrt{a^2-x^2} \,dx 
\\
=\int_{0}^{3}\frac{2}{3}\sqrt{9-x^2} \,dx 
\end{align}
by solving the above equation we get area of ellipse $\frac{3\pi}{2}$

\subsection*{Area of triangle}

\begin{align}
\implies A1=\int_{0}^{a} \frac{b}{a}(a-x) \,dx\\
=\int_{0}^{3} \frac{2}{3}(3-x) \,dx
\end{align}
by solving the above equation we get area of triangle 3 square units.
\\

The total area is

\vspace{3mm}
$\implies A=\frac{3\pi}{2}-3$\vspace{3mm}

The area of the smaller region is ,
\begin{align}
\hspace{-4.25cm}A= 3(\frac{\pi}{2}-1) square units
\end{align}
 \vspace{2mm} \textbf{Construction}
\begin{center}
\setlength{\arrayrulewidth}{0.5mm}
\setlength{\tabcolsep}{6pt}
\renewcommand{\arraystretch}{1.5}
    \begin{tabular}{|l|c|}
    \hline 
    \textbf{Points} & \textbf{coordinates} \\ \hline
   B & $\myvec{
   a\\
   0
   } $ \\\hline
   A & $\myvec{
   0\\
   b
   } $ \\\hline
      \end{tabular}
  \end{center}
  \end{multicols}
 
\textbf{Download code}

\begin{table}[h]
\large
\centering
\framebox{
\url{https://github.com/Siva Krishna/blob/main/conics/code/conic.py}}
\bibliographystyle{ieeetr}
\end{table} 
 
\end{document}

\caption{$\vec{x}^T\vec{V}\vec{x}+2\vec{u}^T\vec{x}+f = 0$  can be expressed in the above standard form for various conics. $\vec{c}$ represents the centre/vertex of the conic. $\vec{q}$ is/are the point(s) of contact for the tangent(s). }
\label{table:conics}
\end{table*}

\end{enumerate}
