
%\renewcommand{\theequation}{\theenumi}
%\begin{enumerate}[label=\arabic*.,ref=\theenumi]
\begin{enumerate}[label=\thesection.\arabic*.,ref=\thesection.\theenumi]
%\begin{enumerate}
%\numberwithin{equation}{enumi}
%\item The area of a triangle with vertices $\vec{A}, \vec{B}, \vec{C}$ is given by 
%\begin{align}
%  \label{eq:area3d}
% \frac{1}{2} \norm{\brak{\vec{A} - \vec{B}} \times \brak{\vec{A} - \vec{C}}}
%\end{align}

\item Points $\vec{A}, \vec{B}, \vec{C}$ are on a line if 
\begin{align}
  \label{eq:line_rank}
  \text{rank}\myvec{\vec{A} \\ \vec{B} \\ \vec{C} }  = 1
\end{align}
\item Points $\vec{A}, \vec{B}, \vec{C}, \vec{D}$ form a paralelogram if 
\begin{align}
  \label{eq:parallelgm_rank}
  \text{rank}\myvec{\vec{A} \\ \vec{B} \\ \vec{C} \\ \vec{D}  }  = 1, 
  \text{rank}\myvec{\vec{A} \\ \vec{B} \\ \vec{C} }  = 2
\end{align}
\item The equation of a line  is given by  
	\eqref{eq:dir_line}
	\item The equation of a plane is given by
	\eqref{eq:normal_line}
	\item The distance from the origin to the line  in 
	\eqref{eq:normal_line}
	is given by 
	\eqref{eq:dist_line_2d_orig}
\item The distance from a point $\vec{P}$  to the line in 
	\eqref{eq:dir_line} is given by 
\begin{align}
	\label{dist_3d_def_final}
		d = \norm{\vec{A} -\vec{P}}^2 - \frac{\cbrak{\vec{m}^{\top}\brak{\vec{A}-\vec{P} 
	}}^2}{\norm{\vec{m}}^2}
%	d =\norm{\vec{A}  -\vec{P}
% -\frac{\vec{m}^{\top}\brak{\vec{A} 
%			-\vec{P}}}
%			{ \norm{\vec{m}}^2}
%	\vec{m}}
		\end{align}
		\solution
%		\solution{\title{Solution:}
\begin{align}
	\label{dist_3d_def}
	d\brak{\lambda } &=\norm{\vec{A} + \lambda \vec{m}-\vec{P}}
	\\
\implies 	d^2\brak{\lambda } &=\norm{\vec{A} + \lambda \vec{m}-\vec{P}}^2
\end{align}
which can be simplified to obtain 
	\begin{multline}
d^2\brak{\lambda } =\lambda^2 \norm{\vec{m}}^2+2\lambda \vec{m}^{\top}\brak{\vec{A} 
		-\vec{P}}
		\\
		+\norm{\vec{A} -\vec{P}}^2
	\end{multline}
which is of the form 
\begin{align}
	\label{dist_3d_def_quad}
	d^2\brak{\lambda } &=a \lambda^2 + 2b\lambda +c
	\\
	&=a \cbrak{\brak{\lambda+ \frac{b}{a}}^2 +\sbrak{\frac{c}{a}-\brak{\frac{b}{a}}^2 }}
\end{align}
with 
\begin{align}
	\label{dist_3d_def_quad_abc}
	a = \norm{\vec{m}}^2, b = \vec{m}^{\top}\brak{\vec{A} 
		-\vec{P}}, c = 
		\norm{\vec{A} -\vec{P}}^2
\end{align}
which can be expressed as 
%		\begin{multline}
%			d^2\brak{\lambda } =\norm{\vec{m}}^2\brak{\lambda + \frac{\vec{m}^{\top}\brak{\vec{A}-\vec{P} }}{\vec{m}}^2}}^2 +2\lambda \vec{m}^{\top}\brak{\vec{A} 
%			-\vec{P}}
%			\\
%			+\norm{\vec{A} -\vec{P}}^2
%		\end{multline}
		From the above, $d^2\brak{\lambda}$ is smallest when upon substituting from 
	\eqref{dist_3d_def_quad_abc}
\begin{align}
	\label{dist_3d_def_quad_small}
	\lambda+ \frac{b}{2a} &= 0 \implies \lambda = - \frac{b}{2a}
	\\
	&= -\frac{\vec{m}^{\top}\brak{\vec{A} 
			-\vec{P}}}
			{ \norm{\vec{m}}^2}
	%		\label{dist_3d_lam}
\end{align}
and consequently, 
\begin{align}
	d_{\min}\brak{\lambda } &=a \brak{\frac{c}{a}-\brak{\frac{b}{a}}^2 } 
	\\
	&=c - \frac{b^2}{a }
\end{align}
yielding
	\eqref{dist_3d_def_final} after substituting from 
	\eqref{dist_3d_def_quad_abc}.
%From 	\eqref{dist_3d_def} and \eqref{dist_3d_lam}, 
%	\eqref{dist_3d_def} is obtained.
\item The distance between the parallel planes 
	\eqref{eq:parallel_lines}
	is given by 
	\eqref{eq:dist_lines_2d}.
\item The plane 
		\begin{align}
		\vec{n}^{\top}
			\vec{x} = c
			\label{eq:plain_contain}
		\end{align}
		contains the line 
		\begin{align}
			\vec{x} = \vec{A}+\lambda \vec{m}
			\label{eq:line_contain}
		\end{align}
		if 
		\begin{align}
		\vec{m}^{\top}\vec{n} = 0
			\label{eq:line_plain_contain}
		\end{align}
		\solution Any point on the line 
			\eqref{eq:line_contain}
			should also satisfy 
			\eqref{eq:plain_contain}.  Hence, 
		\begin{align}
			\vec{n}^{\top}\brak{\vec{A}+\lambda \vec{m}} &= \vec{n}^{\top}\vec{A}=c
		\end{align}
		which can be simplified to obtain
			\eqref{eq:line_plain_contain}
		\item The foot of the perpendicular from a point $\vec{P}$ to the plane 
		\begin{align}
			\vec{n}^{\top}\vec{x} =c
		\end{align}
		is given by 
\begin{align}
	\vec{x} &= \vec{P} + \frac{c - \vec{n}^{\top}\vec{P}}{\norm{\vec{n}}^2}
\vec{n}
	\label{eq:foot_perp_pt_plane}
\end{align}
		\\
		\solution The equation of the line perpendicular to the given plane and passing through $\vec{P}$ is 
		\begin{align}
			\vec{x} = \vec{P} + \lambda 	\vec{n}
		\end{align}
		From 
	\eqref{eq:dir_line_plane_isect}, the intersection of the above line with the given plane is 
	\eqref{eq:foot_perp_pt_plane}.
	\iffalse
\begin{align}
	\vec{x} &= \vec{P} + \frac{c - \vec{n}^{\top}\vec{P}}{\norm{\vec{n}}^2}
\vec{n}
	\label{eq:foot_perp_pt_plane}
\end{align}
\fi
\item The image of a point $\vec{P}$ with respect to the plane 
		\begin{align}
			\vec{n}^{\top}\vec{x} =c
		\end{align}
		is given by 
		\begin{align}
			\vec{R} &=
	  \vec{P} + 2\frac{c - \vec{n}^{\top}\vec{P}}{\norm{\vec{n}}^2}
			\label{eq:image_pt_plane}
		\end{align}
		\solution Let $\vec{R}$ be the desired image.  Then, subtituting the expression for the  foot of the perpendicular from $\vec{P}$ to the given plane using 
	\eqref{eq:foot_perp_pt_plane},
		\begin{align}
			\frac{\vec{P}+\vec{R}}{2} &=
	  \vec{P} + \frac{c - \vec{n}^{\top}\vec{P}}{\norm{\vec{n}}^2}
		\end{align}
		\item Let a plane pass through the points $\vec{A},\vec{B}$ and be perpendicular to the plane 
		\begin{align}
		\vec{n}^{\top}\vec{x} =c 
			\label{eq:plane_3d_2pt_perp_given}
		\end{align}
		Then the equation of this plane is given by 
		\begin{align}
		\vec{p}^{\top}\vec{x} = 1
			\label{eq:plane_3d_2pt}
		\end{align}
		where
		\begin{align}
			\vec{p} = 		\myvec{\vec{A} & \vec{B} & \vec{n}}^{-\top}  \myvec{1 \\ 1 \\ 0}
			\label{eq:plane_3d_2pt_perp_norm}
		\end{align}
	\solution From the given information, 
		\begin{align}
			\vec{p}^{\top}\vec{A} &=d 
			\\
			\vec{p}^{\top}\vec{B} &=d 
			\\
			\vec{p}^{\top}\vec{n} &= 0
			\label{eq:plane_3d_2pt_perp_system}
		\end{align}
		$\because$ the normal vectors to the two planes will also be perpendicular.  The system of equations in 
			\eqref{eq:plane_3d_2pt_perp_system}
			can be expressed as the matrix equation
		\begin{align}
			\myvec{\vec{A} & \vec{B} & \vec{n}}^{\top}\vec{p} = d\myvec{1 \\ 1 \\ 0}
			\label{eq:plane_3d_2pt_perp_system_temp}
		\end{align}
		which yields 
			\eqref{eq:plane_3d_2pt_perp_norm}
			upon normalising with $d$.
		\item The intersection of the line represented by 
	\eqref{eq:dir_line}
	with the plane represented by 
	\eqref{eq:normal_line}
	is given by 
\begin{align}
	\label{eq:dir_line_plane_isect}
	\vec{x} &= \vec{A} + \frac{c - \vec{n}^{\top}\vec{A}}{\vec{n}^{\top}\vec{m}}
\vec{m}
\end{align}
\solution From 
	\eqref{eq:dir_line}
	and 
	\eqref{eq:normal_line},
\begin{align}
	\vec{x} &= \vec{A} + \lambda \vec{m}
	\\
	\vec{n}^{\top}\vec{x} &= c
	\\
	\implies 
	\vec{n}^{\top}\brak{\vec{A} + \lambda \vec{m}}&= c
	\label{eq:dir_line_plane_inter}
\end{align}
which can be simplified to obtain
\begin{align}
	\vec{n}^{\top}\vec{A} + \lambda 	\vec{n}^{\top}\vec{m}&= c
	\\
	\implies \lambda &= \frac{c - \vec{n}^{\top}\vec{A}}{\vec{n}^{\top}\vec{m}}
\end{align}
Substituting the above in 
	\eqref{eq:dir_line_plane_inter}
	yields
	\eqref{eq:dir_line_plane_isect}.
\item The foot of the perpendicular from the point $\vec{P}$ to the line  represented by 
	\eqref{eq:dir_line}
	is given by 
\begin{align}
	\label{eq:plane_line_foot_ans}
	\vec{x} &= \vec{A} + \frac{ \vec{m}^{\top}\brak{\vec{P} - \vec{A}}}{\norm{\vec{m}}^2}
\vec{m}
\end{align}
\solution  Let the equation of the line be 
\begin{align}
	\label{eq:dir_line_foot}
	\vec{x} &= \vec{A} + \lambda \vec{m}
\end{align}
	The equation of the plane perpendicular to the given line passing through $\vec{P}$ is given by
\begin{align}
	\label{eq:plane_line_foot}
	\vec{m}^{\top}\brak{\vec{x}-\vec{P}}  &= 0
	\\
	\implies \vec{m}^{\top}\vec{x}  &= \vec{m}^{\top}\vec{P}
\end{align}
The desired foot of the perpendicular is the intersection of 
	\eqref{eq:dir_line_foot} with 
	\eqref{eq:plane_line_foot}
	which can be obtained from 
	\eqref{eq:dir_line_plane_isect}
	as 
	\eqref{eq:plane_line_foot_ans}
\item The foot of the perpendicular from a point $\vec{P}$ to a plane is $\vec{Q}$.  The equation of the plane is given by 
\begin{align}
	\label{eq:plane_foot_perp}
	\brak{\vec{P}-\vec{Q}}^{\top}\brak{\vec{x}-\vec{Q}} = 0
\end{align}
	\solution  The normal vector to the plane is given by 
\begin{align}
	\vec{n}= \vec{P}-\vec{Q} 
\end{align}
	Hence, the equation of the plane is
	\eqref{eq:plane_foot_perp}.
\item Let $\vec{A}, \vec{B}, \vec{C}$ be  points on a plane.  The equation of the plane is then given by 	
\begin{align}
	\myvec{	\vec{A} & \vec{B}& \vec{C}}^{\top} \vec{n}= \myvec{1\\1\\1}
	\label{eq:plane_3pt}
\end{align}
\solution Let the equation of the plane be 
\begin{align}
	\vec{n}^{\top}	\vec{x} &= 1
\end{align}
Then 
\begin{align}
	\vec{n}^{\top}	\vec{A} &= 1
	\\
	\vec{n}^{\top}	\vec{B} &= 1
	\\
	\vec{n}^{\top}	\vec{C} &= 1
\end{align}
which can be combined to obtain 
	\eqref{eq:plane_3pt}.
\item The lines 
    \begin{align}
        \vec{x} = \vec{x_1} + \lambda_1\vec{m_1} \label{eq:chapters/12/11/2/16/L1-gen} \\
        \vec{x} = \vec{x_2} + \lambda_2\vec{m_2} \label{eq:chapters/12/11/2/16/L2-gen}
    \end{align}
    intersect if
    \begin{align}
    %    \lambda_1\vec{m_1} - \lambda_2\vec{m_2} &= \vec{x_2} - \vec{x_1} \\
       \vec{M}\bm{\lambda} &= \vec{x_2} - \vec{x_1}
        \label{eq:chapters/12/11/2/16/intersect-cond}
    \end{align}
    where
    \begin{align}
        \vec{M} \triangleq \myvec{\vec{m_1} & \vec{m_2}} \label{eq:chapters/12/11/2/16/M-def} \\
        \bm{\lambda} \triangleq \myvec{\lambda_1\\-\lambda_2}
        \label{eq:chapters/12/11/2/16/lambda-def}
    \end{align}
\item 
    \iffalse
    Let 
    \begin{align}
        \vec{x} = \vec{x_1} + \lambda_1\vec{m_1} \label{eq:chapters/12/11/2/16/L1-gen} \\
        \vec{x} = \vec{x_2} + \lambda_2\vec{m_2} \label{eq:chapters/12/11/2/16/L2-gen}
    \end{align}
be two skew lines. 
\fi
	The closest points on two skew lines are given by 
    \begin{align}
	    \vec{M}^\top \vec{M}\bm{\lambda} = \vec{M}^\top\brak{\vec{x_2}-\vec{x_1}}
        \label{eq:chapters/12/11/2/16/lambda-eqn}
    \end{align}
	\solution
	\iffalse
    If these lines intersect, then
    If the lines are skew,
    \fi
    For the lines defined in \eqref{eq:chapters/12/11/2/16/L1-gen} and \eqref{eq:chapters/12/11/2/16/L2-gen},
Suppose the closest points on both lines are
    \begin{align}
        \vec{A} = \vec{x_1} + \lambda_1\vec{m_1} \label{eq:chapters/12/11/2/16/a-def} \\
        \vec{B} = \vec{x_2} + \lambda_2\vec{m_2}
        \label{eq:chapters/12/11/2/16/b-def}
    \end{align}
    Then, $AB$ is perpendicular to both lines, hence
    \begin{align}
        \vec{m_1}^\top\brak{\vec{A}-\vec{B}} = 0 \\
        \vec{m_2}^\top\brak{\vec{A}-\vec{B}} = 0 \\
        \implies \vec{M}^\top\brak{\vec{A}-\vec{B}} = \vec{O}
        \label{eq:chapters/12/11/2/16/perp-vec}
    \end{align}
    Using \eqref{eq:chapters/12/11/2/16/a-def} and \eqref{eq:chapters/12/11/2/16/b-def} in \eqref{eq:chapters/12/11/2/16/perp-vec},
    \begin{align}
        \vec{M}^\top\brak{\vec{x_1}-\vec{x_2} + \vec{M}\bm{\lambda}} = \vec{0} \\
    \end{align}
    yielding
        \ref{eq:chapters/12/11/2/16/lambda-eqn}.
%\renewcommand{\theequation}{\theenumi}
%%\begin{enumerate}[label=\arabic*.,ref=\theenumi]
%\begin{enumerate}[label=\thesubsection.\arabic*.,ref=\thesubsection.\theenumi]
%\numberwithin{equation}{enumi}
%
\item (Parallelogram Law)  Let $\vec{A}, \vec{B}, \vec{D}$ be three vertices of a parallelogram.  Then the vertex $\vec{C}$ is given by 
\begin{align}
  \label{eq:pgm_law}
  \vec{C} = \vec{B}+\vec{C} - \vec{A}
\end{align}
		\solution Shifting $\vec{A}$ to the origin, we obtain a parallelogram with corresponding vertices 
\begin{align}
  \label{eq:pgm_law_org_vert}
  \vec{0}, \vec{B}-\vec{A}, \vec{D} - \vec{A}
\end{align}
The fourth vertex of this parallelogram is then obtained as 
\begin{align}
  \label{eq:pgm_law_org}
	\brak{\vec{B}-\vec{A}}+\brak{ \vec{D} - \vec{A}} = \vec{D}+ \vec{B} - 2\vec{A}
\end{align}
Shifting the origin to $\vec{A}$, the fourth vertex is obtained as 
\begin{align}
  \label{eq:pgm_law_org_C}
		 \vec{C} &= \vec{D}+ \vec{B} - 2\vec{A}+\vec{A} 
		 \\
	 &=
	 \vec{D}+ \vec{B} - \vec{A} 
\end{align}
\item (Affine Transformation) Let $\vec{A},\vec{C}$, be opposite vertices of a square. The other two points can be obtained as  
\begin{align}
  \label{eq:square_points}
  \vec{B} = \frac{\norm{\vec{A}-\vec{C}}}{\sqrt{2}} \vec{P}\vec{e}_1+\vec{A}
  \\
  \vec{D} = \frac{\norm{\vec{A}-\vec{C}}}{\sqrt{2}} \vec{P}\vec{e}_2+\vec{A}
\end{align}
where 
\begin{align}
	\vec{P} = \myvec{\cos \brak{\theta-\frac{\pi}{4}} & \sin  \brak{\theta-\frac{\pi}{4}} \\ \sin \brak{\theta-\frac{\pi}{4}} & \cos \brak{\theta-\frac{\pi}{4}}}
\end{align}
and 
\begin{align}
	\cos\theta = \frac{\brak{\vec{C}-\vec{A}}^{\top}\vec{e}_1}{\norm{\vec{A}-\vec{C}}\norm{\vec{e}_1}}
\end{align}
\end{enumerate}
